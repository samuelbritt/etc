% \iffalse meta-comment
%
% Copyright (C) 2008 by
%    Joseph Wright <joseph.wright@morningstar2.co.uk>
%
% This work may be distributed and/or modified under the
% conditions of the LaTeX Project Public License, either
% version 1.3c of this license or (at your option) any later
% version. The latest version of this license is in
%    http://www.latex-project.org/lppl.txt
% and version 1.3c or later is part of all distributions of
% LaTeX version 2005/12/01 or later.
%
% This work has the LPPL maintenance status `maintained'.
%
% The current maintainer of this work is Joseph Wright.
%
% This work consists of the source file siunitx.dtx
%                 and the derived files siunitx.ins,
%                                       siunitx.sty,
%                                       jawltxdoc.sty,
%                                       si-bug.ltx,
%                                       si-SIunits.cfg,
%                                       si-sistyle.cfg,
%                                       si-unitsdef.cfg,
%                                       si-units.cfg,
%                                       si-hepunits.cfg,
%                                       si-fancynum.cfg,
%                                       si-fancyunits.cfg,
%                                       si-prefix.cfg,
%                                       si-named.cfg,
%                                       si-prefixed.cfg,
%                                       si-addn.cfg,
%                                       si-accepted.cfg,
%                                       si-physical.cfg,
%                                       si-abbr.cfg,
%                                       si-synchem.cfg,
%                                       si-hep.cfg,
%                                       si-astro.cfg,
%                                       si-geophys.cfg,
%                                       si-binary.cfg,
%                                       si-UK.cfg,
%                                       si-USA.cfg,
%                                       si-DE.cfg,
%                                       si-ZA.cfg,
%                                       siunitx.bib and
%                                       siunitx.pdf
%
% TDS-ready files:
%    The compressed file siunitx.tds.zip contains an unpacked version
%    of all of the files included here, and pre-compiled
%    documentation in PDF format.  Simply decompress siunitx.tds.dtx
%    in your local TeX directory, run your hash program (texhash,
%    initexmf --update-fndb, etc.) and everything will be ready to
%    go.
%
% Unpacking:
%    (a) If siunitx.ins is present:
%           tex siunitx.ins
%    (b) Without siunitx.ins:
%           tex siunitx.dtx
%    (c) If you use LaTeX to generate files:
%           latex \let\install=y% \iffalse meta-comment
%
% Copyright (C) 2008 by
%    Joseph Wright <joseph.wright@morningstar2.co.uk>
%
% This work may be distributed and/or modified under the
% conditions of the LaTeX Project Public License, either
% version 1.3c of this license or (at your option) any later
% version. The latest version of this license is in
%    http://www.latex-project.org/lppl.txt
% and version 1.3c or later is part of all distributions of
% LaTeX version 2005/12/01 or later.
%
% This work has the LPPL maintenance status `maintained'.
%
% The current maintainer of this work is Joseph Wright.
%
% This work consists of the source file siunitx.dtx
%                 and the derived files siunitx.ins,
%                                       siunitx.sty,
%                                       jawltxdoc.sty,
%                                       si-bug.ltx,
%                                       si-SIunits.cfg,
%                                       si-sistyle.cfg,
%                                       si-unitsdef.cfg,
%                                       si-units.cfg,
%                                       si-hepunits.cfg,
%                                       si-fancynum.cfg,
%                                       si-fancyunits.cfg,
%                                       si-prefix.cfg,
%                                       si-named.cfg,
%                                       si-prefixed.cfg,
%                                       si-addn.cfg,
%                                       si-accepted.cfg,
%                                       si-physical.cfg,
%                                       si-abbr.cfg,
%                                       si-synchem.cfg,
%                                       si-hep.cfg,
%                                       si-astro.cfg,
%                                       si-geophys.cfg,
%                                       si-binary.cfg,
%                                       si-UK.cfg,
%                                       si-USA.cfg,
%                                       si-DE.cfg,
%                                       si-ZA.cfg,
%                                       siunitx.bib and
%                                       siunitx.pdf
%
% TDS-ready files:
%    The compressed file siunitx.tds.zip contains an unpacked version
%    of all of the files included here, and pre-compiled
%    documentation in PDF format.  Simply decompress siunitx.tds.dtx
%    in your local TeX directory, run your hash program (texhash,
%    initexmf --update-fndb, etc.) and everything will be ready to
%    go.
%
% Unpacking:
%    (a) If siunitx.ins is present:
%           tex siunitx.ins
%    (b) Without siunitx.ins:
%           tex siunitx.dtx
%    (c) If you use LaTeX to generate files:
%           latex \let\install=y% \iffalse meta-comment
%
% Copyright (C) 2008 by
%    Joseph Wright <joseph.wright@morningstar2.co.uk>
%
% This work may be distributed and/or modified under the
% conditions of the LaTeX Project Public License, either
% version 1.3c of this license or (at your option) any later
% version. The latest version of this license is in
%    http://www.latex-project.org/lppl.txt
% and version 1.3c or later is part of all distributions of
% LaTeX version 2005/12/01 or later.
%
% This work has the LPPL maintenance status `maintained'.
%
% The current maintainer of this work is Joseph Wright.
%
% This work consists of the source file siunitx.dtx
%                 and the derived files siunitx.ins,
%                                       siunitx.sty,
%                                       jawltxdoc.sty,
%                                       si-bug.ltx,
%                                       si-SIunits.cfg,
%                                       si-sistyle.cfg,
%                                       si-unitsdef.cfg,
%                                       si-units.cfg,
%                                       si-hepunits.cfg,
%                                       si-fancynum.cfg,
%                                       si-fancyunits.cfg,
%                                       si-prefix.cfg,
%                                       si-named.cfg,
%                                       si-prefixed.cfg,
%                                       si-addn.cfg,
%                                       si-accepted.cfg,
%                                       si-physical.cfg,
%                                       si-abbr.cfg,
%                                       si-synchem.cfg,
%                                       si-hep.cfg,
%                                       si-astro.cfg,
%                                       si-geophys.cfg,
%                                       si-binary.cfg,
%                                       si-UK.cfg,
%                                       si-USA.cfg,
%                                       si-DE.cfg,
%                                       si-ZA.cfg,
%                                       siunitx.bib and
%                                       siunitx.pdf
%
% TDS-ready files:
%    The compressed file siunitx.tds.zip contains an unpacked version
%    of all of the files included here, and pre-compiled
%    documentation in PDF format.  Simply decompress siunitx.tds.dtx
%    in your local TeX directory, run your hash program (texhash,
%    initexmf --update-fndb, etc.) and everything will be ready to
%    go.
%
% Unpacking:
%    (a) If siunitx.ins is present:
%           tex siunitx.ins
%    (b) Without siunitx.ins:
%           tex siunitx.dtx
%    (c) If you use LaTeX to generate files:
%           latex \let\install=y% \iffalse meta-comment
%
% Copyright (C) 2008 by
%    Joseph Wright <joseph.wright@morningstar2.co.uk>
%
% This work may be distributed and/or modified under the
% conditions of the LaTeX Project Public License, either
% version 1.3c of this license or (at your option) any later
% version. The latest version of this license is in
%    http://www.latex-project.org/lppl.txt
% and version 1.3c or later is part of all distributions of
% LaTeX version 2005/12/01 or later.
%
% This work has the LPPL maintenance status `maintained'.
%
% The current maintainer of this work is Joseph Wright.
%
% This work consists of the source file siunitx.dtx
%                 and the derived files siunitx.ins,
%                                       siunitx.sty,
%                                       jawltxdoc.sty,
%                                       si-bug.ltx,
%                                       si-SIunits.cfg,
%                                       si-sistyle.cfg,
%                                       si-unitsdef.cfg,
%                                       si-units.cfg,
%                                       si-hepunits.cfg,
%                                       si-fancynum.cfg,
%                                       si-fancyunits.cfg,
%                                       si-prefix.cfg,
%                                       si-named.cfg,
%                                       si-prefixed.cfg,
%                                       si-addn.cfg,
%                                       si-accepted.cfg,
%                                       si-physical.cfg,
%                                       si-abbr.cfg,
%                                       si-synchem.cfg,
%                                       si-hep.cfg,
%                                       si-astro.cfg,
%                                       si-geophys.cfg,
%                                       si-binary.cfg,
%                                       si-UK.cfg,
%                                       si-USA.cfg,
%                                       si-DE.cfg,
%                                       si-ZA.cfg,
%                                       siunitx.bib and
%                                       siunitx.pdf
%
% TDS-ready files:
%    The compressed file siunitx.tds.zip contains an unpacked version
%    of all of the files included here, and pre-compiled
%    documentation in PDF format.  Simply decompress siunitx.tds.dtx
%    in your local TeX directory, run your hash program (texhash,
%    initexmf --update-fndb, etc.) and everything will be ready to
%    go.
%
% Unpacking:
%    (a) If siunitx.ins is present:
%           tex siunitx.ins
%    (b) Without siunitx.ins:
%           tex siunitx.dtx
%    (c) If you use LaTeX to generate files:
%           latex \let\install=y\input{siunitx.dtx}
%
% Documentation:
%    (a) Without write18 enabled:
%          pdflatex siunitx.dtx
%          bibtex8 --wolfgang siunitx
%          makeindex -s gind.ist siunitx.idx
%          makeindex -s gglo.ist -o siunitx.gls siunitx.glo
%          pdflatex siunitx.dtx
%          pdflatex siunitx.dtx
%    (b) With write18 enabled:
%          pdflatex siunitx.dtx
%          pdflatex siunitx.dtx
%          pdflatex siunitx.dtx
%
% Installation:
%     Copy siunitx.sty and the various .cfg files to a location
%     searched by TeX, and if required by your TeX installation,
%     run the appropriate command to build a hash of files
%     (texhash, initexmf --update-fndb, etc.)
%
% Note:
%     The jawltxdoc.sty file is not needed for installation,
%     only for building the documentation; it may be deleted
%     after producing the documentation (if necessary).
%
%<*ignore>
% This is all taken verbatim from Heiko Oberdiek's packages
\begingroup
  \def\x{LaTeX2e}%
\expandafter\endgroup
\ifcase 0\ifx\install y1\fi\expandafter
         \ifx\csname processbatchFile\endcsname\relax\else1\fi
         \ifx\fmtname\x\else 1\fi\relax
\else\csname fi\endcsname
%</ignore>
%<*install>
\input docstrip.tex
\keepsilent
\askforoverwritefalse
\preamble
 ----------------------------------------------------------------
 The siunitx package --- A comprehensive (SI) units package
 Maintained by Joseph Wright
 E-mail: joseph.wright@morningstar2.co.uk
 Released under the LaTeX Project Public License v1.3c or later
 See http://www.latex-project.org/lppl.txt
 ----------------------------------------------------------------

\endpreamble
\Msg{Generating siunitx files:}
\generate{\file{jawltxdoc.sty}{\from{\jobname.dtx}{jawltxdoc}}
}
\usedir{tex/latex/siunitx}
\generate{\file{\jobname.sty}{\from{\jobname.dtx}{package}}
}
\usedir{source/latex/siunitx}
\generate{\file{\jobname.ins}{\from{\jobname.dtx}{install}}
}
\usedir{doc/latex/siunitx}
\generate{\file{si-bug.ltx}{\from{\jobname.dtx}{bug}}
}
\usedir{tex/latex/siunitx/config}
\generate{\file{si-SIunits.cfg}{\from{\jobname.dtx}{SIunits}}
          \file{si-sistyle.cfg}{\from{\jobname.dtx}{sistyle}}
          \file{si-unitsdef.cfg}{\from{\jobname.dtx}{unitsdef}}
          \file{si-units.cfg}{\from{\jobname.dtx}{units}}
          \file{si-hepunits.cfg}{\from{\jobname.dtx}{hepunits}}
          \file{si-fancynum.cfg}{\from{\jobname.dtx}{fancynum}}
          \file{si-fancyunits.cfg}{\from{\jobname.dtx}{fancyunits}}
}
\generate{\file{si-prefix.cfg}{\from{\jobname.dtx}{prefix}}
          \file{si-named.cfg}{\from{\jobname.dtx}{named}}
          \file{si-prefixed.cfg}{\from{\jobname.dtx}{prefixed}}
          \file{si-addn.cfg}{\from{\jobname.dtx}{addn}}
          \file{si-accepted.cfg}{\from{\jobname.dtx}{accepted}}
          \file{si-physical.cfg}{\from{\jobname.dtx}{physical}}
          \file{si-abbr.cfg}{\from{\jobname.dtx}{abbr}}
}
\generate{\file{si-synchem.cfg}{\from{\jobname.dtx}{synchem}}
          \file{si-hep.cfg}{\from{\jobname.dtx}{hep}}
          \file{si-astro.cfg}{\from{\jobname.dtx}{astro}}
          \file{si-geophys.cfg}{\from{\jobname.dtx}{geophys}}
          \file{si-binary.cfg}{\from{\jobname.dtx}{binary}}
}
\generate{\file{si-UK.cfg}{\from{\jobname.dtx}{UK}}
          \file{si-USA.cfg}{\from{\jobname.dtx}{USA}}
          \file{si-DE.cfg}{\from{\jobname.dtx}{DE}}
          \file{si-ZA.cfg}{\from{\jobname.dtx}{ZA}}
}
\nopreamble\nopostamble
\usedir{doc/latex/siunitx}
\generate{\file{README.txt}{\from{\jobname.dtx}{readme}}
          \file{\jobname.bib}{\from{\jobname.dtx}{refs}}
}
\endbatchfile
%</install>
%<*readme>
----------------------------------------------------------------
The siunitx package --- A comprehensive (SI) units package
Maintained by Joseph Wright
E-mail: joseph.wright@morningstar2.co.uk
Released under the LaTeX Project Public License v1.3c or later
See http://www.latex-project.org/lppl.txt
----------------------------------------------------------------

Typesetting values with units requires care to ensure that the
combined mathematical meaning of the value plus unit combination
is clear.  In particular, the SI units system lays down a
consistent set of units with rules on how these are to be used.
However, different countries and publishers have differing
conventions on the exact appearance of numbers (and units).

The siunitx provides a set of tools for authors to typeset
numbers and units in a consistent way.  The package has an
extended set of configuration options which make it possible to
follow varying typographic conventions with the same input
syntax.  The package includes automated processing of numbers
and units, and the ability to control tabular alignment of
numbers.

A number of LaTeX packages have been developed in the past for
formatting units: SIunits, sistyle, unitsdef, unitspkg,
fancyunits and fancynum.  Support for users of all of these
packages is available as emulation modules in siunitx.  In
addition, siunitx can carry out many of the functions of the
dcolumn, rccol and numprint packages.
%</readme>
%<*ignore>
\fi
% Will Robertson's trick
\immediate\write18{bibtex8 --wolfgang \jobname}
\immediate\write18{makeindex -s gind.ist -o \jobname.ind  \jobname.idx}
\immediate\write18{makeindex -s gglo.ist -o \jobname.gls  \jobname.glo}
%</ignore>
%<*driver>
\PassOptionsToClass{a4paper}{article}
\documentclass[german,english,UKenglish]{ltxdoc}
\EnableCrossrefs
\CodelineIndex
\RecordChanges
%\OnlyDescription
% Various adjustments are needed to the packages loaded.  xfrac is
% only loaded if it is available.
\usepackage[final]{graphicx}
% Currently, xfrac is not working with expl3!
%\IfFileExists{xfrac.sty}
%  {\usepackage{xfrac}}
%  {}
\usepackage{jawltxdoc}
\usepackage{textcomp,tikz,pgfplots,tocloft,datatool}
\usepackage[latin1]{inputenc}
\sisetup{
  eVcorra=0.1ex,
  unitsep=cdot,
  alsoload={synchem,hep,binary,astro,geophys},
  loctolang={UK:UKenglish,DE:german}}
\setlength{\cftsubsecnumwidth}{2.5em}
\begin{document}
  \DocInput{\jobname.dtx}
\end{document}
%</driver>
% \fi
%
%\CheckSum{10177}
%
% \CharacterTable
%  {Upper-case    \A\B\C\D\E\F\G\H\I\J\K\L\M\N\O\P\Q\R\S\T\U\V\W\X\Y\Z
%   Lower-case    \a\b\c\d\e\f\g\h\i\j\k\l\m\n\o\p\q\r\s\t\u\v\w\x\y\z
%   Digits        \0\1\2\3\4\5\6\7\8\9
%   Exclamation   \!     Double quote  \"     Hash (number) \#
%   Dollar        \$     Percent       \%     Ampersand     \&
%   Acute accent  \'     Left paren    \(     Right paren   \)
%   Asterisk      \*     Plus          \+     Comma         \,
%   Minus         \-     Point         \.     Solidus       \/
%   Colon         \:     Semicolon     \;     Less than     \<
%   Equals        \=     Greater than  \>     Question mark \?
%   Commercial at \@     Left bracket  \[     Backslash     \\
%   Right bracket \]     Circumflex    \^     Underscore    \_
%   Grave accent  \`     Left brace    \{     Vertical bar  \|
%   Right brace   \}     Tilde         \~}
%
%\GetFileInfo{\jobname.sty}
%
%\DoNotIndex{\@car,\@eha,\@ehc,\@empty,\@firstofone,\@firstoftwo}
%\DoNotIndex{\@for,\@gobble,\@gobbletwo,\@ifl@aded,\@ifnextchar}
%\DoNotIndex{\@ifpackagelater,\@ifpackageloaded,\@ifstar}
%\DoNotIndex{\@ifundefined,\@makeother,\@namedef,\@nameuse,\@ne}
%\DoNotIndex{\@nil,\@secondoftwo,\@temptokena,\@undefined,\AA}
%\DoNotIndex{\active,\addto,\addtolength,\advance,\afterassignment}
%\DoNotIndex{\aftergroup,\AtBeginDocument,\AtEndOfPackage}
%\DoNotIndex{\begingroup,\bfseries,\bgroup,\boldmath,\box}
%\DoNotIndex{\capitalring,\catcode,\cdot,\chardef,\check@mathfonts}
%\DoNotIndex{\circ,\csname,\DeclareFontEncoding,\DeclareFontFamily}
%\DoNotIndex{\DeclareFontShape,\DeclareFontSubstitution}
%\DoNotIndex{\DeclareMathSymbol,\DeclareRobustCommand}
%\DoNotIndex{\DeclareSymbolFont,\DeclareTextSymbol}
%\DoNotIndex{\DeclareTextSymbolDefault,\def,\define@boolkey}
%\DoNotIndex{\define@choicekey,\define@cmdkey,\define@cmdkeys}
%\DoNotIndex{\define@key,\displaystyle,\displaywidth,\divide,\do}
%\DoNotIndex{\edef,\egroup,\else,\encodingdefault,\end,\endcsname}
%\DoNotIndex{\endgroup,\endinput,\endtabular,\ensuremath}
%\DoNotIndex{\expandafter,\f@family,\f@series,\fam,\fi,\font}
%\DoNotIndex{\fontsize,\frac,\gdef,\global,\hbox,\hfil,\hfill,\hskip}
%\DoNotIndex{\hspace,\hss,\ifcsname,\ifdim,\iffalse,\IfFileExists}
%\DoNotIndex{\ifmmode,\ifnum,\ifx,\ignorespaces,\inputencodingname}
%\DoNotIndex{\InputIfFileExists,\key@ifundefine,\let,\long,\m@ne}
%\DoNotIndex{\makeatletter,\math@version,\mathchoice,\mathnormal}
%\DoNotIndex{\mathord,\mathrm,\mathsf,\mathtt,\mdseries}
%\DoNotIndex{\MessageBreak,\mkern,\mp,\NC@find,\NeedsTeXFormat}
%\DoNotIndex{\newbox,\newcolumntype,\newcommand,\newcount,\newdimen}
%\DoNotIndex{\newif,\newlength,\newtoks,\nobreak,\noexpand,\p@}
%\DoNotIndex{\PackageError,\PackageInfo,\PackageWarning,\phantom}
%\DoNotIndex{\ProcessOptionsX,\protect,\protected@edef,\ProvidesFile}
%\DoNotIndex{\ProvidesPackage,\Q,\raisebox,\relax,\renewcommand}
%\DoNotIndex{\RequirePackage,\rlap,\sbox,\scantokens}
%\DoNotIndex{\scriptscriptstyle,\scriptstyle,\selectfont,\setbox}
%\DoNotIndex{\setkeys,\setlength,\settoheight,\sf@size,\sfdefault}
%\DoNotIndex{\sfrac,\skewchar,\sp,\space,\string,\strip@pt}
%\DoNotIndex{\tabularnewline,\text,\textdegree,\textmu,\textohm}
%\DoNotIndex{\textsc,\textstyle,\textsuperscript,\the,\thr@@,\times}
%\DoNotIndex{\ttdefault,\uccode,\unboldmath,\unhbox,\upmu,\Upomega}
%\DoNotIndex{\uppercase,\upshape,\wd,\xdef,\xspace,\z@}
%
%\DoNotIndex{\if,\lowercase,\+,\,,\-,\.,\:,\;,\@argdef,\@cdr}
%\DoNotIndex{\@ifdefinable,\@testopt,\@yargdef,\color,\endlinechar}
%\DoNotIndex{\key@ifundefined,\newrobustcmd,\protected,\scriptspace}
%\DoNotIndex{\usebox,\tw@,\ifinner,\usepackage,\texttt,\textsf}
%\DoNotIndex{\NC@do,\loop,\l@ngrel@x,\g@addto@macro,\documentclass}
%\DoNotIndex{\document\,\begin,\=,\@ehb,\\,\listfiles,\nullfont}
%\DoNotIndex{\normalsize,\futurelet}
%
%\DoNotIndex{\pascalsecond,\amperemetresecond,\ohmmetre}
%\DoNotIndex{\newtonmetre,\newtonmetrenp}
%\DoNotIndex{\kilogramsquaremetre,\kilogramsquaremetrenp}
%
%\changes{v1.0}{2008/06/12}{First official release}
%
%\setkeys{lst}{language=[LaTeX]{TeX},moretexcs={sisetup,num,ang,SI,
%  si,newunit,newpower,newprefix,metre,candela,kelvin,ampere,eV,
%  second,mole,per,kilogram,Square,squared,cubed,cubic,kilo,centi,
%  gram,tothe,toprule,midrule,bottomrule,newton,micro,joule,mega,
%  raiseto,deci,mathnormal,numprint,electronvolt,selectlanguage,
%  kibi,byte,molar,Molar,mmHg,mebi,torr,parsec,lightyear,clight,
%  eVperc,addplot,color,milli,cm,Hz,kg,DeclareRobustCommand,
%  pascal,dalton,addtolocale,DTLnewdb,DTLnewrow,DTLnewdbentry,
%  DTLforeach,DTLiffirstrow,DTLmul,newqualifier,hour,renewunit,
%  degree}}
%
%\title{\currpkg\ --- A comprehensive (SI) units package^^A
%  \thanks{This file describes version \fileversion, last revised
%    \filedate.}}
%\author{Joseph Wright^^A
%  \thanks{E-mail: joseph.wright@morningstar2.co.uk}}
%\date{Released \filedate}
%
%\maketitle
%
%\newcommand*{\SIunits}{\pkg{SIunits}\xspace}
%\newcommand*{\SIstyle}{\pkg{SIstyle}\xspace}
%\newcommand*{\unitsdef}{\pkg{unitsdef}\xspace}
%\newcommand*{\units}{\pkg{units}\xspace}
%\newcommand*{\numprint}{\pkg{numprint}\xspace}
%\begin{abstract}
% Typesetting values with units requires care to ensure that the
% combined mathematical meaning of the value plus unit combination is
% clear.  In particular, the SI units system lays down a consistent
% set of units with rules on how these are to be used.  However,
% different countries and publishers have differing conventions on
% the exact appearance of numbers (and units).
%
% The \currpkg provides a set of tools for authors to typeset numbers
% and units in a consistent way.  The package has an extended set of
% configuration options which make it possible to follow varying
% typographic conventions with the same input syntax.  The package
% includes automated processing of numbers and units, and the ability
% to control tabular alignment of numbers.
%
% A number of \LaTeX\ packages have been developed in the past for
% formatting units: \SIunits, \SIstyle, \unitsdef, \units,
% \pkg{fancyunits} and \pkg{fancynum}. Support for users of all of
% these packages is available as emulation modules in \currpkg.
% In addition, \currpkg can carry out many of the functions of the
% \pkg{dcolumn}, \pkg{rccol} and \numprint packages.
%\end{abstract}
%
%\begin{multicols}{2}
%  \tableofcontents
%\end{multicols}
%
%\clearpage
%\part{Introduction}
% The correct application of units of measurement is very important
% in technical applications.  For this reason, carefully-crafted
% definitions of a coherent units system have been laid down by the
% \latin{Conf\'erence G\'en\'erale des Poids et
% Mesures}\footnote{General Conference on Weights and Measures.}
% (CGPM): this has resulted in the \latin{Syst\`eme International
% d'Unit\'es}\footnote{International System of Units.} (SI).  At the
% same time, typographic conventions for correctly displaying both
% numbers and units exist to ensure that no loss of meaning occurs in
% printed matter.
%
% \currpkg aims to provide a unified method for \LaTeX\ users to
% typeset units and values correctly and easily.  The design
% philosophy of \currpkg is to follow the agreed rules by default,
% but to allow variation through option settings.  In this way,
% users can use \currpkg to follow the requirements of publishers,
% co-authors, universities, \etc without needing to alter the input
% at all.
%
% \currpkg is intended as a complete replacement for \SIunits,
% \SIstyle, \unitsdef, \units, \pkg{fancyunits} and
% \pkg{fancynum}. As such, emulation modes are provided for all of
% these packages.  Where possible, conventions from the existing
% solutions have been used here.  For example, the macros \cs{num},
% \cs{ang} and \cs{SI} act in a very similar fashion to those in
% existing packages.
%
%\clearpage
%\part{Using the \currpkg package}
%
%\section{For the impatient}
% \currpkg provides the user macros:
%\begin{itemize}
%  \item \cs{SI}\oarg{options}\marg{value}\oarg{pre-unit}\marg{unit}
%  \item \cs{si}\oarg{options}\marg{unit}
%  \item \cs{num}\oarg{options}\marg{number}
%  \item \cs{ang}\oarg{options}\marg{angle}
%  \item \cs{sisetup}\marg{options}
%\end{itemize}
% plus the |S| and |s| column types for decimal alignments and units
% in tables.  These macros are designed for typesetting units and
% values with control of appearance and with intelligent processing.
%
% By default, all text is typeset in the current upright, serif maths
% font.  This can be changed by setting the appropriate package
% options: \opt{obeyall} will use the current font for typesetting.
%
% The package includes a ``unit processor'', which allows the use of
% named units or literal values.  Named units are processed to
% correctly include powers.
%\begin{LaTeXexample}
%  \SI{10}{\gram}\\
%  \SI{23.4}{g.cm^3}\\
%  \num{1e34}\\
%  \ang{1;2;3}\\
%  \emph{\SI{16,7}{\metre\per\second}}\\
%  \textbf{\SI{30e3}{\Hz}}\\
%  \SI{1.2 x 3.56 x 9.2}{\milli\metre}\\
%  \sisetup{obeyall}
%  \textbf{\SI{-4.5}{\cm}}\\
%  \si{\joule\per\mole\per\kelvin}\\
%  \si[per=frac]{\joule\per\mole\per\kelvin}\\
%  \num[dp=4]{1.23456}\\
%  \num[dp=4]{9.8}\\
%  \begin{tabular}{S[tabformat=3.2]}
%    \toprule
%    {Heading}\\
%    \midrule
%    1.3 \\
%    134.2 \\
%    3.56 \\
%    74,7 \\
%    \bottomrule
%  \end{tabular}
%\end{LaTeXexample}
%
%\section{Requirements}
% \currpkg requires a reasonably up to date \TeX\ system.  The
% package requires \eTeX-extensions, which should be available on
% most systems.\footnote{If you have an old \LaTeX\, try
% ``\texttt{elatex}'' rather than ``\texttt{latex}''.}  The following
% packages are also needed:
%\begin{itemize}
%  \item \pkg{array} and \pkg{xspace}: from the \pkg{tools} bundle,
%    which should be available to everyone;
%  \item \pkg{xkeyval}: this processes the option handling, and needs
%    to be at least v2.5;
%  \item \pkg{amstext}: from the \AMS\TeX\ support bundle (the
%    \AMS\ fonts are also needed to provide the default upright
%    \si{\micro}).
%\end{itemize}
% Hopefully most people using the package will have access to all of
% those items.
%
% To use the \opt{fraction=sfrac} option, the \pkg{xfrac} package is
% needed.  This needs various experimental \LaTeX3 packages.  As a
% result, \currpkg does not load \pkg{xfrac}.  If you want to use
% \opt{fraction=sfrac}, \emph{you} need to load \pkg{xfrac} in your
% preamble before \currpkg.\footnote{This document has been compiled
% in this way.  You have to load \pkg{xfrac} first as otherwise very
% nasty things happen with \pkg{xkeyval}.  MiK\TeX\ users should note
% that the packaged versions of \pkg{expl3}, \pkg{template} and
% \pkg{xparse} will not work with \pkg{xfrac}: download copies from
% \textsc{ctan}!} If the package is not loaded, \opt{fraction=sfrac}
% falls back on a \pkg{nicefrac}-like method.  The interested user
% should look at the \pkg{xfrac} documentation for reasons this might
% not be ideal.\footnote{On the other hand, some fractional units
% will look really bad with \cs{sfrac}.  Use this option with
% caution.}
%
%\section{Loading the package}
% \currpkg is loaded by the usual \LaTeX\ method.
%\begin{LaTeXexample}[noexample]
%  \usepackage[<options>]{siunitx}
%\end{LaTeXexample}
% As is shown in the example, the package can be loaded with one or
% more options, using the key--value system.  The full range of
% package options are described in Section~\ref{sec:options}; some
% options are described in the along with the appropriate user
% macros.  Most of the user macros accept the same key--value
% settings as an optional argument.
%
%\section{Numbers}
%\DescribeMacro{\num}
% Numbers are automatically formatted by the \cs{num} macro.  This
% takes one optional and one mandatory argument:
% \cs{num}\oarg{options}\marg{number}.  The contents of \meta{number}
% are automatically formatted, in a similar method to that used by
% \numprint.  The formatter removes ``hard'' spaces (\cs{,} and |~|),
% automatically identifies exponents (by default marked using |e| or
% |d|) and adds the appropriate spacing of large numbers.  A leading
% zero is added before a decimal marker, if needed: both ``|.|'' and
% ``|,|'' are recognised as decimal marker.
%\begin{LaTeXexample}
%  \num{1}    \num{123}     \num{1234}  \num{12345}\\
%  \num{0.1}  \num{0.123}   \num{0,1234} \num{.12345}\\
%  \num{1e10} \num{3.45d-4} \num{-e10}
%\end{LaTeXexample}
% Various error-checking systems are built into the package, so that
% if \meta{number} does not contain any numeric characters, a warning
% is issued.  Isolated signs are also detected.  The package
% recognises |(| and |)| as ``extra'' characters, which can be used
% to indicate the error in a number.\footnote{This is common in
% chemical crystallography, for example.}  The \opt{seperr} causes
% this data to be given as a separate error value.  If the number
% also contains an exponent, then brackets are re-added after the
% separation to ensure that meaning is not lost.
%\begin{LaTeXexample}
%  $\num{1.234(5)} = \num[seperr]{1.234(5)}$\\
%  $\num{1.234(5)e6} = \num[seperr]{1.234(5)e6}$
%\end{LaTeXexample}
% The same applies to the unit and value macro \cs{SI}, described
% later, for example the rest mass of an electron \cite{IUPAC}:
%\begin{LaTeXexample}
%   $ m_{\mathrm{e}}
%     = \SI{9.1093897(54)e-31}{\kg} $\\
%   $ m_{\mathrm{e}}
%     = \SI[seperr]{9.1093897(54)e-31}{\kg} $
%\end{LaTeXexample}
%
% A number of effects are available as options.  These are fully
% explained in Section~\ref{sec:options}.  Some of the more useful
% options are illustrated here.  By default, the output of the
% package is typeset in maths mode.  However, the use of the current
% text font can be forced.\footnote{This document is typeset using
% lowercase numbers in text mode, which emphasises the effect here.}
%\begin{LaTeXexample}
%  \num{1234567890} \num[mode=text]{1234567890}
%\end{LaTeXexample}
% \currpkg can automatically add zeros and signs to numbers.  This
% can be altered as desired.
%\begin{LaTeXexample}
%  \num{1.}  \num[padnumber=all]{1.}\\
%  \num{2}   \num[addsign=all]{2}\\
%  \num{3e4} \num[addsign=mant]{3e4} \num[addsign=all]{3e4}\\
%  \num{.5}  \num[padnumber=none]{.5}
%\end{LaTeXexample}
% The separation of digits can be turned on and off, and the output
% changed.
%\begin{LaTeXexample}
%  \num{1234}  \num[sepfour=true]{1234}\\
%  \num{12345} \num[digitsep=comma]{12345}\\
%  \num[digitsep=none]{12345}
%\end{LaTeXexample}
% The formatting of exponents is also customisable.
%\begin{LaTeXexample}
%  \num{1e10} \num[expproduct=cdot]{1e10}\\
%  \num{2e20} \num[expbase=5]{2e20}\\
%  \num{3e30} \num[expproduct=tighttimes]{3e30}
%\end{LaTeXexample}
%
% \currpkg can automatically add colour to negative numbers, which is
% often useful for highlighting purposes.  This is turned on with the
% \opt{colourneg} option; the colour used is set by \opt{negcolour}.
% Both of these are available with the US spellings: \opt{colorneg}
% and \opt{negcolor}.
%\begin{LaTeXexample}
%  \num{-1}\\
%  \sisetup{colourneg} \num{-2}\\
%  \num{3e-3}\\
%  \num[negcolour=blue]{-4}
%\end{LaTeXexample}
%
% \currpkg can automatically zero-fill and round to a fixed number of
% decimal places.  This is controlled by two options \opt{fixdp} and
% \opt{dp}.  The later is an integer which specifies how many places
% to fix to; setting this option automatically sets \opt{fixdp} to
% \opt{true}.  The place-fixing system will only alter pure numbers: for
% example, any error component will result in the input being left
% unchanged.
%\begin{LaTeXexample}
%  \num{1.23456}\\
%  \sisetup{dp=2}
%  \num{1.23456}\\
%  \num{9.8}\\
%  \num{-10.432}\\
%  \num{44.3221(2)}
%\end{LaTeXexample}
%
%\section{Tabular material}
%\subsection{Aligning numbers}
% Centring numbers in tabular content is handled by a new column
% type, the |S| column.  This is based closely on the \pkg{dcolumn}
% method for centring numbers in columns, but adds the functionality
% of the \cs{num} macro.\footnote{The approach used is actually a
% combination of \pkg{dcolumn} for centring the material and
% \numprint for processing it.  It will therefore give rather
% different results than the \texttt{n} and \texttt{N} column types
% in \numprint.}
%
% By default, the decimal marker of the number is placed at the
% centre of the column, which then resizes to accommodate the width
% of the contents (\ref{tab:default}).  This behaviour is set by the
% \opt{tabnumalign=centredecimal} option.  By setting the
% \opt{tabnumalign} option to \opt{centre}, the centre of the space
% reserved for the number is placed at the centre of the column.  The
% space reserved is stored in \opt{tabformat}, which is of the form
% \texttt{\meta{before}\meta{dec}\meta{after}}, where \meta{before}
% is the number of characters before the decimal marker and
% \meta{after} is the number after. Thus in the example,
% \opt{tabformat=2.4} provides space for two digits before the
% decimal marker and four after.  \opt{tabnumalign} can also be set
% to \opt{left} and \opt{right}, with the expected results.
%\begin{LaTeXexample}[float]
%  \begin{table}
%    \caption{Behaviour of \texttt{S} column type}
%    \label{tab:default}
%    \centering
%    \begin{tabular}{%
%      S%
%      S[tabnumalign=centre,tabformat=2.4]%
%      S[tabnumalign=right,tabformat=2.4]%
%      S[tabnumalign=centre,tabformat=2.4,decimalsymbol=comma]}
%    \toprule
%    {Some Values} & {Some Values} & {Some Values} & {Some Values} \\
%    \midrule
%     2.3456  &  2.3456  &  2.3456 &  2.3456 \\
%    34.2345  & 34.2345  & 34.2345 & 34.2345 \\
%    56.7835  & 56.7835  & 56.7835 & 56.7835 \\
%    90.473   & 90.473   & 90.473  & 90.473  \\
%    \bottomrule
%    \end{tabular}
%  \end{table}
%\end{LaTeXexample}
% The \opt{tabformat} setting can also be used to reserve space for
% numbers containing exponents.  This is given in the same format as
% above, but with a mantissa and exponent part (\ref{tab:exptab}).
% Notice that this is designed to expect that numbers will contain a
% mantissa.  Exponents can either be aligned so that the ``$\times$''
% symbols match up vertically, or the exponent part can be allowed to
% move across as needed. Space for signs is added by using any sign
% in the \opt{tabformat}, so for example \opt{tabformat=+2.2} and
% \opt{tabformat=-2.2} have exactly the same effect. Setting
% \opt{tabformat} will automatically switch \opt{tabnumalign} from
% \opt{centredecimal} to \opt{centre}, if the former is
% currently set. In other cases, the current alignment option is
% retained.
%\begin{LaTeXexample}[float]
%  \begin{table}
%    \caption{Exponents in tables}
%    \label{tab:exptab}
%    \centering
%    \begin{tabular}{%
%      S[tabnumalign=right,tabformat=2.2e2]%
%      S[tabnumalign=centre,tabformat=2.2e1.1]%
%      S[tabnumalign=centre,tabformat=2.2e1.1,tabalignexp=false]%
%      S[tabnumalign=centre,tabformat=+2.2]}
%    \toprule
%        {Longer values}
%      & {Longer values}
%      & {Longer values}
%      & {Values} \\
%    \midrule
%    2.3e1    &  2.34e1  &  2.34e1  & +2.31 \\
%    34.23e45 & 34.23e45 & 34.23e45 & 34.23 \\
%    56.78    & 56.78    & 56.78    & -56.78 \\
%    1.0e34   & 1.0e34   & 1.0e34   & +-1.0 \\
%    \bottomrule
%    \end{tabular}
%  \end{table}
%\end{LaTeXexample}
%
% Data not to be processed as a number should be protected by
% wrapping it in braces: this is most likely to be true for column
% headers (again as illustrated).  By default, the contents of
% non-numeric cells are centred.  This can be altered by setting
% \opt{tabtextalign}, which can be set to \opt{left}, \opt{right} or
% \opt{centre}.  The use of digit separators in table columns is
% accounted for: extra space is reserved if digit separators will be
% added.
%
%\subsection{Columns of units}
% As a complement to the |S| column, \currpkg also provides a second
% column type, |s|.  This is intended for producing columns of units.
% The letters chosen are intended to be similar to \cs{SI} and
% \cs{si}, respectively.  The alignment of material in |s| columns is
% governed by the \opt{tabunitalign} option.
%\begin{LaTeXexample}[float]
%  \begin{table}
%  \centering
%  \caption{Number and units in tables}
%  \label{tab:num-unit}
%    \begin{tabular}{%
%      S[tabformat=1.2e-1,tabnumalign=centre]%
%      s[tabunitalign=left]}
%    \toprule
%      {Value} & \multicolumn{1}{c}{Unit} \\
%    \midrule
%      2.16e-5 & \metre\squared\per\second \\
%      2.83e-6 & \metre\squared\per\second \\
%      7.39e3  & \pascal\cubic\metre\per\mole \\
%      1.0e5   & \pascal \\
%    \bottomrule
%    \end{tabular}
%  \end{table}
%\end{LaTeXexample}
%
% As the \cs{si} macro can take literal or macro input, the |s|
% column cannot validate the input.  \emph{Everything} in an |s|
% column is therefore passed to the \cs{si} macro for processing.  To
% prevent this, you have to use \cs{multicolumn}, as is shown in
% \ref{tab:s-limits}.  Notice that the braces do not prevent
% processing and colouring of the cell contents.
%\begin{LaTeXexample}[float]
%  \begin{table}
%  \centering
%  \caption{The \texttt{s} column processes everything}
%  \label{tab:s-limits}
%    \begin{tabular}{%
%      s[colourall,colour=orange]%
%      s[colourall,colour=orange]}
%    \toprule
%      {Unit}    & \multicolumn{1}{c}{Unit}\\
%    \midrule
%      {m^3}     & \multicolumn{1}{c}{\si{m^3}} \\
%      \kilogram & \kilogram \\
%    \bottomrule
%    \end{tabular}
%  \end{table}
%\end{LaTeXexample}
%
%\section{Angles}
%\DescribeMacro{\ang}
% Angles can be typeset using the \cs{ang} command.  This takes two
% arguments, \cs{ang}\oarg{options}\marg{angle}, where \meta{options}
% can be any of the package options to apply only to this value.
% \meta{angle} can be given either as a decimal number or as a
% semi-colon separated list of degrees, minutes and seconds, \ie\
% \cs{ang}\marg{decimal angle} or
% \cs{ang}|{|\meta{degrees}|;|\meta{minutes}|;|\meta{seconds}|}|. By
% default, no space is introduced between angles and the degrees,
% minutes and seconds markers.
%\begin{LaTeXexample}
%  \ang{10} \ang{12.3} \ang{4,5}\\
%  \ang{1;2;3} \ang{;;1}\\
%  \ang{+10;;} \ang{-0;1;}
%\end{LaTeXexample}
%
% By default, angles with no degrees (or minutes) are zero-filled;
% angles with degrees but no minutes or seconds are not filled.  This
% behaviour can be altered using the package options.
%\begin{LaTeXexample}
%  \ang{;;1} \ang[padangle=none]{;;1}\\
%  \ang{2;;} \ang[padangle=all]{2;;}\\
%  \sisetup{padangle=all} \ang{;3;} \ang{4;;} \ang{;;5}
%\end{LaTeXexample}
% The \cs{num} macro is used to typeset each number of the angle, so
% the options for \cs{num} also apply here.  The \opt{anglesep} value
% can be used to separate degrees, minutes and seconds.
%\begin{LaTeXexample}
%  \ang{1.05}    \ang[decimalsymbol=comma]{1.05}\\
%  \ang{3.67890} \ang[digitsep=comma]{3.67890}\\
%  \ang{9;8;7}   \ang[anglesep=thin]{9;8;7}
%\end{LaTeXexample}
% The degrees, minutes and seconds signs can be placed over the
% decimal sign using the \opt{astroang} option.  This is designed on
% the assumption that only the last number given has a decimal
% part.
%\begin{LaTeXexample}
%  \ang{1.2} \ang[astroang]{1.2}\\
%  \ang{1;2.3;} \ang[astroang]{1;2.3;}\\
%  \ang{1;2;3.4} \ang[astroang]{1;2;3.4}
%\end{LaTeXexample}
%
%\section{Units and values}
%\DescribeMacro{\SI}
% The core aim of \currpkg is correctly typesetting values which have
% units. The main output macro here is \cs{SI}, which has the same
% syntax as the macros with the same name in \SIstyle and \unitsdef
% packages. The \cs{SI} macro takes two mandatory arguments, in
% addition to the optional set up argument, and a second optional
% argument:
% \cs{SI}\oarg{options}\marg{number}\oarg{preunit}\marg{unit}.  The
% \meta{number} argument operates in exactly the same manner as the
% equivalent argument of the \cs{num} macro.  \meta{unit} will be
% typeset with a non-breakable space between it and the preceding
% number, with font control as outlined earlier.  Finally,
% \meta{preunit} is a unit to be typeset \emph{before} the numerical
% value (most likely to be a currency).  Some examples illustrate the
% general power of the macro.
%\begin{LaTeXexample}
%  \SI[mode=text]{1.23}{J.mol^{-1}.K^{-1}}\\
%  \SI{.23e7}{\candela}\\
%  \SI[per=slash]{1.99}[\pounds]{\per\kilogram}\\
%  \SI{70}{\metre\per\second}\\
%  \SI[per=frac,fraction=nice]{1,345}{\ampere\per\mole}
%\end{LaTeXexample}
% The use of unit macros outside of the \cs{SI} macro is described
% later.
%
%\subsection{Literal units}
%\changes{v1.1}{2008/07/23}{Unit processor extended to interpret
%  \texttt{\textunderscore} correctly}
% Units can be input in two ways, inspired by \SIstyle and \SIunits.
% The \SIstyle-like method uses literal input.  Five characters have
% a special meaning:
%\begin{itemize}
%  \item ``|^|'' and ``|_|''  The superscript and subscript
%    characters can be used without the usual
%    need for surrounding maths characters (|$|);^^A$
%  \item ``|.|'' and ``|,|'': the full stop (point) symbol and comma
%    are made active, and produce the current contents of the
%    \opt{unitsep} option;
%  \item ``|~|'' The contents of the \opt{unitspace} option are typeset
%    by a tilde.
%\end{itemize}
% This allows ready input of units.
%\begin{LaTeXexample}
%  \SI{10}{kg.m.s^{-2}}\\
%  \SI{1.453}{g/cm^3}\\
%  \SI{6.345}{kg_{pol}.mmol_{cat}^{-1}}\\
%  \SI{33.562}{cd~s}\\
%  \SI[unitsep=medium]{100}{m.s^{-2}}
%\end{LaTeXexample}
% The literal unit system will correctly typeset input containing the
% symbols \si{\micro} (micro), \si{\degree} (degree) and
% \si{\angstrom} (ring-A).\footnote{Currently this works with \XeTeX\
% and \pkg{inputenc} using the \texttt{latin1}, \texttt{latin5} and
% \texttt{latin9} encodings.}
% \begingroup\makeatletter\si@unt@nonlatin
%\begin{LaTeXexample}
%  \SI{10}{�m}\\
%  \SI{20}{�C}\\
%  \SI{30}{�}
%\end{LaTeXexample}
%\endgroup
%
%\subsection{The unit interpreter}
% The second operation mode for the \cs{SI} macro is based on the
% behaviour of \SIunits.  Here, each unit, SI multiple prefix and
% power is given a macro name.  These are entered in a method very
% similar to the reading of the unit name in English.
%\begin{LaTeXexample}
%  \SI{10}{\kilo\gram\metre\per\second\squared}\\
%  \SI{1.453}{\gram\per\cubic\centi\metre}\\
%  \SI{33.562}{\candela\second}\\
%  \SI[unitsep=medium]{100}{\metre\per\Square\second}\\
%  \SI[prefixsymbolic=false]{4.56}{\kilo\metre\per\second}
%\end{LaTeXexample}
% On its own, this is very similar to \SIunits, and is less
% convenient than the direct input method.\footnote{Users of \SIunits
% should note the lack of need for a \cs{usk}-type macro.}  However,
% the package allows you to define new unit macros; a large number of
% pre-defined abbreviations are also supplied.  More importantly, by
% defining macros for units, instead of literal values, new
% functionality is made available.  Units may be re-defined to give
% different output, and handling of reciprocal values can be altered.
%\begin{LaTeXexample}
%  \SI[per=frac,fraction=frac]{10}{\gram\metre\per\second\squared}\\
%  \SI[per=slash]{1.453}{\gram\per\cubic\centi\metre}\\
%  \SI{33.562}{\candela\second}\\
%  \SI[per=frac,fraction=nice]{100}{\metre\per\Square\second}
%\end{LaTeXexample}
% The unit processor will trap \emph{some} errors in the input and
% give the ``best guess'' result.  However, it is down to the user to
% check the output.
%
%\subsection{Powers of units}
%\DescribeMacro{\Square}
%\DescribeMacro{\ssquare}
%\DescribeMacro{\squared}
%\DescribeMacro{\cubic}
%\DescribeMacro{\cubed}
% Including powers in units is handled using a ``natural language''
% method.  Thus preceding a unit by \cs{Square} or \cs{cubic} which
% raise the unit to the appropriate power, while \cs{squared} or
% \cs{cubed} follow the unit they apply to. The \cs{Square} macro is
% capitalised to avoid a name clash with \pkg{pstricks}; the
% alternative \cs{ssquare} is also provided.
%\begin{LaTeXexample}
%  \SI{10}{\metre\squared}\\
%  \SI{20}{\Square\metre}\\
%  \SI{30}{\metre\cubed}\\
%  \SI{40}{\cubic\metre}
%\end{LaTeXexample}
%
%\DescribeMacro{\per}
% The \cs{per} macro intelligently creates reciprocal powers, and
% also adds the power $-1$ when appropriate.
%\begin{LaTeXexample}
%  \SI{10}{\per\second\squared}\\
%  \SI{20}{\per\Square\second}\\
%  \SI[per=frac,fraction=nice]{30}{\per\second\cubed}\\
%  \SI[per=slash]{40}{\per\cubic\second}\\
%  \SI{50}{\per\second}\\
%  \SI{60}{\per\metre\Square\candela}
%\end{LaTeXexample}
%
%\DescribeMacro{\tothe}
%\DescribeMacro{\raiseto}
% For powers not defined above or with \cs{newpower}, the \cs{tothe}
% macro can be used ``in line'' to produce a power.  As follows from
% standard English usage, this comes after the unit. \cs{raiseto}
% achieves the same, but is used \emph{before} a unit to add a power
% \emph{after}.\footnote{\cs{raiseto} acts in the same way as
% \cs{tothe} when used in a literal context: the power will be
% produced where the macro is, rather than moving after the next
% item.}
%\begin{LaTeXexample}
%  \SI{16.86}{\metre\tothe{4}}\\
%  \SI{7.895}{\raiseto{-6}\newton}\\
%  \SI{1.34}{\per\kelvin\tothe{7}}
%\end{LaTeXexample}
%
%\subsection{Units with no values}
%\DescribeMacro{\si}
% For typesetting the symbol for a unit on its own, with the full
% font control and without extra spaces, the \cs{si} macro is
% provided.\footnote{The same effect can be achieved using the
% \cs{SI} macro with an empty numerical argument.}  The macro name
% avoids a clash with the functionality of the earlier packages, but
% is similar to \cs{ilu} from the \unitsdef package.
%\begin{LaTeXexample}
%  \SI{}{kg.m/s^2}\\
%  \si{kg.m/s^2}\\
%  \si[mode=text,unitsep=thin]{\mole\per\cubic\deci\metre}
%\end{LaTeXexample}
%
%\subsection{Free-standing units}
% Users of the \unitsdef package will be accustomed to using unit
% macros on their own (following a value) or with an optional
% argument containing a number.  In both cases, only a single unit
% macro could be used. \currpkg supports both operation modes, with
% the limitation that units trailing values loose font control of the
% value.  When used in this way, the units \emph{do not} take an
% optional \pkg{keyval} argument.
%\begin{LaTeXexample}
%  \sisetup{prespace,allowoptarg}
%  123\metre\\
%  \kelvin[123]\\
%  \sisetup{mode=text} \ampere[234]\\
%  6\second
%\end{LaTeXexample}
%
%\subsection{Pre-defined units, prefixes and powers}
%\newcommand*{\csindex}[1]{^^A
%  \cs{#1}^^A
%  \expandafter\SpecialUsageIndex\csname#1\endcsname}
%\newcommand*{\unitinfo}[1]{^^A
%  #1 & \csindex{#1} &
%  \expandafter\si\expandafter{\csname#1\endcsname}}
%\begin{table}
%  \caption{The seven base SI units}
%  \label{tab:baseunits}
%  \centering
%  \begin{tabular}{lll}
%    \toprule
%      \multicolumn{1}{c}{Unit} &
%      \multicolumn{1}{c}{Macro} &
%      \multicolumn{1}{c}{Symbol} \\
%    \midrule
%    \unitinfo{kilogram}\\
%    \unitinfo{metre}\\
%    \unitinfo{second}\\
%    \unitinfo{mole}\\
%    \unitinfo{kelvin}\\
%    \unitinfo{ampere}\\
%    \unitinfo{candela}\\
%    \bottomrule
%  \end{tabular}
%\end{table}
%\DescribeMacro{\metre}
%\DescribeMacro{\meter}
% The package always defines the seven base SI units, irrespective of
% any package options given (\ref{tab:baseunits}). The kilogram is
% notable as by default it is a \emph{base} unit with a prefix. Thus,
% when the package is loaded with the option \opt{load=\{\}},
% \cs{kilo} and \cs{gram} \emph{are not defined}.  As metre is often
% spelled as ``meter'' in the US, the macro \cs{meter} is provided in
% addition to the \cs{metre} macro.\footnote{The official SI spelling
% for the unit is ``metre''.}
%
% By default, a number of additional definitions are created by the
% package.  These are controlled by the \opt{load} and \opt{noload}
% options. Unless specifically requested with the option
% \opt{noload=prefix}, \currpkg defines the standard prefixes for
% powers of ten (\ref{tab:prefix}).  This leads to the redefinition
% of \cs{kilogram} as \cs{kilo}\cs{gram}.  The macro \cs{deka} is
% provided, as this is used as an alias for \cs{deca} in some places.
% The package also defines a number of derived SI units which have
% assigned names and symbols (\ref{tab:derived}). Note that \cs{Gray}
% is capitalised to avoid a name clash with the \pkg{pstricks}
% package.\footnote{The macros \cs{ohm} and \cs{celsius} are not
% defined by \currpkg if the \pkg{gensymb} package is loaded.}
%\newcommand*{\prefixunitsym}[1]{\si[prefixsymbolic=false]{#1}}
%\newcommand*{\prefixinfo}[1]{^^A
%  #1 & \csindex{#1} &
%    \expandafter\prefixunitsym\expandafter{\csname#1\endcsname} &
%    \expandafter\si\expandafter{\csname#1\endcsname}}
%\begin{table}
%  \caption{The SI prefixes (\opt{load=prefix})}
%  \label{tab:prefix}
%  \centering
%  \begin{tabular}{lllclllc}
%    \toprule
%      \multicolumn{1}{c}{Prefix} &
%      \multicolumn{1}{c}{Macro} &
%      \multicolumn{1}{c}{Power} &
%      Symbol &
%      \multicolumn{1}{c}{Prefix} &
%      \multicolumn{1}{c}{Macro} &
%      \multicolumn{1}{c}{Power} &
%      Symbol \\
%    \midrule
%    \prefixinfo{yocto} &
%    \prefixinfo{deca} \\
%    \prefixinfo{zepto} &
%    \prefixinfo{hecto} \\
%    \prefixinfo{atto} &
%    \prefixinfo{kilo} \\
%    \prefixinfo{femto} &
%    \prefixinfo{mega} \\
%    \prefixinfo{pico} &
%    \prefixinfo{giga} \\
%    \prefixinfo{nano} &
%    \prefixinfo{tera} \\
%    \prefixinfo{micro} &
%    \prefixinfo{peta} \\
%    \prefixinfo{milli} &
%    \prefixinfo{exa} \\
%    \prefixinfo{centi} &
%    \prefixinfo{zetta} \\
%    \prefixinfo{deci} &
%    \prefixinfo{yotta} \\
%    \bottomrule
%  \end{tabular}
%\end{table}
%\begin{table}
%  \caption{The derived SI units with defined names
%    (\opt{load=derived})}
%  \label{tab:derived}
%  \centering
%  \begin{tabular}{llllll}
%    \toprule
%      \multicolumn{1}{c}{Unit} &
%      \multicolumn{1}{c}{Macro} &
%      \multicolumn{1}{c}{Symbol} &
%      \multicolumn{1}{c}{Unit} &
%      \multicolumn{1}{c}{Macro} &
%      \multicolumn{1}{c}{Symbol} \\
%    \midrule
%    \unitinfo{becquerel} &
%    \unitinfo{newton} \\
%    \unitinfo{celsius} &
%    \unitinfo{ohm} \\
%    \unitinfo{coulomb} &
%    \unitinfo{pascal} \\
%    \unitinfo{farad} &
%    \unitinfo{radian} \\
%    \unitinfo{Gray} &
%    \unitinfo{siemens} \\
%      & \csindex{ggray} & \ggray &
%    \unitinfo{sievert} \\
%    \unitinfo{hertz} &
%    \unitinfo{steradian} \\
%    \unitinfo{henry} &
%    \unitinfo{tesla} \\
%    \unitinfo{joule} &
%    \unitinfo{volt} \\
%    \unitinfo{katal} &
%    \unitinfo{watt} \\
%    \unitinfo{lumen} &
%    \unitinfo{weber} \\
%    \unitinfo{lux} \\
%    \bottomrule
%  \end{tabular}
%\end{table}
%
% In addition to these units, there are three other groups of units
% for use with the SI system which do not fit into the above.  These
% are those derived from physical measurements
% (\ref{tab:experimental}), those considered ``accepted''
% (\ref{tab:accepted}), and those accepted temporarily
% (\ref{tab:addn}).\footnote{These are supposed to be replaced over
% time by SI units.}
%\DescribeMacro{\litre}
%\DescribeMacro{\liter}
% The unit ``litre'' is often spelled ``liter'' in the US; both
% spellings are provided by \currpkg, with \cs{liter} giving
% \si{\liter} and \cs{litre} producing \si{\litre}.
%\begin{table}
%  \caption{Units derived from experiments
%    (\opt{load=physical})}
%  \label{tab:experimental}
%  \centering
%  \begin{tabular}{lll}
%    \toprule
%      \multicolumn{1}{c}{Unit} &
%      \multicolumn{1}{c}{Macro} &
%      \multicolumn{1}{c}{Symbol} \\
%    \midrule
%    electron volt & \csindex{electronvolt} & \electronvolt \\
%    unified atomic mass unit
%      & \csindex{atomicmassunit} & \atomicmassunit \\
%      & \csindex{atomicmass} & \atomicmass \\
%    \bottomrule
%  \end{tabular}
%\end{table}
%\begin{table}
%  \caption{Units accepted for use with SI
%    (\opt{load=accepted})}
%  \label{tab:accepted}
%  \centering
%  \begin{tabular}{lll}
%    \toprule
%      \multicolumn{1}{c}{Unit} &
%      \multicolumn{1}{c}{Macro} &
%      \multicolumn{1}{c}{Symbol} \\
%    \midrule
%    \unitinfo{bel} \\
%    day & \csindex{Day} & \Day \\
%     & \csindex{dday} & \dday \\
%    \unitinfo{degree} \\
%    \unitinfo{hour} \\
%    \unitinfo{litre} \\
%     & \csindex{liter} & \liter \\
%    \unitinfo{minute} \\
%    minute (arc) & \csindex{arcmin} & \arcmin \\
%    \unitinfo{neper} \\
%    \unitinfo{percent} \\
%    second (arc) & \csindex{arcsec} & \arcsec \\
%    \unitinfo{tonne} \\
%    \bottomrule
%  \end{tabular}
%\end{table}
%\begin{table}
%  \centering
%  \caption{Additional (temporary) SI units
%    (\opt{load=addn})}
%  \label{tab:addn}
%  \begin{tabular}{lll}
%    \toprule
%      \multicolumn{1}{c}{Unit} &
%      \multicolumn{1}{c}{Macro} &
%      \multicolumn{1}{c}{Symbol} \\
%    \midrule
%    {\aa}ngstr{\"o}m & \csindex{angstrom} & \angstrom \\
%    \unitinfo{are} \\
%    \unitinfo{curie} \\
%    bar & \csindex{BAR} & \BAR \\
%     & \csindex{bbar} & \bbar \\
%    \unitinfo{barn} \\
%    \unitinfo{gal} \\
%    \unitinfo{hectare} \\
%    \unitinfo{millibar} \\
%    \unitinfo{rad} \\
%    \unitinfo{rem} \\
%    \unitinfo{roentgen} \\
%    \bottomrule
%  \end{tabular}
%\end{table}
%
%\subsection{Prefixed and abbreviated units}
% Many basic units have prefixes which are commonly used with the
% unit, such as centimetre or megahertz.  The package therefore
% defines a number of common prefixed units (\opt{load=prefixed}).
% Several of these also have obvious abbreviations (such as \cs{MHz}
% for \cs{megahertz}), which are made available by \opt{load=abbr}.
% In common with the units discussed above, the prefixed and
% abbreviated unit definitions are loaded by default.
%\newcommand*{\tablesubhead}[1]{^^A
%  \addlinespace \emph{#1} \\*}
%\begin{longtable}{llll}
%    \caption{Prefixed (\opt{load=prefixed}) and abbreviated
%      (\opt{load=abbr}) units\label{tab:abbr}}\\
%    \toprule
%    Unit & Macro & Symbol & Abbreviation \\
%    \midrule
%  \endfirsthead
%    \toprule
%    Unit & Macro & Symbol & Abbreviation \\
%    \midrule
%  \endhead
%    \bottomrule
%    \multicolumn{4}{r}{\emph{Continued on next page}}
%  \endfoot
%    \bottomrule
%  \endlastfoot
%
%  \tablesubhead{Masses}
%  \unitinfo{kilogram} & \csindex{kg} \\
%  \unitinfo{femtogram} & \csindex{fg} \\
%  \unitinfo{picogram} & \csindex{pg} \\
%  \unitinfo{nanogram} & \csindex{nanog} \\
%  \unitinfo{microgram} & \csindex{micg} \\
%  \unitinfo{milligram} & \csindex{mg} \\
%  atomic mass & \csindex{atomicmass} & \amu & \csindex{amu} \\
%
%  \tablesubhead{Lengths}
%  \unitinfo{picometre} & \csindex{picom} \\
%  \unitinfo{nanometre} & \csindex{nm} \\
%  \unitinfo{micrometre} & \csindex{micm} \\
%  \unitinfo{millimetre} & \csindex{mm} \\
%  \unitinfo{centimetre} & \csindex{cm} \\
%  \unitinfo{decimetre} & \csindex{dm} \\
%  \unitinfo{kilometre} & \csindex{km} \\
%
%  \tablesubhead{Times}
%  \unitinfo{second} & \csindex{Sec} \\
%  \unitinfo{attosecond} & \csindex{as} \\
%  \unitinfo{femtosecond} & \csindex{fs} \\
%  \unitinfo{picosecond} & \csindex{ps} \\
%  \unitinfo{nanosecond} & \csindex{ns} \\
%  \unitinfo{microsecond} & \csindex{mics} \\
%  \unitinfo{millisecond} & \csindex{ms} \\
%
%  \tablesubhead{Moles}
%  \unitinfo{femtomole} & \csindex{fmol} \\
%  \unitinfo{picomole} & \csindex{pmol} \\
%  \unitinfo{nanomole} & \csindex{nmol} \\
%  \unitinfo{micromole} & \csindex{micmol} \\
%  \unitinfo{millimole} & \csindex{mmol} \\
%  \unitinfo{kilomole} & \csindex{kmol} \\
%
%  \tablesubhead{Currents}
%  \unitinfo{picoampere} & \csindex{pA} \\
%  \unitinfo{nanoampere} & \csindex{nA} \\
%  \unitinfo{microampere} & \csindex{micA} \\
%  \unitinfo{milliampere} & \csindex{mA} \\
%  \unitinfo{kiloampere} & \csindex{kA} \\
%
%  \tablesubhead{Areas}
%  square centimetre & \csindex{squarecentimetre} &
%    \squarecentimetre & \csindex{cms} \\*
%    & \csindex{centimetresquared} & \centimetresquared \\
%  square metre & \csindex{squaremetre} & \squaremetre \\
%  square kilometre & \csindex{squarekilometre} & \squarekilometre \\
%
%  \tablesubhead{Volumes}
%\changes{v1.1d}{2008/10/29}{New units \cs{microliter}, \cs{milliliter},
%  \cs{micL} and \cs{mL}}
%  \unitinfo{microlitre} & \csindex{micl} \\
%  \unitinfo{millilitre} & \csindex{ml} \\
%  \unitinfo{microliter} & \csindex{micL} \\
%  \unitinfo{milliliter} & \csindex{mL} \\
%  cubic centimetre & \csindex{cubiccentimetre} &
%    \cubiccentimetre & \csindex{cmc} \\*
%    & \csindex{centimetrecubed} & \centimetrecubed \\
%  cubic decimetre & \csindex{cubicdecimetre} &
%    \cubicdecimetre & \csindex{dmc} \\
%
%  \tablesubhead{Frequencies}
%  \unitinfo{hertz} & \csindex{Hz} \\
%  \unitinfo{millihertz} & \csindex{mHz} \\
%  \unitinfo{kilohertz} & \csindex{kHz} \\
%  \unitinfo{megahertz} & \csindex{MHz} \\
%  \unitinfo{gigahertz} & \csindex{GHz} \\
%  \unitinfo{terahertz} & \csindex{THz} \\
%
%  \tablesubhead{Potentials}
%  \unitinfo{millivolt} & \csindex{mV} \\
%  \unitinfo{kilovolt} & \csindex{kV} \\
%
%  \tablesubhead{Energies}
%  \unitinfo{kilojoule}  & \csindex{kJ}\\
%  \unitinfo{electronvolt} & \csindex{eV} \\
%  \unitinfo{millielectronvolt} & \csindex{meV} \\
%  \unitinfo{kiloelectronvolt} & \csindex{keV} \\
%  \unitinfo{megaelectronvolt} & \csindex{MeV} \\
%  \unitinfo{gigaelectronvolt} & \csindex{GeV} \\
%  \unitinfo{teraelectronvolt} & \csindex{TeV} \\
%  \unitinfo{kilowatthour} & \cs{kWh} \\
%
%  \tablesubhead{Powers}
%  \unitinfo{milliwatt} \\
%  \unitinfo{kilowatt} \\
%  \unitinfo{megawatt} \\
%
%  \tablesubhead{Capacitances}
%  \unitinfo{femtofarad} \\
%  \unitinfo{picofarad} \\
%  \unitinfo{nanofarad} \\
%  \unitinfo{microfarad} \\
%  \unitinfo{millifarad} \\
%
%  \tablesubhead{Resistances}
%  \unitinfo{kilohm} \\
%  \unitinfo{megohm} \\
%  \unitinfo{gigaohm} \\
%  \unitinfo{millisiemens} \\
%
%  \tablesubhead{Forces}
%  \unitinfo{millinewton} \\
%  \unitinfo{kilonewton} \\
%
%  \tablesubhead{Other units}
%  \unitinfo{hectopascal} \\
%  \unitinfo{megabecquerel} \\
%  \unitinfo{millisievert} \\
%\end{longtable}
%
%\subsection{Defining new units}
%\DescribeMacro{\newunit}
%\DescribeMacro{\renewunit}
%\DescribeMacro{\provideunit}
% New units are produced using the \cs{newunit} macro.  This works as
% might be expected:
% \cs{newunit}\oarg{options}\marg{unit}\marg{symbol}, where
% \meta{symbol} can contain literal values, other units, multiple
% prefixes, powers and \cs{per}.  The \meta{options} argument can be
% any suitable options, and applies the specific unit macro only. The
% most obvious example for using this macro is the \cs{degree}
% unit.\footnote{Although the \cs{ang} macro is preferred for this
% job.}  The (first) optional argument to \cs{SI} and \cs{si} can be
% used to override the settings for the unit.  The \cs{renewunit} and
% \cs{provideunit} macros take the same arguments.
%\begin{LaTeXexample}
%  \SI{3.1415}{\degree}\\
%  \newunit[valuesep=none]{\oddunit}{XXX}
%  \SI{12345}{\oddunit}
%  \SI[valuesep=thick]{67890}{\oddunit}
%\end{LaTeXexample}
%
% As with the \LaTeX\ commands \cs{newcommand}, \etc, the choice of
% \cs{newunit}, \cs{renewunit} or \cs{provideunit} depends on the
% presence of an existing definition.  While \cs{newunit} should be
% used when a unit has not been previously defined, \cs{renewunit}
% will issue a warning if the named unit does not already exist.
% \cs{provideunit} defines the unit if it does not exist, and
% otherwise does nothing at all.  The same behaviour is seen with
% \cs{providepower} and \cs{provideprefix} (\latin{vide infra}).
%
% Output that is only valid in maths mode requires \cs{ensuremath},
% text-only input requires \cs{text}.  In the example below,
% \cs{mathnormal} is used to force the font choice only for the
% single character.\footnote{The \cs{mathrm} font used for this
% document has an ``\ss'' at the $\pi$ position.}
%\begin{LaTeXexample}
%  \newunit{\SIpi}{\ensuremath{\mathnormal{\pi}}}
%  \SI{10}{\metre\per\SIpi\squared}
%\end{LaTeXexample}
%
%\DescribeMacro{\newpower}
%\DescribeMacro{\renewpower}
%\DescribeMacro{\providepower}
% Powers are defined:
% \cs{newpower}\ofixarg{post}\marg{power}\marg{num}. Here,
% \meta{power} is the name of the power macro and \meta{num} is the
% (positive) number it represents.  The later argument is always
% processed internally by \cs{num}, but \emph{must} be a number.
% Giving the optional argument \opt{post} indicates to the package
% that the power will come after the unit it applies to; by default
% it is assumed that it will come before.
%\begin{LaTeXexample}
%  \newpower{\quartic}{4}
%  \newpower[post]{\totheforth}{4}\\
%  \si{\kilogram\totheforth}\\
%  \si{\quartic\metre}
%\end{LaTeXexample}
%
%\DescribeMacro{\newprefix}
%\DescribeMacro{\renewprefix}
%\DescribeMacro{\provideprefix}
% The standard SI powers of ten are defined by the package, and are
% described above.  However, the user can define new prefixes with
% \cs{newprefix}.  This has syntax
% \cs{newunit}\ofixarg{binary}\marg{prefix}\marg{symbol}\marg{powers-ten},
% where \meta{powers-ten} is the number of powers of ten the prefix
% represents.  When the \opt{binary} option is given, the prefix is a
% power of two.  For example, \cs{kilo} and \cs{kibi} are defined:
%\begin{LaTeXexample}[noexample]
%  \newprefix{\kilo}{k}{3}
%  \newprefix[binary]{\kibi}{Ki}{10}
%\end{LaTeXexample}
%
%\changes{v1.1}{2008/09/22}{Qualifiers introduced}
%\DescribeMacro{\newqualifier}
%\DescribeMacro{\renewqualifier}
%\DescribeMacro{\providequalifier}
% It is possible to create unit ``qualifiers'', which add subscript
% descriptions to units.  Although not encouraged, this is sometimes
% necessary to make the meaning clear.  No qualifiers are pre-defined
% by \currpkg, and so the user needs to declare them using
% \cs{newqualifier}. Currently, only subscript qualifiers can be
% created; a more extended set of options is planned for the next
% release of \currpkg.
%\begin{LaTeXexample}
%  \newqualifier{\polymer}{pol}
%  \newqualifier{\catalyst}{cat}
%  \si{\kg\polymer\per\mole\catalyst\per\hour}
%\end{LaTeXexample}
%
%\section{Specialist units}
% In some subject area, there are units which are in common use even
% though they are outside of the SI system.  Unlike the units
% discussed earlier, these specialist units are not loaded by
% default.  In each case, they should be requested with the option
% \opt{alsoload=\meta{name}}.
%
%\subsection{Binary units (\opt{binary})\label{sec:binary}}
%\DescribeMacro{\bit}
%\DescribeMacro{\byte}
% The binary prefixes, \cs{bit}, \cs{byte} (\ref{tab:binary}) are not
% formally part of the SI system. They are available by giving the
% \opt{alsoload=binary} option.
%\begin{LaTeXexample}
%  \SI{100}{\mebi\byte}
%\end{LaTeXexample}
%\begin{table}
%  \caption{Binary prefixes
%    (\opt{alsoload=binary})}
%  \label{tab:binary}
%  \centering
%  \begin{tabular}{lllc}
%    \toprule
%      \multicolumn{1}{c}{Prefix} &
%      \multicolumn{1}{c}{Macro} &
%      \multicolumn{1}{c}{Power} &
%      Symbol \\
%    \midrule
%      \prefixinfo{kibi} \\
%      \prefixinfo{mebi} \\
%      \prefixinfo{gibi} \\
%      \prefixinfo{tebi} \\
%      \prefixinfo{pebi} \\
%      \prefixinfo{exbi} \\
%    \bottomrule
%  \end{tabular}
%\end{table}
%
%\subsection{Synthetic chemistry (\opt{synchem})}
%\DescribeMacro{\mmHg}
%\DescribeMacro{\molar}
%\DescribeMacro{\Molar}
%\DescribeMacro{\torr}
%\DescribeMacro{\dalton}
% The \opt{synchem} file adds the common chemistry units \cs{mmHg},
% \cs{molar}, \cs{Molar}, \cs{torr} and \cs{dalton} to \currpkg.  The
% \cs{Molar} macro is somewhat awkward, as it can be given as either
% ``\textsc{m}'' or ``M''.  The later is obviously easily confused
% with the sign for the prefix mega.  By default, \currpkg uses the
% UK default of a small-caps symbol.  The \cs{dalton} unit is defined
% here as this name is not recognised by the various international
% bodies: the symbol \si{\amu} is preferred.
%\begin{LaTeXexample}
%  \SI{1}{\Molar} HCl\\
%  \SI{760}{\torr}\\
%  \SI{0.01}{\mmHg}\\
%  \SI{3.0}{\molar}\\
%  \SI{106.42}{\dalton}
%\end{LaTeXexample}
%
%\subsection{High-energy physics (\opt{hep})}
% In contrast to \pkg{hepunits}, \currpkg does not define a long list
% of compound units for high-energy physics.\footnote{Using the
% \opt{emulate=hepunits} option will load a file defining those.}
% Instead, a small selection of new units are defined
% (\ref{tab:hep}).  The mechanisms provided by \currpkg should avoid
% the need for large numbers of abbreviations.  For example, the
% \pkg{hepunits} \cs{MinveV} can be given as |\per\MeV| in \currpkg,
% which requires only one more character.
%\begin{table}
%  \caption{High-energy physics units (\opt{alsoload=hep})}
%  \label{tab:hep}
%  \centering
%  \begin{tabular}{lllc}
%    \toprule
%    Unit & Macro & Symbol & Abbreviation \\
%    \midrule
%    \addlinespace \emph{Areas} \\
%    \unitinfo{yoctobarn} & \csindex{yb} \\
%    \unitinfo{zeptobarn} & \csindex{zb} \\
%    \unitinfo{attobarn} & \csindex{ab} \\
%    \unitinfo{femtobarn} & \csindex{fb} \\
%    \unitinfo{picobarn} & \csindex{pb} \\
%    \unitinfo{nanobarn} & \csindex{nb} \\
%    \addlinespace \emph{Other units} \\
%    \unitinfo{micron} \\
%    millirad & \csindex{mrad} & \mrad \\
%    \unitinfo{gauss} \\
%    \bottomrule
%  \end{tabular}
%\end{table}
%
%\DescribeMacro{\clight}
%\DescribeMacro{\eVperc}
% The \opt{hep} option defines two units which are slightly unusual.
% \cs{clight} gives \clight, which is recognised as a unit when used
% in the appropriate circumstances. The second unit provided is
% \cs{eVperc}, which is commonly-used and clear enough for a compound
% definition. Notice that the value of \opt{eVcorrb} will need to be
% adjusted when using this unit.
%\begin{LaTeXexample}
%  \SI[per=slash,eVcorrb=0.4ex]{4.657}{\mega\eVperc\squared}
%\end{LaTeXexample}
%
%\subsection{Astronomy (\opt{astro})}
%\DescribeMacro{\parsec}
%\DescribeMacro{\lightyear}
% For astronomers, the \cs{parsec} and \cs{lightyear} units are
% available, and give the obvious results.
%\begin{LaTeXexample}
%  \SI{12}{\parsec}\\
%  \SI{1}{\lightyear}
%\end{LaTeXexample}
%
%
%\subsection{Geophysics (\opt{geophys})}
%\changes{v1.1j}{2008/12/22}{New module \opt{geophys} containing \cs{gon}
%  unit}
%\DescribeMacro{\gon}
% For geophysics, the unit \cs{gon} is available.
%\begin{LaTeXexample}
%  \SI{12}{\gon}
%\end{LaTeXexample}
%
%\section{Font control}
% Following the lead of \SIstyle, \currpkg provides control over the
% font used to typeset output.  By default, all text is typeset using
% the current upright serif maths font, whether the macros are given
% in text or maths mode.  Some examples will show the effect.
%\begin{LaTeXexample}
%  \num{10} $\num{10}$\\
%  \sffamily \ang{20} $\ang{20}$\\
%  \textbf{\SI{30}{\kilo\gram}}\\
%  \boldmath $\SI{40}{\kilo\gram}$
%  \[ \num{50} \]
%\end{LaTeXexample}
%
% In contrast, by setting \opt{obeyall}, the current font is used:
% this may be maths or text, depending on the context.
%\begin{LaTeXexample}
%  \sisetup{obeyall}\\
%  \ang{1;1;1} $\ang{1;1;1}$\\
%  \sffamily \ang{2;2;2} $\ang{2;2;2}$\\
%  \textbf{\ang{3;3;3}} \boldmath $\ang{3;3;3}$\\
%  \emph{\ang{4;4;4}} \emph{$\ang{4;4;4}$}
%  \[ \ang{5;5;5} \]
%\end{LaTeXexample}
%
% Fine control of which elements of the local font are used is
% available with the \opt{obeyfamily}, \opt{obeybold},
% \opt{obeyitalic} and \opt{obeymode} options.
%\begin{LaTeXexample}
%  \sisetup{obeyfamily}\\
%  \ang{1;1;1} $\ang{1;1;1}$\\
%  \sffamily \ang{2;2;2} $\ang{2;2;2}$\\
%  \sisetup{obeybold}
%  \textbf{\ang{3;3;3}} \boldmath $\ang{3;3;3}$\\
%  \emph{\ang{4;4;4}} \emph{$\ang{4;4;4}$}
%  \[ \ang{5;5;5} \]
%\end{LaTeXexample}
%
%\section{Package options\label{sec:options}}
%\DescribeMacro{\sisetup}
% The ``native'' options for the package are all given using the
% key--value method. Most of the package options can be given both
% when loading the package and at any point in the document.  This is
% achieved using the \cs{sisetup} macro.
%
% The package options take a number of different forms.
%\begin{itemize}
%  \item \opt{option=\meta{bool}} Simple true/false values.  These
%    macros all default to \opt{option=true}, meaning that giving the
%    option name along will set the appropriate flag.
%  \item \opt{option=\meta{choice}} Take a single item from a
%    pre-determined list. Depending on the value, one or more
%    internal states will be altered.  Values not on the list are
%    ignored (with a warning).
%  \item \opt{option=\meta{choice,literal}} If the given value is a
%    \meta{choice}, then the internal settings for that choice are
%    used.  Any other value is used directly.
%  \item \opt{option=\meta{literal}} The given value is used as a
%    literal by the package.
%  \item \opt{option=\meta{csname}} These options expect a command
%    sequence as a value.
%  \item \opt{option=\meta{length}} Requires a \TeX\ length, for
%    example \opt{0.5ex}.
%  \item \opt{option=\meta{list}} Takes a list of one or more items,
%    which are not determined in advance.
%  \item \opt{option=\meta{number}} Takes a number (possibly
%    including an exponent part).
%\end{itemize}
%
% The package has a large range of options, to allow full control of
% the various features of the package.  These control differing
% aspects of the package, and are given below in groups based on
% function.  Where the key has a default value, it is given in bold.
%
%\subsection{Font family and style}
% The font used when typesetting material can be tightly controlled
% using \currpkg.  A number of options affect how the package matches
% the surrounding font, and the font families used to achieve this.
% The default is to use the current upright maths serif font with no
% variation.
%
%\DescribeOption{mode}
%\DescribeOption{textmode}
%\DescribeOption{valuemode}
%\DescribeOption{unitmode}
%\DescribeOption{obeymode}
% The output of \currpkg can occur using either text or maths mode.
% The package option \opt{mode} determines which is used: valid
% options are \defaultopt{maths} and \opt{text}.\footnote{Here and in
% all other cases, either UK or US spelling may be used.  Thus
% \opt{mode=maths} or \opt{mode=math} have exactly the same effect.}
% The shortcut \opt{textmode} is provided for setting \opt{mode=text}
% quickly. Further refinement is possible using the \opt{valuemode}
% and \opt{unitmode} options.  These apply to numbers (the output of
% \cs{num} and the first mandatory argument of \cs{SI}) and units
% (all other output), respectively. By setting the \opt{obeymode}
% flag, the package will use the local typesetting mode (maths or
% text).
%
%\DescribeOption{obeyall}
%\DescribeOption{obeyfamily}
%\DescribeOption{obeybold}
%\DescribeOption{obeyitalic}
% The detection and matching of surrounding text can be controlled
% using a number of Boolean package options. \opt{obeyall} turns on
% all of the detection.  Thus output with \opt{obeyall} in force will
% always match the local text appearance.  \opt{obeyfamily} instructs
% the package on detecting the surrounding font family (Roman, sans
% serif, fixed width), but does not detect bold or italic.
% \opt{obeybold} detects the local bold setting, whilst
% \opt{obeyitalic} picks up italic fonts.
%
% Bold detection is influenced by the value of \opt{inlinebold},
% which takes values \defaultopt{text} and \opt{maths}.  The package
% can detect the local value of bold for either the surrounding text,
% or the surrounding inline (|$|\ldots|$|) maths.  The
% \opt{obeyitalic} option does \emph{not} have the same facility
% (maths is italic anyway).
%
%\DescribeOption{mathsrm}
%\DescribeOption{mathssf}
%\DescribeOption{mathstt}
%\SpecialOptionIndex{mathrm}
%\SpecialOptionIndex{mathsf}
%\SpecialOptionIndex{mathtt}
%\DescribeOption{textrm}
%\DescribeOption{textsf}
%\DescribeOption{texttt}
% The font commands used by the package to achieve the above are all
% available for user modification.  The options \opt{mathsrm},
% \opt{mathssf} and \opt{mathstt} hold the command sequences used in
% maths mode,\footnote{These can also be set using \opt{mathrm},
% \opt{mathsf} and \opt{mathtt}} while \opt{textrm}, \opt{textsf} and
% \opt{texttt} do the same for text mode.  By default, these contain
% the obvious command names, for example \opt{mathsrm=mathrm} and
% \opt{texttt=ttfamily}.  However, they can be set at will: the macro
% names indicate the nature of the surrounding text detected.  For
% example, the value of \opt{mathssf} is used in maths mode when the
% surrounding text is sans serif.
%
%\SpecialOptionIndex{valuemathsrm}
%\SpecialOptionIndex{valuemathssf}
%\SpecialOptionIndex{valuemathstt}
%\SpecialOptionIndex{valuemathrm}
%\SpecialOptionIndex{valuemathsf}
%\SpecialOptionIndex{valuemathtt}
%\SpecialOptionIndex{unitmathsrm}
%\SpecialOptionIndex{unitmathssf}
%\SpecialOptionIndex{unitmathstt}
%\SpecialOptionIndex{valuetextrm}
%\SpecialOptionIndex{valuetextsf}
%\SpecialOptionIndex{valuetexttt}
%\SpecialOptionIndex{valuetextrm}
%\SpecialOptionIndex{valuetextsf}
%\SpecialOptionIndex{valuetexttt}
%\SpecialOptionIndex{unittextrm}
%\SpecialOptionIndex{unittextsf}
%\SpecialOptionIndex{unittexttt}
% Each of the font options can be given separately for the contents
% of numbers and units.  The option names include \opt{value} or
% \opt{unit} before the mode name.  For example, the \opt{mathsrm}
% option may be split into \opt{valuemathsrm} and \opt{unitmathsrm}.
%
%\DescribeOption{detectdisplay}
% The font detection system can treat displayed mathematical content
% in two ways.  This is controlled by the \opt{detectdisplay} option.
% When set to \defaultopt{true}, display mathematics is treated
% independently from the body of the document.  Thus the local
% \emph{maths} font is checked for matching.  In contrast, when set
% to \opt{false}, display material is treated with the current
% running text font.
%\begin{LaTeXexample}
%  \sffamily
%  Some text
%  \sisetup{obeyall}
%  \[ x = \SI{1.2e3}{\kg\kelvin\candela} \]
%  More text
%  \sisetup{detectdisplay=false}
%  \[ y = \SI{3}{\metre\second\mole} \]
%\end{LaTeXexample}
%
%\subsection{Spacing and separators}
%\DescribeOption{unitsep}
%\DescribeOption{valuesep}
%\DescribeOption{digitsep}
%\DescribeOption{anglesep}
% The separators between items can all be set using options taking a
% list of pre-defined items or a literal value.  The ``\opt{sep}''
% options (\opt{unitsep}, \opt{valuesep}, \opt{digitsep} and
% \opt{anglesep}) all recognise \opt{thin}, \opt{medium}, \opt{med},
% \opt{thick}, \opt{space}, \opt{cdot}, \opt{times}, \opt{tightcdot},
% \opt{tighttimes}, \opt{fullstop}, \opt{stop}, \opt{period} and
% \opt{none}.  The named spaces are the normal maths separations,
% with \opt{space} representing a full (non-breakable text) space,
% and with the obvious meanings for \opt{cdot} and \opt{times}.  The
% \opt{tight} variants reduce the spacing available. Three possible
% values are provided for ``|.|'', and \opt{none} yields no space at
% all. In all cases, other values are treated literally and are
% typeset in maths mode.  The default value is \defaultopt{thin} for
% all separations except \opt{anglesep}, which is set to
% \defaultopt{none}.
%
%\DescribeOption{unitspace}
%\DescribeOption{errspace}
% The \opt{unitspace} and \opt{errspace} options again take a list
% or literal value, but only the ``real'' spaces \opt{thin},
% \opt{medium}, \opt{med}, \opt{thick}, \opt{space} and \opt{none}
% are recognised in the list. The \opt{unitspace} option controls the
% output generated by an explicit space (|~|) inside a unit macro,
% while \opt{errspace} is used to separate a bracketed error from the
% main number.
%
%\subsection{Number formatting}
%\DescribeOption{numdigits}
%\DescribeOption{numdecimal}
%\DescribeOption{numdiv}
%\DescribeOption{numexp}
%\DescribeOption{numsign}
%\DescribeOption{numaddn}
%\DescribeOption{numgobble}
% There are two groups of options for formatting numbers.  The first
% group all begin with ``\opt{num}'', and take literal values used by
% the package to parse numbers.  \opt{numdigits} contains the valid
% number symbols (\defaultopt{0123456789}), with \opt{numdecimal}
% containing the decimal markers (\defaultopt{.,}).  As in the
% \numprint package, \opt{numexp} (the list of exponent markers)
% recognises \defaultopt{deDE} as valid by default.  \opt{numsign}
% contains the sign markers for numbers
% (\defaultopt{+-\cs{pm}\cs{mp}}).  \opt{numdiv} holds the division
% marker in numbers (\defaultopt{/}). \opt{numaddn} and \opt{numgobble}
% both control which other characters do not give an error when present
% in a number.  \opt{numaddn} contains valid characters which should be
% included in the final output ``as is'', whereas \opt{numgobble} lists
% the characters that are completely ignored.  In all cases, the
% content of the options is a simple string, for example
% \opt{numdigits=1234567890}.
%
%\DescribeOption{decimalsymbol}
%\DescribeOption{digitsep}
%\DescribeOption{sepfour}
% The second group of number options control the output of numbers
% after parsing. The symbol used by \currpkg as a decimal marker is
% set by the \opt{decimalsymbol} option, which can take a list of
% choices or a literal.  The valid choices here are
% \defaultopt{fullstop}, \opt{comma}, \opt{cdot} and
% \opt{tightcdot}.\footnote{\opt{fullstop} also has aliases
% \opt{stop} and \opt{period}.}  Notice that this does not have to
% agree with the input marker.  The other separator for numerical
% output is the division of digits into groups of three.  The result
% is dependent on two options.  The previously-described
% \opt{digitsep} option controls the spacing added between groups of
% three numbers.  For numbers consisting of exactly four digits, the
% \opt{sepfour} Boolean option controls whether separation occurs in
% these cases.  The default is \defaultopt{false}.
%\begin{LaTeXexample}
%  \num{1234}\\
%  \num[sepfour]{1234}
%\end{LaTeXexample}
%
%\DescribeOption{seperr}
%\DescribeOption{numopenerr}
%\DescribeOption{numcloseerr}
% For numbers given with an error [\eg \num{1.23(4)}], the package
% can separate out the error part, to give for example
% \num[seperr]{1.23(4)}.  This behaviour is activated by the
% \opt{seperr} option, and requires that \opt{numopenerr} and
% \opt{numcloseerr} contain the left- and right-hand delimiters for
% the error [defaults \opt{numopenerr=(} and \opt{numcloseerr=)}].
%\begin{LaTeXexample}
%  \sisetup{seperr}\\
%  \num{1.234(5)}\\
%  \num{1.234(5)e6}
%\end{LaTeXexample}
%\DescribeOption{trapambigerr}
%\DescribeOption{openerr}
%\DescribeOption{closeerr}
%\DescribeOption{tightpm}
% If the number has an exponent, or if units are not repeated, then
% the result can be considered ambiguous.  By default, the package
% adds the markers stored in \opt{openerr} and \opt{closeerr} to
% remove the ambiguity; the options have the same default values as
% the input error markers.  Detection of a potentially-ambiguous
% error is controlled by the \opt{trapambigerr} option, although for
% numbers with units the \opt{repeatunits} option is also important.
% The spacing around the $\pm$ sign is normally set by \TeX. However,
% using the \opt{tightpm} option will cause this to be reduced to a
% minimum.
%\begin{LaTeXexample}
%  \sisetup{seperr}\\
%  \num[trapambigerr=false]{1.234(5)e6}\\
%  \SI{1.234(5)}{\metre}\\
%  \sisetup{repeatunits=false}
%  \SI{1.234(5)}{\metre}\\
%  \SI[trapambigerr=false]{1.234(5)}{\metre}\\
%  \num[tightpm]{1.234(5)}
%\end{LaTeXexample}
%
%\DescribeOption{numprod}
%\DescribeOption{repeatunits}
% The number processor can recognise products in the numerical input;
% the symbols used for products are stored in \opt{numprod}, with the
% default value of ``\opt{x}''.  By default, the \cs{SI} macro will
% repeat the units for a number given in this way.  This behaviour is
% altered  by the \opt{repeatunits} option, which takes the values
% \defaultopt{true}, \opt{false} and \opt{power}.  The later applies
% only when providing multiplied numbers with units, and converts the
% unit to the appropriate power rather than repeating the
% units.\footnote{This is a very simply option: do not expect it to
% work with anything except areas and volumes.} Notice that when
% applied to errors, \opt{repeatunits} takes priority over
% \opt{trapambigerr}.
%\begin{LaTeXexample}
%  \num{1 x 2 x 3}\\
%  \SI{4 x 5 x 6}{\metre}\\
%  \sisetup{repeatunits=false}
%  \SI{1.2 x 3.4 x 5.6}{\milli\metre\cubed}\\
%  \SI[seperr]{1.234(5)}{\metre}\\
%  \SI[seperr,
%    trapambigerr=false]
%    {9.1093897(54)e-31}{\kilo\gram}\\
%  \SI[seperr,repeatunits,
%    trapambigerr=false]
%    {9.1093897(54)e-31}{\kilo\gram}\\
%  \SI[repeatunits=power]{1 x 2 x 3}{\metre}
%\end{LaTeXexample}
%
%\DescribeOption{expproduct}
%\DescribeOption{expbase}
%\DescribeOption{allowzeroexp}
% The formatting of exponents is controlled by \opt{expproduct} and
% \opt{expbase}.  \opt{expproduct} sets the symbol used to indicate a
% product for exponents (\eg the $\times$ in \num{2e2}), while the
% value of \opt{expbase} sets the power used (the \num{10} in the
% example).  Both options accept a very short list of options:
% \defaultopt{times}, \opt{tighttimes}, \opt{cdot} and
% \opt{tightcdot} for the product, and \defaultopt{ten} and \opt{two}
% for the power.\footnote{The \opt{tighttimes} and \opt{tightcdot}
% options give the rather questionable results:
% \num[expproduct=tighttimes]{1e2} and
% \num[expproduct=tightcdot]{1e2}, as opposed to \num{1e2} and
% \num[expproduct=cdot]{1e2}.} Other choices are used literally. Also
% relevant to exponent processing is the \opt{allowzeroexp} option.
% By default, the package will suppress a zero exponent, but setting
% the flag will allow the output of \num[allowzeroexp]{e0}.
%
%\DescribeOption{addsign}
%\DescribeOption{sign}
%\DescribeOption{retainplus}
% Additions to the input can take the form of implicit signs and
% padded zeros.  The \opt{addsign} option takes a list of potential
% sites to add a sign: \defaultopt{none}, \opt{mantissa},
% \opt{exponent} and \opt{both}.\footnote{Aliases are provided:
% \opt{mant} = \opt{mantissa}, \opt{exp} = \opt{exponent}, \opt{all}
% = \opt{true} = \opt{both}, \opt{false} = \opt{none}.}  If no sign
% is given in the input, the setting here determines if one is added.
% The sign to add is stored in \opt{sign}, which takes the list of
% choices \defaultopt{plus}, \opt{minus}, \opt{pm} and \opt{mp}, or
% uses the input literally (in maths mode).  For positive numbers,
% the \opt{retainplus} option causes a $+$ sign explicitly in the
% input to be retained.  By default, the package will remove such
% signs.
%
%\DescribeOption{padnumber}
% The \opt{padnumber} option controls the addition of zeros to the
% input, to ``pad'' the result.  The option takes a list of choices:
% \defaultopt{leading}, \opt{trailing}, \opt{both} and
% \opt{none}.\footnote{Aliases: \opt{all} = \opt{true} = \opt{both},
% \opt{false} = \opt{none}.}  No additional precision is added by
% this option; integer input will not add a decimal point.
%\begin{LaTeXexample}
%  \num[padnumber=leading]{.1}\\
%  \num[padnumber=leading]{2.}\\
%  \num[padnumber=trailing]{3.}\\
%  \num[padnumber=both]{4.}\\
%  \num[padnumber=both]{.5}\\
%  \num[padnumber=both]{6}\\
%  \num[padnumber=none]{7.}\\
%  \num[padnumber=none]{.8}
%\end{LaTeXexample}
%
%\DescribeOption{fixdp}
%\DescribeOption{dp}
% In contrast to the \opt{padnumber} option, the package can alter
% the precision of the input number if the \opt{fixdp} option is set.
% The \opt{fixdp} option will fix the decimal places of the output to
% the number stored in the \opt{dp} option.  The later should be a 
% positive integer or zero.
%\begin{LaTeXexample}
%  \sisetup{fixdp,dp=3}
%  \num{1}\\
%  \num{53.9}\\
%  \num{4.56783}\\
%  \num{-1.2942}\\
%  \num{-1.2959}\\
%  \num{9.9999}\\
%  \num{2.1264e90}\\
%  \num[dp=0]{12.345}
%\end{LaTeXexample}
%
%\DescribeOption{fixsf}
%\changes{v1.1}{2008/08/22}{New \opt{fixsf} option}
%\DescribeOption{sf}
%\changes{v1.1}{2008/08/22}{New \opt{sf} option}
% In contrast to the \opt{dp} option, the \opt{fixsf} and \opt{sf}
% options control the number of significant figures in the output.
% This never adds extra digits to the output, as significant figures
% are concerned with accuracy.
%\begin{LaTeXexample}
%  \sisetup{fixsf,sf=3}
%  \num{1}\\
%  \num{53.9}\\
%  \num{4.56783}\\
%  \num{-1.2942}\\
%  \num{-1.2959}\\
%  \num{9.9999}\\
%  \num{2.1264e90}\\
%  \num{773322}
%\end{LaTeXexample}
%
%\subsection{Angle formatting}
%\DescribeOption{padangle}
%\DescribeOption{strictarc}
% The angle formatter uses \cs{num} to format numbers: any options
% for numbers are therefore applicable here.  The \opt{padangle}
% option takes choices \defaultopt{small}, \opt{large}, \opt{all} and
% \opt{none}, and controls how angles are padded when given in
% degrees, minutes and seconds.\footnote{Aliases: \opt{all} =
% \opt{true} = \opt{both}, \opt{false} = \opt{none}.}  When giving
% angles as arcs (in degrees, minutes and seconds), the package can
% detect if the correct number of semi-colons have been given. This
% is controlled by the \opt{strictarc} option, which is a Boolean
% switch with a \defaultopt{true} default.  With \opt{strictarc} set
% to \opt{false}, an incomplete arc is interpreted as degrees and
% minute, while an over-complete one will drop excess input.
%\begin{LaTeXexample}
%  \ang[padangle=none]{1;;}\\
%  \ang[padangle=large]{2;;}\\
%  \ang[padangle=small]{3;;}\\
%  \ang[padangle=both]{;4;}\\
%  \ang[padangle=none]{;5;5}\\
%  \ang[strictarc=false]{1;2}\\
%  \ang[strictarc=false]{1;2;3;4;}
%\end{LaTeXexample}
%
%\DescribeOption{angformat}
% The angle formatting system can convert between decimal angles and
% those given as degrees, minutes and seconds.  This is controlled by
% the \opt{angformat} option, which takes choices
% \defaultopt{unchanged}, \opt{decimal} and
% \opt{arc}.\footnote{Aliases: \opt{decimal} = \opt{dec}, \opt{arc} =
% \opt{dms}, \opt{unchanged} = \opt{none}.}  When set to
% \opt{unchanged}, nothing is done to the input.  The conversion is
% based on \TeX\ dimensions, and is therefore limited in accuracy.
% For this reason, the output is automatically rounded: output as a
% decimal angle is limited to three places, and that as an arc is
% given to a single decimal place for the seconds component.
%\begin{LaTeXexample}
%  $\ang{1;2;3} = \ang[angformat=dec]{1;2;3}$\\
%  $\ang{4.56} = \ang[angformat=arc]{4.56}$
%\end{LaTeXexample}
%
%\DescribeOption{astroang}
% For astronomers, the \opt{astroang} option is provided.  This
% moves the degrees, minutes or seconds symbol (as appropriate)
% over the decimal marker rather than after the number.
%\begin{LaTeXexample}
%  \ang{1;2;3.4}\\
%  \ang[astroang]{5;6;7.8}
%\end{LaTeXexample}
%
%\subsection{Tabular material}
%\DescribeOption{tabnumalign}
% Material typeset in |S| columns is processed internally by the
% \cs{num} macro.  Thus, as with angles, the number options also
% apply here.  The positioning of tabular material is controlled by
% the two options \opt{tabnumalign} and \opt{tabformat}.
% \opt{tabnumalign} takes values \defaultopt{centredecimal},
% \opt{centre}, \opt{left} and \opt{right}.\footnote{Aliases
% \opt{centerdecimal} = \opt{centredecimal}; \opt{center} =
% \opt{centre}.}  When using \opt{centredecimal}, the package places
% the decimal marker of the mantissa at the centre of the column,
% which then grows to accommodate the widest number given.  For equal
% numbers of digits before and after the decimal sign, this is the
% easiest option.  The other choices use a fixed-width box to store
% the number; the box is then aligned with the edges of the column.
%
%\DescribeOption{tabformat}
% The \opt{tabformat} option sets the amount of space reserved by
% \currpkg for the alignment box when not using the
% \opt{centredecimal} setting of \opt{tabnumalign}.  The numerical
% parts of \opt{tabformat} are interpreted as
% \texttt{\meta{pre}\meta{dec}\meta{post}}; \meta{pre} and
% \meta{post} are the number of digits before and after the decimal
% sign, respectively.  Both signs and exponents can be included in
% \opt{tabformat}, resulting in appropriate space being reserved. The
% entire \opt{tabformat} input is processed using the \cs{num} macro
% internally.  Thus the decimal and exponent signs used in
% \opt{tabformat} are checked against \opt{numdecimal} and
% \opt{numexp}, respectively.
%
%\DescribeOption{tabalignexp}
%\DescribeOption{tabexpalign}
%\changes{v1.1d}{2008/10/29}{Added \opt{tabexpalign} as an alias for
%  \opt{tabalignexp}}
% When \opt{tabformat} contains exponents, two possibilities are
% available for alignment.  The first method is to place the exponent
% parts so that the ``$\times 10$'' parts form a column, with
% whitespace after shorter mantissa components.  In the second
% method, no additional space is added after the mantissa, and the
% exponents do no line up (\ref{tab:alignexp}).  This is controlled
% by the \opt{tabalignexp} option, which can be set to
% \defaultopt{true} or \opt{false}. The alias \opt{tabexpalign} is
% available for this option.
%\begin{LaTeXexample}[float]
%  \begin{table}
%    \centering
%    \caption{The \opt{tabalignexp} option}
%    \label{tab:alignexp}
%    \sisetup{tabformat=1.3e2,tabnumalign=centre}
%    \begin{tabular}{SS[tabalignexp=false]}
%      \toprule
%      {Header} & {Header} \\
%      \midrule
%      1.2e3    & 1.2e3    \\
%      1.234e56 & 1.234e56 \\
%      \bottomrule
%    \end{tabular}
%  \end{table}
%\end{LaTeXexample}
%
%\DescribeOption{tabtextalign}
%\DescribeOption{tabunitalign}
%\DescribeOption{tabalign}
% Cells containing no numbers are handled by \currpkg in a manner
% similar to \cs{multicolumn}.  The setting of \opt{tabtextalign} is
% taken from the list \defaultopt{centre}, \opt{right} and
% \opt{left}.\footnote{Alias \opt{centre} = \opt{center}.}  As would
% be expected, these settings centre, right- or left-align the cell
% contents. In |s| columns, all content is treated as input to the
% \cs{si} macro.  The alignment of the contents relative to the cell
% is controlled by the \opt{tabunitalign} option, which takes options
% \defaultopt{left}, \opt{right} and \opt{centre}.  The settings for
% \opt{tabnumalign}, \opt{tabtextalign} and \opt{tabunitalign} can be
% set to the same value in one go with the \opt{tabalign} option.
%
%\DescribeOption{tabautofit}
% The contents of table cells can automatically be rounded or
% zero-filled to the number of decimal places given in
% \opt{tabformat}.  This is activated by the \opt{tabautofit} Boolean
% option. As \opt{tabformat} does not apply to columns with alignment
% \opt{centredecimal}, \opt{tabautofit} is also inactive for these
% columns (\ref{tab:autofit}).
%\begin{LaTeXexample}[float]
%  \begin{table}
%    \centering
%    \caption{The \opt{tabautofit} option}
%    \label{tab:autofit}
%    \sisetup{tabformat=1.3,tabnumalign=centre}
%    % Notice the overfull hbox which results with
%    % the first column
%    \begin{tabular}{%
%      S%
%      S[tabautofit]%
%      S[tabautofit,tabnumalign=centredecimal]}
%      \toprule
%      {Header} & {Header} & {Header} \\
%      \midrule
%      1.2    & 1.2    & 1.2    \\
%      1.2345 & 1.2345 & 1.2345 \\
%      \bottomrule
%    \end{tabular}
%  \end{table}
%\end{LaTeXexample}
%
%\DescribeOption{tabparseonly}
% It is possible to turn off the automatic alignment of numbers
% while retaining the ability to parse the |S| column input.
% Setting \opt{tabparseonly} to \opt{true} will still parse the
% column contents, but will not align the decimal markers.  The
% alignment of the cell is governed by \opt{tabnumalign}
% (\ref{tab:parse}).
%\begin{LaTeXexample}[float]
%  \begin{table}
%    \centering
%    \caption{The \opt{tabparseonly} option}
%    \label{tab:parse}
%    \begin{tabular}{%
%      S[tabparseonly]%
%      S[tabparseonly,tabnumalign=left]%
%      S[tabparseonly,tabnumalign=right]}
%      \toprule
%      {Header} & {Header} & {Header} \\
%      \midrule
%      14.2    & 14.2    & 14.2    \\
%      1.23456 & 1.23456 & 1.23456 \\
%      1.2e3   & 1.2e3   & 1.2e3   \\
%      \bottomrule
%    \end{tabular}
%  \end{table}
%\end{LaTeXexample}
%
%\subsection{Units}
%\DescribeOption{per}
% Most of the unit options are concerned with the processing of named
% units.  The processor for units given as macro names can be
% influenced to give a variety of output formats.  The \opt{per}
% option defines how the keyword macro \cs{per} is handled.  This
% option takes a choice from the list \defaultopt{reciprocal},
% \opt{slash} and \opt{fraction}.\footnote{Aliases \opt{reciprocal} =
% \opt{rp} = \opt{power}, \opt{fraction} = \opt{frac}.}  The default
% option uses \cs{per} to indicate reciprocal powers, whereas
% \opt{slash} causes the package to use ``$/$'' to show division.
%
%\DescribeOption{fraction}
%\DescribeOption{slash}
% The \opt{fraction} option defines how \opt{per=fraction} is
% interpreted.  The list of applicable values here is
% \defaultopt{frac}, \opt{nice}, \opt{ugly} and \opt{sfrac}.  In each
% case, the unit is typeset as a fraction, but the macro use to
% achieve this varies. \opt{frac} uses the \TeX\ \cs{frac}, macro,
% while \opt{nice} makes use of a \cs{nicefrac}-like method.  The
% \opt{ugly} option uses a slash in text mode and \cs{frac} in maths
% mode.\footnote{Similar to the \pkg{ugly} option of the
% \pkg{nicefrac} package.}  Finally, the setting \opt{fraction=sfrac}
% uses the \cs{sfrac} macro from the \pkg{xfrac} package, when
% available.\footnote{\pkg{xfrac} is part of the experimental system
% for \LaTeX3.  As it requires a number of additional packages to
% work, \currpkg does not load \pkg{xfrac}. If it is unavailable, the
% \opt{sfrac} setting will fall back to using \cs{nicefrac}.  See the
% \pkg{xfrac} documentation for reasons to prefer \cs{sfrac} to
% \cs{nicefrac}.}  The \opt{slash} option sets the symbol used when
% \opt{per=slash} is in force.  This recognises the single keyword
% \opt{slash}; anything else is used literally.
%\begin{LaTeXexample}
%  \sisetup{per=fraction}
%  \si[fraction=frac]{\metre\per\second}\\
%  \si[fraction=nice]{\metre\per\second}\\
%  \si[fraction=sfrac]{\metre\per\second}\\
%  \si[per=slash]{\metre\per\second}
%\end{LaTeXexample}
%
%\DescribeOption{stickyper}
% By default, \cs{per} applies only to the next unit
% given.\footnote{This is the standard method of reading units in
% English: for example, \si{\joule\per\mole\per\kelvin} is pronounced
% ``joules per mole per kelvin''.}  By setting the \opt{stickyper}
% flag, this behaviour is changed so that \cs{per} applies to all
% subsequent units.\footnote{This is the behaviour in \SIunits.}
%\begin{LaTeXexample}
%  \si{\kilogram\per\metre\ampere}\\
%  \si[stickyper]{\kilogram\per\metre\ampere}
%\end{LaTeXexample}
%
%\DescribeOption{trapambigfrac}
%\DescribeOption{openfrac}
%\DescribeOption{closefrac}
% When using \opt{per=slash}, multiple units in the denominator will
% yield a potentially ambiguous result.  The \opt{trapambigfrac}
% determines whether the package checks for this: this takes
% \defaultopt{true} and \opt{false}.  When set, the contents of
% \opt{openfrac} are inserted before the denominator, and
% \opt{closefrac} is inserted after.
%\begin{LaTeXexample}
%  \sisetup{trapambigfrac,openfrac=(,closefrac=),per=slash}\\
%  \si{\metre\per\second\per\kilogram}\\
%  \si[trapambigfrac=false]{\metre\per\second\per\kilogram}
%\end{LaTeXexample}
%
%\DescribeOption{prefixsymbolic}
%\DescribeOption{prefixbase}
%\DescribeOption{prefixproduct}
% The unit prefixes (\cs{kilo}, \etc) are normally given as letters.
% However, the package can convert these into numerical
% powers.\footnote{Provided things are not too complex!}  This is
% controlled by the \opt{prefixsymbolic} Boolean option, which by
% default is \defaultopt{true}.  If \opt{prefixsymbolic} is set to
% \opt{false}, the format of the prefix is controlled by
% \opt{prefixbase} and \opt{prefixproduct}, which work in the same
% way as \opt{expbase} and \opt{expproduct}.
%
%\DescribeOption{prespace}
%\DescribeOption{xspace}
%\DescribeOption{allowoptarg}
% By default, the single unit macros (\eg \cs{metre}) add no space
% either before or after the unit.  Setting the \opt{xspace} flag to
% true means that the single macros are followed by the \cs{xspace}
% command (when used outside of \cs{SI}/\cs{si}).  For users of
% \unitsdef, the \opt{prespace} macro changes the behaviour of the
% unit macros, so that they can immediately follow a number.  As a
% result, the unit macros will \emph{always} be preceded by a fixed
% space when the \opt{prespace} flag is true: this will be in
% addition to any other space. Also relevant to users moving from
% \unitsdef is the \opt{allowoptarg} option.  This allows single unit
% macros to take an optional numerical argument, in the same way that
% occurs in that package.
%\begin{LaTeXexample}
%  \metre is the symbol for metres\\
%  \sisetup{xspace}
%  No, \metre is the correct symbol\\
%  \sisetup{prespace}
%  30\metre\\
%  Do not use \metre in running text\\
%  \sisetup{allowoptarg}
%  \metre[40]
%\end{LaTeXexample}
%
%\subsection{Symbols}
% User access to control the symbols used for \si{\ohm}, \si{\micro},
% \si{\degree}, \si{\arcmin}, \si{\arcsec}, \si{\angstrom} and
% \si{\celsius} is provided here. These are all literal options,
% which are available in text and maths mode variants.  For example,
% \opt{textmicro} is the code used for the \si{\micro} symbol in text
% mode.  The text mode macros should be safe when forced into text,
% and the maths ones when forced into maths.  The symbols defined in
% this way are:
%\newcommand*{\listsymbopt}[1]{\item \opt{#1}\SpecialOptionIndex{#1}}
%\begin{itemize}
%  \listsymbopt{textOmega};
%  \listsymbopt{mathsOmega};
%  \listsymbopt{textmu};
%  \listsymbopt{mathsmu};
%  \listsymbopt{textdegree};
%  \listsymbopt{mathsdegree};
%  \listsymbopt{textminute};
%  \listsymbopt{mathsminute};
%  \listsymbopt{textsecond};
%  \listsymbopt{mathssecond};
%  \listsymbopt{textringA};
%  \listsymbopt{mathsringA};
%  \listsymbopt{textcelsius};
%  \listsymbopt{mathscelsius}.
%\end{itemize}
%
%\DescribeOption{redefsymbols}
% When \currpkg is loaded, it can check for the presence of the
% \pkg{textcomp} and \pkg{upgreek} packages, to provide better
% symbols for certain items. To prevent this, set the
% \opt{redefsymbols} option \opt{false} (the default is
% \defaultopt{true}).
%
%\DescribeOption{eVcorra}
%\DescribeOption{eVcorrb}
% The \si{\electronvolt} symbol requires some fine-tuning, and so has
% two options of its own, both \TeX\ lengths.  \opt{eVcorra} is the
% correction applied to the gap between ``e'' and ``V'' of the unit:
% the default is \defaultopt{0.3ex}. \opt{eVcorrb} is the correction
% applied to the gap between ``V'' of the unit and whatever follows;
% the default is \defaultopt{0ex}.  The optimal value for these
% options will depend on the current font settings.\footnote{This
% document uses \opt{eVcorra=0.1ex}.}
%\begin{LaTeXexample}
%  \si[per=slash]{\electronvolt\per\metre}\\
%  \si[per=slash,eVcorrb=0.7ex]{\electronvolt\per\metre}
%\end{LaTeXexample}
%
%\subsection{Colour}
%\DescribeOption{colourall}
%\DescribeOption{colourunits}
%\DescribeOption{colourvalues}
%\SpecialOptionIndex{colorall}
%\SpecialOptionIndex{colorunit}
%\SpecialOptionIndex{colorvalue}
% The package provides internal hooks for applying colour to part or
% all of the output.  This requires the user to load the \pkg{color}
% or \pkg{xcolor} package to support colour in the output; \currpkg
% will ignore a colour request if support is unavailable. The Boolean
% options \opt{colourall}, \opt{colourunits} and \opt{colourvalues}
% are used to turn application of a given colour on or off for all
% output, only units and only values, respectively.  All three
% switches are available with US spelling, \eg \opt{colorall} and
% \opt{colourall} behave in the same way.  With colour turned off, no
% \cs{color} command is issued internally, and output follows the
% surrounding text.
%
%\DescribeOption{colour}
%\DescribeOption{unitcolour}
%\DescribeOption{valuecolour}
%\SpecialOptionIndex{color}
%\SpecialOptionIndex{unitcolor}
%\SpecialOptionIndex{valuecolor}
% The colour names to use for colouring output are set by the
% \opt{colour}, \opt{unitcolour} and \opt{valuecolour} options (all
% also available with US spelling).  The \opt{colour} option
% internally sets both \opt{unitcolour} and \opt{valuecolour}.
%\begin{LaTeXexample}
%  {\color{brown} Text \num{50}}\\
%  \sisetup{colourall,colour=green}
%  \SI{10}{\metre}\\
%  {\color{purple} Text \num{70}}\\
%  \sisetup{colourunits=true,colourvalues=false}
%  \SI{40}{\second}
%\end{LaTeXexample}
%
%\DescribeOption{colourneg}
%\DescribeOption{negcolour}
%\SpecialOptionIndex{colorneg}
%\SpecialOptionIndex{negcolor}
% \currpkg can automatically add a colour to negative numbers.  This
% is turned on using the \opt{colourneg} switch.  The colour used is
% set by the \opt{negcolour} option; both options are available using
% US spellings.
%
%\subsection{International support}
%\DescribeOption{locale}
% \currpkg allows the user to switch between the typographic
% conventions of different (geographical) areas by using
% \latin{locales}.  Currently, the package is supplied with
% configurations for locales \opt{UK}, \opt{USA}, \opt{DE} (Germany)
% and \opt{ZA} (South Africa). The \opt{locale} option is used to
% switch to a particular locale.
%\begin{LaTeXexample}
%  \SI{1.234}{\metre}\\
%  \SI[locale=DE]{6.789}{\metre}
%\end{LaTeXexample}
%
%\DescribeOption{loctolang}
% Locales are distinct from \pkg{babel} languages, as typographic
% conventions are not tightly integrated with language.  However, it
% is useful to be able to associate a particular locale with a
% \pkg{babel} language.  The option \opt{loctolang} handles this, and
% expects pairs of values:
% \opt{loctolang=\meta{locale}:\meta{language}}.
%\begin{LaTeXexample}
%  \sisetup{loctolang={UK:UKenglish,DE:german}}\\
%  \SI{6.022e23}{\per\mole}\\
%  \selectlanguage{german}\\
%  \SI{6.022e23}{\per\mole}
%\end{LaTeXexample}
%
%\subsection{Package control}
%\DescribeOption{load}
%\DescribeOption{noload}
%\DescribeOption{alsoload}
% The package keeps most of the unit and abbreviations definitions in
% files separate from \file{siunitx.sty}.  To control what is loaded,
% three complementary options are provided, all of which take a list
% of one or more choices.  \opt{load} and \opt{alsoload} define which
% support configuration files are loaded.  The list in \opt{load}
% recognises the value \opt{default}, which is expanded to the normal
% list before loading. The difference between \opt{load} and
% \opt{alsoload} is that \opt{load} specifies the \emph{complete}
% list of files to load, whereas \opt{alsoload} adds to the existing
% list. To use the \opt{load} option successfully requires knowing
% everything that is needed. The \opt{noload} option can be used to
% delete one or more items from the \opt{load} list, without needing
% to know what is on it.\footnote{\opt{noload} does not prevent the
% loading of a file needed by one which is loaded.  Thus the package
% may internally override a \opt{noload} value if needed.}
%
%\DescribeOption{log}
%\DescribeOption{debug}
% To control data written to the \ext{log} file, the \opt{log} option
% is provided.  This takes a value from the list \defaultopt{normal},
% \opt{none}, \opt{minimal}, \opt{errors} and \opt{debug}.  As would
% be expected, these indicate the amount of detail written to the log
% file. As a shortcut to \opt{log=debug}, the package also recognises
% the \opt{debug} option directly.
%
%\DescribeOption{strict}
% Some users will want to stick closely to the official rules for
% typesetting units.  This could be made complicated if the options
% for non-standards behaviour could not be turned off.  The load-time
% option \opt{strict} resets package behaviour to follow the rules
% closely, and disables options which deviate from this.  If the
% package is loaded with the \opt{strict} option, all output is made
% in maths mode using the upright serif font.
%
%\subsection{Back-compatibility options}
%\DescribeOption{emulate}
% The package can emulate \SIunits, \SIstyle, \unitsdef, \units,
% \pkg{hepunits}, \pkg{fancyunits} and \pkg{fancynum}.  Giving the
% \opt{emulate=\meta{package}} option will give the desired
% emulation, and combinations which would be possible with the real
% packages will also work here. The package will recognise the
% options of the emulated packages.  This will automatically cause
% emulation to be switched on.
%
%\subsection{Summary of all options}
% \ref{tab:allopt} lists a summary of the package options (excluding
% those for backward-compatibility).  A reminder or the input format
% is also provided.
%\begin{longtable}{>{\ttfamily}lll}
%    \caption{All package options\label{tab:allopt}}\\
%    \toprule
%    Option & Type & Description \\
%    \midrule
%  \endfirsthead
%    \toprule
%    Option & Type & Description \\
%    \midrule
%  \endhead
%    \bottomrule
%    \multicolumn{3}{r}{\emph{Continued on next page}}
%  \endfoot
%    \bottomrule
%  \endlastfoot
%  addsign & List & Add sign to number \\
%  allowzeroexp & Boolean & Allow \num[allowzeroexp]{e0} \\
%  angformat & List & Conversion of angle format \\
%  anglesep & List or literal & Space between angle components \\
%  astroang & Boolean & Astronomy-style angles \\
%  closeerr & Literal & Closes potential-ambiguous error \\
%  closefrac & Literal & Closes potential-ambiguous fraction \\
%  color & Literal & Colour used for units and values \\
%  colour & Literal & Colour used for units and values \\
%  colorall & Boolean & Switch for colouring all output \\
%  colourall & Boolean & Switch for colouring all output \\
%  colorneg & Boolean & Colour negative numbers \\
%  colourneg & Boolean & Colour negative numbers \\
%  colorunits & Boolean & Switch for colouring units \\
%  colourunits & Boolean & Switch for colouring units \\
%  colorvalues & Boolean & Switch for colouring values \\
%  colourvalues & Boolean & Switch for colouring values \\
%  decimalsymbol & List or literal & Decimal symbol \\
%  debug & Boolean & Write debugging data to log \\
%  detectdisplay & Boolean & Treat display maths separately \\
%  digitsep & List & Separation of digits in large numbers \\
%  dp & Integer & Number of decimal places to \\
%   & & output numbers to \\
%  emulate & Modules & Emulation modules to load \\
%  errspace & List & Spacing of bracketed error \\
%  eVcorra & Length & Spacing correction in \eV \\
%  eVcorrb & Length & Spacing correction after \eV \\
%  expbase & List or literal & Base used for exponents \\
%  expproduct & List or literal & Product sign for exponents \\
%  fixdp & Boolean & Switch for fixing decimal \\
%   & & places of numbers \\
%  fixsf & Boolean & Switch for fixing significant \\
%    & & figures of numbers \\
%  fraction & List & Method used when setting \opt{per=frac} \\
%  inlinebold & List & Select how inline bold is tested \\
%  load & Modules & Modules to load \\
%  locale & Modules & Locale to follow \\
%  loctolang & Special & Associate locale with \pkg{babel}
%    language \\
%  log & List & Amount of data added to log \\
%  mathOmega & Literal & ``$\makeatletter\si@mathsOmega\makeatother$''
%    symbol in maths mode \\
%  mathcelsius & Literal & ``$\makeatletter\si@mathscelsius\makeatother$''
%    symbol in maths mode \\
%  mathdegree & Literal & ``$\makeatletter\si@mathsdegree\makeatother$''
%    symbol in maths mode \\
%  mathminute & Literal & ``$\makeatletter\si@mathsminute\makeatother$''
%    symbol in maths mode \\
%  mathmu & Literal & ``$\makeatletter\si@mathsmu\makeatother$''
%    symbol in maths mode \\
%  mathringA & Literal & ``$\makeatletter\si@mathsringA\makeatother$''
%    symbol in maths mode \\
%  mathrm & Csname & Roman maths font \\
%  mathsOmega & Literal & ``$\makeatletter\si@mathsOmega\makeatother$''
%    symbol in maths mode \\
%  mathscelsius & Literal & ``$\makeatletter\si@mathscelsius\makeatother$''
%    symbol in maths mode \\
%  mathsdegree & Literal & ``$\makeatletter\si@mathsdegree\makeatother$''
%    symbol in maths mode \\
%  mathsecond & Literal & ``$\makeatletter\si@mathssecond\makeatother$''
%    symbol in maths mode \\
%  mathsf & Csname & Sans serif maths font \\
%  mathsminute & Literal & ``$\makeatletter\si@mathsminute\makeatother$''
%    symbol in maths mode \\
%  mathsmu & Literal & ``$\makeatletter\si@mathsmu\makeatother$''
%    symbol in maths mode \\
%  mathsringA & Literal & ``$\makeatletter\si@mathsringA\makeatother$''
%    symbol in maths mode \\
%  mathsrm & Csname & Roman maths font \\
%  mathssecond & Literal & ``$\makeatletter\si@mathssecond\makeatother$''
%    symbol in maths mode \\
%  mathssf & Csname & Sans serif maths font \\
%  mathstt & Csname & Fixed-width maths font \\
%  mathtt & Csname & Fixed-width maths font \\
%  mode & List & Use text or maths mode for \\
%   & & typesetting \\
%  negcolor & Literal & Colour used for negative numbers \\
%  negcolour & Literal & Colour used for negative numbers \\
%  noload & Modules & Modules not to load \\
%  numaddn & Literal & Additional input allowed in numbers \\
%  numcloseerr & Literal & Character indicating end of \\
%   & & numerical error\\
%  numdecimal & Literal & Decimal symbols in numbers \\
%  numdigits & Literal & Digit characters in numbers \\
%  numdiv & Literal & Division characters in numbers \\
%  numexp & Literal & Exponent characters in numbers \\
%  numgobble & Literal & Characters to ignore in numbers \\
%  numopenerr & Literal & Character indicating start of \\
%   & & numerical error\\
%  numprod & Literal & Characters used for a product \\
%  numsign & Literal & Sign characters in numbers \\
%  obeyall & Boolean & Combination of \opt{obeybold}, \\
%   & & \opt{obeyitalic} and \opt{obeymode} \\
%  obeybold & Boolean & Check local bold setting \\
%  obeyitalic & Boolean & Check local italic setting \\
%  obeymode & Boolean & Check local mode (text/maths) \\
%  openerr & Literal & Opens potentially-ambiguous error \\
%  openfrac & Literal & Opens potentially-ambiguous fraction \\
%  padangle & List & Add zeros to blank parts of angles \\
%  padnumber & List & Add zeros to blank parts of numbers \\
%  per & List & Behaviour of \cs{per} \\
%  prefixbase & List or literal & Base used when making prefixes \\
%   & & numerical \\
%  prefixproduct & List or literal & Product sign for prefixes \\
%  prefixsymbolic & Boolean & Behaviour of unit prefixes\\
%  prespace & Boolean & Add space before units \\
%  redefsymbols & Boolean & Use better symbols if available \\
%  repeatunits & List & Repeat units with separated errors \\
%   & &  and products \\
%  retainplus & Boolean & Retain explicit plus sign\\
%  seperr & Boolean & Separate number and error \\
%  sepfour & Boolean & Separate four-digit numbers \\
%  sf & Integer & Number of significant figures \\
%   & & to output numbers to \\
%  sign & List or literal & Sign to add to numbers \\
%  slash & List or literal & Symbol used for ``/'' \\
%  stickyper & Boolean & Require \cs{per} only once \\
%  strict & Boolean & Obey the rules strictly \\
%  strictarc & Boolean & Require exactly zero or \\
%   & & two semi-colons \\
%  tabalign & List & Positioning of all column data \\
%  tabalignexp & Boolean & Alignment of exponents in tables \\
%  tabautofit & Boolean & Switch for rounding numbers to \\
%   & & length given by \opt{tabformat} \\
%  tabformat & Number & Space reserved in table for numbers \\
%  tabnumalign & List & Alignment of |S| column numbers \\
%  tabparseonly & Boolean & Do not align |S| columns \\
%   & & on decimal marker \\
%  tabtextalign & List & Positioning of text in |S| columns \\
%  tabunitalign & List & Positioning of units in |s| columns \\
%  textcelsius & Literal & ``$^{\circ}$C'' symbol in text mode \\
%  textdegree & Literal & ``${\circ}$'' symbol in text mode \\
%  textminute & Literal & ``$'$'' symbol in text mode \\
%  textmu & Literal & ``$\upmu$'' symbol in text mode \\
%  textomega & Literal & ``$\Omega$'' symbol in text mode \\
%  textringa & Literal & ``\AA'' symbol in maths mode \\
%  textrm & Csname & Roman text font \\
%  textsecond & Literal & ``$''$'' symbol in text mode \\
%  textsf & Csname & Sans serif text font \\
%  texttt & Csname & Fixed-width text font \\
%  tightpm & Boolean & Reduce space around $\pm$ \\
%  trapambigerr & Boolean & Check for ambiguous errors \\
%  trapambigfrac & Boolean & Check for ambiguous fractions \\
%  unitcolor & Literal & Switch for colouring units \\
%  unitcolour & Literal & Switch for colouring units \\
%  unitmathrm & Csname & Roman maths font for units \\
%  unitmathsf & Csname & Sans serif maths font for units \\
%  unitmathsrm & Csname & Roman maths font for units \\
%  unitmathssf & Csname & Sans serif maths font for units \\
%  unitmathstt & Csname & Fixed-width maths font for units \\
%  unitmathtt & Csname & Fixed-width maths font for units \\
%  unitmode & List &  As \opt{mode}, for units only \\
%  unitsep & List or literal & Separator for units \\
%  unitspace & List or literal & Space used for ``|~|'' in units \\
%  valuecolor & Literal & Switch for colouring value \\
%  valuecolour & Literal & Switch for colouring value \\
%  valuemathrm & Csname & Roman maths font for values \\
%  valuemathsf & Csname & Sans serif maths font for values \\
%  valuemathsrm & Csname & Roman maths font for values \\
%  valuemathssf & Csname & Sans serif maths font for values \\
%  valuemathstt & Csname & Fixed-width maths font for values \\
%  valuemathtt & Csname & Fixed-width maths font for values \\
%  valuemode & List &  As \opt{mode}, for values only \\
%  valuesep & List or literal & Separator between value and unit \\
%  xspace & Boolean & Use \pkg{xspace} after units \\
%\end{longtable}
%
%\section{Emulation of other packages}
% \currpkg has been designed as a replacement for \SIunits, \SIstyle,
% \unitsdef, \units, \pkg{hepunits}, \pkg{fancyunits} and
% \pkg{fancynum}. It therefore provides options reproduce the
% functions of all of these packages.  In this way, \currpkg should
% be usable as a straight replacement for the older
% packages.\footnote{User macros means that they are described in the
% package documentation; simply not containing an \texttt{@} does not
% mean they will have been emulated.}  This means for example that
% the \cs{num} macro takes an optional star when emulating \SIstyle.
% However, there are some points that should be remembered.  In
% particular, \currpkg validates numerical input, meaning that places
% where a number is expected in the older packages \emph{require} a
% number when emulated by \currpkg.
%
% The \numprint package has provided many useful ideas for the code
% used here for number formatting. The basic use of the \cs{numprint}
% (or \cs{np}) macro can be reproduced using \currpkg.  However,
% \numprint is large and complex, with its own backward-compatibility
% options.  As a result, emulation of \numprint is not provided here.
% To use a \numprint document with \currpkg, the \cs{numprint} macro
% could be provided using the following code.
%^^A Need to deal with \cs{numprint} being defined.
%\let\numprint\relax
%\begin{LaTeXexample}
%  \newcommand*{\numprint}[2][]{\SI[obeymode]{#2}{#1}}\\
%  \numprint{-123456} \\
%  \numprint[N/mm^2]{-123456}
%\end{LaTeXexample}
%^^A Back to the normal - with \cs{newcommand}
%\newcommand*{\numprint}{\pkg{numprint}\xspace}
%
% \currpkg can be used more-or-less directly to replace both
% \pkg{dcolumn} and \pkg{rccol}.  As is explained in the code
% section, much of the column-alignment system here is taken from
% \pkg{dcolumn}, while \pkg{rccol} provided a model for a
% customisable system.  However, neither package has been directly
% emulated here.  The |S| column type can be used to replace both |D|
% and |R| columns by setting the appropriate package options.
%
%\section{Configuration files}
% \currpkg is a modular package.  The unit definitions, abbreviations
% and locales are all stored in configuration files.  These all take
% names of the form \file{si-\meta{name}.cfg}, where \meta{name} is
% the part of the filename used as an option in \cs{sisetup} or when
% loading \currpkg.  Producing new configuration files therefore
% consists of making a suitably-named file and adding it to the path
% searched by \TeX.  The files should normally consist of settings
% (in \cs{sisetup}) and unit definitions, \etc.
%
%\DescribeMacro{\addtolocale}
% To allow arbitrary macros to be stored in locales, the
% \cs{addtolocale} macro is provided.  This ensures that arbitrary
% text is only executed when using a locale, not when loading it. The
% macro takes two arguments, \marg{locale} and \marg{code}.
%\begin{LaTeXexample}
%  \addtolocale{DE}{TEST}
%  Some text as filler \\
%  \sisetup{locale=DE}
%  This example is rather trivial!
%\end{LaTeXexample}
%
%\DescribeMacro{\requiresiconfigs}
% To load one or more configuration files from inside another
% configuration file, the \cs{requiresiconfigs} macro is provided.
% This accepts a comma-separated list of configuration names, in the
% same way as \opt{load} or \opt{noload}.
%
% In addition to the various configuration files provided with the
% package, a local file \file{siunitx.cfg} may be provided.  This is
% read at the end of loading \currpkg, and allows the user to include
% any local definitions or settings easily.
%
%\section{Common questions}
%\subsection{Why do I need \cs{per} more than once?}
% The unit engine of \currpkg is based around the English method for
% reading units out loud. Thus \si{\joule\per\mole\per\kelvin} is
% pronounced ``joules per mole per kelvin''.  Hence by default you
% need to put \cs{per} before each item in the denominator of a unit.
% The behaviour can be altered by setting the \opt{stickyper} option.
%
%\subsection{Why is the order of my units changed?}
% Then using \opt{per=reciprocal}, the units are typeset as given.
% However, when using \opt{per=slash} or \opt{per=fraction}, the
% package needs to find which ones are in the denominator.  It then
% prints the numerator and denominator separately.  So if you give a
% unit in the denominator \emph{before} one in the numerator, they
% have to be re-ordered.\footnote{``Double division''
% (\SI{1}{s/m/kg}) is mathematically incorrect.}
%\begin{LaTeXexample}
%  \SI{88}{\per\kilogram\metre} \\
%  \SI[per=slash]{66}{\per\kilogram\metre}
%\end{LaTeXexample}
%
%\subsection{Why are compound units not recommended outside of
%  \cs{SI}/\cs{si}?}
% To fully process units made up of several parts, the processor has
% to know where the end of the unit is.  When the unit macros are
% used outside of \cs{SI}/\cs{si}, this is not the case.  The package
% therefore does its best, but results may be sub-optimal.  To get
% consistent results, either define a new \emph{single} unit, or keep
% compound units inside \cs{SI} and \cs{si}.
%\begin{LaTeXexample}
%  \metre\per\micro\second \\
%  \si{\metre\per\micro\second} \\
%  \newunit{\myunit}{\metre\per\micro\second}
%  \myunit
%  \newunit{\myunittwo}{\kilogram\myunit} \\
%  \myunittwo \\
%  \kilogram\myunit
%\end{LaTeXexample}
% Notice the difference in behaviour of \cs{per} in the first two
% lines, and the spacing error on the last line in the example.
%
%\subsection{How do I set superscripts to use lining numbers?}
% Lowercase (``old style'') numbers are favoured by many people for
% use in running text.  However, this does not necessarily look good
% in superscripts. The mode used to typeset data can be varied, so
% that maths mode numbers are used for the unit part of the output.
%\begin{LaTeXexample}
%  \SI[mode=text]{1234}{\metre\per\second}\\
%  \SI[valuemode=text,unitmode=maths]{1234}{\metre\per\second}
%\end{LaTeXexample}
%
%\subsection{Why do most of the examples use
%  \texorpdfstring{\si{\joule\per\mole\per\kelvin}}{J mol-1 K-1}?}
% The package author is a chemist, and this is the unit of entropy
% (disorder).  It nicely demonstrates the use of the \cs{per} macro,
% and so it crops up a lot.  It is also in the subsection heading
% here to act as a test with \pkg{hyperref} and moving arguments!
%
%\subsection{What can \numprint do that \currpkg cannot?}
% \currpkg uses a lot of ideas from \numprint: a reasonable amount of
% the number-processing code here is based on that in \numprint.
% However, the two packages have somewhat different aims, and as a
% result there are things that \numprint can do that \currpkg does
% not implement.  The main features of \numprint not available here
% are:
%\begin{itemize}
%  \item General support for numbers with base other than $10$ (see
%    \pkg{nbaseprt});
%  \item Alignment of the decimal marker of powers in tables;
%  \item Alignment of numbers in running text;
%  \item Specific formatting commands for \TeX\ counters and lengths.
%\end{itemize}
%
%\section{Tricks and known issues}
%
%\subsection{Ensuring maths mode}
% Due to the possibility of output in either maths or text mode, any
% input which requires a particular mode needs to be protected.  You
% cannot use |$|\ldots|$|, as this can get ``caught out'', but also
% as it may give hard-to-follow errors.  Always use \cs{ensuremath}
% to force maths processing, and \cs{text} (from the \AMS\TeX\
% bundle) to ensure text mode.
%
%\subsection{Using \texttt{.} and fixed spaces in units}
% To use a literal |.| in a unit, it has to be within an extra set of
% braces.  This does not need any extra protection, unlike the
% situation with \SIstyle (for example, no \cs{text} macro is
% needed).  The fixed space (|~|) is more problematic: set
% \opt{unitspace=space} to get a full space here.
%\begin{LaTeXexample}
%  \newunit[unitspace=space]{\myunit}{V~vs{.}~NHE}%
%  \SI{10}{\myunit}
%\end{LaTeXexample}
%
%\subsection{Passing unprocessed digits through an \texttt{S}
%  column}
% The method used to detect numbers in an |S| column will pick up
% material wrapped inside braces if there is more than a single
% character inside the braces.  If you want to pass a \emph{single}
% numerical character without processing it, you need two sets of
% braces (\ref{tab:S-limits}).
%\begin{LaTeXexample}[float]
%  \begin{table}
%    \centering
%    \caption{Passing single digit characters}
%    \label{tab:S-limits}
%    \begin{tabular}{S[colourall,colour=orange]}
%      \toprule
%        {Heading} \\
%      \midrule
%        {1-2} \\
%        {1}   \\
%        {{1}} \\
%        {{-}} \\ % Using {-} gives an error!
%      \bottomrule
%    \end{tabular}
%  \end{table}
%\end{LaTeXexample}
%
%\subsection{Limitations of \cs{mathrm}}
% The package uses the \cs{mathrm} font family by default to typeset
% output in maths mode.  This however has a few side-effects.  For
% example, the Greek alphabet can give odd results.\footnote{This
% depends on your font setup; this document uses T1 encoding, which
% shows the issue, whereas using OT1 does not.}  The use of the
% \cs{mathnormal} font \emph{may} get around this issue.
%\begin{LaTeXexample}
%  \num[numaddn=\pi]{4\pi e-7}\\
%  \num[numaddn=\pi,mathsrm=mathnormal]{4\pi e-7}
%\end{LaTeXexample}
% On the other hand, you may want to use text mode, in which case
% \cs{ensuremath} is needed.  Depending on the exact circumstances,
% the \LaTeX\ protection mechanism (\cs{DeclareRobustCommand}) may be
% sufficient; in some cases, this will fail and the \eTeX\
% \cs{protected} system may be required. There are several potential
% pitfalls in this area; experimentation may well be needed.
%\begin{LaTeXexample}
%  \DeclareRobustCommand*{\numpi}{\ensuremath{\pi}}
%  \num[numaddn=\numpi,mode=text]{4\numpi e-7}
%\end{LaTeXexample}
%
%\subsection{Entire document in sans serif font}
% If your entire document is not in a Roman font, using
% the font detection system is not the most efficient method
% for setting the \currpkg output.  Instead, the \opt{mathrm}
% and \opt{textrm} package options can be redefined.
%\begin{LaTeXexample}
%  \sffamily
%  Some text \\
%  \sisetup{obeyfamily=false,mathrm=mathsf,textrm=sffamily}
%  \num{1e2} \\
%  \SI{3}{\newton}
%  \[ \SI{4e5}{\pascal} \]
%\end{LaTeXexample}
%
%\subsection{Effects of emulation}
% The package has been designed so that almost everything can be set
% using the options.  In the emulation code, some internal macros are
% redefined.  This is because the legacy packages do odd things,
% which are deliberately not implemented by \currpkg.  Using an
% emulation file will prevent subsequent loading of the real package.
% This is to prevent errors or, worse, difficult to diagnose changes
% to output.
%
%\subsection{Centring columns on non-decimal input}
% The \pkg{dcolumn} manual suggests using that package to align a
% column on a $\pm$ sign. The same type of output is possible using
% \currpkg, but some care is needed (\ref{tab:dcolumn}).  Odd things
% may happen: use with care!
%\begin{LaTeXexample}[float]
%  \begin{table}
%    \centering
%    \caption{Non-standard \texttt{S} column}
%    \label{tab:dcolumn}
%    \begin{tabular}{%
%      S[digitsep=none,decimalsymbol={\,\pm\,},
%        numdigits={0123456789.},numdecimal=+]}
%    \toprule
%    {Some Values} \\
%    \midrule
%    2.3456  + 0.02  \\
%    34.2345 + 0.001 \\
%    56.7835 + 0.067 \\
%    90.473  + 0.021 \\
%    \bottomrule
%    \end{tabular}
%  \end{table}
%\end{LaTeXexample}
%
%\subsection{Expanding content in tables}
% When processing |S| columns, \currpkg works hard to expand any
% items which may give numbers.  So for example you can define values
% as macros (\ref{tab:vmacros}). To avoid the expansion of single
% macros, you must either wrap them in two sets of braces or make
% them robust (using either \cs{DeclareRobustCommand} or
% \cs{protected}).
%\begin{LaTeXexample}[float]
%  \begin{table}
%    \centering
%    \caption{Values as macros}
%    \label{tab:vmacros}
%    \newcommand*{\myvala}{1.234}%
%    \newcommand*{\myvalb}{20.345}%
%    \newcommand*{\myvalc}{0.987654}%
%    \DeclareRobustCommand*{\myvald}{88.88}%
%    \begin{tabular}{S[tabtextalign=left]}
%    \toprule
%    {Some Values} \\
%    \midrule
%    \myvala  \\
%    \myvalb \\
%    \myvalc \\
%    {{\myvala}} \\
%    \myvald \\
%    \bottomrule
%    \end{tabular}
%  \end{table}
%\end{LaTeXexample}
%
% It is possible to use calculated values in tables. For this to
% work, the calculation must take place before attempting to parse
% the number. This means that any calculation code should be wrapped
% in braces and appear before the use of the number. This is
% illustrated by using the \pkg{datatool} package to set up a
% miniature database, and then output the values doubled
% (\ref{tab:calc}). This example also shows that macros such as
% \cs{DLTforeach}, which construct a table while \TeX\ is running,
% cannot be placed inside an |S| column (or an |s| one). Instead, an
% additional dummy column has been added with no inter-column space.
% This is used to contain the tale-building macro.
%\begin{LaTeXexample}[float]
%  \DTLnewdb{data}
%  \DTLnewrow{data}\DTLnewdbentry{data}{value}{66.7012}
%  \DTLnewrow{data}\DTLnewdbentry{data}{value}{66.0212}
%  \DTLnewrow{data}\DTLnewdbentry{data}{value}{64.9026}
%  \begin{table}
%    \caption{Calculated values}
%    \label{tab:calc}
%    \centering
%    \sisetup{tabformat=2.4}
%    \begin{tabular}{SS@{}l}
%      \toprule
%      {Value} & {Doubled}  &
%      \DTLforeach{data}{\myvalue=value}{%
%        \DTLiffirstrow {\\ \midrule}{\\}%
%        \myvalue & % First column
%        {\DTLmul{\myvalue}{\myvalue}{2}} \myvalue % second column
%        & }\\
%      \bottomrule
%    \end{tabular}
%  \end{table}
%\end{LaTeXexample}
%
%\subsection{Adding items after the last column of a tabular}
% If you use an |S| or |s| column as the last one in a tabular, and
% you use the \pkg{array} ``|<|'' construction to add items after it,
% the spacing may be wrong.  This will occur if the column contents
% are of differing widths.  Changing the \LaTeX\ |\\| line ending for
% the plain \TeX\ \cs{cr} will give the correct spacing, but does not
% allow adjustment of inter-row distance (\ref{tab:cr}).\footnote{For
% the \TeX\ experts, the issue here is that the system to gather up
% cell contents is added in using the \texttt{<} construction.
% Normally, this comes after the cell contents and any other
% \texttt{<} arguments, so collects the user additions.  However, in
% the last cell the contents include \texttt{\bslash\bslash}, which
% is converted to \cs{cr} before gathering can occur.  By using
% \cs{cr} directly, the gathering process receives all of the cell
% contents as normal.}  In most cases, this should not be a serious
% issue. Notice that an extra set of braces is needed here, compared
% to usual \LaTeX\ use; this is to prevent any expansion of the
% material by \currpkg.
%\begin{LaTeXexample}[float]
%  \begin{table}
%    \caption{Correcting spacing in last \texttt{S} column}
%    \label{tab:cr}
%    \hfil
%    \begin{tabular}{S<{{\,\si{\kg}}}S<{{\,\si{\kg}}}}
%      \toprule
%      \multicolumn{1}{c}{Long header} &
%        \multicolumn{1}{c}{Long header} \\
%      \midrule
%      1.23 & 1.23 \\
%      4.56 & 4.56 \\
%      7.8  & 7.8  \\
%      \bottomrule
%    \end{tabular}
%    \hfil
%    \begin{tabular}{S<{{\,\si{\kg}}}S<{{\,\si{\kg}}}}
%      \toprule
%      \multicolumn{1}{c}{Long header} &
%        \multicolumn{1}{c}{Long header} \\
%      \midrule
%      1.23 & 1.23 \cr
%      4.56 & 4.56 \cr
%      7.8  & 7.8  \cr
%      \bottomrule
%    \end{tabular}
%    \hfil
%  \end{table}
%\end{LaTeXexample}
%
%\subsection{Using \currpkg\ with the \pkg{cellspace} package}
% Both \currpkg\ and \pkg{cellspace} use the letter \texttt{S} for a 
% new column type. This obviously leads to a problem. If both are 
% loaded, \currpkg\ will retain the \texttt{S} column, and moves the
% functionality of \pkg{cellspace} to the letter \texttt{C}. This 
% allows the normal use of \pkg{cellspace} with standard column types:
% it does \emph{not} work with the \currpkg\ \texttt{S} or \texttt{s}
% columns.
% 
%\subsection{Using \currpkg\ with the \pkg{mathabx} package}
%\changes{v1.1g}{2008/11/26}{Respect loading of \pkg{mathabx}}
% The \pkg{mathabx} package defines its own \cs{second} and 
% \cs{degree} macros.  This is respected by \currpkg: if you want the
% \currpkg\ definitions instead, put the lines
%\begin{LaTeXexample}[noexample]
%  \makeatletter
%  \renewunit{\second}{s}
%  \renewunit[valuesep=none]{\degree}{\si@sym@degree}
%  \makeatother
%\end{LaTeXexample}
% after |\begin{document}|.
%
%\iffalse
%<*bug>
%\fi
%\section{Reporting a problem}
% \currpkg is quite long and complicated, and works hard to cover all
% possible eventualities.  However, there will be bugs in the code
% and unexpected interactions with other packages.  If you think you
% have found a bug, please report it.  A short test-case
% demonstrating the problem would be very welcome.  The following is
% a suitable template, and is available as \file{si-bug.ltx} in the
% \texttt{doc/latex/siunitx} directory or by running the \ext{dtx} or
% \ext{ins} file through \TeX.
%    \begin{macrocode}
\listfiles
\documentclass{article}
%    \end{macrocode}
% Load other packages needed here.
%    \begin{macrocode}
\usepackage{siunitx}
%    \end{macrocode}
% Normally, debugging the load procedure will not be needed; the
% debug option here means that all run-time information is logged.
%    \begin{macrocode}
\sisetup{debug}
\begin{document}
This is the bug test-case document for the \textsf{siunitx}
package.\\
Please put your demonstration here, and e-mail to the package
author.
\begin{center}
  \texttt{joseph.wright@morningtar2.co.uk}
\end{center}
\end{document}
%    \end{macrocode}
%\iffalse
%</bug>
%\fi
%
%\section{Feature requests}
% Feature requests for \currpkg are welcome.  The package maintainer
% will consider any ideas within the remit of the package (units and
% values).  If suggesting a new feature, an example of how it should
% work would be appreciated.  If new controls are needed, some
% suggestions for option names would be welcome.
%
%\section{Acknowledgements}
% Many thanks indeed to Stefan Pinnow, who has made a very large
% number of suggestions and found numerous bugs in the package.  His
% contribution to the package has been vital. The package author has
% learned \LaTeX\ tricks from far too many people to thank all of
% them.  However, for this package specific thanks must go to the
% authors of the existing ``unit'' packages: Danie Els (\SIstyle),
% Marcel Heldoorn (\SIunits), Patrick Happel (\unitsdef), Axel
% Reichert (\units) and Harald Harders (\numprint).  Will Robertson
% and Heiko Oberdiek deserve much credit for demonstrating \LaTeX\
% coding best practice. Victor Eijkhout's excellent (and free)
% \emph{\TeX\ by Topic} has provided some useful coding hints
% \cite{Eijkhout}.  The idea for combining and extending unit
% provision in \LaTeX\ was heavily inspired by Philip Lehmann's
% \pkg{biblatex}. Thanks to the various contributors of ideas for the
% package: Donald Arseneau, Michele Dondi, Paul Gans, Ben Morrow, Lan
% Thuy Pham, Alan Ristow, Patrick Heinze, Andrea Blomenhofer, Morten
% H{\o}gholm, Burkhard Moddemann  and Patrick Steegstra.
%
%\clearpage
%\part{Correct application of (SI) units}
%\section{Background}
% Consistent and logical units are a necessity for scientific work,
% and have applicability everywhere.  Historically, a number of
% systems have been used for physical units.  SI units were
% introduced by the \latin{Conf\'erence G\'en\'erale des Poids et
% Mesures} (CGPM) in 1960.  SI units are a coherent system based on
% seven base units, from which all other units may be derived.
%
% At the same time, physical quantities with units are mathematical
% entities, and as such way that they are typeset is important.  In
% mathematics, changes of type (such as using bold, italic, sans
% serif typeface and so on) convey information.  This means that
% rules exist not only for the type of units to be used under the SI
% system, but also the way they should appear in print.  Advice on
% best practice has been made available by the \emph{National
% Institute of Standards and Technology} (NIST) \cite{NIST}.
%
% As befits an agreed international standard, the full rules are
% detailed. It is not appropriate to reproduce these in totality
% here; instead, a useful summary of the key points is provided. The
% full details are available from the \latin{Bureau International des
% Poids et Mesures} (BIPM) in French \cite{BIPM:FR} and English
% \cite{BIPM:EN}.  They also publish a very useful and detailed guide
% to using units, values and so on, available online in a number of
% different formats \cite{BIPM:PDF}.
%
% \currpkg takes account of the information given here, so far as is
% possible.  Thus the package defaults follow the recommendations
% made for typesetting units and values.  Spacing and so forth is
% handled in such a way as to make implementing the rules
% (relatively) easy.
%
%\section{Units}
%\subsection{SI base units}
% There are seven base SI units, listed in \ref{tab:baseunits}. Apart
% from the kilogram, these are defined in terms of a measurable
% physical quantity needing the definition alone.\footnote{Some base
% units need others defined first; there is therefore a required
% order of definition.}  The base units have been chosen such that
% all physical quantities can be expressed using an appropriative
% combination of these units, needing no others and with no
% redundancy.  The kilogram is slightly different from the other base
% units as it is still defined in terms of a ``prototype'' held in
% Paris.\footnote{At the time of writing, this is under review and
% will be altered.}
%
%\subsection{SI derived units}
% All other units within the SI system are regarded as ``derived''
% from the seven base units.  At the most basic, all other SI units
% can be expressed as combinations of the base units.  However, many
% units (listed in \ref{tab:derived}, \ref{tab:experimental} and
% \ref{tab:accepted}) have a special name and symbol.\footnote{The
% nautical mile has a given name but no agreed symbol, and although
% accepted by the SI is not provided by \currpkg as a unit macro.}
% Most of these units are simple combinations of one or more base
% units (raised to powers as appropriate).  A small number of units
% derived from experimental data are allowed as SI units
% (\ref{tab:experimental}).
%
% Some of these units (in \ref{tab:accepted}) are regarded as only
% ``temporarily'' accepted, as the use of only the base and
% fully-consistent derived units in \ref{tab:derived} is encouraged.
% They are accepted as they are in common use in one or more
% disciplines; some are still very widespread in the appropriate
% areas.  These units are mainly multiples of base units (for
% example, a tonne is \SI{1000}{\kilo\gram}).
%
% One point to note is that ``unitless ratios'' are regarded as
% having base units which cancel out.  For example, the radian is
% regarded as having base unit \si{\metre\per\metre}.  The result of
% this division (``$1$'') is therefore regarded as a derived SI unit
% in this context.
%
%\subsection{SI prefixes}
% A series of SI prefixes for decimal multiples and submultiples are
% provided, and can be used as modifiers for any SI unit (either base
% or derived units) with the exception of the kilogram.  The prefixes
% are listed in \ref{tab:prefix}.  No space should be used between a
% prefix and the unit, and only a single prefix should be used.  Even
% the degree Celsius can be given a prefix, for example
% \SI{1}{\milli\celsius}.  The only exception to this rule is for
% degrees, minutes and seconds of an arc: \ang{1;2;3}.
%
% It is important to note that the kilogram is the only SI unit with
% a prefix as part of its name and symbol.  Only  single prefix may
% be used, and so in the case of the kilogram prefix names are used
% with the unit name ``gram'' and the prefix symbols are used with
% the unit symbol \si{\gram}.  For example $\SI{1e-6}{\kilogram} =
% \SI{1}{\milli\gram}$.
% 
% Three unit-prefix combinations deserve special note. The units
% hectare, kilohm and megohm elide the additional vowel (these
% would otherwise be hectoare, kiloohm and megaohm).  All other
% vowel combinations are retained.\footnote{Thanks to Charles
% Cameron for raising this issue with the package author.}
%
%\subsection{Other units}
% The application of SI units is meant to provide a single set of
% units which ensure consistency and clarity across all areas.
% However, other units are common is many areas, and are not without
% merit. The units provided by \currpkg by default do not included
% any of these; only units which are part of the SI set or are
% accepted for use with SI units are defined.  However, several other
% sets of units can be loaded as optional modules. The binary
% prefixes and units (Section~\ref{sec:binary} and \ref{tab:binary})
% are the most obvious example.  These are \emph{not} part of the SI
% specifications, but the prefix names are derived from those in
% \ref{tab:prefix}.
%
% Other units (such as those provided by the modules \opt{synchem},
% \opt{hep} and \opt{astro}) are normally to be avoided where
% possible.  SI units should, in the main, be preferred due to the
% advantages of clear definition and self-consistency this brings.
% However, there will probably always be a place for specialist or
% non-standard units.  This is particulary true of units derived from
% basic physical constants; for example reason, the \opt{hep} module
% defines the speed of light, \si{\clight}, as a unit.  For work in
% basic science, a small number of physical constants are recognised
% as units provided the results for comparison with experiment are
% given in SI units.
%
% There are also many areas where non-standard units are used so
% commonly that to do otherwise is difficult or impossible.  For
% example, most synthetic chemists measure the pressure inside vacuum
% apparatus in \si{\mmHg}, partly because the most common gauge for
% the task still uses a column of mercury metal.  For these reasons,
% \currpkg does define non-SI units.
%
%\section{Units and values in print}
%\subsection{Mathematical meaning}
% As explained earlier, a unit--value combination is a single
% mathematical entity.  This has implications for how both the number
% and the unit should be printed.  Firstly, the two parts should not
% be separated.  With the exception of the symbols for plane angles
% (\si{\degree}, \si{\arcmin} and \si{\arcsec}), it is usual to have
% a space between the unit and the value.  This should therefore be a
% non-breaking space between the two.  Different geographical areas
% have different conventions on the size of this space; a ``small''
% space (|\,|) is the \currpkg default.
%\begin{LaTeXexample}
%  A space for \SI{10}{\percent}\\
%  and also for \SI{100}{\celsius}\\
%  but not for \ang{1.23}.
%\end{LaTeXexample}
%
% The mathematical meaning of units also means that the shape, weight
% and family are important.  Units are supposed to be typeset in an
% upright, medium weight serif font.  Italic, bold and sans serif are
% all used mathematically to convey other meanings.  \currpkg package
% defaults again follow this convention: any local settings are
% ignored, and uses the current upright serif maths font.  However,
% there are occasions where this may not be the most desirable
% behaviour.  A classic example would be in an all-bold section
% heading.  As the surrounding text is bold, some people feel that
% any units should follow this.
%\begin{LaTeXexample}
%  Units should \textbf{not be bold: \SI{54}{\farad}}\\
%  \textbf{But perhaps in a running block,\\
%  it might look better: \SI[obeybold]{54}{\farad}}
%\end{LaTeXexample}
%
%\subsection{Unit multiplication and division}
% Symbols for units formed from other units by multiplication are
% indicated by means of either a half-height (that is, centred) dot
% or a (thin) space.  This document uses a half-height dot as
% (\emph{i}) this is the recommendation of \textsc{nist}, amongst
% others and (\emph{ii}) it avoids potential confusion between unit
% prefixes and multiplied units.
%\begin{LaTeXexample}
%  $\si{\metre\second} = \mbox{metre second}$ \\
%  $\si{\milli\second} = \mbox{millisecond}$ \\
%  \sisetup{unitsep=thin}
%  $\si{\metre\second} = \mbox{metre second}$ \\
%  $\si{\milli\second} = \mbox{millisecond}$
%\end{LaTeXexample}
% There are some circumstances under which it is permissable to omit
% any spaces.  The classic example is \si{\kWh}, where
% ``\si[unitsep=thin]{\kWh}'' does not add any useful information.
% If using such a unit repeatedly, users of \currpkg are advised to
% create a custom unit to ensure consistency.\footnote{\cs{kWh} and
% \cs{kilowatthour} are defined by \currpkg in this way.}
%
% Symbols for units formed from other units by division are indicated
% by means of a virgule (oblique stroke, slash, |/|), a horizontal
% line, or negative exponents.\footnote{Notice that a virgule and a
% solidus are not the same symbol.}  However, to avoid ambiguity, the
% virgule must not be repeated on the same line unless parentheses
% are used. This is ensured when using named unit macros in \currpkg,
% which will ``trap'' repeated division and format it correctly.  In
% complicated cases, negative exponents are to be preferred over
% other formats.
%\begin{LaTeXexample}
%  \si{\joule\per\mole\per\kelvin}\\
%  \si[per=fraction]{\joule\per\mole\per\kelvin}\\
%  \si[per=slash]{\joule\per\mole\per\kelvin}
%\end{LaTeXexample}
%
%\subsection{Repeating units}
% Products and errors should show what unit applies to each value
% given.  Thus \SI{2x3}{\metre} is an ordered set of lengths of a
% geometric area, whereas \SI[repeatunits=false]{2x3}{\metre} is a
% length (and equal to \SI{6}{\metre}). Thus, $\times$ is not a
% product but is a mathematical operator; in the same way, a $2
% \times 3$ matrix is not a $6$ matrix!  In some areas, areas and
% volumes are given with separated units but a unit raised to the
% appropriate power: \SI[repeatunits=power]{2 x 3}{\metre}. Although
% this does display the correct overall units, it is
% potentially-confusing and is not encouraged.
%
%\subsection{Clarity in writing values of quantities}
% Care must be taken when writing ranges of numbers.  For purely
% numerical values, it is common to use a en-dash between values, for
% example ``see pages 1--5''.  On the other hand, values with units
% could be misinterpret as negative values if written in this way. As
% the unit--value combination is a single mathematic entity, writing
% the values with an en-dash followed by a single unit is also
% incorrect.  As a result, using the word ``to'' is strongly
% recommended.
%\begin{LaTeXexample}
%  \SI{1}{\metre} to \SI{5}{\metre} long.
%\end{LaTeXexample}
%
%\subsection{Graphs and tables}
% In graphs and tables, repetition of the units following each entry
% or axis mark is confusing and repetitive.  It is therefore best to
% place the unit in the label part of the information.  Placing the
% unit in square brackets is common but mathematically
% poor.\footnote{For example, for an acceleration $a$, the expression
% $[a]$ is the dimensions of $a$, \ie length per time squared in this
% case.}  Much better is to show division of all values by the unit,
% which leaves the entries as unitless ratios.  This is illustrated
% in \ref{tab:label} and \ref{fig:label}.
%\begin{LaTeXexample}[float]
%  \begin{table}
%    \centering
%    \caption{An example of table labelling}
%    \label{tab:label}
%    \begin{tabular}{cS[tabformat=1.4,tabnumalign=centre]}
%      \toprule
%      {Entry} & {Length/\si{\metre}} \\
%      \midrule
%      1 & 1.1234 \\
%      2 & 1.1425 \\
%      3 & 1.7578 \\
%      4 & 1.9560 \\
%      \bottomrule
%    \end{tabular}
%  \end{table}
%\end{LaTeXexample}
%\begin{LaTeXexample}[float]
%  \begin{figure}
%    \centering
%    \begin{tikzpicture}
%      \begin{axis}[xlabel=$t/\si{\second}$,ylabel=$d/\si{\metre}$]
%        \addplot[smooth,mark=x]
%          plot coordinates {
%            (0,0)
%            (1,5)
%            (2,8)
%            (3,9)
%            (4,8)
%            (5,5)
%            (6,0)
%           };
%      \end{axis}
%    \end{tikzpicture}
%    \caption{An example of graph labelling}
%    \label{fig:label}
%  \end{figure}
%\end{LaTeXexample}
%
% In most cases, adding exponent values in the body of a table is
% less desirable than adding a fixed exponent to column headers.  An
% example is shown in \ref{tab:exp}.  The use of \cs{multicolumn} is
% needed here due to the ``|<|''; without \cs{multicolumn}, the
% titles are followed by ``\si{\kilo\gram}''!
%\begin{LaTeXexample}[float]
%  \begin{table}
%    \centering
%    \caption{Good and bad columns}
%    \label{tab:exp}
%    \sisetup{tabnumalign=centre}
%    \begin{tabular}{%
%      c%
%      S[tabformat=1.3e1]<{\,\si{\kilogram}}%
%      S[tabformat=2.2]}
%      \toprule
%      Entry & \multicolumn{1}{c}{Mass} &
%        {Mass/\SI{e3}{\kilogram}} \\
%      \midrule
%      1 & 4.56e3  & 4.56 \\
%      2 & 2.40e3  & 2.40 \\
%      3 & 1.345e4 & 13.45 \\
%      4 & 4.5e2   & 0.45 \\
%      \bottomrule
%    \end{tabular}
%  \end{table}
%\end{LaTeXexample}
%
%\bibliographystyle{unsrt}
%\StopEventually{%
%  \clearpage
%  \part{Notes}
%  \PrintChanges
%  \PrintIndex
%  \bibliography{siunitx}}
%
% \iffalse
%<*package>
% \fi
%
%\clearpage
%\part{Implementation}
%\section{Main package}
% Much of the code here is taken, with little or no modification,
% from the existing packages.  These are all released under the LPPL,
% and so this use is entirely allowed.  Rather than confuse the
% source here with repeated references, note that code here could be
% copied from \SIstyle, \SIunits, \numprint, \unitsdef or \units.
% Some ideas have also been borrowed from \pkg{biblatex}; again these
% will not be specifically noted. Code from other packages will be
% marked when used.
%
% User-space commands (those not containing |@|) defined here should
% give the same result as macros with the same name in the older
% packages.\footnote{Although extra optional arguments may be added.}
% However, internal package macros will behave differently; if the
% user has redefined internal macros, then compatibility will be
% impaired.
%
% The code used here uses \LaTeX\ rather than \TeX\ commands where
% possible.\footnote{This applies to \LaTeX\ kernel commands only;
% for example, \pkg{ifthenelse} is not used.}  For example,
% \cs{newcommand*} is used in place of \cs{def}, unless custom
% parameters are needed.  Hopefully, this will aid future
% maintenance.  Grouping is used where possible to limit the scope of
% temporary assignments.
%
%\subsection{Setup code}
% As always, the package starts with identification.
%    \begin{macrocode}
\NeedsTeXFormat{LaTeX2e}
\ProvidesPackage{siunitx}
  [2008/12/22 v1.1j A comprehensive (SI) units package]
%    \end{macrocode}
% The package requires \eTeX, so the usual test is made.
%    \begin{macrocode}
\begingroup
  \@ifundefined{eTeXversion}
    {\PackageError{siunitx}
       {Not running under e-TeX}
       {This package requires e-TeX. Try compiling the document
        with\MessageBreak `elatex' instead of `latex'. When
        using pdfTeX, try `pdfelatex'\MessageBreak instead of
        `pdflatex'}%
     \endgroup\endinput}
    {\endgroup}
%    \end{macrocode}
%\begin{macro}{\si@catcodes}
% There are circumstances under which some odd category codes might
% be in place.  The category codes for |{|, |}|, |[|, |]| and |#| are
% assumed to be okay; if issues arise, this can be altered.  \LaTeX\
% will have set |@| as a letter on loading \currpkg.
%    \begin{macrocode}
\edef\si@catcodes{%
  \catcode\string`\string ` \the\catcode\string`\`\relax
  \catcode\string`\string = \the\catcode\string`\=\relax
  \catcode\string`\string ^ \the\catcode\string`\^\relax
  \catcode\string`\string _ \the\catcode\string`\_\relax
  \catcode\string`\string ~ \the\catcode\string`\~\relax
  \catcode\string`\string : \the\catcode\string`\:\relax
  \catcode\string`\string - \the\catcode\string`\-\relax
  \catcode\string`\string + \the\catcode\string`\+\relax
  \catcode\string`\string ; \the\catcode\string`\;\relax
  \catcode\string`\string , \the\catcode\string`\,\relax
  \catcode\string`\string . \the\catcode\string`\.\relax}
\catcode\string`\` 12\relax
\catcode`\= 12\relax
\catcode`\^ 7\relax
\catcode`\_ 8\relax
\catcode`\~ \active\relax
\@makeother{\:}
\@makeother{\-}
\@makeother{\+}
\@makeother{\;}
\@makeother{\,}
\@makeother{\.}
%    \end{macrocode}
%\end{macro}
% Packages needed for functionality are loaded.  \pkg{xkeyval}
% handles the package options, while \pkg{amstext} from the \AMS\
% bundle is needed for \cs{text}.  \pkg{array} is needed for the new
% column type for tabular material.  \pkg{xspace} provides ``magic''
% spacing after macros, if requested.
%    \begin{macrocode}
\RequirePackage{xkeyval}[2005/05/07]
\RequirePackage{amstext,array,xspace}
%    \end{macrocode}
%\begin{macro}{\si@tempa}
%\begin{macro}{\si@tempb}
%\begin{macro}{\si@tempc}
% Some scratch commands are defined; apart from where a known value
% is carried through, these could contain anything.
%    \begin{macrocode}
\newcommand*{\si@tempa}{}
\newcommand*{\si@tempb}{}
\newcommand*{\si@tempc}{}
%    \end{macrocode}
%\end{macro}
%\end{macro}
%\end{macro}
%\begin{macro}{\ifsi@switch}
% Various items will need a switch.  To avoid name pollution, a
% single switch is defined here; grouping will keep the definition
% local.
%    \begin{macrocode}
\newif\ifsi@switch
%    \end{macrocode}
%\end{macro}
%\begin{macro}{\si@tempboxa}
%\begin{macro}{\si@tempboxb}
%\begin{macro}{\si@tempboxc}
%\begin{macro}{\si@tempboxd}
% Some boxes are also needed.
%    \begin{macrocode}
\newbox\si@tempboxa
\newbox\si@tempboxb
\newbox\si@tempboxc
\newbox\si@tempboxd
%    \end{macrocode}
%\end{macro}
%\end{macro}
%\end{macro}
%\end{macro}
%\begin{macro}{\si@temptoks}
% A token register is also handy.
%    \begin{macrocode}
\newtoks{\si@temptoks}
%    \end{macrocode}
%\end{macro}
%\begin{macro}{\si@packagecheck}
%\begin{macro}{\si@blockpkgs}
%\begin{macro}{\si@checkpkgs}
% As \currpkg is intended to replace the other unit-management
% packages, these are tested for before any further processing.  If
% any are loaded, the package halts compilation; name clashes or
% unexpected results could occur if this is not tested.  Notice that
% \SIunits and \SIstyle could be loaded with variable capitalisation
% (at least on Windows); both possibilities are tested.
%    \begin{macrocode}
\newcommand*{\si@blockpkgs}{SIunits,sistyle,siunits,SIstyle,%
  unitsdef,fancyunits}
\newcommand*{\si@checkpkgs}{units,hepunits,fancynum}
\newcommand*{\si@packagecheck}{%
  \begingroup
  \@for\si@tempa:=\si@blockpkgs\do{
    \@ifpackageloaded{\si@tempa}
      {\PackageError{siunitx}
         {Package `\si@tempa' incompatible}
         {The \si@tempa\space package and siunitx are
          incompatible.\MessageBreak Use the
          `emulate=\si@tempa' package option when loading
          siunitx}}
      {}}
%    \end{macrocode}
% Some packages should not cause a clash, but are emulated and would
% be better handled that way.
%    \begin{macrocode}
  \@for\si@tempa:=\si@checkpkgs\do{%
    \@ifpackageloaded{\si@tempa}
      {\PackageWarning{siunitx}
         {Consider loading the siunitx package
          with\MessageBreak option `emulate=\si@tempa', rather
          than\MessageBreak loading both \si@tempa\space and
          siunitx}}
      {}}
  \endgroup}
%    \end{macrocode}
% The check is carried out on loading and at the beginning of the
% document, so that packages loaded both before and after \currpkg
% are caught.
%    \begin{macrocode}
\si@packagecheck
\AtBeginDocument{\si@packagecheck}
%    \end{macrocode}
%\end{macro}
%\end{macro}
%\end{macro}
%\begin{macro}{\si@ifdefinable}
%\darg{macro}
% Using \cs{@ifdefinable} to check macro definitions gives a generic
% error.  To give something more helpful, \cs{@ifundefined} is used,
% but this needs some \cs{expandafter} work.  This way it can also be
% used as a form of \cs{@ifundefined} for macro names.
%    \begin{macrocode}
\newcommand*{\si@ifdefinable}[1]{%
  \expandafter\expandafter\expandafter\@ifundefined%
    \expandafter\expandafter\expandafter%
      {\expandafter\@gobble\string#1}}
%    \end{macrocode}
%\end{macro}
%\begin{macro}{\si@addtolist}
%\darg{macro}
%\darg{items}
% It is quite useful to be able to add to a comma-separated list of
% expandable items.
%    \begin{macrocode}
\newcommand*{\si@addtolist}[2]{%
  \ifx\@empty#1\@empty
    \edef#1{#2}%
  \else
    \edef#1{#1,#2}%
  \fi}
%    \end{macrocode}
%\end{macro}
%\begin{macro}{\si@addtocsname}
%\darg{csname}
%\darg{tokens}
% A second item to add to a command sequence.
%    \begin{macrocode}
\newcommand*{\si@addtocsname}[2]{%
  \@ifundefined{#1}
    {\expandafter\gdef\csname #1\endcsname{#2}}
    {\si@temptoks\expandafter\expandafter\expandafter{%
       \csname #1\endcsname#2}%
     \expandafter\xdef\csname #1\endcsname{\the\si@temptoks}}}
%    \end{macrocode}
%\end{macro}
%\begin{macro}{\si@ifmtarg}
%\begin{macro}{\si@xifmtarg}
%\begin{macro}{\si@ifnotmtarg}
% To keep down dependance on other packages, the very short code
% block from \pkg{ifmtarg} is copied here with an internal name.
%    \begin{macrocode}
\begingroup
  \catcode`\Q=3
  \long\gdef\si@ifmtarg#1{%
    \si@xifmtarg#1QQ\@secondoftwo\@firstoftwo\@nil}
  \long\gdef\si@xifmtarg#1#2Q#3#4#5\@nil{#4}
  \long\gdef\si@ifnotmtarg#1{%
    \si@xifmtarg#1QQ\@firstofone\@gobble\@nil}
\endgroup
%    \end{macrocode}
%\end{macro}
%\end{macro}
%\end{macro}
%\begin{macro}{\si@newrobustcmd}
%\begin{macro}{\si@newcommand}
%\begin{macro}{\si@newcmd}
%\begin{macro}{\si@xargdef}
%\darg{macro}
%\doarg{num-args}
%\doarg{default}
%\darg{def}
% Some more copying, this time from \pkg{etoolbox}.  Various macros
% need to be \emph{really} robust.  This is achieved using the \eTeX\
% \cs{protected} primitive in various places.  However, it would be
% nice to have a \cs{protected} version of \cs{newcommand}.
% \pkg{etoolbox} has code for that, but to avoid needing to load it,
% the necessary stuff is copied here.  The only changes form the
% original are names, and the use of \cs{newcommand*} for the
% \cs{si@newcommand} macro.
%    \begin{macrocode}
\@ifpackageloaded{etoolbox}
  {\let\si@newrobustcmd\newrobustcmd}
  {\protected\def\si@newrobustcmd{%
     \@ifstar
       {\let\l@ngrel@x\protected\si@newcommand}
       {\def\l@ngrel@x{\protected\long}\si@newcommand}}
   \newcommand*{\si@newcommand}[1]{\@testopt{\si@newcmd#1}0}
   \def\si@newcmd#1[#2]{%
     \@ifnextchar[%]
       {\si@xargdef#1[#2]}
       {\@argdef#1[#2]}}
   \long\def\si@xargdef#1[#2][#3]#4{%
     \@ifdefinable#1{%
       \expandafter\protected
       \expandafter\def
       \expandafter#1%
       \expandafter{%
         \expandafter\@testopt
         \csname\string#1\endcsname{#3}}%
       \expandafter\@yargdef
       \csname\string#1\endcsname\tw@{#2}{#4}}}}
%    \end{macrocode}
%\end{macro}
%\end{macro}
%\end{macro}
%\end{macro}
%
%\subsection{Logging}
%\begin{macro}{\ifsi@debug}
%\begin{macro}{\ifsi@logmin}
%\begin{macro}{\ifsi@lognone}
% To control logging, some new switches are declared.
%    \begin{macrocode}
\newif\ifsi@debug
\newif\ifsi@logmin
\newif\ifsi@lognone
%    \end{macrocode}
%\end{macro}
%\end{macro}
%\end{macro}
%\begin{macro}{\si@log@err}
%\darg{error}
%\darg{explanation}
% Some handy re-usable macros are defined here.  These all take names
% beginning  These pop up in various places.  First errors are
% handled.
%    \begin{macrocode}
\newcommand*{\si@log@err}[2]{%
  \ifsi@lognone\else
    \ifsi@logmin
      \PackageWarning{siunitx}{#1}%
    \else
      \PackageError{siunitx}{#1}{#2}%
    \fi
  \fi}
%    \end{macrocode}
%\end{macro}
%\begin{macro}{\si@log@warn}
%\begin{macro}{\si@log@inf}
%\darg{text}
% Then warnings and information.
%    \begin{macrocode}
\newcommand*{\si@log@warn}[1]{%
  \ifsi@lognone\else
    \ifsi@logmin\else
      \PackageWarning{siunitx}{#1}%
    \fi
  \fi}
\newcommand*{\si@log@inf}[1]{%
  \ifsi@lognone\else
    \ifsi@logmin\else
      \PackageInfo{siunitx}{#1}%
    \fi
  \fi}
%    \end{macrocode}
%\end{macro}
%\end{macro}
%\begin{macro}{\si@log@debug}
%\darg{debug-info}
% The debug macro only gives output if the appropriate package option
% is set.
%    \begin{macrocode}
\newcommand*{\si@log@debug}[1]{%
  \ifsi@lognone\else
    \ifsi@debug
      \PackageInfo{siunitx}{#1}%
    \fi
  \fi}
%    \end{macrocode}
%\end{macro}
%
%\subsection{String comparison}
%\begin{macro}{\si@str@ifchrstr}
%\begin{macro}{\si@str@chrstr}
%\darg{char}
%\darg{string}
% At various points, the package needs to compare two strings, to
% find if one occurs in the other.  The first test is if a single
% character is part of a second string; this is used, for example, to
% check that a character is valid as input.  The first argument is
% not expanded further, but the second is two allow division into
% individual units.
%    \begin{macrocode}
\newcommand*{\si@str@ifchrstr}[2]{%
  \begingroup
    \si@switchfalse
    \renewcommand*{\si@tempa}{#1}%
    \protected@edef\si@tempb{#2}%
    \expandafter\si@str@chrstr\si@tempb\@empty\@empty\@empty
    \ifsi@switch
      \aftergroup\@firstoftwo
    \else
      \aftergroup\@secondoftwo
    \fi
  \endgroup}
\def\si@str@chrstr#1#2\@empty{%
  \renewcommand*{\si@tempc}{#1}%
  \ifx\si@tempa\si@tempc
    \expandafter\si@switchtrue
  \else
    \ifx\@empty#2\@empty\else
      \si@str@chrstr#2\@empty\@empty
    \fi
  \fi}
%    \end{macrocode}
%\end{macro}
%\end{macro}
%\begin{macro}{\si@str@ifonlychrs}
%\begin{macro}{\si@str@onlychrs}
%\darg{string}
%\darg{chars}
% The second test builds on the first.  Here, a check is made to see
% if the first string is made up only of characters from the second
% string.  In this case, the first string is expanded before
% testing.  The second string will be expanded by the internal
% character by character test.
%    \begin{macrocode}
\newcommand*{\si@str@ifonlychrs}[2]{%
  \begingroup
    \si@switchtrue
    \protected@edef\si@tempa{#1}%
    \renewcommand*{\si@tempb}{#2}%
    \expandafter\si@str@onlychrs\si@tempa\@empty\@empty\@empty
    \ifsi@switch
      \aftergroup\@firstoftwo
    \else
      \aftergroup\@secondoftwo
    \fi
  \endgroup}
\def\si@str@onlychrs#1#2\@empty{%
  \si@str@ifchrstr{#1}{\si@tempb}
    {}{\si@switchfalse}%
  \ifx\@empty#2\@empty\else
    \si@str@onlychrs#2\@empty\@empty
  \fi}
%    \end{macrocode}
%\end{macro}
%\end{macro}
%
%\subsection{Option handling}
%\begin{macro}{\sisetup}
%\darg{options}
% To allow modification of options at run time, a setup macro is
% provided.
%    \begin{macrocode}
\newcommand*{\sisetup}{\setkeys[si]{key}}
%    \end{macrocode}
%\end{macro}
%\begin{macro}{\si@opt@key}
%\darg{keyname}
%\darg{code}
% To aid maintenance, some shortcuts are defined for generating keys.
% These also allow the debugging messages to be added automatically
% to every key.  First of all the basic key definition.
%    \begin{macrocode}
\newcommand*{\si@opt@key}[2]{%
  \define@key[si]{key}{#1}
  {#2\si@log@debug{Option #1 set to ##1}}}
%    \end{macrocode}
%\end{macro}
%\begin{macro}{\si@opt@cmdkey}
%\doarg{default}
%\darg{keyname}
%\darg{function}
% For a single command key, a function must be specified.
%    \begin{macrocode}
\newcommand*{\si@opt@cmdkey}[3][]{%
  \define@cmdkey[si]{key}[si@]{#2}[#1]{#3}}
%    \end{macrocode}
%\end{macro}
%\begin{macro}{\si@opt@cmdkeys}
%\doarg{default}
%\darg{keyname}
% Whereas multiple definitions do not take a function.
%    \begin{macrocode}
\newcommand*{\si@opt@cmdkeys}[2][]{%
  \define@cmdkeys[si]{key}[si@]{#2}[#1]}
%    \end{macrocode}
%\end{macro}
%\begin{macro}{\si@opt@boolkey}
%\doarg{function}
%\darg{keyname}
% Keys which only take switch values; anything other than \opt{true}
% or \opt{false} will generate a warning from \pkg{xkeyval}.
%    \begin{macrocode}
\newcommand*{\si@opt@boolkey}[2][]{%
  \define@boolkey[si]{key}[si@]{#2}[true]
  {#1\si@log@debug{Option #2 set to ##1}}}
%    \end{macrocode}
%\end{macro}
%\begin{macro}{\si@opt@choicekey}
%\doarg{default}
%\darg{keyname}
%\darg{choices}
%\darg{function}
% A ``fill in the blanks'' choice key.  In all cases, \cs{si@tempa}
% is used to hold the value given to the key, so that \cs{ifx}
% testing can occur.
%    \begin{macrocode}
\newcommand*{\si@opt@choicekey}[4][]{%
  \define@choicekey*+[si]{key}{#2}[\si@tempa]{#3}[#1]
    {#4\si@log@debug{Option #2 set to ##1}}
    {\si@log@warn{Unknown value `##1' for option #2}}}
%    \end{macrocode}
%\end{macro}
%\begin{macro}{\si@opt@xchoicekey}
%\darg{keyname}
%\darg{choices}
%\darg{initial-value}
% Several of the package options can take either a choice from a list
% of known options, or a value to be interpreted literally.  To aid
% maintenance, the necessary code can be set up here.  These keys all
% define a new macro, which must exist.  The \cs{si@opt@xchoicekey}
% macro therefore ensures that this is defined, as well as setting up
% the \pkg{xkeyval} key.
%    \begin{macrocode}
\newcommand*{\si@opt@xchoicekey}[3]{%
  \define@choicekey*+[si]{key}{#1}[\si@tempa]{#2}[#1]
%    \end{macrocode}
% This code will execute if the option is on the list.  There will be
% a ``fixed'' macro with a matching name, which is used for this.
%    \begin{macrocode}
    {\si@log@debug{Option #1 set to ##1}%
     \expandafter\renewcommand\expandafter*\expandafter{%
      \csname si@#1\endcsname}{\@nameuse{si@fix@##1}}}
%    \end{macrocode}
% The user has given something that is not on the list as an
% argument.  It is used literally.
%    \begin{macrocode}
    {\si@log@debug{Option #1 set to ##1}%
     \expandafter\renewcommand\expandafter*\expandafter{%
      \csname si@#1\endcsname}{##1}}
%    \end{macrocode}
% Finally, the initial value of the macro is set up.
%    \begin{macrocode}
  \expandafter\newcommand\expandafter*\expandafter{%
    \csname si@#1\endcsname}%
    {\@nameuse{si@fix@#3}}}
%    \end{macrocode}
%\end{macro}
%\begin{macro}{\si@opt@compatkey}
%\darg{package}
%\darg{keyname}
% An all-in-one definition for a back-compatibility key.  These
% should only be used at load time, so are automatically disabled
% once the package is loaded.  Emulation is also automatically turned
% on.
%    \begin{macrocode}
\newcommand*{\si@opt@compatkey}[2]{%
  \define@boolkey[si]{key}[si@old@]{#2}[true]
  {\si@log@debug{Emulating #1 package option\MessageBreak #2}%
   \sisetup{emulate=#1}%
   \si@log@debug{Option #2 set to ##1}}
   \AtEndOfPackage{\si@opt@disablekey{#2}
     {Compatibility option #2 only\MessageBreak
        available when loading siunitx package}}}
%    \end{macrocode}
%\end{macro}
%\begin{macro}{\si@opt@disablekey}
%\darg{keyname}
%\darg{text}
% The ability to disable a key with a meaningful message is a must;
% the warning will come from \currpkg, and not from \pkg{xkeyval}.
%    \begin{macrocode}
\newcommand*{\si@opt@disablekey}[2]{%
  \key@ifundefined[si]{key}{#1}
    {}
    {\si@log@debug{Disabling key #1}%
     \si@opt@key{#1}{\si@log@warn{#2}}}}
%    \end{macrocode}
%\end{macro}
%
% The \pkg{xkeyval} package option for logging is declared. This is
% then processed to set the switches correctly.
%    \begin{macrocode}
\si@opt@choicekey[normal]{log}{debug,verbose,normal,errors,none}
%    \end{macrocode}
% A series of comparisons are made to assign the logging mode.  The
% \opt{normal} option is not tested, as executing the option sets the
% switches appropriately.
%    \begin{macrocode}
  {\si@debugfalse
   \si@logminfalse
   \si@lognonefalse
   \renewcommand*{\si@tempb}{none}%
   \ifx\si@tempa\si@tempb
     \si@lognonetrue
   \fi
   \renewcommand*{\si@tempb}{minimal}%
   \ifx\si@tempa\si@tempb
     \si@logmintrue
   \fi
   \renewcommand*{\si@tempb}{debug}%
   \ifx\si@tempa\si@tempb
     \si@debugtrue
   \fi
   \renewcommand*{\si@tempb}{verbose}%
   \ifx\si@tempa\si@tempb
     \si@debugtrue
   \fi}
%    \end{macrocode}
% A quick method to set \opt{log=debug}.
%    \begin{macrocode}
\si@opt@boolkey{debug}
%    \end{macrocode}
%\begin{macro}{\ifsi@strict}
% It would be useful to be able to disable some keys, when strict
% interpretation of the rules is desired.  This is a load-time
% option, and has to disable various options.
%    \begin{macrocode}
\si@opt@boolkey[%
  \ifsi@strict
    \sisetup{
      obeymode=false,
      obeybold=false,
      obeyitalic=false,
      mode=maths,
      repeatunits=true,
      trapambigerr=true,
      trapambigfrac=true}
    \@for\si@tempa:=obeyall,obeymode,obeyitalic,mode,unitmode,%
      valuemode,textmode,obeybold,repeatunits,trapambigerr,%
      trapambigfrac\do{%
      \begingroup
        \edef\si@tempb{\endgroup
          \noexpand\si@opt@disablekey{\si@tempa}
          {Option `\si@tempa' forbidden in strict mode}}%
      \si@tempb}
  \fi]{strict}
\AtEndOfPackage{
  \si@opt@disablekey{strict}
    {Option `strict' only available when\MessageBreak
     loading package}}
%    \end{macrocode}
%\end{macro}
%\begin{macro}{\si@emulate}
% The \opt{emulate} option is used for back-compatibility mode; the
% option is only valid when loading \currpkg.
%    \begin{macrocode}
\newcommand*{\si@emulate}{}
\si@opt@key{emulate}{\si@addtolist{\si@emulate}{#1}}
\AtEndOfPackage{
  \si@opt@disablekey{emulate}
    {Option `emulate' only available when\MessageBreak
     loading package}}
%    \end{macrocode}
%\end{macro}
%\begin{macro}{\si@unitsep}
%\begin{macro}{\si@unitspace}
%\begin{macro}{\si@errspace}
%\begin{macro}{\si@valuesep}
% The two \opt{\dots sep} options control the size of spaces between
% the number and the unit (\cs{si@valuesep}), and that used to
% represent a product (\cs{si@unitsep}).  Known values here are
% \opt{thin}, \opt{med}, \opt{medium}, \opt{thick}, \opt{cdot},
% \opt{tightcdot}\footnote{Both \cs{cdot}-based options only valid
% for \opt{unitsep}.} and \opt{none};\footnote{Only valid for
% \opt{valuesep}.} other entries will be treated as custom spaces.
%    \begin{macrocode}
\si@opt@xchoicekey{unitsep}
  {thin,med,medium,thick,none,comma,stop,fullstop,period,
    times,tighttimes,cdot,tightcdot}{thin}
\si@opt@xchoicekey{unitspace}{space,thin,med,medium,thick,
  none}{thin}
\si@opt@xchoicekey{errspace}{space,thin,med,medium,thick,
  none}{none}
\si@opt@xchoicekey{valuesep}
  {thin,med,medium,thick,none,comma,stop,fullstop,period,
    times,tighttimes,cdot,tightcdot}{thin}
%    \end{macrocode}
%\end{macro}
%\end{macro}
%\end{macro}
%\end{macro}
%\begin{macro}{\si@digitsep}
% Separation of digits in large numbers is controlled by the
% \opt{digitsep} option.  As with the other \opt{sep} values, this
% one has a choice of possible values.  The list is quite long, so
% that a range of options are handled automatically.  Notice that
% \opt{digitsep=none} will be used for no separation at all.
%    \begin{macrocode}
\si@opt@xchoicekey{digitsep}
  {thin,med,medium,thick,none,comma,stop,fullstop,period,
    times,tighttimes,cdot,tightcdot}{thin}
%    \end{macrocode}
%\end{macro}
%\begin{macro}{\si@decimalsymbol}
% The symbol used for the decimal position is varied here.  There are
% only two real options, but options are given for the name of a full
% stop.
%    \begin{macrocode}
\si@opt@xchoicekey{decimalsymbol}{comma,stop,fullstop,period,
  cdot,tightcdot}{fullstop}
%    \end{macrocode}
%\end{macro}
%\begin{macro}{\si@anglesep}
% The separator between degrees and minutes, and between minutes and
% seconds, when using \cs{ang}.
%    \begin{macrocode}
\si@opt@xchoicekey{anglesep}
    {thin,med,medium,thick,none,comma,stop,fullstop,period,
      times,tighttimes,cdot,tightcdot}{none}
%    \end{macrocode}
%\end{macro}
%\begin{macro}{\ifsi@obeymode}
% The first test for the font control is whether to respect the
% surrounding maths or text mode.
%    \begin{macrocode}
\si@opt@boolkey{obeymode}
%    \end{macrocode}
%\end{macro}
%\begin{macro}{\ifsi@numtextmode}
%\begin{macro}{\ifsi@unittextmode}
% The output of the package can be typeset using either text or maths
% mode fonts. Two switches are needed, for numbers and units.
%    \begin{macrocode}
\newif\ifsi@numtextmode
\newif\ifsi@unittextmode
%    \end{macrocode}
%\end{macro}
%\end{macro}
% The \opt{textmode} option has to set both flags.
%    \begin{macrocode}
\si@opt@choicekey[true]{textmode}{true,false}
  {\si@numtextmodefalse
   \si@unittextmodefalse
   \renewcommand*{\si@tempb}{true}%
   \ifx\si@tempa\si@tempb
     \si@numtextmodetrue
     \si@unittextmodetrue
   \fi}
%    \end{macrocode}
% The \opt{mode} option applies to numbers and units.
%    \begin{macrocode}
\si@opt@choicekey{mode}{math,maths,text}
  {\si@numtextmodefalse
   \si@unittextmodefalse
   \renewcommand*{\si@tempb}{text}%
   \ifx\si@tempa\si@tempb
     \si@numtextmodetrue
     \si@unittextmodetrue
   \fi}
%    \end{macrocode}
% Now the two options for numbers or units alone.
%    \begin{macrocode}
\si@opt@choicekey{valuemode}{math,maths,text}
  {\si@numtextmodefalse
   \renewcommand*{\si@tempb}{text}%
   \ifx\si@tempa\si@tempb
     \si@numtextmodetrue
   \fi}
\si@opt@choicekey{unitmode}{math,maths,text}
  {\si@unittextmodefalse
   \renewcommand*{\si@tempb}{text}%
   \ifx\si@tempa\si@tempb
     \si@unittextmodetrue
   \fi}
%    \end{macrocode}
%\begin{macro}{\ifsi@obeyfamily}
% The package can work to match the font family (serif, sans serif,
% typewriter) of the surrounding text.  This is controlled by a
% Boolean option.
%    \begin{macrocode}
\si@opt@boolkey{obeyfamily}
%    \end{macrocode}
%\end{macro}
%\begin{macro}{\ifsi@obeybold}
% The package can attempt to respect bold, or may ignore it.
%    \begin{macrocode}
\si@opt@boolkey{obeybold}
%    \end{macrocode}
%\end{macro}
%\begin{macro}{\ifsi@inlinebtext}
% For inline maths, two options for checking what is bold are
% available, the maths environment (\ie \cs{boldmath}) and the
% surrounding text (\cs{textbf} or \cs{bffamily}).
%    \begin{macrocode}
\newif\ifsi@inlinebtext
\si@opt@choicekey{inlinebold}{text,maths,math}
  {\si@inlinebtextfalse
   \renewcommand*{\si@tempb}{text}%
   \ifx\si@tempa\si@tempb
     \si@inlinebtexttrue
   \fi}
%    \end{macrocode}
%\end{macro}
%\begin{macro}{\ifsi@obeyitalic}
% Italic is slightly different to bold, as there is no convenient
% switch for maths.  Thus a choice key is used, with the appropriate
% check.
%    \begin{macrocode}
\si@opt@boolkey{obeyitalic}
%    \end{macrocode}
%\end{macro}
%\begin{macro}{\ifsi@detectdisplay}
% For handling display mathematics, a setting is needed for whether to
% treat it differently from other maths.
%    \begin{macrocode}
\si@opt@boolkey{detectdisplay}
%    \end{macrocode}
%\end{macro}
% The option to obey all font switching commands is Boolean-like but
% needs alternative setup.
%    \begin{macrocode}
\si@opt@choicekey[true]{obeyall}{true,false}
  {\si@obeyboldfalse
   \si@obeyitalicfalse
   \si@obeymodefalse
   \si@obeyfamilyfalse
   \renewcommand*{\si@tempb}{true}%
   \ifx\si@tempa\si@tempb
     \si@obeyboldtrue
     \si@obeyitalictrue
     \si@obeymodetrue
     \si@obeyfamilytrue
   \fi}
%    \end{macrocode}
%\begin{macro}{\si@valuemathsrm}
%\begin{macro}{\si@valuemathssf}
%\begin{macro}{\si@valuemathstt}
%\begin{macro}{\si@unitmathsrm}
%\begin{macro}{\si@unitmathssf}
%\begin{macro}{\si@unitmathstt}
% The fonts used by the package default to the obvious \LaTeX\ ones;
% however, this needs to be exposed to user modification.  First the
% maths mode fonts are sorted out.
%    \begin{macrocode}
\si@opt@cmdkeys{valuemathsrm,valuemathssf,valuemathstt}
\si@opt@cmdkeys{unitmathsrm,unitmathssf,unitmathstt}
%    \end{macrocode}
%\end{macro}
%\end{macro}
%\end{macro}
%\end{macro}
%\end{macro}
%\end{macro}
% To make life easier for the user, US spellings are provided for the
% maths keys.
%    \begin{macrocode}
\si@opt@key{valuemathrm}{\sisetup{valuemathsrm=#1}}
\si@opt@key{valuemathsf}{\sisetup{valuemathssf=#1}}
\si@opt@key{valuemathtt}{\sisetup{valuemathstt=#1}}
\si@opt@key{unitmathrm}{\sisetup{unitmathsrm=#1}}
\si@opt@key{unitmathsf}{\sisetup{unitmathssf=#1}}
\si@opt@key{unitmathtt}{\sisetup{unitmathstt=#1}}
%    \end{macrocode}
% The combined options are given, for setting numbers and units
% at the same time.
%    \begin{macrocode}
\si@opt@key{mathsrm}{\sisetup{valuemathsrm=#1,unitmathsrm=#1}}
\si@opt@key{mathssf}{\sisetup{valuemathssf=#1,unitmathssf=#1}}
\si@opt@key{mathstt}{\sisetup{valuemathstt=#1,unitmathstt=#1}}
\si@opt@key{mathrm}{\sisetup{valuemathsrm=#1,unitmathsrm=#1}}
\si@opt@key{mathsf}{\sisetup{valuemathssf=#1,unitmathssf=#1}}
\si@opt@key{mathtt}{\sisetup{valuemathstt=#1,unitmathstt=#1}}
%    \end{macrocode}
%\begin{macro}{\si@valuetextrm}
%\begin{macro}{\si@valuetextsf}
%\begin{macro}{\si@valuetexttt}
%\begin{macro}{\si@unittextrm}
%\begin{macro}{\si@unittextsf}
%\begin{macro}{\si@unittexttt}
% The same thing for text mode fonts.  Once again the default values
% are pretty obvious.
%    \begin{macrocode}
\si@opt@cmdkeys{valuetextrm,valuetextsf,valuetexttt}
\si@opt@cmdkeys{unittextrm,unittextsf,unittexttt}
\si@opt@key{textrm}{\sisetup{unittextrm=#1,valuetextrm=#1}}
\si@opt@key{textsf}{\sisetup{unittextsf=#1,valuetextsf=#1}}
\si@opt@key{texttt}{\sisetup{unittexttt=#1,valuetexttt=#1}}
%    \end{macrocode}
%\end{macro}
%\end{macro}
%\end{macro}
%\end{macro}
%\end{macro}
%\end{macro}
%\changes{v1.1}{2008/09/22}{Added \opt{numdiv} option}
%\begin{macro}{\si@numdigits}
%\begin{macro}{\si@numdiv}
%\begin{macro}{\si@numdecimal}
%\begin{macro}{\si@numexp}
%\begin{macro}{\si@numprod}
%\begin{macro}{\si@numgobble}
%\begin{macro}{\si@numsign}
%\begin{macro}{\si@numcloseerr}
%\begin{macro}{\si@numopenerr}
%\begin{macro}{\si@numaddn}
% The list of possible valid characters for parsing numbers is set
% up.  This is similar to \numprint, but with the \opt{extra} class,
% and with characters ignored with no output renamed as \opt{gobble}.
%    \begin{macrocode}
\si@opt@cmdkeys{numdigits,numdecimal,numexp,numgobble,numsign,%
  numcloseerr,numopenerr,numaddn,numprod,numdiv}
%    \end{macrocode}
%\end{macro}
%\end{macro}
%\end{macro}
%\end{macro}
%\end{macro}
%\end{macro}
%\end{macro}
%\end{macro}
%\end{macro}
%\end{macro}
%\begin{macro}{\si@numvalid}
%\begin{macro}{\si@numextra}
% The various valid characters are collected together in a single
% macro for later.  In common with the above macros, this one starts
% \cs{si@num\ldots}.  The order here is the order the values are
% tested later on.
%    \begin{macrocode}
\newcommand*{\si@numextra}{%
  \si@numopenerr\si@numcloseerr\si@numaddn}
\newcommand*{\si@numvalid}{\si@numgobble\si@numexp\si@numsign
  \si@numdecimal\si@numdigits\si@numextra\si@numprod\si@numdiv}
%    \end{macrocode}
%\end{macro}
%\end{macro}
%\begin{macro}{\ifsi@seperr}
%\begin{macro}{\ifsi@trapambigerr}
%\begin{macro}{\si@openerr}
%\begin{macro}{\si@closeerr}
% An option to control whether numerical error values are printed with
% or separate from the number.
%    \begin{macrocode}
\si@opt@boolkey{seperr}
\si@opt@boolkey{trapambigerr}
\si@opt@cmdkeys{openerr,closeerr}
%    \end{macrocode}
%\end{macro}
%\end{macro}
%\end{macro}
%\end{macro}
%\begin{macro}{\ifsi@sepfour}
% With four digits in a number, separating may or may not be desired.
% Note that this option is the same as one for \numprint.
%    \begin{macrocode}
\si@opt@boolkey{sepfour}
%    \end{macrocode}
%\end{macro}
%\begin{macro}{\ifsi@retainplus}
% An option to keep an explicit positive sign.
%    \begin{macrocode}
\si@opt@boolkey{retainplus}
%    \end{macrocode}
%\end{macro}
%\begin{macro}{\si@expbase}
%\begin{macro}{\si@expproduct}
% The options for exponents are set up.
%    \begin{macrocode}
\si@opt@xchoicekey{expproduct}{times,tighttimes,
  cdot,tightcdot}{times}
\si@opt@xchoicekey{expbase}{ten}{ten}
%    \end{macrocode}
%\end{macro}
%\end{macro}
%\begin{macro}{\ifsi@allowzeroexp}
% The package normally prevents \num[allowzeroexp]{e0}.
%    \begin{macrocode}
\si@opt@boolkey{allowzeroexp}
%    \end{macrocode}
%\end{macro}
%\begin{macro}{\si@prefixproduct}
% The marker for multiplication in prefixes.
%    \begin{macrocode}
\si@opt@xchoicekey{prefixproduct}{times,tighttimes,cdot,
  tightcdot,none}{times}
%    \end{macrocode}
%\end{macro}
%\begin{macro}{\si@prefixbase}
% In the same area, the power for prefixes is variable.  Here, two
% choices ar needed.
%    \begin{macrocode}
\si@opt@xchoicekey{prefixbase}{ten,two}{ten}
%    \end{macrocode}
%\end{macro}
%\begin{macro}{\ifsi@prefixsymbolic}
% Unit prefixes can be given as either symbols or numerically.
%    \begin{macrocode}
\si@opt@boolkey{prefixsymbolic}
%    \end{macrocode}
%\end{macro}
%\begin{macro}{\ifsi@num@padlead}
%\begin{macro}{\ifsi@num@padtrail}
% A setting is needed to indicate when to add zeros to decimal
% numbers, either before the decimal marker (|.1| giving
% ``\num[padnumber=all]{.1}'') or after (|1.| giving
% ``\num[padnumber=all]{1.}'').
%    \begin{macrocode}
\newif\ifsi@num@padlead
\newif\ifsi@num@padtrail
\si@opt@choicekey[all]{padnumber}
  {leading,lead,trailing,trail,all,both,true,none,false}
  {\si@num@padleadfalse
   \si@num@padtrailfalse
   \renewcommand*{\si@tempb}{leading}%
   \ifx\si@tempa\si@tempb
     \si@num@padleadtrue
   \fi
   \renewcommand*{\si@tempb}{lead}%
   \ifx\si@tempa\si@tempb
     \si@num@padleadtrue
   \fi
   \renewcommand*{\si@tempb}{trailing}%
   \ifx\si@tempa\si@tempb
     \si@num@padtrailtrue
   \fi
   \renewcommand*{\si@tempb}{trail}%
   \ifx\si@tempa\si@tempb
     \si@num@padtrailtrue
   \fi
   \renewcommand*{\si@tempb}{all}%
   \ifx\si@tempa\si@tempb
     \si@num@padleadtrue
     \si@num@padtrailtrue
   \fi
   \renewcommand*{\si@tempb}{true}%
   \ifx\si@tempa\si@tempb
     \si@num@padleadtrue
     \si@num@padtrailtrue
   \fi
   \renewcommand*{\si@tempb}{both}%
   \ifx\si@tempa\si@tempb
     \si@num@padleadtrue
     \si@num@padtrailtrue
   \fi}
%    \end{macrocode}
%\end{macro}
%\end{macro}
%\begin{macro}{\si@sign}
%\begin{macro}{\ifsi@num@signmant}
%\begin{macro}{\ifsi@num@signexp}
% Some new switches for adding signs to numbers
%    \begin{macrocode}
\newif\ifsi@num@signmant
\newif\ifsi@num@signexp
%    \end{macrocode}
%\end{macro}
%\end{macro}
% Signs can be added to numbers by default. Two options are needed
% here; whether to add a sign by default, and what the sign is.
%    \begin{macrocode}
\si@opt@xchoicekey{sign}{plus,minus,pm,mp}{plus}
\si@opt@choicekey[all]{addsign}
  {mantissa,exponent,mant,exp,all,both,true,none,false}
%    \end{macrocode}
%\end{macro}
% The option is now processed.
%    \begin{macrocode}
  {\si@num@signmantfalse
   \si@num@signexpfalse
   \renewcommand*{\si@tempb}{mantissa}%
   \ifx\si@tempa\si@tempb
     \si@num@signmanttrue
   \fi
   \renewcommand*{\si@tempb}{mant}%
   \ifx\si@tempa\si@tempb
     \si@num@signmanttrue
   \fi
   \renewcommand*{\si@tempb}{exponent}%
   \ifx\si@tempa\si@tempb
     \si@num@signexptrue
   \fi
   \renewcommand*{\si@tempb}{exp}%
   \ifx\si@tempa\si@tempb
     \si@num@signexptrue
   \fi
   \renewcommand*{\si@tempb}{all}%
   \ifx\si@tempa\si@tempb
     \si@num@signmanttrue
     \si@num@signexptrue
   \fi
   \renewcommand*{\si@tempb}{true}%
   \ifx\si@tempa\si@tempb
     \si@num@signmanttrue
     \si@num@signexptrue
   \fi
   \renewcommand*{\si@tempb}{both}%
   \ifx\si@tempa\si@tempb
     \si@num@signmanttrue
     \si@num@signexptrue
   \fi}
%    \end{macrocode}
%\begin{macro}{\ifsi@tightpm}
%\begin{macro}{\si@pm}
% To reduce spacing, it might be necessary to use a ``tight'' $\pm$
% sign.
%    \begin{macrocode}
\si@opt@boolkey{tightpm}
\newcommand*{\si@pm}{%
  \ifsi@tightpm
    \si@fix@tightpm
  \else
    \si@fix@pm
  \fi}
%    \end{macrocode}
%\end{macro}
%\end{macro}
%\begin{macro}{\ifsi@ang@padsmall}
%\begin{macro}{\ifsi@ang@padlarge}
% A switch for determining whether to typeset |\ang{;;1}| as
% \ang[padangle=both]{;;1} or \ang[padangle=none]{;;1}.  First, two
% new Boolean switches are needed to indicate padding.
%    \begin{macrocode}
\newif\ifsi@ang@padsmall
\newif\ifsi@ang@padlarge
\si@opt@choicekey[all]{padangle}
  {small,large,all,both,true,none,false}
  {\si@ang@padsmallfalse
   \si@ang@padlargefalse
   \renewcommand*{\si@tempb}{small}%
   \ifx\si@tempa\si@tempb
     \si@ang@padsmalltrue
   \fi
   \renewcommand*{\si@tempb}{large}%
   \ifx\si@tempa\si@tempb
     \si@ang@padlargetrue
   \fi
   \renewcommand*{\si@tempb}{all}%
   \ifx\si@tempa\si@tempb
     \si@ang@padsmalltrue
     \si@ang@padlargetrue
   \fi
   \renewcommand*{\si@tempb}{true}%
   \ifx\si@tempa\si@tempb
     \si@ang@padsmalltrue
     \si@ang@padlargetrue
   \fi
   \renewcommand*{\si@tempb}{both}%
   \ifx\si@tempa\si@tempb
     \si@ang@padsmalltrue
     \si@ang@padlargetrue
   \fi}
%    \end{macrocode}
%\end{macro}
%\end{macro}
%\begin{macro}{\ifsi@ang@toarc}
%\begin{macro}{\ifsi@ang@todec}
% To control whether angles are formatted as decimals or
% degrees--minutes--seconds, a package option plus two switches are
% needed.  The later format is referred to as ``\opt{arc}'' most
% readily.  An option to leave the input unchanged is also provided.
%    \begin{macrocode}
\newif\ifsi@ang@toarc
\newif\ifsi@ang@todec
\si@opt@choicekey[all]{angformat}
  {dec,decimal,arc,dms,unchanged,none}
  {\si@ang@toarcfalse
   \si@ang@todecfalse
   \renewcommand*{\si@tempb}{dec}%
   \ifx\si@tempa\si@tempb
     \si@ang@todectrue
   \fi
   \renewcommand*{\si@tempb}{decimal}%
   \ifx\si@tempa\si@tempb
     \si@ang@todectrue
   \fi
   \renewcommand*{\si@tempb}{arc}%
   \ifx\si@tempa\si@tempb
     \si@ang@toarctrue
   \fi
   \renewcommand*{\si@tempb}{dms}%
   \ifx\si@tempa\si@tempb
     \si@ang@toarctrue
   \fi}
%    \end{macrocode}
%\end{macro}
%\end{macro}
%\begin{macro}{\ifsi@astroang}
% A slightly odd option to allow the method used by astronomers for
% angles.
%    \begin{macrocode}
\si@opt@boolkey{astroang}
%    \end{macrocode}
%\end{macro}
%\begin{macro}{\ifsi@strictarc}
% For the \cs{ang} macro, the default is to require two semi-colons
% in the input for arc angles.  This is controlled here.
%    \begin{macrocode}
\si@opt@boolkey{strictarc}
%    \end{macrocode}
%\end{macro}
%\changes{v1.1}{2008/09/03}{Added \opt{tabparseonly} option}
%\begin{macro}{\ifsi@tab@fixed}
%\begin{macro}{\ifsi@tabparseonly}
%\begin{macro}{\si@tabnumalign}
%\begin{macro}{\si@tab@rfill@S}
%\begin{macro}{\si@tab@lfill@S}
% To determine the control of table alignment, two options are
% provided.  The \opt{tabnumalign} option controls which centring method
% is used, and the fills used for achieving this.
%    \begin{macrocode}
\newif\ifsi@tab@fixed
\si@opt@boolkey{tabparseonly}
\si@opt@choicekey{tabnumalign}
  {centredecimal,centerdecimal,right,left,centre,center}
  {\si@tab@fixedtrue
   \def\si@tab@rfill@S{0.5}%
   \def\si@tab@lfill@S{0.5}%
   \renewcommand*{\si@tempb}{right}%
   \ifx\si@tempa\si@tempb
     \def\si@tab@rfill@S{0}%
     \def\si@tab@lfill@S{1}%
   \fi
   \renewcommand*{\si@tempb}{left}%
   \ifx\si@tempa\si@tempb
     \def\si@tab@rfill@S{1}%
     \def\si@tab@lfill@S{0}%
   \fi
   \renewcommand*{\si@tempb}{centredecimal}%
   \ifx\si@tempa\si@tempb
     \expandafter\si@tab@fixedfalse
   \fi
   \renewcommand*{\si@tempb}{centerdecimal}%
   \ifx\si@tempa\si@tempb
     \expandafter\si@tab@fixedfalse
   \fi}
\si@opt@key{tabalign}{\sisetup{tabnumalign=#1,tabtextalign=#1,
  tabunitalign=#1}}
%    \end{macrocode}
%\end{macro}
%\end{macro}
%\end{macro}
%\end{macro}
%\end{macro}
%\begin{macro}{\ifsi@tabalignexp}
% A switch for aligning exponents.
%    \begin{macrocode}
\si@opt@boolkey{tabalignexp}
\si@opt@boolkey[%
  \si@tabalignexpfalse
  \ifsi@tabexpalign
    \expandafter\si@tabalignexptrue
  \fi
  ]{tabexpalign}
%    \end{macrocode}
%\end{macro}
%\begin{macro}{\si@tab@mantprecnt}
%\begin{macro}{\si@tab@mantpostcnt}
%\begin{macro}{\si@tab@expprecnt}
%\begin{macro}{\si@tab@exppostcnt}
%\begin{macro}{\ifsi@tab@mantsign}
%\begin{macro}{\ifsi@tab@expsign}
% To process the format information, various internal
% number-processing macros are used.  First, some
% storage areas are created.
%    \begin{macrocode}
\newcount\si@tab@mantprecnt
\newcount\si@tab@mantpostcnt
\newcount\si@tab@expprecnt
\newcount\si@tab@exppostcnt
\newif\ifsi@tab@mantsign
\newif\ifsi@tab@expsign
%    \end{macrocode}
%\end{macro}
%\end{macro}
%\end{macro}
%\end{macro}
%\end{macro}
%\end{macro}
% The input is split into a mantissa and exponent, then passed to a
% re-useable macro for further processing.
%    \begin{macrocode}
\si@opt@cmdkey{tabformat}
  {\si@num@fixpm
   \renewcommand*{\si@num@arg}{tabformat data}%
   \renewcommand*{\si@num@exp}{}%
   \renewcommand*{\si@num@mant}{}%
   \si@tab@mantsignfalse
   \si@tab@expsignfalse
   \si@switchfalse
   \si@num@sepmantexp{#1}%
%    \end{macrocode}
% When checking for a sign, the internal flag for finding but
% deleting a plus sign is used.
%    \begin{macrocode}
   \si@num@sepsign{mant}%
   \ifx\@empty\si@num@mantsign\@empty
     \ifsi@num@delplus
       \expandafter\expandafter\expandafter\si@tab@mantsigntrue
     \fi
   \else
     \expandafter\si@tab@mantsigntrue
   \fi
   \si@num@sepsign{exp}%
   \ifx\@empty\si@num@expsign\@empty
     \ifsi@num@delplus
       \expandafter\expandafter\expandafter\si@tab@expsigntrue
     \fi
   \else
     \expandafter\si@tab@expsigntrue
   \fi
   \si@opt@proctform{mant}%
   \si@opt@proctform{exp}%
%    \end{macrocode}
%\changes{v1.0h}{2008/08/02}{Simplified warnings for \opt{tabformat}}
% If both the integer and decimal parts are empty, then something has
% probably gone wrong.
%    \begin{macrocode}
   \ifnum\si@tab@mantpostcnt=\z@\relax
     \ifnum\si@tab@mantprecnt=\z@\relax
       \si@log@err{Empty mantissa argument for tabformat}
         {The argument `#1' contains no valid entry for
          a mantissa\MessageBreak It should be of the
          form `m.n', where  m and n are integers}%
     \fi
   \fi
%    \end{macrocode}
% If \opt{tabformat} has been given with
% \opt{tabnumalign=centredecimal} active, then the alignment is
% changed to centred.
%    \begin{macrocode}
   \ifsi@tab@fixed\else
     \sisetup{tabnumalign=centre}%
   \fi
   \let\pm\si@num@pm
   \let\mp\si@num@mp}
%    \end{macrocode}
%\begin{macro}{\si@opt@proctform}
%\darg{either \texttt{mant} or \texttt{exp}}
% Processing the number further uses the \cs{si@num@digits} macro.
% The results are stored in the appropriate counter.
%    \begin{macrocode}
\newcommand*{\si@opt@proctform}[1]{%
  \renewcommand*{\si@num@predec}{}%
  \renewcommand*{\si@num@postdec}{}%
  \si@switchfalse
  \expandafter\si@ifnotmtarg\expandafter{%
    \csname si@num@#1\endcsname}
     {\expandafter\expandafter\expandafter\si@num@digits
        \csname si@num@#1\endcsname\@empty\@empty}%
   \csname si@tab@#1precnt\endcsname\z@\relax
   \csname si@tab@#1postcnt\endcsname\z@\relax
   \ifx\@empty\si@num@predec\@empty\else
     \csname si@tab@#1precnt\endcsname\si@num@predec\relax
   \fi
   \ifx\@empty\si@num@postdec\@empty\else
     \csname si@tab@#1postcnt\endcsname\si@num@postdec\relax
   \fi}
%    \end{macrocode}
%\end{macro}
% The alignment of tabular material when not processed by \cs{num}
% needs to be available.
%    \begin{macrocode}
\si@opt@choicekey{tabtextalign}{left,right,centre,center}
%    \end{macrocode}
%\begin{macro}{\si@tab@rfill@t}
%\begin{macro}{\si@tab@lfill@t}
% By default, centring happens on both sides of the content of
% tabular material.
%    \begin{macrocode}
  {\def\si@tab@rfill@t{0.5}%
   \def\si@tab@lfill@t{0.5}%
   \renewcommand*{\si@tempb}{right}%
   \ifx\si@tempa\si@tempb
     \def\si@tab@rfill@t{0}%
     \def\si@tab@lfill@t{1}%
   \fi
   \renewcommand*{\si@tempb}{left}%
   \ifx\si@tempa\si@tempb
     \def\si@tab@rfill@t{1}%
     \def\si@tab@lfill@t{0}%
   \fi}
%    \end{macrocode}
%\end{macro}
%\end{macro}
% The alignment of unit columns for tabular material has a similar
% control option.
%    \begin{macrocode}
\si@opt@choicekey{tabunitalign}{left,right,centre,center}
%    \end{macrocode}
%\begin{macro}{\si@tab@rfill@s}
%\begin{macro}{\si@tab@lfill@s}
% By default, centring happens on both sides of the content of
% tabular material.
%    \begin{macrocode}
  {\def\si@tab@rfill@s{0.5}%
   \def\si@tab@lfill@s{0.5}%
   \renewcommand*{\si@tempb}{right}%
   \ifx\si@tempa\si@tempb
     \def\si@tab@rfill@s{0}%
     \def\si@tab@lfill@s{1}%
   \fi
   \renewcommand*{\si@tempb}{left}%
   \ifx\si@tempa\si@tempb
     \def\si@tab@rfill@s{1}%
     \def\si@tab@lfill@s{0}%
   \fi}
%    \end{macrocode}
%\end{macro}
%\end{macro}
%\begin{macro}{\ifsi@fixdp}
%\changes{v1.1}{2008/08/22}{Modified to work with \opt{fixsf} option}
%\begin{macro}{\si@num@dp}
%\changes{v1.1}{2008/08/22}{Modified to work with \opt{fixsf} option}
% To allow control of rounding, two options are needed.  One sets how
% many fixed digits to use, the second turns this function on and
% off.
%    \begin{macrocode}
\si@opt@boolkey[%
  \ifsi@fixdp
    \expandafter\si@fixsffalse
  \fi]{fixdp}
\newcount\si@num@dp
\si@opt@key{dp}{%
  \si@str@ifonlychrs{#1}{0123456789}
    {\si@num@dp#1\relax
     \si@fixdptrue
     \si@fixsffalse}
    {\si@log@err{Invalid input for dp option}
      {The dp option must be given a positive integer}}}
%    \end{macrocode}
%\end{macro}
%\end{macro}
%\begin{macro}{\ifsi@fixsf}
%\begin{macro}{\si@num@sf}
% A similar approach for fixing significant figures.
%    \begin{macrocode}
\si@opt@boolkey[%
  \ifsi@fixsf
    \expandafter\si@fixdpfalse
  \fi]{fixsf}
\newcount\si@num@sf
\si@opt@key{sf}{%
  \si@str@ifonlychrs{#1}{0123456789}
    {\si@num@sf#1\relax
     \ifnum\si@num@sf=\z@\relax
       \si@log@warn{Setting sf=0 disables function}%
       \si@fixsffalse
     \else
       \si@fixsftrue
       \si@fixdpfalse
     \fi}
    {\si@log@err{Invalid input for sf option}
      {The sf option must be given a positive integer}}}
%    \end{macrocode}
%\end{macro}
%\end{macro}
%\begin{macro}{\ifsi@tabautofit}
% To apply rounding automatically in a table, a separate option is
% used.
%    \begin{macrocode}
\si@opt@boolkey{tabautofit}
%    \end{macrocode}
%\end{macro}
%\begin{macro}{\ifsi@xspace}
% Unit macros on their own may need \pkg{xpsace}.
%    \begin{macrocode}
\si@opt@boolkey{xspace}
%    \end{macrocode}
%\end{macro}
%\begin{macro}{\ifsi@prespace}
%    \begin{macrocode}
\si@opt@boolkey
  [\si@unt@numfalse
   \ifsi@prespace
     \si@unt@numtrue
  \fi]
  {prespace}
%    \end{macrocode}
%\end{macro}
%\begin{macro}{\ifsi@allowoptarg}
% For \unitsdef users, a method to absorb optional arguments is
% needed.
%    \begin{macrocode}
\si@opt@boolkey{allowoptarg}
%    \end{macrocode}
%\end{macro}
%\changes{v1.1d}{2008/10/29}{Minor bug with \opt{per} option fixed}
%\begin{macro}{\ifsi@frac}
%\begin{macro}{\ifsi@slash}
%\begin{macro}{\ifsi@stickyper}
% The option processing for formatting units with \cs{per} in them
% needs two switches.
%    \begin{macrocode}
\newif\ifsi@slash
\newif\ifsi@frac
\si@opt@boolkey{stickyper}
\si@opt@choicekey[reciprocal]{per}
  {reciprocal,rp,power,slash,frac,fraction}
  {\si@slashfalse
   \si@fracfalse
   \let\si@frac\si@frc@frac
   \renewcommand*{\si@tempb}{slash}%
   \ifx\si@tempa\si@tempb
     \si@fractrue
     \si@slashtrue
     \let\si@frac\si@frc@slash
   \fi
   \renewcommand*{\si@tempb}{frac}%
   \ifx\si@tempa\si@tempb
     \si@fractrue
   \fi
   \renewcommand*{\si@tempb}{fraction}%
   \ifx\si@tempa\si@tempb
     \si@fractrue
   \fi}
%    \end{macrocode}
%\end{macro}
%\end{macro}
%\end{macro}
%\begin{macro}{\si@slash}
% For the \opt{slash} option, the separator can be customised.
%    \begin{macrocode}
\si@opt@xchoicekey{slash}{slash}{slash}
%    \end{macrocode}
%\end{macro}
%\begin{macro}{\ifsi@repeatunits}
%\begin{macro}{\ifsi@addunitpower}
% An option is needed for cases where units should be repeated.
%    \begin{macrocode}
\newif\ifsi@repeatunits
\newif\ifsi@addunitpower
\si@opt@choicekey[true]{repeatunits}{true,false,power}
  {\si@repeatunitsfalse
   \si@addunitpowerfalse
   \renewcommand*{\si@tempb}{true}%
   \ifx\si@tempa\si@tempb
     \si@repeatunitstrue
   \fi
   \renewcommand*{\si@tempb}{power}%
   \ifx\si@tempa\si@tempb
     \si@addunitpowertrue
   \fi}
%    \end{macrocode}
%\end{macro}
%\end{macro}
%\begin{macro}{\ifsi@trapambigfrac}
%\begin{macro}{\si@closefrac}
%\begin{macro}{\si@openfrac}
% Macros for the right and left brackets added to potentially
% ambiguous denominators.
%    \begin{macrocode}
\si@opt@boolkey{trapambigfrac}
\si@opt@cmdkeys{closefrac,openfrac}
%    \end{macrocode}
%\end{macro}
%\end{macro}
%\end{macro}
% In the case of fractional handling of the \cs{per} operator,
% further refinement is available.
%    \begin{macrocode}
\si@opt@choicekey[frac]{fraction}
  {frac,nicefrac,nice,sfrac,xfrac,uglyfrac,ugly}
  {\let\si@frac\si@frc@frac
   \renewcommand*{\si@tempb}{nicefrac}%
   \ifx\si@tempa\si@tempb
     \let\si@frac\si@frc@nice
   \fi
   \renewcommand*{\si@tempb}{uglyfrac}%
   \ifx\si@tempa\si@tempb
     \let\si@frac\si@frc@ugly
   \fi
   \renewcommand*{\si@tempb}{nice}%
   \ifx\si@tempa\si@tempb
     \let\si@frac\si@frc@nice
   \fi
   \renewcommand*{\si@tempb}{sfrac}%
   \ifx\si@tempa\si@tempb
     \let\si@frac\si@frc@sfrac
   \fi
   \renewcommand*{\si@tempb}{xfrac}%
   \ifx\si@tempa\si@tempb
     \let\si@frac\si@frc@sfrac
   \fi
   \renewcommand*{\si@tempb}{ugly}%
   \ifx\si@tempa\si@tempb
     \let\si@frac\si@frc@ugly
   \fi}
%    \end{macrocode}
%\begin{macro}{\si@load}
%\begin{macro}{\si@noload}
% Loading of support files is controlled by two keys.  The first
% defines a list of files that may be loaded, the second a list that
% will not.  This makes it easy to exclude a single file from a long
% list.
%    \begin{macrocode}
\si@opt@cmdkeys{load,noload}
\si@opt@key{alsoload}{\si@addtolist{\si@load}{#1}}
\AtEndOfPackage{
  \si@opt@disablekey{load}
    {Configuration files can only be used\MessageBreak
     when loading package}
  \si@opt@disablekey{noload}
    {Configuration files can only be used\MessageBreak
     when loading package}}
\AtEndOfPackage{
  \si@opt@key{alsoload}{%
    \@for\si@tempa:=#1\do{\si@loadfile{\si@tempa}}}}
%    \end{macrocode}
%\end{macro}
%\end{macro}
%\begin{macro}{\ifsi@colourunits}
%\begin{macro}{\ifsi@colourvalues}
%\begin{macro}{\si@unitcolour}
%\begin{macro}{\si@valuecolour}
% Colour is turned on and off using two switches and the appropriate
% options.  US spellings are also provided.
%    \begin{macrocode}
\si@opt@boolkey{colourunits}
\si@opt@boolkey{colourvalues}
\si@opt@choicekey[true]{colorunits}
  {true,false}
  {\si@colourunitsfalse
   \renewcommand*{\si@tempb}{true}%
   \ifx\si@tempa\si@tempb
     \si@colourunitstrue
   \fi}
\si@opt@choicekey[true]{colorvalues}
  {true,false}
  {\si@colourvaluesfalse
   \renewcommand*{\si@tempb}{true}%
   \ifx\si@tempa\si@tempb
     \si@colourvaluestrue
   \fi}
\si@opt@choicekey[true]{colorall}
  {true,false}
  {\si@colourvaluesfalse
   \si@colourunitsfalse
   \renewcommand*{\si@tempb}{true}%
   \ifx\si@tempa\si@tempb
     \si@colourunitstrue
     \si@colourvaluestrue
   \fi}
\si@opt@choicekey[true]{colourall}
  {true,false}
  {\si@colourvaluesfalse
   \si@colourunitsfalse
   \renewcommand*{\si@tempb}{true}%
   \ifx\si@tempa\si@tempb
     \si@colourunitstrue
     \si@colourvaluestrue
   \fi}
\si@opt@cmdkeys{unitcolour,valuecolour}
\si@opt@key{unitcolor}{\sisetup{unitcolour=#1}}
\si@opt@key{valuecolor}{\sisetup{valuecolour=#1}}
\si@opt@key{colour}{\sisetup{unitcolour=#1,valuecolour=#1}}
\si@opt@key{color}{\sisetup{unitcolour=#1,valuecolour=#1}}
%    \end{macrocode}
%\end{macro}
%\end{macro}
%\end{macro}
%\end{macro}
%\begin{macro}{\ifsi@colourneg}
%\begin{macro}{\si@negcolour}
% The set up for colouring negative numbers is similar.
%    \begin{macrocode}
\si@opt@boolkey{colourneg}
\si@opt@choicekey[true]{colorneg}
  {true,false}
  {\si@colournegfalse
   \renewcommand*{\si@tempb}{true}%
   \ifx\si@tempa\si@tempb
     \si@colournegtrue
   \fi}
\si@opt@cmdkeys{negcolour}
\si@opt@key{negcolor}{\sisetup{negcolour=#1}}
%    \end{macrocode}
%\end{macro}
%\end{macro}
%\begin{macro}{\si@textOmega}
%\begin{macro}{\si@mathsOmega}
% The various non-Latin symbols need to be handled, and given user interfaces.
% Some definitions are more complex than others; for \si{\ohm}
% things are easy.
%    \begin{macrocode}
\si@opt@cmdkeys{textOmega,mathsOmega}
\si@opt@key{mathOmega}{\sisetup{mathsOmega=#1}}
\newcommand*{\si@mathsOmega}{\text{\ensuremath{\Omega}}}
\newcommand*{\si@textOmega}{\ensuremath{\Omega}}
%    \end{macrocode}
%\end{macro}
%\end{macro}
%\changes{v1.0c}{2008/06/25}{Fixed excess loading of maths fonts}
%\begin{macro}{\si@textmu}
%\begin{macro}{\si@mathsmu}
% For the \si{\micro} symbol, some direct loading of symbols is
% needed as the maths mu sign ($\mu$) is wrong.
%    \begin{macrocode}
\si@opt@cmdkeys{textmu,mathsmu}
\si@opt@key{mathmu}{\sisetup{mathsmu=#1}}
\DeclareFontEncoding{TS1}{}{}
\DeclareFontSubstitution{TS1}{cmr}{m}{n}
\DeclareTextSymbol{\si@textmu}{TS1}{181}
\DeclareTextSymbolDefault{\si@textmu}{TS1}
\@ifpackageloaded{upgreek}
  {}
  {\DeclareFontFamily{OML}{eur}{\skewchar\font'177}
   \DeclareFontShape{OML}{eur}{m}{n}{%
    <-6> eurm5 <6-8> eurm7 <8-> eurm10}{}}
\AtBeginDocument{
  \@ifpackageloaded{upgreek}
    {\let\si@mathsmu\upmu}
    {\DeclareSymbolFont{si@greek}{OML}{eur}{m}{n}
     \DeclareMathSymbol{\si@mathsmu}{\mathord}{si@greek}{"16}}}
%    \end{macrocode}
%\end{macro}
%\end{macro}
%\begin{macro}{\si@textdegree}
%\begin{macro}{\si@mathsdegree}
%\begin{macro}{\si@textminute}
%\begin{macro}{\si@mathsminute}
%\begin{macro}{\si@textsecond}
%\begin{macro}{\si@mathssecond}
% The angle signs.
%    \begin{macrocode}
\si@opt@cmdkeys{textdegree,mathsdegree,textminute,mathsminute,
  textsecond,mathssecond}
\si@opt@key{mathdegree}{\sisetup{mathsdegree=#1}}
\si@opt@key{mathminute}{\sisetup{mathsminute=#1}}
\si@opt@key{mathsecond}{\sisetup{mathssecond=#1}}
\newcommand*{\si@textdegree}{\ensuremath{{}^{\circ}}}
\newcommand*{\si@mathsdegree}{{}^{\circ}}
\newcommand*{\si@textminute}{\ensuremath{{}^{\prime}}}
\newcommand*{\si@mathsminute}{{}^{\prime}}
\newcommand*{\si@textsecond}{\ensuremath{{}^{\prime\prime}}}
\newcommand*{\si@mathssecond}{{}^{\prime\prime}}
%    \end{macrocode}
%\end{macro}
%\end{macro}
%\end{macro}
%\end{macro}
%\end{macro}
%\end{macro}
%\begin{macro}{\si@textcelsius}
%\changes{v1.1e}{2008/11/18}{Bug fix in appearance}
%\begin{macro}{\si@mathscelsius}
% Finally, degrees Celsius, which may need the degree symbol.
%    \begin{macrocode}
\si@opt@cmdkeys{textcelsius,mathscelsius}
\si@opt@key{mathcelsius}{\sisetup{mathscelsius=#1}}
\newcommand*{\si@textcelsius}{%
  \si@textdegree\kern-\scriptspace C}
\newcommand*{\si@mathscelsius}{%
  \si@mathsdegree\kern-\scriptspace C}
%    \end{macrocode}
%\end{macro}
%\end{macro}
%\begin{macro}{\si@textringA}
%\begin{macro}{\si@mathsringA}
% The \si{\angstrom} sign.
%    \begin{macrocode}
\si@opt@cmdkeys{textringA,mathsringA}
\si@opt@key{mathringA}{\sisetup{mathsringA=#1}}
\newcommand*{\si@textringA}{\AA}
\newcommand*{\si@mathsringA}{\text{\AA}}
%    \end{macrocode}
%\end{macro}
%\end{macro}
%\begin{macro}{\ifsi@redefsymbols}
% A flag for using \pkg{textcomp} and \pkg{upgreek} to provide better
% symbols.
%    \begin{macrocode}
\si@opt@boolkey{redefsymbols}
\AtBeginDocument{
  \si@opt@disablekey{redefsymbols}
    {Symbols can only be redefined\MessageBreak
     when loading siunitx}}
%    \end{macrocode}
%\end{macro}
%\begin{macro}{\si@eVcorra}
%\begin{macro}{\si@eVcorrb}
%    \begin{macrocode}
\newlength\si@eVcorra
\newlength\si@eVcorrb
\si@opt@key{eVcorra}{\setlength\si@eVcorra{#1}}
\si@opt@key{eVcorrb}{\setlength\si@eVcorrb{#1}}
%    \end{macrocode}
%\end{macro}
%\end{macro}
%\begin{macro}{\si@locale}
%\begin{macro}{\si@loctolang}
% Handling typographic conventions needs three keys.  \opt{locale} is
% used to set the locale, \opt{loctolang} to bind to \pkg{babel}.
%    \begin{macrocode}
\si@opt@key{locale}{%
  \si@loc@load{#1}%
  \si@loc@set{#1}}%
\si@opt@key{loctolang}{\si@loc@ltol{#1}}
%    \end{macrocode}
%\end{macro}
%\end{macro}
%
%\subsection{Compatibility options}
%\begin{macro}{\ifsi@old@ugly}
%\begin{macro}{\ifsi@old@nice}
%\begin{macro}{\ifsi@old@loose}
%\begin{macro}{\ifsi@old@tight}
% With the options for the package set up, the next stage is to
% provide support for users of the older packages.  These all set up
% switches, but do not do anything.  That is left to the emulation
% files, loaded at the end of the package. First of all, the
% \units options are dealt with; there are not many.
%    \begin{macrocode}
\si@opt@compatkey{units}{ugly}
\si@opt@compatkey{units}{nice}
\si@opt@compatkey{units}{loose}
\si@opt@compatkey{units}{tight}
%    \end{macrocode}
%\end{macro}
%\end{macro}
%\end{macro}
%\end{macro}
%\begin{macro}{\ifsi@old@OHM}
%\begin{macro}{\ifsi@old@ohm}
%\begin{macro}{\ifsi@old@redef-gensymb}
%\begin{macro}{\ifsi@gensymb}
% The \unitsdef package is unfortunately much more profligate with
% options. The first set are to do with support for \pkg{gensymb}.
%    \begin{macrocode}
\si@opt@compatkey{unitsdef}{OHM}
\si@opt@compatkey{unitsdef}{ohm}
\si@opt@compatkey{unitsdef}{redef-gensymb}
\newif\ifsi@gensymb
%    \end{macrocode}
%\end{macro}
%\end{macro}
%\end{macro}
%\end{macro}
%\begin{macro}{\ifsi@old@LITER}
%\begin{macro}{\ifsi@old@liter}
%\begin{macro}{\ifsi@old@noxspace}
%\begin{macro}{\ifsi@old@noconfig}
% The second set are more general functionality.
%    \begin{macrocode}
\si@opt@compatkey{unitsdef}{LITER}
\si@opt@compatkey{unitsdef}{liter}
\si@opt@compatkey{unitsdef}{noxspace}
\si@opt@compatkey{unitsdef}{noconfig}
%    \end{macrocode}
%\end{macro}
%\end{macro}
%\end{macro}
%\end{macro}
%\begin{macro}{\ifsi@old@noabbr}
%\begin{macro}{\ifsi@old@noamperageabbr}
%\begin{macro}{\ifsi@old@nofrequncyabbr}
%\begin{macro}{\ifsi@old@nomolabbr}
%\begin{macro}{\ifsi@old@novoltageabbr}
%\begin{macro}{\ifsi@old@novolumeabbr}
%\begin{macro}{\ifsi@old@noweightabbr}
%\begin{macro}{\ifsi@old@noenergyabbr}
%\begin{macro}{\ifsi@old@nolengthabbr}
%\begin{macro}{\ifsi@old@notimeabbr}
% The final set are for control of abbreviations, and are a good
% demonstration of why to use \pkg{xkeyval}!
%    \begin{macrocode}
\si@opt@compatkey{unitsdef}{noabbr}
\si@opt@compatkey{unitsdef}{noampereageabbr}
\si@opt@compatkey{unitsdef}{nofrequncyabbr}
\si@opt@compatkey{unitsdef}{nomolabbr}
\si@opt@compatkey{unitsdef}{novoltageabbr}
\si@opt@compatkey{unitsdef}{novolumeabbr}
\si@opt@compatkey{unitsdef}{noweightabbr}
\si@opt@compatkey{unitsdef}{noenergyabbr}
\si@opt@compatkey{unitsdef}{nolengthabbr}
\si@opt@compatkey{unitsdef}{notimeabbr}
%    \end{macrocode}
%\end{macro}
%\end{macro}
%\end{macro}
%\end{macro}
%\end{macro}
%\end{macro}
%\end{macro}
%\end{macro}
%\end{macro}
%\end{macro}
%\begin{macro}{\ifsi@old@cdot}
%\begin{macro}{\ifsi@old@thickspace}
%\begin{macro}{\ifsi@old@mediumspace}
%\begin{macro}{\ifsi@old@thinspace}
%\begin{macro}{\ifsi@old@thickqspace}
%\begin{macro}{\ifsi@old@mediumqspace}
%\begin{macro}{\ifsi@old@thinqspace}
% The \SIunits package has lots of options.  These ones are all
% related to spacing.
%    \begin{macrocode}
\si@opt@compatkey{SIunits}{cdot}
\si@opt@compatkey{SIunits}{thickspace}
\si@opt@compatkey{SIunits}{mediumspace}
\si@opt@compatkey{SIunits}{thinspace}
\si@opt@compatkey{SIunits}{thickqspace}
\si@opt@compatkey{SIunits}{mediumqspace}
\si@opt@compatkey{SIunits}{thinqspace}
%    \end{macrocode}
%\end{macro}
%\end{macro}
%\end{macro}
%\end{macro}
%\end{macro}
%\end{macro}
%\end{macro}
%\begin{macro}{\ifsi@old@amssymb}
%\begin{macro}{\ifsi@old@squaren}
%\begin{macro}{\ifsi@old@pstricks}
%\begin{macro}{\ifsi@old@Gray}
%\begin{macro}{\ifsi@old@italian}
% These options are used by \SIunits to control clashes with other
% packages.
%    \begin{macrocode}
\si@opt@compatkey{SIunits}{amssymb}
\si@opt@compatkey{SIunits}{squaren}
\si@opt@compatkey{SIunits}{pstricks}
\si@opt@compatkey{SIunits}{Gray}
\si@opt@compatkey{SIunits}{italian}
%    \end{macrocode}
%\end{macro}
%\end{macro}
%\end{macro}
%\end{macro}
%\end{macro}
%\begin{macro}{\ifsi@old@textstyle}
%\begin{macro}{\ifsi@old@binary}
%\begin{macro}{\ifsi@old@noams}
%\begin{macro}{\ifsi@old@derivedinbase}
%\begin{macro}{\ifsi@old@derived}
% The miscellaneous options.
%    \begin{macrocode}
\si@opt@compatkey{SIunits}{textstyle}
\si@opt@compatkey{SIunits}{binary}
\si@opt@compatkey{SIunits}{noams}
\si@opt@compatkey{SIunits}{derivedinbase}
\si@opt@compatkey{SIunits}{derived}
%    \end{macrocode}
%\end{macro}
%\end{macro}
%\end{macro}
%\end{macro}
%\end{macro}
%\begin{macro}{\ifsi@old@noprefixcmds}
% The \pkg{hepunits} package only has one option.
%    \begin{macrocode}
\si@opt@compatkey{hepunits}{noprefixcmds}
%    \end{macrocode}
%\end{macro}
%\begin{macro}{\ifsi@old@english}
%\begin{macro}{\ifsi@old@french}
% \pkg{fancynum} provides a few options.  First the rather oddly
% named \opt{english} and \opt{french} ones.
%    \begin{macrocode}
\si@opt@compatkey{fancynum}{english}
\si@opt@compatkey{fancynum}{french}
%    \end{macrocode}
%\end{macro}
%\end{macro}
%\begin{macro}{\ifsi@old@tight}
%\begin{macro}{\ifsi@old@loose}
% A couple for spacing multiplication.
%    \begin{macrocode}
\si@opt@compatkey{fancynum}{tight}
\si@opt@compatkey{fancynum}{loose}
%    \end{macrocode}
%\end{macro}
%\end{macro}
%\begin{macro}{\ifsi@old@thinspaces}
%\begin{macro}{\ifsi@old@commas}
%\begin{macro}{\ifsi@old@plain}
% Three for digit separation.
%    \begin{macrocode}
\si@opt@compatkey{fancynum}{thinspaces}
\si@opt@compatkey{fancynum}{commas}
\si@opt@compatkey{fancynum}{plain}
%    \end{macrocode}
%\end{macro}
%\end{macro}
%\end{macro}
%\begin{macro}{\ifsi@old@spaceqspace}
% The \pkg{fancyunits} package provides one option not available with
% \SIunits.
%    \begin{macrocode}
\si@opt@compatkey{fancyunits}{spaceqspace}
%    \end{macrocode}
%\end{macro}
%
%\subsection{Constants}
% A number of macros are needed by the package that provide a
% non-changing output.  These are defined here; the intention is that
% these should not be macros that the user is likely to need to
% alter.  All of these macros have preface \cs{si@fix@}, to flag that
% that are intended as constants.  The package may rely on the
% contents of these macros for functionality.
%
%\begin{macro}{\si@fix@thin}
%\begin{macro}{\si@fix@med}
%\begin{macro}{\si@fix@medium}
%\begin{macro}{\si@fix@thick}
%\begin{macro}{\si@fix@space}
%\changes{v1.0a}{2008/06/20}{Fixed problem with \opt{space} option}
% First, there are the various space macros.  To allow both \opt{med}
% and \opt{medium} to be used as a space description, two macros are
% needed for the same output.
%    \begin{macrocode}
\newcommand*{\si@fix@thin}{\,}
\newcommand*{\si@fix@med}{\:}
\newcommand*{\si@fix@medium}{\:}
\newcommand*{\si@fix@thick}{\;}
\newcommand*{\si@fix@space}{\text{ }}
%    \end{macrocode}
%\end{macro}
%\end{macro}
%\end{macro}
%\end{macro}
%\end{macro}
%\begin{macro}{\si@fix@cdot}
%\begin{macro}{\si@fix@comma}
%\begin{macro}{\si@fix@stop}
%\begin{macro}{\si@fix@fullstop}
%\begin{macro}{\si@fix@period}
%\begin{macro}{\si@fix@times}
%\begin{macro}{\si@fix@tighttimes}
%\begin{macro}{\si@fix@tightcdot}
% Next there are macros for material that is not simply whitespace.
% To allow several options, the full-stop gets lots of names.
%    \begin{macrocode}
\newcommand*{\si@fix@cdot}{{}\cdot{}}
\newcommand*{\si@fix@comma}{{,}}
\newcommand*{\si@fix@stop}{{.}}
\newcommand*{\si@fix@fullstop}{{.}}
\newcommand*{\si@fix@period}{{.}}
\newcommand*{\si@fix@times}{\times}
\newcommand*{\si@fix@tighttimes}{\bgroup\times\egroup}
\newcommand*{\si@fix@tightcdot}{\bgroup\cdot\egroup}
%    \end{macrocode}
%\end{macro}
%\end{macro}
%\end{macro}
%\end{macro}
%\end{macro}
%\end{macro}
%\end{macro}
%\end{macro}
%\begin{macro}{\si@fix@plus}
%\begin{macro}{\si@fix@minus}
%\begin{macro}{\si@fix@pm}
%\begin{macro}{\si@fix@tightpm}
%\begin{macro}{\si@fix@mp}
% Signs for numbers are needed.
%    \begin{macrocode}
\newcommand*{\si@fix@plus}{+}
\newcommand*{\si@fix@minus}{-}
\newcommand*{\si@fix@pm}{\pm}
\newcommand*{\si@fix@tightpm}{\bgroup\pm\egroup}
\newcommand*{\si@fix@mp}{\mp}
%    \end{macrocode}
%\end{macro}
%\end{macro}
%\end{macro}
%\end{macro}
%\end{macro}
%\begin{macro}{\si@fix@two}
%\begin{macro}{\si@fix@ten}
% The literals ``2'' and ``10'' are needed for exponents.
%    \begin{macrocode}
\newcommand*{\si@fix@two}{2}
\newcommand*{\si@fix@ten}{10}
%    \end{macrocode}
%\end{macro}
%\end{macro}
%\begin{macro}{\si@fix@slash}
% Another optional component that will probably not be used by many
% people.
%    \begin{macrocode}
\newcommand*{\si@fix@slash}{/}
%    \end{macrocode}
%\end{macro}
%\begin{macro}{\si@fix@none}
% Finally for spacing, there is the possibility of nothing at all
%    \begin{macrocode}
\newcommand*{\si@fix@none}{}
%    \end{macrocode}
%\end{macro}
%
%\subsection{Symbols}
%\begin{macro}{\si@symbol}
% Each of the symbol macros needs to be set up; the options give a
% maths and text mode sign, but internally a single macro is needed
% for each.
%    \begin{macrocode}
\newcommand*{\si@symbol}[1]{%
  \expandafter\protected\expandafter\def
    \csname si@sym@#1\endcsname{%
    \ifmmode
      \expandafter\csname si@maths#1\expandafter\endcsname
    \else
      \expandafter\csname si@text#1\expandafter\endcsname
    \fi}}
%    \end{macrocode}
%\end{macro}
%\begin{macro}{\si@sym@Omega}
%\begin{macro}{\si@sym@ringA}
%\begin{macro}{\si@sym@mu}
%\begin{macro}{\si@sym@degree}
%\begin{macro}{\si@sym@minute}
%\begin{macro}{\si@sym@second}
%\begin{macro}{\si@sym@celsius}
% The various symbols are now declared.
%    \begin{macrocode}
\si@symbol{Omega}
\si@symbol{ringA}
\si@symbol{mu}
\si@symbol{degree}
\si@symbol{minute}
\si@symbol{second}
\si@symbol{celsius}
%    \end{macrocode}
%\end{macro}
%\end{macro}
%\end{macro}
%\end{macro}
%\end{macro}
%\end{macro}
%\end{macro}
% The issue of redefinition of symbols now arises.  \currpkg can
% check for the loading of a number of support package, and can then
% redefine the appropriate symbols.
%    \begin{macrocode}
\AtBeginDocument{%
  \ifsi@redefsymbols
    \@ifpackageloaded{textcomp}
      {\si@log@debug{Redefining symbols using textcomp}%
       \renewcommand*{\si@textdegree}{\textdegree}%
       \renewcommand*{\si@mathsdegree}{\text{\textdegree}}%
%    \end{macrocode}
% \pkg{mathptmx} will give issues with \pkg{textcomp} and the
% \si{\ohm} sign.
%    \begin{macrocode}
       \@ifpackageloaded{mathptmx}{}
         {\renewcommand*{\si@textmu}{\textmu}%
          \renewcommand*{\si@textOmega}{\textohm}}%
%    \end{macrocode}
% The \si{\angstrom} symbol is only redefined if the encoding is
% OT1; other encodings should have a proper glyph used for \cs{AA}.
% The \cs{encodingdefault} macro is \cs{long} for some reason.
%    \begin{macrocode}
       \renewcommand*{\si@tempa}{OT1}%
       \ifx\si@tempa\encodingdefault
         \renewcommand*{\si@mathsringA}{%
           \text{\capitalring{A}}}%
         \renewcommand*{\si@textringA}{\capitalring{A}}
       \fi}{}
    \@ifpackageloaded{upgreek}
      {\si@log@debug{Redefining symbols using upgreek}%
       \renewcommand*{\si@mathsOmega}{\Upomega}}{}
  \fi}
%    \end{macrocode}
%
%\subsection{Handling fractions}
%\begin{macro}{\si@frac}
%\begin{macro}{\si@frc@hook}
%\begin{macro}{\si@frc@frac}
%\begin{macro}{\si@frc@slash}
%\begin{macro}{\si@frc@nice}
%\begin{macro}{\si@frc@sfrac}
% Various methods of handling fractions are provided.
%    \begin{macrocode}
\newcommand*{\si@frc@frac}[2]{%
  \ensuremath{\si@frc@hook\frac{%
    \expandafter\si@unt@out\expandafter{#1}}%
    {\expandafter\si@unt@out\expandafter{#2}}}}
\let\si@frac\si@frc@frac
\newcommand*{\si@frc@hook}{}
\newcommand*{\si@frc@slash}[2]{%
    \expandafter\si@unt@out\expandafter{#1}%
    \si@out{\ensuremath{\si@slash}}%
    \expandafter\si@unt@out\expandafter{#2}}
\newcommand*{\si@frc@nice}[2]{%
  \ensuremath{\si@frc@nicefrac{\expandafter\si@unt@out%
    \expandafter{#1}}{\expandafter\si@unt@out\expandafter
      {#2}}}}
\newcommand*{\si@frc@sfrac}[2]{%
    \sfrac{\expandafter\si@unt@out\expandafter{#1}}%
      {\expandafter\si@unt@out\expandafter{#2}}}
\AtBeginDocument{
  \@ifpackageloaded{xfrac}
    {}
    {\si@log@inf{xfrac package unavailable\MessageBreak
       using `fraction=sfrac' will fall back on\MessageBreak
       nicefrac-like method}%
     \renewcommand*{\si@frc@sfrac}[2]{%
      \si@log@warn{xfrac package unavailable}%
      \si@frc@nice{#1}{#2}}}}
%    \end{macrocode}
%\end{macro}
%\end{macro}
%\end{macro}
%\end{macro}
%\end{macro}
%\end{macro}
%\begin{macro}{\si@frc@nicefrac}
%\begin{macro}{\si@frc@displen}
%\begin{macro}{\si@frc@textlen}
%\begin{macro}{\si@frc@suplen}
%\begin{macro}{\si@frc@ssuplen}
% To avoid needing \units installed, the \cs{nicefrac} macro needs
% to be emulated here. The code is taken (with permission) from
% \kg{nicefrac}.\footnote{The original is licensed under the GPL;
% thanks to the author Axel Reichert for permission to copy the code
% here.}
%    \begin{macrocode}
\newlength\si@frc@displen
\newlength\si@frc@textlen
\newlength\si@frc@suplen
\newlength\si@frc@ssuplen
\newcommand*{\si@frc@nicefrac}{%
  \ifmmode
    \expandafter\si@frc@mathsnf
  \else
    \expandafter\si@frc@textnf
  \fi}
%    \end{macrocode}
%\end{macro}
%\end{macro}
%\end{macro}
%\end{macro}
%\end{macro}
%\begin{macro}{\si@frc@mathsnf}
%\darg{numerator}
%\darg{denominator}
% The maths mode system.
%    \begin{macrocode}
\newcommand*{\si@frc@mathsnf}[2]{%
  \begingroup
    \settoheight{\si@frc@displen}{\ensuremath{%
      \displaystyle{M}}}%
    \settoheight{\si@frc@textlen}{\ensuremath{%
      \textstyle{M}}}%
    \settoheight{\si@frc@suplen}{\ensuremath{%
      \scriptstyle{M}}}%
    \settoheight{\si@frc@ssuplen}{%
      \ensuremath{\scriptscriptstyle{M}}}%
    \addtolength{\si@frc@displen}{-\si@frc@ssuplen}%
    \addtolength{\si@frc@textlen}{-\si@frc@ssuplen}%
    \addtolength{\si@frc@suplen}{-\si@frc@ssuplen}%
    \mathchoice
      {\raisebox{\si@frc@displen}{\ensuremath{%
        \scriptstyle{#1}}}}%
      {\raisebox{\si@frc@textlen}{\ensuremath{%
         \scriptstyle{#1}}}}%
      {\raisebox{\si@frc@suplen}%
        {\ensuremath{\scriptscriptstyle{#1}}}}%
      {\raisebox{\si@frc@ssuplen}%
        {\ensuremath{\scriptscriptstyle{#1}}}}%
    \mkern-2mu\relax/\mkern-1mu\relax
    \bgroup
      \mathchoice
        {\scriptstyle}%
        {\scriptstyle}%
        {\scriptscriptstyle}%
        {\scriptscriptstyle}%
      {#2}%
    \egroup
  \endgroup}
%    \end{macrocode}
%\end{macro}
%\begin{macro}{\si@frc@textnf}
%\darg{numerator}
%\darg{denominator}
% A stripped down version of the \pkg{nicefrac} system for text mode.
%    \begin{macrocode}
\newcommand*{\si@frc@textnf}[2]{%
  \begingroup
    \settoheight{\si@frc@textlen}{M}%
    \settoheight{\si@frc@ssuplen}{\fontsize\sf@size\z@\relax
      \selectfont{M}}%
    \addtolength{\si@frc@textlen}{-\si@frc@ssuplen}%
    \raisebox{\si@frc@textlen}{\fontsize\sf@size\z@\relax
      \selectfont{#1}}%
    \hspace{-0.25ex}/\hspace{-0.25ex}%
    \hbox{\fontsize\sf@size\z@\selectfont{#2}}%
  \endgroup}
%    \end{macrocode}
%\end{macro}
%\begin{macro}{\si@frc@ugly}
%\darg{numerator}
% The \cs{si@frc@ugly} macro is needed to emulate the \opt{ugly}
% option in \units, where output depends on the current mode.
%    \begin{macrocode}
\newcommand*{\si@frc@ugly}[1]{%
  \renewcommand*{\si@tempa}{#1}%
  \ifmmode
    \expandafter\si@frc@frac
  \else
    \renewcommand*{\si@tempb}{1}%
    \ifx\si@tempa\si@tempb
%    \end{macrocode}
% The slash switch cannot be used, so the possibility of the
% numerator being one is handled here.
%    \begin{macrocode}
      \setbox\si@tempboxa=\hbox{\ensuremath{\si@valuesep}}%
      \hskip-\wd\si@tempboxa\relax
      \renewcommand*{\si@tempa}{}%
    \fi
    \expandafter\si@frc@slash
  \fi
  {\si@tempa}}
%    \end{macrocode}
%\end{macro}
%
%\subsection{Font control}
% A number of controls and tests are needed to control the font used
% for output.  Underlying all of this is the \AMS\ package
% \pkg{amstext} package, providing the \cs{text} command.  Much of
% the font control system here is taken more or less verbatim from
% \SIstyle; modifications have been made to fit the \currpkg
% interface.
%
%\begin{macro}{\si@fam@sf}
%\begin{macro}{\si@fam@tt}
% The package needs to know the maths font families in use.  This is
% set right at the start of the document, after any changes can have
% been made by, for example, \pkg{fontspec}.
%    \begin{macrocode}
\g@addto@macro{\document}{%
  \ifdefined\mathsf
    \setbox\si@tempboxa=\hbox{%
      $\mathsf{\global\chardef\si@fam@sf=\fam}$}%
  \else
    \si@log@inf{\string\mathsf not found}%
    \global\chardef\si@fam@sf=99\relax
  \fi
  \ifdefined\mathtt
    \setbox\si@tempboxa=\hbox{%
      $\mathtt{\global\chardef\si@fam@tt=\fam}$}%
  \else
    \si@log@inf{\string\mathtt not found}%
    \global\chardef\si@fam@tt=99\relax
  \fi}
%    \end{macrocode}
%\end{macro}
%\end{macro}
%\begin{macro}{\si@fam@ifbtext}
%\begin{macro}{\si@fam@ifbmaths}
%\darg{code}
% These tests check for bold in text and maths mode, respectively.
%    \begin{macrocode}
\newcommand*{\si@fam@ifbtext}[1]{%
  \if b\expandafter\@car\f@series\@nil
    #1\fi}
\newcommand*{\si@fam@ifbmaths}[1]{%
  \renewcommand*{\si@tempa}{bold}%
  \ifx\math@version\si@tempa
    #1\fi}
%    \end{macrocode}
%\end{macro}
%\end{macro}
%\begin{macro}{\si@fam@ifbinline}
% For compatibility with \units, a method to change the behaviour
% when in inline maths is needed for the bold detector.
%    \begin{macrocode}
\newcommand*{\si@fam@ifbinline}{%
  \ifsi@inlinebtext
    \expandafter\si@fam@ifbtext
  \else
    \expandafter\si@fam@ifbmaths
  \fi}
%    \end{macrocode}
%\end{macro}
%\begin{macro}{\si@fam@ifitext}
%\darg{code}
% This test check for italic or slanted text in text mode, by
% negation (upright text is |n|).
%    \begin{macrocode}
\newcommand*{\si@fam@ifitext}[1]{%
  \if n\expandafter\@car\f@series\@nil\else
    #1\fi}
%    \end{macrocode}
%\end{macro}
%\begin{macro}{\si@fam@mode}
% Detection of the current mode needs to happen``early'' (before any
% change of \cs{ensuremath}).  So a short macro is provided to do the
% job.
%    \begin{macrocode}
\newcommand*{\si@fam@mode}{%
  \ifsi@obeymode
    \ifmmode
      \sisetup{mode=maths}%
    \else
      \sisetup{mode=text}%
    \fi
  \fi}
%    \end{macrocode}
%\end{macro}
%\begin{macro}{\si@fam@colourcmd}
% The colour command is set up.
%    \begin{macrocode}
\AtBeginDocument{
  \@ifpackageloaded{color}
    {\let\si@fam@colourcmd\color}
    {\let\si@fam@colourcmd\@gobble}}
%    \end{macrocode}
%\end{macro}
%
%\begin{macro}{\ifsi@fam@set}
%\begin{macro}{\ifsi@textmode}
% A marker is set up to check if font-matching has been taken place.
% A second flag is used to track the use of text mode.
%    \begin{macrocode}
\newif\ifsi@fam@set
\newif\ifsi@textmode
%    \end{macrocode}
%\end{macro}
%\end{macro}
%\begin{macro}{\si@fam@set}
% Using the code from \SIstyle as a base, a set of tests are used to
% set the current font families and weights.  To begin with, the
% mode to use is set up.
%    \begin{macrocode}
\newcommand*{\si@fam@set}{%
  \ifsi@out@num
    \ifsi@numtextmode
      \expandafter\expandafter\expandafter\si@textmodetrue
    \else
      \expandafter\expandafter\expandafter\si@textmodefalse
    \fi
  \else
    \ifsi@unittextmode
      \expandafter\expandafter\expandafter\si@textmodetrue
    \else
      \expandafter\expandafter\expandafter\si@textmodefalse
    \fi
  \fi
%    \end{macrocode}
%\begin{macro}{\si@mathsrm}
%\begin{macro}{\si@mathssf}
%\begin{macro}{\si@mathstt}
%\begin{macro}{\si@textrm}
%\begin{macro}{\si@textsf}
%\begin{macro}{\si@texttt}
%\begin{macro}{\si@colourcmd}
%\begin{macro}{\si@colour}
% The appropriate font macros are now established, if necessary.
%    \begin{macrocode}
  \ifsi@fam@set\else
    \let\si@colourcmd\@gobble
    \ifsi@out@num
      \let\si@mathsrm\si@valuemathsrm
      \let\si@mathssf\si@valuemathssf
      \let\si@mathstt\si@valuemathstt
      \let\si@textrm\si@valuetextrm
      \let\si@textsf\si@valuetextsf
      \let\si@texttt\si@valuetexttt
      \ifsi@colourvalues
        \let\si@colourcmd\si@fam@colourcmd
      \fi
      \let\si@colour\si@valuecolour
    \else
      \let\si@mathsrm\si@unitmathsrm
      \let\si@mathssf\si@unitmathssf
      \let\si@mathstt\si@unitmathstt
      \let\si@textrm\si@unittextrm
      \let\si@textsf\si@unittextsf
      \let\si@texttt\si@unittexttt
      \ifsi@colourunits
        \let\si@colourcmd\si@fam@colourcmd
      \fi
      \let\si@colour\si@unitcolour
    \fi
  \fi
  \si@fam@settrue
%    \end{macrocode}
%\end{macro}
%\end{macro}
%\end{macro}
%\end{macro}
%\end{macro}
%\end{macro}
%\end{macro}
%\end{macro}
% The temporary macros are needed for the \cs{ifx} tests, which need to
% be expanded once.
%    \begin{macrocode}
  \edef\si@tempa{\sfdefault}%
  \edef\si@tempb{\ttdefault}%
%    \end{macrocode}
%\begin{macro}{\si@fam@maths}
%\begin{macro}{\si@fam@text}
% The surrounding font family is only tested if matching is
% requested.  First, the defaults are set up assuming no detection
% takes place.
%    \begin{macrocode}
  \expandafter\let\expandafter\si@fam@maths
    \csname\si@mathsrm\endcsname
  \expandafter\let\expandafter\si@fam@text
    \csname\si@textrm\endcsname
  \ifsi@obeyfamily
    \si@log@debug{Font detection: checking font}%
%    \end{macrocode}
%\end{macro}
%\end{macro}
% The detection code has to check the mode currently in operation.
% Display mathematics can be handled in two ways, so this means some
% code is repeated: it is spun out to separate routines.
%    \begin{macrocode}
    \ifmmode
      \ifinner
        \si@log@debug{Font detection: inline maths}%
        \si@fam@dettext
      \else
        \si@log@debug{Font detection: display maths}%
        \ifsi@detectdisplay
          \si@fam@detmaths
        \else
          \si@fam@dettext
        \fi
      \fi
    \else
      \si@log@debug{Font detection: text}%
      \si@fam@dettext
    \fi
  \else
    \si@log@debug{Font detection: inactive}%
  \fi
%    \end{macrocode}
%\begin{macro}{\si@fam@bold}
% With the font family set, the next check is for bold text.  This
% again needs to examine the current mode. Things are a bit more
% complex than in \SIstyle as it is possible to be typesetting in
% either text or maths mode.  The bold command is set up with
% \cs{def}, as nested calls can occur.
%    \begin{macrocode}
  \def\si@fam@bold{\unboldmath\mdseries}%
  \ifsi@obeybold
    \si@log@debug{Weight detection: checking weight}%
    \ifmmode
      \ifdim\displaywidth>0pt\relax
        \ifsi@detectdisplay
          \expandafter\si@fam@ifbmaths
        \else
          \expandafter\si@fam@ifbtext
        \fi
        \si@fam@setbold
      \else
        \si@fam@ifbinline\si@fam@setbold
      \fi
    \else
      \si@fam@ifbtext\si@fam@setbold
    \fi
  \fi
%    \end{macrocode}
%\end{macro}
%\begin{macro}{\si@fam@italic}
% The value of \opt{obeyitalic} is now tested; as this does nothing
% in maths mode, a reminder is added to the log.
%    \begin{macrocode}
  \let\si@fam@italic\upshape
  \ifsi@obeyitalic
    \si@log@debug{Italic detection: checking italic}%
    \si@fam@ifitext
      {\let\si@fam@italic\relax
       \si@log@debug{Italic detection: italic}}%
  \fi}
%    \end{macrocode}
%\end{macro}
%\end{macro}
%\begin{macro}{\si@fam@detmaths}
%\begin{macro}{\si@fam@dettext}
% Two detection macros are needed for maths and text mode. This
% allows handling of the various combinations without needing too
% many code lines.
%    \begin{macrocode}
\newcommand*{\si@fam@detmaths}{%
  \ifnum\the\fam=\si@fam@sf
    \si@log@debug{Font detection: sf}%
    \expandafter\let\expandafter\si@fam@maths
      \csname\si@mathssf\endcsname
    \expandafter\let\expandafter\si@fam@text
      \csname\si@textsf\endcsname
  \else
    \ifnum\the\fam=\si@fam@tt
      \si@log@debug{Font detection: tt}%
      \expandafter\let\expandafter\si@fam@maths
        \csname\si@mathstt\endcsname
      \expandafter\let\expandafter\si@fam@text
        \csname\si@texttt\endcsname
    \else
      \si@log@debug{Font detection: rm}%
      \expandafter\let\expandafter\si@fam@maths
        \csname\si@mathsrm\endcsname
      \expandafter\let\expandafter\si@fam@text
        \csname\si@textrm\endcsname
    \fi
  \fi}
\newcommand*{\si@fam@dettext}{%
  \ifx\f@family\si@tempa
    \si@log@debug{Font detection: sf}%
    \expandafter\let\expandafter\si@fam@maths
      \csname\si@mathssf\endcsname
    \expandafter\let\expandafter\si@fam@text
      \csname\si@textsf\endcsname
  \else
    \ifx\f@family\si@tempb
      \si@log@debug{Font detection: tt}%
      \expandafter\let\expandafter\si@fam@maths
        \csname\si@mathstt\endcsname
      \expandafter\let\expandafter\si@fam@text
        \csname\si@texttt\endcsname
    \else
      \si@log@debug{Font detection: rm}%
      \expandafter\let\expandafter\si@fam@maths
        \csname\si@mathsrm\endcsname
      \expandafter\let\expandafter\si@fam@text
        \csname\si@textrm\endcsname
    \fi
  \fi}
%    \end{macrocode}
%\end{macro}
%\end{macro}
%\begin{macro}{\si@fam@setbold}
%\begin{macro}{\si@fam@boldify}
% For setting bold, a couple of control macros are needed.
%    \begin{macrocode}
\newcommand*{\si@fam@setbold}{%
  \si@log@debug{Weight detection: bold weight}%
  \let\si@fam@bold\si@fam@boldify}
\newcommand*{\si@fam@boldify}{\boldmath\bfseries}
%    \end{macrocode}
%\end{macro}
%\end{macro}
%
%\subsection{Formatting numbers}
%\begin{macro}{\num}
%\doarg{keyval options}
%\darg{number}
% The system used here is modelled on that in \numprint; the
% input is broken down into single tokens, each one is examined and
% the result is re-assembled into an output number.  However, various
% changes have been made to the system used, and so the macros here
% are not simply renamed copies of those in \numprint.  The user
% macro \cs{num} sets any local keys, then calls the number
% formatting macro on the processed number.
%    \begin{macrocode}
\si@newrobustcmd*{\num}[2][]{%
  \begingroup
    \sisetup{#1}%
    \si@fam@mode
    \si@num@intabfalse
    \si@log@debug{Processing \string\num\space input `#2'}%
    \expandafter\si@out@num\expandafter{\si@num{#2}}%
  \endgroup}
%    \end{macrocode}
%\end{macro}
%\begin{macro}{\ifsi@num@intab}
% A flag for processing inside a table is needed.
%    \begin{macrocode}
\newif\ifsi@num@intab
%    \end{macrocode}
%\end{macro}
%\begin{macro}{\si@num}
%\darg{number}
% This is the main processing macro.  Unlike the related macro in
% \numprint, the output of this macro is not subjected to any font
% changes.  That is left to one of the \cs{si@out@\ldots} macros.  No
% grouping is applied here; any call to \cs{si@num} (or any of the
% sub-macros) must be within a group as the definitions used rely on
% this.  Grouping is not applied here so that other macros can get
% the various separated parts of the input.
%    \begin{macrocode}
\newcommand*{\si@num}[1]{%
%    \end{macrocode}
% The argument of the macro is fully expanded before any processing.
% By using \cs{scantokens}, any odd problems from packages with
% active characters can be avoided.
%    \begin{macrocode}
  \si@num@fixpm
  \begingroup
    \makeatletter
    \@makeother{\,}%
    \@makeother{\.}%
    \@makeother{\+}%
    \@makeother{\-}%
    \def~{}%
    \def\,{}%
    \catcode`\~=\active\relax
    \catcode`\^=\active\relax
    \catcode`\_=\active\relax
    \everyeof{\noexpand}%
    \endlinechar\m@ne
    \protected@xdef\si@tempa{\scantokens{#1}}%
  \endgroup
%    \end{macrocode}
% Processing only takes place if there is actually something in the
% argument.  This is tested once ``hard'' spaces have been stripped
% out; if there is input other than spaces, the processor first
% checks the validity of the input, then moves on to format it.
%    \begin{macrocode}
  \si@ifnotmtarg{\si@tempa}
    {\si@num@ifvalid{\si@tempa}
      {\si@num@format{\si@tempa}}
%    \end{macrocode}
% The parser has to bailed-out, and so no further processing of
% the input is done.  Instead, whatever was passed to the macro is
% returned as supplied.
%    \begin{macrocode}
      {\si@log@err{Invalid character `#1' in numerical input}%
         {Only characters from the list
          `\si@numvalid'\MessageBreak should be present in the
          argument of the \string\num\space  macro\MessageBreak
         (or derivative such as an `s' column)}%
       {#1}}}}
%    \end{macrocode}
%\end{macro}
%\begin{macro}{\si@num@fixpm}
%\begin{macro}{\si@num@pm}
%\begin{macro}{\si@num@mp}
%\begin{macro}{\pm}
%\begin{macro}{\mp}
% With certain packages loaded, there can be issues with \cs{pm} and
% \cs{mp}.  To avoid this, the \eTeX\ \cs{protected} system is
% employed; this is only used within local groups.
%    \begin{macrocode}
\newcommand*{\si@num@fixpm}{%
  \let\si@num@pm\pm
  \let\si@num@mp\mp
  \protected\def\pm{\si@num@pm}%
  \protected\def\mp{\si@num@mp}}
%    \end{macrocode}
%\end{macro}
%\end{macro}
%\end{macro}
%\end{macro}
%\end{macro}
%\begin{macro}{\si@num@ifvalid}
%\begin{macro}{\si@num@valid}
%\darg{chars}
% Assuming that there is a non-space argument to \cs{si@num}, every
% character is checked to ensure it is valid in the context, so that
% further processing can occur without sanity checks.  If the
% character is valid, recursion occurs.
%    \begin{macrocode}
\newcommand*{\si@num@ifvalid}[1]{%
  \begingroup
    \si@switchtrue
    \expandafter\si@num@valid#1\@empty\@empty
    \ifsi@switch
      \aftergroup\@firstoftwo
    \else
      \aftergroup\@secondoftwo
    \fi
  \endgroup}
\def\si@num@valid#1#2\@empty{%
  \si@str@ifchrstr{#1}{\si@numvalid}
    {\ifx\@empty#2\@empty\else
       \si@num@valid#2\@empty\@empty\@empty
     \fi}
    {\si@switchfalse}}
%    \end{macrocode}
%\end{macro}
%\end{macro}
%\begin{macro}{\si@num@in}
%\begin{macro}{\si@num@out}
%\begin{macro}{\si@num@exp}
%\begin{macro}{\si@num@expsign}
%\begin{macro}{\si@num@mant}
%\begin{macro}{\si@num@mantsign}
%\begin{macro}{\si@num@err}
%\begin{macro}{\si@num@xpart}
%\begin{macro}{\si@num@ambig}
%\begin{macro}{\si@tab@out}
%\begin{macro}{\si@tab@expout}
% Various storage macros are needed.
%    \begin{macrocode}
\newcommand*{\si@num@in}{}
\newcommand*{\si@num@out}{}
\newcommand*{\si@num@exp}{}
\newcommand*{\si@num@expsign}{}
\newcommand*{\si@num@mant}{}
\newcommand*{\si@num@mantsign}{}
\newcommand*{\si@num@err}{}
\newcommand*{\si@num@xpart}{}
\newcommand*{\si@num@ambig}{}
\newcommand*{\si@tab@out}{}
\newcommand*{\si@tab@expout}{}
%    \end{macrocode}
%\end{macro}
%\end{macro}
%\end{macro}
%\end{macro}
%\end{macro}
%\end{macro}
%\end{macro}
%\end{macro}
%\end{macro}
%\end{macro}
%\end{macro}
%\begin{macro}{\ifsi@num@erropen}
% A flag is set up for tracking unclosed errors.
%    \begin{macrocode}
\newif\ifsi@num@erropen
%    \end{macrocode}
%\end{macro}
%\begin{macro}{\si@num@format}
%\begin{macro}{\si@num@arg}
%\darg{number}
% The number processor starts by saving |#1| (odd things happen
% otherwise).  A hook is also provided to allow modifications by
% other macros.
%    \begin{macrocode}
\newcommand*{\si@num@arg}{}
\newcommand*{\si@num@format}[1]{%
  \protected@edef\si@num@arg{#1}%
  \si@log@debug{Formatting number `\si@num@arg'}%
%    \end{macrocode}
%\end{macro}
% The storage areas are emptied.
%    \begin{macrocode}
  \renewcommand*{\si@num@in}{}%
  \renewcommand*{\si@num@exp}{}%
  \renewcommand*{\si@num@expsign}{}%
  \renewcommand*{\si@num@mant}{}%
  \renewcommand*{\si@num@mantsign}{}%
  \renewcommand*{\si@num@err}{}%
  \renewcommand*{\si@num@xpart}{}%
%    \end{macrocode}
% Any ``x-part'' is now found, leaving the first number in
% \cs{si@num@in} and anything else in \cs{si@num@xpart}.
%    \begin{macrocode}
  \si@switchfalse
  \expandafter\si@num@findxpart\si@num@arg\@empty\@empty
%    \end{macrocode}
% The input is split into an mantissa and an exponent; the flag is
% used here to indicate if an exponent is found.  The mantissa will
% end up in \cs{si@num@mant}, and the exponent in \cs{si@num@exp}.
%    \begin{macrocode}
  \si@switchfalse
  \si@num@sepmantexp{\si@num@in}%
%    \end{macrocode}
% The sign and value of the mantissa and exponent are separated; the
% mantissa is done after the exponent as this makes life easier when
% using the table-formatting fork. Checks are then needed, as a sign
% with no value is potentially-valid for the mantissa (for example
% \num{-e10}).
%    \begin{macrocode}
  \si@num@sepsign{exp}%
  \si@num@sepsign{mant}%
  \ifx\@empty\si@num@exp\@empty
    \ifx\@empty\si@num@expsign\@empty\else
      \si@log@warn{Sign but no number for `\si@num@arg'}%
    \fi
    \let\si@num@expsign\@empty
  \fi
  \ifx\@empty\si@num@mant\@empty
    \ifx\@empty\si@num@mantsign\@empty\else
      \ifx\@empty\si@num@exp\@empty
        \si@log@warn{Sign but no number for `\si@num@arg'}%
        \let\si@num@mantsign\@empty
      \fi
    \fi
  \fi
%    \end{macrocode}
% A check for negative mantissa values is made, to allow some
% colour-based trickery.
%    \begin{macrocode}
  \renewcommand*{\si@tempa}{{-}}%
  \ifx\si@num@mantsign\si@tempa
    \ifsi@colourneg
      \expandafter\expandafter\expandafter\si@fam@colourcmd
    \else
      \expandafter\expandafter\expandafter\@gobble
    \fi
  \else
    \expandafter\@gobble
  \fi
  {\si@negcolour}%
%    \end{macrocode}
% The next stage is to process the remaining data, to find the
% decimal marker and reformat correctly.  These two macros control
% this entire process.  The exponent is processed first as this makes
% life easier when the system is used to typeset tabular
% material.\footnote{The contents of \cs{si@num@predec} and
% \cs{si@num@postdec} are needed for the mantissa.}
%    \begin{macrocode}
  \si@num@procnum{exp}%
  \si@num@procnum{mant}%
%    \end{macrocode}
% If the exponent is zero, then if might need to be deleted.
%    \begin{macrocode}
  \si@str@ifonlychrs{\si@num@exp}{0\si@numdecimal}
    {\ifsi@allowzeroexp\else
       \renewcommand*{\si@num@exp}{}%
       \ifx\@empty\si@num@mant\@empty
         \renewcommand*{\si@num@mant}{1}%
       \fi
     \fi}{}%
%    \end{macrocode}
% To build up the number for typesetting, a rather complex series
% of tests is needed.  First, the ``ambiguous error'' flag is
% set if there is an exponent and the package has been asked to check
% this.  The same flag will already be true if a unit is present and
% checking is active.
%    \begin{macrocode}
  \ifx\@empty\si@num@exp\@empty\else
    \ifsi@trapambigerr
      \expandafter\expandafter\expandafter\si@num@ambigerrtrue
    \fi
  \fi
%    \end{macrocode}
%\begin{macro}{\si@num@out}
%\begin{macro}{\si@tab@out}
%\begin{macro}{\si@tab@expout}
% Processing now divides, as when used with the |S| column some extra
% steps are needed.  Inside a table, the macro \cs{si@tab@out} holds
% the part of the mantissa before the decimal sign.  The post-decimal
% part then ends up in \cs{i@num@out}.  On the other hand, under
% normal circumstances the entire mantissa should be copied to
% \cs{si@num@out}.
%    \begin{macrocode}
  \protected@edef\si@num@out{%
    \ensuremath{{\si@num@mantsign}}\si@num@mant}%
  \renewcommand*{\si@tempa}{num}%
  \ifsi@num@intab
    \protected@edef\si@tab@out{%
      \ensuremath{{\si@num@mantsign}}\si@num@predec}%
    \protected@edef\si@num@out{\si@num@postdec}%
    \renewcommand*{\si@tempa}{tab}%
  \fi
%    \end{macrocode}
% Now there is the error part to handle.  For tables, a check has to
% be made so the error ends up in the correct part of the number. The
% value of \cs{si@tempa} is used to track this., and so is set up
% here.
%    \begin{macrocode}
  \ifx\@empty\si@num@postdec\@empty\else
    \renewcommand*{\si@tempa}{num}%
  \fi
  \ifx\@empty\si@num@err\@empty\else
%    \end{macrocode}
% If there is an error, and it is begin separated, the problem arises
% of the potential for ambiguous values.  This can only apply outside
% of a table, as \opt{seperr} is disabled in tabular material.
%    \begin{macrocode}
    \ifsi@seperr
      \ifsi@num@ambigerr
        \protected@edef\si@num@out{%
          \ensuremath{\si@openerr}\si@num@out}%
        \si@repeatunitsfalse
        \expandafter\si@num@erropentrue
      \else
%    \end{macrocode}
% If there is an exponential part and an error, errors are being
% separated and ambiguous errors are not trapped, then there is work
% to do.  The exponent part is added to the \emph{error}, and deleted
% from the mantissa if necessary.
%    \begin{macrocode}
        \ifsi@trapambigerr\else
          \ifx\@empty\si@num@exp\@empty\else
            \protected@edef\si@num@err{%
              \si@num@err\expandafter\@car\si@numexp\@nil
              \si@num@expsign\si@num@exp}%
            \ifsi@repeatunits\else
              \renewcommand*{\si@num@exp}{}%
              \renewcommand*{\si@num@expsign}{}%
            \fi
          \fi
        \fi
      \fi
%    \end{macrocode}
% If \opt{seperr} is not in force, the error and mantissa have to be
% recombined.  This is handled so that the same macro deals with
% tables and normal processing.
%    \begin{macrocode}
    \else
      \expandafter\protected@edef\csname
        si@\si@tempa @out\endcsname{%
          \si@num@out\ensuremath{\si@errspace}\ensuremath
          {\si@openerr}\si@num@err\ensuremath{\si@closeerr}}%
      \renewcommand*{\si@num@err}{}%
    \fi
  \fi
%    \end{macrocode}
% The main reconstruction can now occur.  This performs various
% checks on the validity of the input, and adds the necessary
% ``filler'' between the supplied data.
%    \begin{macrocode}
  \renewcommand*{\si@tempa}{num@out}%
  \ifsi@num@erropen
    \renewcommand*{\si@tempa}{num@ambig}%
  \fi
  \ifsi@num@intab
    \renewcommand*{\si@tempa}{tab@expout}%
  \fi
  \ifx\@empty\si@num@exp\@empty
    \ifx\@empty\si@num@mant\@empty
      \si@log@err{Invalid number format `\si@num@arg'}
        {Something is wrong with the number format; does it
         contain \MessageBreak any numbers (from the list
         `\si@numdigits')?}%
      \renewcommand*\si@num@out{}%
    \fi
  \else
    \ifx\@empty\si@num@mant\@empty\else
      \expandafter\protected@edef\csname
        si@\si@tempa\endcsname{%
          \csname si@\si@tempa\endcsname\ensuremath{{}%
          \si@expproduct{}}}%
    \fi
    \expandafter\protected@edef\csname
      si@\si@tempa\endcsname{%
        \csname si@\si@tempa\endcsname\si@expbase
        \textsuperscript{\ensuremath{\si@num@expsign}%
        \si@num@exp}}%
  \fi
%    \end{macrocode}
%\end{macro}
%\end{macro}
%\end{macro}
% With everything done, the result is output.
%    \begin{macrocode}
  \ifsi@num@intab\else
    \expandafter\si@num@out
  \fi
%    \end{macrocode}
% If there is anything inside the ``x-part'', then there is now
% more work to do.
%    \begin{macrocode}
  \ifx\@empty\si@num@err\@empty\else
    \expandafter\si@num@procerr
  \fi
  \ifsi@num@erropen
    \expandafter\si@out@num\expandafter{%
      \ensuremath{\si@closeerr}}%
      \ifx\@empty\si@num@ambig\@empty\else
        \expandafter\si@out@num\expandafter{\si@num@ambig}%
        \renewcommand*{\si@num@ambig}{}%
    \fi
  \fi
  \si@num@erropenfalse
  \ifx\@empty\si@num@xpart\@empty\else
    \expandafter\si@num@sepxpart
  \fi}
%    \end{macrocode}
%\end{macro}
%\begin{macro}{\si@num@findxpart}
% Before processing the number, any multiplied parts have to be found
% and removed.  As there can be more than one product sign, the
% building process here has no error checking.
%    \begin{macrocode}
\def\si@num@findxpart#1#2\@empty{%
  \si@str@ifchrstr{#1}{\si@numprod}
    {\si@switchtrue\si@seperrfalse}{}%
  \ifsi@switch
    \protected@edef\si@num@xpart{\si@num@xpart#1}%
  \else
    \protected@edef\si@num@in{\si@num@in#1}%
  \fi
  \ifx\@empty#2\@empty\else
    \si@num@findxpart#2\@empty
  \fi}
%    \end{macrocode}
%\end{macro}
%\begin{macro}{\si@num@sepmantexp}
%\begin{macro}{\si@num@mantexp}
%\darg{number}
% Splitting the mantissa and exponent first checks for characters to
% gobble, which are simply thrown away.  For any other input, there
% are two possibilities.  If the character is an exponent marker,
% then the package switches from collecting the mantissa to
% collecting the exponent (after a sanity check).  All other
% characters are added to either the mantissa or the exponent, as
% appropriate.
%    \begin{macrocode}
\newcommand*{\si@num@sepmantexp}[1]{%
  \expandafter\si@num@mantexp#1\@empty\@empty}
\def\si@num@mantexp#1#2\@empty{%
  \si@str@ifchrstr{#1}{\si@numgobble}
    {\si@log@debug{Gobbling `#1' in \si@num@arg}}
    {\si@str@ifchrstr{#1}{\si@numexp}
       {\ifsi@switch
          \si@log@err{Duplicate exponent marker found}
            {Only a single exponent character \MessageBreak
            (from the list `\si@numexp')\MessageBreak may
            occur in a numerical argument}%
        \else
          \si@log@debug{Exponent marker `#1' found in
            `\si@num@arg'}%
        \fi
        \si@switchtrue}%
       {\ifsi@switch
          \expandafter\si@num@addexp
        \else
          \expandafter\si@num@addmnt
        \fi
        {#1}}}%
%    \end{macrocode}
% If the recursion has not bottomed out, another loop occurs.
%    \begin{macrocode}
  \ifx\@empty#2\@empty
    \expandafter\@gobble
  \else
    \expandafter\si@num@sepmantexp
  \fi
  {#2}}
%    \end{macrocode}
%\end{macro}
%\end{macro}
%\begin{macro}{\si@num@addmnt}
%\begin{macro}{\si@num@addexp}
%\darg{char}
% To allow \cs{expandafter} use in the above, the actual addition to
% the appropriate macro is handled here.  Two helper macros are
% needed.
%    \begin{macrocode}
\newcommand*{\si@num@addmnt}[1]{%
  \si@num@addmntexp{#1}{mant}{mantissa}}
\newcommand*{\si@num@addexp}[1]{%
  \si@num@addmntexp{#1}{exp}{exponent}}
%    \end{macrocode}
%\end{macro}
%\end{macro}
%\begin{macro}{\si@num@addmntexp}
%\darg{char}
%\darg{either \texttt{mant} or \texttt{exp}}
%\darg{info-text}
% Then the business end.
%    \begin{macrocode}
\newcommand*{\si@num@addmntexp}[3]{%
  \si@log@debug{Adding `#1' to #3 for `\si@num@arg'}%
  \expandafter\protected@edef\csname si@num@#2\endcsname{%
    \csname si@num@#2\endcsname#1}}
%    \end{macrocode}
%\end{macro}
%\begin{macro}{\si@num@sepsign}
%\darg{either \texttt{mant} or \texttt{exp}}
% The input is now tested for a sign.  If one exists, it is
% transferred into the \cs{si@num@}|#1sign| storage macro, with the
% remained of the number in \cs{si@num@}|#1|.  The only check made
% directly here is that there is something to process.
%    \begin{macrocode}
\newcommand*{\si@num@sepsign}[1]{%
  \expandafter\ifx\expandafter\@empty
    \csname si@num@#1\endcsname\@empty
    \expandafter\@gobble
  \else
    \expandafter\si@num@gensign
  \fi
  {#1}}
%    \end{macrocode}
%\end{macro}
%\begin{macro}{\si@num@gensign}
%\darg{either \texttt{mant} or \texttt{exp}}
% The sign generator starts by calling the procedure to check if the
% input contains a valid one- or two-digit sign.  The results are
% returned as \cs{si@num@sign} and \cs{si@num@value}.
%    \begin{macrocode}
\newcommand*{\si@num@gensign}[1]{%
  \expandafter\expandafter\expandafter\si@num@findsign
    \csname si@num@#1\endcsname\@empty\@empty
%    \end{macrocode}
% If no sign has been found, then there may be a need to add one
% anyway.
%    \begin{macrocode}
  \ifx\@empty\si@num@sign\@empty
    \ifx\@empty\si@num@value\@empty
      \expandafter\expandafter\expandafter\@gobble
    \else
      \expandafter\expandafter\expandafter\si@num@addsign
    \fi
  \else
    \expandafter\@gobble
  \fi
  {#1}%
%    \end{macrocode}
% The appropriate storage areas are now assigned.
%    \begin{macrocode}
  \expandafter\let\csname si@num@#1sign\endcsname\si@num@sign
  \expandafter\let\csname si@num@#1\endcsname\si@num@value}
%    \end{macrocode}
%\end{macro}
%\begin{macro}{\si@num@findsign}
%\begin{macro}{\si@num@sign}
%\begin{macro}{\si@num@value}
% The first one or two characters of the mantissa or exponent may
% contain a sign.  To test for this, the first two characters of the
% number are split off, and examined.  Two characters are used so
% that \cs{pm} and \cs{mp} can be represented by |+-| and |-+|,
% respectively. To allow the user to alter the valid signs, but
% retain this conversion, the generic character test is used before
% checking specific matches.
%    \begin{macrocode}
\newcommand*{\si@num@sign}{}
\def\si@num@findsign#1#2#3\@empty{%
  \si@num@delplusfalse
  \si@str@ifchrstr{#1}{\si@numsign}{%
    \si@str@ifchrstr{#2}{\si@numsign}{%
      \if +#1%
        \if -#2%
          \si@log@debug{Found sign combination +- for
            `\si@num@arg'}%
          \renewcommand*{\si@num@sign}{{\si@pm}}%
        \else
          \si@log@inf{Unknown sign combination `#1#2'}%
          \renewcommand*{\si@num@sign}{{#1#2}}%
        \fi
      \else
        \if -#1%
          \if +#2%
            \si@log@debug{Found sign combination -+ for
              `\si@num@arg'}%
            \renewcommand*{\si@num@sign}{{\mp}}%
          \else
            \si@log@inf{Unknown sign combination `#1#2'}%
            \renewcommand*{\si@num@sign}{{#1#2}}%
          \fi
        \else
          \si@log@inf{Unknown sign combination `#1#2'}%
          \renewcommand*{\si@num@sign}{{#1#2}}%
        \fi
      \fi
      \protected@edef\si@num@value{#3}}%
%    \end{macrocode}
% Only one valid sign character.
%    \begin{macrocode}
      {\si@log@debug{Found single sign character `#1' for
         `\si@num@arg'}%
       \renewcommand*{\si@num@sign}{{#1}}%
       \if +#1%
         \ifsi@retainplus\else
           \expandafter\expandafter\expandafter\si@num@killsign
         \fi
       \fi
       \protected@edef\si@num@value{#2#3}}}%
%    \end{macrocode}
% No valid sign, so \cs{@empty} is returned for the sign .
%    \begin{macrocode}
    {\si@log@debug{No sign found for `\si@num@arg'}%
     \renewcommand*{\si@num@sign}{}%
     \protected@edef\si@num@value{#1#2#3}}}
%    \end{macrocode}
%\end{macro}
%\end{macro}
%\end{macro}
%\begin{macro}{\ifsi@num@delplus}
%\begin{macro}{\si@num@killsign}
% A simple spin-out to remove a plus sign.  A flag is set as it might
% be useful to know this.
%    \begin{macrocode}
\newif\ifsi@num@delplus
\newcommand*{\si@num@killsign}{%
  \si@num@delplustrue
  \renewcommand*{\si@num@sign}{}}
%    \end{macrocode}
%\end{macro}
%\end{macro}
%\begin{macro}{\si@num@addsign}
%\begin{macro}{\si@num@asign}
%\darg{either \texttt{mant} or \texttt{exp}}
% The macro to add a sign to an unsigned number has to check whether
% this is a mantissa or an exponent.  The result is still placed in
% \cs{si@num@sign} for ease of processing later.
%    \begin{macrocode}
\newcommand*{\si@num@addsign}[1]{%
  \begingroup
    \renewcommand*{\si@tempa}{#1}%
    \renewcommand*{\si@tempb}{mant}%
    \ifx\si@tempa\si@tempb
      \aftergroup\@firstoftwo
    \else
      \aftergroup\@secondoftwo
    \fi
  \endgroup
  {\ifsi@num@signmant
     \expandafter\si@num@asign
   \else
     \expandafter\@gobble
   \fi
   {mantissa}}
  {\ifsi@num@signexp
     \expandafter\si@num@asign
   \else
     \expandafter\@gobble
   \fi
   {exponent}}}
\newcommand*{\si@num@asign}[1]{%
  \let\si@num@sign\si@sign
  \si@log@debug{Adding sign \si@sign\space to #1 for
    `\si@num@arg'}}
%    \end{macrocode}
%\end{macro}
%\end{macro}
%\begin{macro}{\si@num@procnum}
%\darg{either \texttt{mant} or \texttt{exp}}
% The control macro for processing the number (plus any extra
% characters).
%    \begin{macrocode}
\newcommand*{\si@num@procnum}[1]{%
  \expandafter\ifx\expandafter\@empty
    \csname si@num@#1\endcsname\@empty
    \expandafter\@gobble
  \else
    \expandafter\si@num@finddigits
  \fi
  {#1}}
%    \end{macrocode}
%\end{macro}
%\begin{macro}{\si@num@predec}
%\begin{macro}{\si@num@postdec}
% Two new storage areas are defined.
%    \begin{macrocode}
\newcommand*{\si@num@predec}{}
\newcommand*{\si@num@postdec}{}
%    \end{macrocode}
%\end{macro}
%\end{macro}
%\begin{macro}{\si@num@finddigits}
%\darg{either \texttt{mant} or \texttt{exp}}
% The core digit processor divides the number into the parts before
% and after the decimal point marker.  The temporary switch is used
% to indicate finding a decimal marker.
%    \begin{macrocode}
\newcommand*{\si@num@finddigits}[1]{%
  \renewcommand*{\si@num@predec}{}%
  \renewcommand*{\si@num@postdec}{}%
  \si@switchfalse
  \expandafter\expandafter\expandafter\si@num@digits
    \csname si@num@#1\endcsname\@empty\@empty
%    \end{macrocode}
% Tests are now made to see if padding zeros are needed.  The
% trailing test needs to verify if a decimal marker was found, as
% well as if a zero is needed.
%    \begin{macrocode}
  \ifx\@empty\si@num@predec\@empty
    \ifsi@num@padlead
      \expandafter\expandafter\expandafter\si@num@addprezero
    \fi
  \fi
  \ifx\@empty\si@num@postdec\@empty
    \ifsi@num@padtrail
      \ifsi@switch
        \expandafter\expandafter\expandafter\expandafter
          \expandafter\expandafter\expandafter
          \si@num@addpostzero
      \fi
    \fi
  \fi
%    \end{macrocode}
% The next checks to make concern input validity in a more
% mathematical sense.  First, if the number is zero, then no sign
% should be given under any circumstances.  Then leading zeros need
% to be removed from the input.  This is slightly complicated by the
% potential presence of ``extra'' characters.
%    \begin{macrocode}
  \si@num@unsign{#1}%
  \ifx\@empty\si@num@predec\@empty
  \else
    \expandafter\si@num@nozero
  \fi
%    \end{macrocode}
% A sanity check is made to ensure that the supplied number consisted
% of more than a decimal marker.
%    \begin{macrocode}
  \ifx\@empty\si@num@predec\@empty
    \ifx\@empty\si@num@postdec\@empty
      \expandafter\expandafter\expandafter\@gobble
    \else
      \expandafter\expandafter\expandafter\si@num@sepdigits
    \fi
  \else
    \expandafter\si@num@sepdigits
  \fi
  {#1}}
%    \end{macrocode}
%\end{macro}
%\begin{macro}{\si@num@digits}
% The \cs{si@num@digits} macro compares each character in the input
% against the list of characters valid at this stage: numbers,
% decimal markers and ``extra'' characters.
%    \begin{macrocode}
\def\si@num@digits#1#2\@empty{%
  \si@str@ifchrstr{#1}{\si@numdecimal}
    {\ifsi@switch
       \si@log@err{Duplicate decimal marker in `\si@num@arg'}
         {Only a single decimal marker (from the list
            `\si@numdecimal')\MessageBreak may occur in a
            numerical argument}%
     \else
       \si@log@debug{Found decimal marker `#1' in
         `\si@num@arg'}%
       \expandafter\si@switchtrue
     \fi}
%    \end{macrocode}
% The earlier code only checks for a sign at the start of the text. A
% check is therefore needed for a sign after the first two
% characters; if one is found, it is ignored.
%    \begin{macrocode}
    {\si@str@ifchrstr{#1}{\si@numsign}
       {\si@log@err{Misplaced sign character
          `#1' in `\si@num@arg'}
          {Sign characters `\si@numsign' can only
            occur\MessageBreak at the start of a number}}
%    \end{macrocode}
% The current character is added to the appropriate stack; this is
% ``spun out'' to avoid problems with expansion of the switch code.
%    \begin{macrocode}
       {\ifsi@switch
          \expandafter\si@num@post
        \else
          \expandafter\si@num@pre
        \fi
        {#1}}}%
  \ifx\@empty#2\@empty\else
    \si@num@digits#2\@empty\@empty
  \fi}
%    \end{macrocode}
%\end{macro}
%\begin{macro}{\si@num@pre}
%\begin{macro}{\si@num@post}
%\darg{char}
% Storage of the result is spun out.
%    \begin{macrocode}
\newcommand*{\si@num@pre}[1]{%
  \si@num@prepost{#1}{pre}{integer}}
\newcommand*{\si@num@post}[1]{%
  \si@num@prepost{#1}{post}{decimal}}
%    \end{macrocode}
%\end{macro}
%\end{macro}
%\begin{macro}{\si@num@prepost}
%\darg{char}
%\darg{either \texttt{pre} or \texttt{post}}
%\darg{info-text}
% Then actually stored here.
%    \begin{macrocode}
\newcommand*{\si@num@prepost}[3]{%
  \expandafter\protected@edef\csname si@num@#2dec\endcsname{%
    \csname si@num@#2dec\endcsname#1}%
  \si@log@debug{Adding `#1' to #3 part for `\si@num@arg'}}
%    \end{macrocode}
%\end{macro}
%\begin{macro}{\si@num@addprezero}
%\begin{macro}{\si@num@addpostzero}
% A similar set of macros are used for the padding zeros.
%    \begin{macrocode}
\newcommand*{\si@num@addprezero}{%
  \si@num@addpzero{pre}{leading}}
\newcommand*{\si@num@addpostzero}{%
  \si@num@addpzero{post}{trailing}}
%    \end{macrocode}
%\end{macro}
%\end{macro}
%\begin{macro}{\si@num@addpzero}
%\darg{either \texttt{pre} or \texttt{post}}
% In this case, there is no argument to be passed along.
%    \begin{macrocode}
\newcommand*{\si@num@addpzero}[2]{%
  \si@log@debug{Adding #2 zero for `\si@num@arg'}%
  \@namedef{si@num@#1dec}{0}}
%    \end{macrocode}
%\end{macro}
%\begin{macro}{\si@num@unsign}
%\begin{macro}{\si@num@nosign}
%\darg{either \texttt{mant} or \texttt{exp}}
% The trap for a sign with zero numerical input is made.  First, a
% check is made to see if there is a sign to worry about.  The pre-
% and post-decimal parts are then examined, to see if they contain
% something other than ``0'' or an extra character.
%    \begin{macrocode}
\newcommand*{\si@num@unsign}[1]{%
  \expandafter\ifx\expandafter\@empty
    \csname si@num@#1sign\endcsname\@empty
    \expandafter\@gobble
  \else
    \expandafter\si@num@nosign
  \fi
  {#1}}
\newcommand*{\si@num@nosign}[1]{%
  \begingroup
    \si@switchtrue
    \si@str@ifonlychrs{\si@num@predec\si@num@postdec}{0}
      {\si@switchfalse}{}%
    \ifsi@switch
      \aftergroup\@gobble
    \else
      \aftergroup\@firstofone
    \fi
  \endgroup
  {\si@log@debug{Zero value: removing any sign}%
   \ifsi@ang@sign\else
     \@namedef{si@num@#1sign}{}%
   \fi}}
%    \end{macrocode}
%\end{macro}
%\end{macro}
%\begin{macro}{\si@num@nozero}
%  A very short test for a totally zero pre-decimal component.
%    \begin{macrocode}
\newcommand*{\si@num@nozero}{%
  \si@str@ifonlychrs{\si@num@predec}{0}
    {\renewcommand*{\si@num@predec}{0}}{}}
%    \end{macrocode}
%\end{macro}
%\begin{macro}{\si@num@decimalhook}
% A hook is needed to attach things inside the group to happen
% afterwards, if the number is a decimal.
%    \begin{macrocode}
\newcommand*{\si@num@decimalhook}{}
%    \end{macrocode}
%\end{macro}
%\begin{macro}{\si@num@sepdigits}
%\darg{either \texttt{mant} or \texttt{exp}}
% The \cs{si@num@sepdigits} macro is only called if at least one of
% the mantissa and exponent contain something to output.
%    \begin{macrocode}
\newcommand*{\si@num@sepdigits}[1]{%
%    \end{macrocode}
% First an overall check is needed for additional characters.  By
% altering the contents of \cs{si@numextra}, the same code can be
% shared by two different checks.
%    \begin{macrocode}
  \begingroup
    \let\si@numextra\si@numaddn
    \protected@edef\si@tempa{\si@num@predec\si@num@postdec}%
    \si@num@ifextra{\si@tempa}
      {\aftergroup\@gobble}
      {\aftergroup\@firstofone}%
  \endgroup
%    \end{macrocode}
% Separation of the error in a number from the number itself is only
% attempted for the mantissa.
%    \begin{macrocode}
  {\renewcommand*{\si@tempb}{mant}%
   \renewcommand*{\si@tempc}{#1}%
   \ifx\si@tempb\si@tempc
     \expandafter\si@num@checkerr
   \fi}%
%    \end{macrocode}
%\changes{v1.1}{2008/08/22}{Fixed-length code generalised}
%\changes{v1.1}{2008/08/26}{Length-fixing only applies to
%  manitssa}
% If both parts of the number contain only digits, then any rounding
% can be attempted if there is no error part.  Otherwise, the input
% must be left alone.
%    \begin{macrocode}
  \protected@edef\si@tempa{\si@num@predec\si@num@postdec}%
  \expandafter\si@str@ifonlychrs\expandafter{\si@tempa}
    {0123456789}
    {\ifx\@empty\si@num@err\@empty
       \renewcommand*{\si@tempb}{mant}%
       \renewcommand*{\si@tempc}{#1}%
       \ifx\si@tempb\si@tempc
         \expandafter\expandafter\expandafter\si@num@fixlength
       \fi
     \fi}{}%
%    \end{macrocode}
% If the pre-decimal part contains nothing except numbers, then
% digit separation is carried out.
%    \begin{macrocode}
  \si@num@ifextra{\si@num@predec}{}
    {\expandafter\si@num@int\expandafter{\si@num@predec}}%
%    \end{macrocode}
% A decision is made about the decimal sign, then digit separation
% occurs on the post-decimal part of the number.
%    \begin{macrocode}
  \renewcommand*{\si@tempc}{}%
  \ifx\@empty\si@num@postdec\@empty\else
    \si@num@decimalhook
    \renewcommand*{\si@tempc}{%
      \ensuremath{{\si@decimalsymbol}}}%
    \si@num@ifextra{\si@num@postdec}{}
      {\expandafter\si@num@dec\expandafter{\si@num@postdec}}%
  \fi
%    \end{macrocode}
% The construction is finalised by re-combining the number.
%    \begin{macrocode}
  \expandafter\protected@edef\csname si@num@#1\endcsname
    {\si@num@predec\si@tempc\si@num@postdec}}
%    \end{macrocode}
%\end{macro}
%\begin{macro}{\si@num@ifextra}
%\begin{macro}{\si@num@extra}
%\darg{number}
% A relatively simple test for ``extra'' characters.  Once again, a
% bit of group trickery is used.
%    \begin{macrocode}
\newcommand*{\si@num@ifextra}[1]{%
  \begingroup
    \si@switchfalse
    \expandafter\si@num@extra#1\@empty\@empty
    \ifsi@switch
      \si@log@debug{Found `extra' characters in `#1'}%
      \aftergroup\@firstoftwo
    \else
      \aftergroup\@secondoftwo
    \fi
  \endgroup}
\def\si@num@extra#1#2\@empty{%
  \ifx\@empty#1\@empty\else
    \si@str@ifchrstr{#1}{\si@numextra}{\si@switchtrue}{}%
    \ifx\@empty#2\@empty\else
      \si@num@extra#2\@empty\@empty
    \fi
  \fi}
%    \end{macrocode}
%\end{macro}
%\end{macro}
%\begin{macro}{\ifsi@num@ambigerr}
%\begin{macro}{\si@num@checkerr}
% When separating out an error from a number, the first step is to
% see if there is a decimal part to the number.  If so, any error
% must be in that part of the decimal part; it is not possible to
% have an error starting in the integer part and continuing into the
% decimal part.
%    \begin{macrocode}
\newif\ifsi@num@ambigerr
\newcommand*{\si@num@checkerr}{%
  \ifx\@empty\si@num@postdec\@empty
    \expandafter\si@num@preerr
  \else
    \expandafter\si@num@posterr
  \fi}
%    \end{macrocode}
%\end{macro}
%\end{macro}
%\begin{macro}{\si@num@preerr}
% Errors in integers are easy to handle.  After finding the error,
% the result is simply stored in the error macro \cs{si@num@err}.
%    \begin{macrocode}
\newcommand*{\si@num@preerr}{%
  \si@num@seperr{pre}%
  \ifx\@empty\si@tempb\@empty\else
    \expandafter\renewcommand\expandafter*\expandafter
      \si@num@err\expandafter{\si@tempb}%
  \fi}
%    \end{macrocode}
%\end{macro}
%\begin{macro}{\si@num@posterr}
%\begin{macro}{\si@num@psterr}
% Life is more complex for an error in the decimal part.  This is
% found, then it may need zero-filling or the insertion of a decimal
% point.  This depends on whether the number of error digits is
% larger than the number of post-decimal digits.
%    \begin{macrocode}
\newcommand*{\si@num@posterr}{%
  \si@num@seperr{post}%
  \ifx\@empty\si@tempb\@empty\else
    \ifsi@seperr
      \expandafter\expandafter\expandafter\si@num@psterr
    \else
      \let\si@num@err\si@tempb
    \fi
  \fi}
\newcommand*{\si@num@psterr}{%
    \si@num@cntdigits{\si@tempb}%
    \si@tempcntb\si@tempcnta\relax
    \si@num@cntdigits{\si@num@postdec}%
    \ifnum\si@tempcnta<\si@tempcntb\relax
      \expandafter\si@num@largeerr
    \else
      \expandafter\si@num@smallerr
    \fi}
%    \end{macrocode}
%\end{macro}
%\end{macro}
%\begin{macro}{\si@num@seperr}
%\begin{macro}{\si@num@finderr}
%\darg{either \texttt{pre} or \texttt{post}}
% The usual hand-off is made for searching for an error.
%    \begin{macrocode}
\newcommand*{\si@num@seperr}[1]{%
  \si@switchfalse
  \renewcommand*{\si@tempa}{}%
  \renewcommand*{\si@tempb}{}%
  \expandafter\expandafter\expandafter\si@num@finderr
    \csname si@num@#1dec\endcsname\@empty\@empty
  \ifx\@empty\si@tempb\@empty\else
    \expandafter\let\csname si@num@#1dec\endcsname\si@tempa
  \fi}
\def\si@num@finderr#1#2\@empty{%
%    \end{macrocode}
% First a check for the opening character of an error.
%    \begin{macrocode}
  \si@str@ifchrstr{#1}{\si@numopenerr}
%    \end{macrocode}
% If the switch is set when an error is found, then something is
% wrong.
%    \begin{macrocode}
    {\ifsi@switch
       \si@log@err{Invalid error in number}
         {The numerical argument \si@num@arg\space has two (or
           more)\MessageBreak error-opening characters}%
     \else
       \expandafter\si@switchtrue
     \fi}
%    \end{macrocode}
% The end-of-error character has to be the last item of the input,
% and an error needs to start before it ends.
%    \begin{macrocode}
    {\si@str@ifchrstr{#1}{\si@numcloseerr}
       {\ifsi@switch
          \ifx\@empty#2\@empty\else
            \si@log@err{Invalid error in number}
              {The numerical argument \si@num@arg\space has an
              error-closing before the last character}%
          \fi
        \else
          \si@log@err{Invalid error in number}
           {The numerical argument \si@num@arg\space has an
              error-closing character\MessageBreak but no
              error-opening one}%
        \fi}
%    \end{macrocode}
% The input must be a digit.  It is stored in the appropriate area.
%    \begin{macrocode}
       {\ifsi@switch
          \expandafter\si@num@addtmpb
        \else
          \expandafter\si@num@addtmpa
        \fi
        {#1}}}%
  \ifx\@empty#2\@empty\else
    \si@num@finderr#2\@empty
  \fi}
%    \end{macrocode}
%\end{macro}
%\end{macro}
%\begin{macro}{\si@num@addtmpa}
%\begin{macro}{\si@num@addtmpb}
%\darg{number}
% Some quick methods for adding to the temporary macros with \cs{if}
% expansion.
%    \begin{macrocode}
\newcommand*{\si@num@addtmpa}[1]{\si@num@addtmp{a}{#1}}
\newcommand*{\si@num@addtmpb}[1]{\si@num@addtmp{b}{#1}}
%    \end{macrocode}
%\end{macro}
%\end{macro}
%\begin{macro}{\si@num@addtmp}
%\darg{either \texttt{a} or \texttt{b}}
%\darg{number}
% The addition is committed.
%    \begin{macrocode}
\newcommand*{\si@num@addtmp}[2]{%
  \expandafter\protected@edef\csname si@temp#1\endcsname{%
    \csname si@temp#1\endcsname#2}}
%    \end{macrocode}
%\end{macro}
%\begin{macro}{\si@num@cntdigits}
%\begin{macro}{\si@num@cntdgt}
%\darg{number}
% A recursive system for counting the number of characters in a given
% input; only used here to count digits in a number.
%    \begin{macrocode}
\newcommand*{\si@num@cntdigits}[1]{%
  \si@tempcnta\z@\relax
  \expandafter\si@num@cntdgt#1\@empty\@empty}
\def\si@num@cntdgt#1#2\@empty{%
  \ifx\@empty#1\@empty\else
    \advance\si@tempcnta\@ne\relax
  \fi
  \ifx\@empty#2\@empty\else
    \expandafter\si@num@cntdgt#2\@empty
  \fi}
%    \end{macrocode}
%\end{macro}
%\end{macro}
%\begin{macro}{\si@num@smallerr}
%\begin{macro}{\si@num@serr}
% For handling an error with no more digits than the post-decimal
% part of a number, zero-padding is undertaken before adding a
% decimal sign and (possibly) leading zero.
%    \begin{macrocode}
\newcommand*{\si@num@smallerr}{%
  \si@tempcntb\si@tempcnta\relax
  \si@num@serr
  \protected@edef\si@num@err{%
    \ifsi@num@padlead0\fi\expandafter\@car\si@numdecimal\@nil
    \si@tempb}}
\newcommand*{\si@num@serr}{%
  \si@num@cntdigits{\si@tempb}%
  \ifnum\si@tempcnta=\si@tempcntb\relax\else
    \protected@edef\si@tempb{0\si@tempb}%
    \expandafter\si@num@serr
  \fi}
%    \end{macrocode}
%\end{macro}
%\end{macro}
%\begin{macro}{\si@num@largeerr}
%\changes{v1.0m}{2008/09/16}{Fixed problem with error separation
%  crossing decimal boundary}
%\begin{macro}{\si@num@lerr}
%\begin{macro}{\si@num@movedigit}
% When the error has more digits than the decimal part, some
% shuffling is needed.
%    \begin{macrocode}
\newcommand*{\si@num@largeerr}{%
  \renewcommand*{\si@tempa}{}%
  \si@tempcntb\si@tempcnta\relax
  \si@num@lerr
  \protected@edef\si@num@err{%
    \si@tempa\expandafter\@car\si@numdecimal\@nil\si@tempb}}
\newcommand*{\si@num@lerr}{%
  \si@num@cntdigits{\si@tempb}%
  \ifnum\si@tempcnta=\si@tempcntb\relax\else
    \expandafter\si@num@movedigit\si@tempb\@empty\@empty
    \si@num@lerr
  \fi}
\def\si@num@movedigit#1#2\@empty{%
  \protected@edef\si@tempa{\si@tempa#1}%
  \protected@edef\si@tempb{#2}}
%    \end{macrocode}
%\end{macro}
%\end{macro}
%\end{macro}
%\begin{macro}{\si@num@fixlength}
% The checks for fixing number length are spun out here.
%    \begin{macrocode}
\newcommand*{\si@num@fixlength}{%
  \ifsi@fixdp
    \expandafter\si@num@fixdp
  \else
    \ifsi@fixsf
      \expandafter\expandafter\expandafter\si@num@fixsf
    \fi
  \fi}
%    \end{macrocode}
%\end{macro}
%\begin{macro}{\si@num@fixdp}
% The test for fixing decimal places starts by counting up the number
% of decimal digits.  The number is then padded, rounded or left
% alone.
%    \begin{macrocode}
\newcommand*{\si@num@fixdp}{%
  \si@num@cntdigits{\si@num@postdec}%
  \ifx\@empty\si@num@postdec\@empty
    \si@tempcnta\z@\relax
  \fi
  \ifnum\si@tempcnta>\si@num@dp\relax
    \expandafter\si@num@round
  \else
    \ifnum\si@tempcnta<\si@num@dp\relax
      \expandafter\expandafter\expandafter\si@num@pad
    \fi
  \fi}
%    \end{macrocode}
%\end{macro}
%\begin{macro}{\si@num@pad}
%\begin{macro}{\si@num@pd}
% Padding a number is a relatively easy matter.  The number of digits
% is increased by adding ``$0$'' recursively.
%    \begin{macrocode}
\newcommand*{\si@num@pad}{%
  \si@log@debug{Padding to \the\si@num@dp\space digits}%
  \loop\ifnum\si@tempcnta<\si@num@dp\si@num@pd\repeat}
\newcommand*{\si@num@pd}{%
  \advance\si@tempcnta\@ne\relax
  \protected@edef\si@num@postdec{\si@num@postdec0}}
%    \end{macrocode}
%\end{macro}
%\end{macro}
%\begin{macro}{\si@num@rndtarget}
%\changes{v1.1}{2008/08/25}{New counter}
% A scratch counter is created to allow some code sharing between
% fixing decimal places and significant figures.
%    \begin{macrocode}
\newcount\si@num@rndtarget
%    \end{macrocode}
%\end{macro}
%\begin{macro}{\si@num@round}
%\begin{macro}{\si@num@prernd}
%\begin{macro}{\si@num@postrnd}
%\begin{macro}{\si@num@rndstart}
%\changes{v1.1}{2008/08/25}{Separated out from \cs{si@num@round}}
% Rounding a number is more complicated.  The main macro here relies
% on several others, and is simply a set-up and hand-off system.
%    \begin{macrocode}
\newcommand*{\si@num@prernd}{}
\newcommand*{\si@num@postrnd}{}
\newcommand*{\si@num@round}{%
  \si@log@debug{Rounding to \the\si@num@dp\space digits}%
  \si@num@rndtarget\si@num@dp\relax
  \si@num@rndstart}
\newcommand*{\si@num@rndstart}{%
  \si@num@reverse{\si@num@postdec}%
  \si@num@reverse{\si@num@predec}%
  \let\si@num@prernd\si@num@predec
  \let\si@num@postrnd\si@num@postdec
  \renewcommand*{\si@num@predec}{}%
  \renewcommand*{\si@num@postdec}{}%
  \si@switchfalse
  \si@num@rnd}
%    \end{macrocode}
%\end{macro}
%\end{macro}
%\end{macro}
%\end{macro}
%\begin{macro}{\si@num@reverse}
%\begin{macro}{\si@num@rev}
%\darg{storage-macro}
% The initial stage is to reverse the entire number, to make life
% easier.  This is then undone by the other macros as the output is
% constructed.
%    \begin{macrocode}
\newcommand*{\si@num@reverse}[1]{%
  \renewcommand*{\si@tempa}{}%
  \expandafter\si@num@rev#1\@empty\@empty
  \let#1\si@tempa}
\def\si@num@rev#1#2\@empty{%
  \edef\si@tempa{#1\si@tempa}%
  \ifx\@empty#2\@empty\else
    \si@num@rev#2\@empty\@empty
  \fi}
%    \end{macrocode}
%\end{macro}
%\end{macro}
%\begin{macro}{\si@num@rnd}
%\changes{v1.1}{2008/08/25}{Altered check for ending to generalise
%  code}
%\begin{macro}{\si@num@rndpre}
%\begin{macro}{\si@num@rndpost}
%\changes{v1.0b}{2008/06/23}{Corrected bug in rounding code}
% The core looping macro of the system simply divides the flow
% between rounding before and after the decimal.  The two routines
% are quite similar, but have subtly different requirements.  Both
% take one character at a time from the input, increment if there is
% a carry digit, check its value, and add to the output string. This
% is very similar to the same routines used in other macro packages
% to achieve the same thing.
%    \begin{macrocode}
\newcommand*{\si@num@rnd}{%
  \ifx\@empty\si@num@postrnd\@empty
    \expandafter\si@num@rndpre
  \else
    \expandafter\si@num@rndpost
  \fi}
\newcommand*{\si@num@rndpre}{%
  \expandafter\edef\expandafter\si@tempa\expandafter{%
    \expandafter\@car\si@num@prernd\@nil}%
  \expandafter\edef\expandafter\si@num@prernd\expandafter{%
    \expandafter\@cdr\si@num@prernd\@nil}%
  \si@tempcntb\si@tempa\relax
  \ifsi@switch
    \advance\si@tempcntb\@ne\relax
  \fi
  \si@switchfalse
  \ifnum\si@tempcntb=10\relax
    \si@tempcntb\z@\relax
    \expandafter\expandafter\expandafter\si@switchtrue
  \fi
  \edef\si@num@predec{\the\si@tempcntb\si@num@predec}%
  \ifx\@empty\si@num@prernd\@empty
    \ifsi@switch
      \edef\si@num@predec{1\si@num@predec}%
    \fi
  \else
    \expandafter\si@num@rnd
  \fi}
\newcommand*{\si@num@rndpost}{%
  \expandafter\edef\expandafter\si@tempa\expandafter{%
    \expandafter\@car\si@num@postrnd\@nil}%
  \expandafter\edef\expandafter\si@num@postrnd\expandafter{%
    \expandafter\@cdr\si@num@postrnd\@nil}%
  \si@tempcntb\si@tempa\relax
  \ifsi@switch
    \advance\si@tempcntb\@ne\relax
  \fi
  \si@switchfalse
  \advance\si@num@rndtarget\@ne\relax
  \ifnum\si@tempcnta>\si@num@rndtarget\relax
    \advance\si@num@rndtarget\m@ne\relax
  \else
    \advance\si@num@rndtarget\m@ne\relax
    \ifnum\si@tempcnta>\si@num@rndtarget\relax
      \ifnum\si@tempcntb>4\relax
        \expandafter\expandafter\expandafter\si@switchtrue
      \fi
    \else
      \ifnum\si@tempcntb=10\relax
        \si@tempcntb\z@\relax
        \expandafter\expandafter\expandafter\si@switchtrue
      \fi
      \edef\si@num@postdec{\the\si@tempcntb\si@num@postdec}%
    \fi
  \fi
  \advance\si@tempcnta\m@ne\relax
%    \end{macrocode}
% If \cs{si@tempcnta} is negative, this must be the significant
% figures routine and one digit needs to be thrown away.
%    \begin{macrocode}
  \ifnum\si@tempcnta=\m@ne\relax
    \si@num@throw
  \fi
  \si@num@rnd}
%    \end{macrocode}
%\end{macro}
%\end{macro}
%\end{macro}
%\begin{macro}{\si@num@throw}
% Throw one digit away from the end of the post-decimal part.
%    \begin{macrocode}
\newcommand*{\si@num@throw}{%
  \si@num@reverse{\si@num@postdec}%
  \edef\si@num@postdec{\expandafter\@cdr\si@num@postdec\@nil}%
  \si@num@reverse{\si@num@postdec}}
%    \end{macrocode}
%\end{macro}
%\begin{macro}{\si@num@fixsf}
% Setting up a fixed number of significant figures is rather more
% complicated than fixing the decimal places.  There are a number of
% tests needed to see what needs to be done.
%    \begin{macrocode}
\newcommand*{\si@num@fixsf}{%
  \si@num@cntdigits{\si@num@predec}%
  \ifx\@empty\si@num@predec\@empty
    \si@tempcnta\z@\relax
  \fi
%    \end{macrocode}
% If the pre-decimal component has only a single digit, it could be
% ``0'': if so, only the significant figures in the decimal part are
% counted.
%    \begin{macrocode}
  \ifnum\si@tempcnta=\@ne\relax
    \renewcommand*{\si@tempa}{0}%
    \ifx\si@num@predec\si@tempa
%    \end{macrocode}
% The number of significant digits after the decimal point have to be
% counted up. If there are enough digits there, then some rounding
% will again be needed.
%    \begin{macrocode}
      \ifx\@empty\si@num@postdec\@empty\else
        \si@tempcnta\z@\relax
        \si@switchfalse
        \expandafter\si@num@cntpostsf\si@num@postdec
          \@empty\@empty
        \ifnum\si@tempcnta>\si@num@sf\relax
          \si@log@debug{Rounding to \the\si@num@sf\space
            figures}%
          \si@num@rndtarget\si@num@sf\relax
          \si@num@rndstart
        \fi
      \fi
    \else
      \si@num@cntprepost
    \fi
  \else
%    \end{macrocode}
% If there are too many digits in the pre-decimal part, a check is
% first made to see if there may be a need to round from the first
% post-decimal digit. If not, the post decimal part is thrown away.
%    \begin{macrocode}
    \ifnum\si@tempcnta<\si@num@sf\relax
      \si@num@cntprepost
    \else
      \advance\si@tempcnta\m@ne\relax
      \si@switchfalse
      \ifnum\si@tempcnta>\si@num@sf\relax\else
        \ifx\@empty\si@num@postdec\@empty\else
          \edef\si@num@postdec{\expandafter\@car
            \si@num@postdec\@nil}%
          \ifnum\si@num@postdec>4\relax
            \expandafter\si@switchtrue
          \fi
        \fi
      \fi
      \advance\si@tempcnta\@ne\relax
      \renewcommand*{\si@num@postdec}{}%
      \si@num@reverse{\si@num@predec}%
      \let\si@num@prernd\si@num@predec
      \renewcommand*{\si@num@predec}{}%
      \si@num@sfpre
      \ifsi@switch
        \edef\si@num@predec{1\si@num@predec}%
      \fi
    \fi
  \fi}
%    \end{macrocode}
%\end{macro}
%\begin{macro}{\si@num@cntpostsf}
% Counting up significant figures in the post-decimal part needs an
% auxiliary macro.
%    \begin{macrocode}
\def\si@num@cntpostsf#1#2\@empty{%
  \ifsi@switch
    \advance\si@tempcnta\@ne\relax
  \else
    \if 0#1\relax\else
      \si@tempcnta\@ne\relax
      \expandafter\expandafter\expandafter\si@switchtrue
    \fi
  \fi
  \ifx\@empty#2\@empty\else
    \si@num@cntpostsf#2\@empty\@empty
  \fi}
%    \end{macrocode}
%\end{macro}
%\begin{macro}{\si@num@cntprepost}
% There are two routes to find that the number of digits in both
% parts of the number are important.  In either case, rounding is
% only needed if the two parts have too many digits.
%    \begin{macrocode}
\newcommand*{\si@num@cntprepost}{%
  \si@tempcntb\si@tempcnta\relax
  \si@num@cntdigits{\si@num@postdec}%
  \advance\si@tempcnta\si@tempcntb\relax
  \ifnum\si@tempcnta>\si@num@sf
    \si@log@debug{Rounding to \the\si@num@sf\space figures}%
    \si@num@rndtarget\si@num@sf\relax
    \si@num@rndstart
    \ifsi@switch
      \si@num@throw
    \fi
  \fi}
%    \end{macrocode}
%\end{macro}
%\begin{macro}{\si@num@sfpre}
% The code here is very similar to that used to round to a fixed
% number of decimal places.
%    \begin{macrocode}
\newcommand*{\si@num@sfpre}{%
  \expandafter\edef\expandafter\si@tempa\expandafter{%
    \expandafter\@car\si@num@prernd\@nil}%
  \expandafter\edef\expandafter\si@num@prernd\expandafter{%
    \expandafter\@cdr\si@num@prernd\@nil}%
  \si@tempcntb\si@tempa\relax
  \ifsi@switch
    \advance\si@tempcntb\@ne\relax
  \fi
  \si@switchfalse
  \advance\si@num@sf\@ne\relax
  \ifnum\si@tempcnta>\si@num@sf\relax
    \advance\si@num@sf\m@ne\relax
    \edef\si@num@predec{0\si@num@predec}%
  \else
    \advance\si@num@sf\m@ne\relax
    \ifnum\si@tempcnta>\si@num@sf\relax
      \ifnum\si@tempcntb>4\relax
        \expandafter\expandafter\expandafter\si@switchtrue
      \fi
      \edef\si@num@predec{0\si@num@predec}%
    \else
      \ifnum\si@tempcntb=10\relax
        \si@tempcntb\z@\relax
        \expandafter\expandafter\expandafter\si@switchtrue
      \fi
      \edef\si@num@predec{\the\si@tempcntb\si@num@predec}%
    \fi
  \fi
  \advance\si@tempcnta\m@ne\relax
  \ifx\@empty\si@num@prernd\@empty\else
    \expandafter\si@num@sfpre
  \fi}
%    \end{macrocode}
%\end{macro}
%
%\begin{macro}{\si@num@int}
%\darg{integer-part}
% The formatting code for separating thousands is taken more-or-less
% directly from \SIstyle.  A few changes are made to fit the various
% conventions here.  Following on from the code above, \cs{si@tempa}
% is used to store the integer part of the number, and \cs{si@tempb}
% is used for the decimal part.
%    \begin{macrocode}
\newcommand*{\si@num@int}[1]{%
  \renewcommand*{\si@num@predec}{}%
  \ifsi@sepfour
    \si@num@intfmt{}#1\@empty\@empty\@empty
  \else
    \si@num@iffive{#1}
      {\si@num@intfmt{}#1\@empty\@empty\@empty}
      {\renewcommand*{\si@num@predec}{#1}}%
  \fi}
%    \end{macrocode}
%\end{macro}
%\begin{macro}{\si@num@iffive}
%\begin{macro}{\si@num@five}
%\darg{number}
% A test is needed for the presence of more than four characters.
%    \begin{macrocode}
\newcommand*{\si@num@iffive}[1]{%
  \si@num@five#1\@empty\@empty\@empty\@empty\@empty\end}
\def\si@num@five#1#2#3#4#5\end{%
   \ifx\@empty#5\@empty
      \expandafter\@secondoftwo
   \else
      \expandafter\@firstoftwo
   \fi}
%    \end{macrocode}
%\end{macro}
%\end{macro}
%\begin{macro}{\si@num@intfmt}
%\begin{macro}{\si@num@fiint}
% The business end of the integer formatter.
%    \begin{macrocode}
\newcommand*{\si@num@intfmt}[4]{%
  \ifx\@empty#2\@empty
    \si@num@intsep#1\relax
  \else
    \ifx\@empty#3\@empty
      \si@num@intsep\@empty\@empty#1#2\relax
    \else
      \ifx\@empty#4\@empty
        \si@num@intsep\@empty#1#2#3\relax
      \else
        \si@num@fiint{#1#2#3#4}%
      \fi
    \fi
  \fi}
\def\si@num@fiint#1\fi\fi\fi{\fi\fi\fi\si@num@intfmt{#1}}
%    \end{macrocode}
%\end{macro}
%\end{macro}
%\begin{macro}{\si@num@intsep}
% For adding separation to integers, an extra function is needed.
%    \begin{macrocode}
\newcommand*{\si@num@intsep}[4]{%
  \protected@edef\si@num@predec{\si@num@predec#1#2#3}%
  \if\relax#4\relax\else
    \protected@edef\si@num@predec{%
      \si@num@predec\ensuremath{\noexpand\si@digitsep}}%
    \expandafter\si@num@intsep\expandafter#4%
  \fi}
%    \end{macrocode}
%\end{macro}
%\begin{macro}{\si@num@dec}
%\begin{macro}{\si@num@decfmt}
%\darg{decimal-part}
% Formatting a decimal uses a similar mechanism, but with a few
% alterations needed.
%    \begin{macrocode}
\newcommand*{\si@num@dec}[1]{%
  \renewcommand*{\si@num@postdec}{}%
  \ifsi@sepfour
    \si@num@decfmt#1\@empty\@empty\@empty\@empty
  \else
    \si@num@iffive{#1}
      {\si@num@decfmt#1\@empty\@empty\@empty\@empty}
      {\protected@edef\si@num@postdec{\si@num@postdec#1}}%
  \fi}
\newcommand*{\si@num@decfmt}[4]{%
  \protected@edef\si@num@postdec{\si@num@postdec#1#2#3}%
  \ifx\@empty#4\@empty%
  \else
    \protected@edef\si@num@postdec{%
      \si@num@postdec\ensuremath{\noexpand\si@digitsep}}%
    \expandafter\si@num@decfmt\expandafter#4%
  \fi}
%    \end{macrocode}
%\end{macro}
%\end{macro}
%\begin{macro}{\si@num@procerr}
% Any error is recycled to to formatted correctly.  The $\pm$ sign,
% and any unit, are also added.
%    \begin{macrocode}
\newcommand*{\si@num@procerr}{%
  \si@num@addunit
  \ensuremath{\si@pm}%
  \expandafter\si@num\expandafter{\si@num@err}}
%    \end{macrocode}
%\end{macro}
%\begin{macro}{\si@num@sepxpart}
% A similar approach for numbers containing products, except the
% first token of the input has to be deleted.
%    \begin{macrocode}
\newcommand*{\si@num@sepxpart}{%
  \si@num@addunit
  \ensuremath{\times}%
  \expandafter\expandafter\expandafter\si@num\expandafter
    \expandafter\expandafter{%
      \expandafter\@cdr\si@num@xpart\@nil}}
%    \end{macrocode}
%\end{macro}
%\begin{macro}{\si@num@addunit}
% A short macro to add units if needed.
%    \begin{macrocode}
\newcommand*{\si@num@addunit}{%
  \si@unt@numtrue
  \ifx\@empty\si@unt@unitarg\@empty\else
    \ifsi@repeatunits
      \si@unt@printunit{\si@unt@unitarg}%
    \fi
  \fi}
%    \end{macrocode}
%\end{macro}
%
%\subsection{Formatting angles}
%\begin{macro}{\ang}
%\doarg{keyval options}
%\darg{angle, either in decimal or deg;min;sec format}
% The approach used here is similar to that in \SIstyle, but has been
% modified in a few ways.
%    \begin{macrocode}
\si@newrobustcmd*{\ang}[2][]{%
  \begingroup
    \sisetup{#1}%
    \si@fam@mode
    \si@log@debug{Processing \string\ang\space input `#2'}%
    \@makeother{\;}%
    \makeatletter
    \scantokens{\si@ang@parse#2;;;\@nil}}
%    \end{macrocode}
%\end{macro}
%\begin{macro}{\si@ang@parse}
% With the correct catcodes in place, processing can take place strip
% out the semi-colons.  Here, the input must either contain no
% semi-colons or two semi-colons.
%    \begin{macrocode}
\def\si@ang@parse#1;#2;#3;#4\@nil{%
  \let\ifsi@ang@fixdp\ifsi@fixdp
  \si@fixdpfalse
  \si@ifmtarg{#4}
    {\si@log@debug{Angle argument contains no
       semi-colons:\MessageBreak decimal angle}%
     \si@ang@dec{#1}{}{}}
    {\si@log@debug{Angle argument contains
       semi-colons:\MessageBreak degree-minute-second angle}%
     \renewcommand*{\si@tempa}{#4}%
     \renewcommand*{\si@tempb}{;;}%
     \ifx\si@tempa\si@tempb\else
       \ifsi@strictarc
         \renewcommand*{\si@tempb}{;}%
         \ifx\si@tempa\si@tempb
           \si@log@err{Insufficient semi-colons in argument
             of \string\ang}{The argument of
             \string\ang\space  must contain either no
             semi-colons or exactly two}%
         \else
           \si@log@err{Excess semi-colons in argument of
             \string\ang}{The argument of \string\ang\space
             must contain either no semi-colons or exactly
             two}%
         \fi
       \fi
     \fi
     \si@ang@arc{#1}{#2}{#3}}}
%    \end{macrocode}
%\end{macro}
%\begin{macro}{\si@ang@dec}
%\begin{macro}{\si@ang@arc}
% Two tests are needed, in case the input format requires conversion.
%    \begin{macrocode}
\newcommand*{\si@ang@dec}{%
  \let\si@ang@fix\@gobble
  \ifsi@ang@toarc
    \expandafter\si@ang@dectoarc
  \else
    \sisetup{padangle=none}\expandafter\si@ang@typeset
  \fi}
\newcommand*{\si@ang@arc}{%
  \let\si@ang@fix\si@ang@arcfix
  \ifsi@ang@todec
    \expandafter\si@ang@arctodec
  \else
    \expandafter\si@ang@typeset
  \fi}
%    \end{macrocode}
%\end{macro}
%\end{macro}
%\begin{macro}{\ifsi@ang@fixdp}
%\begin{macro}{\si@ang@fix}
%\begin{macro}{\si@ang@arcfix}
%\darg{number}
% A check is needed so that rounding and zero filling are only
% applied to the seconds of an arc angle.
%    \begin{macrocode}
\newif\ifsi@ang@fixdp
\newcommand*{\si@ang@fix}[1]{}
\newcommand*{\si@ang@arcfix}[1]{%
  \renewcommand*{\si@tempa}{second}%
  \renewcommand*{\si@tempb}{#1}%
  \ifx\si@tempa\si@tempb
    \ifsi@ang@fixdp
      \expandafter\expandafter\expandafter\si@fixdptrue
    \else
      \expandafter\expandafter\expandafter\si@fixdpfalse
    \fi
  \else
    \expandafter\si@fixdpfalse
  \fi}
%    \end{macrocode}
%\end{macro}
%\end{macro}
%\end{macro}
%\begin{macro}{\si@ang@ifnum}
%\darg{number}
% A test is required to check that the provided data consists of
% numbers which can actually be processed by \TeX.  This is achieved
% by using \cs{si@num@ifvalid} with a fixed list of valid
% characters.
%    \begin{macrocode}
\newcommand*{\si@ang@ifnum}[1]{%
  \begingroup
    \renewcommand*{\si@numvalid}{0123456789,.+-}%
    \ifx\@empty#1\@empty
      \aftergroup\@firstoftwo
    \else
      \si@num@ifvalid{#1}
        {\aftergroup\@firstoftwo}
        {\aftergroup\@secondoftwo}%
    \fi
  \endgroup}
%    \end{macrocode}
%\end{macro}
%\begin{macro}{\si@ang@arctodec}
%\darg{degrees}
%\darg{minutes}
%\darg{seconds}
% The business end of converting arcs to decimal angles is relatively
% straight-forward, as it only needs one calculation.  This has to be
% divided up, so that disaster does not strike if there are empty
% arguments; a check is also needed for the sign of the angle,
% so that the maths makes sense.
%    \begin{macrocode}
\newcommand*{\si@ang@arctodec}[3]{%
  \let\si@ang@fix\@gobble
  \ifnum\si@num@dp>\thr@@\relax
    \si@num@dp\thr@@\relax
  \fi
  \si@fixdptrue
  \si@ang@ifnum{#1}
    {\si@ang@ifnum{#2}
      {\si@ang@ifnum{#3}
        {\si@tempdima\z@\relax
         \renewcommand*{\si@tempa}{+}%
         \ifx\@empty#1\@empty\else
           \si@tempdima #1pt\relax
         \fi
         \ifdim\si@tempdima<\z@\relax
           \renewcommand*{\si@tempa}{-}%
         \fi
         \ifx\@empty#2\@empty\else
           \si@tempdima\dimexpr\si@tempdima\si@tempa
             #2pt/60\relax
         \fi
         \ifdim\si@tempdima<\z@\relax
           \renewcommand*{\si@tempa}{-}%
         \else
         \fi
         \ifx\@empty#3\@empty\else
           \si@tempdima\dimexpr\si@tempdima\si@tempa
             #3pt/3600\relax
         \fi
         \sisetup{numdecimal=.}%
         \expandafter\si@ang@typeset\expandafter{%
           \strip@pt\si@tempdima}{}{}}
        {\si@ang@notnum{#1}{#2}{#3}}}
      {\si@ang@notnum{#1}{#2}{#3}}}
    {\si@ang@notnum{#1}{#2}{#3}}}
%    \end{macrocode}
%\end{macro}
%\begin{macro}{\si@ang@dectoarc}
%\begin{macro}{\si@ang@arcdeg}
%\begin{macro}{\si@ang@arcmin}
%\begin{macro}{\si@ang@arcsec}
%\darg{number}
% The conversion macros for decimal to arc angles.  Life is more
% complex here than above, even without the need for checks on the
% input.  A number of separation steps are needed, each of which
% needs a separate macro.
%    \begin{macrocode}
\newcommand*{\si@ang@dectoarc}[1]{%
  \let\si@ang@fix\si@ang@arcfix
  \si@ang@fixdptrue
  \ifnum\si@num@dp>\@ne\relax
    \si@num@dp\@ne\relax
  \fi
  \si@ang@ifnum{#1}
    {\si@tempdima\z@\relax
     \ifx\@empty#1\@empty\else
       \si@tempdima #1pt\relax
     \fi
     \si@ang@sepint{deg}%
     \si@tempdima\dimexpr\si@tempdima *60\relax
     \si@ang@sepint{min}%
     \edef\si@tempa{\the\dimexpr\si@tempdima *60\relax}%
     \expandafter\newcommand\expandafter*\expandafter{%
       \expandafter\si@ang@arcsec\expandafter}\expandafter{%
       \expandafter\si@ang@strippt\si@tempa}%
%    \end{macrocode}
% A check is made for ``0.0'' seconds, which should be converted to
% simply ``0''.
%    \begin{macrocode}
     \si@tempdima\z@\relax
     \edef\si@tempa{\the\si@tempdima}%
     \expandafter\renewcommand\expandafter*\expandafter{%
       \expandafter\si@tempa\expandafter}\expandafter{%
       \expandafter\si@ang@strippt\si@tempa}%
     \ifx\si@tempa\si@ang@arcsec
       \renewcommand*{\si@ang@arcsec}{0}%
     \fi
%    \end{macrocode}
% To avoid adding zeros where they are not required, each part of the
% angle is now checked.
%    \begin{macrocode}
     \renewcommand*{\si@tempa}{0}%
     \ifx\si@ang@arcdeg\si@tempa
       \si@temptoks{{}}%
     \else
       \si@temptoks{{\si@ang@arcdeg}}%
     \fi
     \ifx\si@ang@arcmin\si@tempa
       \si@temptoks\expandafter{\the\si@temptoks{}}%
     \else
       \si@temptoks\expandafter{\the\si@temptoks{%
         \si@ang@arcmin}}%
     \fi
     \ifx\si@ang@arcsec\si@tempa
       \si@temptoks\expandafter{\the\si@temptoks{}}%
     \else
       \si@temptoks\expandafter{\the\si@temptoks{%
         \si@ang@arcsec}}%
     \fi
     \expandafter\si@ang@typeset\the\si@temptoks}
    {\si@ang@notnum{#1}{}{}}}
%    \end{macrocode}
%\end{macro}
%\end{macro}
%\end{macro}
%\end{macro}
%\begin{macro}{\si@ang@sepint}
%\begin{macro}{\si@ang@sint}
%\begin{macro}{\si@ang@strippt}
%\darg{either \texttt{deg} or \texttt{min}}
% Support macros for the conversion from decimal to arc angles.
%    \begin{macrocode}
\newcommand*{\si@ang@sepint}[1]{%
  \expandafter\si@ang@sint\the\si@tempdima\@empty
  \expandafter\let\csname si@ang@arc#1\endcsname\si@tempa}
\def\si@ang@sint#1.#2\@empty{%
  \renewcommand*{\si@tempa}{#1}%
  \si@tempdima 0.#2\relax}
\begingroup
  \catcode`P=12
  \catcode`T=12
  \lowercase{
    \renewcommand*{\si@tempa}{%
      \def\si@ang@strippt##1PT{##1}}}
  \expandafter\endgroup
\si@tempa
%    \end{macrocode}
%\end{macro}
%\end{macro}
%\end{macro}
%\begin{macro}{\si@ang@notnum}
%\darg{degrees}
%\darg{minutes}
%\darg{seconds}
% Not a \TeX\ number: complain.
%    \begin{macrocode}
\newcommand*{\si@ang@notnum}[3]{%
  \si@log@warn{Angle `#1;#2;#3' is not a pure
    number:\MessageBreak output will be as given}%
  \si@ang@typeset{#1}{#2}{#3}}
%    \end{macrocode}
%\end{macro}
%\begin{macro}{\ifsi@ang@sign}
% A flag is needed to leave signs along for angles, where
% a zero value may still need a sign.
%    \begin{macrocode}
\newif\ifsi@ang@sign
%    \end{macrocode}
%\end{macro}
%\begin{macro}{\si@ang@typeset}
%\darg{degrees}
%\darg{minutes}
%\darg{seconds}
% The \cs{si@ang@set} macro does the work of assigning the degrees,
% minutes and seconds, and actually typesetting the result.
%    \begin{macrocode}
\newcommand*{\si@ang@typeset}[3]{%
%    \end{macrocode}
%\begin{macro}{\si@ang@degs}
%\begin{macro}{\si@ang@mins}
%\begin{macro}{\si@ang@secs}
% First, the three macros that will contain the measures must exist.
%    \begin{macrocode}
  \ifsi@ang@padlarge
    \newcommand*{\si@ang@degs}{0\si@sym@degree}%
    \newcommand*{\si@ang@mins}{0\si@sym@minute}%
    \newcommand*{\si@ang@secs}{0\si@sym@second}%
  \else
    \newcommand*{\si@ang@degs}{}%
    \newcommand*{\si@ang@mins}{}%
    \newcommand*{\si@ang@secs}{}%
  \fi
%    \end{macrocode}
%\begin{macro}{\si@ang@decimalsymbol}
% The current definition of \cs{si@decimalsymbol} needs to be saved.
%    \begin{macrocode}
  \protected@edef\si@ang@decimalsymbol{\si@decimalsymbol}%
%    \end{macrocode}
%\end{macro}
%\begin{macro}{\si@ang@movesign}
% Either the signs need to be moved, or this needs to be killed off.
%    \begin{macrocode}
  \ifsi@astroang
    \let\si@ang@movesign\si@ang@astrosign
  \else
    \let\si@ang@movesign\@gobble
  \fi
%    \end{macrocode}
%\end{macro}
%\begin{macro}{\si@ang@secnum}
%\begin{macro}{\si@ang@minnum}
% The arguments are now examined in reverse order.  If they are
% empty, then nothing is done.  Otherwise, the larger measures are
% zero-filled, if this has been requested. Some steps are needed
% to allow for addition of signs to numbers.
%    \begin{macrocode}
  \newcommand*{\si@ang@secnum}{\si@ang@num{second}}%
  \newcommand*{\si@ang@minnum}{\si@ang@num{minute}}%
   \si@ifnotmtarg{#3}
     {\si@log@debug{Found seconds `#3'}%
      \si@ang@ifnum{#3}
        {\ifdim #3 pt=\z@\relax\else
           \si@ang@signtrue
         \fi}{}%
      \renewcommand*{\si@ang@secs}
       {\si@ang@secnum{#3}\si@sym@second}%
      \renewcommand*{\si@ang@mins}
        {\si@ang@pad{0\si@sym@minute}}%
      \renewcommand*{\si@ang@degs}
        {\si@ang@pad{0\si@sym@degree}}}%
   \si@ifnotmtarg{#2}
     {\si@log@debug{Found minutes `#2'}%
      \si@ang@ifnum{#2}
        {\ifdim #2 pt=\z@\relax\else
           \si@ang@signtrue
         \fi}{}%
      \renewcommand*{\si@ang@secnum}{%
        \si@ang@signlessnum{second}}%
      \renewcommand*{\si@ang@mins}
       {\si@ang@minnum{#2}\si@sym@minute}%
      \renewcommand*{\si@ang@degs}
        {\si@ang@pad{0\si@sym@degree}}}%
   \si@ifnotmtarg{#1}
     {\si@log@debug{Found degrees `#1'}%
      \renewcommand*{\si@ang@secnum}{%
        \si@ang@signlessnum{second}}%
      \renewcommand*{\si@ang@minnum}{%
        \si@ang@signlessnum{minute}}%
      \renewcommand*{\si@ang@degs}
%    \end{macrocode}
% The group here is needed to get the mechanism to move the symbol to
% work properly.
%    \begin{macrocode}
       {\si@ang@num{degree}{#1}%
        \si@sym@degree}}%
   \si@out@num
     {\si@ang@degs\si@anglesep\si@ang@mins\si@anglesep
        \si@ang@secs}%
%    \end{macrocode}
%\end{macro}
%\end{macro}
% The group opened by \cs{ang} is closed.
%    \begin{macrocode}
   \endgroup}
%    \end{macrocode}
%\end{macro}
%\end{macro}
%\end{macro}
%\end{macro}
%\begin{macro}{\si@ang@pad}
%\darg{number}
% Padding is only added if requested; the zero is a literal.
%    \begin{macrocode}
\newcommand*{\si@ang@pad}[1]{\ifsi@ang@padsmall #1\fi}
%    \end{macrocode}
%\end{macro}
%\begin{macro}{\si@ang@num}
%\begin{macro}{\si@ang@signlessnum}
%\darg{one of \texttt{degree}, \texttt{minute} or \texttt{second}}
%\darg{number}
% Modified versions of \cs{num}, one to typeset angles without a
% leading sign and the other with.
%    \begin{macrocode}
\newcommand*{\si@ang@num}[2]{%
  \begingroup
    \si@ang@fix{#1}%
    \si@ang@movesign{#1}%
    \si@num{#2}%
  \endgroup}
\newcommand*{\si@ang@signlessnum}[2]{%
  \begingroup
    \si@ang@fix{#1}%
    \si@ang@movesign{#1}%
    \sisetup{addsign=none}%
    \si@num{#2}%
  \endgroup}
%    \end{macrocode}
%\end{macro}
%\end{macro}
%\begin{macro}{\si@ang@killdegree}
%\begin{macro}{\si@ang@killminute}
%\begin{macro}{\si@ang@killsecond}
% A mechanism is needed to handle moving the angle unit signs for the
% \opt{astroang} option.  First, some support macros are needed.
%    \begin{macrocode}
\newcommand*{\si@ang@killdegree}{\let\si@sym@degree\relax}
\newcommand*{\si@ang@killminute}{\let\si@sym@minute\relax}
\newcommand*{\si@ang@killsecond}{\let\si@sym@second\relax}
%    \end{macrocode}
%\end{macro}
%\end{macro}
%\end{macro}
%\begin{macro}{\si@ang@astrosign}
%\darg{one of \texttt{degree}, \texttt{minute} or \texttt{second}}
% The method needs two steps, producing the sign over the decimal
% sign and preventing duplicate symbols appearing. This is based on a
% suggestion from Morten H{\o}gholm, but using \TeX\ internals as
% \cs{makebox} does not work here.  Note the need to correct for
% \cs{scriptspace} (thanks to Donald Arseneau for that).
%    \begin{macrocode}
\newcommand*{\si@ang@astrosign}[1]{%
  \renewcommand*{\si@decimalsymbol}{%
    \setbox\si@tempboxa=\hbox{%
      \ensuremath{{\si@ang@decimalsymbol}}}%
    \si@tempdima\wd\si@tempboxa\relax
    \setbox\si@tempboxb=\hbox to\z@{%
      \hss\unhbox\si@tempboxa\hss}%
    \setbox\si@tempboxa=\hbox{%
      \csname si@sym@#1\endcsname\hskip-\scriptspace}%
    \si@tempdimb\wd\si@tempboxa\relax
    \setbox\si@tempboxc=\hbox to\z@{%
      \hss\unhbox\si@tempboxa\hss}%
    \setbox\si@tempboxd=\hbox{%
      \usebox\si@tempboxb\usebox\si@tempboxc}%
    \ifdim\si@tempdima>\si@tempdimb\relax
      \setbox\si@tempboxa=\hbox to\si@tempdima{%
        \hss\unhbox\si@tempboxd\hss}%
    \else
      \setbox\si@tempboxa=\hbox to\si@tempdimb{%
        \hss\unhbox\si@tempboxd\hss}%
    \fi
    \usebox\si@tempboxa%
    \ifdim\si@tempdima>\si@tempdimb\relax\else
      \hskip\scriptspace
    \fi}%
  \renewcommand*{\si@num@decimalhook}{\expandafter\aftergroup
    \csname si@ang@kill#1\endcsname}}%
%    \end{macrocode}
%\end{macro}
%
%\subsection{Tabular material}
% The automatic formatting and alignment of numerical data in columns
% is handled here.  The various other packages that work in this area
% are basically ripped-off here.  The letters |D|, |N| and |R| are
% already taken by the other packages for numerical alignment, and so
% |S| ($=$ \currpkg) is chosen for the alignment of numerical
% material.  The package also provides a second column type, |s|, for
% units (the letter is taken from \cs{si}).
%
%\begin{macro}{\NC@list}
%\changes{v1.1e}{2008/11/18}{Fixed issue with \pkg{cellspace} package}
% The first part of the job is to create the basic apparatus for the
% columns using the \pkg{array} package.  To prevent any issues with
% the content of optional arguments to the new columns, so
% rearrangement is needed.  The \currpkg columns have to come
% \emph{before} any other column definitions.  This is achieved by
% saving \cs{NC@list}, creating the columns then restoring the list
% with the appropriate extra parts.
%    \begin{macrocode}
\@ifpackageloaded{cellspace}
  {\newcolumntype{C}[1]{>{\bcolumn #1\@nil}#1<{\ecolumn}}}
  {\AtBeginDocument{
    \@ifpackageloaded{cellspace}
     {\newcolumntype{C}[1]{>{\bcolumn #1\@nil}#1<{\ecolumn}}%
      \renewcommand*{\NC@rewrite@S}[1][]{%
        \edef\si@tempa{\the\@temptokena
        >{\noexpand\si@tab@begin@S[#1]}c%
        <{\noexpand\si@tab@end@S}}%
        \@temptokena\expandafter{\si@tempa}%
        \NC@find}}
     {}}}
\edef\si@tempa{%
  \noexpand\NC@do S\noexpand\NC@do s\the\NC@list}
\newcolumntype{S}{}
\newcolumntype{s}{}
\NC@list\expandafter{\si@tempa}
%    \end{macrocode}
%\end{macro}
%\changes{v1.1h}{2008/11/30}{Fixed problem with \pkg{colortbl} package}
%\changes{v1.1i}{2008/11/30}{Better solution to \pkg{colortbl} issue}
%\begin{macro}{\si@tab@corr@colortbl}
% A correction if \pkg{colortbl} is loaded.
%    \begin{macrocode}
\newcommand*{\si@tab@corr@colortbl}{\relax}
\AtBeginDocument{
  \@ifpackageloaded{colortbl}{%
    \renewcommand*{\si@tab@corr@colortbl}{%
      \hskip\stretch{-0.5}\kern\z@
    }
  }{}
}
%    \end{macrocode}
%\end{macro}
%\begin{macro}{\NC@rewrite@S}
%\begin{macro}{\NC@rewrite@s}
%\doarg{keyval options}
% Following the \numprint approach, the \cs{NC@rewrite@\ldots} macros
% are rewritten to collect the cell contents.  This means messing
% with the internal macros of another package, but there is no other
% way to do this.  As \pkg{array} is a standard package from the
% \pkg{tools} bundle, this should be reasonably safe. Here the begin
% and end code needed is added to the existing list if
% \cs{@temptokena}, with the start and end macros unexpanded.
% Argument |#1| contains any user setup options for this column.
% Passing an argument at this stage will cause issues, so each column
% type needs its own \texttt{begin} and \texttt{end} macros.
%    \begin{macrocode}
\renewcommand*{\NC@rewrite@S}[1][]{%
  \edef\si@tempa{\the\@temptokena
    >{\noexpand\si@tab@begin@S[#1]}c%
    <{\noexpand\si@tab@end@S}}%
  \@temptokena\expandafter{\si@tempa}%
  \NC@find}
\renewcommand*{\NC@rewrite@s}[1][]{%
  \edef\si@tempa{\the\@temptokena
    >{\noexpand\si@tab@begin@s[#1]}c%
    <{\noexpand\si@tab@end@s}}%
  \@temptokena\expandafter{\si@tempa}%
  \NC@find}
%    \end{macrocode}
%\end{macro}
%\end{macro}
%\begin{macro}{\si@tab@begin@S}
%\begin{macro}{\si@tab@begin@s}
%\begin{macro}{\si@tab@gettok}
%\doarg{keyval options}
% At this stage, the appropriate token gathering macro is activated,
% and the common starting macro is called.  For the |S| column, the
% \opt{seperr} is turned off, and an error is set to be raised by any
% ``x-part'' input.
%    \begin{macrocode}
\newcommand*{\si@tab@begin@S}[1][]{%
  \si@log@debug{Processing S column cell contents}%
  \let\si@tab@gettok\si@tab@gettok@S
  \si@seperrfalse
  \renewcommand*{\si@num@sepxpart}{%
    \si@log@err{Multiple numbers not allowed in
       tables\MessageBreak Only the first number used}
       \@ehb}%
  \si@tab@begin[#1]}
\newcommand*{\si@tab@begin@s}[1][]{%
  \si@log@debug{Processing s column cell contents}%
  \let\si@tab@gettok\si@tab@gettok@s
  \si@tab@begin[#1]}
%    \end{macrocode}
%\end{macro}
%\end{macro}
%\end{macro}
%\begin{macro}{\si@tab@toks}
%\begin{macro}{\si@tab@pretoks}
%\begin{macro}{\si@tab@posttoks}
% Some storage is needed for the data to build up.  In common with
% \pkg{rccol} and \numprint, token registers are used for this (thus
% leaving problematic input to be handled later).
%    \begin{macrocode}
\newtoks\si@tab@toks
\newtoks\si@tab@pretoks
\newtoks\si@tab@posttoks
%    \end{macrocode}
%\end{macro}
%\end{macro}
%\end{macro}
%\begin{macro}{\si@tab@begin}
%\doarg{keyval options}
% The macro for gathering up input is common to both column types. It
% uses the method for \pkg{rccol}; the cell contents are collected by
% a second macro, which then stores all of the data in an appropriate
% token store.
%    \begin{macrocode}
\newcommand*{\si@tab@begin}[1][]{%
  \begingroup
    \sisetup{#1}%
    \si@tab@toks{}%
    \si@tab@pretoks{}%
    \si@tab@posttoks{}%
    \si@switchfalse
    \si@tab@gettok}
%    \end{macrocode}
%\end{macro}
%\begin{macro}{\si@tab@next}
% A macro is needed for recursion when collecting cell contents.
%    \begin{macrocode}
\newcommand*{\si@tab@next}{}
%    \end{macrocode}
%\end{macro}
%\begin{macro}{\si@tab@gettok@S}
%\changes{v1.0j}{2008/08/21}{Token gathering code now checks for
%  expandable content}
%\darg{token}
% The cell-content collection system is inherited from \pkg{rccol}.
% First, the current input is compared with a fixed list of macros
% which need custom handling.
%    \begin{macrocode}
\newcommand*{\si@tab@gettok@S}[1]{%
  \ifx\tabularnewline#1\relax
    \let\si@tab@next\si@tab@newline@S
  \else
    \ifx\end#1\relax
      \let\si@tab@next\end
    \else
      \ifx\si@tab@end@S#1\relax
        \let\si@tab@next\si@tab@end@S
      \else
        \ifx\endtabular#1\relax
          \let\si@tab@next\endtabular
        \else
          \ifx\csname#1\relax
            \let\si@tab@next\csname
          \else
            \ifx\relax#1\relax
              \let\si@tab@next\relax
            \else
%    \end{macrocode}
% If the code gets to here, then recursion should occur.
%    \begin{macrocode}
              \let\si@tab@next\si@tab@gettok@S
%    \end{macrocode}
% Cells always contain \cs{ignorespaces} and \cs{unskip}. These can
% both be filtered out here and are saved to either the pre-number or
% post-number token store, as appropriate. This is done without
% fixing the destination in case the user adds addition copies to the
% input.
%    \begin{macrocode}
              \ifx\ignorespaces#1\relax
                \si@tab@othertok{#1}%
              \else
                \ifx\unskip#1\relax
                  \si@tab@othertok{#1}%
                \else
%    \end{macrocode}
% If the current token is a valid numerical character, then the flag
% is set and the input is saved into the ``processing'' token
% register.
%    \begin{macrocode}
                  \si@str@ifchrstr{#1}{\si@numvalid}
                     {\si@switchtrue
                      \si@log@debug{Found numerical cell
                        contents `#1'}%
                      \si@tab@toks=\expandafter{%
                        \the\si@tab@toks#1}}
%    \end{macrocode}
% For non-numerical content, some testing is needed. A check is made
% to see if the current ``token'' is a single item. If it was wrapped
% in one set of braces before being picked up here, they will have
% been removed. So the input may now have more than one character to
% it. If it is a single token, testing for expendability first
% requires a check for another set of braces.  This is handed off
% using \cs{futurelet}.
%    \begin{macrocode}
                     {\si@tab@ifonechar{#1}
                       {\futurelet\si@tempa\si@tab@bracetest#1}%
                       {\si@tab@othertok{#1}}}%
                \fi
              \fi
            \fi
          \fi
        \fi
      \fi
    \fi
  \fi
  \si@tab@next}
%    \end{macrocode}
%\end{macro}
%\begin{macro}{\si@tab@othertok}
%\darg{token}
% Tokens which are not part of a number can occur before or after a
% number. This is all sorted out here, with some debugging
% information for good measure.
%    \begin{macrocode}
\newcommand*{\si@tab@othertok}[1]{%
  \si@log@debug{Found other cell contents `\unexpanded{#1}'}%
  \ifsi@switch
    \si@log@debug{Adding to post-numeral part}%
    \si@tab@posttoks=\expandafter{\the\si@tab@posttoks#1}%
  \else
    \si@log@debug{Adding to pre-numeral part}%
    \si@tab@pretoks=\expandafter{\the\si@tab@pretoks#1}%
  \fi}
%    \end{macrocode}
%\end{macro}
%\begin{macro}{\si@tab@ifonechar}
%\darg{token}
%\begin{macro}{\si@tab@onechar}
% A quick check for a single character.
%    \begin{macrocode}
\newcommand*{\si@tab@ifonechar}[1]{\si@tab@onechar#1\@empty}
\def\si@tab@onechar#1#2\@empty{%
  \ifx\@empty#2\@empty
    \expandafter\@firstoftwo
  \else
    \expandafter\@secondoftwo
  \fi}
%    \end{macrocode}
%\end{macro}
%\end{macro}
%\begin{macro}{\si@tab@bracetest}
% For the \cs{futurelet} test, a separate macro is needed. If the
% current token starts with a brace, then it is added to the
% ``other'' stack.  If not, a test can be made on its definition.
%    \begin{macrocode}
\newcommand*{\si@tab@bracetest}{%
  \ifx\si@tempa\bgroup
    \expandafter\si@tab@othertok
  \else
    \expandafter\si@tab@exptest
  \fi}
%    \end{macrocode}
%\end{macro}
%\begin{macro}{\si@tab@exptest}
%\begin{macro}{\si@tab@meaning}
%\begin{macro}{\si@tab@strip}
%\darg{token}
% The meaning of the token is compared to that for a short macro. If
% the two are of the same type, expansion is attempted. In this case,
% the current pre-number tokens are also typeset.
%    \begin{macrocode}
\newcommand*{\si@tab@exptest}[1]{%
  \edef\si@tempa{\si@tab@meaning{#1}}%
  \ifx\si@tempa\si@tab@macro
    \si@log@debug{Expanding cell contents `\unexpanded{#1}'}%
    \the\si@tab@pretoks
    \si@tab@pretoks{}%
    \protected@edef\si@tab@next{\noexpand\si@tab@gettok@S#1}%
  \else
    \si@tab@othertok{#1}%
  \fi}
\newcommand*{\si@tab@meaning}[1]{%
  \expandafter\si@tab@strip\meaning#1:->:->\@empty}
\def\si@tab@strip#1:->#2:->#3\@empty{#1}
%    \end{macrocode}
%\end{macro}
%\end{macro}
%\end{macro}
%\begin{macro}{\si@tab@macro}
% A supporter macro for the above which always expands to ``macro''
% with everything with category code 12.
%    \begin{macrocode}
\newcommand*{\si@tab@macro}{}
\edef\si@tab@macro{\si@tab@meaning{\si@tab@macro}}
%    \end{macrocode}
%\end{macro}
%\begin{macro}{\si@tab@gettok@s}
%\changes{v1.0a}{2008/06/20}{Fixed issues with alignment of
%  \texttt{s} column contents}
%\darg{token}
% The situation for the |s| column is less complex, as there is only
% one store and expansion is not an issue.
%    \begin{macrocode}
\newcommand*{\si@tab@gettok@s}[1]{%
  \ifx\tabularnewline#1\relax
    \let\si@tab@next\si@tab@newline@s
  \else
    \ifx\end#1\relax
      \let\si@tab@next\end
    \else
      \ifx\si@tab@end@s#1\relax
        \let\si@tab@next\si@tab@end@s
      \else
        \ifx\endtabular#1\relax
          \let\si@tab@next\endtabular
        \else
          \ifx\csname#1\relax
            \let\si@tab@next\csname
          \else
            \ifx\relax#1\relax
              \let\si@tab@next\relax
            \else
              \let\si@tab@next\si@tab@gettok@s
              \ifx\ignorespaces#1\relax\else
                \ifx\unskip#1\relax\else
                  \si@tab@toks=\expandafter{%
                    \the\si@tab@toks#1}%
                  \si@log@debug{Found cell contents `#1'}%
                \fi
              \fi
            \fi
          \fi
        \fi
      \fi
    \fi
  \fi
  \si@tab@next}
%    \end{macrocode}
%\end{macro}
%\begin{macro}{\si@tab@end@S}
%\changes{v1.0a}{2008/06/20}{Fixed mixed number/text column
%  alignment}
%\changes{v1.0k}{2008/09/02}{Fixed problem with coloured cells}
%\begin{macro}{\si@tab@rfill}
%\begin{macro}{\si@tab@lfill}
%\begin{macro}{\si@tab@end@s}
% The end macros are similar, but with some minor differences. In
% both cases, the appropriate filling is carried out.  For the |S|
% column, this depends on the cell contents, whereas for the |s|
% column the fill is always the same.  Output of a number in an |S|
% column is only attempted if one was found, otherwise the cell
% contents will all be in \cs{si@tab@pretoks}.
%    \begin{macrocode}
\newcommand*{\si@tab@end@S}{%
    \ifsi@switch
      \let\si@tab@lfill\si@tab@lfill@S
      \let\si@tab@rfill\si@tab@rfill@S
    \else
      \let\si@tab@rfill\si@tab@rfill@t
      \let\si@tab@lfill\si@tab@lfill@t
    \fi
    \si@tab@corr@colortbl
    \hskip\stretch{\si@tab@lfill}\kern\z@
    \ifsi@switch
      \the\si@tab@pretoks
      \si@tab@numout
      \the\si@tab@posttoks
    \else
      \the\si@tab@pretoks
    \fi
    \hskip\stretch{\si@tab@rfill}\kern\z@
    \si@tab@corr@colortbl
  \endgroup}
\newcommand*{\si@tab@end@s}{%
    \si@tab@corr@colortbl
    \hskip\stretch{\si@tab@lfill@s}\kern\z@ 
    \ignorespaces
    \expandafter\si\expandafter{\the\si@tab@toks}%
    \unskip
    \hskip\stretch{\si@tab@rfill@s}\kern\z@ 
    \si@tab@corr@colortbl
  \endgroup}
%    \end{macrocode}
%\end{macro}
%\end{macro}
%\end{macro}
%\end{macro}
%\begin{macro}{\si@tab@newline@S}
%\begin{macro}{\si@tab@newline@s}
%\begin{macro}{\si@tab@end}
% If the column is the final one read, then some work is needed with
% the \cs{tabularnewline} macro.  Output has to happen \emph{before}
% the new line, then the ending macro is made safe before calling the
% \LaTeX\ line end.  If the user makes use of the primitive \cs{cr}
% then this problem does not arise as \cs{si@tab@end@\ldots} is
% called correctly.
%    \begin{macrocode}
\newcommand*{\si@tab@newline@S}{%
  \si@tab@end@S
  \hfil\relax
  \let\si@tab@end\si@tab@end@S
  \renewcommand*{\si@tab@end@S}{\let\si@tab@end@S\si@tab@end}%
  \tabularnewline}
\newcommand*{\si@tab@newline@s}{%
  \si@tab@end@s
  \hfil\relax
  \let\si@tab@end\si@tab@end@s
  \renewcommand*{\si@tab@end@s}{\let\si@tab@end@s\si@tab@end}%
  \tabularnewline}
%    \end{macrocode}
%\end{macro}
%\end{macro}
%\end{macro}
%
%\begin{macro}{\si@tempcnta}
%\begin{macro}{\si@tempcntb}
% The second part of the tabular code is concerned with typesetting
% numbers in |S| columns with the appropriate alignment. Counters are
% needed for the digit-counting system.
%    \begin{macrocode}
\newcount\si@tempcnta
\newcount\si@tempcntb
%    \end{macrocode}
%\end{macro}
%\end{macro}
%\begin{macro}{\si@tab@numout}
%\changes{v1.1}{2008/09/03}{Extended for non-aligned output}
% If a number is found, then some secondary processing is needed to
% format it correctly.
%    \begin{macrocode}
\newcommand*{\si@tab@numout}{%
  \si@num@intabtrue
  \ifsi@tabparseonly\else
    \ifsi@tab@fixed
      \ifsi@tabautofit
        \si@num@dp\si@tab@mantpostcnt\relax
        \expandafter\expandafter\expandafter\expandafter
          \expandafter\expandafter\expandafter\si@fixdptrue
      \fi
    \fi
  \fi
  \expandafter\si@num\expandafter{\the\si@tab@toks}%
  \si@tab@format}
%    \end{macrocode}
%\end{macro}
%\begin{macro}{\si@tab@prebox}
%\begin{macro}{\si@tab@postbox}
%\begin{macro}{\si@tab@midbox}
%\begin{macro}{\si@tab@expbox}
% The various boxes needed for the column centring are declared
% Unlike the \pkg{dcolumn} original, private boxes are used here.
% \cs{si@tempboxa} is used when a space to measure one of the
% constituents is needed; it is never used for output.
%    \begin{macrocode}
\newbox\si@tab@prebox
\newbox\si@tab@midbox
\newbox\si@tab@postbox
\newbox\si@tab@expbox
%    \end{macrocode}
%\end{macro}
%\end{macro}
%\end{macro}
%\end{macro}
%\begin{macro}{\si@tab@format}
%\changes{v1.1}{2008/09/03}{Extended for non-aligned output}
% The formatting set up is taken from \pkg{dcolumn}, but with
% control of the output form the stored information.  The choice
% of a variable (decimal-centred) column or fixed width boxes is
% made.
%    \begin{macrocode}
\newcommand*{\si@tab@format}{%
  \ifsi@tabparseonly
    \expandafter\si@tab@outonly
  \else
    \ifsi@tab@fixed
      \expandafter\expandafter\expandafter\si@tab@fixed
    \else
      \expandafter\expandafter\expandafter\si@tab@unfixed
    \fi
  \fi
%    \end{macrocode}
% A hack to get the correct colour everywhere without too much work.
%    \begin{macrocode}
  \ifsi@colourvalues
    \si@fam@colourcmd{\si@valuecolour}%
  \fi
  \box\si@tab@prebox\box\si@tab@midbox\box\si@tab@postbox
    \box\si@tab@expbox}
%    \end{macrocode}
%\end{macro}
%\begin{macro}{\si@tab@outonly}
% The non-alignment option means that no real processing is done: only font
% control is required.
%    \begin{macrocode}
\newcommand*{\si@tab@outonly}{%
  \si@log@debug{Using non-aligned S column}%
  \setbox\si@tab@prebox=\hbox{}%
  \setbox\si@tab@midbox=\hbox{}%
  \setbox\si@tab@postbox=\hbox{}%
  \setbox\si@tab@expbox=\hbox{}%
    \protected@edef\si@tempa{%
      \si@tab@out
    \ifx\@empty\si@num@out\@empty\else
      \ensuremath{{\si@decimalsymbol}}%
    \fi
    \si@num@out
    \si@tab@expout}%
  \expandafter\si@out@num\expandafter{\si@tempa}}
%    \end{macrocode}
%\end{macro}
%\begin{macro}{\si@tab@unfixed}
%\changes{v1.0l}{2008/09/03}{Fixed issue with decimal sign and
%  integers}
% When the width of the contents is not fixed, the system creates a
% block in which the decimal marker is always at the centre. This is
% achieved by placing the pre- and post-decimal parts of the number
% in boxes.  The wider one is then used to set up the column width,
% by resizing the other one.
%    \begin{macrocode}
\newcommand*{\si@tab@unfixed}{%
  \si@log@debug{Using variable width S column}%
  \ifx\@empty\si@num@out\@empty
    \setbox\si@tab@midbox=\hbox
      {\phantom{\ensuremath{{\si@decimalsymbol}}}}%
  \else
    \setbox\si@tab@midbox=\hbox
      {\ensuremath{{\si@decimalsymbol}}}%
  \fi
  \protected@edef\si@num@out{\si@num@out\si@tab@expout}%
  \setbox\si@tab@prebox=\hbox
    {\expandafter\si@out@num\expandafter{\si@tab@out}}%
  \setbox\si@tab@postbox=\hbox
    {\expandafter\si@out@num\expandafter{\si@num@out}}%
  \ifdim\wd\si@tab@prebox>\wd\si@tab@postbox\relax
    \setbox\si@tab@postbox=\hbox to\wd\si@tab@prebox%
      {\unhbox\si@tab@postbox\hfill}%
  \else
    \setbox\si@tab@prebox=\hbox to\wd\si@tab@postbox%
      {\hfill\unhbox\si@tab@prebox}%
  \fi
  \setbox\si@tab@expbox=\hbox{}}
%    \end{macrocode}
%\end{macro}
%\begin{macro}{\si@tab@predim}
%\begin{macro}{\si@tab@postdim}
%\begin{macro}{\si@tab@expdim}
%\begin{macro}{\si@tempdima}
%\begin{macro}{\si@tempdimb}
% Some storage dimensions are declared.
%    \begin{macrocode}
\newdimen\si@tab@predim
\newdimen\si@tab@postdim
\newdimen\si@tab@expdim
\newdimen\si@tempdima
\newdimen\si@tempdimb
%    \end{macrocode}
%\end{macro}
%\end{macro}
%\end{macro}
%\end{macro}
%\end{macro}
%\begin{macro}{\si@tab@sp}
% A short macro to control superscript.
%    \begin{macrocode}
\newcommand*{\si@tab@sp}{}
%    \end{macrocode}
%\end{macro}
%\begin{macro}{\si@tab@fixed}
%\changes{v1.0l}{2008/09/03}{Fixed issue with spacing when no digits
%  present after decimal marker}
% The column is not centred on the decimal marker; the user specifies
% how many characters on each side are allowed for.  First, the width
% of a character is measured, and stored.
%    \begin{macrocode}
\newcommand*{\si@tab@fixed}{%
  \si@log@debug{Using fixed-width S column}%
  \let\si@tab@sp\relax
  \setbox\si@tab@midbox=\hbox{}%
  \setbox\si@tab@expbox=\hbox{}%
  \setbox\si@tempboxa=\hbox{\si@out@num{1}}%
  \si@tempdima\wd\si@tempboxa\relax
%    \end{macrocode}
% The width for the two output boxes is set up.
%    \begin{macrocode}
  \si@tab@predim\the\si@tab@mantprecnt\si@tempdima\relax
  \si@tab@sepcorr{mantpre}{pre}%
  \si@tab@postdim\si@tab@mantpostcnt\si@tempdima\relax
  \setbox\si@tempboxa=\hbox{\ensuremath{{\si@decimalsymbol}}}%
  \ifnum\si@tab@mantpostcnt=\z@\relax\else
    \advance\si@tab@postdim\wd\si@tempboxa\relax
  \fi
  \si@tab@sepcorr{mantpost}{post}%
%    \end{macrocode}
% If space is needed for an exponent, it needs to be allowed for in
% the exponent box dimension.  First, the digits of the two parts are
% checked for; the width of a character is altered to be superscript.
%    \begin{macrocode}
  \setbox\si@tempboxa=\hbox{\si@out@num{^{1}}}%
  \si@tempdima\wd\si@tempboxa\relax
  \ifnum\si@tab@expprecnt>\z@\relax
    \si@tab@expdim\si@tab@expprecnt\si@tempdima\relax
    \si@tab@sepcorr{exppre}{exp}%
  \fi
  \let\si@tab@sp\sp
  \ifnum\si@tab@exppostcnt>\z@\relax
    \advance\si@tab@expdim\si@tab@exppostcnt\si@tempdima\relax
    \setbox\si@tempboxa=\hbox{%
      \ensuremath{^{\si@decimalsymbol}}}%
    \advance\si@tab@expdim\wd\si@tempboxa\relax
    \si@tab@sepcorr{exppost}{exp}%
  \fi
%    \end{macrocode}
% Space is reserved for signs.
%    \begin{macrocode}
  \setbox\si@tempboxa=\hbox{\ensuremath{-}}%
  \ifsi@tab@mantsign
    \advance\si@tab@predim\wd\si@tempboxa\relax
  \fi
  \setbox\si@tempboxa=\hbox{\ensuremath{^{-}}}%
  \ifsi@tab@expsign
    \advance\si@tab@expdim\wd\si@tempboxa\relax
  \fi
%    \end{macrocode}
% Now, if there is space to be saved for an exponent under any
% circumstances, the space for the ``$\times 10$'' part is needed.
% This is done by adding both counters together, then using this
% result for the logic.
%    \begin{macrocode}
  \si@tempcnta\si@tab@expprecnt\relax
  \advance\si@tempcnta\si@tab@exppostcnt\relax
  \ifnum\si@tempcnta>\z@\relax
    \setbox\si@tempboxa=\hbox{\ensuremath{%
      {}\si@expproduct{}\si@expbase}}%
    \advance\si@tab@expdim\wd\si@tempboxa\relax
  \fi
%    \end{macrocode}
% Finally for the box dimensions, if the exponent is not aligned,
% the space reserved for it is added to the post box.
%    \begin{macrocode}
  \ifsi@tabalignexp\else
    \advance\si@tab@postdim\si@tab@expdim\relax
  \fi
%    \end{macrocode}
% With the boxes set up, the contents can be sorted out.  A bit of
% shuffling may be needed, depending on the treatment of exponents.
%    \begin{macrocode}
  \setbox\si@tab@prebox=\hbox to\si@tab@predim{\hss\hfill
    \expandafter\si@out@num\expandafter{\si@tab@out}}%
  \ifx\@empty\si@num@out\@empty
    \setbox\si@tab@postbox=\hbox to\si@tab@postdim{\hfil}%
  \else
    \ifsi@tabalignexp\else
      \protected@edef\si@num@out{\si@num@out\si@tab@expout}%
    \fi
    \setbox\si@tab@postbox=\hbox to\si@tab@postdim
      {\ensuremath{{\si@decimalsymbol}}\expandafter\si@out@num
        \expandafter{\si@num@out}\hfil}%
  \fi
  \ifx\@empty\si@tab@expout\@empty
    \ifsi@tabalignexp
      \setbox\si@tab@expbox=\hbox to\si@tab@expdim{\hfil}%
    \fi
  \else
    \ifsi@tabalignexp
      \setbox\si@tab@expbox=\hbox to\si@tab@expdim
        {\expandafter\si@out@num\expandafter{\si@tab@expout}%
           \hfil}%
    \fi
  \fi}
%    \end{macrocode}
%\end{macro}
%\begin{macro}{\si@tab@sepcorr}
%\darg{one of \texttt{mantpre}, \texttt{mantpost}, \texttt{exppre} or
%  \texttt{exppost}}
%\darg{one of \texttt{pre}, \texttt{post} or \texttt{exp}}
% A spacing correction is needed \emph{if} the number of digits to be
% allowed for will lead to the introduction of a separator.  A counter
% and dimension are needed for the testing.
%    \begin{macrocode}
\newcommand*{\si@tab@sepcorr}[2]{%
  \expandafter\si@tempcnta\expandafter\the
    \csname si@tab@#1cnt\endcsname\relax
%    \end{macrocode}
% Calculate how many groups of three there are, then allow for not
% separating four characters if \cs{ifsi@sepfour} is false.
%    \begin{macrocode}
  \divide\si@tempcnta\thr@@\relax
  \ifsi@sepfour\else
    \expandafter\ifnum\expandafter\the
      \csname si@tab@#1cnt\endcsname=4\relax
      \si@tempcnta\z@\relax
    \fi
  \fi
%    \end{macrocode}
% The width of the separators is measured, and the correct number of
% separator widths are added to the box dimension.
%    \begin{macrocode}
  \setbox\si@tempboxa=\hbox{%
    \ensuremath{\si@tab@sp{\si@digitsep}}}%
  \expandafter\advance\csname si@tab@#2dim\endcsname
    \si@tempcnta\wd\si@tempboxa}
%    \end{macrocode}
%\end{macro}
%
%\subsection{Units}
%\begin{macro}{\SI}
%\doarg{keyval options}
%\darg{number}
% There are two types of user macros for the units system; those for
% defining new units, prefixes and powers, and those for using them.
% There are two macros for using units, \cs{SI} and \cs{si}, which
% work in very similar ways. argument to \cs{SI}
%    \begin{macrocode}
\si@newrobustcmd*{\SI}[2][]{%
  \@ifnextchar[%]
    {\si@SI[#1]{#2}}
    {\si@SI[#1]{#2}[]}}
%    \end{macrocode}
%\end{macro}
%\begin{macro}{\si}
%\doarg{keyval options}
%\darg{unit}
% \cs{si} is just an alias for \cs{SI} with no number; everything is
% handed off into an internal macro.  The internal macro also handles
% the optional prefix
%    \begin{macrocode}
\si@newrobustcmd*{\si}[2][]{\si@SI[#1]{}[]{#2}}
%    \end{macrocode}
%\end{macro}

%\begin{macro}{\newunit}
%\begin{macro}{\renewunit}
%\begin{macro}{\provideunit}
%\doarg{keyval options}
%\darg{unit}
%\darg{symbol}
% The \cs{newunit} and \cs{renewunit} macros create the new unit
% macros. To allow a mechanism for checking an existing definition,
% these macros simply carry out the appropriate tests, before handing
% off to the internal macro.  \cs{@ifdefinable} is not used here as a
% customised error is desirable. Other than that, the code here gives
% very similar results to \cs{newcommand} and \cs{renewcommand}.
% Finally, \cs{provideunit} adds the unit definition only if it does
% not already exist.
%    \begin{macrocode}
\newcommand*{\newunit}[3][]{%
  \si@ifdefinable{#2}
    {\si@unt@defunit[#1]{#2}{#3}}
    {\si@log@err{Unit \string#2 already defined!}\@eha}}
\newcommand*{\renewunit}[3][]{%
  \si@ifdefinable{#2}
    {\si@log@err{Unit \string#2 undefined}\@ehc
     \si@unt@defunit[#1]{#2}{#3}}
    {\si@log@inf{Redefining unit \string#2}%
     \si@unt@defunit[#1]{#2}{#3}}}
\newcommand*{\provideunit}[3][]{%
  \si@ifdefinable{#2}
    {\si@unt@defunit[#1]{#2}{#3}}
    {}}
%    \end{macrocode}
%\end{macro}
%\end{macro}
%\end{macro}
%\begin{macro}{\newprefix}
%\begin{macro}{\renewprefix}
%\begin{macro}{\provideprefix}
%\doarg{\texttt{binary}}
%\darg{multiple}
%\darg{powers-ten}
%\darg{symbol}
% The multiples of units are defined here; very similar code is used
% to the \cs{newunit}, \etc, macros.  The multiple prefixes cannot
% take an optional argument, and must represent some power.  Hence
% the arguments required are different.
%    \begin{macrocode}
\newcommand*{\newprefix}[4][]{%
  \si@ifdefinable{#2}
    {\si@unt@defprefix[#1]{#2}{#3}{#4}}
    {\si@log@err{Prefix \string#2 already defined!}\@eha}}
\newcommand*{\renewprefix}[4][]{%
  \si@ifdefinable{#2}
    {\si@log@err{Prefix \string#2 undefined}\@ehc
     \si@unt@defprefix[#1]{#2}{#3}{#4}}
    {\si@log@inf{Redefining prefix \string#2}%
     \si@unt@defprefix[#1]{#2}{#3}{#4}}}
\newcommand*{\provideprefix}[4][]{%
  \si@ifdefinable{#2}
    {\si@unt@defprefix[#1]{#2}{#3}{#4}}
    {}}
%    \end{macrocode}
%\end{macro}
%\end{macro}
%\end{macro}
%\begin{macro}{\newpower}
%\begin{macro}{\renewpower}
%\begin{macro}{\providepower}
%\doarg{\texttt{post}}
%\darg{number}
%\darg{power}
% Here power multiples for units are set up.  As with units and
% multiples, a layered approach is used to keep things easy to
% maintain.
%    \begin{macrocode}
\newcommand*{\newpower}[3][]{%
  \si@ifdefinable{#2}
    {\si@unt@defpower[#1]{#2}{#3}}
    {\si@log@err{Power \string#2 already defined!}\@eha}}
\newcommand*{\renewpower}[3][]{%
  \si@ifdefinable{#2}
    {\si@log@err{Power \string#2 undefined}\@ehc
     \si@unt@defpower[#1]{#2}{#3}}
    {\si@log@inf{Redefining power \string#2}%
     \si@unt@defpower[#1]{#2}{#3}}}
\newcommand*{\providepower}[3][]{%
  \si@ifdefinable{#2}
    {\si@unt@defpower[#1]{#2}{#3}}
    {}}
%    \end{macrocode}
%\end{macro}
%\end{macro}
%\end{macro}
%\begin{macro}{\newqualifier}
%\changes{v1.1}{2008/09/04}{New user macro}
%\begin{macro}{\renewqualifier}
%\changes{v1.1}{2008/09/04}{New user macro}
%\begin{macro}{\providequalifier}
%\changes{v1.1}{2008/09/04}{New user macro}
%\darg{qualifier}
%\darg{symbol}
% Creating unit qualifiers is much the same method.
%    \begin{macrocode}
\newcommand*{\newqualifier}[2]{%
  \si@ifdefinable{#1}
    {\si@unt@defqual{#1}{#2}}
    {\si@log@err{Qualifier \string#1 already defined!}\@eha}}
\newcommand*{\renewqualifier}[2]{%
  \si@ifdefinable{#1}
    {\si@log@err{Qualifier \string#1 undefined}\@ehc
     \si@unt@defqual{#1}{#2}}
    {\si@unt@defqual{#1}{#2}}}
\newcommand*{\providequalifier}[2]{%
  \si@ifdefinable{#1}
    {\si@unt@defqual{#1}{#2}}
    {}}
%    \end{macrocode}
%\end{macro}
%\end{macro}
%\end{macro}
%\begin{macro}{\ifsi@unt@num}
% A flag is needed to tell the processor whether there is a number,
% to get the correct spacing.
%    \begin{macrocode}
\newif\ifsi@unt@num
%    \end{macrocode}
%\end{macro}
%\begin{macro}{\si@unt@unitarg}
% A storage macro for the argument of the unit macro.
%    \begin{macrocode}
\newcommand*{\si@unt@unitarg}{}
%    \end{macrocode}
%\end{macro}
%\begin{macro}{\si@SI}
%\begin{macro}{\si@unt@SIopts}
%\doarg{keyval options}
%\darg{unit}
%\doarg{preunit}
%\darg{unit}
% The internal processing starts with \cs{si@SI}, which processes the
% second optional argument to \cs{SI} (which is empty for \cs{si}).
% Everything is set up in a group, and processing begins by handling
% the options.
%    \begin{macrocode}
\newcommand*{\si@unt@SIopts}{}
\def\si@SI[#1]#2[#3]#4{%
  \begingroup
    \let\fg\SIfg
    \si@ifnotmtarg{#1}
      {\sisetup{#1}%
       \renewcommand*{\si@unt@SIopts}{#1}}%
%    \end{macrocode}
%\end{macro}
% The prefix unit is handled before any processing of the number; the
% flags are set to get spacing correct.
%    \begin{macrocode}
    \si@unt@numfalse
    \si@xspacefalse
    \si@ifnotmtarg{#3}
      {\si@log@debug{Prefix unit found}%
       \si@unt@printunit{#3}}%
%    \end{macrocode}
% The numerical argument may be empty, in which case no extra space
% should be produced.
%    \begin{macrocode}
    \si@ifnotmtarg{#4}
      {\renewcommand*{\si@unt@unitarg}{#4}}%
    \si@ifnotmtarg{#2}
      {\si@log@debug{Number found in \string\SI\space
         argument}%
       \ifsi@repeatunits\else
         \ifsi@trapambigerr
           \expandafter\expandafter\expandafter
             \si@num@ambigerrtrue
         \fi
       \fi
       \num{#2}%
       \si@unt@numtrue}%
%    \end{macrocode}
% If there is a unit, a check is needed in case the units need to
% have a power added.
%    \begin{macrocode}
    \si@ifnotmtarg{#4}
      {\si@ifmtarg{#2}
         {\si@unt@printunit{#4}}
         {\si@tempcnta\z@\relax
         \ifsi@addunitpower
           \si@unt@countx{#2}%
         \fi
          \ifnum\si@tempcnta>\z@\relax
            \advance\si@tempcnta\@ne\relax
            \edef\si@tempa{\noexpand\tothe{\si@tempcnta}}%
            \renewcommand*{\si@tempb}{#4}%
            \expandafter\expandafter\expandafter
            \si@unt@printunit\expandafter\expandafter
            \expandafter{%
              \expandafter\si@tempb\si@tempa}%
          \else
            \si@unt@printunit{#4}%
          \fi}}%
  \endgroup}
%    \end{macrocode}
%\end{macro}
%\begin{macro}{\si@unt@countx}
% A short macro to count up any multiplication in numerical input.
%    \begin{macrocode}
\newcommand*{\si@unt@countx}[1]{%
  \si@tempcnta\z@\relax
  \expandafter\si@unt@cntx#1\@empty\@empty}
\def\si@unt@cntx#1#2\@empty{%
  \si@str@ifchrstr{#1}{\si@numprod}
    {\advance\si@tempcnta\@ne\relax}
    {}%
  \ifx\@empty#2\@empty\else
    \si@unt@cntx#2\@empty\@empty
  \fi}
%    \end{macrocode}
%\end{macro}
%\begin{macro}{\si@unt@ifliteral}
%\begin{macro}{\ifsi@unt@littest}
%\darg{unit}
% The next stage of the processor is to determine whether or not the
% argument of the unit macro is processable.  For literal arguments,
% this is not the case, and the argument is typeset ``as is''.  On
% the other hand, any units, \etc., declared by the package will work
% with the processor, and so need to be executed before typesetting
% the result.
%    \begin{macrocode}
\newif\ifsi@unt@littest
\newcommand*{\si@unt@ifliteral}[1]{%
  \begingroup
    \si@unt@littesttrue
%    \end{macrocode}
% The test relies on any non-processable test having some width;
% hopefully, this should be the case.
%    \begin{macrocode}
    \setbox\si@tempboxa=\hbox{\si@unt@out{#1}}%
    \ifdim\wd\si@tempboxa>\z@\relax
      \aftergroup\@firstoftwo
    \else
      \aftergroup\@secondoftwo
    \fi
  \endgroup}
%    \end{macrocode}
%\end{macro}
%\end{macro}
%\begin{macro}{\ifsi@unt@litout}
%\begin{macro}{\si@unt@printunit}
%\darg{unit}
% The printing macro uses the above test to determine how to act.  It
% then carries out the appropriate action: either typesetting or
% executing.  A flag is also provided so that any macro units inside
% a partially-literal argument will work (this is also needed to
% emulate \unitsdef).
%    \begin{macrocode}
\newif\ifsi@unt@litout
\newcommand*{\si@unt@printunit}[1]{%
  \si@unt@ifliteral{#1}
%    \end{macrocode}
% The unit includes one or more literal items; typeset using the
% unit typesetting macro.
%    \begin{macrocode}
    {\si@log@debug{Literal items found in unit
      argument:\MessageBreak outputting without further
      processing}%
     \si@unt@litouttrue
     \si@unt@addvaluesep
     \si@unt@out{#1}}
%    \end{macrocode}
% For processable output, the argument is executed; the macros are
% all designed for this.
%    \begin{macrocode}
    {\si@log@debug{Macro unit found:\MessageBreak
       processing to format output}%
     \si@unt@init
     \advance\si@unt@depthcnt\@ne\relax
     #1%
     \si@unt@final}}
%    \end{macrocode}
%\end{macro}
%\end{macro}
%\begin{macro}{\si@unt@addvaluesep}
%\begin{macro}{\si@unt@addvalsep}
%\begin{macro}{\si@unt@litvalsep}
%\begin{macro}{\si@unt@stackvalsep}
% To ensure no problems pop up with expansion, adding the value--unit
% space is handled by a macro.
%    \begin{macrocode}
\newcommand*{\si@unt@addvaluesep}{%
  \ifsi@unt@num
    \expandafter\si@unt@addvalsep
  \fi}
\newcommand*{\si@unt@addvalsep}{%
  \ifsi@unt@litout
    \expandafter\si@unt@litvalsep
  \else
    \expandafter\si@unt@stackvalsep
  \fi}
\newcommand*{\si@unt@stackvalsep}{%
  \protected@edef\si@unt@spstack{\si@valuesep}}
\newcommand*{\si@unt@litvalsep}{%
  \nobreak\ensuremath{\si@valuesep}\nobreak}
%    \end{macrocode}
%\end{macro}
%\end{macro}
%\end{macro}
%\end{macro}
%\begin{macro}{\si@unt@spstack}
%\begin{macro}{\si@unt@stacka}
%\begin{macro}{\si@unt@stackb}
%\begin{macro}{\si@unt@unitcnta}
%\begin{macro}{\si@unt@unitcntb}
%\begin{macro}{\si@unt@depthcnt}
%\begin{macro}{\ifsi@unt@first}
%\begin{macro}{\ifsi@unt@first}
%\begin{macro}{\si@unt@init}
% The initialisation macro sets up the various switches, and clears
% the storage areas for the formatted output.  There are two stacks,
% as when typesetting as fractions, the two parts of the number have
% to be stored separately.  The depth counter is used to tell when
% recursion ends in the processor.  The ``first'' switch is needed as
% the depth counter will not be at one for items processed by
% \cs{SI}.
%    \begin{macrocode}
\newcommand*{\si@unt@spstack}{}
\newcommand*{\si@unt@stacka}{}
\newcommand*{\si@unt@stackb}{}
\newcount\si@unt@unitcnta
\newcount\si@unt@unitcntb
\newcount\si@unt@depthcnt
\newif\ifsi@unt@first
\si@unt@depthcnt\m@ne\relax
\newcommand*{\si@unt@init}{%
  \begingroup
    \si@unt@litoutfalse
    \si@unt@litprefixfalse
    \si@unt@firsttrue
    \si@unt@perfalse
    \si@unt@perseenfalse
    \si@unt@prepowerfalse
    \si@unt@depthcnt\z@\relax
    \si@unt@powerdim\z@\relax
    \si@unt@unitcnta\z@\relax
    \si@unt@unitcntb\z@\relax
    \si@unt@prefixcnt\z@\relax
    \renewcommand*{\si@unt@spstack}{}%
    \renewcommand*{\si@unt@stacka}{}%
    \renewcommand*{\si@unt@stackb}{}%
    \renewcommand*{\si@unt@spacestacka}{}%
    \renewcommand*{\si@unt@spacestackb}{}%
    \renewcommand*{\si@unt@lastadda}{space}%
    \renewcommand*{\si@unt@lastaddb}{space}}
%    \end{macrocode}
%\end{macro}
%\end{macro}
%\end{macro}
%\end{macro}
%\end{macro}
%\end{macro}
%\end{macro}
%\end{macro}
%\end{macro}
%\begin{macro}{\si@unt@final}
% The finalisation macro finishes off the output and resets the flags.
%    \begin{macrocode}
\newcommand*{\si@unt@final}{%
    \si@unt@third
    \si@unt@stackout
  \endgroup
  \ifsi@xspace
    \expandafter\expandafter\expandafter\xspace
  \fi}
%    \end{macrocode}
%\end{macro}
%\begin{macro}{\si@unt@defunit}
%\doarg{keyval options}
%\darg{unit}
%\darg{symbol}
% The internal macro for defining a unit does not check for
% redefinition; that is done by the user macros.
%    \begin{macrocode}
\newcommand*{\si@unt@defunit}[3][]{%
  \si@log@debug{Declaring unit \string#2 with \MessageBreak
    meaning \string#3}%
%    \end{macrocode}
% The optional argument needs to be saved. The macro name is reversed
% so that life is easier with the expansions here.
%    \begin{macrocode}
  \si@ifnotmtarg{#1}
    {\expandafter\@namedef\expandafter{%
      \expandafter\@gobble\string#2@opt@unt@si}{#1}}%
%    \end{macrocode}
% The unit macro itself is now defined.  The definition simply
% selects the correct path for the rest of the processing to go down.
%    \begin{macrocode}
  \protected\def#2{%
    \ifsi@allowoptarg
      \expandafter\si@unt@withopt
    \else
      \expandafter\si@unt@noopt
    \fi
    {#2}{#3}}}
%    \end{macrocode}
%\end{macro}
%\begin{macro}{\si@unt@withopt}
%\begin{macro}{\si@unt@noopt}
%\darg{unit}
%\darg{symbol}
% To allow the correct expansion, the potential optional argument to
% a unit macro has to come last.  So \cs{@ifnextchar} is needed to
% detect it and pass data through.  To keep variation down, when the
% argument is not allowed, the empty |[]| is supplied.
%    \begin{macrocode}
\newcommand*{\si@unt@withopt}[2]{%
  \@ifnextchar[%]
    {\si@unt@opt{#1}{#2}}
    {\si@unt@opt{#1}{#2}[]}}
\newcommand*{\si@unt@noopt}[2]{\si@unt@opt{#1}{#2}[]}
%    \end{macrocode}
%\end{macro}
%\end{macro}
%\begin{macro}{\si@unt@opt}
%\darg{unit}
%\darg{symbol}
%\doarg{number}
% The optional argument to the unit macro (if present) is converted
% to a normal one for ease.  The correct route for processing is then
% picked.
%    \begin{macrocode}
\def\si@unt@opt#1#2[#3]{%
  \ifsi@unt@littest
    \expandafter\si@gobblethree
  \else
%    \end{macrocode}
% For literal output, the second argument is all that is needed.
%    \begin{macrocode}
    \ifsi@unt@litout
      \expandafter\expandafter\expandafter\@gobbletwo
    \else
      \expandafter\expandafter\expandafter\si@unt@unit
    \fi
  \fi
  {#3}{#1}{#2}}
%    \end{macrocode}
%\end{macro}
%\begin{macro}{\si@gobblethree}
% \LaTeX\ does not have a \cs{@gobblethree} macro, but one is needed.
%    \begin{macrocode}
\long\def\si@gobblethree #1#2#3{}
%    \end{macrocode}
%\end{macro}
%\begin{macro}{\si@unt@defprefix}
%\begin{macro}{\ifsi@unt@litprefix}
%\doarg{\texttt{binary}}
%\darg{multiple}
%\darg{powers-ten}
%\darg{symbol}
% As with units, multiples are defined by an internal macro.
%    \begin{macrocode}
\newif\ifsi@unt@litprefix
\si@unt@litprefixtrue
\newcommand*{\si@unt@defprefix}[4][]{%
  \si@log@debug{Declaring multiple \string#1 with\MessageBreak
    meaning \string#4}%
%    \end{macrocode}
% The optional argument is saved, using \cs{def} as no check is made
% on an existing definition of the storage macro.
%    \begin{macrocode}
  \expandafter\expandafter\expandafter\def\expandafter
    \csname\expandafter\@gobble\string#2@opt@si\endcsname{#1}%
  \protected\def#2{%
    \ifsi@unt@littest
      \expandafter\si@gobblethree
    \else
      \ifsi@unt@litout
        \expandafter\expandafter\expandafter\@gobbletwo
      \else
        \ifsi@unt@litprefix
          \expandafter\expandafter\expandafter\expandafter
            \expandafter\expandafter\expandafter\@gobbletwo
        \else
          \expandafter\expandafter\expandafter\expandafter
            \expandafter\expandafter\expandafter\si@unt@prefix
        \fi
      \fi
    \fi
    {#2}{#3}{#4}}}
%    \end{macrocode}
%\end{macro}
%\end{macro}
%\begin{macro}{\si@unt@defpower}
%\doarg{\texttt{post}}
%\darg{number}
%\darg{power}
% The definition of powers is complicated by the need to handle both
% those given before units (such as \cs{cubic}) and those given after
% (\eg \cs{cubed}).  This means that an optional argument is needed.
%    \begin{macrocode}
\newcommand*{\si@unt@defpower}[3][]{%
  \si@log@debug{Declaring power \string#2 with\MessageBreak
    meaning \string#3}%
%    \end{macrocode}
% Once again the optional argument is saved.
%    \begin{macrocode}
  \expandafter\expandafter\expandafter\def\expandafter
    \csname\expandafter\@gobble\string#2@opt@si\endcsname{#1}%
  \protected\def#2{%
    \ifsi@unt@littest
      \expandafter\@gobbletwo
    \else
%    \end{macrocode}
% The literal output here does not need to gobble anything.
%    \begin{macrocode}
      \ifsi@unt@litout
        \expandafter\expandafter\expandafter\si@unt@litpower
      \else
        \expandafter\expandafter\expandafter\si@unt@power
      \fi
    \fi
    {#2}{#3}}}
%    \end{macrocode}
%\end{macro}
%\begin{macro}{\si@unt@defqual}
%\darg{qualifier}
%\darg{symbol}
% Qualifiers are defined in a similar manner.
%    \begin{macrocode}
\newcommand*{\si@unt@defqual}[2]{%
  \si@log@debug{Declaring qualifier \string#1 with\MessageBreak
    meaning \string#2}%
  \protected\def#1{%
    \ifsi@unt@littest
      \expandafter\@gobble
    \else
      \ifsi@unt@litout
        \expandafter\expandafter\expandafter\si@unt@litqual
      \else
        \expandafter\expandafter\expandafter\si@unt@qual
      \fi
    \fi
    {#2}}}
%    \end{macrocode}
%\end{macro}
%
%\begin{macro}{\si@unt@unithook}
%\begin{macro}{\si@unt@unit}
%\darg{number}
%\darg{unit}
%\darg{symbol}
% The macro for units is actually a processor, rather than
% typesetting anything, which is handled elsewhere.  The first
% argument to the macro is optional, but does not have square
% brackets to keep things simple with gobbling.
%    \begin{macrocode}
\newcommand*{\si@unt@unithook}{}
\newcommand*{\si@unt@unit}[3]{%
%    \end{macrocode}
% When the count is minus one at the start of the processor, then the
% unit is begin used on its own: initialisation occurs.
%    \begin{macrocode}
  \ifnum\si@unt@depthcnt=\m@ne\relax
    \expandafter\si@unt@init
  \fi
  \advance\si@unt@depthcnt\@ne\relax
  \si@log@debug{Unit processing: level \the\si@unt@depthcnt,
    \MessageBreak unit \string#2}%
  \si@unt@firstorsecond{#1}{#2}%
%    \end{macrocode}
% The core of the \cs{si@unt@unit} macro is testing if the symbol for
% the unit is a literal value or another macro.  Depending on the
% result, the symbol is either used as a literal or executed.
%    \begin{macrocode}
  \si@unt@ifliteral{#3}
    {\si@unt@addtostack{unit}{#3}%
     \ifsi@unt@prepower
       \expandafter\si@unt@stkpower
     \fi}
    {#3}%
%    \end{macrocode}
% The counter is now stepped down, before checking if this is
% the end of a compound unit.
%    \begin{macrocode}
  \advance\si@unt@depthcnt\m@ne\relax
  \ifnum\si@unt@depthcnt=\z@\relax
    \expandafter\si@unt@final
  \fi}
%    \end{macrocode}
%\end{macro}
%\end{macro}
%\begin{macro}{\si@unt@firstorsecond}
%\darg{number}
%\darg{unit-macro}
% At this stage, the flag will be set for the first item to be
% processed whichever route the unit has been called by.
%    \begin{macrocode}
\newcommand*{\si@unt@firstorsecond}[2]{%
  \ifsi@unt@first
    \expandafter\si@unt@first
  \else
    \expandafter\si@unt@second
  \fi
  {#1}{#2}}%
%    \end{macrocode}
%\end{macro}
%\begin{macro}{\si@unt@first}
%\darg{number}
%\darg{unit-macro}
% For the first unit in the input, some extra tasks are needed.
% First, the optional argument for the unit macro needs to be
% tested.
%    \begin{macrocode}
\newcommand*{\si@unt@first}[2]{%
  \si@ifnotmtarg{#1}
    {\num{#1}%
     \si@unt@numtrue}%
  \si@unt@unithook
%    \end{macrocode}
% To avoid filling up the macro list with useless values, the \eTeX\
% primitive \cs{ifcsname} is employed here (it also avoids complex
% expansion issues).  If some options exist, they are set.
%    \begin{macrocode}
  \ifcsname\expandafter\@gobble\string#2@opt@unt@si\endcsname
    \expandafter\si@unt@setopts
  \else
    \expandafter\@gobble
  \fi
  {#2}%
  \si@unt@addvaluesep
  \si@unt@firstfalse}
%    \end{macrocode}
%\end{macro}
%\begin{macro}{\si@unt@setopts}
%\begin{macro}{\si@unt@setSIopts}
%\darg{unit}
% A rather long set of \cs{expandafter} commands to get the options
% to set safely.
%    \begin{macrocode}
\newcommand*{\si@unt@setopts}[1]{%
  \expandafter\expandafter\expandafter\expandafter\expandafter
    \expandafter\expandafter\si@temptoks\expandafter
    \expandafter\expandafter\expandafter\expandafter
    \expandafter\expandafter{\expandafter%
      \csname\expandafter\@gobble\string#1@opt@unt@si%
      \endcsname}%
  \expandafter\sisetup\expandafter{\the\si@temptoks}%
  \si@log@debug{Applying options `\the\si@temptoks'
    for\MessageBreak unit \string#1}%
%    \end{macrocode}
% The user options are reloaded, if defined, to ensure that they
% still work as expected.
%    \begin{macrocode}
  \@ifundefined{si@unt@SIopts}{}
    {\ifx\@empty\si@unt@SIopts\@empty\else
       \expandafter\expandafter\si@unt@setSIopts
     \fi}}
\newcommand*{\si@unt@setSIopts}{%
  \expandafter\si@temptoks\expandafter{\si@unt@SIopts}%
  \expandafter\sisetup\expandafter{\the\si@temptoks}}
%    \end{macrocode}
%\end{macro}
%\end{macro}
%\begin{macro}{\si@unt@second}
%\begin{macro}{\si@unt@third}
%\darg{number}
%\darg{unit}
% For everything apart from the first item to be processed, spacing
% may need to be added to separated different units.  The macro is
% divided into two, so that everything except the space can be added
% in finalisation.
%    \begin{macrocode}
\newcommand*{\si@unt@second}[2]{%
  \si@ifnotmtarg{#1}
    {\si@log@warn{Optional argument to unit macro\MessageBreak
       allowed only for outer unit}}%
  \si@unt@third
  \si@unt@addtostack{space}{\ensuremath{\si@unitsep}}}
\newcommand*{\si@unt@third}{%
  \ifsi@unt@prepower\else
    \expandafter\si@unt@stkpower
  \fi
%    \end{macrocode}
% A check is made to avoid adding $-1$ to prefixes.  If \opt{frac} is
% active, then the |b| stack will be in use, otherwise it will be
% |a|.
%    \begin{macrocode}
  \renewcommand*{\si@tempa}{prefix}%
  \expandafter\ifx
    \csname si@unt@lastadd\si@unt@checkstack\endcsname\si@tempa
  \else
    \expandafter\si@unt@spacecheck
  \fi
  \ifsi@unt@per
    \expandafter\si@unt@perseentrue
  \fi}
%    \end{macrocode}
%\end{macro}
%\end{macro}
%\begin{macro}{\si@unt@spacecheck}
% A check to prevent adding $-1$ at the very beginning of the unit,
% where there is a space on the stack.
%    \begin{macrocode}
\newcommand*{\si@unt@spacecheck}{%
  \renewcommand*{\si@tempa}{space}%
  \expandafter\ifx
    \csname si@unt@lastadd\si@unt@checkstack\endcsname\si@tempa
  \else
    \expandafter\si@unt@reciptest
  \fi}
%    \end{macrocode}
%\end{macro}
%\begin{macro}{\si@unt@prefix}
%\darg{multiple}
%\darg{power-ten}
%\darg{symbol}
% Actual output of the prefixes.
%    \begin{macrocode}
\newcommand*{\si@unt@prefix}[3]{%
  \si@unt@firstorsecond{}{#1}%
  \ifsi@prefixsymbolic
    \expandafter\si@unt@addprefix
  \else
    \expandafter\si@unt@countprefix
  \fi
  {#1}{#2}{#3}}
%    \end{macrocode}
%\end{macro}
%\begin{macro}{\si@unt@addprefix}
%\darg{gobbled}
%\darg{gobbled}
%\darg{symbol}
% To add the prefix, a little translation is needed.
%    \begin{macrocode}
\newcommand*{\si@unt@addprefix}[3]{%
  \si@unt@addtostack{prefix}{#3}}
%    \end{macrocode}
%\end{macro}
%\begin{macro}{\si@unt@prefixcnt}
% A storage area is created.
%    \begin{macrocode}
\newcount\si@unt@prefixcnt
%    \end{macrocode}
%\end{macro}
%\begin{macro}{\si@unt@countprefix}
%\begin{macro}{\si@unt@invprefix}
%\darg{multiple}
%\darg{powers-ten}
%\darg{gobbled}
% To count the prefix numeral, the symbol is thrown away.  First, a
% check is made for binary units.
%    \begin{macrocode}
\newcommand*{\si@unt@countprefix}[3]{%
  \renewcommand*{\si@tempa}{binary}%
  \expandafter\expandafter\expandafter\ifx\expandafter
    \csname\expandafter\@gobble\string#1@opt@si\endcsname
    \si@tempa
    \expandafter\sisetup
  \else
    \expandafter\@gobble
  \fi
  {prefixbase=two}%
  \si@tempcnta#2\relax
  \ifsi@unt@per
    \expandafter\si@unt@invprefix
  \fi
  \advance\si@unt@prefixcnt\si@tempcnta\relax}
\newcommand*{\si@unt@invprefix}{%
  \si@tempcntb\si@tempcnta\relax
  \si@tempcnta -\si@tempcntb\relax}
%    \end{macrocode}
%\end{macro}
%\end{macro}
%\begin{macro}{\si@unt@litpower}
%\darg{gobbled}
%\darg{number}
% For literal power output, the number can't simply be dumped, so
% a macro is needed.
%    \begin{macrocode}
\newcommand*{\si@unt@litpower}[2]{\textsuperscript{#2}}
%    \end{macrocode}
%\end{macro}
%\begin{macro}{\ifsi@unt@prepower}
% If a power is seen before a unit, tracking is needed.
%    \begin{macrocode}
\newif\ifsi@unt@prepower
%    \end{macrocode}
%\end{macro}
%\begin{macro}{\si@unt@power}
%\darg{power}
%\darg{number}
% The handling of powers starts by ensuring that ``pre'' powers
% follow \cs{per} cleanly.  Then a check is needed for inversion.
%    \begin{macrocode}
\newcommand*{\si@unt@power}[2]{%
  \renewcommand*{\si@tempa}{post}%
  \expandafter\expandafter\expandafter\ifx\expandafter
    \csname\expandafter\@gobble\string#1@opt@si\endcsname
    \si@tempa
    \expandafter\@gobbletwo
  \else
    \expandafter\si@unt@firstorsecond
  \fi
  {}{\power}%
  \si@unt@powerdim #2 pt\relax
  \ifsi@frac\else
    \ifsi@unt@per
      \expandafter\expandafter\expandafter\si@unt@invpower
    \fi
  \fi
  \renewcommand*{\si@tempa}{post}%
  \si@unt@prepowertrue
  \expandafter\expandafter\expandafter\ifx\expandafter
    \csname\expandafter\@gobble\string#1@opt@si\endcsname
    \si@tempa
    \expandafter\si@unt@stackpower
  \else
    \si@log@debug{Power \strip@pt\si@unt@powerdim\space saved
      to be added after\MessageBreak next unit}%
  \fi}
%    \end{macrocode}
%\end{macro}
%\begin{macro}{\si@unt@powerdim}
% To do sign-inversion on the power, a dimension is used (this allows
% non-integer values to be handled).
%    \begin{macrocode}
\newdimen\si@unt@powerdim
%    \end{macrocode}
%\end{macro}
%\begin{macro}{\si@unt@stackpower}
%\begin{macro}{\si@unt@stkpower}
%\begin{macro}{\si@unt@stkpwr}
%\begin{macro}{\si@unt@skpwr}
% Adding powers to the stack should also clear the power list.  If
% the number is already zero, then  of course nothing happens.
%    \begin{macrocode}
\newcommand*{\si@unt@stackpower}{%
  \si@unt@prepowerfalse
%    \end{macrocode}
% A trap is used for $-1$ added to the denominator of a fraction.
%    \begin{macrocode}
  \si@unt@stkpower
  \ifsi@stickyper\else
    \si@unt@perfalse
    \si@unt@perseenfalse
  \fi}
%    \end{macrocode}
% The \cs{si@unt@stkpower} macro needs to be called from a few
% places, so is spun out from the above.
%    \begin{macrocode}
\newcommand*{\si@unt@stkpower}{%
  \ifdim\si@unt@powerdim=\m@ne pt\relax
    \ifsi@frac\else
      \expandafter\expandafter\expandafter\si@unt@stkpwr
    \fi
  \else
    \expandafter\si@unt@stkpwr
  \fi}
%    \end{macrocode}
%\changes{v1.1}{2008/09/04}{Altered to account for qualifiers}
% Finally, the actual adding (set up to avoid problems with the
% \cs{if} above).
%    \begin{macrocode}
\newcommand*{\si@unt@stkpwr}{%
  \ifdim\si@unt@powerdim=\z@\relax\else
    \renewcommand*{\si@tempa}{unit}%
    \expandafter\ifx
      \csname si@unt@lastadd\si@unt@checkstack\endcsname
      \si@tempa
      \si@unt@skpwr
    \else
      \renewcommand*{\si@tempa}{qual}%
      \expandafter\ifx
        \csname si@unt@lastadd\si@unt@checkstack\endcsname
        \si@tempa
        \si@unt@skpwrqual
      \fi
    \fi
  \fi
  \si@unt@powerdim\z@\relax}
\newcommand*{\si@unt@skpwr}{%
  \si@log@debug{Adding power
    \strip@pt\si@unt@powerdim\space to output stack}%
  \si@unt@addtostack{power}{^{\num[fixdp=false]{%
    \strip@pt\si@unt@powerdim}}}}
%    \end{macrocode}
%\end{macro}
%\end{macro}
%\end{macro}
%\end{macro}
%\begin{macro}{\si@unt@invpower}
% A macro to change the sign of the current power.
%    \begin{macrocode}
\newcommand*{\si@unt@invpower}{%
  \si@tempdima\si@unt@powerdim\relax
  \si@unt@powerdim -\si@tempdima\relax
%    \end{macrocode}
% The power might end up as ``1'', which is not wanted.  So it is
% chucked away.
%    \begin{macrocode}
  \ifdim\si@unt@powerdim=\p@\relax
    \si@unt@powerdim\z@\relax
  \fi}
%    \end{macrocode}
%\end{macro}
%\begin{macro}{\si@unt@litqual}
%    \begin{macrocode}
\newcommand*{\si@unt@litqual}[1]{(#1)}
%    \end{macrocode}
%\end{macro}
%\begin{macro}{\si@unt@qual}
%    \begin{macrocode}
\newcommand*{\si@unt@qual}[1]{%
  \ifsi@unt@first
    \si@log@err{Qualifier before units}
      {A unit qualifier can only be given after\MessageBreak
       one or more units.  You tried to open a unit
       with\MessageBreak \string#1}%
  \fi
%    \end{macrocode}
% The space after the current unit is shuffled so it will come
% after the qualifier.
%    \begin{macrocode}
  \expandafter\let\expandafter\si@unt@spacestack
    \csname si@unt@spacestack\si@unt@checkstack\endcsname
  \expandafter\let\csname si@unt@spacestack\si@unt@checkstack
    \endcsname\@empty
%    \end{macrocode}
%\begin{macro}{\si@unt@skpwrqual}
% Some protection against trying to add two powers.
%    \begin{macrocode}
  \renewcommand*{\si@tempa}{power}%
    \expandafter\ifx
    \csname si@unt@lastadd\si@unt@checkstack\endcsname\si@tempa
    \let\si@unt@skpwrqual\relax
  \else
    \let\si@unt@skpwrqual\si@unt@skpwr
  \fi
  \si@unt@addtostack{qual}{_{#1}}%
  \expandafter\let\csname si@unt@spacestack\si@unt@checkstack
    \endcsname\si@unt@spacestack}
%    \end{macrocode}
%\end{macro}
%\end{macro}
%\begin{macro}{\ifsi@unt@per}
%\begin{macro}{\ifsi@unt@perseen}
%\begin{macro}{\per}
%\begin{macro}{\si@per}
%\begin{macro}{\si@unt@per}
% The \cs{per} macro sets the correct flags; almost everything else
% is done elsewhere.  There is always the case of two \cs{per}
% instructions; so the flag is flipped rather than set arbitrarily.
% The second flag is needed so that \cs{per} can give powers of $-1$
% properly.
%    \begin{macrocode}
\newif\ifsi@unt@per
\newif\ifsi@unt@perseen
\si@newrobustcmd*{\si@per}{%
  \ifsi@unt@littest\else
    \ifsi@unt@litout
      \expandafter\expandafter\expandafter /%
    \else
      \ifsi@unt@litprefix
        \expandafter\expandafter\expandafter\expandafter
          \expandafter\expandafter\expandafter /%
      \else
        \expandafter\expandafter\expandafter\expandafter
          \expandafter\expandafter\expandafter\si@unt@per
      \fi
    \fi
  \fi}
\newcommand*{\si@unt@per}{%
  \si@unt@firstorsecond{}{\per}%
  \ifsi@unt@per
    \ifsi@stickyper\else
      \expandafter\expandafter\expandafter\si@unt@perfalse
    \fi
  \else
    \expandafter\si@unt@pertrue
  \fi}
\let\per\si@per
%    \end{macrocode}
%\end{macro}
%\end{macro}
%\end{macro}
%\end{macro}
%\end{macro}
%\begin{macro}{\si@unt@reciptest}
%\begin{macro}{\si@unt@recip}
% A test is needed for adding $-1$ when needed. The second macro is
% fired only if the power should be reciprocal.
%    \begin{macrocode}
\newcommand*{\si@unt@reciptest}{%
  \ifsi@unt@per
    \ifsi@unt@perseen
      \expandafter\expandafter\expandafter\si@unt@recip
    \fi
  \fi}
\newcommand*{\si@unt@recip}{%
  \si@unt@powerdim\m@ne pt\relax
  \si@unt@stackpower}
%    \end{macrocode}
%\end{macro}
%\end{macro}
%\begin{macro}{\si@unt@lastadda}
%\begin{macro}{\si@unt@lastaddb}
% Items cannot be added directly to the stacks (except the spacing
% stack, a) as the fractional handling may need to add the item to
% either storage area. First, a track is kept of what has
% been added at each stage.
%    \begin{macrocode}
\newcommand*{\si@unt@lastadda}{}
\newcommand*{\si@unt@lastaddb}{}
%    \end{macrocode}
%\end{macro}
%\end{macro}
%\begin{macro}{\si@unt@addtostack}
%\changes{v1.1b}{2008/10/12}{Added warning for multiple prefixes}
%\darg{type}
%\darg{token}
% By indicating the type of data to be added to the stack, problems
% can be avoided.
%    \begin{macrocode}
\newcommand*{\si@unt@addtostack}[2]{%
  \renewcommand*{\si@tempa}{#1}%
%    \end{macrocode}
% Two consecutive items cannot be of the same type; there must be spaces
% between units, units between prefixes, \etc.
%    \begin{macrocode}
  \expandafter\ifx
    \csname si@unt@lastadd\si@unt@checkstack\endcsname\si@tempa
      \renewcommand*{\si@tempa}{prefix}%
      \expandafter\ifx\csname si@unt@lastadd\si@unt@checkstack
        \endcsname\si@tempa
        \si@log@warn{Two prefixes found: only first printed}%
      \fi
    \expandafter\@gobbletwo
  \else
    \expandafter\si@unt@preplussp
  \fi
  {#1}{#2}}
%    \end{macrocode}
%\end{macro}
%\begin{macro}{\si@unt@preplussp}
%\darg{type}
%\darg{stack}
%\darg{token}
%\darg{gobbled}
% The space added after a prefix needs to be ignored.
%    \begin{macrocode}
\newcommand*{\si@unt@preplussp}[2]{%
  \renewcommand*{\si@tempa}{prefix+space}%
  \edef\si@tempb{%
    \csname si@unt@lastadd\si@unt@checkstack\endcsname+#1}%
  \ifx\si@tempa\si@tempb
    \expandafter\@gobbletwo
  \else
    \expandafter\si@unt@stack
  \fi
  {#1}{#2}}
%    \end{macrocode}
%\end{macro}
%\begin{macro}{\si@unt@stack}
%\darg{type}
%\darg{token}
% The macro for actually doing the stacking up.
%    \begin{macrocode}
\newcommand*{\si@unt@stack}[2]{%
  \expandafter\renewcommand\expandafter*\expandafter{%
    \csname si@unt@lastadd\si@unt@checkstack\endcsname}{#1}%
%    \end{macrocode}
% A count is kept of the number of \emph{units} added to each stack.
%    \begin{macrocode}
  \renewcommand*{\si@tempa}{#1}%
  \renewcommand*{\si@tempb}{unit}%
  \ifx\si@tempa\si@tempb
    \expandafter\si@unt@inccnt
  \fi
%    \end{macrocode}
% If a space is added, it is actually held until the next add.
%    \begin{macrocode}
  \renewcommand*{\si@tempb}{space}%
  \ifx\si@tempa\si@tempb
    \expandafter\si@unt@holdspace
  \else
    \expandafter\si@unt@addstack
  \fi
  {#2}}
%    \end{macrocode}
%\end{macro}
%\begin{macro}{\si@unt@inccnt}
% The appropriate counter is added to.
%    \begin{macrocode}
\newcommand*{\si@unt@inccnt}{%
  \expandafter\advance
    \csname si@unt@unitcnt\si@unt@checkstack\endcsname
    \@ne\relax}
%    \end{macrocode}
%\end{macro}
%\begin{macro}{\si@unt@spacestack}
%\begin{macro}{\si@unt@spacestacka}
%\begin{macro}{\si@unt@spacestackb}
% The stacked material needs somewhere to live.
%    \begin{macrocode}
\newcommand*{\si@unt@spacestack}{}
\newcommand*{\si@unt@spacestacka}{}
\newcommand*{\si@unt@spacestackb}{}
%    \end{macrocode}
%\end{macro}
%\end{macro}
%\end{macro}
%\begin{macro}{\si@unt@holdspace}
%\begin{macro}{\si@unt@addstack}
%\darg{token}
% Depending on the nature of the addition, it is either held or added
% to the stack.  For the ``b'' space stack, a check is made to ensure
% that a space cannot be added before the first item.
%    \begin{macrocode}
\newcommand*{\si@unt@holdspace}[1]{%
  \renewcommand*{\si@tempa}{b}%
  \edef\si@tempb{\si@unt@checkstack}%
  \ifx\si@tempa\si@tempb
    \ifx\@empty\si@unt@stackb\@empty
    \else
      \protected@edef\si@unt@spacestackb{#1}%
    \fi
  \else
    \protected@edef\si@unt@spacestacka{#1}%
  \fi}
\newcommand*{\si@unt@addstack}[1]{%
  \expandafter\protected@edef
    \csname si@unt@stack\si@unt@checkstack\endcsname
    {\csname si@unt@stack\si@unt@checkstack\endcsname
     \csname si@unt@spacestack\si@unt@checkstack\endcsname#1}%
  \expandafter\renewcommand\expandafter*\expandafter{%
    \csname si@unt@spacestack\si@unt@checkstack\endcsname}{}}
%    \end{macrocode}
%\end{macro}
%\end{macro}
%\begin{macro}{\si@unt@stackout}
% The stack contents are actually typeset here.  First the spacing
% between units and values is added.
%    \begin{macrocode}
\newcommand*{\si@unt@stackout}{%
  \si@unt@litouttrue
  \ifsi@frac
    \expandafter\si@unt@fracout
  \else
    \expandafter\si@unt@normout
  \fi}
%    \end{macrocode}
%\end{macro}
%\begin{macro}{\si@unt@checkstack}
% Which stack is in use needs to be tested.
%    \begin{macrocode}
\newcommand*{\si@unt@checkstack}{%
  \ifsi@frac
    \ifsi@unt@per
      \expandafter\expandafter\expandafter b%
    \else
      \expandafter\expandafter\expandafter a%
    \fi
  \else
    \expandafter a%
  \fi}
%    \end{macrocode}
%\end{macro}
%\begin{macro}{\si@unt@spaceout}
% The space before a unit might not be needed, so it crops up a few
% times in the output routine.
%    \begin{macrocode}
\newcommand*{\si@unt@spaceout}{%
   \ensuremath{\si@unt@spstack}}
%    \end{macrocode}
%\end{macro}
%\begin{macro}{\si@unt@prefixout}
% If the prefix counter is not zero, then there is something to
% typeset.
%    \begin{macrocode}
\newcommand*{\si@unt@prefixout}{%
  \ifnum\si@unt@prefixcnt=\z@\relax\else
    \ifsi@unt@num
      \si@out{\ensuremath{{}\si@prefixproduct{}}}%
    \fi
    \si@unt@stackvalsep
    \let\si@expbase\si@prefixbase
    \num[fixdp=false]{e\the\si@unt@prefixcnt}%
  \fi}
%    \end{macrocode}
%\end{macro}
%\begin{macro}{\si@unt@normout}
% The normal output mode is set up here; nothing much needs to be
% done as there is no need for complex checks.
%    \begin{macrocode}
\newcommand*{\si@unt@normout}{%
  \si@unt@prefixout
  \si@unt@spaceout
  \expandafter\si@unt@out\expandafter{\si@unt@stacka}}
%    \end{macrocode}
%\end{macro}
%\begin{macro}{\si@unt@fracout}
% For fractions, some checks are needed.
%    \begin{macrocode}
\newcommand*{\si@unt@fracout}{%
  \si@unt@notambig
  \ifx\@empty\si@unt@stacka\@empty
    \ifx\@empty\si@unt@stackb\@empty
      \ifsi@unt@litout\else
        \si@log@err{Empty fractional unit}{The unit
          argument\MessageBreak given does not contain any
          symbols}%
      \fi
    \else
%    \end{macrocode}
% With an empty numerator, no space is added
%    \begin{macrocode}
      \ifsi@slash
        \si@unt@prefixout
        \si@frac{}{\si@unt@stackb}%
      \else
        \si@unt@prefixout
        \si@unt@spaceout
        \si@frac{1}{\si@unt@stackb}%
      \fi
    \fi
  \else
%    \end{macrocode}
% If the denominator is empty, then the usual output system can be
% used.
%    \begin{macrocode}
    \ifx\@empty\si@unt@stackb\@empty
      \si@unt@normout
    \else
      \si@unt@prefixout
      \si@unt@spaceout
      \si@frac{\si@unt@stacka}{\si@unt@stackb}%
    \fi
  \fi}
%    \end{macrocode}
%\end{macro}
%\begin{macro}{\si@unt@notambig}
%\begin{macro}{\si@unt@notabg}
% A trap is set for adding brackets to units using a slash, when more
% than one unit is in the denominator.
%    \begin{macrocode}
\newcommand*{\si@unt@notambig}{%
  \ifnum\si@unt@unitcntb>\@ne\relax
    \ifsi@slash
        \ifsi@trapambigfrac
          \expandafter\expandafter\expandafter\expandafter
            \expandafter\expandafter\expandafter\si@unt@notabg
        \fi
    \fi
  \fi}
\newcommand*{\si@unt@notabg}{%
  \protected@edef\si@unt@stackb{\si@openfrac\si@unt@stackb
    \si@closefrac}}
%    \end{macrocode}
%\end{macro}
%\end{macro}
%
%\begin{macro}{\si@unt@out}
%\changes{v1.0d}{2008/06/25}{Sanitise one level of unit input
%  only}
%\darg{unit}
% The final part of the units system is the output routine.  This has
% to cope with units not only as macros but also as direct input
% (\SIstyle-type input).  Non-Latin characters are also handled
% cleanly.  The \cs{scantokens} system deals with everything except
% full stops; these are left out so that a single level system can be
% used \latin{via} a token register.
%    \begin{macrocode}
\begingroup
  \catcode`\~=\active
  \catcode`\.=\active
  \gdef\si@unt@out#1{%
    \si@temptoks{#1}%
    \si@unt@fullstop
    \def.{\ensuremath{\si@unitsep}}%
    \def~{\ensuremath{\si@unitspace}}%
    \expandafter\protected@edef\expandafter\si@tempa
      \expandafter{\the\si@temptoks}%
    \begingroup
      \si@unt@nonlatin
      \makeatletter
      \endlinechar\m@ne
      \expandafter\si@out\expandafter{%
        \expandafter\scantokens\expandafter{\si@tempa}}%
    \endgroup}
\endgroup
%    \end{macrocode}
%\end{macro}
%\begin{macro}{\si@unt@fullstop}
%\begin{macro}{\si@unt@stp}
% Two macros modified from \pkg{kvsetkeys} to deal with a single
% level of active full stops only.
%    \begin{macrocode}
\begingroup
  \catcode`\.=\active
  \catcode`\&=12\relax
  \begingroup
    \lccode`\.=`\.\relax
    \lccode`\&=`\.\relax
  \lowercase{\endgroup
    \gdef\si@unt@fullstop{%
      \si@temptoks\expandafter{\expandafter}\expandafter
      \si@unt@stp\the\si@temptoks&\@nil}
  \gdef\si@unt@stp#1&#2\@nil{%
    \edef\si@tempa{\the\si@temptoks}%
    \ifx\si@tempa\@empty
      \expandafter\@firstoftwo
    \else
      \expandafter\@secondoftwo
    \fi
      {\si@temptoks{#1}}
      {\si@temptoks\expandafter{\the\si@temptoks.#1}}%
    \si@ifmtarg{#2}
      {}
      {\si@unt@stp#2\@nil}}}
\endgroup
%    \end{macrocode}
%\end{macro}
%\end{macro}
%\begin{macro}{\si@unt@nonlatin}
% To handle non-Latin symbols in the input, a single macro is
% provided.  If \XeTeX\ is in use, this can be detected immediately.
%    \begin{macrocode}
\newcommand*{\si@unt@nonlatin}{}
\ifdefined\XeTeXrevision
  \renewcommand*{\si@unt@nonlatin}{%
    \catcode176=\active
    \catcode181=\active
    \catcode197=\active
    \si@unt@sym{176}{\si@sym@degree}%
    \si@unt@sym{181}{\si@sym@mu}%
    \si@unt@sym{197}{\si@sym@ringA}}%
\fi
%    \end{macrocode}
% If \pkg{inputenc} has been loaded, then a check is made that the
% encoding is correct.  If all is well, the non-Latin symbols are
% handled.  The degree symbol is character 176, the micro
% symbol is character 181 and ring capital A is character 197 in
% latin1.
%    \begin{macrocode}
\AtBeginDocument{
  \@ifpackageloaded{inputenc}
    {\@for\si@tempa:=latin1,latin5,latin9\do{
       \ifx\inputencodingname\si@tempa
         \renewcommand*{\si@unt@nonlatin}{%
           \catcode176=\active
           \catcode181=\active
           \catcode197=\active
           \si@unt@sym{176}{\si@sym@degree}%
           \si@unt@sym{181}{\si@sym@mu}%
           \si@unt@sym{197}{\si@sym@ringA}}%
       \fi}}
    {}}
%    \end{macrocode}
%\end{macro}
%\begin{macro}{\si@unt@sym}
%\darg{charcode}
% A macro for declaring symbols: a copy of \cs{DeclareInputMath} from
% \pkg{inputenc}.
%    \begin{macrocode}
\newcommand*{\si@unt@sym}[1]{%
  \bgroup
    \uccode`\~#1%
    \uppercase{%
      \egroup
      \def~}}
%    \end{macrocode}
%\end{macro}
%
%\begin{macro}{\kilogram}
%\begin{macro}{\metre}
%\begin{macro}{\meter}
%\begin{macro}{\mole}
%\begin{macro}{\kelvin}
%\begin{macro}{\candela}
%\begin{macro}{\second}
%\begin{macro}{\ampere}
% With the system set up, the basic unit macros are implemented.  The
% only units defined whatever options are given are the base SI
% units.
%    \begin{macrocode}
\newunit{\kilogram}{kg}
\newunit{\metre}{m}
\newunit{\meter}{\metre}
\newunit{\mole}{mol}
\AtBeginDocument{\provideunit{\second}{s}}
\newunit{\ampere}{A}
\newunit{\kelvin}{K}
\newunit{\candela}{cd}
%    \end{macrocode}
%\end{macro}
%\end{macro}
%\end{macro}
%\end{macro}
%\end{macro}
%\end{macro}
%\end{macro}
%\end{macro}
%\begin{macro}{\Square}
%\changes{v1.1b}{2008/10/12}{Only defined if not otherwise available}
%\begin{macro}{\ssquare}
%\begin{macro}{\squared}
%\begin{macro}{\cubic}
%\begin{macro}{\cubed}
% Unlike multiples (which can be skipped if needed), the basic powers
% are also always defined.
%    \begin{macrocode}
\AtBeginDocument{
  \@ifundefined{Square}{%
    \newpower{\Square}{2}}{}}
\newpower{\ssquare}{2}
\newpower[post]{\squared}{2}
\newpower{\cubic}{3}
\newpower[post]{\cubed}{3}
%    \end{macrocode}
%\end{macro}
%\end{macro}
%\end{macro}
%\end{macro}
%\end{macro}
%\begin{macro}{\tothe}
%\begin{macro}{\raiseto}
% The user macros for arbitraty powers are simple
% calling macros to a common internal macro.
%    \begin{macrocode}
\newcommand*{\tothe}{\si@tothe{\tothe}}
\newcommand*{\raiseto}{\si@tothe{\raiseto}}
%    \end{macrocode}
%\end{macro}
%\end{macro}
%\begin{macro}{\si@tothe}
%\begin{macro}{\tothe@opt@si}
%\begin{macro}{\raiseto@opt@si}
%\darg{either \texttt{\cs{tothe}} or \texttt{\cs{raiseto}}}
%\darg{number}
% A macro for arbitrary powers, which comes after the unit and so
% needs to be marked as such.  Two fake option-storage macros are
% created so that everything works correctly.
%    \begin{macrocode}
\newcommand*{\si@tothe}[2]{%
  \ifsi@unt@littest
    \expandafter\@gobbletwo
  \else
    \ifsi@unt@litout
      \expandafter\expandafter\expandafter\si@unt@litpower
    \else
      \expandafter\expandafter\expandafter\si@unt@power
    \fi
  \fi
  {#1}{#2}}
\newcommand*{\tothe@opt@si}{post}
\newcommand*{\raiseto@opt@si}{}
%    \end{macrocode}
%\end{macro}
%\end{macro}
%\end{macro}
%
%\subsection{Locales}
%\begin{macro}{\si@loc@load}
%\begin{macro}{\si@loc@sisetup}
%\darg{locale}
% When loading a locale, the setup is saved rather than applied.
% Anything other than simple settings should be inside
% \cs{addtolocale}, which is already defined.
%    \begin{macrocode}
\newcommand*{\si@loc@load}[1]{%
  \let\si@loc@sisetup\sisetup
  \renewcommand*{\sisetup}[1]{%
    \expandafter\gdef\csname si@loc@#1\endcsname{##1}}%
  \si@loadfile{#1}%
  \let\sisetup\si@loc@sisetup}
%    \end{macrocode}
%\end{macro}
%\end{macro}
%\begin{macro}{\si@loc@set}
%\darg{locale}
% Setting the locale transfers the settings to \cs{sisetup}, and runs
% any extra macros.
%    \begin{macrocode}
\newcommand*{\si@loc@set}[1]{%
  \ifcsname si@loc@#1\endcsname
    \si@log@inf{Setting locale to `#1'}%
    \expandafter\expandafter\expandafter\expandafter
      \expandafter\expandafter\expandafter\si@temptoks
      \expandafter\expandafter\expandafter\expandafter
      \expandafter\expandafter\expandafter{%
        \expandafter\csname si@loc@#1\endcsname}%
    \expandafter\sisetup\expandafter{\the\si@temptoks}%
    \ifcsname si@loc@#1@extra\endcsname
      \csname si@loc@#1@extra\endcsname
    \fi
  \else
    \ifcsname si@loc@#1@extra\endcsname
      \si@log@inf{Setting locale to `#1'}%
      \csname si@loc@#1@extra\endcsname
    \else
      \si@log@warn{Unknown locale `#1'}%
    \fi
  \fi}
%    \end{macrocode}
%\end{macro}
%\begin{macro}{\si@loc@ltol}
%\changes{v1.0a}{2008/06/20}{\pkg{babel} support fixed for default
%  document language}
%\darg{list of locales and languages}
% The necessary loading for language modules occurs.
%    \begin{macrocode}
\newcommand*{\si@loc@ltol}[1]{%
  \def\si@tempa##1:##2\@nil{\si@loc@load{##1}}
  \@for\si@tempb:=#1\do{%
    \expandafter\si@tempa\si@tempb:\@nil}
  \AtBeginDocument{
    \@ifpackageloaded{babel}
      {\def\si@tempa##1:##2:##3\@nil{%
        \expandafter\addto\expandafter{%
          \csname extras##2\endcsname}%
          {\si@loc@set{##1}}}%
       \@for\si@tempb:=#1\do{%
         \expandafter\si@tempa\si@tempb::\@nil}%
       \expandafter\selectlanguage\expandafter{\languagename}}
      {\si@log@warn{babel not loaded \MessageBreak
        loctolang option ignored}}}}
\AtBeginDocument{
  \@ifpackageloaded{babel}
    {\renewcommand*{\si@loc@ltol}[1]{%
       \def\si@tempa##1:##2\@nil{\si@loc@load{##1}}%
       \@for\si@tempb:=#1\do{%
         \expandafter\si@tempa\si@tempb:\@nil}%
       \def\si@tempa##1:##2:##3\@nil{%
         \expandafter\addto\expandafter{%
           \csname extras##2\endcsname}%
           {\si@loc@set{##1}}}%
       \@for\si@tempb:=#1\do{%
         \expandafter\si@tempa\si@tempb::\@nil}}}
    {\renewcommand*{\si@loc@ltol}[1]{%
       \si@log@warn{babel not loaded \MessageBreak
         loctolang option ignored}}}}
%    \end{macrocode}
%\end{macro}
%\begin{macro}{\addtolocale}
%\darg{locale}
%\darg{commands}
% Arbitrary macros may need to be added to the locale.
%    \begin{macrocode}
\newcommand*{\addtolocale}[2]{%
  \si@addtocsname{si@loc@#1@extra}{#2}}
%    \end{macrocode}
%\end{macro}
%
%\subsection{Output routine}
%\begin{macro}{\si@out}
%\changes{v1.1}{2008/07/22}{Extended handling to subscripts}
%\darg{text}
% With all of the setup done, the text can finally be typeset.  This
% is done inside a \cs{text} block, which takes care of
% \cs{ensuremath}, \etc.  First of all, the various catcode settings
% needed to get maths-in-text mode are made.  \cs{makeatletter} is
% needed so that \cs{scantokens} still allows internal macros to
% work.
%    \begin{macrocode}
\begingroup
  \catcode`\^=\active
  \catcode`\-=\active
  \catcode`\_=\active
  \gdef\si@out#1{%
    \begingroup
      \catcode`\^=\active
      \catcode`\_=\active
      \makeatletter
      \endlinechar\m@ne
%    \end{macrocode}
%\changes{v1.0f}{2008/07/17}{Fixed bug with colour and spacing}
% The various font families can now be set up.  First a check is made
% in case there are nested calls to \cs{si@out@text}.
%    \begin{macrocode}
      \ifsi@fam@set\else
        \expandafter\si@fam@set
      \fi
      \text{%
        \si@colourcmd{\si@colour}%
        \si@fam@italic\si@fam@bold\si@fam@text
%    \end{macrocode}
%  The correct mode is selected, and the input is handed off for
%  typesetting.
%    \begin{macrocode}
        \ifsi@textmode
          \expandafter\si@out@text
        \else
          \expandafter\si@out@maths
        \fi
        {\scantokens{#1}}}%
    \endgroup
    \check@mathfonts}
%    \end{macrocode}
%\end{macro}
%\begin{macro}{\si@out@text}
%\begin{macro}{\si@out@maths}
%\darg{text}
% Output occurs with the correct changes to superscript behaviour.
%    \begin{macrocode}
  \gdef\si@out@text#1{%
    \let^\si@out@sp
    \let_\si@out@sb
    \let\textsuperscript\si@out@sp
    \let\textsubscript\si@out@sb
    \catcode`\-=\active\relax
    \let-\si@out@minus
    #1}
  \gdef\si@out@maths#1{%
    \let^\sp
    \let_\sb
    \let\textsuperscript\sp
    \let\textsubscript\sb
    $\si@fam@maths{#1}$}
\endgroup
%    \end{macrocode}
%\end{macro}
%\end{macro}
%\begin{macro}{\si@out@sp}
%\begin{macro}{\si@out@sb}
%\darg{text}
% \cs{textsuperscript} gives slightly different alignment of numbers
% to using |^| in text mode.  To avoid this, a slightly different
% definition is used.  Elsewhere \cs{textsuperscript} is used, as the
% code above sorts out the text/maths mode issues. The same is done
% for subscripts.
%    \begin{macrocode}
\newcommand*{\si@out@sp}[1]{\ensuremath{^{\text{#1}}}}
\newcommand*{\si@out@sb}[1]{\ensuremath{_{\text{#1}}}}
%    \end{macrocode}
%\end{macro}
%\end{macro}
%\begin{macro}{\si@out@minus}
%\changes{v1.1j}{2008/12/22}{Fixed infinite loop with \pkg{breqn}}
% An active minus sign is needed.
%    \begin{macrocode}
\newcommand*{\si@out@minus}{}
\DeclareTextSymbolDefault{\si@out@minus}{TS1}
\DeclareTextSymbol{\si@out@minus}{TS1}{61}
%    \end{macrocode}
%\end{macro}
%\begin{macro}{\ifsi@out@num}
% A flag is needed to control output settings.  This will be
% false unless inside \cs{si@out@num}.
%    \begin{macrocode}
\newif\ifsi@out@num
%    \end{macrocode}
%\end{macro}
%\begin{macro}{\si@out@num}
%\darg{number}
% For numerical output, the default fonts are controlled slightly
% differently to text output.
%    \begin{macrocode}
\newcommand*{\si@out@num}[1]{%
  \begingroup
    \si@out@numtrue
    \si@out{#1}%
  \endgroup}
%    \end{macrocode}
%\end{macro}
%
%\subsection{Finalisation}
%\begin{macro}{\si@extension}
%\begin{macro}{\si@fileprefix}
% To keep the code easy to maintain, the reusable filename components
% are macros rather than literals.
%    \begin{macrocode}
\newcommand*{\si@extension}{cfg}
\newcommand*{\si@fileprefix}{si-}
%    \end{macrocode}
%\end{macro}
%\end{macro}
%\begin{macro}{\si@ifl@aded}
%\begin{macro}{\si@ifloaded}
%\darg{package}
% A bit of borrowing from the \LaTeX\ kernel.  A copy of
% \cs{@ifl@aded} is needed as things aren't always done in the
% preamble by \currpkg.
%    \begin{macrocode}
\newcommand*{\si@ifl@aded}{}
\let\si@ifl@aded\@ifl@aded
\newcommand*{\si@ifloaded}[1]{%
  \si@ifl@aded\si@extension{\si@fileprefix#1}}
%    \end{macrocode}
%\end{macro}
%\end{macro}
%\begin{macro}{\si@loadfile}
%\darg{file}
% Loading configuration files is handled here.
%    \begin{macrocode}
\newcommand*{\si@loadfile}[1]{%
  \si@ifloaded{#1}{}
    {\si@InputIfFileExists{\si@fileprefix#1.\si@extension}
      {}
      {\si@log@err{Failed to load file
         \si@fileprefix#1.\si@extension}
         {The configuration file requested could not be
          found}}}}
%    \end{macrocode}
%\end{macro}
%\begin{macro}{\ifsi@outerinput}
% At various points, the package needs to know if the
% currently-read configuration file is the
% outer one.
%    \begin{macrocode}
\newif\ifsi@outerinput\si@outerinputtrue
%    \end{macrocode}
%\end{macro}
%\begin{macro}{\si@InputIfFileExists}
% To allow reading configuration files after the start of the
% document, a private copy of \cs{InputIfFileExists} is needed.  The
% main macro here simply checks if the modified version is needed.
%    \begin{macrocode}
\let\si@InputIfFileExists\InputIfFileExists
\AtBeginDocument{
  \renewcommand*{\si@InputIfFileExists}{%
    \ifsi@outerinput
      \expandafter\si@outerinput
    \else
      \expandafter\InputIfFileExists
    \fi}}
%    \end{macrocode}
%\end{macro}
%\begin{macro}{\si@outerinput}
%\darg{file}
%\darg{true-code}
%\darg{false-code}
% The outer call to load a file needs to set |@| to a letter and turn
% of printing using \cs{nullfont}.
%    \begin{macrocode}
\newcommand*{\si@outerinput}[3]{%
  \makeatletter
  \nullfont
  \si@outerinputfalse
  \IfFileExists{#1}%
    {#2\@addtofilelist{#1}\@@input \@filef@und\normalsize}
    {#3}%
  \normalsize
  \makeatother}
%    \end{macrocode}
%\end{macro}
%\begin{macro}{\requiresiconfigs}
%\darg{config-file}
% The configuration files depend on each other.
%    \begin{macrocode}
\newcommand*{\requiresiconfigs}[1]{%
  \@for\si@tempb:=#1\do{\si@loadfile{\si@tempb}}}
%    \end{macrocode}
%\end{macro}
%\begin{macro}{\si@loademfile}
%\darg{file}
% For emulation files, an additional check is made.
%    \begin{macrocode}
\newcommand*{\si@loademfile}[1]{%
  \@ifpackageloaded{#1}
    {\si@log@err{Emulation clash for package `#1'}
      {You have asked for emulation of package
       `#1'\MessageBreak
       (perhaps by giving siunitx a back-compatibility
       option)\MessageBreak but the package is already
       loaded!}}
    {\si@loadfile{#1}}}
%    \end{macrocode}
%\end{macro}
%\begin{macro}{\si@emclash}
%\darg{package}
%\darg{package}
% A macro for emulation clashes.
%    \begin{macrocode}
\newcommand*{\si@emclash}[2]{%
  \si@log@err{Emulation clash: `#1' and `#2'}
    {You have asked for emulation of package `#1'\MessageBreak
     but have already loaded emulation of `#2'}}
%    \end{macrocode}
%\end{macro}
%\begin{macro}{\si@emulating}
%\darg{package}
%\darg{version}
% For packages that are emulated, the \LaTeX\ mechanism to prevent
% re-loading is used.  The list of packages to check at the start
% of the document also has to be altered.
%    \begin{macrocode}
\newcommand*{\si@emulating}[2]{%
  \@namedef{ver@#1.sty}{#2 siunitx emulation of #1}%
  \let\si@tempa\si@blockpkgs
  \renewcommand*{\si@blockpkgs}{}%
  \@for\si@tempb:=\si@tempa\do{%
    \renewcommand*{\si@tempa}{#1}%
    \ifx\si@tempa\si@tempb\else
      \lowercase{\edef\si@tempa{#1}}%
      \lowercase{\edef\si@tempc{\si@tempb}}%
      \ifx\si@tempa\si@tempc
        \@namedef{ver@\si@tempc.sty}{#2 siunitx emulation of
          #1}%
      \else
        \si@addtolist{\si@blockpkgs}{\si@tempb}%
      \fi
    \fi}%
  \let\si@tempa\si@checkpkgs
  \renewcommand*{\si@checkpkgs}{}%
  \renewcommand*{\si@tempb}{#1}%
  \@for\si@tempc:=\si@tempa\do{%
    \ifx\si@tempb\si@tempc\else
      \si@addtolist{\si@checkpkgs}{\si@tempc}%
    \fi}}
%    \end{macrocode}
%\end{macro}
% With the \currpkg kernel macros defined, the package can now run
% through finalisation.  First, the default key values are set. The
% user options are then processed.
%    \begin{macrocode}
\sisetup{
  addsign=none,
  allowzeroexp=false,
  angformat=unchanged,
  astroang=false,%(
  closeerr=),%(
  closefrac=),
  colour=black,
  colourall=false,
  colourneg=false,
  decimalsymbol=fullstop,
  detectdisplay=true,
  digitsep=thin,
  dp=3,
  eVcorra=0.3ex,
  eVcorrb=0ex,
  errspace=none,
  fixdp=false,
  inlinebold=text,
  load=default,
  mathsrm=mathrm,
  mathssf=mathsf,
  mathstt=mathtt,
  mode=maths,
  negcolour=red,
  noload={},
  numaddn={},%(
  numcloseerr=),
  numdecimal={.,},
  numdigits=0123456789,
  numdiv=/,
  numexp=eEdD,
  numgobble={},
  numopenerr=(,%)
  numprod=x,
  numsign=+-\pm\mp,
  obeybold=false,
  obeyitalic=false,
  obeymode=false,
  openerr=(,%)
  openfrac=(,%)
  padangle=small,
  padnumber=lead,
  per=reciprocal,
  prefixbase=ten,
  prefixproduct=times,
  prefixsymbolic=true,
  prespace=false,
  redefsymbols=true,
  repeatunits=true,
  retainplus=false,
  seperr=false,
  sepfour=false,
  sign=plus,
  slash=slash,
  stickyper=false,
  strictarc=true,
  tabalignexp=true,
  tabautofit=false,
  tabformat=3.2,
  tabnumalign=centredecimal,
  tabparseonly=false,
  tabtextalign=centre,
  tabunitalign=left,
  textrm=rmfamily,
  textsf=sffamily,
  texttt=ttfamily,
  tightpm=false,
  trapambigerr=true,
  trapambigfrac=true,
  unitsep=thin,
  valuesep=thin,
  xspace=false}
\ProcessOptionsX[si]<key>
%    \end{macrocode}
% A check is now made so that emulation takes place one file at a
% time, and that each file is loaded only once.
%    \begin{macrocode}
\ifx\@empty\si@emulate\@empty\else
  \@for\si@tempa:=\si@emulate\do{%
    \si@loademfile{\si@tempa}}
\fi
%    \end{macrocode}
%\begin{macro}{\si@expanddefault}
% For turning the list of default choices into a loadable list.
%    \begin{macrocode}
\newcommand*{\si@expanddefault}[2]{%
  \expandafter\ifx\expandafter\@empty\csname si@#1\endcsname
    \@empty
  \else
    \renewcommand*{\si@tempb}{default}%
    \renewcommand*{\si@tempc}{}%
    \expandafter\@for\expandafter\si@tempa\expandafter
      :\expandafter=\csname si@#1\endcsname\do{%
      \ifx\si@tempa\si@tempb
        \si@addtolist{\si@tempc}{#2}%
      \else
        \si@addtolist{\si@tempc}{\si@tempa}%
      \fi}
    \expandafter\edef\csname si@#1\endcsname{\si@tempc}%
    \expandafter\si@addtolist\expandafter{%
      \csname si@no#1\endcsname}%
      {default}%
    \renewcommand*{\si@tempc}{}%
    \expandafter\@for\expandafter\si@tempa\expandafter
      :\expandafter=\csname si@#1\endcsname\do{%
      \si@switchfalse
      \expandafter\@for\expandafter\si@tempb\expandafter
        :\expandafter=\csname si@no#1\endcsname\do{%
        \ifx\si@tempa\si@tempb
          \si@switchtrue
        \fi
        \ifsi@switch\else
          \si@addtolist{\si@tempc}{\si@tempa}%
        \fi}}
    \@for\si@tempa:=\si@tempc\do{%
      \si@loadfile{\si@tempa}}%
  \fi}
%    \end{macrocode}
%\end{macro}
% The configuration and abbreviation files are loaded.
%    \begin{macrocode}
\si@expanddefault{load}{prefix,named,addn,prefixed,accepted,%
  physical,abbr}
%    \end{macrocode}
% The very last job is to load a local configuration file, if one
% exists, and restore catcodes.
%    \begin{macrocode}
\IfFileExists{siunitx.cfg}
  {\si@log@inf{Local configuration file found}%
   \InputIfFileExists{siunitx.cfg}{}{}}
  {}
\si@catcodes
%    \end{macrocode}
%
%\iffalse
%</package>
%\fi
%
%\section{Loadable modules}
% To keep the package relatively clear, and to make maintenance
% easier, the only units defined in the package itself are the base
% units. Everything else is a loadable module (similar to the
% approach in \unitsdef).
%
%\iffalse
%<*prefix>
%\fi
%\subsection{Multiple prefixes}
%\begin{macro}{\yocto}
%\begin{macro}{\zepto}
%\begin{macro}{\atto}
%\begin{macro}{\femto}
%\begin{macro}{\pico}
%\begin{macro}{\nano}
%\begin{macro}{\micro}
%\begin{macro}{\Micro}
%\begin{macro}{\milli}
%\begin{macro}{\centi}
%\begin{macro}{\deci}
% The various SI multiple prefixes are defined here.  First the small
% powers.
%    \begin{macrocode}
\ProvidesFile{si-prefix.cfg}
  [2008/12/22 v1.1j siunitx: SI Multiple prefixes]
\newprefix{\yocto}{-24}{y}
\newprefix{\zepto}{-21}{z}
\newprefix{\atto}{-18}{a}
\newprefix{\femto}{-15}{f}
\newprefix{\pico}{-12}{p}
\newprefix{\nano}{-9}{n}
%    \end{macrocode}
% Some testing is needed for \unitsdef compatibility.
%    \begin{macrocode}
\ifsi@old@OHM
  \newprefix{\Micro}{-6}{\si@sym@mu}
\else
  \ifsi@gensymb\else
    \newprefix{\micro}{-6}{\si@sym@mu}
  \fi
\fi
\newprefix{\milli}{-3}{m}
\newprefix{\centi}{-2}{c}
\newprefix{\deci}{-1}{d}
%    \end{macrocode}
%\end{macro}
%\end{macro}
%\end{macro}
%\end{macro}
%\end{macro}
%\end{macro}
%\end{macro}
%\end{macro}
%\end{macro}
%\end{macro}
%\end{macro}
%\begin{macro}{\deca}
%\begin{macro}{\hecto}
%\begin{macro}{\kilo}
%\begin{macro}{\mega}
%\begin{macro}{\giga}
%\begin{macro}{\tera}
%\begin{macro}{\peta}
%\begin{macro}{\exa}
%\begin{macro}{\zetta}
%\begin{macro}{\yotta}
% Now the large ones.
%    \begin{macrocode}
\newprefix{\deca}{1}{da}
\newprefix{\hecto}{2}{h}
\newprefix{\kilo}{3}{k}
\newprefix{\mega}{6}{M}
\newprefix{\giga}{9}{G}
\newprefix{\tera}{12}{T}
\newprefix{\peta}{15}{P}
\newprefix{\exa}{18}{E}
\newprefix{\zetta}{21}{Z}
\newprefix{\yotta}{24}{Y}
%    \end{macrocode}
%\end{macro}
%\end{macro}
%\end{macro}
%\end{macro}
%\end{macro}
%\end{macro}
%\end{macro}
%\end{macro}
%\end{macro}
%\end{macro}
%\begin{macro}{\deka}
% Apparently, ``deka'' is common in the US for deca.
%    \begin{macrocode}
\newprefix{\deka}{1}{da}
%    \end{macrocode}
%\end{macro}
%\begin{macro}{\gram}
%\begin{macro}{\kilogram}
% As the base unit of mass is the kilogram, rather than the gram, a
% bit of extra work is needed; by default the package only defines
% \cs{kilogram}, but with the prefixes available, this is altered to
% be \cs{kilo}\cs{gram}.  For that, the \cs{gram} must be defined first.
%    \begin{macrocode}
\newunit{\gram}{g}
\renewunit{\kilogram}{\kilo\gram}
%    \end{macrocode}
%\end{macro}
%\end{macro}
%\iffalse
%</prefix>
%<*named>
%\fi
%\subsection{Derived units with specific names}
%\begin{macro}{\becquerel}
%\begin{macro}{\coulomb}
%\begin{macro}{\farad}
%\begin{macro}{\Gray}
%\begin{macro}{\ggray}
%\begin{macro}{\hertz}
%\begin{macro}{\henry}
%\begin{macro}{\joule}
%\begin{macro}{\katal}
%\begin{macro}{\lumen}
%\begin{macro}{\lux}
%\begin{macro}{\newton}
% Derived units with specific names and symbols are defined.  Litre
% is an awkward one, but here the UK standard is used.
%    \begin{macrocode}
\ProvidesFile{si-named.cfg}
  [2008/12/22 v1.1j siunitx: SI Named units]
\newunit{\becquerel}{Bq}
\newunit{\coulomb}{C}
\newunit{\farad}{F}
\newunit{\Gray}{Gy}
\newunit{\ggray}{Gy}
\newunit{\hertz}{Hz}
\newunit{\henry}{H}
\newunit{\joule}{J}
\newunit{\katal}{kat}
\newunit{\lumen}{lm}
\newunit{\lux}{lx}
\newunit{\newton}{N}
%    \end{macrocode}
%\end{macro}
%\end{macro}
%\end{macro}
%\end{macro}
%\end{macro}
%\end{macro}
%\end{macro}
%\end{macro}
%\end{macro}
%\end{macro}
%\end{macro}
%\end{macro}
%\begin{macro}{\ohm}
%\begin{macro}{\Ohm}
%\begin{macro}{\pascal}
%\begin{macro}{\siemens}
%\begin{macro}{\sievert}
%\begin{macro}{\tesla}
%\begin{macro}{\volt}
%\begin{macro}{\watt}
%\begin{macro}{\weber}
% Some testing is needed for \unitsdef compatibility.
%    \begin{macrocode}
\ifsi@old@OHM
  \newunit{\Ohm}{\si@sym@Omega}
\else
  \ifsi@gensymb\else
%    \end{macrocode}
% To be on the safe side, use \cs{provideunit}.
%    \begin{macrocode}
    \provideunit{\ohm}{\si@sym@Omega}
  \fi
\fi
\newunit{\pascal}{Pa}
\newunit{\siemens}{S}
\newunit{\sievert}{Sv}
\newunit{\tesla}{T}
\newunit{\volt}{V}
\newunit{\watt}{W}
\newunit{\weber}{Wb}
%    \end{macrocode}
%\end{macro}
%\end{macro}
%\end{macro}
%\end{macro}
%\end{macro}
%\end{macro}
%\end{macro}
%\end{macro}
%\end{macro}
%\begin{macro}{\celsius}
%\begin{macro}{\Celsius}
% The degree celsius is a named unit.
%    \begin{macrocode}
\ifsi@old@OHM
  \newunit{\Celsius}{\si@sym@celsius}
\else
  \ifsi@gensymb\else
    \newunit{\celsius}{\si@sym@celsius}
  \fi
\fi
%    \end{macrocode}
%\end{macro}
%\end{macro}
%\begin{macro}{\radian}
%\begin{macro}{\steradian}
% The radian and steradian are officially derived units.
%    \begin{macrocode}
\newunit{\radian}{rad}
\newunit{\steradian}{sr}
%    \end{macrocode}
%\end{macro}
%\end{macro}
%\iffalse
%</named>
%<*prefixed>
%\fi
%\subsection{Units with prefixes}
% As in \unitsdef, some commonly used prefixed units are set up.
% This requires \file{si-prefix.cfg} and \file{si-named.cfg}.
%    \begin{macrocode}
\ProvidesFile{si-prefixed.cfg}
  [2008/12/22 v1.1j siunitx: SI Prefixed units]
\requiresiconfigs{prefix,named,accepted,physical}
%    \end{macrocode}
%\begin{macro}{\picometre}
%\begin{macro}{\nanometre}
%\begin{macro}{\micrometre}
%\begin{macro}{\millimetre}
%\begin{macro}{\centimetre}
%\begin{macro}{\decimetre}
%\begin{macro}{\kilometre}
% Extra distances.
%    \begin{macrocode}
\newunit{\picometre}{\pico\metre}
\newunit{\nanometre}{\nano\metre}
\newunit{\micrometre}{\micro\metre}
\newunit{\millimetre}{\milli\metre}
\newunit{\centimetre}{\centi\metre}
\newunit{\decimetre}{\deci\metre}
\newunit{\kilometre}{\kilo\metre}
%    \end{macrocode}
%\end{macro}
%\end{macro}
%\end{macro}
%\end{macro}
%\end{macro}
%\end{macro}
%\end{macro}
%\begin{macro}{\femtogram}
%\begin{macro}{\picogram}
%\begin{macro}{\nanogram}
%\begin{macro}{\microgram}
%\begin{macro}{\milligram}
% Extra masses.
%    \begin{macrocode}
\newunit{\femtogram}{\femto\gram}
\newunit{\picogram}{\pico\gram}
\newunit{\nanogram}{\nano\gram}
\newunit{\microgram}{\micro\gram}
\newunit{\milligram}{\milli\gram}
%    \end{macrocode}
%\end{macro}
%\end{macro}
%\end{macro}
%\end{macro}
%\end{macro}
%\begin{macro}{\femtomole}
%\begin{macro}{\picomole}
%\begin{macro}{\nanomole}
%\begin{macro}{\micromole}
%\begin{macro}{\millimole}
%\begin{macro}{\kilomole}
%\changes{v1.1e}{2008/11/18}{New unit}
% Now some moles.
%    \begin{macrocode}
\newunit{\femtomole}{\femto\mole}
\newunit{\picomole}{\pico\mole}
\newunit{\nanomole}{\nano\mole}
\newunit{\micromole}{\micro\mole}
\newunit{\millimole}{\milli\mole}
\newunit{\kilomole}{\kilo\mole}
%    \end{macrocode}
%\end{macro}
%\end{macro}
%\end{macro}
%\end{macro}
%\end{macro}
%\end{macro}
%\begin{macro}{\attosecond}
%\begin{macro}{\femtosecond}
%\begin{macro}{\picosecond}
%\begin{macro}{\nanosecond}
%\begin{macro}{\microsecond}
%\begin{macro}{\millisecond}
% Prefixed seconds.
%    \begin{macrocode}
\newunit{\attosecond}{\atto\second}
\newunit{\femtosecond}{\femto\second}
\newunit{\picosecond}{\pico\second}
\newunit{\nanosecond}{\nano\second}
\newunit{\microsecond}{\micro\second}
\newunit{\millisecond}{\milli\second}
%    \end{macrocode}
%\end{macro}
%\end{macro}
%\end{macro}
%\end{macro}
%\end{macro}
%\end{macro}
%\begin{macro}{\picoampere}
%\begin{macro}{\nanoampere}
%\begin{macro}{\microampere}
%\begin{macro}{\milliampere}
%\begin{macro}{\kiloampere}
% The last prefixed base units are amperes.
%    \begin{macrocode}
\newunit{\picoampere}{\pico\ampere}
\newunit{\nanoampere}{\nano\ampere}
\newunit{\microampere}{\micro\ampere}
\newunit{\milliampere}{\milli\ampere}
\newunit{\kiloampere}{\kilo\ampere}
%    \end{macrocode}
%\end{macro}
%\end{macro}
%\end{macro}
%\end{macro}
%\end{macro}
%\begin{macro}{\millivolt}
%\begin{macro}{\kilovolt}
%\begin{macro}{\milliwatt}
%\begin{macro}{\kilowatt}
%\begin{macro}{\megawatt}
%\begin{macro}{\femtofarad}
%\begin{macro}{\picofarad}
%\begin{macro}{\nanofarad}
%\begin{macro}{\microfarad}
%\begin{macro}{\millifarad}
%\begin{macro}{\millisiemens}
% More electricity-related units.
%    \begin{macrocode}
\newunit{\millivolt}{\milli\volt}
\newunit{\kilovolt}{\kilo\volt}
\newunit{\milliwatt}{\milli\watt}
\newunit{\kilowatt}{\kilo\watt}
\newunit{\megawatt}{\mega\watt}
\newunit{\femtofarad}{\femto\farad}
\newunit{\picofarad}{\pico\farad}
\newunit{\nanofarad}{\nano\farad}
\newunit{\microfarad}{\micro\farad}
\newunit{\millifarad}{\milli\farad}
\newunit{\millisiemens}{\milli\siemens}
%    \end{macrocode}
%\end{macro}
%\end{macro}
%\end{macro}
%\end{macro}
%\end{macro}
%\end{macro}
%\end{macro}
%\end{macro}
%\end{macro}
%\end{macro}
%\end{macro}
%\begin{macro}{\kilohm}
%\changes{v1.1f}{2008/11/25}{Renamed from \cs{kiloohm}}
%\begin{macro}{\megohm}
%\changes{v1.1f}{2008/11/25}{Renamed from \cs{megaohm}}
%\begin{macro}{\gigaohm}
% For resistance, checks are needed again for the definition of
% \cs{ohm}.
%    \begin{macrocode}
\ifsi@old@OHM
  \newunit{\kilohm}{\kilo\Ohm}
  \newunit{\megohm}{\mega\Ohm}
  \newunit{\gigaohm}{\giga\Ohm}
\else
  \ifsi@gensymb\else
  \newunit{\kilohm}{\kilo\ohm}
  \newunit{\megohm}{\mega\ohm}
  \newunit{\gigaohm}{\giga\ohm}
  \fi
\fi
%    \end{macrocode}
%\end{macro}
%\end{macro}
%\end{macro}
%\begin{macro}{\microlitre}
%\begin{macro}{\millilitre}
%\begin{macro}{\microliter}
%\begin{macro}{\milliliter}
%\begin{macro}{\cubicmetre}
%\begin{macro}{\cubiccentimetre}
%\begin{macro}{\centimetrecubed}
%\begin{macro}{\cubicmicrometre}
%\begin{macro}{\cubicmillimetre}
%\begin{macro}{\cubicdecimetre}
% Volumes (unlike \unitsdef, with litre and metre spelled
% correctly).  \cs{millilitre} and \cs{microlitre} are defined
% as they are the two officially-sanctioned prefixes for the litre.
%    \begin{macrocode}
\newunit{\microlitre}{\micro\litre}
\newunit{\millilitre}{\milli\litre}
\newunit{\microliter}{\micro\liter}
\newunit{\milliliter}{\milli\liter}
\newunit{\cubicmetre}{\metre\cubed}
\newunit{\cubiccentimetre}{\centi\metre\cubed}
\newunit{\centimetrecubed}{\centi\metre\cubed}
\newunit{\cubicmicrometre}{\micro\metre\cubed}
\newunit{\cubicmillimetre}{\milli\metre\cubed}
\newunit{\cubicdecimetre}{\cubic\deci\metre}
%    \end{macrocode}
%\end{macro}
%\end{macro}
%\end{macro}
%\end{macro}
%\end{macro}
%\end{macro}
%\end{macro}
%\end{macro}
%\end{macro}
%\end{macro}
%\begin{macro}{\squaremetre}
%\begin{macro}{\squarecentimetre}
%\begin{macro}{\centimetresquared}
%\begin{macro}{\squarekilometre}
% Areas, with metre spelled corrected; \cs{are} and \cs{hectare} are
% in the ``temporarily accepted'' file.
%    \begin{macrocode}
\newunit{\squaremetre}{\Square\metre}
\newunit{\squarecentimetre}{\Square\centi\metre}
\newunit{\centimetresquared}{\centi\metre\squared}
\newunit{\squarekilometre}{\Square\kilo\metre}
%    \end{macrocode}
%\end{macro}
%\end{macro}
%\end{macro}
%\end{macro}
%\begin{macro}{\millijoule}
%\begin{macro}{\kilojoule}
%\begin{macro}{\megajoule}
%\begin{macro}{\millielectronvolt}
%\begin{macro}{\kiloelectronvolt}
%\begin{macro}{\megaelectronvolt}
%\begin{macro}{\gigaelectronvolt}
%\begin{macro}{\teraelectronvolt}
%\begin{macro}{\kilowatthour}
% Some energy is needed by now!  Notice the definition of
% \cs{kilowatthour}
%    \begin{macrocode}
\newunit{\millijoule}{\milli\joule}
\newunit{\kilojoule}{\kilo\joule}
\newunit{\megajoule}{\mega\joule}
\newunit{\millielectronvolt}{\milli\electronvolt}
\newunit{\kiloelectronvolt}{\kilo\electronvolt}
\newunit{\megaelectronvolt}{\mega\electronvolt}
\newunit{\gigaelectronvolt}{\giga\electronvolt}
\newunit{\teraelectronvolt}{\tera\electronvolt}
\newunit[unitsep=none]{\kilowatthour}{\kilo\watt\hour}
%    \end{macrocode}
%\end{macro}
%\end{macro}
%\end{macro}
%\end{macro}
%\end{macro}
%\end{macro}
%\end{macro}
%\end{macro}
%\end{macro}
%\begin{macro}{\millihertz}
%\begin{macro}{\kilohertz}
%\begin{macro}{\megahertz}
%\begin{macro}{\gigahertz}
%\begin{macro}{\terahertz}
% Frequencies.
%    \begin{macrocode}
\newunit{\millihertz}{\milli\hertz}
\newunit{\kilohertz}{\kilo\hertz}
\newunit{\megahertz}{\mega\hertz}
\newunit{\gigahertz}{\giga\hertz}
\newunit{\terahertz}{\tera\hertz}
%    \end{macrocode}
%\end{macro}
%\end{macro}
%\end{macro}
%\end{macro}
%\end{macro}
%\begin{macro}{\millinewton}
%\begin{macro}{\kilonewton}
%\begin{macro}{\hectopascal}
%\begin{macro}{\megabecquerel}
%\begin{macro}{\millisievert}
% A few more from various areas.
%    \begin{macrocode}
\newunit{\millinewton}{\milli\newton}
\newunit{\kilonewton}{\kilo\newton}
\newunit{\hectopascal}{\hecto\pascal}
\newunit{\megabecquerel}{\mega\becquerel}
\newunit{\millisievert}{\milli\sievert}
%    \end{macrocode}
%\end{macro}
%\end{macro}
%\end{macro}
%\end{macro}
%\end{macro}
%\iffalse
%</prefixed>
%<*abbr>
%\fi
%\subsection{Abbreviated units}
%\begin{macro}{\pA}
%\begin{macro}{\nA}
%\begin{macro}{\micA}
%\begin{macro}{\mA}
%\begin{macro}{\kA}
% The abbreviated units are sorted in one file.  To allow
% back-compatibility with \unitsdef, each one is inside an \cs{if}
% block for the appropriate option.  First currents.
%    \begin{macrocode}
\ProvidesFile{si-abbr.cfg}
  [2008/12/22 v1.1j siunitx: Abbreviated units]
\requiresiconfigs{prefix,named,accepted,physical}
\newunit{\pA}{\pico\ampere}
\newunit{\nA}{\nano\ampere}
\newunit{\micA}{\micro\ampere}
\newunit{\mA}{\milli\ampere}
\newunit{\kA}{\kilo\ampere}
%    \end{macrocode}
%\end{macro}
%\end{macro}
%\end{macro}
%\end{macro}
%\end{macro}
%\begin{macro}{\Hz}
%\begin{macro}{\mHz}
%\begin{macro}{\kHz}
%\begin{macro}{\MHz}
%\begin{macro}{\GHz}
%\begin{macro}{\THz}
% Then frequencies.
%    \begin{macrocode}
\newunit{\Hz}{\hertz}
\newunit{\mHz}{\milli\hertz}
\newunit{\kHz}{\kilo\hertz}
\newunit{\MHz}{\mega\hertz}
\newunit{\GHz}{\giga\hertz}
\newunit{\THz}{\tera\hertz}
%    \end{macrocode}
%\end{macro}
%\end{macro}
%\end{macro}
%\end{macro}
%\end{macro}
%\end{macro}
%\begin{macro}{\fmol}
%\begin{macro}{\pmol}
%\begin{macro}{\nmol}
%\begin{macro}{\micmol}
%\begin{macro}{\mmol}
%\begin{macro}{\kmol}
%\changes{v1.1e}{2008/11/18}{New unit}
% Amounts of substance.
%    \begin{macrocode}
\newunit{\fmol}{\femto\mole}
\newunit{\pmol}{\pico\mole}
\newunit{\nmol}{\nano\mole}
\newunit{\micmol}{\micro\mole}
\newunit{\mmol}{\milli\mole}
\newunit{\kmol}{\kilo\mole}
%    \end{macrocode}
%\end{macro}
%\end{macro}
%\end{macro}
%\end{macro}
%\end{macro}
%\end{macro}
%\begin{macro}{\kV}
%\begin{macro}{\mV}
% Potentials.
%    \begin{macrocode}
\newunit{\kV}{\kilo\volt}
\newunit{\mV}{\milli\volt}
%    \end{macrocode}
%\end{macro}
%\end{macro}
%\begin{macro}{\ml}
%\begin{macro}{\micl}
%\begin{macro}{\mL}
%\begin{macro}{\micL}
%\begin{macro}{\cmc}
%\begin{macro}{\dmc}
%\begin{macro}{\cms}
% Volumes and areas
%    \begin{macrocode}
\provideunit{\ml}{\milli\litre}
\provideunit{\micl}{\micro\litre}
\provideunit{\mL}{\milli\liter}
\provideunit{\micL}{\micro\liter}
\newunit{\cmc}{\centi\metre\cubed}
\newunit{\dmc}{\deci\metre\cubed}
\newunit{\cms}{\centi\metre\squared}
%    \end{macrocode}
%\end{macro}
%\end{macro}
%\end{macro}
%\end{macro}
%\end{macro}
%\end{macro}
%\end{macro}
%\begin{macro}{\fg}
%\begin{macro}{\SIfg}
%\begin{macro}{\kg}
%\begin{macro}{\fg}
%\begin{macro}{\pg}
%\begin{macro}{\nanog}
%\begin{macro}{\micg}
%\begin{macro}{\mg}
%\begin{macro}{\amu}
% Masses.
%    \begin{macrocode}
\newunit{\kg}{\kilo\gram}
%    \end{macrocode}
% There is a name clash with \pkg{babel} here in French; hopefully
% there will not be too many complaints.
%    \begin{macrocode}
\AtBeginDocument{\provideunit{\fg}{\femto\gram}}
\newunit{\SIfg}{\femto\gram}
\newunit{\pg}{\pico\gram}
\newunit{\nanog}{\nano\gram}
\newunit{\micg}{\micro\gram}
\newunit{\mg}{\milli\gram}
\newunit{\amu}{\atomicmass}
%    \end{macrocode}
%\end{macro}
%\end{macro}
%\end{macro}
%\end{macro}
%\end{macro}
%\end{macro}
%\end{macro}
%\end{macro}
%\end{macro}
%\begin{macro}{\kJ}
%\begin{macro}{\eV}
%\begin{macro}{\meV}
%\begin{macro}{\keV}
%\begin{macro}{\MeV}
%\begin{macro}{\GeV}
%\begin{macro}{\TeV}
%\begin{macro}{\kWh}
% Energies.
%    \begin{macrocode}
\newunit{\kJ}{\kilo\joule}
\newunit{\eV}{\electronvolt}
\newunit{\meV}{\milli\electronvolt}
\newunit{\keV}{\kilo\electronvolt}
\newunit{\MeV}{\mega\electronvolt}
\newunit{\GeV}{\giga\electronvolt}
\newunit{\TeV}{\tera\electronvolt}
\newunit[unitsep=none]{\kWh}{\kilo\watt\hour}
%    \end{macrocode}
%\end{macro}
%\end{macro}
%\end{macro}
%\end{macro}
%\end{macro}
%\end{macro}
%\end{macro}
%\end{macro}
%\begin{macro}{\picom}
%\begin{macro}{\nm}
%\begin{macro}{\micm}
%\begin{macro}{\mm}
%\begin{macro}{\cm}
%\begin{macro}{\dm}
%\begin{macro}{\km}
% Lengths.
%    \begin{macrocode}
\newunit{\picom}{\pico\metre}
\newunit{\nm}{\nano\metre}
\newunit{\micm}{\micro\metre}
\newunit{\mm}{\milli\metre}
\newunit{\cm}{\centi\metre}
\newunit{\dm}{\deci\metre}
\newunit{\km}{\kilo\metre}
%    \end{macrocode}
%\end{macro}
%\end{macro}
%\end{macro}
%\end{macro}
%\end{macro}
%\end{macro}
%\end{macro}
%\begin{macro}{\Sec}
%\begin{macro}{\as}
%\begin{macro}{\fs}
%\begin{macro}{\ps}
%\begin{macro}{\ns}
%\begin{macro}{\mics}
%\begin{macro}{\ms}
% Finally, times.
%    \begin{macrocode}
\newunit{\Sec}{\second}
\newunit{\as}{\atto\second}
\newunit{\fs}{\femto\second}
%    \end{macrocode}
% The letter class (and others) define \cs{ps} for postscripts, so
% \cs{provideunit} is best here.
%    \begin{macrocode}
\provideunit{\ps}{\pico\second}
\newunit{\ns}{\nano\second}
\newunit{\mics}{\micro\second}
\newunit{\ms}{\milli\second}
%    \end{macrocode}
%\end{macro}
%\end{macro}
%\end{macro}
%\end{macro}
%\end{macro}
%\end{macro}
%\end{macro}
%\iffalse
%</abbr>
%<*addn>
%\fi
%\subsection{Additional (temporary) SI units}
%\begin{macro}{\angstrom}
%\begin{macro}{\are}
%\begin{macro}{\hectare}
%\begin{macro}{\barn}
%\begin{macro}{\BAR}
%\begin{macro}{\bbar}
%\begin{macro}{\millibar}
%\begin{macro}{\gal}
%\begin{macro}{\curie}
%\begin{macro}{\roentgen}
%\begin{macro}{\rad}
%\begin{macro}{\rem}
%\changes{v1.1c}{2008/10/22}{Uses \cs{provideunit}}
% Some units are ``temporarily'' acceptable for use in the SI system.
% These are defined here, although some are in very general use.
%    \begin{macrocode}
\ProvidesFile{si-addn.cfg}
  [2008/12/22 v1.1j siunitx: SI Additional units]
\newunit{\angstrom}{\si@sym@ringA}
\newunit{\are}{a}
\newunit{\hectare}{\hecto\are}
\newunit{\barn}{b}
\newunit{\BAR}{bar}
\newunit{\bbar}{bar}
\newunit{\millibar}{\milli\BAR}
\newunit{\gal}{Gal}
\newunit{\curie}{Ci}
\newunit{\roentgen}{R}
\newunit{\rad}{rad}
\provideunit{\rem}{rem}
%    \end{macrocode}
%\end{macro}
%\end{macro}
%\end{macro}
%\end{macro}
%\end{macro}
%\end{macro}
%\end{macro}
%\end{macro}
%\end{macro}
%\end{macro}
%\end{macro}
%\end{macro}
%\iffalse
%</addn>
%<*accepted>
%\fi
%\subsection{Units accepted for use with SI}
% The units which are accepted but do not fit in the above, plus
% \cs{percent} which seems to fit into this category.
%\begin{macro}{\minute}
%\begin{macro}{\hour}
%\begin{macro}{\Day}
%\begin{macro}{\dday}
%\begin{macro}{\degree}
%\changes{v1.1c}{2008/10/22}{Uses \cs{provideunit}}
%\begin{macro}{\Degree}
%\begin{macro}{\arcmin}
%\begin{macro}{\arcsec}
%\begin{macro}{\litre}
%\begin{macro}{\liter}
%\begin{macro}{\tonne}
%\begin{macro}{\neper}
%\begin{macro}{\bel}
%\begin{macro}{\percent}
%    \begin{macrocode}
\ProvidesFile{si-accepted.cfg}
  [2008/12/22 v1.1j siunitx: SI Accepted units]
\newunit{\minute}{min}
\newunit{\hour}{h}
\newunit{\Day}{d}
\newunit{\dday}{d}
\ifsi@old@OHM
  \AtBeginDocument{
    \provideunit[valuesep=none]{\Degree}{\si@sym@degree}
  }
\else
  \ifsi@gensymb\else
    \AtBeginDocument{
      \provideunit[valuesep=none]{\degree}{\si@sym@degree}
    }
  \fi
\fi
\newunit[valuesep=none]{\arcmin}{\si@sym@minute}
\newunit[valuesep=none]{\arcsec}{\si@sym@second}
\newunit{\litre}{l}
\newunit{\liter}{L}
\newunit{\tonne}{t}
\newunit{\neper}{Np}
\newunit{\bel}{B}
\newunit{\percent}{\%}
%    \end{macrocode}
%\end{macro}
%\end{macro}
%\end{macro}
%\end{macro}
%\end{macro}
%\end{macro}
%\end{macro}
%\end{macro}
%\end{macro}
%\end{macro}
%\end{macro}
%\end{macro}
%\end{macro}
%\end{macro}
%\iffalse
%</accepted>
%<*physical>
%\fi
%\subsection{Units based on physical measurements}
%\begin{macro}{\si@eVspacea}
%\begin{macro}{\si@eVspaceb}
%\begin{macro}{\electronvolt}
%\begin{macro}{\atomicmassunit}
%\begin{macro}{\atomicmass}
% A few units based on physical measurements exist.  For \cs{eV}, the need
% for a negative kern does make things a bit complicated.
%    \begin{macrocode}
\ProvidesFile{si-physical.cfg}
  [2008/12/22 v1.1j siunitx: SI Physically-measured units]
\newcommand*{\si@eVspacea}{\text{\kern-\si@eVcorra}}%
\newcommand*{\si@eVspaceb}{\text{\kern-\si@eVcorrb}}%
\newunit{\electronvolt}{e\protect\si@eVspacea V\protect%
  \si@eVspaceb}
\newunit{\atomicmass}{u}
\newunit{\atomicmassunit}{u}
%    \end{macrocode}
%\end{macro}
%\end{macro}
%\end{macro}
%\end{macro}
%\end{macro}
%\iffalse
%</physical>
%\fi
%
%\section{Additional configurations}
% To provide flexibility for people in specific areas, specialised
% units can be set up.  These are then stored separately to ease use.
%
%\iffalse
%<*synchem>
%\fi
%\subsection{Synthetic chemistry}
%\begin{macro}{\mmHg}
%\begin{macro}{\molar}
%\begin{macro}{\Molar}
%\begin{macro}{\torr}
%\begin{macro}{\dalton}
% Some useful units for synthetic chemists; although \cs{mmHg} and
% \cs{Molar} are outside of the SI system, they are used a lot.
% These are set up using \cs{provideunit} as people may have their
% own definitions.
%    \begin{macrocode}
\ProvidesFile{si-synchem.cfg}
  [2008/12/22 v1.1j siunitx: Units for synthetic chemists]
\requiresiconfigs{prefix}
\newunit{\mmHg}{mmHg}
\newunit{\molar}{\mole\per\cubic\deci\metre}
\newunit{\Molar}{\textsc{m}}
\newunit{\torr}{Torr}
\newunit{\dalton}{Da}
%    \end{macrocode}
%\end{macro}
%\end{macro}
%\end{macro}
%\end{macro}
%\end{macro}
%\iffalse
%</synchem>
%<*hep>
%\fi
%\subsection{High-energy physics}
% Some units inspired by \pkg{hepunits}.
%    \begin{macrocode}
\ProvidesFile{si-hep.cfg}
  [2008/12/22 v1.1j siunitx: Units for high-energy physics]
\requiresiconfigs{prefix,named}
%    \end{macrocode}
%\begin{macro}{\micron}
%\begin{macro}{\mrad}
%\begin{macro}{\gauss}
% These specific to high-energy physics, but are not defined
% elsewhere in \currpkg.\footnote{It is not clear from
% \pkg{hepunits}, but the assumption is that \cs{mrad} represents
% millirad and not milliradian.}
%    \begin{macrocode}
\provideunit{\micron}{\micro\metre}
\newunit{\mrad}{\milli\rad}
\newunit{\gauss}{G}
%    \end{macrocode}
%\end{macro}
%\end{macro}
%\end{macro}
%\begin{macro}{\clight}
%\begin{macro}{\eVperc}
% The speed of light is used in units for the area, although of
% course it is not strictly a unit.  However, this makes it easier to
% use in other units.
%    \begin{macrocode}
\newunit{\clight}{\ensuremath{\mathnormal{c}}}
\newunit{\eVperc}{\eV\per\clight}
%    \end{macrocode}
%\end{macro}
%\end{macro}
%\begin{macro}{\nanobarn}
%\begin{macro}{\picobarn}
%\begin{macro}{\femtobarn}
%\begin{macro}{\attobarn}
%\begin{macro}{\zeptobarn}
%\begin{macro}{\yoctobarn}
% Various prefixed barns.
%    \begin{macrocode}
\newunit{\nanobarn}{\nano\barn}
\newunit{\picobarn}{\pico\barn}
\newunit{\femtobarn}{\femto\barn}
\newunit{\attobarn}{\atto\barn}
\newunit{\zeptobarn}{\zepto\barn}
\newunit{\yoctobarn}{\yocto\barn}
%    \end{macrocode}
%\end{macro}
%\end{macro}
%\end{macro}
%\end{macro}
%\end{macro}
%\end{macro}
%\begin{macro}{\nb}
%\begin{macro}{\pb}
%\begin{macro}{\fb}
%\begin{macro}{\ab}
%\begin{macro}{\zb}
%\begin{macro}{\yb}
% Abbreviations for the same, but using \cs{provideunit} to avoid any
% clashes.
%    \begin{macrocode}
\provideunit{\nb}{\nano\barn}
\provideunit{\pb}{\pico\barn}
\provideunit{\fb}{\femto\barn}
\provideunit{\ab}{\atto\barn}
\provideunit{\zb}{\zepto\barn}
\provideunit{\yb}{\yocto\barn}
%    \end{macrocode}
%\end{macro}
%\end{macro}
%\end{macro}
%\end{macro}
%\end{macro}
%\end{macro}
%\iffalse
%</hep>
%<*astro>
%\fi
%\subsection{Astronomy}
%\begin{macro}{\parsec}
%\begin{macro}{\lightyear}
% For astronomy, the \cs{parsec} unit is needed.
%    \begin{macrocode}
\ProvidesFile{si-astro.cfg}
  [2008/12/22 v1.1j siunitx: Units for astronomy]
\newunit{\parsec}{pc}
\newunit{\lightyear}{ly}
%    \end{macrocode}
%\end{macro}
%\end{macro}
%\iffalse
%</astro>
%<*geophys>
%\fi
%\subsection{Geophysics}
%\begin{macro}{\lightyear}
% The unit \cs{gon} is useful here.
%    \begin{macrocode}
\ProvidesFile{si-geophys.cfg}
  [2008/12/22 v1.1j siunitx: Units for astronomy]
\newunit{\gon}{gon}
%    \end{macrocode}
%\end{macro}
%\iffalse
%</geophys>
%<*binary>
%\fi
%\subsection{Binary units}
%\begin{macro}{\kibi}
%\begin{macro}{\mebi}
%\begin{macro}{\gibi}
%\begin{macro}{\tebi}
%\begin{macro}{\pebi}
%\begin{macro}{\exbi}
%\begin{macro}{\zebi}
%\begin{macro}{\yobi}
% The binary units, as specified by the IEC and made available by
% \SIunits. First, the binary prefixes.
%    \begin{macrocode}
\ProvidesFile{si-binary.cfg}
  [2008/12/22 v1.1j siunitx: Binary units]
\newprefix[binary]{\kibi}{10}{Ki}
\newprefix[binary]{\mebi}{20}{Mi}
\newprefix[binary]{\gibi}{30}{Gi}
\newprefix[binary]{\tebi}{40}{Ti}
\newprefix[binary]{\pebi}{50}{Pi}
\newprefix[binary]{\exbi}{60}{Ei}
\newprefix[binary]{\zebi}{70}{Zi}
\newprefix[binary]{\yobi}{80}{Yi}
%    \end{macrocode}
%\end{macro}
%\end{macro}
%\end{macro}
%\end{macro}
%\end{macro}
%\end{macro}
%\end{macro}
%\end{macro}
%\begin{macro}{\bit}
%\begin{macro}{\byte}
% Now the units.
%    \begin{macrocode}
\newunit{\bit}{bit}
\newunit{\byte}{B}
%    \end{macrocode}
%\end{macro}
%\end{macro}
%
%\iffalse
%</binary>
%\fi
%
%\section{Loadable locales}
% Some short files to provide the correct settings for various places.
%
%\iffalse
%<*UK>
%\fi
%\subsection{United Kingdom}
% This is identical to the USA locale, and contains the package
% default values.\footnote{The package author is in the UK.}
%    \begin{macrocode}
\ProvidesFile{si-UK.cfg}
  [2008/12/22 v1.1j siunitx: UK locale]
\sisetup{
  unitsep=thin,
  expproduct=times,
  valuesep=thin,
  decimalsymbol=fullstop,
  digitsep=thin,
  sepfour=false}
%    \end{macrocode}
%\iffalse
%</UK>
%<*USA>
%\fi
%\subsection{United States}
% The same as for the UK.
%    \begin{macrocode}
\ProvidesFile{si-USA.cfg}
  [2008/12/22 v1.1j siunitx: USA locale]
\sisetup{
  unitsep=thin,
  expproduct=times,
  valuesep=thin,
  decimalsymbol=fullstop,
  digitsep=thin,
  sepfour=false}
%    \end{macrocode}
%\iffalse
%</USA>
%<*DE>
%\fi
%\subsection{Germany}
% Germany, hopefully.
%    \begin{macrocode}
\ProvidesFile{si-DE.cfg}
  [2008/12/22 v1.1j siunitx: Germany locale]
\sisetup{
  unitsep=cdot,
  valuesep=thin,
  decimalsymbol=comma,
  expproduct=cdot,
  digitsep=thin,
  sepfour=false}
%    \end{macrocode}
%\iffalse
%</DE>
%<*ZA>
%\fi
%\subsection{South Africa}
% Design decisions taken from the same section of \SIstyle.
%    \begin{macrocode}
\ProvidesFile{si-ZA.cfg}
  [2008/12/22 v1.1j siunitx: South Africa locale]
\sisetup{
  unitsep=cdot,
  valuesep=thin,
  expproduct=times,
  decimalsymbol=comma,
  digitsep=thin,
  sepfour=false}
%    \end{macrocode}
%\iffalse
%</ZA>
%\fi
%
%\section{Emulation code}
% Each emulation mode loads an appropriate definition file.  This
% then alters the package defaults, and defines new macros provided
% by the emulated package.
%
%\iffalse
%<*units>
%\fi
%\subsection{\units}
% The very first thing to do here is a reload check, as things
% could go wrong with \unitsdef emulation.
%    \begin{macrocode}
\si@ifloaded{units}{\endinput}{}
%    \end{macrocode}
% The \units package is quite easy to emulate, as it only has a
% few options and macros. There is also no error checking in
% \units for conflicting options, so users probably expect none.
%    \begin{macrocode}
\ProvidesFile{si-units.cfg}
  [2008/12/22 v1.1j siunitx: Emulation of units]
\si@emulating{units}{1998/08/04 v0.9b}
\si@ifloaded{SIunits}
  {\si@emclash{units}{SIunits}\endinput}{}
\si@ifloaded{sistyle}
  {\si@emclash{units}{sistyle}\endinput}{}
%    \end{macrocode}
% To emulate \units, \cs{per} must give fractions.
%    \begin{macrocode}
\sisetup{per=fraction,fraction=nice,obeybold,inlinebold=maths,
  ,obeymode}
\ifsi@old@tight
  \sisetup{valuesep=thin}
\fi
\ifsi@old@loose
  \sisetup{valuesep=space}
\fi
\ifsi@old@ugly
  \sisetup{fraction=ugly}
\fi
%    \end{macrocode}
%\begin{macro}{\unit}
%\doarg{number}
%\darg{unit}
% The \units version of \cs{unit} is similar to \cs{SI}.  Here and
% in \cs{unitfrac} the \cs{num} macro is used; thus the number given
% really has to be a number.  However, if users are using \currpkg
% rather than \units they should expect more checking of input.  As
% the \units package uses the current mode, this has to be
% detected.
%    \begin{macrocode}
\si@newrobustcmd*{\unit}[2][]{%
  \ifmmode
    \SI{#1}{#2}%
  \else
    \SI[obeyfamily,obeyitalic]{#1}{#2}%
  \fi}
%    \end{macrocode}
%\end{macro}
%\begin{macro}{\unitfrac}
%\doarg{number}
%\darg{numerator}
%\darg{denominator}
% \cs{unitfrac} is a bit more of a hack.
%    \begin{macrocode}
\si@newrobustcmd*{\unitfrac}[3][]{%
  \begingroup
    \si@fam@mode%
    \ifmmode\else
      \sisetup{obeyfamily,obeyitalic}%
    \fi
    \si@ifnotmtarg{#1}
      {\num{#1}\ensuremath{\si@valuesep}}%
    \si@frac{#2}{#3}
  \endgroup}
%    \end{macrocode}
%\end{macro}
%\iffalse
%</units>
%\fi
%\iffalse
%<*unitsdef>
%\fi
%\subsection{\unitsdef}
% The package begins with the usual identification of what is
% happening.  Although \file{si-units.cfg} makes the same checks,
% the error will make more sense if it comes here, in the event of a
% clash.
%    \begin{macrocode}
\ProvidesFile{si-unitsdef.cfg}
  [2008/12/22 v1.1j siunitx:
   Emulation of unitsdef]
\si@emulating{unitsdef}{2005/01/04 v0.2}
\si@ifloaded{SIunits}
  {\si@emclash{unitsdef}{SIunits}\endinput}{}
\si@ifloaded{sistyle}
  {\si@emclash{unitsdef}{sistyle}\endinput}{}
%    \end{macrocode}
% Emulation of \units is needed for \unitsdef to work.
%    \begin{macrocode}
\requiresiconfigs{units}
%    \end{macrocode}
% The \unitsdef package loads some packages that \currpkg does not.
% In particular, it loads \pkg{textcomp} and \pkg{fontenc}.  This
% could be important for output, and so the same is done here.
%    \begin{macrocode}
\RequirePackage{textcomp}
\RequirePackage[T1]{fontenc}
%    \end{macrocode}
% The multitude of package options for \unitsdef need to be handled.
%    \begin{macrocode}
\sisetup{mode=text,allowoptarg,prespace}
\ifsi@old@noxspace
  \sisetup{xspace=false}
\fi
%    \end{macrocode}
% The various options for loading unit abbreviations have to be
% handled. Here, any request to avoid abbreviations prevents any
% loading.
%    \begin{macrocode}
\ifsi@old@noabbr
  \sisetup{noload=abbr}
\fi
\ifsi@old@nofrequncyabbr
  \sisetup{noload=abbr}
\fi
\ifsi@old@nomolabbr
  \sisetup{noload=abbr}
\fi
\ifsi@old@novoltageabbr
  \sisetup{noload=abbr}
\fi
\ifsi@old@novolumeabbr
  \sisetup{noload=abbr}
\fi
\ifsi@old@noweightabbr
  \sisetup{noload=abbr}
\fi
\ifsi@old@noenergyabbr
  \sisetup{noload=abbr}
\fi
\ifsi@old@nolengthabbr
  \sisetup{noload=abbr}
\fi
\ifsi@old@notimeabbr
  \sisetup{noload=abbr}
\fi
%    \end{macrocode}
%\begin{macro}{\unitvaluesep}
% To emulate the \cs{unitvaluesep} macro, a hack is needed of the original
% \pkg{xkeyval} macro for \opt{valuesep}, as well of course as a definition
% of the macro itself.
%    \begin{macrocode}
\newcommand*{\unitvaluesep}{\,}
\renewcommand*{\si@valuesep}{\text{\unitvaluesep}}
\define@choicekey*+[si]{key}{valuesep}[\si@tempa]
  {space,thin,med,medium,thick,none}
  {\renewcommand*\unitvaluesep\@nameuse{si@fix@##1}%
   \si@log@debug{Option valuesep set to ##1}}
  {\si@log@debug{Option valuesep set to ##1}%
   \renewcommand*\unitvaluesep{##1}}
%    \end{macrocode}
%\end{macro}
%\begin{macro}{\unitsignonly}
%\begin{macro}{\ilu}
%\begin{macro}{\arc}
% Some rather straight-forward definitions, with just a bit of fun to
% get the spacing correct.
%    \begin{macrocode}
\let\unitsignonly\si
\si@newrobustcmd*{\ilu}[2][]{%
  \begingroup
    #1\unitvaluesep%
    \unit{#2}%
  \endgroup}
\let\arc\ang
%    \end{macrocode}
%\end{macro}
%\end{macro}
%\end{macro}
%\begin{macro}{\unitSIdef}
%\begin{macro}{\si@unitSIdef}
% The \unitsdef package uses a different approach to setting the font
% inside its version of \cs{SI}.  The problem is the same as for
% \cs{unitvaluesep}, but with the added problem that \currpkg uses
% \cs{csname} \ldots \cs{endcsname}.
%    \begin{macrocode}
\newcommand*{\unitSIdef}{\upshape}
\newcommand*{\si@unitSIdef}{\unitSIdef\selectfont}
\sisetup{textrm=si@unitSIdef}
%    \end{macrocode}
%\end{macro}
%\end{macro}
%\begin{macro}{\per}
% Rather awkwardly, \unitsdef uses \cs{per} in a different way to
% \currpkg.
%    \begin{macrocode}
\let\per\relax
\si@newrobustcmd*{\per}[2]{%
  \begingroup
    \si@xspacefalse
    \renewcommand*{\unitvaluesep}{}%
    \unitfrac{#1}{#2}%
  \endgroup}
%    \end{macrocode}
%\end{macro}
%\begin{macro}{\unittimes}
%\begin{macro}{\unitsep}
%\begin{macro}{\unitsuperscript}
% Some pretty straight-forward stuff again; notice that the automatic
% analyser for units has to be turned off for this to work.
%    \begin{macrocode}
\newcommand*{\unittimes}{\ensuremath{\cdot}}
\newcommand*{\unitsep}{\,}
\renewcommand*{\si@unt@unithook}{\si@unt@litouttrue}
\sisetup{unitsep=none}
\newcommand*{\unitsuperscript}{\tothe}
%    \end{macrocode}
%\end{macro}
%\end{macro}
%\end{macro}
%\begin{macro}{\newnosepunit}
%\begin{macro}{\renewnosepunit}
% Simple aliases.
%    \begin{macrocode}
\newcommand*{\newnosepunit}{\newunit[valuesep=none]}
\newcommand*{\renewnosepunit}{\renewunit[valuesep=none]}
%    \end{macrocode}
%\end{macro}
%\end{macro}
%\begin{macro}{\setTextOmega}
%\begin{macro}{\setMathOmega}
%\begin{macro}{\setTextmu}
%\begin{macro}{\setMathmu}
%\begin{macro}{\setTextCelsius}
%\begin{macro}{\setMathCelsius}
%\begin{macro}{\setMathDegree}
%\begin{macro}{\setTextDegree}
% Controlling symbols is a simple translation job; as only one
% setting is used by \currpkg in text mode, a bit of extra work is
% needed.
%    \begin{macrocode}
\newcommand*{\setTextOmega}[2]{%
  \renewcommand*{\si@textOmega}{%
      \begingroup
      \edef\si@tempa{\sfdefault}%
      \ifx\f@family\si@tempa
        \expandafter#2%
      \else
        \expandafter#1%
      \fi
    \endgroup}}
\newcommand*{\setMathOmega}[1]{\sisetup{mathsOmega=#1}}
\newcommand*{\setTextmu}[2]{%
  \renewcommand*{\si@textmu}{%
    \begingroup
      \edef\si@tempa{\sfdefault}%
      \ifx\f@family\si@tempa
        \expandafter#2%
      \else
        \expandafter#1%
      \fi
    \endgroup}}
\newcommand*{\setMathmu}[1]{\sisetup{mathsmu=#1}}
\newcommand*{\setTextCelsius}[2]{%
  \renewcommand*{\si@textcelsius}{%
    \begingroup
      \edef\si@tempa{\sfdefault}%
      \ifx\f@family\si@tempa
        \expandafter#2%
      \else
        \expandafter#1%
      \fi
    \endgroup}}
\newcommand*{\setMathCelsius}[1]{\sisetup{mathscelsius=#1}}
\newcommand*{\setMathDegree}[2]{%
  \renewcommand*{\si@textdegree}{%
    \begingroup%
      \edef\si@tempa{\sfdefault}%
      \ifx\f@family\si@tempa
        \expandafter#2%
      \else
        \expandafter#1%
      \fi
    \endgroup}}
\newcommand*{\setTextDegree}[1]{\sisetup{textdegree=#1}}
%    \end{macrocode}
%\end{macro}
%\end{macro}
%\end{macro}
%\end{macro}
%\end{macro}
%\end{macro}
%\end{macro}
%\end{macro}
% The \opt{ohm} and \opt{OHM} options are checked, and some sanity is
% ensured.  This needs to happen before loading the configuration
% files.
%    \begin{macrocode}
\ifsi@old@OHM
  \ifsi@old@ohm
    \si@log@inf{Both `ohm' and `OHM' options given\MessageBreak
      Using default behaviour for unitsdef}
    \expandafter\expandafter\expandafter\si@old@OHMfalse
  \fi
\fi
%    \end{macrocode}
%\begin{macro}{\liter}
%\begin{macro}{\ton}
%\begin{macro}{\days}
% Tonne is spelled as ``ton'' by \unitsdef, which is wrong in the UK
% at least (\SI{1}{ton} $=$ \SI{40}{cwt} $=$ \SI{2240}{lb}!).
%    \begin{macrocode}
\ifsi@old@liter
  \ifsi@old@LITER
    \si@log@inf{Both `liter' and `LITER' options
      given\MessageBreak Using default behaviour for unitsdef}
  \else
    \renewunit{\liter}{l}
  \fi
\fi
\newunit{\ton}{t}
\newunit{\days}{d}
%    \end{macrocode}
%\end{macro}
%\end{macro}
%\end{macro}
%\begin{macro}{\picometer}
%\begin{macro}{\nanometer}
%\begin{macro}{\micrometer}
%\begin{macro}{\millimeter}
%\begin{macro}{\centimeter}
%\begin{macro}{\decimeter}
%\begin{macro}{\kilometer}
% Extra distances.
%    \begin{macrocode}
\newunit{\picometer}{\pico\meter}
\newunit{\nanometer}{\nano\meter}
\newunit{\micrometer}{\micro\meter}
\newunit{\millimeter}{\milli\meter}
\newunit{\centimeter}{\centi\meter}
\newunit{\decimeter}{\deci\meter}
\newunit{\kilometer}{\kilo\meter}
%    \end{macrocode}
%\end{macro}
%\end{macro}
%\end{macro}
%\end{macro}
%\end{macro}
%\end{macro}
%\end{macro}
%\begin{macro}{\femtoliter}
%\begin{macro}{\picoliter}
%\begin{macro}{\nanoliter}
%\begin{macro}{\microliter}
%\begin{macro}{\milliliter}
%\begin{macro}{\centiliter}
%\begin{macro}{\deciliter}
%\begin{macro}{\hectoliter}
%\begin{macro}{\cubicmeter}
%\begin{macro}{\cubicmicrometer}
%\begin{macro}{\cubicmillimeter}
% Volumes with US spellings.
%    \begin{macrocode}
\newunit{\femtoliter}{\femto\liter}
\newunit{\picoliter}{\pico\liter}
\newunit{\nanoliter}{\nano\liter}
\newunit{\microliter}{\micro\liter}
\newunit{\milliliter}{\milli\liter}
\newunit{\centiliter}{\centi\liter}
\newunit{\deciliter}{\deci\liter}
\newunit{\hectoliter}{\hecto\liter}
\newunit{\cubicmeter}{\meter\cubed}
\newunit{\cubicmicrometer}{\micro\meter\cubed}
\newunit{\cubicmillimeter}{\milli\meter\cubed}
%    \end{macrocode}
%\end{macro}
%\end{macro}
%\end{macro}
%\end{macro}
%\end{macro}
%\end{macro}
%\end{macro}
%\end{macro}
%\end{macro}
%\end{macro}
%\end{macro}
%\begin{macro}{\squaremeter}
%\begin{macro}{\squarecentimeter}
%\begin{macro}{\squarekilometer}
%\begin{macro}{\ar}
%\begin{macro}{\hectar}
% Areas, including the mis-spellings for \cs{are} and
% \cs{hectare}.
%    \begin{macrocode}
\newunit{\squaremeter}{\Square\meter}
\newunit{\squarecentimeter}{\Square\centi\meter}
\newunit{\squarekilometer}{\Square\kilo\meter}
\newunit{\ar}{a}
\newunit{\hectar}{\hecto\ar}
%    \end{macrocode}
%\end{macro}
%\end{macro}
%\end{macro}
%\end{macro}
%\end{macro}
%\begin{macro}{\kv}
%\begin{macro}{\mv}
% The code for \unitsdef has the capitalisation wrong for \cs{kV} and
% \cs{mV}.
%    \begin{macrocode}
\ifsi@old@noabbr
\else
  \ifsi@old@novoltageabbr\else
    \newunit{\kv}{\kilo\volt}
    \newunit{\mv}{\milli\volt}
  \fi
\fi
%    \end{macrocode}
%\end{macro}
%\end{macro}
%\begin{macro}{\sek}
%\begin{macro}{\fg}
%\begin{macro}{\fl}
%\begin{macro}{\pl}
%\begin{macro}{\nl}
%\begin{macro}{\micl}
%\begin{macro}{\ml}
%\begin{macro}{\cl}
%\begin{macro}{\dl}
%\begin{macro}{\hl}
% There are some slightly different abbreviations, plus some which
% are not officially allowed.
%    \begin{macrocode}
\ifsi@old@noabbr\else
  \ifsi@old@notimeabbr\else
    \newunit{\sek}{\second}
  \fi
  \ifsi@old@noweightabbr\else
    \newunit{\fg}{\femto\gram}
  \fi
  \ifsi@old@novolumeabbr\else
    \newunit{\fl}{\femto\liter}
    \newunit{\pl}{\pico\liter}
    \newunit{\nl}{\nano\liter}
    \newunit{\micl}{\micro\liter}
    \newunit{\ml}{\milli\liter}
    \newunit{\cl}{\centi\liter}
    \newunit{\dl}{\deci\liter}
    \newunit{\hl}{\hecto\liter}
  \fi
\fi
%    \end{macrocode}
%\end{macro}
%\end{macro}
%\end{macro}
%\end{macro}
%\end{macro}
%\end{macro}
%\end{macro}
%\end{macro}
%\end{macro}
%\end{macro}
%\begin{macro}{\calory}
%\begin{macro}{\kilocalory}
% \unitsdef spells calorie incorrectly, and it is also not an SI unit.
%    \begin{macrocode}
\newunit{\calory}{cal}
\newunit{\kilocalory}{\kilo\calory}
%    \end{macrocode}
%\end{macro}
%\end{macro}
%\begin{macro}{\uBar}
% \unitsdef uses \cs{ubar} for bar.
%    \begin{macrocode}
\newunit{\uBar}{ba}
%    \end{macrocode}
%\end{macro}
%\begin{macro}{\gensymbohm}
%\begin{macro}{\gensymbcelsius}
%\begin{macro}{\gensymbmicro}
%\begin{macro}{\gensymbdegree}
% If the options relating to \pkg{gensymb} are given, then the
% package \emph{has} to be loaded.  The definitions are then renamed;
% a slight awkward feature is that the hyphen character needs to be a
% letter.  To avoid needing to worry about this again, a second
% switch is set up.
%    \begin{macrocode}
\catcode`\-=11\relax
\ifsi@old@redef-gensymb
  \expandafter\si@gensymbtrue
\fi
\catcode`\-=12\relax
\ifsi@gensymb
  \RequirePackage{gensymb}
  \AtBeginDocument{
    \let\gensymbohm\ohm
    \let\gensymbcelsius\celsius
    \let\gensymbmicro\micro
    \let\gensymbdegree\degree
    \let\ohm\@undefined
    \let\celsius\@undefined
    \let\micro\@undefined
    \let\degree\@undefined
    \ifsi@old@OHM\else
      \newunit{\ohm}{\si@sym@Omega}
      \newunit{\celsius}{\si@sym@celsius}
      \newprefix{\micro}{\si@sym@mu}{-6}
      \newunit{\degree}{\si@sym@degree}
    \fi}
\fi
%    \end{macrocode}
%\end{macro}
%\end{macro}
%\end{macro}
%\end{macro}
% The configuration files can now be loaded.
%    \begin{macrocode}
\requiresiconfigs{prefix,named,addn,accepted}
%    \end{macrocode}
% The \opt{noconfig} option could be ignored, but it costs little to
% let is be used.
%    \begin{macrocode}
\ifsi@old@noconfig\else
  \InputIfFileExists{unitsdef.cfg}
    {\si@log@inf{unitsdef config file loaded}}
    {\si@log@inf{unitsdef config file not found}}
\fi
%    \end{macrocode}
%
%\iffalse
%</unitsdef>
%<*sistyle>
%\fi
%\subsection{\SIstyle}
% After setting the necessary defaults, the emulation code defines
% the macros in \SIstyle as given in the manual for that package.
%    \begin{macrocode}
\ProvidesFile{si-sistyle.cfg}
  [2008/12/22 v1.1j siunitx: Emulation of
     SIstyle]
\si@emulating{sistyle}{2006/12/20 v2.3}
\sisetup{%
  sepfour=true,
  obeyfamily,
  obeyitalic=true,
  numsign=+-,
  strictarc=false,
  unitsep=cdot}
%    \end{macrocode}
%\begin{macro}{\SIobeyboldtrue}
%\begin{macro}{\SIobeyboldfalse}
% Some simple switches, but not using \cs{newif}.
%    \begin{macrocode}
\newcommand*{\SIobeyboldtrue}{\sisetup{obeybold=true}}
\newcommand*{\SIobeyboldfalse}{\sisetup{obeybold=false}}
%    \end{macrocode}
%\end{macro}
%\end{macro}
%\begin{macro}{\num}
%\begin{macro}{\si@sis@num}
%\begin{macro}{\si@sis@numstar}
% To get the correct behaviour for \cs{num}, some redefinitions are
% needed to handle to optional |*|.
%    \begin{macrocode}
\let\num\relax
\si@newrobustcmd*{\num}{%
  \@ifstar
    {\si@sis@numstar}
    {\si@sis@num}}
\newcommand*{\si@sis@num}[2][]{%
  \begingroup%
    \sisetup{#1}%
    \expandafter\si@out@num\expandafter{\si@num{#2}}%
  \endgroup}
\newcommand*{\si@sis@numstar}[2][]{%
  \begingroup%
    \sisetup{mode=text,obeybold}%
    \sisetup{#1}%
    \expandafter\si@out@num\expandafter{\si@num{#2}}%
  \endgroup}
%    \end{macrocode}
%\end{macro}
%\end{macro}
%\end{macro}
%\begin{macro}{\pnt}
% The \cs{pnt} macro is needed as |.| is active inside \cs{SI}.  The
% name is exactly the same as in \SIstyle, but the implementation is
% different.  This is not defined by the main package as there are
% better ways of including numbers in the output than this.
%    \begin{macrocode}
\newcommand*{\pnt}{\ensuremath{\si@decimalsymbol}}
%    \end{macrocode}
%\end{macro}
%\begin{macro}{\SIgroupfourtrue}
%\begin{macro}{\SIgroupfourfalse}
% Switches for grouping four characters.
%    \begin{macrocode}
\newcommand*{\SIgroupfourtrue}{\sisetup{sepfour=true}}
\newcommand*{\SIgroupfourfalse}{\sisetup{sepfour=false}}
%    \end{macrocode}
%\end{macro}
%\end{macro}
%\begin{macro}{\SIunitsep}
%\begin{macro}{\SIunitspace}
%\begin{macro}{\SIunitdot}
% Whatever is given here is passed through to \cs{sisetup}.
%    \begin{macrocode}
\newcommand*{\SIunitsep}[1]{\sisetup{valuesep={#1}}}
\newcommand*{\SIunitspace}[1]{\sisetup{unitspace={#1}}}
\newcommand*{\SIunitdot}[1]{\sisetup{unitsep={#1}}}
%    \end{macrocode}
%\end{macro}
%\end{macro}
%\end{macro}
%\begin{macro}{\SIdecimalsymbol}
%\begin{macro}{\SIthousandsep}
%\begin{macro}{\SIproductsign}
%\begin{macro}{\SIdecimalsign}
% The same is true here, with the appropriate translation.
%    \begin{macrocode}
\newcommand*{\SIdecimalsymbol}[1]{\sisetup{decimalsymbol={#1}}}
\newcommand*{\SIthousandsep}[1]{\sisetup{digitsep={#1}}}
\newcommand*{\SIproductsign}[1]{\sisetup{expproduct={#1}}}
\newcommand*{\SIdecimalsign}[1]{\sisetup{decimalsymbol={#1}}}
%    \end{macrocode}
%\end{macro}
%\end{macro}
%\end{macro}
%\end{macro}
%\begin{macro}{\si@sis@savefont}
%\darg{setting}
%\darg{argument}
% The font definitions need a bit of extra work doing.  As both
% settings here have |@| as a letter, all should be fine.
%    \begin{macrocode}
\newcommand*{\si@sis@savefont}[2]{%
  \@namedef{si@sis@#1}{#2}%
  \sisetup{#1=si@sis@#1}}
%    \end{macrocode}
%\end{macro}
%\begin{macro}{\SImathrm}
%\begin{macro}{\SImathsf}
%\begin{macro}{\SImathtt}
% The font control macros have to ensure that a macro name is passed
% to \cs{sisetup}.
%    \begin{macrocode}
\newcommand*{\SImathrm}[1]{\si@sis@savefont{mathrm}{#1}}
\newcommand*{\SImathsf}[1]{\si@sis@savefont{mathsf}{#1}}
\newcommand*{\SImathtt}[1]{\si@sis@savefont{mathtt}{#1}}
%    \end{macrocode}
%\end{macro}
%\end{macro}
%\end{macro}
%\begin{macro}{\SIdefaultMfam}
%\begin{macro}{\SIdefaultNfam}
%\begin{macro}{\SIdefaultTfam}
% The same for the default keys.
%    \begin{macrocode}
\newcommand*{\SIdefaultMfam}[1]{\si@sis@savefont{mathrm}{#1}}
\newcommand*{\SIdefaultNfam}[1]{\si@sis@savefont{mathnumrm}{#1}}
\newcommand*{\SIdefaultTfam}[1]{\si@sis@savefont{textrm}{#1}}
%    \end{macrocode}
%\end{macro}
%\end{macro}
%\end{macro}
%\begin{macro}{\ensureupmath}
% The \cs{ensureupmath} command guarantees processing by the
% font-matching system. The argument cannot be processed here, so
% care is needed.
%    \begin{macrocode}
\si@newrobustcmd*{\ensureupmath}[1]{%
  \begingroup
    \sisetup{mode=maths,obeyitalic=false}%
    \si@out{#1}%
  \endgroup}
%    \end{macrocode}
%\end{macro}
%\begin{macro}{\degC}
%\begin{macro}{\degF}
%\begin{macro}{\arcdeg}
% A few extra symbol names are needed.
%    \begin{macrocode}
\newcommand*{\degC}{\si@sym@celsius}
\newcommand*{\arcdeg}{\si@sym@degree}
\newcommand*{\degF}{\si@sym@degree F}
%    \end{macrocode}
%\end{macro}
%\end{macro}
%\end{macro}
%\begin{macro}{\AddToSIstyle}
%\begin{macro}{\SIstyle}
%\begin{macro}{\SIstyleToLang}
%\begin{macro}{\si@sis@addtolocale}
% Finally, the locale control.
%    \begin{macrocode}
\newcommand*{\SIstyle}[1]{\sisetup{locale=#1}}
\newcommand*{\SIstyleToLang}[2]{\sisetup{loctolang=#1:#2}}
\newcommand*{\AddToSIstyle}{%
  \si@switchfalse
  \@ifstar
    {\si@switchtrue
     \si@sis@addtolocale}
    {\si@sis@addtolocale}}
\newcommand*{\si@sis@addtolocale}[2]{%
  \ifsi@switch
    \expandafter\let\csname si@loc@#1@extra\endcsname\relax
  \fi
  \addtolocale{#1}{#2}}
%    \end{macrocode}
%\end{macro}
%\end{macro}
%\end{macro}
%\end{macro}
%
%\iffalse
%</sistyle>
%<*SIunits>
%\fi
%\subsection{\SIunits}
% \SIunits emulation starts in much the same way.
%    \begin{macrocode}
\ProvidesFile{si-SIunits.cfg}
  [2008/12/22 v1.1j siunitx: Emulation of
     SIunits]
\si@emulating{SIunits}{2007/12/02 v1.36}
\sisetup{
  unitsep=thick,
  valuesep=thick,
  prefixproduct=\si@valuesep,
  trapambigfrac=false,
  stickyper}
\requiresiconfigs{prefix,named,accepted,physical}
%    \end{macrocode}
%\begin{macro}{\reciprocal}
%\begin{macro}{\rp}
%\begin{macro}{\per}
%\begin{macro}{\usk}
%\begin{macro}{\power}
%\begin{macro}{\rpsquare}
%\begin{macro}{\rpcubic}
%\begin{macro}{\fourth}
%\begin{macro}{\rpfourth}
% A few very simple translations, using the internal version of
% \cs{per} to allow changes of output style.
%    \begin{macrocode}
\newcommand*{\reciprocal}{\sisetup{per=reciprocal}\si@per}
\let\rp\reciprocal
\renewcommand*{\per}{\sisetup{per=slash}\si@per}
\newcommand*{\usk}{}
\newcommand*{\power}[1]{#1\tothe}
\newcommand*{\rpsquare}{\sisetup{per=reciprocal}\si@per\Square}
\newcommand*{\rpcubic}{\sisetup{per=reciprocal}\si@per\cubic}
\newpower{\fourth}{4}
\newcommand*{\rpfourth}{\sisetup{per=reciprocal}\si@per\fourth}
%    \end{macrocode}
%\end{macro}
%\end{macro}
%\end{macro}
%\end{macro}
%\end{macro}
%\end{macro}
%\end{macro}
%\end{macro}
%\end{macro}
%\begin{macro}{\rpsquared}
%\begin{macro}{\rpcubed}
% Here, some low-level switch changing is needed.
%    \begin{macrocode}
\newcommand*{\rpsquared}{%
  \sisetup{per=reciprocal}\si@unt@pertrue\si@unt@perseentrue%
    \squared}
\newcommand*{\rpcubed}{%
  \sisetup{per=reciprocal}\si@unt@pertrue\cubed}
%    \end{macrocode}
%\end{macro}
%\end{macro}
%\begin{macro}{\SIsetup}
%\begin{macro}{\si@siu@setup}
% The various package spacing options are processed.  They also have
% to be correctly handled by the \cs{SIsetup} macro.
%    \begin{macrocode}
\newcommand*{\SIsetup}[1]{%
  \@for\si@tempa:=#1\do{%
    \@ifundefined{ifsi@old@#1}
      {\si@log@warn{Unknown SIunits option `#1'}}
      {\csname si@old@#1true\endcsname}}
  \si@siu@setup}
\newcommand*{\si@siu@setup}{%
  \ifsi@old@cdot
    \sisetup{unitsep=cdot}%
  \fi
  \ifsi@old@thickspace
    \sisetup{unitsep=thick}%
  \fi
  \ifsi@old@mediumspace
    \sisetup{unitsep=medium}%
  \fi
  \ifsi@old@thinspace
    \sisetup{unitsep=thin}%
  \fi
  \ifsi@old@thickqspace
    \sisetup{valuesep=thick}%
  \fi
  \ifsi@old@mediumqspace
    \sisetup{valuesep=medium}%
  \fi
  \ifsi@old@thinqspace
    \sisetup{valuesep=thin}%
  \fi}
\si@siu@setup
%    \end{macrocode}
%\end{macro}
%\end{macro}
%\begin{macro}{\square}
%\begin{macro}{\squaren}
% \SIunits does slightly different things about the clash with
% \cs{square}, and either redefines this macro or provides
% \cs{squaren}.
%    \begin{macrocode}
\ifsi@old@squaren
  \newpower{\squaren}{2}
\fi
\AtBeginDocument{%
  \@ifundefined{square}
    {\newpower{\square}{2}}
    {\ifsi@old@amssymb
       \renewpower{\square}{2}
     \else
       \ifsi@old@squaren\else
         \si@log@warn{\string\square\space already
           defined\MessageBreak SIunits mode may cause
           errors}%
       \fi
     \fi}}
%    \end{macrocode}
%\end{macro}
%\end{macro}
%\begin{macro}{\gray}
% The potential clash with \pkg{PStricks} is also handled
% differently; here, \cs{Gray} will already be defined by the
%  \currpkg kernel.
%    \begin{macrocode}
\AtBeginDocument{
  \@ifundefined{gray}
    {\newunit{\gray}{Gy}}
    {\ifsi@old@pstricks
       \renewunit{\gray}{Gy}
     \else
       \ifsi@old@Gray\else
         \si@log@warn{\string\gray\space already
           defined\MessageBreak SIunits mode may cause
           errors}%
       \fi
     \fi}}
%    \end{macrocode}
%\end{macro}
%\begin{macro}{\unit}
%\begin{macro}{\unita}
% The \cs{unit} macro is defined.
%    \begin{macrocode}
\ifsi@old@italian
  \let\unita\SI
\else
  \let\unit\SI
\fi
%    \end{macrocode}
%\end{macro}
%\end{macro}
% The miscellaneous options are moped up.
%    \begin{macrocode}
\ifsi@old@textstyle
  \sisetup{mode=text}
\fi
\ifsi@old@binary
  \sisetup{alsoload=binary}
\fi
\ifsi@old@noams
  \AtBeginDocument{%
    \renewcommand*{\si@textmu}{\ensuremath\si@mathsmu}}
\fi
%    \end{macrocode}
%\begin{macro}{\arcminute}
%\begin{macro}{\arcsecond}
%\begin{macro}{\rperminute}
%\begin{macro}{\ton}
%\begin{macro}{\degreecelsius}
% The unit macros defined by \SIunits that are not defined by
% \currpkg (by default).
%    \begin{macrocode}
\newunit[valuesep=none]{\arcminute}{\si@sym@minute}
\newunit[valuesep=none]{\arcsecond}{\si@sym@second}
\newunit{\rperminute}{r/min}
\newunit{\ton}{t}
\newunit{\degreecelsius}{\celsius}
%    \end{macrocode}
%\end{macro}
%\end{macro}
%\end{macro}
%\end{macro}
%\end{macro}
%\begin{macro}{\addunit}
% This is an alias for \cs{newunit}.
%    \begin{macrocode}
\let\addunit\newunit
%    \end{macrocode}
%\end{macro}
%\begin{macro}{\addprefix}
% A little more work for \cs{addprefix}.
%    \begin{macrocode}
\newcommand*{\addprefix}[2]{\newprefix{#1}{#2}}
%    \end{macrocode}
%\end{macro}
%\begin{macro}{\si@siu@newunit}
%\changes{v1.1a}{2008/09/30}{Corrected very minor bug in definition}
%\begin{macro}{\si@siu@power}
%\begin{macro}{\si@siu@newunithook}
%\doarg{power}
%\darg{numerator}
%\darg{denominator}
% To save some code, making new units which need a \texttt{\ldots np}
% variant is handled by a dedicated macro.
%    \begin{macrocode}
\newcommand*{\si@siu@newunit}[3][]{%
%    \end{macrocode}
% A test is needed to sort out \cs{square}.
%    \begin{macrocode}
  \renewcommand*{\si@tempa}{#1}%
  \renewcommand*{\si@tempb}{square}%
  \def\si@siu@power{}%
  \ifx\@empty\si@tempa\@empty\else
    \ifx\si@tempa\si@tempb
      \renewcommand*{\si@siu@power}{\ssquare}%
    \else
      \edef\si@siu@power{%
        \expandafter\noexpand\csname #1\endcsname}%
    \fi
  \fi
%    \end{macrocode}
% The necessary information is now stored in temporary macros.
%    \begin{macrocode}
  \edef\si@tempa{%
    \expandafter\noexpand\csname #2per#1#3\endcsname}%
  \edef\si@tempb{%
    \expandafter\noexpand\csname #2\endcsname\noexpand\per
    \expandafter\noexpand\si@siu@power
    \expandafter\noexpand\csname #3\endcsname}%
  \expandafter\expandafter\expandafter\newunit\expandafter%
    \expandafter\expandafter{\expandafter\si@tempa\expandafter}%
      \expandafter{\si@tempb}
  \edef\si@tempa{%
    \expandafter\noexpand\csname #2per#1#3np\endcsname}%
  \edef\si@tempb{%
    \expandafter\noexpand\csname #2\endcsname\noexpand
    \reciprocal\expandafter\noexpand\si@siu@power
    \expandafter\noexpand\csname #3\endcsname}%
  \expandafter\expandafter\expandafter\newunit\expandafter
    \expandafter\expandafter{\expandafter\si@tempa\expandafter}%
      \expandafter{\si@tempb}%
  \si@siu@newunithook[#1]{#2}{#3}}
\providecommand*{\si@siu@newunithook}[3][]{}
%    \end{macrocode}
%\end{macro}
%\end{macro}
%\end{macro}
% The basic units are now defined; these only have a single csname in
% each of the numerator and denominator.
%    \begin{macrocode}
\si@siu@newunit{gray}{second}
\si@siu@newunit[square]{metre}{second}
\si@siu@newunit{joule}{mole}
\si@siu@newunit[cubic]{mole}{metre}
\si@siu@newunit[square]{radian}{second}
\si@siu@newunit{radian}{second}
\si@siu@newunit[cubic]{squaremetre}{metre}
\si@siu@newunit[cubic]{katal}{metre}
\si@siu@newunit{coulomb}{mol}
\si@siu@newunit[square]{ampere}{metre}
\si@siu@newunit[cubic]{kilogram}{metre}
\si@siu@newunit[cubic]{coulomb}{metre}
\si@siu@newunit{volt}{metre}
\si@siu@newunit[square]{coulomb}{squaremetre}
\si@siu@newunit{farad}{metre}
\si@siu@newunit[square]{watt}{metre}
\si@siu@newunit[square]{joule}{metre}
\si@siu@newunit[cubic]{newton}{metre}
\si@siu@newunit{newton}{kilogram}
\si@siu@newunit{joule}{kelvin}
\si@siu@newunit{joule}{kilogram}
\si@siu@newunit{coulomb}{kilogram}
\si@siu@newunit{squaremetre}{second}
\si@siu@newunit[square]{squaremetre}{second}
\si@siu@newunit[square]{candela}{metre}
\si@siu@newunit{ampere}{metre}
\si@siu@newunit{joule}{tesla}
\si@siu@newunit{henry}{metre}
\si@siu@newunit{kilogram}{second}
\si@siu@newunit[square]{kilogram}{metre}
\si@siu@newunit{kilogram}{metre}
\si@siu@newunit[square]{newton}{metre}
\si@siu@newunit{watt}{kilogram}
\si@siu@newunit[cubic]{watt}{metre}
\si@siu@newunit{squaremetre}{kilogram}
\si@siu@newunit{cubicmetre}{kilogram}
\si@siu@newunit{newton}{metre}
\si@siu@newunit[cubic]{squaremetre}{second}
\si@siu@newunit{metre}{second}
\si@siu@newunit[cubic]{joule}{metre}
\si@siu@newunit{cubicmetre}{second}
%    \end{macrocode}
%\begin{macro}{\si@siu@newunitx}
%\begin{macro}{\si@siu@newunitxhook}
% For the more complex units, a slightly different approach is used;
% four arguments are required, and have to cover everything.
%    \begin{macrocode}
\newcommand*{\si@siu@newunitx}[4]{%
  \expandafter\newunit\expandafter{\csname #1per#2\endcsname}
    {#3\per#4}%
  \expandafter\newunit\expandafter{\csname #1per#2np\endcsname}
    {#3\reciprocal#4}
  \si@siu@newunitxhook{#1}{#2}{#3}{#4}}
\providecommand*{\si@siu@newunitxhook}[4]{}
%    \end{macrocode}
%\end{macro}
%\end{macro}
% The units are defined.
%    \begin{macrocode}
\si@siu@newunitx{kilogramsquaremetre}{second}
  {\kilogram\squaremetre}{\second}
\si@siu@newunitx{squaremetre}{newtonsecond}{\squaremetre}
  {\newton\second}
\si@siu@newunitx{kilogrammetre}{second}{\kilogram\metre}
  {\second}
\si@siu@newunitx{kilogram}{squaremetresecond}{\kilogram}
  {\squaremetre\second}
\si@siu@newunitx{joule}{molekelvin}{\joule}{\mole\kelvin}
\si@siu@newunitx{kilogram}{kilomole}{\kilogram}{\kilo\mole}
\si@siu@newunitx{kilogrammetre}{squaresecond}{\kilogram\metre}
  {\second\squared}
\si@siu@newunitx{watt}{squaremetresteradian}{\watt}
  {\squaremetre\steradian}
\si@siu@newunitx{joule}{kilogramkelvin}{\joule}
  {\kilogram\kelvin}
\si@siu@newunitx{watt}{metrekelvin}{\watt}{\metre\kelvin}
\si@siu@newunitx{kilogram}{cubicmetrecoulomb}{\kilogram}
  {\cubic\metre\coulomb}
\si@siu@newunitx{kilogram}{secondcubicmetre}{\kilogram}
  {\second\cubicmetre}
%    \end{macrocode}
%\begin{macro}{\si@siu@unity}
% A bit of cleverness to get the ``1'' correct; to avoid any clash, the unit
% is given an internal name.
%    \begin{macrocode}
\newunit{\si@siu@unity}{1}
\si@siu@newunitx{}{squaremetresecond}{\si@siu@unity}
  {\squaremetre\second}
%    \end{macrocode}
%\end{macro}
% A few compound units that are best defined directly.
%    \begin{macrocode}
\newunit{\pascalsecond}{\pascal\second}
\newunit{\amperemetresecond}{\ampere\metre\second}
\newunit{\ohmmetre}{\ohm\metre}
\newunit{\newtonmetre}{\newton\metre}
\let\newtonmetrenp\newtonmetre
\newunit{\kilogramsquaremetre}{\kilogram\squaremetre}
\let\kilogramsquaremetrenp\kilogramsquaremetre
%    \end{macrocode}
%\begin{macro}{\si@siu@newprefix}
%\darg{prefix}
% To generate the prefixes correctly, a small saving in repetition.
%    \begin{macrocode}
\newcommand*{\si@siu@newprefix}[1]{%
  \edef\si@tempa{\expandafter\noexpand\csname #1d\endcsname}%
  \edef\si@tempb{\expandafter\noexpand\csname #1\endcsname}%
  \expandafter\expandafter\expandafter\newcommand\expandafter
    \expandafter\expandafter*\expandafter\expandafter
    \expandafter{\expandafter\si@tempa\expandafter}\expandafter
      {\expandafter\si@prefixsymbolicfalse\si@tempb}}
%    \end{macrocode}
%\end{macro}
% This is now implemented.
%    \begin{macrocode}
\si@siu@newprefix{yocto}
\si@siu@newprefix{zepto}
\si@siu@newprefix{atto}
\si@siu@newprefix{femto}
\si@siu@newprefix{pico}
\si@siu@newprefix{nano}
\si@siu@newprefix{micro}
\si@siu@newprefix{milli}
\si@siu@newprefix{centi}
\si@siu@newprefix{deca}
\si@siu@newprefix{deka}
\si@siu@newprefix{hecto}
\si@siu@newprefix{kilo}
\si@siu@newprefix{mega}
\si@siu@newprefix{giga}
\si@siu@newprefix{tera}
\si@siu@newprefix{peta}
\si@siu@newprefix{exa}
\si@siu@newprefix{zetta}
\si@siu@newprefix{yotta}
\ifsi@old@binary
  \si@siu@newprefix{kibi}
  \si@siu@newprefix{mebi}
  \si@siu@newprefix{gibi}
  \si@siu@newprefix{tebi}
  \si@siu@newprefix{pebi}
  \si@siu@newprefix{exbi}
\fi
%    \end{macrocode}
%\begin{macro}{\derradian}
%\begin{macro}{\dersteradian}
%\begin{macro}{\derhertz}
%\begin{macro}{\dernewton}
%\begin{macro}{\derpascal}
%\begin{macro}{\derjoule}
%\begin{macro}{\derwatt}
%\begin{macro}{\dercoulomb}
%\begin{macro}{\dervolt}
%\begin{macro}{\derfarad}
%\begin{macro}{\derohm}
% The derived units may need to be defined.
%    \begin{macrocode}
\ifsi@old@derived
  \newunit{\derradian}{\metre\reciprocal\metre}
  \newunit{\dersteradian}{\squaremetre\rpsquare\metre}
  \newunit{\derhertz}{\reciprocal\second}
  \newunit{\dernewton}{\metre\kilogram\second\rpsquared}
  \newunit{\derpascal}{\newton\rpsquare\metre}
  \newunit{\derjoule}{\newton\metre}
  \newunit{\derwatt}{\joule\reciprocal\second}
  \newunit{\dercoulomb}{\ampere\second}
  \newunit{\dervolt}{\watt\reciprocal\ampere}
  \newunit{\derfarad}{\coulomb\reciprocal\volt}
  \newunit{\derohm}{\volt\reciprocal\ampere}
%    \end{macrocode}
%\end{macro}
%\end{macro}
%\end{macro}
%\end{macro}
%\end{macro}
%\end{macro}
%\end{macro}
%\end{macro}
%\end{macro}
%\end{macro}
%\end{macro}
%\begin{macro}{\dersiemens}
%\begin{macro}{\derweber}
%\begin{macro}{\dertesla}
%\begin{macro}{\derhenry}
%\begin{macro}{\dercelsius}
%\begin{macro}{\derlumen}
%\begin{macro}{\derlux}
%\begin{macro}{\derbecquerel}
%\begin{macro}{\dergray}
%\begin{macro}{\dersievert}
%\begin{macro}{\derkatal}
% In two blocks!
%    \begin{macrocode}
  \newunit{\dersiemens}{\ampere\reciprocal\volt}
  \newunit{\derweber}
    {\squaremetre\kilogram\second\rpsquared\reciprocal\ampere}
  \newunit{\dertesla}{\weber\rpsquare\metre}
  \newunit{\derhenry}{\weber\reciprocal\ampere}
  \newunit{\dercelsius}{\kelvin}
  \newunit{\derlumen}{\candela\steradian}
  \newunit{\derlux}{\lumen\rpsquare\metre}
  \newunit{\derbecquerel}{\derhertz}
  \newunit{\dergray}{\joule\reciprocal\kilogram}
  \newunit{\dersievert}{\dergray}
  \newunit{\derkatal}{\rp\second\usk\mole}
\fi
%    \end{macrocode}
%\end{macro}
%\end{macro}
%\end{macro}
%\end{macro}
%\end{macro}
%\end{macro}
%\end{macro}
%\end{macro}
%\end{macro}
%\end{macro}
%\end{macro}
%\begin{macro}{\radianbase}
%\begin{macro}{\steradianbase}
%\begin{macro}{\hertzbase}
%\begin{macro}{\newtonbase}
%\begin{macro}{\pascalbase}
%\begin{macro}{\joulebase}
%\begin{macro}{\wattbase}
%\begin{macro}{\coulombbase}
%\begin{macro}{\voltbase}
%\begin{macro}{\faradbase}
%\begin{macro}{\ohmbase}
% Also the ``derived-in-base''.
%    \begin{macrocode}
\ifsi@old@derivedinbase
  \newunit{\radianbase}{\metre\reciprocal\metre}
  \newunit{\steradianbase}{\squaremetre\rpsquare\metre}
  \newunit{\hertzbase}{\reciprocal\second}
  \newunit{\newtonbase}{\metre\kilogram\second\rpsquared}
  \newunit{\pascalbase}{\reciprocal\metre\kilogram\second%
    \rpsquared}
  \newunit{\joulebase}{\squaremetre\kilogram\second\rpsquared}
  \newunit{\wattbase}{\squaremetre\kilogram\rpcubic\second}
  \newunit{\coulombbase}{\ampere\second}
  \newunit{\voltbase}
    {\squaremetre\kilogram\rpcubic\second\reciprocal\ampere}
  \newunit{\faradbase}
    {\rpsquare\metre\reciprocal\kilogram\fourth\second\ampere%
      \squared}
  \newunit{\ohmbase}
    {\squaremetre\kilogram\rpcubic\second\rpsquare\ampere}
%    \end{macrocode}
%\end{macro}
%\end{macro}
%\end{macro}
%\end{macro}
%\end{macro}
%\end{macro}
%\end{macro}
%\end{macro}
%\end{macro}
%\end{macro}
%\end{macro}
%\begin{macro}{\siemensbase}
%\begin{macro}{\weberbase}
%\begin{macro}{\teslabase}
%\begin{macro}{\henrybase}
%\begin{macro}{\celsiusbase}
%\begin{macro}{\lumenbase}
%\begin{macro}{\luxbase}
%\begin{macro}{\becquerelbase}
%\begin{macro}{\graybase}
%\begin{macro}{\sievertbase}
%\begin{macro}{\katalbase}
% Also in two blocks.
%    \begin{macrocode}
  \newunit{\siemensbase}
    {\rpsquare\metre\reciprocal\kilogram\cubic\second\ampere%
       \squared}
  \newunit{\weberbase}
    {\squaremetre\kilogram\second\rpsquared\reciprocal\ampere}
  \newunit{\teslabase}{\kilogram\second\rpsquared\reciprocal%
     \ampere}
  \newunit{\henrybase}
    {\squaremetre\kilogram\second\rpsquared\rpsquare\ampere}
  \newunit{\celsiusbase}{\kelvin}
  \newunit{\lumenbase}{\candela\squaremetre\rpsquare\metre}
  \newunit{\luxbase}{\candela\squaremetre\rpfourth\metre}
  \newunit{\becquerelbase}{\hertzbase}
  \newunit{\graybase}{\squaremetre\second\rpsquared}
  \newunit{\sievertbase}{\graybase}
  \newunit{\katalbase}{\rp\second\mole}
\fi
%    \end{macrocode}
%\end{macro}
%\end{macro}
%\end{macro}
%\end{macro}
%\end{macro}
%\end{macro}
%\end{macro}
%\end{macro}
%\end{macro}
%\end{macro}
%\end{macro}
% Any configuration file is used if found.
%    \begin{macrocode}
\InputIfFileExists{SIunits.cfg}
  {\si@log@inf{SIunits config file loaded}}
  {\si@log@inf{SIunits config file not found}}
%    \end{macrocode}
%\iffalse
%</SIunits>
%<*hepunits>
%\fi
%\subsection{\pkg{hepunits}}
% The \pkg{hepunits} package provides some rather odd unit names,
% which are not really to be encouraged.
%    \begin{macrocode}
\ProvidesFile{si-hepunits.cfg}
  [2008/12/22 v1.1j siunitx: Emulation of
     hepunits]
\si@emulating{hepunits}{2007/09/27}
\requiresiconfigs{SIunits,accepted,prefix,hep}
%    \end{macrocode}
%\begin{macro}{\invbarn}
%\begin{macro}{\invnanobarn}
%\begin{macro}{\invpicobarn}
%\begin{macro}{\invfemtobarn}
%\begin{macro}{\invattobarn}
%\begin{macro}{\invzeptobarn}
%\begin{macro}{\invyoctobarn}
% Inverses barn units.
%    \begin{macrocode}
\ifsi@old@noprefixcmds\else
  \newunit{\invbarn}{\per\barn}
  \newunit{\invnanobarn}{\per\nano\barn}
  \newunit{\invpicobarn}{\per\pico\barn}
  \newunit{\invfemtobarn}{\per\femto\barn}
  \newunit{\invattobarn}{\per\atto\barn}
  \newunit{\invzeptobarn}{\per\zepto\barn}
  \newunit{\invyoctobarn}{\per\yocto\barn}
%    \end{macrocode}
%\end{macro}
%\end{macro}
%\end{macro}
%\end{macro}
%\end{macro}
%\end{macro}
%\begin{macro}{\invnb}
%\begin{macro}{\invpb}
%\begin{macro}{\invfb}
%\begin{macro}{\invab}
%\begin{macro}{\invzb}
%\begin{macro}{\invyb}
% Also available abbreviated.
%    \begin{macrocode}
  \newunit{\invnb}{\per\nano\barn}
  \newunit{\invpb}{\per\pico\barn}
  \newunit{\invfb}{\per\femto\barn}
  \newunit{\invab}{\per\atto\barn}
  \newunit{\invzb}{\per\zepto\barn}
  \newunit{\invyb}{\per\yocto\barn}
\fi
%    \end{macrocode}
%\end{macro}
%\end{macro}
%\end{macro}
%\end{macro}
%\end{macro}
%\end{macro}
%\end{macro}
%\begin{macro}{\invcmsqpersecond}
%\begin{macro}{\invcmsqpersec}
%\begin{macro}{\lumiunits}
% Luminosity.
%    \begin{macrocode}
\newunit{\invcmsqpersecond}{\per\Square\centi\metre\per\second}
\newunit{\invcmsqpersec}{\per\Square\centi\metre\per\second}
\newunit{\lumiunits}{\per\Square\centi\metre\per\second}
%    \end{macrocode}
%\end{macro}
%\end{macro}
%\end{macro}
%\begin{macro}{\inveV}
%\begin{macro}{\minveV}
%\begin{macro}{\minveV}
%\begin{macro}{\kinveV}
%\begin{macro}{\MinveV}
%\begin{macro}{\GinveV}
%\begin{macro}{\TinveV}
% The inverse of an electron-volt, plus prefixes.
%    \begin{macrocode}
\newunit{\inveV}{\per\electronvolt}
\newunit{\minveV}{\per\milli\electronvolt}
\newunit{\kinveV}{\per\kilo\electronvolt}
\newunit{\MinveV}{\per\mega\electronvolt}
\newunit{\GinveV}{\per\giga\electronvolt}
\newunit{\TinveV}{\per\tera\electronvolt}
%    \end{macrocode}
%\end{macro}
%\end{macro}
%\end{macro}
%\end{macro}
%\end{macro}
%\end{macro}
%\end{macro}
%\begin{macro}{\eVoverc}
%\begin{macro}{\eVovercsq}
% Some combinations of electron-volts and the speed of light.  As
% these are called |over|, they are set with a slash.  The
% \opt{eVcorrb} values have been set for Computer Modern.
%    \begin{macrocode}
\newunit[per=slash,eVcorrb=0.6ex]{\eVoverc}
  {\electronvolt\per\clight}
\newunit[per=slash,eVcorrb=0.6ex]{\eVovercsq}
  {\electronvolt\per\Square\clight}
%    \end{macrocode}
%\end{macro}
%\end{macro}
%\begin{macro}{\meVoverc}
%\begin{macro}{\keVoverc}
%\begin{macro}{\MeVoverc}
%\begin{macro}{\GeVoverc}
%\begin{macro}{\TeVoverc}
% Prefixed combinations, first of the speed of light.
%    \begin{macrocode}
\newunit[per=slash,eVcorrb=0.6ex]{\meVoverc}
  {\milli\electronvolt\per\clight}
\newunit[per=slash,eVcorrb=0.6ex]{\keVoverc}
  {\kilo\electronvolt\per\clight}
\newunit[per=slash,eVcorrb=0.6ex]{\MeVoverc}
  {\mega\electronvolt\per\clight}
\newunit[per=slash,eVcorrb=0.6ex]{\GeVoverc}
  {\giga\electronvolt\per\clight}
\newunit[per=slash,eVcorrb=0.6ex]{\TeVoverc}
  {\tera\electronvolt\per\clight}
%    \end{macrocode}
%\end{macro}
%\end{macro}
%\end{macro}
%\end{macro}
%\end{macro}
%\begin{macro}{\meVovercsq}
%\begin{macro}{\keVovercsq}
%\begin{macro}{\MeVovercsq}
%\begin{macro}{\GeVovercsq}
%\begin{macro}{\TeVovercsq}
% Then of the square.
%    \begin{macrocode}
\newunit[per=slash,eVcorrb=0.6ex]{\meVovercsq}
  {\milli\electronvolt\per\Square\clight}
\newunit[per=slash,eVcorrb=0.6ex]{\keVovercsq}
  {\kilo\electronvolt\per\Square\clight}
\newunit[per=slash,eVcorrb=0.6ex]{\MeVovercsq}
  {\mega\electronvolt\per\Square\clight}
\newunit[per=slash,eVcorrb=0.6ex]{\GeVovercsq}
  {\giga\electronvolt\per\Square\clight}
\newunit[per=slash,eVcorrb=0.6ex]{\TeVovercsq}
  {\tera\electronvolt\per\Square\clight}
%    \end{macrocode}
%\end{macro}
%\end{macro}
%\end{macro}
%\end{macro}
%\end{macro}
%\iffalse
%</hepunits>
%<*fancynum>
%\fi
%\subsection{\pkg{fancynum}}
% \pkg{fancynum} only does things with numbers, so there is only
% a little emulation and a few macros needed.
%    \begin{macrocode}
\ProvidesFile{si-fancynum.cfg}
  [2008/12/22 v1.1j siunitx: Emulation of
     fancynum]
\si@emulating{fancynum}{2000/08/08 0.92}
\sisetup{decimalsymbol=cdot,digitsep=comma}
%    \end{macrocode}
%\begin{macro}{\fnum}
% The \cs{fnum} macro is rather restricted, but this is not
% reproduced.  Instead, it is \cs{let} as an alias to the \cs{num}
% macro.
%    \begin{macrocode}
\let\fnum\num
%    \end{macrocode}
%\end{macro}
%\begin{macro}{\setfnumdsym}
%\begin{macro}{\setfnumgsym}
%\begin{macro}{\setfnummsym}
% The control macros are defined.
%    \begin{macrocode}
\newcommand*{\setfnumdsym}[1]{\sisetup{decimalsymbol={#1}}}
\newcommand*{\setfnumgsym}[1]{\sisetup{digitsep={#1}}}
\newcommand*{\setfnummsym}[1]{\sisetup{expproduct={#1}}}
%    \end{macrocode}
%\end{macro}
%\end{macro}
%\end{macro}
% The various package options are now processed if necessary.
%    \begin{macrocode}
\ifsi@old@english
  \sisetup{decimalsymbol=cdot,digitsep=comma}
\fi
\ifsi@old@french
  \sisetup{decimalsymbol=comma,digitsep=fullstop}
\fi
\ifsi@old@tight
  \sisetup{expproduct=tighttimes}
\fi
\ifsi@old@loose
  \sisetup{expproduct=times}
\fi
\ifsi@old@thinspaces
  \sisetup{digitsep=thin}
\fi
\ifsi@old@commas
  \sisetup{digitsep=comma}
\fi
\ifsi@old@plain
  \sisetup{digitsep=none}
\fi
%    \end{macrocode}
%
%\iffalse
%</fancynum>
%<*fancyunits>
%\fi
%\subsection{\pkg{fancyunits}}
% The \pkg{fancyunits} package is not available on \textsc{ctan}, but
% is available from its authors homepage \cite{fancyunits}.  It is
% similar to \SIunits, and so most of the code is shared here.
% However, a few bits of set up occur first, and an emulation-clash
% test is needed.
%    \begin{macrocode}
\ProvidesFile{si-fancyunits.cfg}
  [2008/12/22 v1.1j siunitx: Emulation of
     fancyunits]
\si@emulating{fancyunits}{2007/02/01 v1.0.1}
\si@ifloaded{SIunits}
  {\si@log@err{SIunits emulation loaded\MessageBreak before
    fancyunits emulation}{You need to load the fancyunits
    emualtion\MessageBreak code before that for
    SIunits.\MessageBreak Try emulate=fancyunits as the first
    option when\MessageBreak loading siunitx}}{}
%    \end{macrocode}
%\begin{macro}{\si@siu@newunithook}
%\begin{macro}{\si@siu@newunitxhook}
% To create the extra macros provided by \pkg{fancyunits}, the
% \SIunits emulation code is changed to add the ``\texttt{uf}''
% variants.
%    \begin{macrocode}
\newcommand*{\si@siu@newunithook}[3][]{%
  \edef\si@tempa{%
    \expandafter\noexpand\csname #2per#1#3uf\endcsname}%
  \renewcommand*{\si@tempb}{stickyper,per=fraction,
    fraction=nice}%
  \edef\si@tempc{%
    \noexpand\sisetup{\si@tempb}%
    \expandafter\noexpand\csname #2\endcsname\noexpand\si@per%
    \expandafter\noexpand\si@siu@power%
    \expandafter\noexpand\csname #3\endcsname}%
  \expandafter\expandafter\expandafter\newunit\expandafter
    \expandafter\expandafter{\expandafter\si@tempa\expandafter}%
      \expandafter{\si@tempc}%
  \edef\si@tempa{%
    \expandafter\noexpand\csname #2per#1#3Uf\endcsname}%
  \renewcommand*{\si@tempb}{stickyper,per=fraction,
    fraction=frac}%
  \edef\si@tempc{%
    \noexpand\sisetup{\si@tempb}%
    \noexpand\def\noexpand\si@frc@hook{\noexpand\textstyle}%
    \expandafter\noexpand\csname #2\endcsname\noexpand\si@per%
    \expandafter\noexpand\si@siu@power%
    \expandafter\noexpand\csname #3\endcsname}%
  \expandafter\expandafter\expandafter\newunit\expandafter
    \expandafter\expandafter{\expandafter\si@tempa\expandafter}%
      \expandafter{\si@tempc}%
  \edef\si@tempa{%
    \expandafter\noexpand\csname #2per#1#3UF\endcsname}%
  \edef\si@tempc{%
    \noexpand\sisetup{\si@tempb}%
    \noexpand\def\noexpand\si@frc@hook{\noexpand\displaystyle}%
    \expandafter\noexpand\csname #2\endcsname\noexpand\si@per%
    \expandafter\noexpand\si@siu@power%
    \expandafter\noexpand\csname #3\endcsname}%
  \expandafter\expandafter\expandafter\newunit\expandafter
    \expandafter\expandafter{\expandafter\si@tempa\expandafter}%
      \expandafter{\si@tempc}}
\newcommand*{\si@siu@newunitxhook}[4]{%
  \expandafter\newunit\expandafter{\csname #1per#2uf\endcsname}
    {\sisetup{stickyper,per=fraction,fraction=nice}%
     #3\si@per#4}%
  \expandafter\newunit\expandafter{\csname #1per#2Uf\endcsname}
    {\sisetup{stickyper,per=fraction,fraction=frac}%
     \renewcommand*{\si@frc@hook}{\textstyle}%
     #3\si@per#4}%
  \expandafter\newunit\expandafter{\csname #1per#2UF\endcsname}
    {\sisetup{stickyper,per=fraction,fraction=frac}%
     \renewcommand*{\si@frc@hook}{\displaystyle}%
     #3\si@per#4}}
%    \end{macrocode}
%\end{macro}
%\end{macro}
% With that done, the emulation modules can be loaded.
%    \begin{macrocode}
\requiresiconfigs{SIunits,addn,astro}
\sisetup{obeyall}
%    \end{macrocode}
% There is one \pkg{fancyunits}-specific option to handle.  The other
% options all get sent through to the \SIunits system.\footnote{As
% \SIunits is rather more widely known than \pkg{fancyunits}, any
% other options which could be for either are assumed to be for
% \SIunits.}
%    \begin{macrocode}
\ifsi@old@spaceqspace
  \sisetup{valuesep=space}
\fi
%    \end{macrocode}
%\begin{macro}{\paminute}
%\begin{macro}{\parsecond}
%\begin{macro}{\AstroE}
%\begin{macro}{\oersted}
%\begin{macro}{\ton}
% \pkg{fancyunits} provides some extra units, plus tonne spelled
% incorrectly (again).
%    \begin{macrocode}
\newunit{\paminute}{'}
\newunit{\parsecond}{''}
\newunit{\AstroE}{AE}
\newunit{\oersted}{OE}
\provideunit{\ton}{t}
%    \end{macrocode}
%\end{macro}
%\end{macro}
%\end{macro}
%\end{macro}
%\end{macro}
%\begin{macro}{\decaD}
% An additional prefix.
%    \begin{macrocode}
\let\decaD\decad
%    \end{macrocode}
%\end{macro}
%\begin{macro}{\ufrac}
%\begin{macro}{\Ufrac}
%\begin{macro}{\UFrac}
% The fractional unit macros need to be reproduced.
%    \begin{macrocode}
\newcommand*{\ufrac}[2]{%
  \si[stickyper,per=fraction,fraction=nice]{#1\si@per#2}}
\newcommand*{\Ufrac}[2]{%
  \ensuremath{\textstyle{%
    \si[stickyper,per=fraction,fraction=frac]{#1\si@per#2}}}}
\newcommand*{\UFrac}[2]{%
  \ensuremath{\displaystyle{%
    \si[stickyper,per=fraction,fraction=frac]{#1\si@per#2}}}}
%    \end{macrocode}
%\end{macro}
%\end{macro}
%\end{macro}
%\begin{macro}{\pow}
% A slightly-shorted named \cs{power}.
%    \begin{macrocode}
\let\pow\power
%    \end{macrocode}
%\end{macro}
%\begin{macro}{\Squaremetre}
% An alias for \cs{squaremetre}.
%    \begin{macrocode}
\let\Squaremetre\squaremetre
%    \end{macrocode}
%\end{macro}
% As with \SIunits, there is now a list of compound units to add.
% Only a few are not covered by the \SIunits emulation.
%    \begin{macrocode}
\si@siu@newunit{Gray}{second}
\si@siu@newunit[square]{Squaremetre}{metre}
\si@siu@newunitx{Squaremetre}{newtonsecond}{\Square\metre}
  {\newton\second}
\si@siu@newunit{Squaremetre}{second}
\si@siu@newunit[square]{Squaremetre}{squaresecond}
\si@siu@newunit{Squaremetre}{kilogram}
\si@siu@newunit[cubic]{Squaremetre}{second}
%    \end{macrocode}
%\iffalse
%</fancyunits>
%\fi
%
%\Finale
%\iffalse
%<*refs>
@misc{BIPM:EN,
  note = {\url{http://www.bipm.org/en/si/}},
}

@misc{BIPM:FR,
  note = {\url{http://www.bipm.org/fr/si/}},
}

@misc{BIPM:PDF,
  note = {\url{http://www.bipm.org/en/si/si_brochure/}},
}

@misc{Eijkhout,
  author =       {Victor Eijkhout},
  note =         {\url{http://www.eijkhout.net/tbt/}},
  title =        {{\TeX} by {T}opic},
  year =         {2007},
}

@misc{fancyunits,
  author = {Heiko Bauke},
  note = {\url{http://www.mpi-hd.mpg.de/personalhomes/bauke/LaTeX/Tips_und_Tricks/fancyunits/index.php}},
  title = {\pkg{fancyunits}},
  year = 2007,
}

@misc{IUPAC,
  note = {\url{http://old.iupac.org/publications/books/gbook/green_book_2ed.pdf}},
  title = {The \textsc{iupac} {G}reen {B}ook},
  year = 1993,
}

@misc{NIST,
  note = {\url{http://physics.nist.gov/cuu/Units/index.html}},
}
%</refs>
%<*jawltxdoc>
\NeedsTeXFormat{LaTeX2e}
\ProvidesPackage{jawltxdoc}
\usepackage[T1]{fontenc}
\usepackage{lmodern}
\usepackage[final]{listings,graphicx,microtype}
\usepackage[scaled=0.95]{helvet}
\usepackage[version=3]{mhchem}
\usepackage[osf]{mathpazo}
\usepackage{booktabs,array,url,courier,xspace,varioref}
\usepackage{upgreek,ifpdf,float,caption,longtable,babel}
\begingroup
  \@ifundefined{eTeXversion}
    {\aftergroup\@gobble}
    {\aftergroup\@firstofone}
\endgroup
{\usepackage{etoolbox}}
\floatstyle{plaintop}
\restylefloat{table}
\labelformat{figure}{\figurename~#1}
\labelformat{table}{\tablename~#1}
\IfFileExists{\jobname.sty}
  {\usepackage{\jobname}}{}
\usepackage[numbered]{hypdoc}
\setcounter{IndexColumns}{2}
\newlength\LaTeXwidth
\newlength\LaTeXoutdent
\newlength\LaTeXgap
\setlength\LaTeXgap{1em}
\setlength\LaTeXoutdent{-0.15\textwidth}
\newbox\lst@samplebox
\edef\LaTeXexamplefile{\jobname.tmp}
\lst@RequireAspects{writefile}
\lstnewenvironment{LaTeXexample}[1][example]{%
  \global\let\lst@intname\@empty
  \ifcsname LaTeXcode#1\endcsname
    \expandafter\let\expandafter\LaTeXcode
      \csname LaTeXcode#1\endcsname
    \expandafter\let\expandafter\LaTeXcodeend
      \csname LaTeXcode#1end\endcsname
  \else
    \PackageError{jawltxdoc}
      {Undefined example type `#1'}
      \@ehd
    \let\LaTeXcode\relax
    \let\LaTeXcodeend\relax
  \fi
  \LaTeXcode}
  {\lst@EndWriteFile
   \LaTeXcodeend}
\newcommand*{\LaTeXcodeexample}{%
  \setbox\lst@samplebox=\hbox\bgroup
  \LaTeXcodefloat}
\let\LaTeXcoderesultonly\LaTeXcodeexample
\newcommand*{\LaTeXcodeexampleend}{%
  \egroup
  \setlength\LaTeXwidth{\wd\lst@samplebox}%
  \begin{list}{}{%
    \setlength\itemindent{0pt}
    \setlength\leftmargin\LaTeXoutdent
    \setlength\rightmargin{0pt}}%
    \item
      \setlength\LaTeXoutdent{-0.15\textwidth}
      \begin{minipage}[c]{%
        \textwidth-\LaTeXwidth-\LaTeXoutdent-\LaTeXgap}
        \LaTeXcodefloatend
      \end{minipage}%
      \hfill
      \begin{minipage}[c]{\LaTeXwidth}%
        \hbox to\linewidth{\box\lst@samplebox\hss}%
      \end{minipage}%
  \end{list}}
\newcommand*{\LaTeXcodefloat}{%
  \setkeys{lst}{tabsize=4,gobble=3,breakindent=0pt,
    basicstyle=\small\ttfamily,basewidth=0.51em,
    keywordstyle=\color{blue}}%
  \lst@BeginAlsoWriteFile{\LaTeXexamplefile}}
\let\LaTeXcodenoexample\LaTeXcodefloat
\let\LaTeXcodenoexampleend\@empty
\newcommand*{\LaTeXcodefloatend}{%
  \MakePercentComment\catcode`\^^M=10\relax
  \small
  {\setkeys{lst}{SelectCharTable=\lst@ReplaceInput{\^\^I}%
    {\lst@ProcessTabulator}}%
    \leavevmode \input{\LaTeXexamplefile}}%
  \MakePercentIgnore}
\newcommand*{\LaTeXcoderesultonlyend}{\egroup\LaTeXcodefloatend}
\lstnewenvironment{BibTeXexample}{%
  \global\let\lst@intname\@empty
  \setbox\lst@samplebox=\hbox\bgroup
  \setkeys{lst}{tabsize=4,gobble=3,breakindent=0pt,
    basicstyle=\small\ttfamily,basewidth=0.51em,
    keywordstyle=\color{black}}
  \lst@BeginAlsoWriteFile{\LaTeXexamplefile}}
 {\lst@EndWriteFile
   \LaTeXcodeexampleend}
\newcommand*{\DescribeOption}{%
  \leavevmode\@bsphack\begingroup\MakePrivateLetters
  \Describe@Option}
\newcommand*{\Describe@Option}[1]{\endgroup
              \marginpar{\raggedleft\PrintDescribeEnv{#1}}%
              \SpecialOptionIndex{#1}\@esphack\ignorespaces}
\newcommand*{\SpecialOptionIndex}[1]{\@bsphack
    \index{#1\actualchar{\protect\ttfamily#1}
           (option)\encapchar usage}%
    \index{options:\levelchar#1\actualchar{\protect\ttfamily#1}%
      \encapchar usage}\@esphack}
\newcommand*{\indexopt}[1]{\DescribeOption{#1}\opt{#1}}
\newcommand*{\DescribeOptionInfo}[2]{%
  \DescribeOption{#1}%
  \opt{#1=\meta{#2}}\xspace}
\newcommand*{\ofixarg}[1]{%
  {\ttfamily[}%
  \ifmmode \expandafter \nfss@text \fi
  {%
    \meta@font@select
    \edef\meta@hyphen@restore{%
      \hyphenchar\the\font\the\hyphenchar\font}%
    \hyphenchar\font\m@ne
    \language\l@nohyphenation
    #1\/%
    \meta@hyphen@restore
    }%
    {\ttfamily]}}
\newcommand*{\pkg}[1]{\textsf{#1}}
\newcommand*{\currpkg}{\pkg{\jobname}\xspace}
\newcommand*{\opt}[1]{\texttt{#1}}
\newcommand*{\defaultopt}[1]{\opt{\textbf{#1}}}
\newcommand*{\file}[1]{\texttt{#1}}
\newcommand*{\ext}[1]{\file{.#1}}
\newcommand*{\latin}[1]{\emph{#1}}
\newcommand*{\etc}{%
  \@ifnextchar.
    {\latin{etc}}
    {\latin{etc}.\xspace}}
\newcommand*{\eg}{%
  \@ifnextchar.
    {\latin{e.g}}
    {\latin{e.g}.\xspace}}
\newcommand*{\ie}{%
  \@ifnextchar.
    {\latin{i.e}}
    {\latin{i.e}.\xspace}}
\newcommand*{\etal}{%
  \@ifnextchar.
    {\latin{et~al.}}
    {\latin{et~al}.\xspace}}
\newcommand*{\AMS}{{\protect\usefont{OMS}{cmsy}{m}{n}%
  A\kern-.1667em\lower.5ex\hbox{M}\kern-.125emS}}
\providecommand*{\eTeX}{\ensuremath{\varepsilon}-\TeX}
\DeclareRobustCommand*{\XeTeX}
  {X\kern-.125em\lower.5ex\hbox{\reflectbox{E}}\kern-.1667em\TeX}
\providecommand*{\CTAN}{\textsc{ctan}}
\@ifpackageloaded{etoolbox}
  {\patchcmd{\@addmarginpar}
    {\@latex@warning@no@line {Marginpar on page \thepage\space moved}}
    {\relax}{}{}}
  {}
\newcounter{argument}
\g@addto@macro\endmacro{\setcounter{argument}{0}}
\newcommand*\darg[1]{%
  \stepcounter{argument}%
  {\ttfamily\char`\#\theargument~:~}#1\par\noindent\ignorespaces}
\newcommand*\doarg[1]{%
  \stepcounter{argument}%
  {\ttfamily\makebox[0pt][r]{[}%
   \char`\#\theargument]:~}#1\par\noindent\ignorespaces}
%</jawltxdoc>
%\fi

%
% Documentation:
%    (a) Without write18 enabled:
%          pdflatex siunitx.dtx
%          bibtex8 --wolfgang siunitx
%          makeindex -s gind.ist siunitx.idx
%          makeindex -s gglo.ist -o siunitx.gls siunitx.glo
%          pdflatex siunitx.dtx
%          pdflatex siunitx.dtx
%    (b) With write18 enabled:
%          pdflatex siunitx.dtx
%          pdflatex siunitx.dtx
%          pdflatex siunitx.dtx
%
% Installation:
%     Copy siunitx.sty and the various .cfg files to a location
%     searched by TeX, and if required by your TeX installation,
%     run the appropriate command to build a hash of files
%     (texhash, initexmf --update-fndb, etc.)
%
% Note:
%     The jawltxdoc.sty file is not needed for installation,
%     only for building the documentation; it may be deleted
%     after producing the documentation (if necessary).
%
%<*ignore>
% This is all taken verbatim from Heiko Oberdiek's packages
\begingroup
  \def\x{LaTeX2e}%
\expandafter\endgroup
\ifcase 0\ifx\install y1\fi\expandafter
         \ifx\csname processbatchFile\endcsname\relax\else1\fi
         \ifx\fmtname\x\else 1\fi\relax
\else\csname fi\endcsname
%</ignore>
%<*install>
\input docstrip.tex
\keepsilent
\askforoverwritefalse
\preamble
 ----------------------------------------------------------------
 The siunitx package --- A comprehensive (SI) units package
 Maintained by Joseph Wright
 E-mail: joseph.wright@morningstar2.co.uk
 Released under the LaTeX Project Public License v1.3c or later
 See http://www.latex-project.org/lppl.txt
 ----------------------------------------------------------------

\endpreamble
\Msg{Generating siunitx files:}
\generate{\file{jawltxdoc.sty}{\from{\jobname.dtx}{jawltxdoc}}
}
\usedir{tex/latex/siunitx}
\generate{\file{\jobname.sty}{\from{\jobname.dtx}{package}}
}
\usedir{source/latex/siunitx}
\generate{\file{\jobname.ins}{\from{\jobname.dtx}{install}}
}
\usedir{doc/latex/siunitx}
\generate{\file{si-bug.ltx}{\from{\jobname.dtx}{bug}}
}
\usedir{tex/latex/siunitx/config}
\generate{\file{si-SIunits.cfg}{\from{\jobname.dtx}{SIunits}}
          \file{si-sistyle.cfg}{\from{\jobname.dtx}{sistyle}}
          \file{si-unitsdef.cfg}{\from{\jobname.dtx}{unitsdef}}
          \file{si-units.cfg}{\from{\jobname.dtx}{units}}
          \file{si-hepunits.cfg}{\from{\jobname.dtx}{hepunits}}
          \file{si-fancynum.cfg}{\from{\jobname.dtx}{fancynum}}
          \file{si-fancyunits.cfg}{\from{\jobname.dtx}{fancyunits}}
}
\generate{\file{si-prefix.cfg}{\from{\jobname.dtx}{prefix}}
          \file{si-named.cfg}{\from{\jobname.dtx}{named}}
          \file{si-prefixed.cfg}{\from{\jobname.dtx}{prefixed}}
          \file{si-addn.cfg}{\from{\jobname.dtx}{addn}}
          \file{si-accepted.cfg}{\from{\jobname.dtx}{accepted}}
          \file{si-physical.cfg}{\from{\jobname.dtx}{physical}}
          \file{si-abbr.cfg}{\from{\jobname.dtx}{abbr}}
}
\generate{\file{si-synchem.cfg}{\from{\jobname.dtx}{synchem}}
          \file{si-hep.cfg}{\from{\jobname.dtx}{hep}}
          \file{si-astro.cfg}{\from{\jobname.dtx}{astro}}
          \file{si-geophys.cfg}{\from{\jobname.dtx}{geophys}}
          \file{si-binary.cfg}{\from{\jobname.dtx}{binary}}
}
\generate{\file{si-UK.cfg}{\from{\jobname.dtx}{UK}}
          \file{si-USA.cfg}{\from{\jobname.dtx}{USA}}
          \file{si-DE.cfg}{\from{\jobname.dtx}{DE}}
          \file{si-ZA.cfg}{\from{\jobname.dtx}{ZA}}
}
\nopreamble\nopostamble
\usedir{doc/latex/siunitx}
\generate{\file{README.txt}{\from{\jobname.dtx}{readme}}
          \file{\jobname.bib}{\from{\jobname.dtx}{refs}}
}
\endbatchfile
%</install>
%<*readme>
----------------------------------------------------------------
The siunitx package --- A comprehensive (SI) units package
Maintained by Joseph Wright
E-mail: joseph.wright@morningstar2.co.uk
Released under the LaTeX Project Public License v1.3c or later
See http://www.latex-project.org/lppl.txt
----------------------------------------------------------------

Typesetting values with units requires care to ensure that the
combined mathematical meaning of the value plus unit combination
is clear.  In particular, the SI units system lays down a
consistent set of units with rules on how these are to be used.
However, different countries and publishers have differing
conventions on the exact appearance of numbers (and units).

The siunitx provides a set of tools for authors to typeset
numbers and units in a consistent way.  The package has an
extended set of configuration options which make it possible to
follow varying typographic conventions with the same input
syntax.  The package includes automated processing of numbers
and units, and the ability to control tabular alignment of
numbers.

A number of LaTeX packages have been developed in the past for
formatting units: SIunits, sistyle, unitsdef, unitspkg,
fancyunits and fancynum.  Support for users of all of these
packages is available as emulation modules in siunitx.  In
addition, siunitx can carry out many of the functions of the
dcolumn, rccol and numprint packages.
%</readme>
%<*ignore>
\fi
% Will Robertson's trick
\immediate\write18{bibtex8 --wolfgang \jobname}
\immediate\write18{makeindex -s gind.ist -o \jobname.ind  \jobname.idx}
\immediate\write18{makeindex -s gglo.ist -o \jobname.gls  \jobname.glo}
%</ignore>
%<*driver>
\PassOptionsToClass{a4paper}{article}
\documentclass[german,english,UKenglish]{ltxdoc}
\EnableCrossrefs
\CodelineIndex
\RecordChanges
%\OnlyDescription
% Various adjustments are needed to the packages loaded.  xfrac is
% only loaded if it is available.
\usepackage[final]{graphicx}
% Currently, xfrac is not working with expl3!
%\IfFileExists{xfrac.sty}
%  {\usepackage{xfrac}}
%  {}
\usepackage{jawltxdoc}
\usepackage{textcomp,tikz,pgfplots,tocloft,datatool}
\usepackage[latin1]{inputenc}
\sisetup{
  eVcorra=0.1ex,
  unitsep=cdot,
  alsoload={synchem,hep,binary,astro,geophys},
  loctolang={UK:UKenglish,DE:german}}
\setlength{\cftsubsecnumwidth}{2.5em}
\begin{document}
  \DocInput{\jobname.dtx}
\end{document}
%</driver>
% \fi
%
%\CheckSum{10177}
%
% \CharacterTable
%  {Upper-case    \A\B\C\D\E\F\G\H\I\J\K\L\M\N\O\P\Q\R\S\T\U\V\W\X\Y\Z
%   Lower-case    \a\b\c\d\e\f\g\h\i\j\k\l\m\n\o\p\q\r\s\t\u\v\w\x\y\z
%   Digits        \0\1\2\3\4\5\6\7\8\9
%   Exclamation   \!     Double quote  \"     Hash (number) \#
%   Dollar        \$     Percent       \%     Ampersand     \&
%   Acute accent  \'     Left paren    \(     Right paren   \)
%   Asterisk      \*     Plus          \+     Comma         \,
%   Minus         \-     Point         \.     Solidus       \/
%   Colon         \:     Semicolon     \;     Less than     \<
%   Equals        \=     Greater than  \>     Question mark \?
%   Commercial at \@     Left bracket  \[     Backslash     \\
%   Right bracket \]     Circumflex    \^     Underscore    \_
%   Grave accent  \`     Left brace    \{     Vertical bar  \|
%   Right brace   \}     Tilde         \~}
%
%\GetFileInfo{\jobname.sty}
%
%\DoNotIndex{\@car,\@eha,\@ehc,\@empty,\@firstofone,\@firstoftwo}
%\DoNotIndex{\@for,\@gobble,\@gobbletwo,\@ifl@aded,\@ifnextchar}
%\DoNotIndex{\@ifpackagelater,\@ifpackageloaded,\@ifstar}
%\DoNotIndex{\@ifundefined,\@makeother,\@namedef,\@nameuse,\@ne}
%\DoNotIndex{\@nil,\@secondoftwo,\@temptokena,\@undefined,\AA}
%\DoNotIndex{\active,\addto,\addtolength,\advance,\afterassignment}
%\DoNotIndex{\aftergroup,\AtBeginDocument,\AtEndOfPackage}
%\DoNotIndex{\begingroup,\bfseries,\bgroup,\boldmath,\box}
%\DoNotIndex{\capitalring,\catcode,\cdot,\chardef,\check@mathfonts}
%\DoNotIndex{\circ,\csname,\DeclareFontEncoding,\DeclareFontFamily}
%\DoNotIndex{\DeclareFontShape,\DeclareFontSubstitution}
%\DoNotIndex{\DeclareMathSymbol,\DeclareRobustCommand}
%\DoNotIndex{\DeclareSymbolFont,\DeclareTextSymbol}
%\DoNotIndex{\DeclareTextSymbolDefault,\def,\define@boolkey}
%\DoNotIndex{\define@choicekey,\define@cmdkey,\define@cmdkeys}
%\DoNotIndex{\define@key,\displaystyle,\displaywidth,\divide,\do}
%\DoNotIndex{\edef,\egroup,\else,\encodingdefault,\end,\endcsname}
%\DoNotIndex{\endgroup,\endinput,\endtabular,\ensuremath}
%\DoNotIndex{\expandafter,\f@family,\f@series,\fam,\fi,\font}
%\DoNotIndex{\fontsize,\frac,\gdef,\global,\hbox,\hfil,\hfill,\hskip}
%\DoNotIndex{\hspace,\hss,\ifcsname,\ifdim,\iffalse,\IfFileExists}
%\DoNotIndex{\ifmmode,\ifnum,\ifx,\ignorespaces,\inputencodingname}
%\DoNotIndex{\InputIfFileExists,\key@ifundefine,\let,\long,\m@ne}
%\DoNotIndex{\makeatletter,\math@version,\mathchoice,\mathnormal}
%\DoNotIndex{\mathord,\mathrm,\mathsf,\mathtt,\mdseries}
%\DoNotIndex{\MessageBreak,\mkern,\mp,\NC@find,\NeedsTeXFormat}
%\DoNotIndex{\newbox,\newcolumntype,\newcommand,\newcount,\newdimen}
%\DoNotIndex{\newif,\newlength,\newtoks,\nobreak,\noexpand,\p@}
%\DoNotIndex{\PackageError,\PackageInfo,\PackageWarning,\phantom}
%\DoNotIndex{\ProcessOptionsX,\protect,\protected@edef,\ProvidesFile}
%\DoNotIndex{\ProvidesPackage,\Q,\raisebox,\relax,\renewcommand}
%\DoNotIndex{\RequirePackage,\rlap,\sbox,\scantokens}
%\DoNotIndex{\scriptscriptstyle,\scriptstyle,\selectfont,\setbox}
%\DoNotIndex{\setkeys,\setlength,\settoheight,\sf@size,\sfdefault}
%\DoNotIndex{\sfrac,\skewchar,\sp,\space,\string,\strip@pt}
%\DoNotIndex{\tabularnewline,\text,\textdegree,\textmu,\textohm}
%\DoNotIndex{\textsc,\textstyle,\textsuperscript,\the,\thr@@,\times}
%\DoNotIndex{\ttdefault,\uccode,\unboldmath,\unhbox,\upmu,\Upomega}
%\DoNotIndex{\uppercase,\upshape,\wd,\xdef,\xspace,\z@}
%
%\DoNotIndex{\if,\lowercase,\+,\,,\-,\.,\:,\;,\@argdef,\@cdr}
%\DoNotIndex{\@ifdefinable,\@testopt,\@yargdef,\color,\endlinechar}
%\DoNotIndex{\key@ifundefined,\newrobustcmd,\protected,\scriptspace}
%\DoNotIndex{\usebox,\tw@,\ifinner,\usepackage,\texttt,\textsf}
%\DoNotIndex{\NC@do,\loop,\l@ngrel@x,\g@addto@macro,\documentclass}
%\DoNotIndex{\document\,\begin,\=,\@ehb,\\,\listfiles,\nullfont}
%\DoNotIndex{\normalsize,\futurelet}
%
%\DoNotIndex{\pascalsecond,\amperemetresecond,\ohmmetre}
%\DoNotIndex{\newtonmetre,\newtonmetrenp}
%\DoNotIndex{\kilogramsquaremetre,\kilogramsquaremetrenp}
%
%\changes{v1.0}{2008/06/12}{First official release}
%
%\setkeys{lst}{language=[LaTeX]{TeX},moretexcs={sisetup,num,ang,SI,
%  si,newunit,newpower,newprefix,metre,candela,kelvin,ampere,eV,
%  second,mole,per,kilogram,Square,squared,cubed,cubic,kilo,centi,
%  gram,tothe,toprule,midrule,bottomrule,newton,micro,joule,mega,
%  raiseto,deci,mathnormal,numprint,electronvolt,selectlanguage,
%  kibi,byte,molar,Molar,mmHg,mebi,torr,parsec,lightyear,clight,
%  eVperc,addplot,color,milli,cm,Hz,kg,DeclareRobustCommand,
%  pascal,dalton,addtolocale,DTLnewdb,DTLnewrow,DTLnewdbentry,
%  DTLforeach,DTLiffirstrow,DTLmul,newqualifier,hour,renewunit,
%  degree}}
%
%\title{\currpkg\ --- A comprehensive (SI) units package^^A
%  \thanks{This file describes version \fileversion, last revised
%    \filedate.}}
%\author{Joseph Wright^^A
%  \thanks{E-mail: joseph.wright@morningstar2.co.uk}}
%\date{Released \filedate}
%
%\maketitle
%
%\newcommand*{\SIunits}{\pkg{SIunits}\xspace}
%\newcommand*{\SIstyle}{\pkg{SIstyle}\xspace}
%\newcommand*{\unitsdef}{\pkg{unitsdef}\xspace}
%\newcommand*{\units}{\pkg{units}\xspace}
%\newcommand*{\numprint}{\pkg{numprint}\xspace}
%\begin{abstract}
% Typesetting values with units requires care to ensure that the
% combined mathematical meaning of the value plus unit combination is
% clear.  In particular, the SI units system lays down a consistent
% set of units with rules on how these are to be used.  However,
% different countries and publishers have differing conventions on
% the exact appearance of numbers (and units).
%
% The \currpkg provides a set of tools for authors to typeset numbers
% and units in a consistent way.  The package has an extended set of
% configuration options which make it possible to follow varying
% typographic conventions with the same input syntax.  The package
% includes automated processing of numbers and units, and the ability
% to control tabular alignment of numbers.
%
% A number of \LaTeX\ packages have been developed in the past for
% formatting units: \SIunits, \SIstyle, \unitsdef, \units,
% \pkg{fancyunits} and \pkg{fancynum}. Support for users of all of
% these packages is available as emulation modules in \currpkg.
% In addition, \currpkg can carry out many of the functions of the
% \pkg{dcolumn}, \pkg{rccol} and \numprint packages.
%\end{abstract}
%
%\begin{multicols}{2}
%  \tableofcontents
%\end{multicols}
%
%\clearpage
%\part{Introduction}
% The correct application of units of measurement is very important
% in technical applications.  For this reason, carefully-crafted
% definitions of a coherent units system have been laid down by the
% \latin{Conf\'erence G\'en\'erale des Poids et
% Mesures}\footnote{General Conference on Weights and Measures.}
% (CGPM): this has resulted in the \latin{Syst\`eme International
% d'Unit\'es}\footnote{International System of Units.} (SI).  At the
% same time, typographic conventions for correctly displaying both
% numbers and units exist to ensure that no loss of meaning occurs in
% printed matter.
%
% \currpkg aims to provide a unified method for \LaTeX\ users to
% typeset units and values correctly and easily.  The design
% philosophy of \currpkg is to follow the agreed rules by default,
% but to allow variation through option settings.  In this way,
% users can use \currpkg to follow the requirements of publishers,
% co-authors, universities, \etc without needing to alter the input
% at all.
%
% \currpkg is intended as a complete replacement for \SIunits,
% \SIstyle, \unitsdef, \units, \pkg{fancyunits} and
% \pkg{fancynum}. As such, emulation modes are provided for all of
% these packages.  Where possible, conventions from the existing
% solutions have been used here.  For example, the macros \cs{num},
% \cs{ang} and \cs{SI} act in a very similar fashion to those in
% existing packages.
%
%\clearpage
%\part{Using the \currpkg package}
%
%\section{For the impatient}
% \currpkg provides the user macros:
%\begin{itemize}
%  \item \cs{SI}\oarg{options}\marg{value}\oarg{pre-unit}\marg{unit}
%  \item \cs{si}\oarg{options}\marg{unit}
%  \item \cs{num}\oarg{options}\marg{number}
%  \item \cs{ang}\oarg{options}\marg{angle}
%  \item \cs{sisetup}\marg{options}
%\end{itemize}
% plus the |S| and |s| column types for decimal alignments and units
% in tables.  These macros are designed for typesetting units and
% values with control of appearance and with intelligent processing.
%
% By default, all text is typeset in the current upright, serif maths
% font.  This can be changed by setting the appropriate package
% options: \opt{obeyall} will use the current font for typesetting.
%
% The package includes a ``unit processor'', which allows the use of
% named units or literal values.  Named units are processed to
% correctly include powers.
%\begin{LaTeXexample}
%  \SI{10}{\gram}\\
%  \SI{23.4}{g.cm^3}\\
%  \num{1e34}\\
%  \ang{1;2;3}\\
%  \emph{\SI{16,7}{\metre\per\second}}\\
%  \textbf{\SI{30e3}{\Hz}}\\
%  \SI{1.2 x 3.56 x 9.2}{\milli\metre}\\
%  \sisetup{obeyall}
%  \textbf{\SI{-4.5}{\cm}}\\
%  \si{\joule\per\mole\per\kelvin}\\
%  \si[per=frac]{\joule\per\mole\per\kelvin}\\
%  \num[dp=4]{1.23456}\\
%  \num[dp=4]{9.8}\\
%  \begin{tabular}{S[tabformat=3.2]}
%    \toprule
%    {Heading}\\
%    \midrule
%    1.3 \\
%    134.2 \\
%    3.56 \\
%    74,7 \\
%    \bottomrule
%  \end{tabular}
%\end{LaTeXexample}
%
%\section{Requirements}
% \currpkg requires a reasonably up to date \TeX\ system.  The
% package requires \eTeX-extensions, which should be available on
% most systems.\footnote{If you have an old \LaTeX\, try
% ``\texttt{elatex}'' rather than ``\texttt{latex}''.}  The following
% packages are also needed:
%\begin{itemize}
%  \item \pkg{array} and \pkg{xspace}: from the \pkg{tools} bundle,
%    which should be available to everyone;
%  \item \pkg{xkeyval}: this processes the option handling, and needs
%    to be at least v2.5;
%  \item \pkg{amstext}: from the \AMS\TeX\ support bundle (the
%    \AMS\ fonts are also needed to provide the default upright
%    \si{\micro}).
%\end{itemize}
% Hopefully most people using the package will have access to all of
% those items.
%
% To use the \opt{fraction=sfrac} option, the \pkg{xfrac} package is
% needed.  This needs various experimental \LaTeX3 packages.  As a
% result, \currpkg does not load \pkg{xfrac}.  If you want to use
% \opt{fraction=sfrac}, \emph{you} need to load \pkg{xfrac} in your
% preamble before \currpkg.\footnote{This document has been compiled
% in this way.  You have to load \pkg{xfrac} first as otherwise very
% nasty things happen with \pkg{xkeyval}.  MiK\TeX\ users should note
% that the packaged versions of \pkg{expl3}, \pkg{template} and
% \pkg{xparse} will not work with \pkg{xfrac}: download copies from
% \textsc{ctan}!} If the package is not loaded, \opt{fraction=sfrac}
% falls back on a \pkg{nicefrac}-like method.  The interested user
% should look at the \pkg{xfrac} documentation for reasons this might
% not be ideal.\footnote{On the other hand, some fractional units
% will look really bad with \cs{sfrac}.  Use this option with
% caution.}
%
%\section{Loading the package}
% \currpkg is loaded by the usual \LaTeX\ method.
%\begin{LaTeXexample}[noexample]
%  \usepackage[<options>]{siunitx}
%\end{LaTeXexample}
% As is shown in the example, the package can be loaded with one or
% more options, using the key--value system.  The full range of
% package options are described in Section~\ref{sec:options}; some
% options are described in the along with the appropriate user
% macros.  Most of the user macros accept the same key--value
% settings as an optional argument.
%
%\section{Numbers}
%\DescribeMacro{\num}
% Numbers are automatically formatted by the \cs{num} macro.  This
% takes one optional and one mandatory argument:
% \cs{num}\oarg{options}\marg{number}.  The contents of \meta{number}
% are automatically formatted, in a similar method to that used by
% \numprint.  The formatter removes ``hard'' spaces (\cs{,} and |~|),
% automatically identifies exponents (by default marked using |e| or
% |d|) and adds the appropriate spacing of large numbers.  A leading
% zero is added before a decimal marker, if needed: both ``|.|'' and
% ``|,|'' are recognised as decimal marker.
%\begin{LaTeXexample}
%  \num{1}    \num{123}     \num{1234}  \num{12345}\\
%  \num{0.1}  \num{0.123}   \num{0,1234} \num{.12345}\\
%  \num{1e10} \num{3.45d-4} \num{-e10}
%\end{LaTeXexample}
% Various error-checking systems are built into the package, so that
% if \meta{number} does not contain any numeric characters, a warning
% is issued.  Isolated signs are also detected.  The package
% recognises |(| and |)| as ``extra'' characters, which can be used
% to indicate the error in a number.\footnote{This is common in
% chemical crystallography, for example.}  The \opt{seperr} causes
% this data to be given as a separate error value.  If the number
% also contains an exponent, then brackets are re-added after the
% separation to ensure that meaning is not lost.
%\begin{LaTeXexample}
%  $\num{1.234(5)} = \num[seperr]{1.234(5)}$\\
%  $\num{1.234(5)e6} = \num[seperr]{1.234(5)e6}$
%\end{LaTeXexample}
% The same applies to the unit and value macro \cs{SI}, described
% later, for example the rest mass of an electron \cite{IUPAC}:
%\begin{LaTeXexample}
%   $ m_{\mathrm{e}}
%     = \SI{9.1093897(54)e-31}{\kg} $\\
%   $ m_{\mathrm{e}}
%     = \SI[seperr]{9.1093897(54)e-31}{\kg} $
%\end{LaTeXexample}
%
% A number of effects are available as options.  These are fully
% explained in Section~\ref{sec:options}.  Some of the more useful
% options are illustrated here.  By default, the output of the
% package is typeset in maths mode.  However, the use of the current
% text font can be forced.\footnote{This document is typeset using
% lowercase numbers in text mode, which emphasises the effect here.}
%\begin{LaTeXexample}
%  \num{1234567890} \num[mode=text]{1234567890}
%\end{LaTeXexample}
% \currpkg can automatically add zeros and signs to numbers.  This
% can be altered as desired.
%\begin{LaTeXexample}
%  \num{1.}  \num[padnumber=all]{1.}\\
%  \num{2}   \num[addsign=all]{2}\\
%  \num{3e4} \num[addsign=mant]{3e4} \num[addsign=all]{3e4}\\
%  \num{.5}  \num[padnumber=none]{.5}
%\end{LaTeXexample}
% The separation of digits can be turned on and off, and the output
% changed.
%\begin{LaTeXexample}
%  \num{1234}  \num[sepfour=true]{1234}\\
%  \num{12345} \num[digitsep=comma]{12345}\\
%  \num[digitsep=none]{12345}
%\end{LaTeXexample}
% The formatting of exponents is also customisable.
%\begin{LaTeXexample}
%  \num{1e10} \num[expproduct=cdot]{1e10}\\
%  \num{2e20} \num[expbase=5]{2e20}\\
%  \num{3e30} \num[expproduct=tighttimes]{3e30}
%\end{LaTeXexample}
%
% \currpkg can automatically add colour to negative numbers, which is
% often useful for highlighting purposes.  This is turned on with the
% \opt{colourneg} option; the colour used is set by \opt{negcolour}.
% Both of these are available with the US spellings: \opt{colorneg}
% and \opt{negcolor}.
%\begin{LaTeXexample}
%  \num{-1}\\
%  \sisetup{colourneg} \num{-2}\\
%  \num{3e-3}\\
%  \num[negcolour=blue]{-4}
%\end{LaTeXexample}
%
% \currpkg can automatically zero-fill and round to a fixed number of
% decimal places.  This is controlled by two options \opt{fixdp} and
% \opt{dp}.  The later is an integer which specifies how many places
% to fix to; setting this option automatically sets \opt{fixdp} to
% \opt{true}.  The place-fixing system will only alter pure numbers: for
% example, any error component will result in the input being left
% unchanged.
%\begin{LaTeXexample}
%  \num{1.23456}\\
%  \sisetup{dp=2}
%  \num{1.23456}\\
%  \num{9.8}\\
%  \num{-10.432}\\
%  \num{44.3221(2)}
%\end{LaTeXexample}
%
%\section{Tabular material}
%\subsection{Aligning numbers}
% Centring numbers in tabular content is handled by a new column
% type, the |S| column.  This is based closely on the \pkg{dcolumn}
% method for centring numbers in columns, but adds the functionality
% of the \cs{num} macro.\footnote{The approach used is actually a
% combination of \pkg{dcolumn} for centring the material and
% \numprint for processing it.  It will therefore give rather
% different results than the \texttt{n} and \texttt{N} column types
% in \numprint.}
%
% By default, the decimal marker of the number is placed at the
% centre of the column, which then resizes to accommodate the width
% of the contents (\ref{tab:default}).  This behaviour is set by the
% \opt{tabnumalign=centredecimal} option.  By setting the
% \opt{tabnumalign} option to \opt{centre}, the centre of the space
% reserved for the number is placed at the centre of the column.  The
% space reserved is stored in \opt{tabformat}, which is of the form
% \texttt{\meta{before}\meta{dec}\meta{after}}, where \meta{before}
% is the number of characters before the decimal marker and
% \meta{after} is the number after. Thus in the example,
% \opt{tabformat=2.4} provides space for two digits before the
% decimal marker and four after.  \opt{tabnumalign} can also be set
% to \opt{left} and \opt{right}, with the expected results.
%\begin{LaTeXexample}[float]
%  \begin{table}
%    \caption{Behaviour of \texttt{S} column type}
%    \label{tab:default}
%    \centering
%    \begin{tabular}{%
%      S%
%      S[tabnumalign=centre,tabformat=2.4]%
%      S[tabnumalign=right,tabformat=2.4]%
%      S[tabnumalign=centre,tabformat=2.4,decimalsymbol=comma]}
%    \toprule
%    {Some Values} & {Some Values} & {Some Values} & {Some Values} \\
%    \midrule
%     2.3456  &  2.3456  &  2.3456 &  2.3456 \\
%    34.2345  & 34.2345  & 34.2345 & 34.2345 \\
%    56.7835  & 56.7835  & 56.7835 & 56.7835 \\
%    90.473   & 90.473   & 90.473  & 90.473  \\
%    \bottomrule
%    \end{tabular}
%  \end{table}
%\end{LaTeXexample}
% The \opt{tabformat} setting can also be used to reserve space for
% numbers containing exponents.  This is given in the same format as
% above, but with a mantissa and exponent part (\ref{tab:exptab}).
% Notice that this is designed to expect that numbers will contain a
% mantissa.  Exponents can either be aligned so that the ``$\times$''
% symbols match up vertically, or the exponent part can be allowed to
% move across as needed. Space for signs is added by using any sign
% in the \opt{tabformat}, so for example \opt{tabformat=+2.2} and
% \opt{tabformat=-2.2} have exactly the same effect. Setting
% \opt{tabformat} will automatically switch \opt{tabnumalign} from
% \opt{centredecimal} to \opt{centre}, if the former is
% currently set. In other cases, the current alignment option is
% retained.
%\begin{LaTeXexample}[float]
%  \begin{table}
%    \caption{Exponents in tables}
%    \label{tab:exptab}
%    \centering
%    \begin{tabular}{%
%      S[tabnumalign=right,tabformat=2.2e2]%
%      S[tabnumalign=centre,tabformat=2.2e1.1]%
%      S[tabnumalign=centre,tabformat=2.2e1.1,tabalignexp=false]%
%      S[tabnumalign=centre,tabformat=+2.2]}
%    \toprule
%        {Longer values}
%      & {Longer values}
%      & {Longer values}
%      & {Values} \\
%    \midrule
%    2.3e1    &  2.34e1  &  2.34e1  & +2.31 \\
%    34.23e45 & 34.23e45 & 34.23e45 & 34.23 \\
%    56.78    & 56.78    & 56.78    & -56.78 \\
%    1.0e34   & 1.0e34   & 1.0e34   & +-1.0 \\
%    \bottomrule
%    \end{tabular}
%  \end{table}
%\end{LaTeXexample}
%
% Data not to be processed as a number should be protected by
% wrapping it in braces: this is most likely to be true for column
% headers (again as illustrated).  By default, the contents of
% non-numeric cells are centred.  This can be altered by setting
% \opt{tabtextalign}, which can be set to \opt{left}, \opt{right} or
% \opt{centre}.  The use of digit separators in table columns is
% accounted for: extra space is reserved if digit separators will be
% added.
%
%\subsection{Columns of units}
% As a complement to the |S| column, \currpkg also provides a second
% column type, |s|.  This is intended for producing columns of units.
% The letters chosen are intended to be similar to \cs{SI} and
% \cs{si}, respectively.  The alignment of material in |s| columns is
% governed by the \opt{tabunitalign} option.
%\begin{LaTeXexample}[float]
%  \begin{table}
%  \centering
%  \caption{Number and units in tables}
%  \label{tab:num-unit}
%    \begin{tabular}{%
%      S[tabformat=1.2e-1,tabnumalign=centre]%
%      s[tabunitalign=left]}
%    \toprule
%      {Value} & \multicolumn{1}{c}{Unit} \\
%    \midrule
%      2.16e-5 & \metre\squared\per\second \\
%      2.83e-6 & \metre\squared\per\second \\
%      7.39e3  & \pascal\cubic\metre\per\mole \\
%      1.0e5   & \pascal \\
%    \bottomrule
%    \end{tabular}
%  \end{table}
%\end{LaTeXexample}
%
% As the \cs{si} macro can take literal or macro input, the |s|
% column cannot validate the input.  \emph{Everything} in an |s|
% column is therefore passed to the \cs{si} macro for processing.  To
% prevent this, you have to use \cs{multicolumn}, as is shown in
% \ref{tab:s-limits}.  Notice that the braces do not prevent
% processing and colouring of the cell contents.
%\begin{LaTeXexample}[float]
%  \begin{table}
%  \centering
%  \caption{The \texttt{s} column processes everything}
%  \label{tab:s-limits}
%    \begin{tabular}{%
%      s[colourall,colour=orange]%
%      s[colourall,colour=orange]}
%    \toprule
%      {Unit}    & \multicolumn{1}{c}{Unit}\\
%    \midrule
%      {m^3}     & \multicolumn{1}{c}{\si{m^3}} \\
%      \kilogram & \kilogram \\
%    \bottomrule
%    \end{tabular}
%  \end{table}
%\end{LaTeXexample}
%
%\section{Angles}
%\DescribeMacro{\ang}
% Angles can be typeset using the \cs{ang} command.  This takes two
% arguments, \cs{ang}\oarg{options}\marg{angle}, where \meta{options}
% can be any of the package options to apply only to this value.
% \meta{angle} can be given either as a decimal number or as a
% semi-colon separated list of degrees, minutes and seconds, \ie\
% \cs{ang}\marg{decimal angle} or
% \cs{ang}|{|\meta{degrees}|;|\meta{minutes}|;|\meta{seconds}|}|. By
% default, no space is introduced between angles and the degrees,
% minutes and seconds markers.
%\begin{LaTeXexample}
%  \ang{10} \ang{12.3} \ang{4,5}\\
%  \ang{1;2;3} \ang{;;1}\\
%  \ang{+10;;} \ang{-0;1;}
%\end{LaTeXexample}
%
% By default, angles with no degrees (or minutes) are zero-filled;
% angles with degrees but no minutes or seconds are not filled.  This
% behaviour can be altered using the package options.
%\begin{LaTeXexample}
%  \ang{;;1} \ang[padangle=none]{;;1}\\
%  \ang{2;;} \ang[padangle=all]{2;;}\\
%  \sisetup{padangle=all} \ang{;3;} \ang{4;;} \ang{;;5}
%\end{LaTeXexample}
% The \cs{num} macro is used to typeset each number of the angle, so
% the options for \cs{num} also apply here.  The \opt{anglesep} value
% can be used to separate degrees, minutes and seconds.
%\begin{LaTeXexample}
%  \ang{1.05}    \ang[decimalsymbol=comma]{1.05}\\
%  \ang{3.67890} \ang[digitsep=comma]{3.67890}\\
%  \ang{9;8;7}   \ang[anglesep=thin]{9;8;7}
%\end{LaTeXexample}
% The degrees, minutes and seconds signs can be placed over the
% decimal sign using the \opt{astroang} option.  This is designed on
% the assumption that only the last number given has a decimal
% part.
%\begin{LaTeXexample}
%  \ang{1.2} \ang[astroang]{1.2}\\
%  \ang{1;2.3;} \ang[astroang]{1;2.3;}\\
%  \ang{1;2;3.4} \ang[astroang]{1;2;3.4}
%\end{LaTeXexample}
%
%\section{Units and values}
%\DescribeMacro{\SI}
% The core aim of \currpkg is correctly typesetting values which have
% units. The main output macro here is \cs{SI}, which has the same
% syntax as the macros with the same name in \SIstyle and \unitsdef
% packages. The \cs{SI} macro takes two mandatory arguments, in
% addition to the optional set up argument, and a second optional
% argument:
% \cs{SI}\oarg{options}\marg{number}\oarg{preunit}\marg{unit}.  The
% \meta{number} argument operates in exactly the same manner as the
% equivalent argument of the \cs{num} macro.  \meta{unit} will be
% typeset with a non-breakable space between it and the preceding
% number, with font control as outlined earlier.  Finally,
% \meta{preunit} is a unit to be typeset \emph{before} the numerical
% value (most likely to be a currency).  Some examples illustrate the
% general power of the macro.
%\begin{LaTeXexample}
%  \SI[mode=text]{1.23}{J.mol^{-1}.K^{-1}}\\
%  \SI{.23e7}{\candela}\\
%  \SI[per=slash]{1.99}[\pounds]{\per\kilogram}\\
%  \SI{70}{\metre\per\second}\\
%  \SI[per=frac,fraction=nice]{1,345}{\ampere\per\mole}
%\end{LaTeXexample}
% The use of unit macros outside of the \cs{SI} macro is described
% later.
%
%\subsection{Literal units}
%\changes{v1.1}{2008/07/23}{Unit processor extended to interpret
%  \texttt{\textunderscore} correctly}
% Units can be input in two ways, inspired by \SIstyle and \SIunits.
% The \SIstyle-like method uses literal input.  Five characters have
% a special meaning:
%\begin{itemize}
%  \item ``|^|'' and ``|_|''  The superscript and subscript
%    characters can be used without the usual
%    need for surrounding maths characters (|$|);^^A$
%  \item ``|.|'' and ``|,|'': the full stop (point) symbol and comma
%    are made active, and produce the current contents of the
%    \opt{unitsep} option;
%  \item ``|~|'' The contents of the \opt{unitspace} option are typeset
%    by a tilde.
%\end{itemize}
% This allows ready input of units.
%\begin{LaTeXexample}
%  \SI{10}{kg.m.s^{-2}}\\
%  \SI{1.453}{g/cm^3}\\
%  \SI{6.345}{kg_{pol}.mmol_{cat}^{-1}}\\
%  \SI{33.562}{cd~s}\\
%  \SI[unitsep=medium]{100}{m.s^{-2}}
%\end{LaTeXexample}
% The literal unit system will correctly typeset input containing the
% symbols \si{\micro} (micro), \si{\degree} (degree) and
% \si{\angstrom} (ring-A).\footnote{Currently this works with \XeTeX\
% and \pkg{inputenc} using the \texttt{latin1}, \texttt{latin5} and
% \texttt{latin9} encodings.}
% \begingroup\makeatletter\si@unt@nonlatin
%\begin{LaTeXexample}
%  \SI{10}{�m}\\
%  \SI{20}{�C}\\
%  \SI{30}{�}
%\end{LaTeXexample}
%\endgroup
%
%\subsection{The unit interpreter}
% The second operation mode for the \cs{SI} macro is based on the
% behaviour of \SIunits.  Here, each unit, SI multiple prefix and
% power is given a macro name.  These are entered in a method very
% similar to the reading of the unit name in English.
%\begin{LaTeXexample}
%  \SI{10}{\kilo\gram\metre\per\second\squared}\\
%  \SI{1.453}{\gram\per\cubic\centi\metre}\\
%  \SI{33.562}{\candela\second}\\
%  \SI[unitsep=medium]{100}{\metre\per\Square\second}\\
%  \SI[prefixsymbolic=false]{4.56}{\kilo\metre\per\second}
%\end{LaTeXexample}
% On its own, this is very similar to \SIunits, and is less
% convenient than the direct input method.\footnote{Users of \SIunits
% should note the lack of need for a \cs{usk}-type macro.}  However,
% the package allows you to define new unit macros; a large number of
% pre-defined abbreviations are also supplied.  More importantly, by
% defining macros for units, instead of literal values, new
% functionality is made available.  Units may be re-defined to give
% different output, and handling of reciprocal values can be altered.
%\begin{LaTeXexample}
%  \SI[per=frac,fraction=frac]{10}{\gram\metre\per\second\squared}\\
%  \SI[per=slash]{1.453}{\gram\per\cubic\centi\metre}\\
%  \SI{33.562}{\candela\second}\\
%  \SI[per=frac,fraction=nice]{100}{\metre\per\Square\second}
%\end{LaTeXexample}
% The unit processor will trap \emph{some} errors in the input and
% give the ``best guess'' result.  However, it is down to the user to
% check the output.
%
%\subsection{Powers of units}
%\DescribeMacro{\Square}
%\DescribeMacro{\ssquare}
%\DescribeMacro{\squared}
%\DescribeMacro{\cubic}
%\DescribeMacro{\cubed}
% Including powers in units is handled using a ``natural language''
% method.  Thus preceding a unit by \cs{Square} or \cs{cubic} which
% raise the unit to the appropriate power, while \cs{squared} or
% \cs{cubed} follow the unit they apply to. The \cs{Square} macro is
% capitalised to avoid a name clash with \pkg{pstricks}; the
% alternative \cs{ssquare} is also provided.
%\begin{LaTeXexample}
%  \SI{10}{\metre\squared}\\
%  \SI{20}{\Square\metre}\\
%  \SI{30}{\metre\cubed}\\
%  \SI{40}{\cubic\metre}
%\end{LaTeXexample}
%
%\DescribeMacro{\per}
% The \cs{per} macro intelligently creates reciprocal powers, and
% also adds the power $-1$ when appropriate.
%\begin{LaTeXexample}
%  \SI{10}{\per\second\squared}\\
%  \SI{20}{\per\Square\second}\\
%  \SI[per=frac,fraction=nice]{30}{\per\second\cubed}\\
%  \SI[per=slash]{40}{\per\cubic\second}\\
%  \SI{50}{\per\second}\\
%  \SI{60}{\per\metre\Square\candela}
%\end{LaTeXexample}
%
%\DescribeMacro{\tothe}
%\DescribeMacro{\raiseto}
% For powers not defined above or with \cs{newpower}, the \cs{tothe}
% macro can be used ``in line'' to produce a power.  As follows from
% standard English usage, this comes after the unit. \cs{raiseto}
% achieves the same, but is used \emph{before} a unit to add a power
% \emph{after}.\footnote{\cs{raiseto} acts in the same way as
% \cs{tothe} when used in a literal context: the power will be
% produced where the macro is, rather than moving after the next
% item.}
%\begin{LaTeXexample}
%  \SI{16.86}{\metre\tothe{4}}\\
%  \SI{7.895}{\raiseto{-6}\newton}\\
%  \SI{1.34}{\per\kelvin\tothe{7}}
%\end{LaTeXexample}
%
%\subsection{Units with no values}
%\DescribeMacro{\si}
% For typesetting the symbol for a unit on its own, with the full
% font control and without extra spaces, the \cs{si} macro is
% provided.\footnote{The same effect can be achieved using the
% \cs{SI} macro with an empty numerical argument.}  The macro name
% avoids a clash with the functionality of the earlier packages, but
% is similar to \cs{ilu} from the \unitsdef package.
%\begin{LaTeXexample}
%  \SI{}{kg.m/s^2}\\
%  \si{kg.m/s^2}\\
%  \si[mode=text,unitsep=thin]{\mole\per\cubic\deci\metre}
%\end{LaTeXexample}
%
%\subsection{Free-standing units}
% Users of the \unitsdef package will be accustomed to using unit
% macros on their own (following a value) or with an optional
% argument containing a number.  In both cases, only a single unit
% macro could be used. \currpkg supports both operation modes, with
% the limitation that units trailing values loose font control of the
% value.  When used in this way, the units \emph{do not} take an
% optional \pkg{keyval} argument.
%\begin{LaTeXexample}
%  \sisetup{prespace,allowoptarg}
%  123\metre\\
%  \kelvin[123]\\
%  \sisetup{mode=text} \ampere[234]\\
%  6\second
%\end{LaTeXexample}
%
%\subsection{Pre-defined units, prefixes and powers}
%\newcommand*{\csindex}[1]{^^A
%  \cs{#1}^^A
%  \expandafter\SpecialUsageIndex\csname#1\endcsname}
%\newcommand*{\unitinfo}[1]{^^A
%  #1 & \csindex{#1} &
%  \expandafter\si\expandafter{\csname#1\endcsname}}
%\begin{table}
%  \caption{The seven base SI units}
%  \label{tab:baseunits}
%  \centering
%  \begin{tabular}{lll}
%    \toprule
%      \multicolumn{1}{c}{Unit} &
%      \multicolumn{1}{c}{Macro} &
%      \multicolumn{1}{c}{Symbol} \\
%    \midrule
%    \unitinfo{kilogram}\\
%    \unitinfo{metre}\\
%    \unitinfo{second}\\
%    \unitinfo{mole}\\
%    \unitinfo{kelvin}\\
%    \unitinfo{ampere}\\
%    \unitinfo{candela}\\
%    \bottomrule
%  \end{tabular}
%\end{table}
%\DescribeMacro{\metre}
%\DescribeMacro{\meter}
% The package always defines the seven base SI units, irrespective of
% any package options given (\ref{tab:baseunits}). The kilogram is
% notable as by default it is a \emph{base} unit with a prefix. Thus,
% when the package is loaded with the option \opt{load=\{\}},
% \cs{kilo} and \cs{gram} \emph{are not defined}.  As metre is often
% spelled as ``meter'' in the US, the macro \cs{meter} is provided in
% addition to the \cs{metre} macro.\footnote{The official SI spelling
% for the unit is ``metre''.}
%
% By default, a number of additional definitions are created by the
% package.  These are controlled by the \opt{load} and \opt{noload}
% options. Unless specifically requested with the option
% \opt{noload=prefix}, \currpkg defines the standard prefixes for
% powers of ten (\ref{tab:prefix}).  This leads to the redefinition
% of \cs{kilogram} as \cs{kilo}\cs{gram}.  The macro \cs{deka} is
% provided, as this is used as an alias for \cs{deca} in some places.
% The package also defines a number of derived SI units which have
% assigned names and symbols (\ref{tab:derived}). Note that \cs{Gray}
% is capitalised to avoid a name clash with the \pkg{pstricks}
% package.\footnote{The macros \cs{ohm} and \cs{celsius} are not
% defined by \currpkg if the \pkg{gensymb} package is loaded.}
%\newcommand*{\prefixunitsym}[1]{\si[prefixsymbolic=false]{#1}}
%\newcommand*{\prefixinfo}[1]{^^A
%  #1 & \csindex{#1} &
%    \expandafter\prefixunitsym\expandafter{\csname#1\endcsname} &
%    \expandafter\si\expandafter{\csname#1\endcsname}}
%\begin{table}
%  \caption{The SI prefixes (\opt{load=prefix})}
%  \label{tab:prefix}
%  \centering
%  \begin{tabular}{lllclllc}
%    \toprule
%      \multicolumn{1}{c}{Prefix} &
%      \multicolumn{1}{c}{Macro} &
%      \multicolumn{1}{c}{Power} &
%      Symbol &
%      \multicolumn{1}{c}{Prefix} &
%      \multicolumn{1}{c}{Macro} &
%      \multicolumn{1}{c}{Power} &
%      Symbol \\
%    \midrule
%    \prefixinfo{yocto} &
%    \prefixinfo{deca} \\
%    \prefixinfo{zepto} &
%    \prefixinfo{hecto} \\
%    \prefixinfo{atto} &
%    \prefixinfo{kilo} \\
%    \prefixinfo{femto} &
%    \prefixinfo{mega} \\
%    \prefixinfo{pico} &
%    \prefixinfo{giga} \\
%    \prefixinfo{nano} &
%    \prefixinfo{tera} \\
%    \prefixinfo{micro} &
%    \prefixinfo{peta} \\
%    \prefixinfo{milli} &
%    \prefixinfo{exa} \\
%    \prefixinfo{centi} &
%    \prefixinfo{zetta} \\
%    \prefixinfo{deci} &
%    \prefixinfo{yotta} \\
%    \bottomrule
%  \end{tabular}
%\end{table}
%\begin{table}
%  \caption{The derived SI units with defined names
%    (\opt{load=derived})}
%  \label{tab:derived}
%  \centering
%  \begin{tabular}{llllll}
%    \toprule
%      \multicolumn{1}{c}{Unit} &
%      \multicolumn{1}{c}{Macro} &
%      \multicolumn{1}{c}{Symbol} &
%      \multicolumn{1}{c}{Unit} &
%      \multicolumn{1}{c}{Macro} &
%      \multicolumn{1}{c}{Symbol} \\
%    \midrule
%    \unitinfo{becquerel} &
%    \unitinfo{newton} \\
%    \unitinfo{celsius} &
%    \unitinfo{ohm} \\
%    \unitinfo{coulomb} &
%    \unitinfo{pascal} \\
%    \unitinfo{farad} &
%    \unitinfo{radian} \\
%    \unitinfo{Gray} &
%    \unitinfo{siemens} \\
%      & \csindex{ggray} & \ggray &
%    \unitinfo{sievert} \\
%    \unitinfo{hertz} &
%    \unitinfo{steradian} \\
%    \unitinfo{henry} &
%    \unitinfo{tesla} \\
%    \unitinfo{joule} &
%    \unitinfo{volt} \\
%    \unitinfo{katal} &
%    \unitinfo{watt} \\
%    \unitinfo{lumen} &
%    \unitinfo{weber} \\
%    \unitinfo{lux} \\
%    \bottomrule
%  \end{tabular}
%\end{table}
%
% In addition to these units, there are three other groups of units
% for use with the SI system which do not fit into the above.  These
% are those derived from physical measurements
% (\ref{tab:experimental}), those considered ``accepted''
% (\ref{tab:accepted}), and those accepted temporarily
% (\ref{tab:addn}).\footnote{These are supposed to be replaced over
% time by SI units.}
%\DescribeMacro{\litre}
%\DescribeMacro{\liter}
% The unit ``litre'' is often spelled ``liter'' in the US; both
% spellings are provided by \currpkg, with \cs{liter} giving
% \si{\liter} and \cs{litre} producing \si{\litre}.
%\begin{table}
%  \caption{Units derived from experiments
%    (\opt{load=physical})}
%  \label{tab:experimental}
%  \centering
%  \begin{tabular}{lll}
%    \toprule
%      \multicolumn{1}{c}{Unit} &
%      \multicolumn{1}{c}{Macro} &
%      \multicolumn{1}{c}{Symbol} \\
%    \midrule
%    electron volt & \csindex{electronvolt} & \electronvolt \\
%    unified atomic mass unit
%      & \csindex{atomicmassunit} & \atomicmassunit \\
%      & \csindex{atomicmass} & \atomicmass \\
%    \bottomrule
%  \end{tabular}
%\end{table}
%\begin{table}
%  \caption{Units accepted for use with SI
%    (\opt{load=accepted})}
%  \label{tab:accepted}
%  \centering
%  \begin{tabular}{lll}
%    \toprule
%      \multicolumn{1}{c}{Unit} &
%      \multicolumn{1}{c}{Macro} &
%      \multicolumn{1}{c}{Symbol} \\
%    \midrule
%    \unitinfo{bel} \\
%    day & \csindex{Day} & \Day \\
%     & \csindex{dday} & \dday \\
%    \unitinfo{degree} \\
%    \unitinfo{hour} \\
%    \unitinfo{litre} \\
%     & \csindex{liter} & \liter \\
%    \unitinfo{minute} \\
%    minute (arc) & \csindex{arcmin} & \arcmin \\
%    \unitinfo{neper} \\
%    \unitinfo{percent} \\
%    second (arc) & \csindex{arcsec} & \arcsec \\
%    \unitinfo{tonne} \\
%    \bottomrule
%  \end{tabular}
%\end{table}
%\begin{table}
%  \centering
%  \caption{Additional (temporary) SI units
%    (\opt{load=addn})}
%  \label{tab:addn}
%  \begin{tabular}{lll}
%    \toprule
%      \multicolumn{1}{c}{Unit} &
%      \multicolumn{1}{c}{Macro} &
%      \multicolumn{1}{c}{Symbol} \\
%    \midrule
%    {\aa}ngstr{\"o}m & \csindex{angstrom} & \angstrom \\
%    \unitinfo{are} \\
%    \unitinfo{curie} \\
%    bar & \csindex{BAR} & \BAR \\
%     & \csindex{bbar} & \bbar \\
%    \unitinfo{barn} \\
%    \unitinfo{gal} \\
%    \unitinfo{hectare} \\
%    \unitinfo{millibar} \\
%    \unitinfo{rad} \\
%    \unitinfo{rem} \\
%    \unitinfo{roentgen} \\
%    \bottomrule
%  \end{tabular}
%\end{table}
%
%\subsection{Prefixed and abbreviated units}
% Many basic units have prefixes which are commonly used with the
% unit, such as centimetre or megahertz.  The package therefore
% defines a number of common prefixed units (\opt{load=prefixed}).
% Several of these also have obvious abbreviations (such as \cs{MHz}
% for \cs{megahertz}), which are made available by \opt{load=abbr}.
% In common with the units discussed above, the prefixed and
% abbreviated unit definitions are loaded by default.
%\newcommand*{\tablesubhead}[1]{^^A
%  \addlinespace \emph{#1} \\*}
%\begin{longtable}{llll}
%    \caption{Prefixed (\opt{load=prefixed}) and abbreviated
%      (\opt{load=abbr}) units\label{tab:abbr}}\\
%    \toprule
%    Unit & Macro & Symbol & Abbreviation \\
%    \midrule
%  \endfirsthead
%    \toprule
%    Unit & Macro & Symbol & Abbreviation \\
%    \midrule
%  \endhead
%    \bottomrule
%    \multicolumn{4}{r}{\emph{Continued on next page}}
%  \endfoot
%    \bottomrule
%  \endlastfoot
%
%  \tablesubhead{Masses}
%  \unitinfo{kilogram} & \csindex{kg} \\
%  \unitinfo{femtogram} & \csindex{fg} \\
%  \unitinfo{picogram} & \csindex{pg} \\
%  \unitinfo{nanogram} & \csindex{nanog} \\
%  \unitinfo{microgram} & \csindex{micg} \\
%  \unitinfo{milligram} & \csindex{mg} \\
%  atomic mass & \csindex{atomicmass} & \amu & \csindex{amu} \\
%
%  \tablesubhead{Lengths}
%  \unitinfo{picometre} & \csindex{picom} \\
%  \unitinfo{nanometre} & \csindex{nm} \\
%  \unitinfo{micrometre} & \csindex{micm} \\
%  \unitinfo{millimetre} & \csindex{mm} \\
%  \unitinfo{centimetre} & \csindex{cm} \\
%  \unitinfo{decimetre} & \csindex{dm} \\
%  \unitinfo{kilometre} & \csindex{km} \\
%
%  \tablesubhead{Times}
%  \unitinfo{second} & \csindex{Sec} \\
%  \unitinfo{attosecond} & \csindex{as} \\
%  \unitinfo{femtosecond} & \csindex{fs} \\
%  \unitinfo{picosecond} & \csindex{ps} \\
%  \unitinfo{nanosecond} & \csindex{ns} \\
%  \unitinfo{microsecond} & \csindex{mics} \\
%  \unitinfo{millisecond} & \csindex{ms} \\
%
%  \tablesubhead{Moles}
%  \unitinfo{femtomole} & \csindex{fmol} \\
%  \unitinfo{picomole} & \csindex{pmol} \\
%  \unitinfo{nanomole} & \csindex{nmol} \\
%  \unitinfo{micromole} & \csindex{micmol} \\
%  \unitinfo{millimole} & \csindex{mmol} \\
%  \unitinfo{kilomole} & \csindex{kmol} \\
%
%  \tablesubhead{Currents}
%  \unitinfo{picoampere} & \csindex{pA} \\
%  \unitinfo{nanoampere} & \csindex{nA} \\
%  \unitinfo{microampere} & \csindex{micA} \\
%  \unitinfo{milliampere} & \csindex{mA} \\
%  \unitinfo{kiloampere} & \csindex{kA} \\
%
%  \tablesubhead{Areas}
%  square centimetre & \csindex{squarecentimetre} &
%    \squarecentimetre & \csindex{cms} \\*
%    & \csindex{centimetresquared} & \centimetresquared \\
%  square metre & \csindex{squaremetre} & \squaremetre \\
%  square kilometre & \csindex{squarekilometre} & \squarekilometre \\
%
%  \tablesubhead{Volumes}
%\changes{v1.1d}{2008/10/29}{New units \cs{microliter}, \cs{milliliter},
%  \cs{micL} and \cs{mL}}
%  \unitinfo{microlitre} & \csindex{micl} \\
%  \unitinfo{millilitre} & \csindex{ml} \\
%  \unitinfo{microliter} & \csindex{micL} \\
%  \unitinfo{milliliter} & \csindex{mL} \\
%  cubic centimetre & \csindex{cubiccentimetre} &
%    \cubiccentimetre & \csindex{cmc} \\*
%    & \csindex{centimetrecubed} & \centimetrecubed \\
%  cubic decimetre & \csindex{cubicdecimetre} &
%    \cubicdecimetre & \csindex{dmc} \\
%
%  \tablesubhead{Frequencies}
%  \unitinfo{hertz} & \csindex{Hz} \\
%  \unitinfo{millihertz} & \csindex{mHz} \\
%  \unitinfo{kilohertz} & \csindex{kHz} \\
%  \unitinfo{megahertz} & \csindex{MHz} \\
%  \unitinfo{gigahertz} & \csindex{GHz} \\
%  \unitinfo{terahertz} & \csindex{THz} \\
%
%  \tablesubhead{Potentials}
%  \unitinfo{millivolt} & \csindex{mV} \\
%  \unitinfo{kilovolt} & \csindex{kV} \\
%
%  \tablesubhead{Energies}
%  \unitinfo{kilojoule}  & \csindex{kJ}\\
%  \unitinfo{electronvolt} & \csindex{eV} \\
%  \unitinfo{millielectronvolt} & \csindex{meV} \\
%  \unitinfo{kiloelectronvolt} & \csindex{keV} \\
%  \unitinfo{megaelectronvolt} & \csindex{MeV} \\
%  \unitinfo{gigaelectronvolt} & \csindex{GeV} \\
%  \unitinfo{teraelectronvolt} & \csindex{TeV} \\
%  \unitinfo{kilowatthour} & \cs{kWh} \\
%
%  \tablesubhead{Powers}
%  \unitinfo{milliwatt} \\
%  \unitinfo{kilowatt} \\
%  \unitinfo{megawatt} \\
%
%  \tablesubhead{Capacitances}
%  \unitinfo{femtofarad} \\
%  \unitinfo{picofarad} \\
%  \unitinfo{nanofarad} \\
%  \unitinfo{microfarad} \\
%  \unitinfo{millifarad} \\
%
%  \tablesubhead{Resistances}
%  \unitinfo{kilohm} \\
%  \unitinfo{megohm} \\
%  \unitinfo{gigaohm} \\
%  \unitinfo{millisiemens} \\
%
%  \tablesubhead{Forces}
%  \unitinfo{millinewton} \\
%  \unitinfo{kilonewton} \\
%
%  \tablesubhead{Other units}
%  \unitinfo{hectopascal} \\
%  \unitinfo{megabecquerel} \\
%  \unitinfo{millisievert} \\
%\end{longtable}
%
%\subsection{Defining new units}
%\DescribeMacro{\newunit}
%\DescribeMacro{\renewunit}
%\DescribeMacro{\provideunit}
% New units are produced using the \cs{newunit} macro.  This works as
% might be expected:
% \cs{newunit}\oarg{options}\marg{unit}\marg{symbol}, where
% \meta{symbol} can contain literal values, other units, multiple
% prefixes, powers and \cs{per}.  The \meta{options} argument can be
% any suitable options, and applies the specific unit macro only. The
% most obvious example for using this macro is the \cs{degree}
% unit.\footnote{Although the \cs{ang} macro is preferred for this
% job.}  The (first) optional argument to \cs{SI} and \cs{si} can be
% used to override the settings for the unit.  The \cs{renewunit} and
% \cs{provideunit} macros take the same arguments.
%\begin{LaTeXexample}
%  \SI{3.1415}{\degree}\\
%  \newunit[valuesep=none]{\oddunit}{XXX}
%  \SI{12345}{\oddunit}
%  \SI[valuesep=thick]{67890}{\oddunit}
%\end{LaTeXexample}
%
% As with the \LaTeX\ commands \cs{newcommand}, \etc, the choice of
% \cs{newunit}, \cs{renewunit} or \cs{provideunit} depends on the
% presence of an existing definition.  While \cs{newunit} should be
% used when a unit has not been previously defined, \cs{renewunit}
% will issue a warning if the named unit does not already exist.
% \cs{provideunit} defines the unit if it does not exist, and
% otherwise does nothing at all.  The same behaviour is seen with
% \cs{providepower} and \cs{provideprefix} (\latin{vide infra}).
%
% Output that is only valid in maths mode requires \cs{ensuremath},
% text-only input requires \cs{text}.  In the example below,
% \cs{mathnormal} is used to force the font choice only for the
% single character.\footnote{The \cs{mathrm} font used for this
% document has an ``\ss'' at the $\pi$ position.}
%\begin{LaTeXexample}
%  \newunit{\SIpi}{\ensuremath{\mathnormal{\pi}}}
%  \SI{10}{\metre\per\SIpi\squared}
%\end{LaTeXexample}
%
%\DescribeMacro{\newpower}
%\DescribeMacro{\renewpower}
%\DescribeMacro{\providepower}
% Powers are defined:
% \cs{newpower}\ofixarg{post}\marg{power}\marg{num}. Here,
% \meta{power} is the name of the power macro and \meta{num} is the
% (positive) number it represents.  The later argument is always
% processed internally by \cs{num}, but \emph{must} be a number.
% Giving the optional argument \opt{post} indicates to the package
% that the power will come after the unit it applies to; by default
% it is assumed that it will come before.
%\begin{LaTeXexample}
%  \newpower{\quartic}{4}
%  \newpower[post]{\totheforth}{4}\\
%  \si{\kilogram\totheforth}\\
%  \si{\quartic\metre}
%\end{LaTeXexample}
%
%\DescribeMacro{\newprefix}
%\DescribeMacro{\renewprefix}
%\DescribeMacro{\provideprefix}
% The standard SI powers of ten are defined by the package, and are
% described above.  However, the user can define new prefixes with
% \cs{newprefix}.  This has syntax
% \cs{newunit}\ofixarg{binary}\marg{prefix}\marg{symbol}\marg{powers-ten},
% where \meta{powers-ten} is the number of powers of ten the prefix
% represents.  When the \opt{binary} option is given, the prefix is a
% power of two.  For example, \cs{kilo} and \cs{kibi} are defined:
%\begin{LaTeXexample}[noexample]
%  \newprefix{\kilo}{k}{3}
%  \newprefix[binary]{\kibi}{Ki}{10}
%\end{LaTeXexample}
%
%\changes{v1.1}{2008/09/22}{Qualifiers introduced}
%\DescribeMacro{\newqualifier}
%\DescribeMacro{\renewqualifier}
%\DescribeMacro{\providequalifier}
% It is possible to create unit ``qualifiers'', which add subscript
% descriptions to units.  Although not encouraged, this is sometimes
% necessary to make the meaning clear.  No qualifiers are pre-defined
% by \currpkg, and so the user needs to declare them using
% \cs{newqualifier}. Currently, only subscript qualifiers can be
% created; a more extended set of options is planned for the next
% release of \currpkg.
%\begin{LaTeXexample}
%  \newqualifier{\polymer}{pol}
%  \newqualifier{\catalyst}{cat}
%  \si{\kg\polymer\per\mole\catalyst\per\hour}
%\end{LaTeXexample}
%
%\section{Specialist units}
% In some subject area, there are units which are in common use even
% though they are outside of the SI system.  Unlike the units
% discussed earlier, these specialist units are not loaded by
% default.  In each case, they should be requested with the option
% \opt{alsoload=\meta{name}}.
%
%\subsection{Binary units (\opt{binary})\label{sec:binary}}
%\DescribeMacro{\bit}
%\DescribeMacro{\byte}
% The binary prefixes, \cs{bit}, \cs{byte} (\ref{tab:binary}) are not
% formally part of the SI system. They are available by giving the
% \opt{alsoload=binary} option.
%\begin{LaTeXexample}
%  \SI{100}{\mebi\byte}
%\end{LaTeXexample}
%\begin{table}
%  \caption{Binary prefixes
%    (\opt{alsoload=binary})}
%  \label{tab:binary}
%  \centering
%  \begin{tabular}{lllc}
%    \toprule
%      \multicolumn{1}{c}{Prefix} &
%      \multicolumn{1}{c}{Macro} &
%      \multicolumn{1}{c}{Power} &
%      Symbol \\
%    \midrule
%      \prefixinfo{kibi} \\
%      \prefixinfo{mebi} \\
%      \prefixinfo{gibi} \\
%      \prefixinfo{tebi} \\
%      \prefixinfo{pebi} \\
%      \prefixinfo{exbi} \\
%    \bottomrule
%  \end{tabular}
%\end{table}
%
%\subsection{Synthetic chemistry (\opt{synchem})}
%\DescribeMacro{\mmHg}
%\DescribeMacro{\molar}
%\DescribeMacro{\Molar}
%\DescribeMacro{\torr}
%\DescribeMacro{\dalton}
% The \opt{synchem} file adds the common chemistry units \cs{mmHg},
% \cs{molar}, \cs{Molar}, \cs{torr} and \cs{dalton} to \currpkg.  The
% \cs{Molar} macro is somewhat awkward, as it can be given as either
% ``\textsc{m}'' or ``M''.  The later is obviously easily confused
% with the sign for the prefix mega.  By default, \currpkg uses the
% UK default of a small-caps symbol.  The \cs{dalton} unit is defined
% here as this name is not recognised by the various international
% bodies: the symbol \si{\amu} is preferred.
%\begin{LaTeXexample}
%  \SI{1}{\Molar} HCl\\
%  \SI{760}{\torr}\\
%  \SI{0.01}{\mmHg}\\
%  \SI{3.0}{\molar}\\
%  \SI{106.42}{\dalton}
%\end{LaTeXexample}
%
%\subsection{High-energy physics (\opt{hep})}
% In contrast to \pkg{hepunits}, \currpkg does not define a long list
% of compound units for high-energy physics.\footnote{Using the
% \opt{emulate=hepunits} option will load a file defining those.}
% Instead, a small selection of new units are defined
% (\ref{tab:hep}).  The mechanisms provided by \currpkg should avoid
% the need for large numbers of abbreviations.  For example, the
% \pkg{hepunits} \cs{MinveV} can be given as |\per\MeV| in \currpkg,
% which requires only one more character.
%\begin{table}
%  \caption{High-energy physics units (\opt{alsoload=hep})}
%  \label{tab:hep}
%  \centering
%  \begin{tabular}{lllc}
%    \toprule
%    Unit & Macro & Symbol & Abbreviation \\
%    \midrule
%    \addlinespace \emph{Areas} \\
%    \unitinfo{yoctobarn} & \csindex{yb} \\
%    \unitinfo{zeptobarn} & \csindex{zb} \\
%    \unitinfo{attobarn} & \csindex{ab} \\
%    \unitinfo{femtobarn} & \csindex{fb} \\
%    \unitinfo{picobarn} & \csindex{pb} \\
%    \unitinfo{nanobarn} & \csindex{nb} \\
%    \addlinespace \emph{Other units} \\
%    \unitinfo{micron} \\
%    millirad & \csindex{mrad} & \mrad \\
%    \unitinfo{gauss} \\
%    \bottomrule
%  \end{tabular}
%\end{table}
%
%\DescribeMacro{\clight}
%\DescribeMacro{\eVperc}
% The \opt{hep} option defines two units which are slightly unusual.
% \cs{clight} gives \clight, which is recognised as a unit when used
% in the appropriate circumstances. The second unit provided is
% \cs{eVperc}, which is commonly-used and clear enough for a compound
% definition. Notice that the value of \opt{eVcorrb} will need to be
% adjusted when using this unit.
%\begin{LaTeXexample}
%  \SI[per=slash,eVcorrb=0.4ex]{4.657}{\mega\eVperc\squared}
%\end{LaTeXexample}
%
%\subsection{Astronomy (\opt{astro})}
%\DescribeMacro{\parsec}
%\DescribeMacro{\lightyear}
% For astronomers, the \cs{parsec} and \cs{lightyear} units are
% available, and give the obvious results.
%\begin{LaTeXexample}
%  \SI{12}{\parsec}\\
%  \SI{1}{\lightyear}
%\end{LaTeXexample}
%
%
%\subsection{Geophysics (\opt{geophys})}
%\changes{v1.1j}{2008/12/22}{New module \opt{geophys} containing \cs{gon}
%  unit}
%\DescribeMacro{\gon}
% For geophysics, the unit \cs{gon} is available.
%\begin{LaTeXexample}
%  \SI{12}{\gon}
%\end{LaTeXexample}
%
%\section{Font control}
% Following the lead of \SIstyle, \currpkg provides control over the
% font used to typeset output.  By default, all text is typeset using
% the current upright serif maths font, whether the macros are given
% in text or maths mode.  Some examples will show the effect.
%\begin{LaTeXexample}
%  \num{10} $\num{10}$\\
%  \sffamily \ang{20} $\ang{20}$\\
%  \textbf{\SI{30}{\kilo\gram}}\\
%  \boldmath $\SI{40}{\kilo\gram}$
%  \[ \num{50} \]
%\end{LaTeXexample}
%
% In contrast, by setting \opt{obeyall}, the current font is used:
% this may be maths or text, depending on the context.
%\begin{LaTeXexample}
%  \sisetup{obeyall}\\
%  \ang{1;1;1} $\ang{1;1;1}$\\
%  \sffamily \ang{2;2;2} $\ang{2;2;2}$\\
%  \textbf{\ang{3;3;3}} \boldmath $\ang{3;3;3}$\\
%  \emph{\ang{4;4;4}} \emph{$\ang{4;4;4}$}
%  \[ \ang{5;5;5} \]
%\end{LaTeXexample}
%
% Fine control of which elements of the local font are used is
% available with the \opt{obeyfamily}, \opt{obeybold},
% \opt{obeyitalic} and \opt{obeymode} options.
%\begin{LaTeXexample}
%  \sisetup{obeyfamily}\\
%  \ang{1;1;1} $\ang{1;1;1}$\\
%  \sffamily \ang{2;2;2} $\ang{2;2;2}$\\
%  \sisetup{obeybold}
%  \textbf{\ang{3;3;3}} \boldmath $\ang{3;3;3}$\\
%  \emph{\ang{4;4;4}} \emph{$\ang{4;4;4}$}
%  \[ \ang{5;5;5} \]
%\end{LaTeXexample}
%
%\section{Package options\label{sec:options}}
%\DescribeMacro{\sisetup}
% The ``native'' options for the package are all given using the
% key--value method. Most of the package options can be given both
% when loading the package and at any point in the document.  This is
% achieved using the \cs{sisetup} macro.
%
% The package options take a number of different forms.
%\begin{itemize}
%  \item \opt{option=\meta{bool}} Simple true/false values.  These
%    macros all default to \opt{option=true}, meaning that giving the
%    option name along will set the appropriate flag.
%  \item \opt{option=\meta{choice}} Take a single item from a
%    pre-determined list. Depending on the value, one or more
%    internal states will be altered.  Values not on the list are
%    ignored (with a warning).
%  \item \opt{option=\meta{choice,literal}} If the given value is a
%    \meta{choice}, then the internal settings for that choice are
%    used.  Any other value is used directly.
%  \item \opt{option=\meta{literal}} The given value is used as a
%    literal by the package.
%  \item \opt{option=\meta{csname}} These options expect a command
%    sequence as a value.
%  \item \opt{option=\meta{length}} Requires a \TeX\ length, for
%    example \opt{0.5ex}.
%  \item \opt{option=\meta{list}} Takes a list of one or more items,
%    which are not determined in advance.
%  \item \opt{option=\meta{number}} Takes a number (possibly
%    including an exponent part).
%\end{itemize}
%
% The package has a large range of options, to allow full control of
% the various features of the package.  These control differing
% aspects of the package, and are given below in groups based on
% function.  Where the key has a default value, it is given in bold.
%
%\subsection{Font family and style}
% The font used when typesetting material can be tightly controlled
% using \currpkg.  A number of options affect how the package matches
% the surrounding font, and the font families used to achieve this.
% The default is to use the current upright maths serif font with no
% variation.
%
%\DescribeOption{mode}
%\DescribeOption{textmode}
%\DescribeOption{valuemode}
%\DescribeOption{unitmode}
%\DescribeOption{obeymode}
% The output of \currpkg can occur using either text or maths mode.
% The package option \opt{mode} determines which is used: valid
% options are \defaultopt{maths} and \opt{text}.\footnote{Here and in
% all other cases, either UK or US spelling may be used.  Thus
% \opt{mode=maths} or \opt{mode=math} have exactly the same effect.}
% The shortcut \opt{textmode} is provided for setting \opt{mode=text}
% quickly. Further refinement is possible using the \opt{valuemode}
% and \opt{unitmode} options.  These apply to numbers (the output of
% \cs{num} and the first mandatory argument of \cs{SI}) and units
% (all other output), respectively. By setting the \opt{obeymode}
% flag, the package will use the local typesetting mode (maths or
% text).
%
%\DescribeOption{obeyall}
%\DescribeOption{obeyfamily}
%\DescribeOption{obeybold}
%\DescribeOption{obeyitalic}
% The detection and matching of surrounding text can be controlled
% using a number of Boolean package options. \opt{obeyall} turns on
% all of the detection.  Thus output with \opt{obeyall} in force will
% always match the local text appearance.  \opt{obeyfamily} instructs
% the package on detecting the surrounding font family (Roman, sans
% serif, fixed width), but does not detect bold or italic.
% \opt{obeybold} detects the local bold setting, whilst
% \opt{obeyitalic} picks up italic fonts.
%
% Bold detection is influenced by the value of \opt{inlinebold},
% which takes values \defaultopt{text} and \opt{maths}.  The package
% can detect the local value of bold for either the surrounding text,
% or the surrounding inline (|$|\ldots|$|) maths.  The
% \opt{obeyitalic} option does \emph{not} have the same facility
% (maths is italic anyway).
%
%\DescribeOption{mathsrm}
%\DescribeOption{mathssf}
%\DescribeOption{mathstt}
%\SpecialOptionIndex{mathrm}
%\SpecialOptionIndex{mathsf}
%\SpecialOptionIndex{mathtt}
%\DescribeOption{textrm}
%\DescribeOption{textsf}
%\DescribeOption{texttt}
% The font commands used by the package to achieve the above are all
% available for user modification.  The options \opt{mathsrm},
% \opt{mathssf} and \opt{mathstt} hold the command sequences used in
% maths mode,\footnote{These can also be set using \opt{mathrm},
% \opt{mathsf} and \opt{mathtt}} while \opt{textrm}, \opt{textsf} and
% \opt{texttt} do the same for text mode.  By default, these contain
% the obvious command names, for example \opt{mathsrm=mathrm} and
% \opt{texttt=ttfamily}.  However, they can be set at will: the macro
% names indicate the nature of the surrounding text detected.  For
% example, the value of \opt{mathssf} is used in maths mode when the
% surrounding text is sans serif.
%
%\SpecialOptionIndex{valuemathsrm}
%\SpecialOptionIndex{valuemathssf}
%\SpecialOptionIndex{valuemathstt}
%\SpecialOptionIndex{valuemathrm}
%\SpecialOptionIndex{valuemathsf}
%\SpecialOptionIndex{valuemathtt}
%\SpecialOptionIndex{unitmathsrm}
%\SpecialOptionIndex{unitmathssf}
%\SpecialOptionIndex{unitmathstt}
%\SpecialOptionIndex{valuetextrm}
%\SpecialOptionIndex{valuetextsf}
%\SpecialOptionIndex{valuetexttt}
%\SpecialOptionIndex{valuetextrm}
%\SpecialOptionIndex{valuetextsf}
%\SpecialOptionIndex{valuetexttt}
%\SpecialOptionIndex{unittextrm}
%\SpecialOptionIndex{unittextsf}
%\SpecialOptionIndex{unittexttt}
% Each of the font options can be given separately for the contents
% of numbers and units.  The option names include \opt{value} or
% \opt{unit} before the mode name.  For example, the \opt{mathsrm}
% option may be split into \opt{valuemathsrm} and \opt{unitmathsrm}.
%
%\DescribeOption{detectdisplay}
% The font detection system can treat displayed mathematical content
% in two ways.  This is controlled by the \opt{detectdisplay} option.
% When set to \defaultopt{true}, display mathematics is treated
% independently from the body of the document.  Thus the local
% \emph{maths} font is checked for matching.  In contrast, when set
% to \opt{false}, display material is treated with the current
% running text font.
%\begin{LaTeXexample}
%  \sffamily
%  Some text
%  \sisetup{obeyall}
%  \[ x = \SI{1.2e3}{\kg\kelvin\candela} \]
%  More text
%  \sisetup{detectdisplay=false}
%  \[ y = \SI{3}{\metre\second\mole} \]
%\end{LaTeXexample}
%
%\subsection{Spacing and separators}
%\DescribeOption{unitsep}
%\DescribeOption{valuesep}
%\DescribeOption{digitsep}
%\DescribeOption{anglesep}
% The separators between items can all be set using options taking a
% list of pre-defined items or a literal value.  The ``\opt{sep}''
% options (\opt{unitsep}, \opt{valuesep}, \opt{digitsep} and
% \opt{anglesep}) all recognise \opt{thin}, \opt{medium}, \opt{med},
% \opt{thick}, \opt{space}, \opt{cdot}, \opt{times}, \opt{tightcdot},
% \opt{tighttimes}, \opt{fullstop}, \opt{stop}, \opt{period} and
% \opt{none}.  The named spaces are the normal maths separations,
% with \opt{space} representing a full (non-breakable text) space,
% and with the obvious meanings for \opt{cdot} and \opt{times}.  The
% \opt{tight} variants reduce the spacing available. Three possible
% values are provided for ``|.|'', and \opt{none} yields no space at
% all. In all cases, other values are treated literally and are
% typeset in maths mode.  The default value is \defaultopt{thin} for
% all separations except \opt{anglesep}, which is set to
% \defaultopt{none}.
%
%\DescribeOption{unitspace}
%\DescribeOption{errspace}
% The \opt{unitspace} and \opt{errspace} options again take a list
% or literal value, but only the ``real'' spaces \opt{thin},
% \opt{medium}, \opt{med}, \opt{thick}, \opt{space} and \opt{none}
% are recognised in the list. The \opt{unitspace} option controls the
% output generated by an explicit space (|~|) inside a unit macro,
% while \opt{errspace} is used to separate a bracketed error from the
% main number.
%
%\subsection{Number formatting}
%\DescribeOption{numdigits}
%\DescribeOption{numdecimal}
%\DescribeOption{numdiv}
%\DescribeOption{numexp}
%\DescribeOption{numsign}
%\DescribeOption{numaddn}
%\DescribeOption{numgobble}
% There are two groups of options for formatting numbers.  The first
% group all begin with ``\opt{num}'', and take literal values used by
% the package to parse numbers.  \opt{numdigits} contains the valid
% number symbols (\defaultopt{0123456789}), with \opt{numdecimal}
% containing the decimal markers (\defaultopt{.,}).  As in the
% \numprint package, \opt{numexp} (the list of exponent markers)
% recognises \defaultopt{deDE} as valid by default.  \opt{numsign}
% contains the sign markers for numbers
% (\defaultopt{+-\cs{pm}\cs{mp}}).  \opt{numdiv} holds the division
% marker in numbers (\defaultopt{/}). \opt{numaddn} and \opt{numgobble}
% both control which other characters do not give an error when present
% in a number.  \opt{numaddn} contains valid characters which should be
% included in the final output ``as is'', whereas \opt{numgobble} lists
% the characters that are completely ignored.  In all cases, the
% content of the options is a simple string, for example
% \opt{numdigits=1234567890}.
%
%\DescribeOption{decimalsymbol}
%\DescribeOption{digitsep}
%\DescribeOption{sepfour}
% The second group of number options control the output of numbers
% after parsing. The symbol used by \currpkg as a decimal marker is
% set by the \opt{decimalsymbol} option, which can take a list of
% choices or a literal.  The valid choices here are
% \defaultopt{fullstop}, \opt{comma}, \opt{cdot} and
% \opt{tightcdot}.\footnote{\opt{fullstop} also has aliases
% \opt{stop} and \opt{period}.}  Notice that this does not have to
% agree with the input marker.  The other separator for numerical
% output is the division of digits into groups of three.  The result
% is dependent on two options.  The previously-described
% \opt{digitsep} option controls the spacing added between groups of
% three numbers.  For numbers consisting of exactly four digits, the
% \opt{sepfour} Boolean option controls whether separation occurs in
% these cases.  The default is \defaultopt{false}.
%\begin{LaTeXexample}
%  \num{1234}\\
%  \num[sepfour]{1234}
%\end{LaTeXexample}
%
%\DescribeOption{seperr}
%\DescribeOption{numopenerr}
%\DescribeOption{numcloseerr}
% For numbers given with an error [\eg \num{1.23(4)}], the package
% can separate out the error part, to give for example
% \num[seperr]{1.23(4)}.  This behaviour is activated by the
% \opt{seperr} option, and requires that \opt{numopenerr} and
% \opt{numcloseerr} contain the left- and right-hand delimiters for
% the error [defaults \opt{numopenerr=(} and \opt{numcloseerr=)}].
%\begin{LaTeXexample}
%  \sisetup{seperr}\\
%  \num{1.234(5)}\\
%  \num{1.234(5)e6}
%\end{LaTeXexample}
%\DescribeOption{trapambigerr}
%\DescribeOption{openerr}
%\DescribeOption{closeerr}
%\DescribeOption{tightpm}
% If the number has an exponent, or if units are not repeated, then
% the result can be considered ambiguous.  By default, the package
% adds the markers stored in \opt{openerr} and \opt{closeerr} to
% remove the ambiguity; the options have the same default values as
% the input error markers.  Detection of a potentially-ambiguous
% error is controlled by the \opt{trapambigerr} option, although for
% numbers with units the \opt{repeatunits} option is also important.
% The spacing around the $\pm$ sign is normally set by \TeX. However,
% using the \opt{tightpm} option will cause this to be reduced to a
% minimum.
%\begin{LaTeXexample}
%  \sisetup{seperr}\\
%  \num[trapambigerr=false]{1.234(5)e6}\\
%  \SI{1.234(5)}{\metre}\\
%  \sisetup{repeatunits=false}
%  \SI{1.234(5)}{\metre}\\
%  \SI[trapambigerr=false]{1.234(5)}{\metre}\\
%  \num[tightpm]{1.234(5)}
%\end{LaTeXexample}
%
%\DescribeOption{numprod}
%\DescribeOption{repeatunits}
% The number processor can recognise products in the numerical input;
% the symbols used for products are stored in \opt{numprod}, with the
% default value of ``\opt{x}''.  By default, the \cs{SI} macro will
% repeat the units for a number given in this way.  This behaviour is
% altered  by the \opt{repeatunits} option, which takes the values
% \defaultopt{true}, \opt{false} and \opt{power}.  The later applies
% only when providing multiplied numbers with units, and converts the
% unit to the appropriate power rather than repeating the
% units.\footnote{This is a very simply option: do not expect it to
% work with anything except areas and volumes.} Notice that when
% applied to errors, \opt{repeatunits} takes priority over
% \opt{trapambigerr}.
%\begin{LaTeXexample}
%  \num{1 x 2 x 3}\\
%  \SI{4 x 5 x 6}{\metre}\\
%  \sisetup{repeatunits=false}
%  \SI{1.2 x 3.4 x 5.6}{\milli\metre\cubed}\\
%  \SI[seperr]{1.234(5)}{\metre}\\
%  \SI[seperr,
%    trapambigerr=false]
%    {9.1093897(54)e-31}{\kilo\gram}\\
%  \SI[seperr,repeatunits,
%    trapambigerr=false]
%    {9.1093897(54)e-31}{\kilo\gram}\\
%  \SI[repeatunits=power]{1 x 2 x 3}{\metre}
%\end{LaTeXexample}
%
%\DescribeOption{expproduct}
%\DescribeOption{expbase}
%\DescribeOption{allowzeroexp}
% The formatting of exponents is controlled by \opt{expproduct} and
% \opt{expbase}.  \opt{expproduct} sets the symbol used to indicate a
% product for exponents (\eg the $\times$ in \num{2e2}), while the
% value of \opt{expbase} sets the power used (the \num{10} in the
% example).  Both options accept a very short list of options:
% \defaultopt{times}, \opt{tighttimes}, \opt{cdot} and
% \opt{tightcdot} for the product, and \defaultopt{ten} and \opt{two}
% for the power.\footnote{The \opt{tighttimes} and \opt{tightcdot}
% options give the rather questionable results:
% \num[expproduct=tighttimes]{1e2} and
% \num[expproduct=tightcdot]{1e2}, as opposed to \num{1e2} and
% \num[expproduct=cdot]{1e2}.} Other choices are used literally. Also
% relevant to exponent processing is the \opt{allowzeroexp} option.
% By default, the package will suppress a zero exponent, but setting
% the flag will allow the output of \num[allowzeroexp]{e0}.
%
%\DescribeOption{addsign}
%\DescribeOption{sign}
%\DescribeOption{retainplus}
% Additions to the input can take the form of implicit signs and
% padded zeros.  The \opt{addsign} option takes a list of potential
% sites to add a sign: \defaultopt{none}, \opt{mantissa},
% \opt{exponent} and \opt{both}.\footnote{Aliases are provided:
% \opt{mant} = \opt{mantissa}, \opt{exp} = \opt{exponent}, \opt{all}
% = \opt{true} = \opt{both}, \opt{false} = \opt{none}.}  If no sign
% is given in the input, the setting here determines if one is added.
% The sign to add is stored in \opt{sign}, which takes the list of
% choices \defaultopt{plus}, \opt{minus}, \opt{pm} and \opt{mp}, or
% uses the input literally (in maths mode).  For positive numbers,
% the \opt{retainplus} option causes a $+$ sign explicitly in the
% input to be retained.  By default, the package will remove such
% signs.
%
%\DescribeOption{padnumber}
% The \opt{padnumber} option controls the addition of zeros to the
% input, to ``pad'' the result.  The option takes a list of choices:
% \defaultopt{leading}, \opt{trailing}, \opt{both} and
% \opt{none}.\footnote{Aliases: \opt{all} = \opt{true} = \opt{both},
% \opt{false} = \opt{none}.}  No additional precision is added by
% this option; integer input will not add a decimal point.
%\begin{LaTeXexample}
%  \num[padnumber=leading]{.1}\\
%  \num[padnumber=leading]{2.}\\
%  \num[padnumber=trailing]{3.}\\
%  \num[padnumber=both]{4.}\\
%  \num[padnumber=both]{.5}\\
%  \num[padnumber=both]{6}\\
%  \num[padnumber=none]{7.}\\
%  \num[padnumber=none]{.8}
%\end{LaTeXexample}
%
%\DescribeOption{fixdp}
%\DescribeOption{dp}
% In contrast to the \opt{padnumber} option, the package can alter
% the precision of the input number if the \opt{fixdp} option is set.
% The \opt{fixdp} option will fix the decimal places of the output to
% the number stored in the \opt{dp} option.  The later should be a 
% positive integer or zero.
%\begin{LaTeXexample}
%  \sisetup{fixdp,dp=3}
%  \num{1}\\
%  \num{53.9}\\
%  \num{4.56783}\\
%  \num{-1.2942}\\
%  \num{-1.2959}\\
%  \num{9.9999}\\
%  \num{2.1264e90}\\
%  \num[dp=0]{12.345}
%\end{LaTeXexample}
%
%\DescribeOption{fixsf}
%\changes{v1.1}{2008/08/22}{New \opt{fixsf} option}
%\DescribeOption{sf}
%\changes{v1.1}{2008/08/22}{New \opt{sf} option}
% In contrast to the \opt{dp} option, the \opt{fixsf} and \opt{sf}
% options control the number of significant figures in the output.
% This never adds extra digits to the output, as significant figures
% are concerned with accuracy.
%\begin{LaTeXexample}
%  \sisetup{fixsf,sf=3}
%  \num{1}\\
%  \num{53.9}\\
%  \num{4.56783}\\
%  \num{-1.2942}\\
%  \num{-1.2959}\\
%  \num{9.9999}\\
%  \num{2.1264e90}\\
%  \num{773322}
%\end{LaTeXexample}
%
%\subsection{Angle formatting}
%\DescribeOption{padangle}
%\DescribeOption{strictarc}
% The angle formatter uses \cs{num} to format numbers: any options
% for numbers are therefore applicable here.  The \opt{padangle}
% option takes choices \defaultopt{small}, \opt{large}, \opt{all} and
% \opt{none}, and controls how angles are padded when given in
% degrees, minutes and seconds.\footnote{Aliases: \opt{all} =
% \opt{true} = \opt{both}, \opt{false} = \opt{none}.}  When giving
% angles as arcs (in degrees, minutes and seconds), the package can
% detect if the correct number of semi-colons have been given. This
% is controlled by the \opt{strictarc} option, which is a Boolean
% switch with a \defaultopt{true} default.  With \opt{strictarc} set
% to \opt{false}, an incomplete arc is interpreted as degrees and
% minute, while an over-complete one will drop excess input.
%\begin{LaTeXexample}
%  \ang[padangle=none]{1;;}\\
%  \ang[padangle=large]{2;;}\\
%  \ang[padangle=small]{3;;}\\
%  \ang[padangle=both]{;4;}\\
%  \ang[padangle=none]{;5;5}\\
%  \ang[strictarc=false]{1;2}\\
%  \ang[strictarc=false]{1;2;3;4;}
%\end{LaTeXexample}
%
%\DescribeOption{angformat}
% The angle formatting system can convert between decimal angles and
% those given as degrees, minutes and seconds.  This is controlled by
% the \opt{angformat} option, which takes choices
% \defaultopt{unchanged}, \opt{decimal} and
% \opt{arc}.\footnote{Aliases: \opt{decimal} = \opt{dec}, \opt{arc} =
% \opt{dms}, \opt{unchanged} = \opt{none}.}  When set to
% \opt{unchanged}, nothing is done to the input.  The conversion is
% based on \TeX\ dimensions, and is therefore limited in accuracy.
% For this reason, the output is automatically rounded: output as a
% decimal angle is limited to three places, and that as an arc is
% given to a single decimal place for the seconds component.
%\begin{LaTeXexample}
%  $\ang{1;2;3} = \ang[angformat=dec]{1;2;3}$\\
%  $\ang{4.56} = \ang[angformat=arc]{4.56}$
%\end{LaTeXexample}
%
%\DescribeOption{astroang}
% For astronomers, the \opt{astroang} option is provided.  This
% moves the degrees, minutes or seconds symbol (as appropriate)
% over the decimal marker rather than after the number.
%\begin{LaTeXexample}
%  \ang{1;2;3.4}\\
%  \ang[astroang]{5;6;7.8}
%\end{LaTeXexample}
%
%\subsection{Tabular material}
%\DescribeOption{tabnumalign}
% Material typeset in |S| columns is processed internally by the
% \cs{num} macro.  Thus, as with angles, the number options also
% apply here.  The positioning of tabular material is controlled by
% the two options \opt{tabnumalign} and \opt{tabformat}.
% \opt{tabnumalign} takes values \defaultopt{centredecimal},
% \opt{centre}, \opt{left} and \opt{right}.\footnote{Aliases
% \opt{centerdecimal} = \opt{centredecimal}; \opt{center} =
% \opt{centre}.}  When using \opt{centredecimal}, the package places
% the decimal marker of the mantissa at the centre of the column,
% which then grows to accommodate the widest number given.  For equal
% numbers of digits before and after the decimal sign, this is the
% easiest option.  The other choices use a fixed-width box to store
% the number; the box is then aligned with the edges of the column.
%
%\DescribeOption{tabformat}
% The \opt{tabformat} option sets the amount of space reserved by
% \currpkg for the alignment box when not using the
% \opt{centredecimal} setting of \opt{tabnumalign}.  The numerical
% parts of \opt{tabformat} are interpreted as
% \texttt{\meta{pre}\meta{dec}\meta{post}}; \meta{pre} and
% \meta{post} are the number of digits before and after the decimal
% sign, respectively.  Both signs and exponents can be included in
% \opt{tabformat}, resulting in appropriate space being reserved. The
% entire \opt{tabformat} input is processed using the \cs{num} macro
% internally.  Thus the decimal and exponent signs used in
% \opt{tabformat} are checked against \opt{numdecimal} and
% \opt{numexp}, respectively.
%
%\DescribeOption{tabalignexp}
%\DescribeOption{tabexpalign}
%\changes{v1.1d}{2008/10/29}{Added \opt{tabexpalign} as an alias for
%  \opt{tabalignexp}}
% When \opt{tabformat} contains exponents, two possibilities are
% available for alignment.  The first method is to place the exponent
% parts so that the ``$\times 10$'' parts form a column, with
% whitespace after shorter mantissa components.  In the second
% method, no additional space is added after the mantissa, and the
% exponents do no line up (\ref{tab:alignexp}).  This is controlled
% by the \opt{tabalignexp} option, which can be set to
% \defaultopt{true} or \opt{false}. The alias \opt{tabexpalign} is
% available for this option.
%\begin{LaTeXexample}[float]
%  \begin{table}
%    \centering
%    \caption{The \opt{tabalignexp} option}
%    \label{tab:alignexp}
%    \sisetup{tabformat=1.3e2,tabnumalign=centre}
%    \begin{tabular}{SS[tabalignexp=false]}
%      \toprule
%      {Header} & {Header} \\
%      \midrule
%      1.2e3    & 1.2e3    \\
%      1.234e56 & 1.234e56 \\
%      \bottomrule
%    \end{tabular}
%  \end{table}
%\end{LaTeXexample}
%
%\DescribeOption{tabtextalign}
%\DescribeOption{tabunitalign}
%\DescribeOption{tabalign}
% Cells containing no numbers are handled by \currpkg in a manner
% similar to \cs{multicolumn}.  The setting of \opt{tabtextalign} is
% taken from the list \defaultopt{centre}, \opt{right} and
% \opt{left}.\footnote{Alias \opt{centre} = \opt{center}.}  As would
% be expected, these settings centre, right- or left-align the cell
% contents. In |s| columns, all content is treated as input to the
% \cs{si} macro.  The alignment of the contents relative to the cell
% is controlled by the \opt{tabunitalign} option, which takes options
% \defaultopt{left}, \opt{right} and \opt{centre}.  The settings for
% \opt{tabnumalign}, \opt{tabtextalign} and \opt{tabunitalign} can be
% set to the same value in one go with the \opt{tabalign} option.
%
%\DescribeOption{tabautofit}
% The contents of table cells can automatically be rounded or
% zero-filled to the number of decimal places given in
% \opt{tabformat}.  This is activated by the \opt{tabautofit} Boolean
% option. As \opt{tabformat} does not apply to columns with alignment
% \opt{centredecimal}, \opt{tabautofit} is also inactive for these
% columns (\ref{tab:autofit}).
%\begin{LaTeXexample}[float]
%  \begin{table}
%    \centering
%    \caption{The \opt{tabautofit} option}
%    \label{tab:autofit}
%    \sisetup{tabformat=1.3,tabnumalign=centre}
%    % Notice the overfull hbox which results with
%    % the first column
%    \begin{tabular}{%
%      S%
%      S[tabautofit]%
%      S[tabautofit,tabnumalign=centredecimal]}
%      \toprule
%      {Header} & {Header} & {Header} \\
%      \midrule
%      1.2    & 1.2    & 1.2    \\
%      1.2345 & 1.2345 & 1.2345 \\
%      \bottomrule
%    \end{tabular}
%  \end{table}
%\end{LaTeXexample}
%
%\DescribeOption{tabparseonly}
% It is possible to turn off the automatic alignment of numbers
% while retaining the ability to parse the |S| column input.
% Setting \opt{tabparseonly} to \opt{true} will still parse the
% column contents, but will not align the decimal markers.  The
% alignment of the cell is governed by \opt{tabnumalign}
% (\ref{tab:parse}).
%\begin{LaTeXexample}[float]
%  \begin{table}
%    \centering
%    \caption{The \opt{tabparseonly} option}
%    \label{tab:parse}
%    \begin{tabular}{%
%      S[tabparseonly]%
%      S[tabparseonly,tabnumalign=left]%
%      S[tabparseonly,tabnumalign=right]}
%      \toprule
%      {Header} & {Header} & {Header} \\
%      \midrule
%      14.2    & 14.2    & 14.2    \\
%      1.23456 & 1.23456 & 1.23456 \\
%      1.2e3   & 1.2e3   & 1.2e3   \\
%      \bottomrule
%    \end{tabular}
%  \end{table}
%\end{LaTeXexample}
%
%\subsection{Units}
%\DescribeOption{per}
% Most of the unit options are concerned with the processing of named
% units.  The processor for units given as macro names can be
% influenced to give a variety of output formats.  The \opt{per}
% option defines how the keyword macro \cs{per} is handled.  This
% option takes a choice from the list \defaultopt{reciprocal},
% \opt{slash} and \opt{fraction}.\footnote{Aliases \opt{reciprocal} =
% \opt{rp} = \opt{power}, \opt{fraction} = \opt{frac}.}  The default
% option uses \cs{per} to indicate reciprocal powers, whereas
% \opt{slash} causes the package to use ``$/$'' to show division.
%
%\DescribeOption{fraction}
%\DescribeOption{slash}
% The \opt{fraction} option defines how \opt{per=fraction} is
% interpreted.  The list of applicable values here is
% \defaultopt{frac}, \opt{nice}, \opt{ugly} and \opt{sfrac}.  In each
% case, the unit is typeset as a fraction, but the macro use to
% achieve this varies. \opt{frac} uses the \TeX\ \cs{frac}, macro,
% while \opt{nice} makes use of a \cs{nicefrac}-like method.  The
% \opt{ugly} option uses a slash in text mode and \cs{frac} in maths
% mode.\footnote{Similar to the \pkg{ugly} option of the
% \pkg{nicefrac} package.}  Finally, the setting \opt{fraction=sfrac}
% uses the \cs{sfrac} macro from the \pkg{xfrac} package, when
% available.\footnote{\pkg{xfrac} is part of the experimental system
% for \LaTeX3.  As it requires a number of additional packages to
% work, \currpkg does not load \pkg{xfrac}. If it is unavailable, the
% \opt{sfrac} setting will fall back to using \cs{nicefrac}.  See the
% \pkg{xfrac} documentation for reasons to prefer \cs{sfrac} to
% \cs{nicefrac}.}  The \opt{slash} option sets the symbol used when
% \opt{per=slash} is in force.  This recognises the single keyword
% \opt{slash}; anything else is used literally.
%\begin{LaTeXexample}
%  \sisetup{per=fraction}
%  \si[fraction=frac]{\metre\per\second}\\
%  \si[fraction=nice]{\metre\per\second}\\
%  \si[fraction=sfrac]{\metre\per\second}\\
%  \si[per=slash]{\metre\per\second}
%\end{LaTeXexample}
%
%\DescribeOption{stickyper}
% By default, \cs{per} applies only to the next unit
% given.\footnote{This is the standard method of reading units in
% English: for example, \si{\joule\per\mole\per\kelvin} is pronounced
% ``joules per mole per kelvin''.}  By setting the \opt{stickyper}
% flag, this behaviour is changed so that \cs{per} applies to all
% subsequent units.\footnote{This is the behaviour in \SIunits.}
%\begin{LaTeXexample}
%  \si{\kilogram\per\metre\ampere}\\
%  \si[stickyper]{\kilogram\per\metre\ampere}
%\end{LaTeXexample}
%
%\DescribeOption{trapambigfrac}
%\DescribeOption{openfrac}
%\DescribeOption{closefrac}
% When using \opt{per=slash}, multiple units in the denominator will
% yield a potentially ambiguous result.  The \opt{trapambigfrac}
% determines whether the package checks for this: this takes
% \defaultopt{true} and \opt{false}.  When set, the contents of
% \opt{openfrac} are inserted before the denominator, and
% \opt{closefrac} is inserted after.
%\begin{LaTeXexample}
%  \sisetup{trapambigfrac,openfrac=(,closefrac=),per=slash}\\
%  \si{\metre\per\second\per\kilogram}\\
%  \si[trapambigfrac=false]{\metre\per\second\per\kilogram}
%\end{LaTeXexample}
%
%\DescribeOption{prefixsymbolic}
%\DescribeOption{prefixbase}
%\DescribeOption{prefixproduct}
% The unit prefixes (\cs{kilo}, \etc) are normally given as letters.
% However, the package can convert these into numerical
% powers.\footnote{Provided things are not too complex!}  This is
% controlled by the \opt{prefixsymbolic} Boolean option, which by
% default is \defaultopt{true}.  If \opt{prefixsymbolic} is set to
% \opt{false}, the format of the prefix is controlled by
% \opt{prefixbase} and \opt{prefixproduct}, which work in the same
% way as \opt{expbase} and \opt{expproduct}.
%
%\DescribeOption{prespace}
%\DescribeOption{xspace}
%\DescribeOption{allowoptarg}
% By default, the single unit macros (\eg \cs{metre}) add no space
% either before or after the unit.  Setting the \opt{xspace} flag to
% true means that the single macros are followed by the \cs{xspace}
% command (when used outside of \cs{SI}/\cs{si}).  For users of
% \unitsdef, the \opt{prespace} macro changes the behaviour of the
% unit macros, so that they can immediately follow a number.  As a
% result, the unit macros will \emph{always} be preceded by a fixed
% space when the \opt{prespace} flag is true: this will be in
% addition to any other space. Also relevant to users moving from
% \unitsdef is the \opt{allowoptarg} option.  This allows single unit
% macros to take an optional numerical argument, in the same way that
% occurs in that package.
%\begin{LaTeXexample}
%  \metre is the symbol for metres\\
%  \sisetup{xspace}
%  No, \metre is the correct symbol\\
%  \sisetup{prespace}
%  30\metre\\
%  Do not use \metre in running text\\
%  \sisetup{allowoptarg}
%  \metre[40]
%\end{LaTeXexample}
%
%\subsection{Symbols}
% User access to control the symbols used for \si{\ohm}, \si{\micro},
% \si{\degree}, \si{\arcmin}, \si{\arcsec}, \si{\angstrom} and
% \si{\celsius} is provided here. These are all literal options,
% which are available in text and maths mode variants.  For example,
% \opt{textmicro} is the code used for the \si{\micro} symbol in text
% mode.  The text mode macros should be safe when forced into text,
% and the maths ones when forced into maths.  The symbols defined in
% this way are:
%\newcommand*{\listsymbopt}[1]{\item \opt{#1}\SpecialOptionIndex{#1}}
%\begin{itemize}
%  \listsymbopt{textOmega};
%  \listsymbopt{mathsOmega};
%  \listsymbopt{textmu};
%  \listsymbopt{mathsmu};
%  \listsymbopt{textdegree};
%  \listsymbopt{mathsdegree};
%  \listsymbopt{textminute};
%  \listsymbopt{mathsminute};
%  \listsymbopt{textsecond};
%  \listsymbopt{mathssecond};
%  \listsymbopt{textringA};
%  \listsymbopt{mathsringA};
%  \listsymbopt{textcelsius};
%  \listsymbopt{mathscelsius}.
%\end{itemize}
%
%\DescribeOption{redefsymbols}
% When \currpkg is loaded, it can check for the presence of the
% \pkg{textcomp} and \pkg{upgreek} packages, to provide better
% symbols for certain items. To prevent this, set the
% \opt{redefsymbols} option \opt{false} (the default is
% \defaultopt{true}).
%
%\DescribeOption{eVcorra}
%\DescribeOption{eVcorrb}
% The \si{\electronvolt} symbol requires some fine-tuning, and so has
% two options of its own, both \TeX\ lengths.  \opt{eVcorra} is the
% correction applied to the gap between ``e'' and ``V'' of the unit:
% the default is \defaultopt{0.3ex}. \opt{eVcorrb} is the correction
% applied to the gap between ``V'' of the unit and whatever follows;
% the default is \defaultopt{0ex}.  The optimal value for these
% options will depend on the current font settings.\footnote{This
% document uses \opt{eVcorra=0.1ex}.}
%\begin{LaTeXexample}
%  \si[per=slash]{\electronvolt\per\metre}\\
%  \si[per=slash,eVcorrb=0.7ex]{\electronvolt\per\metre}
%\end{LaTeXexample}
%
%\subsection{Colour}
%\DescribeOption{colourall}
%\DescribeOption{colourunits}
%\DescribeOption{colourvalues}
%\SpecialOptionIndex{colorall}
%\SpecialOptionIndex{colorunit}
%\SpecialOptionIndex{colorvalue}
% The package provides internal hooks for applying colour to part or
% all of the output.  This requires the user to load the \pkg{color}
% or \pkg{xcolor} package to support colour in the output; \currpkg
% will ignore a colour request if support is unavailable. The Boolean
% options \opt{colourall}, \opt{colourunits} and \opt{colourvalues}
% are used to turn application of a given colour on or off for all
% output, only units and only values, respectively.  All three
% switches are available with US spelling, \eg \opt{colorall} and
% \opt{colourall} behave in the same way.  With colour turned off, no
% \cs{color} command is issued internally, and output follows the
% surrounding text.
%
%\DescribeOption{colour}
%\DescribeOption{unitcolour}
%\DescribeOption{valuecolour}
%\SpecialOptionIndex{color}
%\SpecialOptionIndex{unitcolor}
%\SpecialOptionIndex{valuecolor}
% The colour names to use for colouring output are set by the
% \opt{colour}, \opt{unitcolour} and \opt{valuecolour} options (all
% also available with US spelling).  The \opt{colour} option
% internally sets both \opt{unitcolour} and \opt{valuecolour}.
%\begin{LaTeXexample}
%  {\color{brown} Text \num{50}}\\
%  \sisetup{colourall,colour=green}
%  \SI{10}{\metre}\\
%  {\color{purple} Text \num{70}}\\
%  \sisetup{colourunits=true,colourvalues=false}
%  \SI{40}{\second}
%\end{LaTeXexample}
%
%\DescribeOption{colourneg}
%\DescribeOption{negcolour}
%\SpecialOptionIndex{colorneg}
%\SpecialOptionIndex{negcolor}
% \currpkg can automatically add a colour to negative numbers.  This
% is turned on using the \opt{colourneg} switch.  The colour used is
% set by the \opt{negcolour} option; both options are available using
% US spellings.
%
%\subsection{International support}
%\DescribeOption{locale}
% \currpkg allows the user to switch between the typographic
% conventions of different (geographical) areas by using
% \latin{locales}.  Currently, the package is supplied with
% configurations for locales \opt{UK}, \opt{USA}, \opt{DE} (Germany)
% and \opt{ZA} (South Africa). The \opt{locale} option is used to
% switch to a particular locale.
%\begin{LaTeXexample}
%  \SI{1.234}{\metre}\\
%  \SI[locale=DE]{6.789}{\metre}
%\end{LaTeXexample}
%
%\DescribeOption{loctolang}
% Locales are distinct from \pkg{babel} languages, as typographic
% conventions are not tightly integrated with language.  However, it
% is useful to be able to associate a particular locale with a
% \pkg{babel} language.  The option \opt{loctolang} handles this, and
% expects pairs of values:
% \opt{loctolang=\meta{locale}:\meta{language}}.
%\begin{LaTeXexample}
%  \sisetup{loctolang={UK:UKenglish,DE:german}}\\
%  \SI{6.022e23}{\per\mole}\\
%  \selectlanguage{german}\\
%  \SI{6.022e23}{\per\mole}
%\end{LaTeXexample}
%
%\subsection{Package control}
%\DescribeOption{load}
%\DescribeOption{noload}
%\DescribeOption{alsoload}
% The package keeps most of the unit and abbreviations definitions in
% files separate from \file{siunitx.sty}.  To control what is loaded,
% three complementary options are provided, all of which take a list
% of one or more choices.  \opt{load} and \opt{alsoload} define which
% support configuration files are loaded.  The list in \opt{load}
% recognises the value \opt{default}, which is expanded to the normal
% list before loading. The difference between \opt{load} and
% \opt{alsoload} is that \opt{load} specifies the \emph{complete}
% list of files to load, whereas \opt{alsoload} adds to the existing
% list. To use the \opt{load} option successfully requires knowing
% everything that is needed. The \opt{noload} option can be used to
% delete one or more items from the \opt{load} list, without needing
% to know what is on it.\footnote{\opt{noload} does not prevent the
% loading of a file needed by one which is loaded.  Thus the package
% may internally override a \opt{noload} value if needed.}
%
%\DescribeOption{log}
%\DescribeOption{debug}
% To control data written to the \ext{log} file, the \opt{log} option
% is provided.  This takes a value from the list \defaultopt{normal},
% \opt{none}, \opt{minimal}, \opt{errors} and \opt{debug}.  As would
% be expected, these indicate the amount of detail written to the log
% file. As a shortcut to \opt{log=debug}, the package also recognises
% the \opt{debug} option directly.
%
%\DescribeOption{strict}
% Some users will want to stick closely to the official rules for
% typesetting units.  This could be made complicated if the options
% for non-standards behaviour could not be turned off.  The load-time
% option \opt{strict} resets package behaviour to follow the rules
% closely, and disables options which deviate from this.  If the
% package is loaded with the \opt{strict} option, all output is made
% in maths mode using the upright serif font.
%
%\subsection{Back-compatibility options}
%\DescribeOption{emulate}
% The package can emulate \SIunits, \SIstyle, \unitsdef, \units,
% \pkg{hepunits}, \pkg{fancyunits} and \pkg{fancynum}.  Giving the
% \opt{emulate=\meta{package}} option will give the desired
% emulation, and combinations which would be possible with the real
% packages will also work here. The package will recognise the
% options of the emulated packages.  This will automatically cause
% emulation to be switched on.
%
%\subsection{Summary of all options}
% \ref{tab:allopt} lists a summary of the package options (excluding
% those for backward-compatibility).  A reminder or the input format
% is also provided.
%\begin{longtable}{>{\ttfamily}lll}
%    \caption{All package options\label{tab:allopt}}\\
%    \toprule
%    Option & Type & Description \\
%    \midrule
%  \endfirsthead
%    \toprule
%    Option & Type & Description \\
%    \midrule
%  \endhead
%    \bottomrule
%    \multicolumn{3}{r}{\emph{Continued on next page}}
%  \endfoot
%    \bottomrule
%  \endlastfoot
%  addsign & List & Add sign to number \\
%  allowzeroexp & Boolean & Allow \num[allowzeroexp]{e0} \\
%  angformat & List & Conversion of angle format \\
%  anglesep & List or literal & Space between angle components \\
%  astroang & Boolean & Astronomy-style angles \\
%  closeerr & Literal & Closes potential-ambiguous error \\
%  closefrac & Literal & Closes potential-ambiguous fraction \\
%  color & Literal & Colour used for units and values \\
%  colour & Literal & Colour used for units and values \\
%  colorall & Boolean & Switch for colouring all output \\
%  colourall & Boolean & Switch for colouring all output \\
%  colorneg & Boolean & Colour negative numbers \\
%  colourneg & Boolean & Colour negative numbers \\
%  colorunits & Boolean & Switch for colouring units \\
%  colourunits & Boolean & Switch for colouring units \\
%  colorvalues & Boolean & Switch for colouring values \\
%  colourvalues & Boolean & Switch for colouring values \\
%  decimalsymbol & List or literal & Decimal symbol \\
%  debug & Boolean & Write debugging data to log \\
%  detectdisplay & Boolean & Treat display maths separately \\
%  digitsep & List & Separation of digits in large numbers \\
%  dp & Integer & Number of decimal places to \\
%   & & output numbers to \\
%  emulate & Modules & Emulation modules to load \\
%  errspace & List & Spacing of bracketed error \\
%  eVcorra & Length & Spacing correction in \eV \\
%  eVcorrb & Length & Spacing correction after \eV \\
%  expbase & List or literal & Base used for exponents \\
%  expproduct & List or literal & Product sign for exponents \\
%  fixdp & Boolean & Switch for fixing decimal \\
%   & & places of numbers \\
%  fixsf & Boolean & Switch for fixing significant \\
%    & & figures of numbers \\
%  fraction & List & Method used when setting \opt{per=frac} \\
%  inlinebold & List & Select how inline bold is tested \\
%  load & Modules & Modules to load \\
%  locale & Modules & Locale to follow \\
%  loctolang & Special & Associate locale with \pkg{babel}
%    language \\
%  log & List & Amount of data added to log \\
%  mathOmega & Literal & ``$\makeatletter\si@mathsOmega\makeatother$''
%    symbol in maths mode \\
%  mathcelsius & Literal & ``$\makeatletter\si@mathscelsius\makeatother$''
%    symbol in maths mode \\
%  mathdegree & Literal & ``$\makeatletter\si@mathsdegree\makeatother$''
%    symbol in maths mode \\
%  mathminute & Literal & ``$\makeatletter\si@mathsminute\makeatother$''
%    symbol in maths mode \\
%  mathmu & Literal & ``$\makeatletter\si@mathsmu\makeatother$''
%    symbol in maths mode \\
%  mathringA & Literal & ``$\makeatletter\si@mathsringA\makeatother$''
%    symbol in maths mode \\
%  mathrm & Csname & Roman maths font \\
%  mathsOmega & Literal & ``$\makeatletter\si@mathsOmega\makeatother$''
%    symbol in maths mode \\
%  mathscelsius & Literal & ``$\makeatletter\si@mathscelsius\makeatother$''
%    symbol in maths mode \\
%  mathsdegree & Literal & ``$\makeatletter\si@mathsdegree\makeatother$''
%    symbol in maths mode \\
%  mathsecond & Literal & ``$\makeatletter\si@mathssecond\makeatother$''
%    symbol in maths mode \\
%  mathsf & Csname & Sans serif maths font \\
%  mathsminute & Literal & ``$\makeatletter\si@mathsminute\makeatother$''
%    symbol in maths mode \\
%  mathsmu & Literal & ``$\makeatletter\si@mathsmu\makeatother$''
%    symbol in maths mode \\
%  mathsringA & Literal & ``$\makeatletter\si@mathsringA\makeatother$''
%    symbol in maths mode \\
%  mathsrm & Csname & Roman maths font \\
%  mathssecond & Literal & ``$\makeatletter\si@mathssecond\makeatother$''
%    symbol in maths mode \\
%  mathssf & Csname & Sans serif maths font \\
%  mathstt & Csname & Fixed-width maths font \\
%  mathtt & Csname & Fixed-width maths font \\
%  mode & List & Use text or maths mode for \\
%   & & typesetting \\
%  negcolor & Literal & Colour used for negative numbers \\
%  negcolour & Literal & Colour used for negative numbers \\
%  noload & Modules & Modules not to load \\
%  numaddn & Literal & Additional input allowed in numbers \\
%  numcloseerr & Literal & Character indicating end of \\
%   & & numerical error\\
%  numdecimal & Literal & Decimal symbols in numbers \\
%  numdigits & Literal & Digit characters in numbers \\
%  numdiv & Literal & Division characters in numbers \\
%  numexp & Literal & Exponent characters in numbers \\
%  numgobble & Literal & Characters to ignore in numbers \\
%  numopenerr & Literal & Character indicating start of \\
%   & & numerical error\\
%  numprod & Literal & Characters used for a product \\
%  numsign & Literal & Sign characters in numbers \\
%  obeyall & Boolean & Combination of \opt{obeybold}, \\
%   & & \opt{obeyitalic} and \opt{obeymode} \\
%  obeybold & Boolean & Check local bold setting \\
%  obeyitalic & Boolean & Check local italic setting \\
%  obeymode & Boolean & Check local mode (text/maths) \\
%  openerr & Literal & Opens potentially-ambiguous error \\
%  openfrac & Literal & Opens potentially-ambiguous fraction \\
%  padangle & List & Add zeros to blank parts of angles \\
%  padnumber & List & Add zeros to blank parts of numbers \\
%  per & List & Behaviour of \cs{per} \\
%  prefixbase & List or literal & Base used when making prefixes \\
%   & & numerical \\
%  prefixproduct & List or literal & Product sign for prefixes \\
%  prefixsymbolic & Boolean & Behaviour of unit prefixes\\
%  prespace & Boolean & Add space before units \\
%  redefsymbols & Boolean & Use better symbols if available \\
%  repeatunits & List & Repeat units with separated errors \\
%   & &  and products \\
%  retainplus & Boolean & Retain explicit plus sign\\
%  seperr & Boolean & Separate number and error \\
%  sepfour & Boolean & Separate four-digit numbers \\
%  sf & Integer & Number of significant figures \\
%   & & to output numbers to \\
%  sign & List or literal & Sign to add to numbers \\
%  slash & List or literal & Symbol used for ``/'' \\
%  stickyper & Boolean & Require \cs{per} only once \\
%  strict & Boolean & Obey the rules strictly \\
%  strictarc & Boolean & Require exactly zero or \\
%   & & two semi-colons \\
%  tabalign & List & Positioning of all column data \\
%  tabalignexp & Boolean & Alignment of exponents in tables \\
%  tabautofit & Boolean & Switch for rounding numbers to \\
%   & & length given by \opt{tabformat} \\
%  tabformat & Number & Space reserved in table for numbers \\
%  tabnumalign & List & Alignment of |S| column numbers \\
%  tabparseonly & Boolean & Do not align |S| columns \\
%   & & on decimal marker \\
%  tabtextalign & List & Positioning of text in |S| columns \\
%  tabunitalign & List & Positioning of units in |s| columns \\
%  textcelsius & Literal & ``$^{\circ}$C'' symbol in text mode \\
%  textdegree & Literal & ``${\circ}$'' symbol in text mode \\
%  textminute & Literal & ``$'$'' symbol in text mode \\
%  textmu & Literal & ``$\upmu$'' symbol in text mode \\
%  textomega & Literal & ``$\Omega$'' symbol in text mode \\
%  textringa & Literal & ``\AA'' symbol in maths mode \\
%  textrm & Csname & Roman text font \\
%  textsecond & Literal & ``$''$'' symbol in text mode \\
%  textsf & Csname & Sans serif text font \\
%  texttt & Csname & Fixed-width text font \\
%  tightpm & Boolean & Reduce space around $\pm$ \\
%  trapambigerr & Boolean & Check for ambiguous errors \\
%  trapambigfrac & Boolean & Check for ambiguous fractions \\
%  unitcolor & Literal & Switch for colouring units \\
%  unitcolour & Literal & Switch for colouring units \\
%  unitmathrm & Csname & Roman maths font for units \\
%  unitmathsf & Csname & Sans serif maths font for units \\
%  unitmathsrm & Csname & Roman maths font for units \\
%  unitmathssf & Csname & Sans serif maths font for units \\
%  unitmathstt & Csname & Fixed-width maths font for units \\
%  unitmathtt & Csname & Fixed-width maths font for units \\
%  unitmode & List &  As \opt{mode}, for units only \\
%  unitsep & List or literal & Separator for units \\
%  unitspace & List or literal & Space used for ``|~|'' in units \\
%  valuecolor & Literal & Switch for colouring value \\
%  valuecolour & Literal & Switch for colouring value \\
%  valuemathrm & Csname & Roman maths font for values \\
%  valuemathsf & Csname & Sans serif maths font for values \\
%  valuemathsrm & Csname & Roman maths font for values \\
%  valuemathssf & Csname & Sans serif maths font for values \\
%  valuemathstt & Csname & Fixed-width maths font for values \\
%  valuemathtt & Csname & Fixed-width maths font for values \\
%  valuemode & List &  As \opt{mode}, for values only \\
%  valuesep & List or literal & Separator between value and unit \\
%  xspace & Boolean & Use \pkg{xspace} after units \\
%\end{longtable}
%
%\section{Emulation of other packages}
% \currpkg has been designed as a replacement for \SIunits, \SIstyle,
% \unitsdef, \units, \pkg{hepunits}, \pkg{fancyunits} and
% \pkg{fancynum}. It therefore provides options reproduce the
% functions of all of these packages.  In this way, \currpkg should
% be usable as a straight replacement for the older
% packages.\footnote{User macros means that they are described in the
% package documentation; simply not containing an \texttt{@} does not
% mean they will have been emulated.}  This means for example that
% the \cs{num} macro takes an optional star when emulating \SIstyle.
% However, there are some points that should be remembered.  In
% particular, \currpkg validates numerical input, meaning that places
% where a number is expected in the older packages \emph{require} a
% number when emulated by \currpkg.
%
% The \numprint package has provided many useful ideas for the code
% used here for number formatting. The basic use of the \cs{numprint}
% (or \cs{np}) macro can be reproduced using \currpkg.  However,
% \numprint is large and complex, with its own backward-compatibility
% options.  As a result, emulation of \numprint is not provided here.
% To use a \numprint document with \currpkg, the \cs{numprint} macro
% could be provided using the following code.
%^^A Need to deal with \cs{numprint} being defined.
%\let\numprint\relax
%\begin{LaTeXexample}
%  \newcommand*{\numprint}[2][]{\SI[obeymode]{#2}{#1}}\\
%  \numprint{-123456} \\
%  \numprint[N/mm^2]{-123456}
%\end{LaTeXexample}
%^^A Back to the normal - with \cs{newcommand}
%\newcommand*{\numprint}{\pkg{numprint}\xspace}
%
% \currpkg can be used more-or-less directly to replace both
% \pkg{dcolumn} and \pkg{rccol}.  As is explained in the code
% section, much of the column-alignment system here is taken from
% \pkg{dcolumn}, while \pkg{rccol} provided a model for a
% customisable system.  However, neither package has been directly
% emulated here.  The |S| column type can be used to replace both |D|
% and |R| columns by setting the appropriate package options.
%
%\section{Configuration files}
% \currpkg is a modular package.  The unit definitions, abbreviations
% and locales are all stored in configuration files.  These all take
% names of the form \file{si-\meta{name}.cfg}, where \meta{name} is
% the part of the filename used as an option in \cs{sisetup} or when
% loading \currpkg.  Producing new configuration files therefore
% consists of making a suitably-named file and adding it to the path
% searched by \TeX.  The files should normally consist of settings
% (in \cs{sisetup}) and unit definitions, \etc.
%
%\DescribeMacro{\addtolocale}
% To allow arbitrary macros to be stored in locales, the
% \cs{addtolocale} macro is provided.  This ensures that arbitrary
% text is only executed when using a locale, not when loading it. The
% macro takes two arguments, \marg{locale} and \marg{code}.
%\begin{LaTeXexample}
%  \addtolocale{DE}{TEST}
%  Some text as filler \\
%  \sisetup{locale=DE}
%  This example is rather trivial!
%\end{LaTeXexample}
%
%\DescribeMacro{\requiresiconfigs}
% To load one or more configuration files from inside another
% configuration file, the \cs{requiresiconfigs} macro is provided.
% This accepts a comma-separated list of configuration names, in the
% same way as \opt{load} or \opt{noload}.
%
% In addition to the various configuration files provided with the
% package, a local file \file{siunitx.cfg} may be provided.  This is
% read at the end of loading \currpkg, and allows the user to include
% any local definitions or settings easily.
%
%\section{Common questions}
%\subsection{Why do I need \cs{per} more than once?}
% The unit engine of \currpkg is based around the English method for
% reading units out loud. Thus \si{\joule\per\mole\per\kelvin} is
% pronounced ``joules per mole per kelvin''.  Hence by default you
% need to put \cs{per} before each item in the denominator of a unit.
% The behaviour can be altered by setting the \opt{stickyper} option.
%
%\subsection{Why is the order of my units changed?}
% Then using \opt{per=reciprocal}, the units are typeset as given.
% However, when using \opt{per=slash} or \opt{per=fraction}, the
% package needs to find which ones are in the denominator.  It then
% prints the numerator and denominator separately.  So if you give a
% unit in the denominator \emph{before} one in the numerator, they
% have to be re-ordered.\footnote{``Double division''
% (\SI{1}{s/m/kg}) is mathematically incorrect.}
%\begin{LaTeXexample}
%  \SI{88}{\per\kilogram\metre} \\
%  \SI[per=slash]{66}{\per\kilogram\metre}
%\end{LaTeXexample}
%
%\subsection{Why are compound units not recommended outside of
%  \cs{SI}/\cs{si}?}
% To fully process units made up of several parts, the processor has
% to know where the end of the unit is.  When the unit macros are
% used outside of \cs{SI}/\cs{si}, this is not the case.  The package
% therefore does its best, but results may be sub-optimal.  To get
% consistent results, either define a new \emph{single} unit, or keep
% compound units inside \cs{SI} and \cs{si}.
%\begin{LaTeXexample}
%  \metre\per\micro\second \\
%  \si{\metre\per\micro\second} \\
%  \newunit{\myunit}{\metre\per\micro\second}
%  \myunit
%  \newunit{\myunittwo}{\kilogram\myunit} \\
%  \myunittwo \\
%  \kilogram\myunit
%\end{LaTeXexample}
% Notice the difference in behaviour of \cs{per} in the first two
% lines, and the spacing error on the last line in the example.
%
%\subsection{How do I set superscripts to use lining numbers?}
% Lowercase (``old style'') numbers are favoured by many people for
% use in running text.  However, this does not necessarily look good
% in superscripts. The mode used to typeset data can be varied, so
% that maths mode numbers are used for the unit part of the output.
%\begin{LaTeXexample}
%  \SI[mode=text]{1234}{\metre\per\second}\\
%  \SI[valuemode=text,unitmode=maths]{1234}{\metre\per\second}
%\end{LaTeXexample}
%
%\subsection{Why do most of the examples use
%  \texorpdfstring{\si{\joule\per\mole\per\kelvin}}{J mol-1 K-1}?}
% The package author is a chemist, and this is the unit of entropy
% (disorder).  It nicely demonstrates the use of the \cs{per} macro,
% and so it crops up a lot.  It is also in the subsection heading
% here to act as a test with \pkg{hyperref} and moving arguments!
%
%\subsection{What can \numprint do that \currpkg cannot?}
% \currpkg uses a lot of ideas from \numprint: a reasonable amount of
% the number-processing code here is based on that in \numprint.
% However, the two packages have somewhat different aims, and as a
% result there are things that \numprint can do that \currpkg does
% not implement.  The main features of \numprint not available here
% are:
%\begin{itemize}
%  \item General support for numbers with base other than $10$ (see
%    \pkg{nbaseprt});
%  \item Alignment of the decimal marker of powers in tables;
%  \item Alignment of numbers in running text;
%  \item Specific formatting commands for \TeX\ counters and lengths.
%\end{itemize}
%
%\section{Tricks and known issues}
%
%\subsection{Ensuring maths mode}
% Due to the possibility of output in either maths or text mode, any
% input which requires a particular mode needs to be protected.  You
% cannot use |$|\ldots|$|, as this can get ``caught out'', but also
% as it may give hard-to-follow errors.  Always use \cs{ensuremath}
% to force maths processing, and \cs{text} (from the \AMS\TeX\
% bundle) to ensure text mode.
%
%\subsection{Using \texttt{.} and fixed spaces in units}
% To use a literal |.| in a unit, it has to be within an extra set of
% braces.  This does not need any extra protection, unlike the
% situation with \SIstyle (for example, no \cs{text} macro is
% needed).  The fixed space (|~|) is more problematic: set
% \opt{unitspace=space} to get a full space here.
%\begin{LaTeXexample}
%  \newunit[unitspace=space]{\myunit}{V~vs{.}~NHE}%
%  \SI{10}{\myunit}
%\end{LaTeXexample}
%
%\subsection{Passing unprocessed digits through an \texttt{S}
%  column}
% The method used to detect numbers in an |S| column will pick up
% material wrapped inside braces if there is more than a single
% character inside the braces.  If you want to pass a \emph{single}
% numerical character without processing it, you need two sets of
% braces (\ref{tab:S-limits}).
%\begin{LaTeXexample}[float]
%  \begin{table}
%    \centering
%    \caption{Passing single digit characters}
%    \label{tab:S-limits}
%    \begin{tabular}{S[colourall,colour=orange]}
%      \toprule
%        {Heading} \\
%      \midrule
%        {1-2} \\
%        {1}   \\
%        {{1}} \\
%        {{-}} \\ % Using {-} gives an error!
%      \bottomrule
%    \end{tabular}
%  \end{table}
%\end{LaTeXexample}
%
%\subsection{Limitations of \cs{mathrm}}
% The package uses the \cs{mathrm} font family by default to typeset
% output in maths mode.  This however has a few side-effects.  For
% example, the Greek alphabet can give odd results.\footnote{This
% depends on your font setup; this document uses T1 encoding, which
% shows the issue, whereas using OT1 does not.}  The use of the
% \cs{mathnormal} font \emph{may} get around this issue.
%\begin{LaTeXexample}
%  \num[numaddn=\pi]{4\pi e-7}\\
%  \num[numaddn=\pi,mathsrm=mathnormal]{4\pi e-7}
%\end{LaTeXexample}
% On the other hand, you may want to use text mode, in which case
% \cs{ensuremath} is needed.  Depending on the exact circumstances,
% the \LaTeX\ protection mechanism (\cs{DeclareRobustCommand}) may be
% sufficient; in some cases, this will fail and the \eTeX\
% \cs{protected} system may be required. There are several potential
% pitfalls in this area; experimentation may well be needed.
%\begin{LaTeXexample}
%  \DeclareRobustCommand*{\numpi}{\ensuremath{\pi}}
%  \num[numaddn=\numpi,mode=text]{4\numpi e-7}
%\end{LaTeXexample}
%
%\subsection{Entire document in sans serif font}
% If your entire document is not in a Roman font, using
% the font detection system is not the most efficient method
% for setting the \currpkg output.  Instead, the \opt{mathrm}
% and \opt{textrm} package options can be redefined.
%\begin{LaTeXexample}
%  \sffamily
%  Some text \\
%  \sisetup{obeyfamily=false,mathrm=mathsf,textrm=sffamily}
%  \num{1e2} \\
%  \SI{3}{\newton}
%  \[ \SI{4e5}{\pascal} \]
%\end{LaTeXexample}
%
%\subsection{Effects of emulation}
% The package has been designed so that almost everything can be set
% using the options.  In the emulation code, some internal macros are
% redefined.  This is because the legacy packages do odd things,
% which are deliberately not implemented by \currpkg.  Using an
% emulation file will prevent subsequent loading of the real package.
% This is to prevent errors or, worse, difficult to diagnose changes
% to output.
%
%\subsection{Centring columns on non-decimal input}
% The \pkg{dcolumn} manual suggests using that package to align a
% column on a $\pm$ sign. The same type of output is possible using
% \currpkg, but some care is needed (\ref{tab:dcolumn}).  Odd things
% may happen: use with care!
%\begin{LaTeXexample}[float]
%  \begin{table}
%    \centering
%    \caption{Non-standard \texttt{S} column}
%    \label{tab:dcolumn}
%    \begin{tabular}{%
%      S[digitsep=none,decimalsymbol={\,\pm\,},
%        numdigits={0123456789.},numdecimal=+]}
%    \toprule
%    {Some Values} \\
%    \midrule
%    2.3456  + 0.02  \\
%    34.2345 + 0.001 \\
%    56.7835 + 0.067 \\
%    90.473  + 0.021 \\
%    \bottomrule
%    \end{tabular}
%  \end{table}
%\end{LaTeXexample}
%
%\subsection{Expanding content in tables}
% When processing |S| columns, \currpkg works hard to expand any
% items which may give numbers.  So for example you can define values
% as macros (\ref{tab:vmacros}). To avoid the expansion of single
% macros, you must either wrap them in two sets of braces or make
% them robust (using either \cs{DeclareRobustCommand} or
% \cs{protected}).
%\begin{LaTeXexample}[float]
%  \begin{table}
%    \centering
%    \caption{Values as macros}
%    \label{tab:vmacros}
%    \newcommand*{\myvala}{1.234}%
%    \newcommand*{\myvalb}{20.345}%
%    \newcommand*{\myvalc}{0.987654}%
%    \DeclareRobustCommand*{\myvald}{88.88}%
%    \begin{tabular}{S[tabtextalign=left]}
%    \toprule
%    {Some Values} \\
%    \midrule
%    \myvala  \\
%    \myvalb \\
%    \myvalc \\
%    {{\myvala}} \\
%    \myvald \\
%    \bottomrule
%    \end{tabular}
%  \end{table}
%\end{LaTeXexample}
%
% It is possible to use calculated values in tables. For this to
% work, the calculation must take place before attempting to parse
% the number. This means that any calculation code should be wrapped
% in braces and appear before the use of the number. This is
% illustrated by using the \pkg{datatool} package to set up a
% miniature database, and then output the values doubled
% (\ref{tab:calc}). This example also shows that macros such as
% \cs{DLTforeach}, which construct a table while \TeX\ is running,
% cannot be placed inside an |S| column (or an |s| one). Instead, an
% additional dummy column has been added with no inter-column space.
% This is used to contain the tale-building macro.
%\begin{LaTeXexample}[float]
%  \DTLnewdb{data}
%  \DTLnewrow{data}\DTLnewdbentry{data}{value}{66.7012}
%  \DTLnewrow{data}\DTLnewdbentry{data}{value}{66.0212}
%  \DTLnewrow{data}\DTLnewdbentry{data}{value}{64.9026}
%  \begin{table}
%    \caption{Calculated values}
%    \label{tab:calc}
%    \centering
%    \sisetup{tabformat=2.4}
%    \begin{tabular}{SS@{}l}
%      \toprule
%      {Value} & {Doubled}  &
%      \DTLforeach{data}{\myvalue=value}{%
%        \DTLiffirstrow {\\ \midrule}{\\}%
%        \myvalue & % First column
%        {\DTLmul{\myvalue}{\myvalue}{2}} \myvalue % second column
%        & }\\
%      \bottomrule
%    \end{tabular}
%  \end{table}
%\end{LaTeXexample}
%
%\subsection{Adding items after the last column of a tabular}
% If you use an |S| or |s| column as the last one in a tabular, and
% you use the \pkg{array} ``|<|'' construction to add items after it,
% the spacing may be wrong.  This will occur if the column contents
% are of differing widths.  Changing the \LaTeX\ |\\| line ending for
% the plain \TeX\ \cs{cr} will give the correct spacing, but does not
% allow adjustment of inter-row distance (\ref{tab:cr}).\footnote{For
% the \TeX\ experts, the issue here is that the system to gather up
% cell contents is added in using the \texttt{<} construction.
% Normally, this comes after the cell contents and any other
% \texttt{<} arguments, so collects the user additions.  However, in
% the last cell the contents include \texttt{\bslash\bslash}, which
% is converted to \cs{cr} before gathering can occur.  By using
% \cs{cr} directly, the gathering process receives all of the cell
% contents as normal.}  In most cases, this should not be a serious
% issue. Notice that an extra set of braces is needed here, compared
% to usual \LaTeX\ use; this is to prevent any expansion of the
% material by \currpkg.
%\begin{LaTeXexample}[float]
%  \begin{table}
%    \caption{Correcting spacing in last \texttt{S} column}
%    \label{tab:cr}
%    \hfil
%    \begin{tabular}{S<{{\,\si{\kg}}}S<{{\,\si{\kg}}}}
%      \toprule
%      \multicolumn{1}{c}{Long header} &
%        \multicolumn{1}{c}{Long header} \\
%      \midrule
%      1.23 & 1.23 \\
%      4.56 & 4.56 \\
%      7.8  & 7.8  \\
%      \bottomrule
%    \end{tabular}
%    \hfil
%    \begin{tabular}{S<{{\,\si{\kg}}}S<{{\,\si{\kg}}}}
%      \toprule
%      \multicolumn{1}{c}{Long header} &
%        \multicolumn{1}{c}{Long header} \\
%      \midrule
%      1.23 & 1.23 \cr
%      4.56 & 4.56 \cr
%      7.8  & 7.8  \cr
%      \bottomrule
%    \end{tabular}
%    \hfil
%  \end{table}
%\end{LaTeXexample}
%
%\subsection{Using \currpkg\ with the \pkg{cellspace} package}
% Both \currpkg\ and \pkg{cellspace} use the letter \texttt{S} for a 
% new column type. This obviously leads to a problem. If both are 
% loaded, \currpkg\ will retain the \texttt{S} column, and moves the
% functionality of \pkg{cellspace} to the letter \texttt{C}. This 
% allows the normal use of \pkg{cellspace} with standard column types:
% it does \emph{not} work with the \currpkg\ \texttt{S} or \texttt{s}
% columns.
% 
%\subsection{Using \currpkg\ with the \pkg{mathabx} package}
%\changes{v1.1g}{2008/11/26}{Respect loading of \pkg{mathabx}}
% The \pkg{mathabx} package defines its own \cs{second} and 
% \cs{degree} macros.  This is respected by \currpkg: if you want the
% \currpkg\ definitions instead, put the lines
%\begin{LaTeXexample}[noexample]
%  \makeatletter
%  \renewunit{\second}{s}
%  \renewunit[valuesep=none]{\degree}{\si@sym@degree}
%  \makeatother
%\end{LaTeXexample}
% after |\begin{document}|.
%
%\iffalse
%<*bug>
%\fi
%\section{Reporting a problem}
% \currpkg is quite long and complicated, and works hard to cover all
% possible eventualities.  However, there will be bugs in the code
% and unexpected interactions with other packages.  If you think you
% have found a bug, please report it.  A short test-case
% demonstrating the problem would be very welcome.  The following is
% a suitable template, and is available as \file{si-bug.ltx} in the
% \texttt{doc/latex/siunitx} directory or by running the \ext{dtx} or
% \ext{ins} file through \TeX.
%    \begin{macrocode}
\listfiles
\documentclass{article}
%    \end{macrocode}
% Load other packages needed here.
%    \begin{macrocode}
\usepackage{siunitx}
%    \end{macrocode}
% Normally, debugging the load procedure will not be needed; the
% debug option here means that all run-time information is logged.
%    \begin{macrocode}
\sisetup{debug}
\begin{document}
This is the bug test-case document for the \textsf{siunitx}
package.\\
Please put your demonstration here, and e-mail to the package
author.
\begin{center}
  \texttt{joseph.wright@morningtar2.co.uk}
\end{center}
\end{document}
%    \end{macrocode}
%\iffalse
%</bug>
%\fi
%
%\section{Feature requests}
% Feature requests for \currpkg are welcome.  The package maintainer
% will consider any ideas within the remit of the package (units and
% values).  If suggesting a new feature, an example of how it should
% work would be appreciated.  If new controls are needed, some
% suggestions for option names would be welcome.
%
%\section{Acknowledgements}
% Many thanks indeed to Stefan Pinnow, who has made a very large
% number of suggestions and found numerous bugs in the package.  His
% contribution to the package has been vital. The package author has
% learned \LaTeX\ tricks from far too many people to thank all of
% them.  However, for this package specific thanks must go to the
% authors of the existing ``unit'' packages: Danie Els (\SIstyle),
% Marcel Heldoorn (\SIunits), Patrick Happel (\unitsdef), Axel
% Reichert (\units) and Harald Harders (\numprint).  Will Robertson
% and Heiko Oberdiek deserve much credit for demonstrating \LaTeX\
% coding best practice. Victor Eijkhout's excellent (and free)
% \emph{\TeX\ by Topic} has provided some useful coding hints
% \cite{Eijkhout}.  The idea for combining and extending unit
% provision in \LaTeX\ was heavily inspired by Philip Lehmann's
% \pkg{biblatex}. Thanks to the various contributors of ideas for the
% package: Donald Arseneau, Michele Dondi, Paul Gans, Ben Morrow, Lan
% Thuy Pham, Alan Ristow, Patrick Heinze, Andrea Blomenhofer, Morten
% H{\o}gholm, Burkhard Moddemann  and Patrick Steegstra.
%
%\clearpage
%\part{Correct application of (SI) units}
%\section{Background}
% Consistent and logical units are a necessity for scientific work,
% and have applicability everywhere.  Historically, a number of
% systems have been used for physical units.  SI units were
% introduced by the \latin{Conf\'erence G\'en\'erale des Poids et
% Mesures} (CGPM) in 1960.  SI units are a coherent system based on
% seven base units, from which all other units may be derived.
%
% At the same time, physical quantities with units are mathematical
% entities, and as such way that they are typeset is important.  In
% mathematics, changes of type (such as using bold, italic, sans
% serif typeface and so on) convey information.  This means that
% rules exist not only for the type of units to be used under the SI
% system, but also the way they should appear in print.  Advice on
% best practice has been made available by the \emph{National
% Institute of Standards and Technology} (NIST) \cite{NIST}.
%
% As befits an agreed international standard, the full rules are
% detailed. It is not appropriate to reproduce these in totality
% here; instead, a useful summary of the key points is provided. The
% full details are available from the \latin{Bureau International des
% Poids et Mesures} (BIPM) in French \cite{BIPM:FR} and English
% \cite{BIPM:EN}.  They also publish a very useful and detailed guide
% to using units, values and so on, available online in a number of
% different formats \cite{BIPM:PDF}.
%
% \currpkg takes account of the information given here, so far as is
% possible.  Thus the package defaults follow the recommendations
% made for typesetting units and values.  Spacing and so forth is
% handled in such a way as to make implementing the rules
% (relatively) easy.
%
%\section{Units}
%\subsection{SI base units}
% There are seven base SI units, listed in \ref{tab:baseunits}. Apart
% from the kilogram, these are defined in terms of a measurable
% physical quantity needing the definition alone.\footnote{Some base
% units need others defined first; there is therefore a required
% order of definition.}  The base units have been chosen such that
% all physical quantities can be expressed using an appropriative
% combination of these units, needing no others and with no
% redundancy.  The kilogram is slightly different from the other base
% units as it is still defined in terms of a ``prototype'' held in
% Paris.\footnote{At the time of writing, this is under review and
% will be altered.}
%
%\subsection{SI derived units}
% All other units within the SI system are regarded as ``derived''
% from the seven base units.  At the most basic, all other SI units
% can be expressed as combinations of the base units.  However, many
% units (listed in \ref{tab:derived}, \ref{tab:experimental} and
% \ref{tab:accepted}) have a special name and symbol.\footnote{The
% nautical mile has a given name but no agreed symbol, and although
% accepted by the SI is not provided by \currpkg as a unit macro.}
% Most of these units are simple combinations of one or more base
% units (raised to powers as appropriate).  A small number of units
% derived from experimental data are allowed as SI units
% (\ref{tab:experimental}).
%
% Some of these units (in \ref{tab:accepted}) are regarded as only
% ``temporarily'' accepted, as the use of only the base and
% fully-consistent derived units in \ref{tab:derived} is encouraged.
% They are accepted as they are in common use in one or more
% disciplines; some are still very widespread in the appropriate
% areas.  These units are mainly multiples of base units (for
% example, a tonne is \SI{1000}{\kilo\gram}).
%
% One point to note is that ``unitless ratios'' are regarded as
% having base units which cancel out.  For example, the radian is
% regarded as having base unit \si{\metre\per\metre}.  The result of
% this division (``$1$'') is therefore regarded as a derived SI unit
% in this context.
%
%\subsection{SI prefixes}
% A series of SI prefixes for decimal multiples and submultiples are
% provided, and can be used as modifiers for any SI unit (either base
% or derived units) with the exception of the kilogram.  The prefixes
% are listed in \ref{tab:prefix}.  No space should be used between a
% prefix and the unit, and only a single prefix should be used.  Even
% the degree Celsius can be given a prefix, for example
% \SI{1}{\milli\celsius}.  The only exception to this rule is for
% degrees, minutes and seconds of an arc: \ang{1;2;3}.
%
% It is important to note that the kilogram is the only SI unit with
% a prefix as part of its name and symbol.  Only  single prefix may
% be used, and so in the case of the kilogram prefix names are used
% with the unit name ``gram'' and the prefix symbols are used with
% the unit symbol \si{\gram}.  For example $\SI{1e-6}{\kilogram} =
% \SI{1}{\milli\gram}$.
% 
% Three unit-prefix combinations deserve special note. The units
% hectare, kilohm and megohm elide the additional vowel (these
% would otherwise be hectoare, kiloohm and megaohm).  All other
% vowel combinations are retained.\footnote{Thanks to Charles
% Cameron for raising this issue with the package author.}
%
%\subsection{Other units}
% The application of SI units is meant to provide a single set of
% units which ensure consistency and clarity across all areas.
% However, other units are common is many areas, and are not without
% merit. The units provided by \currpkg by default do not included
% any of these; only units which are part of the SI set or are
% accepted for use with SI units are defined.  However, several other
% sets of units can be loaded as optional modules. The binary
% prefixes and units (Section~\ref{sec:binary} and \ref{tab:binary})
% are the most obvious example.  These are \emph{not} part of the SI
% specifications, but the prefix names are derived from those in
% \ref{tab:prefix}.
%
% Other units (such as those provided by the modules \opt{synchem},
% \opt{hep} and \opt{astro}) are normally to be avoided where
% possible.  SI units should, in the main, be preferred due to the
% advantages of clear definition and self-consistency this brings.
% However, there will probably always be a place for specialist or
% non-standard units.  This is particulary true of units derived from
% basic physical constants; for example reason, the \opt{hep} module
% defines the speed of light, \si{\clight}, as a unit.  For work in
% basic science, a small number of physical constants are recognised
% as units provided the results for comparison with experiment are
% given in SI units.
%
% There are also many areas where non-standard units are used so
% commonly that to do otherwise is difficult or impossible.  For
% example, most synthetic chemists measure the pressure inside vacuum
% apparatus in \si{\mmHg}, partly because the most common gauge for
% the task still uses a column of mercury metal.  For these reasons,
% \currpkg does define non-SI units.
%
%\section{Units and values in print}
%\subsection{Mathematical meaning}
% As explained earlier, a unit--value combination is a single
% mathematical entity.  This has implications for how both the number
% and the unit should be printed.  Firstly, the two parts should not
% be separated.  With the exception of the symbols for plane angles
% (\si{\degree}, \si{\arcmin} and \si{\arcsec}), it is usual to have
% a space between the unit and the value.  This should therefore be a
% non-breaking space between the two.  Different geographical areas
% have different conventions on the size of this space; a ``small''
% space (|\,|) is the \currpkg default.
%\begin{LaTeXexample}
%  A space for \SI{10}{\percent}\\
%  and also for \SI{100}{\celsius}\\
%  but not for \ang{1.23}.
%\end{LaTeXexample}
%
% The mathematical meaning of units also means that the shape, weight
% and family are important.  Units are supposed to be typeset in an
% upright, medium weight serif font.  Italic, bold and sans serif are
% all used mathematically to convey other meanings.  \currpkg package
% defaults again follow this convention: any local settings are
% ignored, and uses the current upright serif maths font.  However,
% there are occasions where this may not be the most desirable
% behaviour.  A classic example would be in an all-bold section
% heading.  As the surrounding text is bold, some people feel that
% any units should follow this.
%\begin{LaTeXexample}
%  Units should \textbf{not be bold: \SI{54}{\farad}}\\
%  \textbf{But perhaps in a running block,\\
%  it might look better: \SI[obeybold]{54}{\farad}}
%\end{LaTeXexample}
%
%\subsection{Unit multiplication and division}
% Symbols for units formed from other units by multiplication are
% indicated by means of either a half-height (that is, centred) dot
% or a (thin) space.  This document uses a half-height dot as
% (\emph{i}) this is the recommendation of \textsc{nist}, amongst
% others and (\emph{ii}) it avoids potential confusion between unit
% prefixes and multiplied units.
%\begin{LaTeXexample}
%  $\si{\metre\second} = \mbox{metre second}$ \\
%  $\si{\milli\second} = \mbox{millisecond}$ \\
%  \sisetup{unitsep=thin}
%  $\si{\metre\second} = \mbox{metre second}$ \\
%  $\si{\milli\second} = \mbox{millisecond}$
%\end{LaTeXexample}
% There are some circumstances under which it is permissable to omit
% any spaces.  The classic example is \si{\kWh}, where
% ``\si[unitsep=thin]{\kWh}'' does not add any useful information.
% If using such a unit repeatedly, users of \currpkg are advised to
% create a custom unit to ensure consistency.\footnote{\cs{kWh} and
% \cs{kilowatthour} are defined by \currpkg in this way.}
%
% Symbols for units formed from other units by division are indicated
% by means of a virgule (oblique stroke, slash, |/|), a horizontal
% line, or negative exponents.\footnote{Notice that a virgule and a
% solidus are not the same symbol.}  However, to avoid ambiguity, the
% virgule must not be repeated on the same line unless parentheses
% are used. This is ensured when using named unit macros in \currpkg,
% which will ``trap'' repeated division and format it correctly.  In
% complicated cases, negative exponents are to be preferred over
% other formats.
%\begin{LaTeXexample}
%  \si{\joule\per\mole\per\kelvin}\\
%  \si[per=fraction]{\joule\per\mole\per\kelvin}\\
%  \si[per=slash]{\joule\per\mole\per\kelvin}
%\end{LaTeXexample}
%
%\subsection{Repeating units}
% Products and errors should show what unit applies to each value
% given.  Thus \SI{2x3}{\metre} is an ordered set of lengths of a
% geometric area, whereas \SI[repeatunits=false]{2x3}{\metre} is a
% length (and equal to \SI{6}{\metre}). Thus, $\times$ is not a
% product but is a mathematical operator; in the same way, a $2
% \times 3$ matrix is not a $6$ matrix!  In some areas, areas and
% volumes are given with separated units but a unit raised to the
% appropriate power: \SI[repeatunits=power]{2 x 3}{\metre}. Although
% this does display the correct overall units, it is
% potentially-confusing and is not encouraged.
%
%\subsection{Clarity in writing values of quantities}
% Care must be taken when writing ranges of numbers.  For purely
% numerical values, it is common to use a en-dash between values, for
% example ``see pages 1--5''.  On the other hand, values with units
% could be misinterpret as negative values if written in this way. As
% the unit--value combination is a single mathematic entity, writing
% the values with an en-dash followed by a single unit is also
% incorrect.  As a result, using the word ``to'' is strongly
% recommended.
%\begin{LaTeXexample}
%  \SI{1}{\metre} to \SI{5}{\metre} long.
%\end{LaTeXexample}
%
%\subsection{Graphs and tables}
% In graphs and tables, repetition of the units following each entry
% or axis mark is confusing and repetitive.  It is therefore best to
% place the unit in the label part of the information.  Placing the
% unit in square brackets is common but mathematically
% poor.\footnote{For example, for an acceleration $a$, the expression
% $[a]$ is the dimensions of $a$, \ie length per time squared in this
% case.}  Much better is to show division of all values by the unit,
% which leaves the entries as unitless ratios.  This is illustrated
% in \ref{tab:label} and \ref{fig:label}.
%\begin{LaTeXexample}[float]
%  \begin{table}
%    \centering
%    \caption{An example of table labelling}
%    \label{tab:label}
%    \begin{tabular}{cS[tabformat=1.4,tabnumalign=centre]}
%      \toprule
%      {Entry} & {Length/\si{\metre}} \\
%      \midrule
%      1 & 1.1234 \\
%      2 & 1.1425 \\
%      3 & 1.7578 \\
%      4 & 1.9560 \\
%      \bottomrule
%    \end{tabular}
%  \end{table}
%\end{LaTeXexample}
%\begin{LaTeXexample}[float]
%  \begin{figure}
%    \centering
%    \begin{tikzpicture}
%      \begin{axis}[xlabel=$t/\si{\second}$,ylabel=$d/\si{\metre}$]
%        \addplot[smooth,mark=x]
%          plot coordinates {
%            (0,0)
%            (1,5)
%            (2,8)
%            (3,9)
%            (4,8)
%            (5,5)
%            (6,0)
%           };
%      \end{axis}
%    \end{tikzpicture}
%    \caption{An example of graph labelling}
%    \label{fig:label}
%  \end{figure}
%\end{LaTeXexample}
%
% In most cases, adding exponent values in the body of a table is
% less desirable than adding a fixed exponent to column headers.  An
% example is shown in \ref{tab:exp}.  The use of \cs{multicolumn} is
% needed here due to the ``|<|''; without \cs{multicolumn}, the
% titles are followed by ``\si{\kilo\gram}''!
%\begin{LaTeXexample}[float]
%  \begin{table}
%    \centering
%    \caption{Good and bad columns}
%    \label{tab:exp}
%    \sisetup{tabnumalign=centre}
%    \begin{tabular}{%
%      c%
%      S[tabformat=1.3e1]<{\,\si{\kilogram}}%
%      S[tabformat=2.2]}
%      \toprule
%      Entry & \multicolumn{1}{c}{Mass} &
%        {Mass/\SI{e3}{\kilogram}} \\
%      \midrule
%      1 & 4.56e3  & 4.56 \\
%      2 & 2.40e3  & 2.40 \\
%      3 & 1.345e4 & 13.45 \\
%      4 & 4.5e2   & 0.45 \\
%      \bottomrule
%    \end{tabular}
%  \end{table}
%\end{LaTeXexample}
%
%\bibliographystyle{unsrt}
%\StopEventually{%
%  \clearpage
%  \part{Notes}
%  \PrintChanges
%  \PrintIndex
%  \bibliography{siunitx}}
%
% \iffalse
%<*package>
% \fi
%
%\clearpage
%\part{Implementation}
%\section{Main package}
% Much of the code here is taken, with little or no modification,
% from the existing packages.  These are all released under the LPPL,
% and so this use is entirely allowed.  Rather than confuse the
% source here with repeated references, note that code here could be
% copied from \SIstyle, \SIunits, \numprint, \unitsdef or \units.
% Some ideas have also been borrowed from \pkg{biblatex}; again these
% will not be specifically noted. Code from other packages will be
% marked when used.
%
% User-space commands (those not containing |@|) defined here should
% give the same result as macros with the same name in the older
% packages.\footnote{Although extra optional arguments may be added.}
% However, internal package macros will behave differently; if the
% user has redefined internal macros, then compatibility will be
% impaired.
%
% The code used here uses \LaTeX\ rather than \TeX\ commands where
% possible.\footnote{This applies to \LaTeX\ kernel commands only;
% for example, \pkg{ifthenelse} is not used.}  For example,
% \cs{newcommand*} is used in place of \cs{def}, unless custom
% parameters are needed.  Hopefully, this will aid future
% maintenance.  Grouping is used where possible to limit the scope of
% temporary assignments.
%
%\subsection{Setup code}
% As always, the package starts with identification.
%    \begin{macrocode}
\NeedsTeXFormat{LaTeX2e}
\ProvidesPackage{siunitx}
  [2008/12/22 v1.1j A comprehensive (SI) units package]
%    \end{macrocode}
% The package requires \eTeX, so the usual test is made.
%    \begin{macrocode}
\begingroup
  \@ifundefined{eTeXversion}
    {\PackageError{siunitx}
       {Not running under e-TeX}
       {This package requires e-TeX. Try compiling the document
        with\MessageBreak `elatex' instead of `latex'. When
        using pdfTeX, try `pdfelatex'\MessageBreak instead of
        `pdflatex'}%
     \endgroup\endinput}
    {\endgroup}
%    \end{macrocode}
%\begin{macro}{\si@catcodes}
% There are circumstances under which some odd category codes might
% be in place.  The category codes for |{|, |}|, |[|, |]| and |#| are
% assumed to be okay; if issues arise, this can be altered.  \LaTeX\
% will have set |@| as a letter on loading \currpkg.
%    \begin{macrocode}
\edef\si@catcodes{%
  \catcode\string`\string ` \the\catcode\string`\`\relax
  \catcode\string`\string = \the\catcode\string`\=\relax
  \catcode\string`\string ^ \the\catcode\string`\^\relax
  \catcode\string`\string _ \the\catcode\string`\_\relax
  \catcode\string`\string ~ \the\catcode\string`\~\relax
  \catcode\string`\string : \the\catcode\string`\:\relax
  \catcode\string`\string - \the\catcode\string`\-\relax
  \catcode\string`\string + \the\catcode\string`\+\relax
  \catcode\string`\string ; \the\catcode\string`\;\relax
  \catcode\string`\string , \the\catcode\string`\,\relax
  \catcode\string`\string . \the\catcode\string`\.\relax}
\catcode\string`\` 12\relax
\catcode`\= 12\relax
\catcode`\^ 7\relax
\catcode`\_ 8\relax
\catcode`\~ \active\relax
\@makeother{\:}
\@makeother{\-}
\@makeother{\+}
\@makeother{\;}
\@makeother{\,}
\@makeother{\.}
%    \end{macrocode}
%\end{macro}
% Packages needed for functionality are loaded.  \pkg{xkeyval}
% handles the package options, while \pkg{amstext} from the \AMS\
% bundle is needed for \cs{text}.  \pkg{array} is needed for the new
% column type for tabular material.  \pkg{xspace} provides ``magic''
% spacing after macros, if requested.
%    \begin{macrocode}
\RequirePackage{xkeyval}[2005/05/07]
\RequirePackage{amstext,array,xspace}
%    \end{macrocode}
%\begin{macro}{\si@tempa}
%\begin{macro}{\si@tempb}
%\begin{macro}{\si@tempc}
% Some scratch commands are defined; apart from where a known value
% is carried through, these could contain anything.
%    \begin{macrocode}
\newcommand*{\si@tempa}{}
\newcommand*{\si@tempb}{}
\newcommand*{\si@tempc}{}
%    \end{macrocode}
%\end{macro}
%\end{macro}
%\end{macro}
%\begin{macro}{\ifsi@switch}
% Various items will need a switch.  To avoid name pollution, a
% single switch is defined here; grouping will keep the definition
% local.
%    \begin{macrocode}
\newif\ifsi@switch
%    \end{macrocode}
%\end{macro}
%\begin{macro}{\si@tempboxa}
%\begin{macro}{\si@tempboxb}
%\begin{macro}{\si@tempboxc}
%\begin{macro}{\si@tempboxd}
% Some boxes are also needed.
%    \begin{macrocode}
\newbox\si@tempboxa
\newbox\si@tempboxb
\newbox\si@tempboxc
\newbox\si@tempboxd
%    \end{macrocode}
%\end{macro}
%\end{macro}
%\end{macro}
%\end{macro}
%\begin{macro}{\si@temptoks}
% A token register is also handy.
%    \begin{macrocode}
\newtoks{\si@temptoks}
%    \end{macrocode}
%\end{macro}
%\begin{macro}{\si@packagecheck}
%\begin{macro}{\si@blockpkgs}
%\begin{macro}{\si@checkpkgs}
% As \currpkg is intended to replace the other unit-management
% packages, these are tested for before any further processing.  If
% any are loaded, the package halts compilation; name clashes or
% unexpected results could occur if this is not tested.  Notice that
% \SIunits and \SIstyle could be loaded with variable capitalisation
% (at least on Windows); both possibilities are tested.
%    \begin{macrocode}
\newcommand*{\si@blockpkgs}{SIunits,sistyle,siunits,SIstyle,%
  unitsdef,fancyunits}
\newcommand*{\si@checkpkgs}{units,hepunits,fancynum}
\newcommand*{\si@packagecheck}{%
  \begingroup
  \@for\si@tempa:=\si@blockpkgs\do{
    \@ifpackageloaded{\si@tempa}
      {\PackageError{siunitx}
         {Package `\si@tempa' incompatible}
         {The \si@tempa\space package and siunitx are
          incompatible.\MessageBreak Use the
          `emulate=\si@tempa' package option when loading
          siunitx}}
      {}}
%    \end{macrocode}
% Some packages should not cause a clash, but are emulated and would
% be better handled that way.
%    \begin{macrocode}
  \@for\si@tempa:=\si@checkpkgs\do{%
    \@ifpackageloaded{\si@tempa}
      {\PackageWarning{siunitx}
         {Consider loading the siunitx package
          with\MessageBreak option `emulate=\si@tempa', rather
          than\MessageBreak loading both \si@tempa\space and
          siunitx}}
      {}}
  \endgroup}
%    \end{macrocode}
% The check is carried out on loading and at the beginning of the
% document, so that packages loaded both before and after \currpkg
% are caught.
%    \begin{macrocode}
\si@packagecheck
\AtBeginDocument{\si@packagecheck}
%    \end{macrocode}
%\end{macro}
%\end{macro}
%\end{macro}
%\begin{macro}{\si@ifdefinable}
%\darg{macro}
% Using \cs{@ifdefinable} to check macro definitions gives a generic
% error.  To give something more helpful, \cs{@ifundefined} is used,
% but this needs some \cs{expandafter} work.  This way it can also be
% used as a form of \cs{@ifundefined} for macro names.
%    \begin{macrocode}
\newcommand*{\si@ifdefinable}[1]{%
  \expandafter\expandafter\expandafter\@ifundefined%
    \expandafter\expandafter\expandafter%
      {\expandafter\@gobble\string#1}}
%    \end{macrocode}
%\end{macro}
%\begin{macro}{\si@addtolist}
%\darg{macro}
%\darg{items}
% It is quite useful to be able to add to a comma-separated list of
% expandable items.
%    \begin{macrocode}
\newcommand*{\si@addtolist}[2]{%
  \ifx\@empty#1\@empty
    \edef#1{#2}%
  \else
    \edef#1{#1,#2}%
  \fi}
%    \end{macrocode}
%\end{macro}
%\begin{macro}{\si@addtocsname}
%\darg{csname}
%\darg{tokens}
% A second item to add to a command sequence.
%    \begin{macrocode}
\newcommand*{\si@addtocsname}[2]{%
  \@ifundefined{#1}
    {\expandafter\gdef\csname #1\endcsname{#2}}
    {\si@temptoks\expandafter\expandafter\expandafter{%
       \csname #1\endcsname#2}%
     \expandafter\xdef\csname #1\endcsname{\the\si@temptoks}}}
%    \end{macrocode}
%\end{macro}
%\begin{macro}{\si@ifmtarg}
%\begin{macro}{\si@xifmtarg}
%\begin{macro}{\si@ifnotmtarg}
% To keep down dependance on other packages, the very short code
% block from \pkg{ifmtarg} is copied here with an internal name.
%    \begin{macrocode}
\begingroup
  \catcode`\Q=3
  \long\gdef\si@ifmtarg#1{%
    \si@xifmtarg#1QQ\@secondoftwo\@firstoftwo\@nil}
  \long\gdef\si@xifmtarg#1#2Q#3#4#5\@nil{#4}
  \long\gdef\si@ifnotmtarg#1{%
    \si@xifmtarg#1QQ\@firstofone\@gobble\@nil}
\endgroup
%    \end{macrocode}
%\end{macro}
%\end{macro}
%\end{macro}
%\begin{macro}{\si@newrobustcmd}
%\begin{macro}{\si@newcommand}
%\begin{macro}{\si@newcmd}
%\begin{macro}{\si@xargdef}
%\darg{macro}
%\doarg{num-args}
%\doarg{default}
%\darg{def}
% Some more copying, this time from \pkg{etoolbox}.  Various macros
% need to be \emph{really} robust.  This is achieved using the \eTeX\
% \cs{protected} primitive in various places.  However, it would be
% nice to have a \cs{protected} version of \cs{newcommand}.
% \pkg{etoolbox} has code for that, but to avoid needing to load it,
% the necessary stuff is copied here.  The only changes form the
% original are names, and the use of \cs{newcommand*} for the
% \cs{si@newcommand} macro.
%    \begin{macrocode}
\@ifpackageloaded{etoolbox}
  {\let\si@newrobustcmd\newrobustcmd}
  {\protected\def\si@newrobustcmd{%
     \@ifstar
       {\let\l@ngrel@x\protected\si@newcommand}
       {\def\l@ngrel@x{\protected\long}\si@newcommand}}
   \newcommand*{\si@newcommand}[1]{\@testopt{\si@newcmd#1}0}
   \def\si@newcmd#1[#2]{%
     \@ifnextchar[%]
       {\si@xargdef#1[#2]}
       {\@argdef#1[#2]}}
   \long\def\si@xargdef#1[#2][#3]#4{%
     \@ifdefinable#1{%
       \expandafter\protected
       \expandafter\def
       \expandafter#1%
       \expandafter{%
         \expandafter\@testopt
         \csname\string#1\endcsname{#3}}%
       \expandafter\@yargdef
       \csname\string#1\endcsname\tw@{#2}{#4}}}}
%    \end{macrocode}
%\end{macro}
%\end{macro}
%\end{macro}
%\end{macro}
%
%\subsection{Logging}
%\begin{macro}{\ifsi@debug}
%\begin{macro}{\ifsi@logmin}
%\begin{macro}{\ifsi@lognone}
% To control logging, some new switches are declared.
%    \begin{macrocode}
\newif\ifsi@debug
\newif\ifsi@logmin
\newif\ifsi@lognone
%    \end{macrocode}
%\end{macro}
%\end{macro}
%\end{macro}
%\begin{macro}{\si@log@err}
%\darg{error}
%\darg{explanation}
% Some handy re-usable macros are defined here.  These all take names
% beginning  These pop up in various places.  First errors are
% handled.
%    \begin{macrocode}
\newcommand*{\si@log@err}[2]{%
  \ifsi@lognone\else
    \ifsi@logmin
      \PackageWarning{siunitx}{#1}%
    \else
      \PackageError{siunitx}{#1}{#2}%
    \fi
  \fi}
%    \end{macrocode}
%\end{macro}
%\begin{macro}{\si@log@warn}
%\begin{macro}{\si@log@inf}
%\darg{text}
% Then warnings and information.
%    \begin{macrocode}
\newcommand*{\si@log@warn}[1]{%
  \ifsi@lognone\else
    \ifsi@logmin\else
      \PackageWarning{siunitx}{#1}%
    \fi
  \fi}
\newcommand*{\si@log@inf}[1]{%
  \ifsi@lognone\else
    \ifsi@logmin\else
      \PackageInfo{siunitx}{#1}%
    \fi
  \fi}
%    \end{macrocode}
%\end{macro}
%\end{macro}
%\begin{macro}{\si@log@debug}
%\darg{debug-info}
% The debug macro only gives output if the appropriate package option
% is set.
%    \begin{macrocode}
\newcommand*{\si@log@debug}[1]{%
  \ifsi@lognone\else
    \ifsi@debug
      \PackageInfo{siunitx}{#1}%
    \fi
  \fi}
%    \end{macrocode}
%\end{macro}
%
%\subsection{String comparison}
%\begin{macro}{\si@str@ifchrstr}
%\begin{macro}{\si@str@chrstr}
%\darg{char}
%\darg{string}
% At various points, the package needs to compare two strings, to
% find if one occurs in the other.  The first test is if a single
% character is part of a second string; this is used, for example, to
% check that a character is valid as input.  The first argument is
% not expanded further, but the second is two allow division into
% individual units.
%    \begin{macrocode}
\newcommand*{\si@str@ifchrstr}[2]{%
  \begingroup
    \si@switchfalse
    \renewcommand*{\si@tempa}{#1}%
    \protected@edef\si@tempb{#2}%
    \expandafter\si@str@chrstr\si@tempb\@empty\@empty\@empty
    \ifsi@switch
      \aftergroup\@firstoftwo
    \else
      \aftergroup\@secondoftwo
    \fi
  \endgroup}
\def\si@str@chrstr#1#2\@empty{%
  \renewcommand*{\si@tempc}{#1}%
  \ifx\si@tempa\si@tempc
    \expandafter\si@switchtrue
  \else
    \ifx\@empty#2\@empty\else
      \si@str@chrstr#2\@empty\@empty
    \fi
  \fi}
%    \end{macrocode}
%\end{macro}
%\end{macro}
%\begin{macro}{\si@str@ifonlychrs}
%\begin{macro}{\si@str@onlychrs}
%\darg{string}
%\darg{chars}
% The second test builds on the first.  Here, a check is made to see
% if the first string is made up only of characters from the second
% string.  In this case, the first string is expanded before
% testing.  The second string will be expanded by the internal
% character by character test.
%    \begin{macrocode}
\newcommand*{\si@str@ifonlychrs}[2]{%
  \begingroup
    \si@switchtrue
    \protected@edef\si@tempa{#1}%
    \renewcommand*{\si@tempb}{#2}%
    \expandafter\si@str@onlychrs\si@tempa\@empty\@empty\@empty
    \ifsi@switch
      \aftergroup\@firstoftwo
    \else
      \aftergroup\@secondoftwo
    \fi
  \endgroup}
\def\si@str@onlychrs#1#2\@empty{%
  \si@str@ifchrstr{#1}{\si@tempb}
    {}{\si@switchfalse}%
  \ifx\@empty#2\@empty\else
    \si@str@onlychrs#2\@empty\@empty
  \fi}
%    \end{macrocode}
%\end{macro}
%\end{macro}
%
%\subsection{Option handling}
%\begin{macro}{\sisetup}
%\darg{options}
% To allow modification of options at run time, a setup macro is
% provided.
%    \begin{macrocode}
\newcommand*{\sisetup}{\setkeys[si]{key}}
%    \end{macrocode}
%\end{macro}
%\begin{macro}{\si@opt@key}
%\darg{keyname}
%\darg{code}
% To aid maintenance, some shortcuts are defined for generating keys.
% These also allow the debugging messages to be added automatically
% to every key.  First of all the basic key definition.
%    \begin{macrocode}
\newcommand*{\si@opt@key}[2]{%
  \define@key[si]{key}{#1}
  {#2\si@log@debug{Option #1 set to ##1}}}
%    \end{macrocode}
%\end{macro}
%\begin{macro}{\si@opt@cmdkey}
%\doarg{default}
%\darg{keyname}
%\darg{function}
% For a single command key, a function must be specified.
%    \begin{macrocode}
\newcommand*{\si@opt@cmdkey}[3][]{%
  \define@cmdkey[si]{key}[si@]{#2}[#1]{#3}}
%    \end{macrocode}
%\end{macro}
%\begin{macro}{\si@opt@cmdkeys}
%\doarg{default}
%\darg{keyname}
% Whereas multiple definitions do not take a function.
%    \begin{macrocode}
\newcommand*{\si@opt@cmdkeys}[2][]{%
  \define@cmdkeys[si]{key}[si@]{#2}[#1]}
%    \end{macrocode}
%\end{macro}
%\begin{macro}{\si@opt@boolkey}
%\doarg{function}
%\darg{keyname}
% Keys which only take switch values; anything other than \opt{true}
% or \opt{false} will generate a warning from \pkg{xkeyval}.
%    \begin{macrocode}
\newcommand*{\si@opt@boolkey}[2][]{%
  \define@boolkey[si]{key}[si@]{#2}[true]
  {#1\si@log@debug{Option #2 set to ##1}}}
%    \end{macrocode}
%\end{macro}
%\begin{macro}{\si@opt@choicekey}
%\doarg{default}
%\darg{keyname}
%\darg{choices}
%\darg{function}
% A ``fill in the blanks'' choice key.  In all cases, \cs{si@tempa}
% is used to hold the value given to the key, so that \cs{ifx}
% testing can occur.
%    \begin{macrocode}
\newcommand*{\si@opt@choicekey}[4][]{%
  \define@choicekey*+[si]{key}{#2}[\si@tempa]{#3}[#1]
    {#4\si@log@debug{Option #2 set to ##1}}
    {\si@log@warn{Unknown value `##1' for option #2}}}
%    \end{macrocode}
%\end{macro}
%\begin{macro}{\si@opt@xchoicekey}
%\darg{keyname}
%\darg{choices}
%\darg{initial-value}
% Several of the package options can take either a choice from a list
% of known options, or a value to be interpreted literally.  To aid
% maintenance, the necessary code can be set up here.  These keys all
% define a new macro, which must exist.  The \cs{si@opt@xchoicekey}
% macro therefore ensures that this is defined, as well as setting up
% the \pkg{xkeyval} key.
%    \begin{macrocode}
\newcommand*{\si@opt@xchoicekey}[3]{%
  \define@choicekey*+[si]{key}{#1}[\si@tempa]{#2}[#1]
%    \end{macrocode}
% This code will execute if the option is on the list.  There will be
% a ``fixed'' macro with a matching name, which is used for this.
%    \begin{macrocode}
    {\si@log@debug{Option #1 set to ##1}%
     \expandafter\renewcommand\expandafter*\expandafter{%
      \csname si@#1\endcsname}{\@nameuse{si@fix@##1}}}
%    \end{macrocode}
% The user has given something that is not on the list as an
% argument.  It is used literally.
%    \begin{macrocode}
    {\si@log@debug{Option #1 set to ##1}%
     \expandafter\renewcommand\expandafter*\expandafter{%
      \csname si@#1\endcsname}{##1}}
%    \end{macrocode}
% Finally, the initial value of the macro is set up.
%    \begin{macrocode}
  \expandafter\newcommand\expandafter*\expandafter{%
    \csname si@#1\endcsname}%
    {\@nameuse{si@fix@#3}}}
%    \end{macrocode}
%\end{macro}
%\begin{macro}{\si@opt@compatkey}
%\darg{package}
%\darg{keyname}
% An all-in-one definition for a back-compatibility key.  These
% should only be used at load time, so are automatically disabled
% once the package is loaded.  Emulation is also automatically turned
% on.
%    \begin{macrocode}
\newcommand*{\si@opt@compatkey}[2]{%
  \define@boolkey[si]{key}[si@old@]{#2}[true]
  {\si@log@debug{Emulating #1 package option\MessageBreak #2}%
   \sisetup{emulate=#1}%
   \si@log@debug{Option #2 set to ##1}}
   \AtEndOfPackage{\si@opt@disablekey{#2}
     {Compatibility option #2 only\MessageBreak
        available when loading siunitx package}}}
%    \end{macrocode}
%\end{macro}
%\begin{macro}{\si@opt@disablekey}
%\darg{keyname}
%\darg{text}
% The ability to disable a key with a meaningful message is a must;
% the warning will come from \currpkg, and not from \pkg{xkeyval}.
%    \begin{macrocode}
\newcommand*{\si@opt@disablekey}[2]{%
  \key@ifundefined[si]{key}{#1}
    {}
    {\si@log@debug{Disabling key #1}%
     \si@opt@key{#1}{\si@log@warn{#2}}}}
%    \end{macrocode}
%\end{macro}
%
% The \pkg{xkeyval} package option for logging is declared. This is
% then processed to set the switches correctly.
%    \begin{macrocode}
\si@opt@choicekey[normal]{log}{debug,verbose,normal,errors,none}
%    \end{macrocode}
% A series of comparisons are made to assign the logging mode.  The
% \opt{normal} option is not tested, as executing the option sets the
% switches appropriately.
%    \begin{macrocode}
  {\si@debugfalse
   \si@logminfalse
   \si@lognonefalse
   \renewcommand*{\si@tempb}{none}%
   \ifx\si@tempa\si@tempb
     \si@lognonetrue
   \fi
   \renewcommand*{\si@tempb}{minimal}%
   \ifx\si@tempa\si@tempb
     \si@logmintrue
   \fi
   \renewcommand*{\si@tempb}{debug}%
   \ifx\si@tempa\si@tempb
     \si@debugtrue
   \fi
   \renewcommand*{\si@tempb}{verbose}%
   \ifx\si@tempa\si@tempb
     \si@debugtrue
   \fi}
%    \end{macrocode}
% A quick method to set \opt{log=debug}.
%    \begin{macrocode}
\si@opt@boolkey{debug}
%    \end{macrocode}
%\begin{macro}{\ifsi@strict}
% It would be useful to be able to disable some keys, when strict
% interpretation of the rules is desired.  This is a load-time
% option, and has to disable various options.
%    \begin{macrocode}
\si@opt@boolkey[%
  \ifsi@strict
    \sisetup{
      obeymode=false,
      obeybold=false,
      obeyitalic=false,
      mode=maths,
      repeatunits=true,
      trapambigerr=true,
      trapambigfrac=true}
    \@for\si@tempa:=obeyall,obeymode,obeyitalic,mode,unitmode,%
      valuemode,textmode,obeybold,repeatunits,trapambigerr,%
      trapambigfrac\do{%
      \begingroup
        \edef\si@tempb{\endgroup
          \noexpand\si@opt@disablekey{\si@tempa}
          {Option `\si@tempa' forbidden in strict mode}}%
      \si@tempb}
  \fi]{strict}
\AtEndOfPackage{
  \si@opt@disablekey{strict}
    {Option `strict' only available when\MessageBreak
     loading package}}
%    \end{macrocode}
%\end{macro}
%\begin{macro}{\si@emulate}
% The \opt{emulate} option is used for back-compatibility mode; the
% option is only valid when loading \currpkg.
%    \begin{macrocode}
\newcommand*{\si@emulate}{}
\si@opt@key{emulate}{\si@addtolist{\si@emulate}{#1}}
\AtEndOfPackage{
  \si@opt@disablekey{emulate}
    {Option `emulate' only available when\MessageBreak
     loading package}}
%    \end{macrocode}
%\end{macro}
%\begin{macro}{\si@unitsep}
%\begin{macro}{\si@unitspace}
%\begin{macro}{\si@errspace}
%\begin{macro}{\si@valuesep}
% The two \opt{\dots sep} options control the size of spaces between
% the number and the unit (\cs{si@valuesep}), and that used to
% represent a product (\cs{si@unitsep}).  Known values here are
% \opt{thin}, \opt{med}, \opt{medium}, \opt{thick}, \opt{cdot},
% \opt{tightcdot}\footnote{Both \cs{cdot}-based options only valid
% for \opt{unitsep}.} and \opt{none};\footnote{Only valid for
% \opt{valuesep}.} other entries will be treated as custom spaces.
%    \begin{macrocode}
\si@opt@xchoicekey{unitsep}
  {thin,med,medium,thick,none,comma,stop,fullstop,period,
    times,tighttimes,cdot,tightcdot}{thin}
\si@opt@xchoicekey{unitspace}{space,thin,med,medium,thick,
  none}{thin}
\si@opt@xchoicekey{errspace}{space,thin,med,medium,thick,
  none}{none}
\si@opt@xchoicekey{valuesep}
  {thin,med,medium,thick,none,comma,stop,fullstop,period,
    times,tighttimes,cdot,tightcdot}{thin}
%    \end{macrocode}
%\end{macro}
%\end{macro}
%\end{macro}
%\end{macro}
%\begin{macro}{\si@digitsep}
% Separation of digits in large numbers is controlled by the
% \opt{digitsep} option.  As with the other \opt{sep} values, this
% one has a choice of possible values.  The list is quite long, so
% that a range of options are handled automatically.  Notice that
% \opt{digitsep=none} will be used for no separation at all.
%    \begin{macrocode}
\si@opt@xchoicekey{digitsep}
  {thin,med,medium,thick,none,comma,stop,fullstop,period,
    times,tighttimes,cdot,tightcdot}{thin}
%    \end{macrocode}
%\end{macro}
%\begin{macro}{\si@decimalsymbol}
% The symbol used for the decimal position is varied here.  There are
% only two real options, but options are given for the name of a full
% stop.
%    \begin{macrocode}
\si@opt@xchoicekey{decimalsymbol}{comma,stop,fullstop,period,
  cdot,tightcdot}{fullstop}
%    \end{macrocode}
%\end{macro}
%\begin{macro}{\si@anglesep}
% The separator between degrees and minutes, and between minutes and
% seconds, when using \cs{ang}.
%    \begin{macrocode}
\si@opt@xchoicekey{anglesep}
    {thin,med,medium,thick,none,comma,stop,fullstop,period,
      times,tighttimes,cdot,tightcdot}{none}
%    \end{macrocode}
%\end{macro}
%\begin{macro}{\ifsi@obeymode}
% The first test for the font control is whether to respect the
% surrounding maths or text mode.
%    \begin{macrocode}
\si@opt@boolkey{obeymode}
%    \end{macrocode}
%\end{macro}
%\begin{macro}{\ifsi@numtextmode}
%\begin{macro}{\ifsi@unittextmode}
% The output of the package can be typeset using either text or maths
% mode fonts. Two switches are needed, for numbers and units.
%    \begin{macrocode}
\newif\ifsi@numtextmode
\newif\ifsi@unittextmode
%    \end{macrocode}
%\end{macro}
%\end{macro}
% The \opt{textmode} option has to set both flags.
%    \begin{macrocode}
\si@opt@choicekey[true]{textmode}{true,false}
  {\si@numtextmodefalse
   \si@unittextmodefalse
   \renewcommand*{\si@tempb}{true}%
   \ifx\si@tempa\si@tempb
     \si@numtextmodetrue
     \si@unittextmodetrue
   \fi}
%    \end{macrocode}
% The \opt{mode} option applies to numbers and units.
%    \begin{macrocode}
\si@opt@choicekey{mode}{math,maths,text}
  {\si@numtextmodefalse
   \si@unittextmodefalse
   \renewcommand*{\si@tempb}{text}%
   \ifx\si@tempa\si@tempb
     \si@numtextmodetrue
     \si@unittextmodetrue
   \fi}
%    \end{macrocode}
% Now the two options for numbers or units alone.
%    \begin{macrocode}
\si@opt@choicekey{valuemode}{math,maths,text}
  {\si@numtextmodefalse
   \renewcommand*{\si@tempb}{text}%
   \ifx\si@tempa\si@tempb
     \si@numtextmodetrue
   \fi}
\si@opt@choicekey{unitmode}{math,maths,text}
  {\si@unittextmodefalse
   \renewcommand*{\si@tempb}{text}%
   \ifx\si@tempa\si@tempb
     \si@unittextmodetrue
   \fi}
%    \end{macrocode}
%\begin{macro}{\ifsi@obeyfamily}
% The package can work to match the font family (serif, sans serif,
% typewriter) of the surrounding text.  This is controlled by a
% Boolean option.
%    \begin{macrocode}
\si@opt@boolkey{obeyfamily}
%    \end{macrocode}
%\end{macro}
%\begin{macro}{\ifsi@obeybold}
% The package can attempt to respect bold, or may ignore it.
%    \begin{macrocode}
\si@opt@boolkey{obeybold}
%    \end{macrocode}
%\end{macro}
%\begin{macro}{\ifsi@inlinebtext}
% For inline maths, two options for checking what is bold are
% available, the maths environment (\ie \cs{boldmath}) and the
% surrounding text (\cs{textbf} or \cs{bffamily}).
%    \begin{macrocode}
\newif\ifsi@inlinebtext
\si@opt@choicekey{inlinebold}{text,maths,math}
  {\si@inlinebtextfalse
   \renewcommand*{\si@tempb}{text}%
   \ifx\si@tempa\si@tempb
     \si@inlinebtexttrue
   \fi}
%    \end{macrocode}
%\end{macro}
%\begin{macro}{\ifsi@obeyitalic}
% Italic is slightly different to bold, as there is no convenient
% switch for maths.  Thus a choice key is used, with the appropriate
% check.
%    \begin{macrocode}
\si@opt@boolkey{obeyitalic}
%    \end{macrocode}
%\end{macro}
%\begin{macro}{\ifsi@detectdisplay}
% For handling display mathematics, a setting is needed for whether to
% treat it differently from other maths.
%    \begin{macrocode}
\si@opt@boolkey{detectdisplay}
%    \end{macrocode}
%\end{macro}
% The option to obey all font switching commands is Boolean-like but
% needs alternative setup.
%    \begin{macrocode}
\si@opt@choicekey[true]{obeyall}{true,false}
  {\si@obeyboldfalse
   \si@obeyitalicfalse
   \si@obeymodefalse
   \si@obeyfamilyfalse
   \renewcommand*{\si@tempb}{true}%
   \ifx\si@tempa\si@tempb
     \si@obeyboldtrue
     \si@obeyitalictrue
     \si@obeymodetrue
     \si@obeyfamilytrue
   \fi}
%    \end{macrocode}
%\begin{macro}{\si@valuemathsrm}
%\begin{macro}{\si@valuemathssf}
%\begin{macro}{\si@valuemathstt}
%\begin{macro}{\si@unitmathsrm}
%\begin{macro}{\si@unitmathssf}
%\begin{macro}{\si@unitmathstt}
% The fonts used by the package default to the obvious \LaTeX\ ones;
% however, this needs to be exposed to user modification.  First the
% maths mode fonts are sorted out.
%    \begin{macrocode}
\si@opt@cmdkeys{valuemathsrm,valuemathssf,valuemathstt}
\si@opt@cmdkeys{unitmathsrm,unitmathssf,unitmathstt}
%    \end{macrocode}
%\end{macro}
%\end{macro}
%\end{macro}
%\end{macro}
%\end{macro}
%\end{macro}
% To make life easier for the user, US spellings are provided for the
% maths keys.
%    \begin{macrocode}
\si@opt@key{valuemathrm}{\sisetup{valuemathsrm=#1}}
\si@opt@key{valuemathsf}{\sisetup{valuemathssf=#1}}
\si@opt@key{valuemathtt}{\sisetup{valuemathstt=#1}}
\si@opt@key{unitmathrm}{\sisetup{unitmathsrm=#1}}
\si@opt@key{unitmathsf}{\sisetup{unitmathssf=#1}}
\si@opt@key{unitmathtt}{\sisetup{unitmathstt=#1}}
%    \end{macrocode}
% The combined options are given, for setting numbers and units
% at the same time.
%    \begin{macrocode}
\si@opt@key{mathsrm}{\sisetup{valuemathsrm=#1,unitmathsrm=#1}}
\si@opt@key{mathssf}{\sisetup{valuemathssf=#1,unitmathssf=#1}}
\si@opt@key{mathstt}{\sisetup{valuemathstt=#1,unitmathstt=#1}}
\si@opt@key{mathrm}{\sisetup{valuemathsrm=#1,unitmathsrm=#1}}
\si@opt@key{mathsf}{\sisetup{valuemathssf=#1,unitmathssf=#1}}
\si@opt@key{mathtt}{\sisetup{valuemathstt=#1,unitmathstt=#1}}
%    \end{macrocode}
%\begin{macro}{\si@valuetextrm}
%\begin{macro}{\si@valuetextsf}
%\begin{macro}{\si@valuetexttt}
%\begin{macro}{\si@unittextrm}
%\begin{macro}{\si@unittextsf}
%\begin{macro}{\si@unittexttt}
% The same thing for text mode fonts.  Once again the default values
% are pretty obvious.
%    \begin{macrocode}
\si@opt@cmdkeys{valuetextrm,valuetextsf,valuetexttt}
\si@opt@cmdkeys{unittextrm,unittextsf,unittexttt}
\si@opt@key{textrm}{\sisetup{unittextrm=#1,valuetextrm=#1}}
\si@opt@key{textsf}{\sisetup{unittextsf=#1,valuetextsf=#1}}
\si@opt@key{texttt}{\sisetup{unittexttt=#1,valuetexttt=#1}}
%    \end{macrocode}
%\end{macro}
%\end{macro}
%\end{macro}
%\end{macro}
%\end{macro}
%\end{macro}
%\changes{v1.1}{2008/09/22}{Added \opt{numdiv} option}
%\begin{macro}{\si@numdigits}
%\begin{macro}{\si@numdiv}
%\begin{macro}{\si@numdecimal}
%\begin{macro}{\si@numexp}
%\begin{macro}{\si@numprod}
%\begin{macro}{\si@numgobble}
%\begin{macro}{\si@numsign}
%\begin{macro}{\si@numcloseerr}
%\begin{macro}{\si@numopenerr}
%\begin{macro}{\si@numaddn}
% The list of possible valid characters for parsing numbers is set
% up.  This is similar to \numprint, but with the \opt{extra} class,
% and with characters ignored with no output renamed as \opt{gobble}.
%    \begin{macrocode}
\si@opt@cmdkeys{numdigits,numdecimal,numexp,numgobble,numsign,%
  numcloseerr,numopenerr,numaddn,numprod,numdiv}
%    \end{macrocode}
%\end{macro}
%\end{macro}
%\end{macro}
%\end{macro}
%\end{macro}
%\end{macro}
%\end{macro}
%\end{macro}
%\end{macro}
%\end{macro}
%\begin{macro}{\si@numvalid}
%\begin{macro}{\si@numextra}
% The various valid characters are collected together in a single
% macro for later.  In common with the above macros, this one starts
% \cs{si@num\ldots}.  The order here is the order the values are
% tested later on.
%    \begin{macrocode}
\newcommand*{\si@numextra}{%
  \si@numopenerr\si@numcloseerr\si@numaddn}
\newcommand*{\si@numvalid}{\si@numgobble\si@numexp\si@numsign
  \si@numdecimal\si@numdigits\si@numextra\si@numprod\si@numdiv}
%    \end{macrocode}
%\end{macro}
%\end{macro}
%\begin{macro}{\ifsi@seperr}
%\begin{macro}{\ifsi@trapambigerr}
%\begin{macro}{\si@openerr}
%\begin{macro}{\si@closeerr}
% An option to control whether numerical error values are printed with
% or separate from the number.
%    \begin{macrocode}
\si@opt@boolkey{seperr}
\si@opt@boolkey{trapambigerr}
\si@opt@cmdkeys{openerr,closeerr}
%    \end{macrocode}
%\end{macro}
%\end{macro}
%\end{macro}
%\end{macro}
%\begin{macro}{\ifsi@sepfour}
% With four digits in a number, separating may or may not be desired.
% Note that this option is the same as one for \numprint.
%    \begin{macrocode}
\si@opt@boolkey{sepfour}
%    \end{macrocode}
%\end{macro}
%\begin{macro}{\ifsi@retainplus}
% An option to keep an explicit positive sign.
%    \begin{macrocode}
\si@opt@boolkey{retainplus}
%    \end{macrocode}
%\end{macro}
%\begin{macro}{\si@expbase}
%\begin{macro}{\si@expproduct}
% The options for exponents are set up.
%    \begin{macrocode}
\si@opt@xchoicekey{expproduct}{times,tighttimes,
  cdot,tightcdot}{times}
\si@opt@xchoicekey{expbase}{ten}{ten}
%    \end{macrocode}
%\end{macro}
%\end{macro}
%\begin{macro}{\ifsi@allowzeroexp}
% The package normally prevents \num[allowzeroexp]{e0}.
%    \begin{macrocode}
\si@opt@boolkey{allowzeroexp}
%    \end{macrocode}
%\end{macro}
%\begin{macro}{\si@prefixproduct}
% The marker for multiplication in prefixes.
%    \begin{macrocode}
\si@opt@xchoicekey{prefixproduct}{times,tighttimes,cdot,
  tightcdot,none}{times}
%    \end{macrocode}
%\end{macro}
%\begin{macro}{\si@prefixbase}
% In the same area, the power for prefixes is variable.  Here, two
% choices ar needed.
%    \begin{macrocode}
\si@opt@xchoicekey{prefixbase}{ten,two}{ten}
%    \end{macrocode}
%\end{macro}
%\begin{macro}{\ifsi@prefixsymbolic}
% Unit prefixes can be given as either symbols or numerically.
%    \begin{macrocode}
\si@opt@boolkey{prefixsymbolic}
%    \end{macrocode}
%\end{macro}
%\begin{macro}{\ifsi@num@padlead}
%\begin{macro}{\ifsi@num@padtrail}
% A setting is needed to indicate when to add zeros to decimal
% numbers, either before the decimal marker (|.1| giving
% ``\num[padnumber=all]{.1}'') or after (|1.| giving
% ``\num[padnumber=all]{1.}'').
%    \begin{macrocode}
\newif\ifsi@num@padlead
\newif\ifsi@num@padtrail
\si@opt@choicekey[all]{padnumber}
  {leading,lead,trailing,trail,all,both,true,none,false}
  {\si@num@padleadfalse
   \si@num@padtrailfalse
   \renewcommand*{\si@tempb}{leading}%
   \ifx\si@tempa\si@tempb
     \si@num@padleadtrue
   \fi
   \renewcommand*{\si@tempb}{lead}%
   \ifx\si@tempa\si@tempb
     \si@num@padleadtrue
   \fi
   \renewcommand*{\si@tempb}{trailing}%
   \ifx\si@tempa\si@tempb
     \si@num@padtrailtrue
   \fi
   \renewcommand*{\si@tempb}{trail}%
   \ifx\si@tempa\si@tempb
     \si@num@padtrailtrue
   \fi
   \renewcommand*{\si@tempb}{all}%
   \ifx\si@tempa\si@tempb
     \si@num@padleadtrue
     \si@num@padtrailtrue
   \fi
   \renewcommand*{\si@tempb}{true}%
   \ifx\si@tempa\si@tempb
     \si@num@padleadtrue
     \si@num@padtrailtrue
   \fi
   \renewcommand*{\si@tempb}{both}%
   \ifx\si@tempa\si@tempb
     \si@num@padleadtrue
     \si@num@padtrailtrue
   \fi}
%    \end{macrocode}
%\end{macro}
%\end{macro}
%\begin{macro}{\si@sign}
%\begin{macro}{\ifsi@num@signmant}
%\begin{macro}{\ifsi@num@signexp}
% Some new switches for adding signs to numbers
%    \begin{macrocode}
\newif\ifsi@num@signmant
\newif\ifsi@num@signexp
%    \end{macrocode}
%\end{macro}
%\end{macro}
% Signs can be added to numbers by default. Two options are needed
% here; whether to add a sign by default, and what the sign is.
%    \begin{macrocode}
\si@opt@xchoicekey{sign}{plus,minus,pm,mp}{plus}
\si@opt@choicekey[all]{addsign}
  {mantissa,exponent,mant,exp,all,both,true,none,false}
%    \end{macrocode}
%\end{macro}
% The option is now processed.
%    \begin{macrocode}
  {\si@num@signmantfalse
   \si@num@signexpfalse
   \renewcommand*{\si@tempb}{mantissa}%
   \ifx\si@tempa\si@tempb
     \si@num@signmanttrue
   \fi
   \renewcommand*{\si@tempb}{mant}%
   \ifx\si@tempa\si@tempb
     \si@num@signmanttrue
   \fi
   \renewcommand*{\si@tempb}{exponent}%
   \ifx\si@tempa\si@tempb
     \si@num@signexptrue
   \fi
   \renewcommand*{\si@tempb}{exp}%
   \ifx\si@tempa\si@tempb
     \si@num@signexptrue
   \fi
   \renewcommand*{\si@tempb}{all}%
   \ifx\si@tempa\si@tempb
     \si@num@signmanttrue
     \si@num@signexptrue
   \fi
   \renewcommand*{\si@tempb}{true}%
   \ifx\si@tempa\si@tempb
     \si@num@signmanttrue
     \si@num@signexptrue
   \fi
   \renewcommand*{\si@tempb}{both}%
   \ifx\si@tempa\si@tempb
     \si@num@signmanttrue
     \si@num@signexptrue
   \fi}
%    \end{macrocode}
%\begin{macro}{\ifsi@tightpm}
%\begin{macro}{\si@pm}
% To reduce spacing, it might be necessary to use a ``tight'' $\pm$
% sign.
%    \begin{macrocode}
\si@opt@boolkey{tightpm}
\newcommand*{\si@pm}{%
  \ifsi@tightpm
    \si@fix@tightpm
  \else
    \si@fix@pm
  \fi}
%    \end{macrocode}
%\end{macro}
%\end{macro}
%\begin{macro}{\ifsi@ang@padsmall}
%\begin{macro}{\ifsi@ang@padlarge}
% A switch for determining whether to typeset |\ang{;;1}| as
% \ang[padangle=both]{;;1} or \ang[padangle=none]{;;1}.  First, two
% new Boolean switches are needed to indicate padding.
%    \begin{macrocode}
\newif\ifsi@ang@padsmall
\newif\ifsi@ang@padlarge
\si@opt@choicekey[all]{padangle}
  {small,large,all,both,true,none,false}
  {\si@ang@padsmallfalse
   \si@ang@padlargefalse
   \renewcommand*{\si@tempb}{small}%
   \ifx\si@tempa\si@tempb
     \si@ang@padsmalltrue
   \fi
   \renewcommand*{\si@tempb}{large}%
   \ifx\si@tempa\si@tempb
     \si@ang@padlargetrue
   \fi
   \renewcommand*{\si@tempb}{all}%
   \ifx\si@tempa\si@tempb
     \si@ang@padsmalltrue
     \si@ang@padlargetrue
   \fi
   \renewcommand*{\si@tempb}{true}%
   \ifx\si@tempa\si@tempb
     \si@ang@padsmalltrue
     \si@ang@padlargetrue
   \fi
   \renewcommand*{\si@tempb}{both}%
   \ifx\si@tempa\si@tempb
     \si@ang@padsmalltrue
     \si@ang@padlargetrue
   \fi}
%    \end{macrocode}
%\end{macro}
%\end{macro}
%\begin{macro}{\ifsi@ang@toarc}
%\begin{macro}{\ifsi@ang@todec}
% To control whether angles are formatted as decimals or
% degrees--minutes--seconds, a package option plus two switches are
% needed.  The later format is referred to as ``\opt{arc}'' most
% readily.  An option to leave the input unchanged is also provided.
%    \begin{macrocode}
\newif\ifsi@ang@toarc
\newif\ifsi@ang@todec
\si@opt@choicekey[all]{angformat}
  {dec,decimal,arc,dms,unchanged,none}
  {\si@ang@toarcfalse
   \si@ang@todecfalse
   \renewcommand*{\si@tempb}{dec}%
   \ifx\si@tempa\si@tempb
     \si@ang@todectrue
   \fi
   \renewcommand*{\si@tempb}{decimal}%
   \ifx\si@tempa\si@tempb
     \si@ang@todectrue
   \fi
   \renewcommand*{\si@tempb}{arc}%
   \ifx\si@tempa\si@tempb
     \si@ang@toarctrue
   \fi
   \renewcommand*{\si@tempb}{dms}%
   \ifx\si@tempa\si@tempb
     \si@ang@toarctrue
   \fi}
%    \end{macrocode}
%\end{macro}
%\end{macro}
%\begin{macro}{\ifsi@astroang}
% A slightly odd option to allow the method used by astronomers for
% angles.
%    \begin{macrocode}
\si@opt@boolkey{astroang}
%    \end{macrocode}
%\end{macro}
%\begin{macro}{\ifsi@strictarc}
% For the \cs{ang} macro, the default is to require two semi-colons
% in the input for arc angles.  This is controlled here.
%    \begin{macrocode}
\si@opt@boolkey{strictarc}
%    \end{macrocode}
%\end{macro}
%\changes{v1.1}{2008/09/03}{Added \opt{tabparseonly} option}
%\begin{macro}{\ifsi@tab@fixed}
%\begin{macro}{\ifsi@tabparseonly}
%\begin{macro}{\si@tabnumalign}
%\begin{macro}{\si@tab@rfill@S}
%\begin{macro}{\si@tab@lfill@S}
% To determine the control of table alignment, two options are
% provided.  The \opt{tabnumalign} option controls which centring method
% is used, and the fills used for achieving this.
%    \begin{macrocode}
\newif\ifsi@tab@fixed
\si@opt@boolkey{tabparseonly}
\si@opt@choicekey{tabnumalign}
  {centredecimal,centerdecimal,right,left,centre,center}
  {\si@tab@fixedtrue
   \def\si@tab@rfill@S{0.5}%
   \def\si@tab@lfill@S{0.5}%
   \renewcommand*{\si@tempb}{right}%
   \ifx\si@tempa\si@tempb
     \def\si@tab@rfill@S{0}%
     \def\si@tab@lfill@S{1}%
   \fi
   \renewcommand*{\si@tempb}{left}%
   \ifx\si@tempa\si@tempb
     \def\si@tab@rfill@S{1}%
     \def\si@tab@lfill@S{0}%
   \fi
   \renewcommand*{\si@tempb}{centredecimal}%
   \ifx\si@tempa\si@tempb
     \expandafter\si@tab@fixedfalse
   \fi
   \renewcommand*{\si@tempb}{centerdecimal}%
   \ifx\si@tempa\si@tempb
     \expandafter\si@tab@fixedfalse
   \fi}
\si@opt@key{tabalign}{\sisetup{tabnumalign=#1,tabtextalign=#1,
  tabunitalign=#1}}
%    \end{macrocode}
%\end{macro}
%\end{macro}
%\end{macro}
%\end{macro}
%\end{macro}
%\begin{macro}{\ifsi@tabalignexp}
% A switch for aligning exponents.
%    \begin{macrocode}
\si@opt@boolkey{tabalignexp}
\si@opt@boolkey[%
  \si@tabalignexpfalse
  \ifsi@tabexpalign
    \expandafter\si@tabalignexptrue
  \fi
  ]{tabexpalign}
%    \end{macrocode}
%\end{macro}
%\begin{macro}{\si@tab@mantprecnt}
%\begin{macro}{\si@tab@mantpostcnt}
%\begin{macro}{\si@tab@expprecnt}
%\begin{macro}{\si@tab@exppostcnt}
%\begin{macro}{\ifsi@tab@mantsign}
%\begin{macro}{\ifsi@tab@expsign}
% To process the format information, various internal
% number-processing macros are used.  First, some
% storage areas are created.
%    \begin{macrocode}
\newcount\si@tab@mantprecnt
\newcount\si@tab@mantpostcnt
\newcount\si@tab@expprecnt
\newcount\si@tab@exppostcnt
\newif\ifsi@tab@mantsign
\newif\ifsi@tab@expsign
%    \end{macrocode}
%\end{macro}
%\end{macro}
%\end{macro}
%\end{macro}
%\end{macro}
%\end{macro}
% The input is split into a mantissa and exponent, then passed to a
% re-useable macro for further processing.
%    \begin{macrocode}
\si@opt@cmdkey{tabformat}
  {\si@num@fixpm
   \renewcommand*{\si@num@arg}{tabformat data}%
   \renewcommand*{\si@num@exp}{}%
   \renewcommand*{\si@num@mant}{}%
   \si@tab@mantsignfalse
   \si@tab@expsignfalse
   \si@switchfalse
   \si@num@sepmantexp{#1}%
%    \end{macrocode}
% When checking for a sign, the internal flag for finding but
% deleting a plus sign is used.
%    \begin{macrocode}
   \si@num@sepsign{mant}%
   \ifx\@empty\si@num@mantsign\@empty
     \ifsi@num@delplus
       \expandafter\expandafter\expandafter\si@tab@mantsigntrue
     \fi
   \else
     \expandafter\si@tab@mantsigntrue
   \fi
   \si@num@sepsign{exp}%
   \ifx\@empty\si@num@expsign\@empty
     \ifsi@num@delplus
       \expandafter\expandafter\expandafter\si@tab@expsigntrue
     \fi
   \else
     \expandafter\si@tab@expsigntrue
   \fi
   \si@opt@proctform{mant}%
   \si@opt@proctform{exp}%
%    \end{macrocode}
%\changes{v1.0h}{2008/08/02}{Simplified warnings for \opt{tabformat}}
% If both the integer and decimal parts are empty, then something has
% probably gone wrong.
%    \begin{macrocode}
   \ifnum\si@tab@mantpostcnt=\z@\relax
     \ifnum\si@tab@mantprecnt=\z@\relax
       \si@log@err{Empty mantissa argument for tabformat}
         {The argument `#1' contains no valid entry for
          a mantissa\MessageBreak It should be of the
          form `m.n', where  m and n are integers}%
     \fi
   \fi
%    \end{macrocode}
% If \opt{tabformat} has been given with
% \opt{tabnumalign=centredecimal} active, then the alignment is
% changed to centred.
%    \begin{macrocode}
   \ifsi@tab@fixed\else
     \sisetup{tabnumalign=centre}%
   \fi
   \let\pm\si@num@pm
   \let\mp\si@num@mp}
%    \end{macrocode}
%\begin{macro}{\si@opt@proctform}
%\darg{either \texttt{mant} or \texttt{exp}}
% Processing the number further uses the \cs{si@num@digits} macro.
% The results are stored in the appropriate counter.
%    \begin{macrocode}
\newcommand*{\si@opt@proctform}[1]{%
  \renewcommand*{\si@num@predec}{}%
  \renewcommand*{\si@num@postdec}{}%
  \si@switchfalse
  \expandafter\si@ifnotmtarg\expandafter{%
    \csname si@num@#1\endcsname}
     {\expandafter\expandafter\expandafter\si@num@digits
        \csname si@num@#1\endcsname\@empty\@empty}%
   \csname si@tab@#1precnt\endcsname\z@\relax
   \csname si@tab@#1postcnt\endcsname\z@\relax
   \ifx\@empty\si@num@predec\@empty\else
     \csname si@tab@#1precnt\endcsname\si@num@predec\relax
   \fi
   \ifx\@empty\si@num@postdec\@empty\else
     \csname si@tab@#1postcnt\endcsname\si@num@postdec\relax
   \fi}
%    \end{macrocode}
%\end{macro}
% The alignment of tabular material when not processed by \cs{num}
% needs to be available.
%    \begin{macrocode}
\si@opt@choicekey{tabtextalign}{left,right,centre,center}
%    \end{macrocode}
%\begin{macro}{\si@tab@rfill@t}
%\begin{macro}{\si@tab@lfill@t}
% By default, centring happens on both sides of the content of
% tabular material.
%    \begin{macrocode}
  {\def\si@tab@rfill@t{0.5}%
   \def\si@tab@lfill@t{0.5}%
   \renewcommand*{\si@tempb}{right}%
   \ifx\si@tempa\si@tempb
     \def\si@tab@rfill@t{0}%
     \def\si@tab@lfill@t{1}%
   \fi
   \renewcommand*{\si@tempb}{left}%
   \ifx\si@tempa\si@tempb
     \def\si@tab@rfill@t{1}%
     \def\si@tab@lfill@t{0}%
   \fi}
%    \end{macrocode}
%\end{macro}
%\end{macro}
% The alignment of unit columns for tabular material has a similar
% control option.
%    \begin{macrocode}
\si@opt@choicekey{tabunitalign}{left,right,centre,center}
%    \end{macrocode}
%\begin{macro}{\si@tab@rfill@s}
%\begin{macro}{\si@tab@lfill@s}
% By default, centring happens on both sides of the content of
% tabular material.
%    \begin{macrocode}
  {\def\si@tab@rfill@s{0.5}%
   \def\si@tab@lfill@s{0.5}%
   \renewcommand*{\si@tempb}{right}%
   \ifx\si@tempa\si@tempb
     \def\si@tab@rfill@s{0}%
     \def\si@tab@lfill@s{1}%
   \fi
   \renewcommand*{\si@tempb}{left}%
   \ifx\si@tempa\si@tempb
     \def\si@tab@rfill@s{1}%
     \def\si@tab@lfill@s{0}%
   \fi}
%    \end{macrocode}
%\end{macro}
%\end{macro}
%\begin{macro}{\ifsi@fixdp}
%\changes{v1.1}{2008/08/22}{Modified to work with \opt{fixsf} option}
%\begin{macro}{\si@num@dp}
%\changes{v1.1}{2008/08/22}{Modified to work with \opt{fixsf} option}
% To allow control of rounding, two options are needed.  One sets how
% many fixed digits to use, the second turns this function on and
% off.
%    \begin{macrocode}
\si@opt@boolkey[%
  \ifsi@fixdp
    \expandafter\si@fixsffalse
  \fi]{fixdp}
\newcount\si@num@dp
\si@opt@key{dp}{%
  \si@str@ifonlychrs{#1}{0123456789}
    {\si@num@dp#1\relax
     \si@fixdptrue
     \si@fixsffalse}
    {\si@log@err{Invalid input for dp option}
      {The dp option must be given a positive integer}}}
%    \end{macrocode}
%\end{macro}
%\end{macro}
%\begin{macro}{\ifsi@fixsf}
%\begin{macro}{\si@num@sf}
% A similar approach for fixing significant figures.
%    \begin{macrocode}
\si@opt@boolkey[%
  \ifsi@fixsf
    \expandafter\si@fixdpfalse
  \fi]{fixsf}
\newcount\si@num@sf
\si@opt@key{sf}{%
  \si@str@ifonlychrs{#1}{0123456789}
    {\si@num@sf#1\relax
     \ifnum\si@num@sf=\z@\relax
       \si@log@warn{Setting sf=0 disables function}%
       \si@fixsffalse
     \else
       \si@fixsftrue
       \si@fixdpfalse
     \fi}
    {\si@log@err{Invalid input for sf option}
      {The sf option must be given a positive integer}}}
%    \end{macrocode}
%\end{macro}
%\end{macro}
%\begin{macro}{\ifsi@tabautofit}
% To apply rounding automatically in a table, a separate option is
% used.
%    \begin{macrocode}
\si@opt@boolkey{tabautofit}
%    \end{macrocode}
%\end{macro}
%\begin{macro}{\ifsi@xspace}
% Unit macros on their own may need \pkg{xpsace}.
%    \begin{macrocode}
\si@opt@boolkey{xspace}
%    \end{macrocode}
%\end{macro}
%\begin{macro}{\ifsi@prespace}
%    \begin{macrocode}
\si@opt@boolkey
  [\si@unt@numfalse
   \ifsi@prespace
     \si@unt@numtrue
  \fi]
  {prespace}
%    \end{macrocode}
%\end{macro}
%\begin{macro}{\ifsi@allowoptarg}
% For \unitsdef users, a method to absorb optional arguments is
% needed.
%    \begin{macrocode}
\si@opt@boolkey{allowoptarg}
%    \end{macrocode}
%\end{macro}
%\changes{v1.1d}{2008/10/29}{Minor bug with \opt{per} option fixed}
%\begin{macro}{\ifsi@frac}
%\begin{macro}{\ifsi@slash}
%\begin{macro}{\ifsi@stickyper}
% The option processing for formatting units with \cs{per} in them
% needs two switches.
%    \begin{macrocode}
\newif\ifsi@slash
\newif\ifsi@frac
\si@opt@boolkey{stickyper}
\si@opt@choicekey[reciprocal]{per}
  {reciprocal,rp,power,slash,frac,fraction}
  {\si@slashfalse
   \si@fracfalse
   \let\si@frac\si@frc@frac
   \renewcommand*{\si@tempb}{slash}%
   \ifx\si@tempa\si@tempb
     \si@fractrue
     \si@slashtrue
     \let\si@frac\si@frc@slash
   \fi
   \renewcommand*{\si@tempb}{frac}%
   \ifx\si@tempa\si@tempb
     \si@fractrue
   \fi
   \renewcommand*{\si@tempb}{fraction}%
   \ifx\si@tempa\si@tempb
     \si@fractrue
   \fi}
%    \end{macrocode}
%\end{macro}
%\end{macro}
%\end{macro}
%\begin{macro}{\si@slash}
% For the \opt{slash} option, the separator can be customised.
%    \begin{macrocode}
\si@opt@xchoicekey{slash}{slash}{slash}
%    \end{macrocode}
%\end{macro}
%\begin{macro}{\ifsi@repeatunits}
%\begin{macro}{\ifsi@addunitpower}
% An option is needed for cases where units should be repeated.
%    \begin{macrocode}
\newif\ifsi@repeatunits
\newif\ifsi@addunitpower
\si@opt@choicekey[true]{repeatunits}{true,false,power}
  {\si@repeatunitsfalse
   \si@addunitpowerfalse
   \renewcommand*{\si@tempb}{true}%
   \ifx\si@tempa\si@tempb
     \si@repeatunitstrue
   \fi
   \renewcommand*{\si@tempb}{power}%
   \ifx\si@tempa\si@tempb
     \si@addunitpowertrue
   \fi}
%    \end{macrocode}
%\end{macro}
%\end{macro}
%\begin{macro}{\ifsi@trapambigfrac}
%\begin{macro}{\si@closefrac}
%\begin{macro}{\si@openfrac}
% Macros for the right and left brackets added to potentially
% ambiguous denominators.
%    \begin{macrocode}
\si@opt@boolkey{trapambigfrac}
\si@opt@cmdkeys{closefrac,openfrac}
%    \end{macrocode}
%\end{macro}
%\end{macro}
%\end{macro}
% In the case of fractional handling of the \cs{per} operator,
% further refinement is available.
%    \begin{macrocode}
\si@opt@choicekey[frac]{fraction}
  {frac,nicefrac,nice,sfrac,xfrac,uglyfrac,ugly}
  {\let\si@frac\si@frc@frac
   \renewcommand*{\si@tempb}{nicefrac}%
   \ifx\si@tempa\si@tempb
     \let\si@frac\si@frc@nice
   \fi
   \renewcommand*{\si@tempb}{uglyfrac}%
   \ifx\si@tempa\si@tempb
     \let\si@frac\si@frc@ugly
   \fi
   \renewcommand*{\si@tempb}{nice}%
   \ifx\si@tempa\si@tempb
     \let\si@frac\si@frc@nice
   \fi
   \renewcommand*{\si@tempb}{sfrac}%
   \ifx\si@tempa\si@tempb
     \let\si@frac\si@frc@sfrac
   \fi
   \renewcommand*{\si@tempb}{xfrac}%
   \ifx\si@tempa\si@tempb
     \let\si@frac\si@frc@sfrac
   \fi
   \renewcommand*{\si@tempb}{ugly}%
   \ifx\si@tempa\si@tempb
     \let\si@frac\si@frc@ugly
   \fi}
%    \end{macrocode}
%\begin{macro}{\si@load}
%\begin{macro}{\si@noload}
% Loading of support files is controlled by two keys.  The first
% defines a list of files that may be loaded, the second a list that
% will not.  This makes it easy to exclude a single file from a long
% list.
%    \begin{macrocode}
\si@opt@cmdkeys{load,noload}
\si@opt@key{alsoload}{\si@addtolist{\si@load}{#1}}
\AtEndOfPackage{
  \si@opt@disablekey{load}
    {Configuration files can only be used\MessageBreak
     when loading package}
  \si@opt@disablekey{noload}
    {Configuration files can only be used\MessageBreak
     when loading package}}
\AtEndOfPackage{
  \si@opt@key{alsoload}{%
    \@for\si@tempa:=#1\do{\si@loadfile{\si@tempa}}}}
%    \end{macrocode}
%\end{macro}
%\end{macro}
%\begin{macro}{\ifsi@colourunits}
%\begin{macro}{\ifsi@colourvalues}
%\begin{macro}{\si@unitcolour}
%\begin{macro}{\si@valuecolour}
% Colour is turned on and off using two switches and the appropriate
% options.  US spellings are also provided.
%    \begin{macrocode}
\si@opt@boolkey{colourunits}
\si@opt@boolkey{colourvalues}
\si@opt@choicekey[true]{colorunits}
  {true,false}
  {\si@colourunitsfalse
   \renewcommand*{\si@tempb}{true}%
   \ifx\si@tempa\si@tempb
     \si@colourunitstrue
   \fi}
\si@opt@choicekey[true]{colorvalues}
  {true,false}
  {\si@colourvaluesfalse
   \renewcommand*{\si@tempb}{true}%
   \ifx\si@tempa\si@tempb
     \si@colourvaluestrue
   \fi}
\si@opt@choicekey[true]{colorall}
  {true,false}
  {\si@colourvaluesfalse
   \si@colourunitsfalse
   \renewcommand*{\si@tempb}{true}%
   \ifx\si@tempa\si@tempb
     \si@colourunitstrue
     \si@colourvaluestrue
   \fi}
\si@opt@choicekey[true]{colourall}
  {true,false}
  {\si@colourvaluesfalse
   \si@colourunitsfalse
   \renewcommand*{\si@tempb}{true}%
   \ifx\si@tempa\si@tempb
     \si@colourunitstrue
     \si@colourvaluestrue
   \fi}
\si@opt@cmdkeys{unitcolour,valuecolour}
\si@opt@key{unitcolor}{\sisetup{unitcolour=#1}}
\si@opt@key{valuecolor}{\sisetup{valuecolour=#1}}
\si@opt@key{colour}{\sisetup{unitcolour=#1,valuecolour=#1}}
\si@opt@key{color}{\sisetup{unitcolour=#1,valuecolour=#1}}
%    \end{macrocode}
%\end{macro}
%\end{macro}
%\end{macro}
%\end{macro}
%\begin{macro}{\ifsi@colourneg}
%\begin{macro}{\si@negcolour}
% The set up for colouring negative numbers is similar.
%    \begin{macrocode}
\si@opt@boolkey{colourneg}
\si@opt@choicekey[true]{colorneg}
  {true,false}
  {\si@colournegfalse
   \renewcommand*{\si@tempb}{true}%
   \ifx\si@tempa\si@tempb
     \si@colournegtrue
   \fi}
\si@opt@cmdkeys{negcolour}
\si@opt@key{negcolor}{\sisetup{negcolour=#1}}
%    \end{macrocode}
%\end{macro}
%\end{macro}
%\begin{macro}{\si@textOmega}
%\begin{macro}{\si@mathsOmega}
% The various non-Latin symbols need to be handled, and given user interfaces.
% Some definitions are more complex than others; for \si{\ohm}
% things are easy.
%    \begin{macrocode}
\si@opt@cmdkeys{textOmega,mathsOmega}
\si@opt@key{mathOmega}{\sisetup{mathsOmega=#1}}
\newcommand*{\si@mathsOmega}{\text{\ensuremath{\Omega}}}
\newcommand*{\si@textOmega}{\ensuremath{\Omega}}
%    \end{macrocode}
%\end{macro}
%\end{macro}
%\changes{v1.0c}{2008/06/25}{Fixed excess loading of maths fonts}
%\begin{macro}{\si@textmu}
%\begin{macro}{\si@mathsmu}
% For the \si{\micro} symbol, some direct loading of symbols is
% needed as the maths mu sign ($\mu$) is wrong.
%    \begin{macrocode}
\si@opt@cmdkeys{textmu,mathsmu}
\si@opt@key{mathmu}{\sisetup{mathsmu=#1}}
\DeclareFontEncoding{TS1}{}{}
\DeclareFontSubstitution{TS1}{cmr}{m}{n}
\DeclareTextSymbol{\si@textmu}{TS1}{181}
\DeclareTextSymbolDefault{\si@textmu}{TS1}
\@ifpackageloaded{upgreek}
  {}
  {\DeclareFontFamily{OML}{eur}{\skewchar\font'177}
   \DeclareFontShape{OML}{eur}{m}{n}{%
    <-6> eurm5 <6-8> eurm7 <8-> eurm10}{}}
\AtBeginDocument{
  \@ifpackageloaded{upgreek}
    {\let\si@mathsmu\upmu}
    {\DeclareSymbolFont{si@greek}{OML}{eur}{m}{n}
     \DeclareMathSymbol{\si@mathsmu}{\mathord}{si@greek}{"16}}}
%    \end{macrocode}
%\end{macro}
%\end{macro}
%\begin{macro}{\si@textdegree}
%\begin{macro}{\si@mathsdegree}
%\begin{macro}{\si@textminute}
%\begin{macro}{\si@mathsminute}
%\begin{macro}{\si@textsecond}
%\begin{macro}{\si@mathssecond}
% The angle signs.
%    \begin{macrocode}
\si@opt@cmdkeys{textdegree,mathsdegree,textminute,mathsminute,
  textsecond,mathssecond}
\si@opt@key{mathdegree}{\sisetup{mathsdegree=#1}}
\si@opt@key{mathminute}{\sisetup{mathsminute=#1}}
\si@opt@key{mathsecond}{\sisetup{mathssecond=#1}}
\newcommand*{\si@textdegree}{\ensuremath{{}^{\circ}}}
\newcommand*{\si@mathsdegree}{{}^{\circ}}
\newcommand*{\si@textminute}{\ensuremath{{}^{\prime}}}
\newcommand*{\si@mathsminute}{{}^{\prime}}
\newcommand*{\si@textsecond}{\ensuremath{{}^{\prime\prime}}}
\newcommand*{\si@mathssecond}{{}^{\prime\prime}}
%    \end{macrocode}
%\end{macro}
%\end{macro}
%\end{macro}
%\end{macro}
%\end{macro}
%\end{macro}
%\begin{macro}{\si@textcelsius}
%\changes{v1.1e}{2008/11/18}{Bug fix in appearance}
%\begin{macro}{\si@mathscelsius}
% Finally, degrees Celsius, which may need the degree symbol.
%    \begin{macrocode}
\si@opt@cmdkeys{textcelsius,mathscelsius}
\si@opt@key{mathcelsius}{\sisetup{mathscelsius=#1}}
\newcommand*{\si@textcelsius}{%
  \si@textdegree\kern-\scriptspace C}
\newcommand*{\si@mathscelsius}{%
  \si@mathsdegree\kern-\scriptspace C}
%    \end{macrocode}
%\end{macro}
%\end{macro}
%\begin{macro}{\si@textringA}
%\begin{macro}{\si@mathsringA}
% The \si{\angstrom} sign.
%    \begin{macrocode}
\si@opt@cmdkeys{textringA,mathsringA}
\si@opt@key{mathringA}{\sisetup{mathsringA=#1}}
\newcommand*{\si@textringA}{\AA}
\newcommand*{\si@mathsringA}{\text{\AA}}
%    \end{macrocode}
%\end{macro}
%\end{macro}
%\begin{macro}{\ifsi@redefsymbols}
% A flag for using \pkg{textcomp} and \pkg{upgreek} to provide better
% symbols.
%    \begin{macrocode}
\si@opt@boolkey{redefsymbols}
\AtBeginDocument{
  \si@opt@disablekey{redefsymbols}
    {Symbols can only be redefined\MessageBreak
     when loading siunitx}}
%    \end{macrocode}
%\end{macro}
%\begin{macro}{\si@eVcorra}
%\begin{macro}{\si@eVcorrb}
%    \begin{macrocode}
\newlength\si@eVcorra
\newlength\si@eVcorrb
\si@opt@key{eVcorra}{\setlength\si@eVcorra{#1}}
\si@opt@key{eVcorrb}{\setlength\si@eVcorrb{#1}}
%    \end{macrocode}
%\end{macro}
%\end{macro}
%\begin{macro}{\si@locale}
%\begin{macro}{\si@loctolang}
% Handling typographic conventions needs three keys.  \opt{locale} is
% used to set the locale, \opt{loctolang} to bind to \pkg{babel}.
%    \begin{macrocode}
\si@opt@key{locale}{%
  \si@loc@load{#1}%
  \si@loc@set{#1}}%
\si@opt@key{loctolang}{\si@loc@ltol{#1}}
%    \end{macrocode}
%\end{macro}
%\end{macro}
%
%\subsection{Compatibility options}
%\begin{macro}{\ifsi@old@ugly}
%\begin{macro}{\ifsi@old@nice}
%\begin{macro}{\ifsi@old@loose}
%\begin{macro}{\ifsi@old@tight}
% With the options for the package set up, the next stage is to
% provide support for users of the older packages.  These all set up
% switches, but do not do anything.  That is left to the emulation
% files, loaded at the end of the package. First of all, the
% \units options are dealt with; there are not many.
%    \begin{macrocode}
\si@opt@compatkey{units}{ugly}
\si@opt@compatkey{units}{nice}
\si@opt@compatkey{units}{loose}
\si@opt@compatkey{units}{tight}
%    \end{macrocode}
%\end{macro}
%\end{macro}
%\end{macro}
%\end{macro}
%\begin{macro}{\ifsi@old@OHM}
%\begin{macro}{\ifsi@old@ohm}
%\begin{macro}{\ifsi@old@redef-gensymb}
%\begin{macro}{\ifsi@gensymb}
% The \unitsdef package is unfortunately much more profligate with
% options. The first set are to do with support for \pkg{gensymb}.
%    \begin{macrocode}
\si@opt@compatkey{unitsdef}{OHM}
\si@opt@compatkey{unitsdef}{ohm}
\si@opt@compatkey{unitsdef}{redef-gensymb}
\newif\ifsi@gensymb
%    \end{macrocode}
%\end{macro}
%\end{macro}
%\end{macro}
%\end{macro}
%\begin{macro}{\ifsi@old@LITER}
%\begin{macro}{\ifsi@old@liter}
%\begin{macro}{\ifsi@old@noxspace}
%\begin{macro}{\ifsi@old@noconfig}
% The second set are more general functionality.
%    \begin{macrocode}
\si@opt@compatkey{unitsdef}{LITER}
\si@opt@compatkey{unitsdef}{liter}
\si@opt@compatkey{unitsdef}{noxspace}
\si@opt@compatkey{unitsdef}{noconfig}
%    \end{macrocode}
%\end{macro}
%\end{macro}
%\end{macro}
%\end{macro}
%\begin{macro}{\ifsi@old@noabbr}
%\begin{macro}{\ifsi@old@noamperageabbr}
%\begin{macro}{\ifsi@old@nofrequncyabbr}
%\begin{macro}{\ifsi@old@nomolabbr}
%\begin{macro}{\ifsi@old@novoltageabbr}
%\begin{macro}{\ifsi@old@novolumeabbr}
%\begin{macro}{\ifsi@old@noweightabbr}
%\begin{macro}{\ifsi@old@noenergyabbr}
%\begin{macro}{\ifsi@old@nolengthabbr}
%\begin{macro}{\ifsi@old@notimeabbr}
% The final set are for control of abbreviations, and are a good
% demonstration of why to use \pkg{xkeyval}!
%    \begin{macrocode}
\si@opt@compatkey{unitsdef}{noabbr}
\si@opt@compatkey{unitsdef}{noampereageabbr}
\si@opt@compatkey{unitsdef}{nofrequncyabbr}
\si@opt@compatkey{unitsdef}{nomolabbr}
\si@opt@compatkey{unitsdef}{novoltageabbr}
\si@opt@compatkey{unitsdef}{novolumeabbr}
\si@opt@compatkey{unitsdef}{noweightabbr}
\si@opt@compatkey{unitsdef}{noenergyabbr}
\si@opt@compatkey{unitsdef}{nolengthabbr}
\si@opt@compatkey{unitsdef}{notimeabbr}
%    \end{macrocode}
%\end{macro}
%\end{macro}
%\end{macro}
%\end{macro}
%\end{macro}
%\end{macro}
%\end{macro}
%\end{macro}
%\end{macro}
%\end{macro}
%\begin{macro}{\ifsi@old@cdot}
%\begin{macro}{\ifsi@old@thickspace}
%\begin{macro}{\ifsi@old@mediumspace}
%\begin{macro}{\ifsi@old@thinspace}
%\begin{macro}{\ifsi@old@thickqspace}
%\begin{macro}{\ifsi@old@mediumqspace}
%\begin{macro}{\ifsi@old@thinqspace}
% The \SIunits package has lots of options.  These ones are all
% related to spacing.
%    \begin{macrocode}
\si@opt@compatkey{SIunits}{cdot}
\si@opt@compatkey{SIunits}{thickspace}
\si@opt@compatkey{SIunits}{mediumspace}
\si@opt@compatkey{SIunits}{thinspace}
\si@opt@compatkey{SIunits}{thickqspace}
\si@opt@compatkey{SIunits}{mediumqspace}
\si@opt@compatkey{SIunits}{thinqspace}
%    \end{macrocode}
%\end{macro}
%\end{macro}
%\end{macro}
%\end{macro}
%\end{macro}
%\end{macro}
%\end{macro}
%\begin{macro}{\ifsi@old@amssymb}
%\begin{macro}{\ifsi@old@squaren}
%\begin{macro}{\ifsi@old@pstricks}
%\begin{macro}{\ifsi@old@Gray}
%\begin{macro}{\ifsi@old@italian}
% These options are used by \SIunits to control clashes with other
% packages.
%    \begin{macrocode}
\si@opt@compatkey{SIunits}{amssymb}
\si@opt@compatkey{SIunits}{squaren}
\si@opt@compatkey{SIunits}{pstricks}
\si@opt@compatkey{SIunits}{Gray}
\si@opt@compatkey{SIunits}{italian}
%    \end{macrocode}
%\end{macro}
%\end{macro}
%\end{macro}
%\end{macro}
%\end{macro}
%\begin{macro}{\ifsi@old@textstyle}
%\begin{macro}{\ifsi@old@binary}
%\begin{macro}{\ifsi@old@noams}
%\begin{macro}{\ifsi@old@derivedinbase}
%\begin{macro}{\ifsi@old@derived}
% The miscellaneous options.
%    \begin{macrocode}
\si@opt@compatkey{SIunits}{textstyle}
\si@opt@compatkey{SIunits}{binary}
\si@opt@compatkey{SIunits}{noams}
\si@opt@compatkey{SIunits}{derivedinbase}
\si@opt@compatkey{SIunits}{derived}
%    \end{macrocode}
%\end{macro}
%\end{macro}
%\end{macro}
%\end{macro}
%\end{macro}
%\begin{macro}{\ifsi@old@noprefixcmds}
% The \pkg{hepunits} package only has one option.
%    \begin{macrocode}
\si@opt@compatkey{hepunits}{noprefixcmds}
%    \end{macrocode}
%\end{macro}
%\begin{macro}{\ifsi@old@english}
%\begin{macro}{\ifsi@old@french}
% \pkg{fancynum} provides a few options.  First the rather oddly
% named \opt{english} and \opt{french} ones.
%    \begin{macrocode}
\si@opt@compatkey{fancynum}{english}
\si@opt@compatkey{fancynum}{french}
%    \end{macrocode}
%\end{macro}
%\end{macro}
%\begin{macro}{\ifsi@old@tight}
%\begin{macro}{\ifsi@old@loose}
% A couple for spacing multiplication.
%    \begin{macrocode}
\si@opt@compatkey{fancynum}{tight}
\si@opt@compatkey{fancynum}{loose}
%    \end{macrocode}
%\end{macro}
%\end{macro}
%\begin{macro}{\ifsi@old@thinspaces}
%\begin{macro}{\ifsi@old@commas}
%\begin{macro}{\ifsi@old@plain}
% Three for digit separation.
%    \begin{macrocode}
\si@opt@compatkey{fancynum}{thinspaces}
\si@opt@compatkey{fancynum}{commas}
\si@opt@compatkey{fancynum}{plain}
%    \end{macrocode}
%\end{macro}
%\end{macro}
%\end{macro}
%\begin{macro}{\ifsi@old@spaceqspace}
% The \pkg{fancyunits} package provides one option not available with
% \SIunits.
%    \begin{macrocode}
\si@opt@compatkey{fancyunits}{spaceqspace}
%    \end{macrocode}
%\end{macro}
%
%\subsection{Constants}
% A number of macros are needed by the package that provide a
% non-changing output.  These are defined here; the intention is that
% these should not be macros that the user is likely to need to
% alter.  All of these macros have preface \cs{si@fix@}, to flag that
% that are intended as constants.  The package may rely on the
% contents of these macros for functionality.
%
%\begin{macro}{\si@fix@thin}
%\begin{macro}{\si@fix@med}
%\begin{macro}{\si@fix@medium}
%\begin{macro}{\si@fix@thick}
%\begin{macro}{\si@fix@space}
%\changes{v1.0a}{2008/06/20}{Fixed problem with \opt{space} option}
% First, there are the various space macros.  To allow both \opt{med}
% and \opt{medium} to be used as a space description, two macros are
% needed for the same output.
%    \begin{macrocode}
\newcommand*{\si@fix@thin}{\,}
\newcommand*{\si@fix@med}{\:}
\newcommand*{\si@fix@medium}{\:}
\newcommand*{\si@fix@thick}{\;}
\newcommand*{\si@fix@space}{\text{ }}
%    \end{macrocode}
%\end{macro}
%\end{macro}
%\end{macro}
%\end{macro}
%\end{macro}
%\begin{macro}{\si@fix@cdot}
%\begin{macro}{\si@fix@comma}
%\begin{macro}{\si@fix@stop}
%\begin{macro}{\si@fix@fullstop}
%\begin{macro}{\si@fix@period}
%\begin{macro}{\si@fix@times}
%\begin{macro}{\si@fix@tighttimes}
%\begin{macro}{\si@fix@tightcdot}
% Next there are macros for material that is not simply whitespace.
% To allow several options, the full-stop gets lots of names.
%    \begin{macrocode}
\newcommand*{\si@fix@cdot}{{}\cdot{}}
\newcommand*{\si@fix@comma}{{,}}
\newcommand*{\si@fix@stop}{{.}}
\newcommand*{\si@fix@fullstop}{{.}}
\newcommand*{\si@fix@period}{{.}}
\newcommand*{\si@fix@times}{\times}
\newcommand*{\si@fix@tighttimes}{\bgroup\times\egroup}
\newcommand*{\si@fix@tightcdot}{\bgroup\cdot\egroup}
%    \end{macrocode}
%\end{macro}
%\end{macro}
%\end{macro}
%\end{macro}
%\end{macro}
%\end{macro}
%\end{macro}
%\end{macro}
%\begin{macro}{\si@fix@plus}
%\begin{macro}{\si@fix@minus}
%\begin{macro}{\si@fix@pm}
%\begin{macro}{\si@fix@tightpm}
%\begin{macro}{\si@fix@mp}
% Signs for numbers are needed.
%    \begin{macrocode}
\newcommand*{\si@fix@plus}{+}
\newcommand*{\si@fix@minus}{-}
\newcommand*{\si@fix@pm}{\pm}
\newcommand*{\si@fix@tightpm}{\bgroup\pm\egroup}
\newcommand*{\si@fix@mp}{\mp}
%    \end{macrocode}
%\end{macro}
%\end{macro}
%\end{macro}
%\end{macro}
%\end{macro}
%\begin{macro}{\si@fix@two}
%\begin{macro}{\si@fix@ten}
% The literals ``2'' and ``10'' are needed for exponents.
%    \begin{macrocode}
\newcommand*{\si@fix@two}{2}
\newcommand*{\si@fix@ten}{10}
%    \end{macrocode}
%\end{macro}
%\end{macro}
%\begin{macro}{\si@fix@slash}
% Another optional component that will probably not be used by many
% people.
%    \begin{macrocode}
\newcommand*{\si@fix@slash}{/}
%    \end{macrocode}
%\end{macro}
%\begin{macro}{\si@fix@none}
% Finally for spacing, there is the possibility of nothing at all
%    \begin{macrocode}
\newcommand*{\si@fix@none}{}
%    \end{macrocode}
%\end{macro}
%
%\subsection{Symbols}
%\begin{macro}{\si@symbol}
% Each of the symbol macros needs to be set up; the options give a
% maths and text mode sign, but internally a single macro is needed
% for each.
%    \begin{macrocode}
\newcommand*{\si@symbol}[1]{%
  \expandafter\protected\expandafter\def
    \csname si@sym@#1\endcsname{%
    \ifmmode
      \expandafter\csname si@maths#1\expandafter\endcsname
    \else
      \expandafter\csname si@text#1\expandafter\endcsname
    \fi}}
%    \end{macrocode}
%\end{macro}
%\begin{macro}{\si@sym@Omega}
%\begin{macro}{\si@sym@ringA}
%\begin{macro}{\si@sym@mu}
%\begin{macro}{\si@sym@degree}
%\begin{macro}{\si@sym@minute}
%\begin{macro}{\si@sym@second}
%\begin{macro}{\si@sym@celsius}
% The various symbols are now declared.
%    \begin{macrocode}
\si@symbol{Omega}
\si@symbol{ringA}
\si@symbol{mu}
\si@symbol{degree}
\si@symbol{minute}
\si@symbol{second}
\si@symbol{celsius}
%    \end{macrocode}
%\end{macro}
%\end{macro}
%\end{macro}
%\end{macro}
%\end{macro}
%\end{macro}
%\end{macro}
% The issue of redefinition of symbols now arises.  \currpkg can
% check for the loading of a number of support package, and can then
% redefine the appropriate symbols.
%    \begin{macrocode}
\AtBeginDocument{%
  \ifsi@redefsymbols
    \@ifpackageloaded{textcomp}
      {\si@log@debug{Redefining symbols using textcomp}%
       \renewcommand*{\si@textdegree}{\textdegree}%
       \renewcommand*{\si@mathsdegree}{\text{\textdegree}}%
%    \end{macrocode}
% \pkg{mathptmx} will give issues with \pkg{textcomp} and the
% \si{\ohm} sign.
%    \begin{macrocode}
       \@ifpackageloaded{mathptmx}{}
         {\renewcommand*{\si@textmu}{\textmu}%
          \renewcommand*{\si@textOmega}{\textohm}}%
%    \end{macrocode}
% The \si{\angstrom} symbol is only redefined if the encoding is
% OT1; other encodings should have a proper glyph used for \cs{AA}.
% The \cs{encodingdefault} macro is \cs{long} for some reason.
%    \begin{macrocode}
       \renewcommand*{\si@tempa}{OT1}%
       \ifx\si@tempa\encodingdefault
         \renewcommand*{\si@mathsringA}{%
           \text{\capitalring{A}}}%
         \renewcommand*{\si@textringA}{\capitalring{A}}
       \fi}{}
    \@ifpackageloaded{upgreek}
      {\si@log@debug{Redefining symbols using upgreek}%
       \renewcommand*{\si@mathsOmega}{\Upomega}}{}
  \fi}
%    \end{macrocode}
%
%\subsection{Handling fractions}
%\begin{macro}{\si@frac}
%\begin{macro}{\si@frc@hook}
%\begin{macro}{\si@frc@frac}
%\begin{macro}{\si@frc@slash}
%\begin{macro}{\si@frc@nice}
%\begin{macro}{\si@frc@sfrac}
% Various methods of handling fractions are provided.
%    \begin{macrocode}
\newcommand*{\si@frc@frac}[2]{%
  \ensuremath{\si@frc@hook\frac{%
    \expandafter\si@unt@out\expandafter{#1}}%
    {\expandafter\si@unt@out\expandafter{#2}}}}
\let\si@frac\si@frc@frac
\newcommand*{\si@frc@hook}{}
\newcommand*{\si@frc@slash}[2]{%
    \expandafter\si@unt@out\expandafter{#1}%
    \si@out{\ensuremath{\si@slash}}%
    \expandafter\si@unt@out\expandafter{#2}}
\newcommand*{\si@frc@nice}[2]{%
  \ensuremath{\si@frc@nicefrac{\expandafter\si@unt@out%
    \expandafter{#1}}{\expandafter\si@unt@out\expandafter
      {#2}}}}
\newcommand*{\si@frc@sfrac}[2]{%
    \sfrac{\expandafter\si@unt@out\expandafter{#1}}%
      {\expandafter\si@unt@out\expandafter{#2}}}
\AtBeginDocument{
  \@ifpackageloaded{xfrac}
    {}
    {\si@log@inf{xfrac package unavailable\MessageBreak
       using `fraction=sfrac' will fall back on\MessageBreak
       nicefrac-like method}%
     \renewcommand*{\si@frc@sfrac}[2]{%
      \si@log@warn{xfrac package unavailable}%
      \si@frc@nice{#1}{#2}}}}
%    \end{macrocode}
%\end{macro}
%\end{macro}
%\end{macro}
%\end{macro}
%\end{macro}
%\end{macro}
%\begin{macro}{\si@frc@nicefrac}
%\begin{macro}{\si@frc@displen}
%\begin{macro}{\si@frc@textlen}
%\begin{macro}{\si@frc@suplen}
%\begin{macro}{\si@frc@ssuplen}
% To avoid needing \units installed, the \cs{nicefrac} macro needs
% to be emulated here. The code is taken (with permission) from
% \kg{nicefrac}.\footnote{The original is licensed under the GPL;
% thanks to the author Axel Reichert for permission to copy the code
% here.}
%    \begin{macrocode}
\newlength\si@frc@displen
\newlength\si@frc@textlen
\newlength\si@frc@suplen
\newlength\si@frc@ssuplen
\newcommand*{\si@frc@nicefrac}{%
  \ifmmode
    \expandafter\si@frc@mathsnf
  \else
    \expandafter\si@frc@textnf
  \fi}
%    \end{macrocode}
%\end{macro}
%\end{macro}
%\end{macro}
%\end{macro}
%\end{macro}
%\begin{macro}{\si@frc@mathsnf}
%\darg{numerator}
%\darg{denominator}
% The maths mode system.
%    \begin{macrocode}
\newcommand*{\si@frc@mathsnf}[2]{%
  \begingroup
    \settoheight{\si@frc@displen}{\ensuremath{%
      \displaystyle{M}}}%
    \settoheight{\si@frc@textlen}{\ensuremath{%
      \textstyle{M}}}%
    \settoheight{\si@frc@suplen}{\ensuremath{%
      \scriptstyle{M}}}%
    \settoheight{\si@frc@ssuplen}{%
      \ensuremath{\scriptscriptstyle{M}}}%
    \addtolength{\si@frc@displen}{-\si@frc@ssuplen}%
    \addtolength{\si@frc@textlen}{-\si@frc@ssuplen}%
    \addtolength{\si@frc@suplen}{-\si@frc@ssuplen}%
    \mathchoice
      {\raisebox{\si@frc@displen}{\ensuremath{%
        \scriptstyle{#1}}}}%
      {\raisebox{\si@frc@textlen}{\ensuremath{%
         \scriptstyle{#1}}}}%
      {\raisebox{\si@frc@suplen}%
        {\ensuremath{\scriptscriptstyle{#1}}}}%
      {\raisebox{\si@frc@ssuplen}%
        {\ensuremath{\scriptscriptstyle{#1}}}}%
    \mkern-2mu\relax/\mkern-1mu\relax
    \bgroup
      \mathchoice
        {\scriptstyle}%
        {\scriptstyle}%
        {\scriptscriptstyle}%
        {\scriptscriptstyle}%
      {#2}%
    \egroup
  \endgroup}
%    \end{macrocode}
%\end{macro}
%\begin{macro}{\si@frc@textnf}
%\darg{numerator}
%\darg{denominator}
% A stripped down version of the \pkg{nicefrac} system for text mode.
%    \begin{macrocode}
\newcommand*{\si@frc@textnf}[2]{%
  \begingroup
    \settoheight{\si@frc@textlen}{M}%
    \settoheight{\si@frc@ssuplen}{\fontsize\sf@size\z@\relax
      \selectfont{M}}%
    \addtolength{\si@frc@textlen}{-\si@frc@ssuplen}%
    \raisebox{\si@frc@textlen}{\fontsize\sf@size\z@\relax
      \selectfont{#1}}%
    \hspace{-0.25ex}/\hspace{-0.25ex}%
    \hbox{\fontsize\sf@size\z@\selectfont{#2}}%
  \endgroup}
%    \end{macrocode}
%\end{macro}
%\begin{macro}{\si@frc@ugly}
%\darg{numerator}
% The \cs{si@frc@ugly} macro is needed to emulate the \opt{ugly}
% option in \units, where output depends on the current mode.
%    \begin{macrocode}
\newcommand*{\si@frc@ugly}[1]{%
  \renewcommand*{\si@tempa}{#1}%
  \ifmmode
    \expandafter\si@frc@frac
  \else
    \renewcommand*{\si@tempb}{1}%
    \ifx\si@tempa\si@tempb
%    \end{macrocode}
% The slash switch cannot be used, so the possibility of the
% numerator being one is handled here.
%    \begin{macrocode}
      \setbox\si@tempboxa=\hbox{\ensuremath{\si@valuesep}}%
      \hskip-\wd\si@tempboxa\relax
      \renewcommand*{\si@tempa}{}%
    \fi
    \expandafter\si@frc@slash
  \fi
  {\si@tempa}}
%    \end{macrocode}
%\end{macro}
%
%\subsection{Font control}
% A number of controls and tests are needed to control the font used
% for output.  Underlying all of this is the \AMS\ package
% \pkg{amstext} package, providing the \cs{text} command.  Much of
% the font control system here is taken more or less verbatim from
% \SIstyle; modifications have been made to fit the \currpkg
% interface.
%
%\begin{macro}{\si@fam@sf}
%\begin{macro}{\si@fam@tt}
% The package needs to know the maths font families in use.  This is
% set right at the start of the document, after any changes can have
% been made by, for example, \pkg{fontspec}.
%    \begin{macrocode}
\g@addto@macro{\document}{%
  \ifdefined\mathsf
    \setbox\si@tempboxa=\hbox{%
      $\mathsf{\global\chardef\si@fam@sf=\fam}$}%
  \else
    \si@log@inf{\string\mathsf not found}%
    \global\chardef\si@fam@sf=99\relax
  \fi
  \ifdefined\mathtt
    \setbox\si@tempboxa=\hbox{%
      $\mathtt{\global\chardef\si@fam@tt=\fam}$}%
  \else
    \si@log@inf{\string\mathtt not found}%
    \global\chardef\si@fam@tt=99\relax
  \fi}
%    \end{macrocode}
%\end{macro}
%\end{macro}
%\begin{macro}{\si@fam@ifbtext}
%\begin{macro}{\si@fam@ifbmaths}
%\darg{code}
% These tests check for bold in text and maths mode, respectively.
%    \begin{macrocode}
\newcommand*{\si@fam@ifbtext}[1]{%
  \if b\expandafter\@car\f@series\@nil
    #1\fi}
\newcommand*{\si@fam@ifbmaths}[1]{%
  \renewcommand*{\si@tempa}{bold}%
  \ifx\math@version\si@tempa
    #1\fi}
%    \end{macrocode}
%\end{macro}
%\end{macro}
%\begin{macro}{\si@fam@ifbinline}
% For compatibility with \units, a method to change the behaviour
% when in inline maths is needed for the bold detector.
%    \begin{macrocode}
\newcommand*{\si@fam@ifbinline}{%
  \ifsi@inlinebtext
    \expandafter\si@fam@ifbtext
  \else
    \expandafter\si@fam@ifbmaths
  \fi}
%    \end{macrocode}
%\end{macro}
%\begin{macro}{\si@fam@ifitext}
%\darg{code}
% This test check for italic or slanted text in text mode, by
% negation (upright text is |n|).
%    \begin{macrocode}
\newcommand*{\si@fam@ifitext}[1]{%
  \if n\expandafter\@car\f@series\@nil\else
    #1\fi}
%    \end{macrocode}
%\end{macro}
%\begin{macro}{\si@fam@mode}
% Detection of the current mode needs to happen``early'' (before any
% change of \cs{ensuremath}).  So a short macro is provided to do the
% job.
%    \begin{macrocode}
\newcommand*{\si@fam@mode}{%
  \ifsi@obeymode
    \ifmmode
      \sisetup{mode=maths}%
    \else
      \sisetup{mode=text}%
    \fi
  \fi}
%    \end{macrocode}
%\end{macro}
%\begin{macro}{\si@fam@colourcmd}
% The colour command is set up.
%    \begin{macrocode}
\AtBeginDocument{
  \@ifpackageloaded{color}
    {\let\si@fam@colourcmd\color}
    {\let\si@fam@colourcmd\@gobble}}
%    \end{macrocode}
%\end{macro}
%
%\begin{macro}{\ifsi@fam@set}
%\begin{macro}{\ifsi@textmode}
% A marker is set up to check if font-matching has been taken place.
% A second flag is used to track the use of text mode.
%    \begin{macrocode}
\newif\ifsi@fam@set
\newif\ifsi@textmode
%    \end{macrocode}
%\end{macro}
%\end{macro}
%\begin{macro}{\si@fam@set}
% Using the code from \SIstyle as a base, a set of tests are used to
% set the current font families and weights.  To begin with, the
% mode to use is set up.
%    \begin{macrocode}
\newcommand*{\si@fam@set}{%
  \ifsi@out@num
    \ifsi@numtextmode
      \expandafter\expandafter\expandafter\si@textmodetrue
    \else
      \expandafter\expandafter\expandafter\si@textmodefalse
    \fi
  \else
    \ifsi@unittextmode
      \expandafter\expandafter\expandafter\si@textmodetrue
    \else
      \expandafter\expandafter\expandafter\si@textmodefalse
    \fi
  \fi
%    \end{macrocode}
%\begin{macro}{\si@mathsrm}
%\begin{macro}{\si@mathssf}
%\begin{macro}{\si@mathstt}
%\begin{macro}{\si@textrm}
%\begin{macro}{\si@textsf}
%\begin{macro}{\si@texttt}
%\begin{macro}{\si@colourcmd}
%\begin{macro}{\si@colour}
% The appropriate font macros are now established, if necessary.
%    \begin{macrocode}
  \ifsi@fam@set\else
    \let\si@colourcmd\@gobble
    \ifsi@out@num
      \let\si@mathsrm\si@valuemathsrm
      \let\si@mathssf\si@valuemathssf
      \let\si@mathstt\si@valuemathstt
      \let\si@textrm\si@valuetextrm
      \let\si@textsf\si@valuetextsf
      \let\si@texttt\si@valuetexttt
      \ifsi@colourvalues
        \let\si@colourcmd\si@fam@colourcmd
      \fi
      \let\si@colour\si@valuecolour
    \else
      \let\si@mathsrm\si@unitmathsrm
      \let\si@mathssf\si@unitmathssf
      \let\si@mathstt\si@unitmathstt
      \let\si@textrm\si@unittextrm
      \let\si@textsf\si@unittextsf
      \let\si@texttt\si@unittexttt
      \ifsi@colourunits
        \let\si@colourcmd\si@fam@colourcmd
      \fi
      \let\si@colour\si@unitcolour
    \fi
  \fi
  \si@fam@settrue
%    \end{macrocode}
%\end{macro}
%\end{macro}
%\end{macro}
%\end{macro}
%\end{macro}
%\end{macro}
%\end{macro}
%\end{macro}
% The temporary macros are needed for the \cs{ifx} tests, which need to
% be expanded once.
%    \begin{macrocode}
  \edef\si@tempa{\sfdefault}%
  \edef\si@tempb{\ttdefault}%
%    \end{macrocode}
%\begin{macro}{\si@fam@maths}
%\begin{macro}{\si@fam@text}
% The surrounding font family is only tested if matching is
% requested.  First, the defaults are set up assuming no detection
% takes place.
%    \begin{macrocode}
  \expandafter\let\expandafter\si@fam@maths
    \csname\si@mathsrm\endcsname
  \expandafter\let\expandafter\si@fam@text
    \csname\si@textrm\endcsname
  \ifsi@obeyfamily
    \si@log@debug{Font detection: checking font}%
%    \end{macrocode}
%\end{macro}
%\end{macro}
% The detection code has to check the mode currently in operation.
% Display mathematics can be handled in two ways, so this means some
% code is repeated: it is spun out to separate routines.
%    \begin{macrocode}
    \ifmmode
      \ifinner
        \si@log@debug{Font detection: inline maths}%
        \si@fam@dettext
      \else
        \si@log@debug{Font detection: display maths}%
        \ifsi@detectdisplay
          \si@fam@detmaths
        \else
          \si@fam@dettext
        \fi
      \fi
    \else
      \si@log@debug{Font detection: text}%
      \si@fam@dettext
    \fi
  \else
    \si@log@debug{Font detection: inactive}%
  \fi
%    \end{macrocode}
%\begin{macro}{\si@fam@bold}
% With the font family set, the next check is for bold text.  This
% again needs to examine the current mode. Things are a bit more
% complex than in \SIstyle as it is possible to be typesetting in
% either text or maths mode.  The bold command is set up with
% \cs{def}, as nested calls can occur.
%    \begin{macrocode}
  \def\si@fam@bold{\unboldmath\mdseries}%
  \ifsi@obeybold
    \si@log@debug{Weight detection: checking weight}%
    \ifmmode
      \ifdim\displaywidth>0pt\relax
        \ifsi@detectdisplay
          \expandafter\si@fam@ifbmaths
        \else
          \expandafter\si@fam@ifbtext
        \fi
        \si@fam@setbold
      \else
        \si@fam@ifbinline\si@fam@setbold
      \fi
    \else
      \si@fam@ifbtext\si@fam@setbold
    \fi
  \fi
%    \end{macrocode}
%\end{macro}
%\begin{macro}{\si@fam@italic}
% The value of \opt{obeyitalic} is now tested; as this does nothing
% in maths mode, a reminder is added to the log.
%    \begin{macrocode}
  \let\si@fam@italic\upshape
  \ifsi@obeyitalic
    \si@log@debug{Italic detection: checking italic}%
    \si@fam@ifitext
      {\let\si@fam@italic\relax
       \si@log@debug{Italic detection: italic}}%
  \fi}
%    \end{macrocode}
%\end{macro}
%\end{macro}
%\begin{macro}{\si@fam@detmaths}
%\begin{macro}{\si@fam@dettext}
% Two detection macros are needed for maths and text mode. This
% allows handling of the various combinations without needing too
% many code lines.
%    \begin{macrocode}
\newcommand*{\si@fam@detmaths}{%
  \ifnum\the\fam=\si@fam@sf
    \si@log@debug{Font detection: sf}%
    \expandafter\let\expandafter\si@fam@maths
      \csname\si@mathssf\endcsname
    \expandafter\let\expandafter\si@fam@text
      \csname\si@textsf\endcsname
  \else
    \ifnum\the\fam=\si@fam@tt
      \si@log@debug{Font detection: tt}%
      \expandafter\let\expandafter\si@fam@maths
        \csname\si@mathstt\endcsname
      \expandafter\let\expandafter\si@fam@text
        \csname\si@texttt\endcsname
    \else
      \si@log@debug{Font detection: rm}%
      \expandafter\let\expandafter\si@fam@maths
        \csname\si@mathsrm\endcsname
      \expandafter\let\expandafter\si@fam@text
        \csname\si@textrm\endcsname
    \fi
  \fi}
\newcommand*{\si@fam@dettext}{%
  \ifx\f@family\si@tempa
    \si@log@debug{Font detection: sf}%
    \expandafter\let\expandafter\si@fam@maths
      \csname\si@mathssf\endcsname
    \expandafter\let\expandafter\si@fam@text
      \csname\si@textsf\endcsname
  \else
    \ifx\f@family\si@tempb
      \si@log@debug{Font detection: tt}%
      \expandafter\let\expandafter\si@fam@maths
        \csname\si@mathstt\endcsname
      \expandafter\let\expandafter\si@fam@text
        \csname\si@texttt\endcsname
    \else
      \si@log@debug{Font detection: rm}%
      \expandafter\let\expandafter\si@fam@maths
        \csname\si@mathsrm\endcsname
      \expandafter\let\expandafter\si@fam@text
        \csname\si@textrm\endcsname
    \fi
  \fi}
%    \end{macrocode}
%\end{macro}
%\end{macro}
%\begin{macro}{\si@fam@setbold}
%\begin{macro}{\si@fam@boldify}
% For setting bold, a couple of control macros are needed.
%    \begin{macrocode}
\newcommand*{\si@fam@setbold}{%
  \si@log@debug{Weight detection: bold weight}%
  \let\si@fam@bold\si@fam@boldify}
\newcommand*{\si@fam@boldify}{\boldmath\bfseries}
%    \end{macrocode}
%\end{macro}
%\end{macro}
%
%\subsection{Formatting numbers}
%\begin{macro}{\num}
%\doarg{keyval options}
%\darg{number}
% The system used here is modelled on that in \numprint; the
% input is broken down into single tokens, each one is examined and
% the result is re-assembled into an output number.  However, various
% changes have been made to the system used, and so the macros here
% are not simply renamed copies of those in \numprint.  The user
% macro \cs{num} sets any local keys, then calls the number
% formatting macro on the processed number.
%    \begin{macrocode}
\si@newrobustcmd*{\num}[2][]{%
  \begingroup
    \sisetup{#1}%
    \si@fam@mode
    \si@num@intabfalse
    \si@log@debug{Processing \string\num\space input `#2'}%
    \expandafter\si@out@num\expandafter{\si@num{#2}}%
  \endgroup}
%    \end{macrocode}
%\end{macro}
%\begin{macro}{\ifsi@num@intab}
% A flag for processing inside a table is needed.
%    \begin{macrocode}
\newif\ifsi@num@intab
%    \end{macrocode}
%\end{macro}
%\begin{macro}{\si@num}
%\darg{number}
% This is the main processing macro.  Unlike the related macro in
% \numprint, the output of this macro is not subjected to any font
% changes.  That is left to one of the \cs{si@out@\ldots} macros.  No
% grouping is applied here; any call to \cs{si@num} (or any of the
% sub-macros) must be within a group as the definitions used rely on
% this.  Grouping is not applied here so that other macros can get
% the various separated parts of the input.
%    \begin{macrocode}
\newcommand*{\si@num}[1]{%
%    \end{macrocode}
% The argument of the macro is fully expanded before any processing.
% By using \cs{scantokens}, any odd problems from packages with
% active characters can be avoided.
%    \begin{macrocode}
  \si@num@fixpm
  \begingroup
    \makeatletter
    \@makeother{\,}%
    \@makeother{\.}%
    \@makeother{\+}%
    \@makeother{\-}%
    \def~{}%
    \def\,{}%
    \catcode`\~=\active\relax
    \catcode`\^=\active\relax
    \catcode`\_=\active\relax
    \everyeof{\noexpand}%
    \endlinechar\m@ne
    \protected@xdef\si@tempa{\scantokens{#1}}%
  \endgroup
%    \end{macrocode}
% Processing only takes place if there is actually something in the
% argument.  This is tested once ``hard'' spaces have been stripped
% out; if there is input other than spaces, the processor first
% checks the validity of the input, then moves on to format it.
%    \begin{macrocode}
  \si@ifnotmtarg{\si@tempa}
    {\si@num@ifvalid{\si@tempa}
      {\si@num@format{\si@tempa}}
%    \end{macrocode}
% The parser has to bailed-out, and so no further processing of
% the input is done.  Instead, whatever was passed to the macro is
% returned as supplied.
%    \begin{macrocode}
      {\si@log@err{Invalid character `#1' in numerical input}%
         {Only characters from the list
          `\si@numvalid'\MessageBreak should be present in the
          argument of the \string\num\space  macro\MessageBreak
         (or derivative such as an `s' column)}%
       {#1}}}}
%    \end{macrocode}
%\end{macro}
%\begin{macro}{\si@num@fixpm}
%\begin{macro}{\si@num@pm}
%\begin{macro}{\si@num@mp}
%\begin{macro}{\pm}
%\begin{macro}{\mp}
% With certain packages loaded, there can be issues with \cs{pm} and
% \cs{mp}.  To avoid this, the \eTeX\ \cs{protected} system is
% employed; this is only used within local groups.
%    \begin{macrocode}
\newcommand*{\si@num@fixpm}{%
  \let\si@num@pm\pm
  \let\si@num@mp\mp
  \protected\def\pm{\si@num@pm}%
  \protected\def\mp{\si@num@mp}}
%    \end{macrocode}
%\end{macro}
%\end{macro}
%\end{macro}
%\end{macro}
%\end{macro}
%\begin{macro}{\si@num@ifvalid}
%\begin{macro}{\si@num@valid}
%\darg{chars}
% Assuming that there is a non-space argument to \cs{si@num}, every
% character is checked to ensure it is valid in the context, so that
% further processing can occur without sanity checks.  If the
% character is valid, recursion occurs.
%    \begin{macrocode}
\newcommand*{\si@num@ifvalid}[1]{%
  \begingroup
    \si@switchtrue
    \expandafter\si@num@valid#1\@empty\@empty
    \ifsi@switch
      \aftergroup\@firstoftwo
    \else
      \aftergroup\@secondoftwo
    \fi
  \endgroup}
\def\si@num@valid#1#2\@empty{%
  \si@str@ifchrstr{#1}{\si@numvalid}
    {\ifx\@empty#2\@empty\else
       \si@num@valid#2\@empty\@empty\@empty
     \fi}
    {\si@switchfalse}}
%    \end{macrocode}
%\end{macro}
%\end{macro}
%\begin{macro}{\si@num@in}
%\begin{macro}{\si@num@out}
%\begin{macro}{\si@num@exp}
%\begin{macro}{\si@num@expsign}
%\begin{macro}{\si@num@mant}
%\begin{macro}{\si@num@mantsign}
%\begin{macro}{\si@num@err}
%\begin{macro}{\si@num@xpart}
%\begin{macro}{\si@num@ambig}
%\begin{macro}{\si@tab@out}
%\begin{macro}{\si@tab@expout}
% Various storage macros are needed.
%    \begin{macrocode}
\newcommand*{\si@num@in}{}
\newcommand*{\si@num@out}{}
\newcommand*{\si@num@exp}{}
\newcommand*{\si@num@expsign}{}
\newcommand*{\si@num@mant}{}
\newcommand*{\si@num@mantsign}{}
\newcommand*{\si@num@err}{}
\newcommand*{\si@num@xpart}{}
\newcommand*{\si@num@ambig}{}
\newcommand*{\si@tab@out}{}
\newcommand*{\si@tab@expout}{}
%    \end{macrocode}
%\end{macro}
%\end{macro}
%\end{macro}
%\end{macro}
%\end{macro}
%\end{macro}
%\end{macro}
%\end{macro}
%\end{macro}
%\end{macro}
%\end{macro}
%\begin{macro}{\ifsi@num@erropen}
% A flag is set up for tracking unclosed errors.
%    \begin{macrocode}
\newif\ifsi@num@erropen
%    \end{macrocode}
%\end{macro}
%\begin{macro}{\si@num@format}
%\begin{macro}{\si@num@arg}
%\darg{number}
% The number processor starts by saving |#1| (odd things happen
% otherwise).  A hook is also provided to allow modifications by
% other macros.
%    \begin{macrocode}
\newcommand*{\si@num@arg}{}
\newcommand*{\si@num@format}[1]{%
  \protected@edef\si@num@arg{#1}%
  \si@log@debug{Formatting number `\si@num@arg'}%
%    \end{macrocode}
%\end{macro}
% The storage areas are emptied.
%    \begin{macrocode}
  \renewcommand*{\si@num@in}{}%
  \renewcommand*{\si@num@exp}{}%
  \renewcommand*{\si@num@expsign}{}%
  \renewcommand*{\si@num@mant}{}%
  \renewcommand*{\si@num@mantsign}{}%
  \renewcommand*{\si@num@err}{}%
  \renewcommand*{\si@num@xpart}{}%
%    \end{macrocode}
% Any ``x-part'' is now found, leaving the first number in
% \cs{si@num@in} and anything else in \cs{si@num@xpart}.
%    \begin{macrocode}
  \si@switchfalse
  \expandafter\si@num@findxpart\si@num@arg\@empty\@empty
%    \end{macrocode}
% The input is split into an mantissa and an exponent; the flag is
% used here to indicate if an exponent is found.  The mantissa will
% end up in \cs{si@num@mant}, and the exponent in \cs{si@num@exp}.
%    \begin{macrocode}
  \si@switchfalse
  \si@num@sepmantexp{\si@num@in}%
%    \end{macrocode}
% The sign and value of the mantissa and exponent are separated; the
% mantissa is done after the exponent as this makes life easier when
% using the table-formatting fork. Checks are then needed, as a sign
% with no value is potentially-valid for the mantissa (for example
% \num{-e10}).
%    \begin{macrocode}
  \si@num@sepsign{exp}%
  \si@num@sepsign{mant}%
  \ifx\@empty\si@num@exp\@empty
    \ifx\@empty\si@num@expsign\@empty\else
      \si@log@warn{Sign but no number for `\si@num@arg'}%
    \fi
    \let\si@num@expsign\@empty
  \fi
  \ifx\@empty\si@num@mant\@empty
    \ifx\@empty\si@num@mantsign\@empty\else
      \ifx\@empty\si@num@exp\@empty
        \si@log@warn{Sign but no number for `\si@num@arg'}%
        \let\si@num@mantsign\@empty
      \fi
    \fi
  \fi
%    \end{macrocode}
% A check for negative mantissa values is made, to allow some
% colour-based trickery.
%    \begin{macrocode}
  \renewcommand*{\si@tempa}{{-}}%
  \ifx\si@num@mantsign\si@tempa
    \ifsi@colourneg
      \expandafter\expandafter\expandafter\si@fam@colourcmd
    \else
      \expandafter\expandafter\expandafter\@gobble
    \fi
  \else
    \expandafter\@gobble
  \fi
  {\si@negcolour}%
%    \end{macrocode}
% The next stage is to process the remaining data, to find the
% decimal marker and reformat correctly.  These two macros control
% this entire process.  The exponent is processed first as this makes
% life easier when the system is used to typeset tabular
% material.\footnote{The contents of \cs{si@num@predec} and
% \cs{si@num@postdec} are needed for the mantissa.}
%    \begin{macrocode}
  \si@num@procnum{exp}%
  \si@num@procnum{mant}%
%    \end{macrocode}
% If the exponent is zero, then if might need to be deleted.
%    \begin{macrocode}
  \si@str@ifonlychrs{\si@num@exp}{0\si@numdecimal}
    {\ifsi@allowzeroexp\else
       \renewcommand*{\si@num@exp}{}%
       \ifx\@empty\si@num@mant\@empty
         \renewcommand*{\si@num@mant}{1}%
       \fi
     \fi}{}%
%    \end{macrocode}
% To build up the number for typesetting, a rather complex series
% of tests is needed.  First, the ``ambiguous error'' flag is
% set if there is an exponent and the package has been asked to check
% this.  The same flag will already be true if a unit is present and
% checking is active.
%    \begin{macrocode}
  \ifx\@empty\si@num@exp\@empty\else
    \ifsi@trapambigerr
      \expandafter\expandafter\expandafter\si@num@ambigerrtrue
    \fi
  \fi
%    \end{macrocode}
%\begin{macro}{\si@num@out}
%\begin{macro}{\si@tab@out}
%\begin{macro}{\si@tab@expout}
% Processing now divides, as when used with the |S| column some extra
% steps are needed.  Inside a table, the macro \cs{si@tab@out} holds
% the part of the mantissa before the decimal sign.  The post-decimal
% part then ends up in \cs{i@num@out}.  On the other hand, under
% normal circumstances the entire mantissa should be copied to
% \cs{si@num@out}.
%    \begin{macrocode}
  \protected@edef\si@num@out{%
    \ensuremath{{\si@num@mantsign}}\si@num@mant}%
  \renewcommand*{\si@tempa}{num}%
  \ifsi@num@intab
    \protected@edef\si@tab@out{%
      \ensuremath{{\si@num@mantsign}}\si@num@predec}%
    \protected@edef\si@num@out{\si@num@postdec}%
    \renewcommand*{\si@tempa}{tab}%
  \fi
%    \end{macrocode}
% Now there is the error part to handle.  For tables, a check has to
% be made so the error ends up in the correct part of the number. The
% value of \cs{si@tempa} is used to track this., and so is set up
% here.
%    \begin{macrocode}
  \ifx\@empty\si@num@postdec\@empty\else
    \renewcommand*{\si@tempa}{num}%
  \fi
  \ifx\@empty\si@num@err\@empty\else
%    \end{macrocode}
% If there is an error, and it is begin separated, the problem arises
% of the potential for ambiguous values.  This can only apply outside
% of a table, as \opt{seperr} is disabled in tabular material.
%    \begin{macrocode}
    \ifsi@seperr
      \ifsi@num@ambigerr
        \protected@edef\si@num@out{%
          \ensuremath{\si@openerr}\si@num@out}%
        \si@repeatunitsfalse
        \expandafter\si@num@erropentrue
      \else
%    \end{macrocode}
% If there is an exponential part and an error, errors are being
% separated and ambiguous errors are not trapped, then there is work
% to do.  The exponent part is added to the \emph{error}, and deleted
% from the mantissa if necessary.
%    \begin{macrocode}
        \ifsi@trapambigerr\else
          \ifx\@empty\si@num@exp\@empty\else
            \protected@edef\si@num@err{%
              \si@num@err\expandafter\@car\si@numexp\@nil
              \si@num@expsign\si@num@exp}%
            \ifsi@repeatunits\else
              \renewcommand*{\si@num@exp}{}%
              \renewcommand*{\si@num@expsign}{}%
            \fi
          \fi
        \fi
      \fi
%    \end{macrocode}
% If \opt{seperr} is not in force, the error and mantissa have to be
% recombined.  This is handled so that the same macro deals with
% tables and normal processing.
%    \begin{macrocode}
    \else
      \expandafter\protected@edef\csname
        si@\si@tempa @out\endcsname{%
          \si@num@out\ensuremath{\si@errspace}\ensuremath
          {\si@openerr}\si@num@err\ensuremath{\si@closeerr}}%
      \renewcommand*{\si@num@err}{}%
    \fi
  \fi
%    \end{macrocode}
% The main reconstruction can now occur.  This performs various
% checks on the validity of the input, and adds the necessary
% ``filler'' between the supplied data.
%    \begin{macrocode}
  \renewcommand*{\si@tempa}{num@out}%
  \ifsi@num@erropen
    \renewcommand*{\si@tempa}{num@ambig}%
  \fi
  \ifsi@num@intab
    \renewcommand*{\si@tempa}{tab@expout}%
  \fi
  \ifx\@empty\si@num@exp\@empty
    \ifx\@empty\si@num@mant\@empty
      \si@log@err{Invalid number format `\si@num@arg'}
        {Something is wrong with the number format; does it
         contain \MessageBreak any numbers (from the list
         `\si@numdigits')?}%
      \renewcommand*\si@num@out{}%
    \fi
  \else
    \ifx\@empty\si@num@mant\@empty\else
      \expandafter\protected@edef\csname
        si@\si@tempa\endcsname{%
          \csname si@\si@tempa\endcsname\ensuremath{{}%
          \si@expproduct{}}}%
    \fi
    \expandafter\protected@edef\csname
      si@\si@tempa\endcsname{%
        \csname si@\si@tempa\endcsname\si@expbase
        \textsuperscript{\ensuremath{\si@num@expsign}%
        \si@num@exp}}%
  \fi
%    \end{macrocode}
%\end{macro}
%\end{macro}
%\end{macro}
% With everything done, the result is output.
%    \begin{macrocode}
  \ifsi@num@intab\else
    \expandafter\si@num@out
  \fi
%    \end{macrocode}
% If there is anything inside the ``x-part'', then there is now
% more work to do.
%    \begin{macrocode}
  \ifx\@empty\si@num@err\@empty\else
    \expandafter\si@num@procerr
  \fi
  \ifsi@num@erropen
    \expandafter\si@out@num\expandafter{%
      \ensuremath{\si@closeerr}}%
      \ifx\@empty\si@num@ambig\@empty\else
        \expandafter\si@out@num\expandafter{\si@num@ambig}%
        \renewcommand*{\si@num@ambig}{}%
    \fi
  \fi
  \si@num@erropenfalse
  \ifx\@empty\si@num@xpart\@empty\else
    \expandafter\si@num@sepxpart
  \fi}
%    \end{macrocode}
%\end{macro}
%\begin{macro}{\si@num@findxpart}
% Before processing the number, any multiplied parts have to be found
% and removed.  As there can be more than one product sign, the
% building process here has no error checking.
%    \begin{macrocode}
\def\si@num@findxpart#1#2\@empty{%
  \si@str@ifchrstr{#1}{\si@numprod}
    {\si@switchtrue\si@seperrfalse}{}%
  \ifsi@switch
    \protected@edef\si@num@xpart{\si@num@xpart#1}%
  \else
    \protected@edef\si@num@in{\si@num@in#1}%
  \fi
  \ifx\@empty#2\@empty\else
    \si@num@findxpart#2\@empty
  \fi}
%    \end{macrocode}
%\end{macro}
%\begin{macro}{\si@num@sepmantexp}
%\begin{macro}{\si@num@mantexp}
%\darg{number}
% Splitting the mantissa and exponent first checks for characters to
% gobble, which are simply thrown away.  For any other input, there
% are two possibilities.  If the character is an exponent marker,
% then the package switches from collecting the mantissa to
% collecting the exponent (after a sanity check).  All other
% characters are added to either the mantissa or the exponent, as
% appropriate.
%    \begin{macrocode}
\newcommand*{\si@num@sepmantexp}[1]{%
  \expandafter\si@num@mantexp#1\@empty\@empty}
\def\si@num@mantexp#1#2\@empty{%
  \si@str@ifchrstr{#1}{\si@numgobble}
    {\si@log@debug{Gobbling `#1' in \si@num@arg}}
    {\si@str@ifchrstr{#1}{\si@numexp}
       {\ifsi@switch
          \si@log@err{Duplicate exponent marker found}
            {Only a single exponent character \MessageBreak
            (from the list `\si@numexp')\MessageBreak may
            occur in a numerical argument}%
        \else
          \si@log@debug{Exponent marker `#1' found in
            `\si@num@arg'}%
        \fi
        \si@switchtrue}%
       {\ifsi@switch
          \expandafter\si@num@addexp
        \else
          \expandafter\si@num@addmnt
        \fi
        {#1}}}%
%    \end{macrocode}
% If the recursion has not bottomed out, another loop occurs.
%    \begin{macrocode}
  \ifx\@empty#2\@empty
    \expandafter\@gobble
  \else
    \expandafter\si@num@sepmantexp
  \fi
  {#2}}
%    \end{macrocode}
%\end{macro}
%\end{macro}
%\begin{macro}{\si@num@addmnt}
%\begin{macro}{\si@num@addexp}
%\darg{char}
% To allow \cs{expandafter} use in the above, the actual addition to
% the appropriate macro is handled here.  Two helper macros are
% needed.
%    \begin{macrocode}
\newcommand*{\si@num@addmnt}[1]{%
  \si@num@addmntexp{#1}{mant}{mantissa}}
\newcommand*{\si@num@addexp}[1]{%
  \si@num@addmntexp{#1}{exp}{exponent}}
%    \end{macrocode}
%\end{macro}
%\end{macro}
%\begin{macro}{\si@num@addmntexp}
%\darg{char}
%\darg{either \texttt{mant} or \texttt{exp}}
%\darg{info-text}
% Then the business end.
%    \begin{macrocode}
\newcommand*{\si@num@addmntexp}[3]{%
  \si@log@debug{Adding `#1' to #3 for `\si@num@arg'}%
  \expandafter\protected@edef\csname si@num@#2\endcsname{%
    \csname si@num@#2\endcsname#1}}
%    \end{macrocode}
%\end{macro}
%\begin{macro}{\si@num@sepsign}
%\darg{either \texttt{mant} or \texttt{exp}}
% The input is now tested for a sign.  If one exists, it is
% transferred into the \cs{si@num@}|#1sign| storage macro, with the
% remained of the number in \cs{si@num@}|#1|.  The only check made
% directly here is that there is something to process.
%    \begin{macrocode}
\newcommand*{\si@num@sepsign}[1]{%
  \expandafter\ifx\expandafter\@empty
    \csname si@num@#1\endcsname\@empty
    \expandafter\@gobble
  \else
    \expandafter\si@num@gensign
  \fi
  {#1}}
%    \end{macrocode}
%\end{macro}
%\begin{macro}{\si@num@gensign}
%\darg{either \texttt{mant} or \texttt{exp}}
% The sign generator starts by calling the procedure to check if the
% input contains a valid one- or two-digit sign.  The results are
% returned as \cs{si@num@sign} and \cs{si@num@value}.
%    \begin{macrocode}
\newcommand*{\si@num@gensign}[1]{%
  \expandafter\expandafter\expandafter\si@num@findsign
    \csname si@num@#1\endcsname\@empty\@empty
%    \end{macrocode}
% If no sign has been found, then there may be a need to add one
% anyway.
%    \begin{macrocode}
  \ifx\@empty\si@num@sign\@empty
    \ifx\@empty\si@num@value\@empty
      \expandafter\expandafter\expandafter\@gobble
    \else
      \expandafter\expandafter\expandafter\si@num@addsign
    \fi
  \else
    \expandafter\@gobble
  \fi
  {#1}%
%    \end{macrocode}
% The appropriate storage areas are now assigned.
%    \begin{macrocode}
  \expandafter\let\csname si@num@#1sign\endcsname\si@num@sign
  \expandafter\let\csname si@num@#1\endcsname\si@num@value}
%    \end{macrocode}
%\end{macro}
%\begin{macro}{\si@num@findsign}
%\begin{macro}{\si@num@sign}
%\begin{macro}{\si@num@value}
% The first one or two characters of the mantissa or exponent may
% contain a sign.  To test for this, the first two characters of the
% number are split off, and examined.  Two characters are used so
% that \cs{pm} and \cs{mp} can be represented by |+-| and |-+|,
% respectively. To allow the user to alter the valid signs, but
% retain this conversion, the generic character test is used before
% checking specific matches.
%    \begin{macrocode}
\newcommand*{\si@num@sign}{}
\def\si@num@findsign#1#2#3\@empty{%
  \si@num@delplusfalse
  \si@str@ifchrstr{#1}{\si@numsign}{%
    \si@str@ifchrstr{#2}{\si@numsign}{%
      \if +#1%
        \if -#2%
          \si@log@debug{Found sign combination +- for
            `\si@num@arg'}%
          \renewcommand*{\si@num@sign}{{\si@pm}}%
        \else
          \si@log@inf{Unknown sign combination `#1#2'}%
          \renewcommand*{\si@num@sign}{{#1#2}}%
        \fi
      \else
        \if -#1%
          \if +#2%
            \si@log@debug{Found sign combination -+ for
              `\si@num@arg'}%
            \renewcommand*{\si@num@sign}{{\mp}}%
          \else
            \si@log@inf{Unknown sign combination `#1#2'}%
            \renewcommand*{\si@num@sign}{{#1#2}}%
          \fi
        \else
          \si@log@inf{Unknown sign combination `#1#2'}%
          \renewcommand*{\si@num@sign}{{#1#2}}%
        \fi
      \fi
      \protected@edef\si@num@value{#3}}%
%    \end{macrocode}
% Only one valid sign character.
%    \begin{macrocode}
      {\si@log@debug{Found single sign character `#1' for
         `\si@num@arg'}%
       \renewcommand*{\si@num@sign}{{#1}}%
       \if +#1%
         \ifsi@retainplus\else
           \expandafter\expandafter\expandafter\si@num@killsign
         \fi
       \fi
       \protected@edef\si@num@value{#2#3}}}%
%    \end{macrocode}
% No valid sign, so \cs{@empty} is returned for the sign .
%    \begin{macrocode}
    {\si@log@debug{No sign found for `\si@num@arg'}%
     \renewcommand*{\si@num@sign}{}%
     \protected@edef\si@num@value{#1#2#3}}}
%    \end{macrocode}
%\end{macro}
%\end{macro}
%\end{macro}
%\begin{macro}{\ifsi@num@delplus}
%\begin{macro}{\si@num@killsign}
% A simple spin-out to remove a plus sign.  A flag is set as it might
% be useful to know this.
%    \begin{macrocode}
\newif\ifsi@num@delplus
\newcommand*{\si@num@killsign}{%
  \si@num@delplustrue
  \renewcommand*{\si@num@sign}{}}
%    \end{macrocode}
%\end{macro}
%\end{macro}
%\begin{macro}{\si@num@addsign}
%\begin{macro}{\si@num@asign}
%\darg{either \texttt{mant} or \texttt{exp}}
% The macro to add a sign to an unsigned number has to check whether
% this is a mantissa or an exponent.  The result is still placed in
% \cs{si@num@sign} for ease of processing later.
%    \begin{macrocode}
\newcommand*{\si@num@addsign}[1]{%
  \begingroup
    \renewcommand*{\si@tempa}{#1}%
    \renewcommand*{\si@tempb}{mant}%
    \ifx\si@tempa\si@tempb
      \aftergroup\@firstoftwo
    \else
      \aftergroup\@secondoftwo
    \fi
  \endgroup
  {\ifsi@num@signmant
     \expandafter\si@num@asign
   \else
     \expandafter\@gobble
   \fi
   {mantissa}}
  {\ifsi@num@signexp
     \expandafter\si@num@asign
   \else
     \expandafter\@gobble
   \fi
   {exponent}}}
\newcommand*{\si@num@asign}[1]{%
  \let\si@num@sign\si@sign
  \si@log@debug{Adding sign \si@sign\space to #1 for
    `\si@num@arg'}}
%    \end{macrocode}
%\end{macro}
%\end{macro}
%\begin{macro}{\si@num@procnum}
%\darg{either \texttt{mant} or \texttt{exp}}
% The control macro for processing the number (plus any extra
% characters).
%    \begin{macrocode}
\newcommand*{\si@num@procnum}[1]{%
  \expandafter\ifx\expandafter\@empty
    \csname si@num@#1\endcsname\@empty
    \expandafter\@gobble
  \else
    \expandafter\si@num@finddigits
  \fi
  {#1}}
%    \end{macrocode}
%\end{macro}
%\begin{macro}{\si@num@predec}
%\begin{macro}{\si@num@postdec}
% Two new storage areas are defined.
%    \begin{macrocode}
\newcommand*{\si@num@predec}{}
\newcommand*{\si@num@postdec}{}
%    \end{macrocode}
%\end{macro}
%\end{macro}
%\begin{macro}{\si@num@finddigits}
%\darg{either \texttt{mant} or \texttt{exp}}
% The core digit processor divides the number into the parts before
% and after the decimal point marker.  The temporary switch is used
% to indicate finding a decimal marker.
%    \begin{macrocode}
\newcommand*{\si@num@finddigits}[1]{%
  \renewcommand*{\si@num@predec}{}%
  \renewcommand*{\si@num@postdec}{}%
  \si@switchfalse
  \expandafter\expandafter\expandafter\si@num@digits
    \csname si@num@#1\endcsname\@empty\@empty
%    \end{macrocode}
% Tests are now made to see if padding zeros are needed.  The
% trailing test needs to verify if a decimal marker was found, as
% well as if a zero is needed.
%    \begin{macrocode}
  \ifx\@empty\si@num@predec\@empty
    \ifsi@num@padlead
      \expandafter\expandafter\expandafter\si@num@addprezero
    \fi
  \fi
  \ifx\@empty\si@num@postdec\@empty
    \ifsi@num@padtrail
      \ifsi@switch
        \expandafter\expandafter\expandafter\expandafter
          \expandafter\expandafter\expandafter
          \si@num@addpostzero
      \fi
    \fi
  \fi
%    \end{macrocode}
% The next checks to make concern input validity in a more
% mathematical sense.  First, if the number is zero, then no sign
% should be given under any circumstances.  Then leading zeros need
% to be removed from the input.  This is slightly complicated by the
% potential presence of ``extra'' characters.
%    \begin{macrocode}
  \si@num@unsign{#1}%
  \ifx\@empty\si@num@predec\@empty
  \else
    \expandafter\si@num@nozero
  \fi
%    \end{macrocode}
% A sanity check is made to ensure that the supplied number consisted
% of more than a decimal marker.
%    \begin{macrocode}
  \ifx\@empty\si@num@predec\@empty
    \ifx\@empty\si@num@postdec\@empty
      \expandafter\expandafter\expandafter\@gobble
    \else
      \expandafter\expandafter\expandafter\si@num@sepdigits
    \fi
  \else
    \expandafter\si@num@sepdigits
  \fi
  {#1}}
%    \end{macrocode}
%\end{macro}
%\begin{macro}{\si@num@digits}
% The \cs{si@num@digits} macro compares each character in the input
% against the list of characters valid at this stage: numbers,
% decimal markers and ``extra'' characters.
%    \begin{macrocode}
\def\si@num@digits#1#2\@empty{%
  \si@str@ifchrstr{#1}{\si@numdecimal}
    {\ifsi@switch
       \si@log@err{Duplicate decimal marker in `\si@num@arg'}
         {Only a single decimal marker (from the list
            `\si@numdecimal')\MessageBreak may occur in a
            numerical argument}%
     \else
       \si@log@debug{Found decimal marker `#1' in
         `\si@num@arg'}%
       \expandafter\si@switchtrue
     \fi}
%    \end{macrocode}
% The earlier code only checks for a sign at the start of the text. A
% check is therefore needed for a sign after the first two
% characters; if one is found, it is ignored.
%    \begin{macrocode}
    {\si@str@ifchrstr{#1}{\si@numsign}
       {\si@log@err{Misplaced sign character
          `#1' in `\si@num@arg'}
          {Sign characters `\si@numsign' can only
            occur\MessageBreak at the start of a number}}
%    \end{macrocode}
% The current character is added to the appropriate stack; this is
% ``spun out'' to avoid problems with expansion of the switch code.
%    \begin{macrocode}
       {\ifsi@switch
          \expandafter\si@num@post
        \else
          \expandafter\si@num@pre
        \fi
        {#1}}}%
  \ifx\@empty#2\@empty\else
    \si@num@digits#2\@empty\@empty
  \fi}
%    \end{macrocode}
%\end{macro}
%\begin{macro}{\si@num@pre}
%\begin{macro}{\si@num@post}
%\darg{char}
% Storage of the result is spun out.
%    \begin{macrocode}
\newcommand*{\si@num@pre}[1]{%
  \si@num@prepost{#1}{pre}{integer}}
\newcommand*{\si@num@post}[1]{%
  \si@num@prepost{#1}{post}{decimal}}
%    \end{macrocode}
%\end{macro}
%\end{macro}
%\begin{macro}{\si@num@prepost}
%\darg{char}
%\darg{either \texttt{pre} or \texttt{post}}
%\darg{info-text}
% Then actually stored here.
%    \begin{macrocode}
\newcommand*{\si@num@prepost}[3]{%
  \expandafter\protected@edef\csname si@num@#2dec\endcsname{%
    \csname si@num@#2dec\endcsname#1}%
  \si@log@debug{Adding `#1' to #3 part for `\si@num@arg'}}
%    \end{macrocode}
%\end{macro}
%\begin{macro}{\si@num@addprezero}
%\begin{macro}{\si@num@addpostzero}
% A similar set of macros are used for the padding zeros.
%    \begin{macrocode}
\newcommand*{\si@num@addprezero}{%
  \si@num@addpzero{pre}{leading}}
\newcommand*{\si@num@addpostzero}{%
  \si@num@addpzero{post}{trailing}}
%    \end{macrocode}
%\end{macro}
%\end{macro}
%\begin{macro}{\si@num@addpzero}
%\darg{either \texttt{pre} or \texttt{post}}
% In this case, there is no argument to be passed along.
%    \begin{macrocode}
\newcommand*{\si@num@addpzero}[2]{%
  \si@log@debug{Adding #2 zero for `\si@num@arg'}%
  \@namedef{si@num@#1dec}{0}}
%    \end{macrocode}
%\end{macro}
%\begin{macro}{\si@num@unsign}
%\begin{macro}{\si@num@nosign}
%\darg{either \texttt{mant} or \texttt{exp}}
% The trap for a sign with zero numerical input is made.  First, a
% check is made to see if there is a sign to worry about.  The pre-
% and post-decimal parts are then examined, to see if they contain
% something other than ``0'' or an extra character.
%    \begin{macrocode}
\newcommand*{\si@num@unsign}[1]{%
  \expandafter\ifx\expandafter\@empty
    \csname si@num@#1sign\endcsname\@empty
    \expandafter\@gobble
  \else
    \expandafter\si@num@nosign
  \fi
  {#1}}
\newcommand*{\si@num@nosign}[1]{%
  \begingroup
    \si@switchtrue
    \si@str@ifonlychrs{\si@num@predec\si@num@postdec}{0}
      {\si@switchfalse}{}%
    \ifsi@switch
      \aftergroup\@gobble
    \else
      \aftergroup\@firstofone
    \fi
  \endgroup
  {\si@log@debug{Zero value: removing any sign}%
   \ifsi@ang@sign\else
     \@namedef{si@num@#1sign}{}%
   \fi}}
%    \end{macrocode}
%\end{macro}
%\end{macro}
%\begin{macro}{\si@num@nozero}
%  A very short test for a totally zero pre-decimal component.
%    \begin{macrocode}
\newcommand*{\si@num@nozero}{%
  \si@str@ifonlychrs{\si@num@predec}{0}
    {\renewcommand*{\si@num@predec}{0}}{}}
%    \end{macrocode}
%\end{macro}
%\begin{macro}{\si@num@decimalhook}
% A hook is needed to attach things inside the group to happen
% afterwards, if the number is a decimal.
%    \begin{macrocode}
\newcommand*{\si@num@decimalhook}{}
%    \end{macrocode}
%\end{macro}
%\begin{macro}{\si@num@sepdigits}
%\darg{either \texttt{mant} or \texttt{exp}}
% The \cs{si@num@sepdigits} macro is only called if at least one of
% the mantissa and exponent contain something to output.
%    \begin{macrocode}
\newcommand*{\si@num@sepdigits}[1]{%
%    \end{macrocode}
% First an overall check is needed for additional characters.  By
% altering the contents of \cs{si@numextra}, the same code can be
% shared by two different checks.
%    \begin{macrocode}
  \begingroup
    \let\si@numextra\si@numaddn
    \protected@edef\si@tempa{\si@num@predec\si@num@postdec}%
    \si@num@ifextra{\si@tempa}
      {\aftergroup\@gobble}
      {\aftergroup\@firstofone}%
  \endgroup
%    \end{macrocode}
% Separation of the error in a number from the number itself is only
% attempted for the mantissa.
%    \begin{macrocode}
  {\renewcommand*{\si@tempb}{mant}%
   \renewcommand*{\si@tempc}{#1}%
   \ifx\si@tempb\si@tempc
     \expandafter\si@num@checkerr
   \fi}%
%    \end{macrocode}
%\changes{v1.1}{2008/08/22}{Fixed-length code generalised}
%\changes{v1.1}{2008/08/26}{Length-fixing only applies to
%  manitssa}
% If both parts of the number contain only digits, then any rounding
% can be attempted if there is no error part.  Otherwise, the input
% must be left alone.
%    \begin{macrocode}
  \protected@edef\si@tempa{\si@num@predec\si@num@postdec}%
  \expandafter\si@str@ifonlychrs\expandafter{\si@tempa}
    {0123456789}
    {\ifx\@empty\si@num@err\@empty
       \renewcommand*{\si@tempb}{mant}%
       \renewcommand*{\si@tempc}{#1}%
       \ifx\si@tempb\si@tempc
         \expandafter\expandafter\expandafter\si@num@fixlength
       \fi
     \fi}{}%
%    \end{macrocode}
% If the pre-decimal part contains nothing except numbers, then
% digit separation is carried out.
%    \begin{macrocode}
  \si@num@ifextra{\si@num@predec}{}
    {\expandafter\si@num@int\expandafter{\si@num@predec}}%
%    \end{macrocode}
% A decision is made about the decimal sign, then digit separation
% occurs on the post-decimal part of the number.
%    \begin{macrocode}
  \renewcommand*{\si@tempc}{}%
  \ifx\@empty\si@num@postdec\@empty\else
    \si@num@decimalhook
    \renewcommand*{\si@tempc}{%
      \ensuremath{{\si@decimalsymbol}}}%
    \si@num@ifextra{\si@num@postdec}{}
      {\expandafter\si@num@dec\expandafter{\si@num@postdec}}%
  \fi
%    \end{macrocode}
% The construction is finalised by re-combining the number.
%    \begin{macrocode}
  \expandafter\protected@edef\csname si@num@#1\endcsname
    {\si@num@predec\si@tempc\si@num@postdec}}
%    \end{macrocode}
%\end{macro}
%\begin{macro}{\si@num@ifextra}
%\begin{macro}{\si@num@extra}
%\darg{number}
% A relatively simple test for ``extra'' characters.  Once again, a
% bit of group trickery is used.
%    \begin{macrocode}
\newcommand*{\si@num@ifextra}[1]{%
  \begingroup
    \si@switchfalse
    \expandafter\si@num@extra#1\@empty\@empty
    \ifsi@switch
      \si@log@debug{Found `extra' characters in `#1'}%
      \aftergroup\@firstoftwo
    \else
      \aftergroup\@secondoftwo
    \fi
  \endgroup}
\def\si@num@extra#1#2\@empty{%
  \ifx\@empty#1\@empty\else
    \si@str@ifchrstr{#1}{\si@numextra}{\si@switchtrue}{}%
    \ifx\@empty#2\@empty\else
      \si@num@extra#2\@empty\@empty
    \fi
  \fi}
%    \end{macrocode}
%\end{macro}
%\end{macro}
%\begin{macro}{\ifsi@num@ambigerr}
%\begin{macro}{\si@num@checkerr}
% When separating out an error from a number, the first step is to
% see if there is a decimal part to the number.  If so, any error
% must be in that part of the decimal part; it is not possible to
% have an error starting in the integer part and continuing into the
% decimal part.
%    \begin{macrocode}
\newif\ifsi@num@ambigerr
\newcommand*{\si@num@checkerr}{%
  \ifx\@empty\si@num@postdec\@empty
    \expandafter\si@num@preerr
  \else
    \expandafter\si@num@posterr
  \fi}
%    \end{macrocode}
%\end{macro}
%\end{macro}
%\begin{macro}{\si@num@preerr}
% Errors in integers are easy to handle.  After finding the error,
% the result is simply stored in the error macro \cs{si@num@err}.
%    \begin{macrocode}
\newcommand*{\si@num@preerr}{%
  \si@num@seperr{pre}%
  \ifx\@empty\si@tempb\@empty\else
    \expandafter\renewcommand\expandafter*\expandafter
      \si@num@err\expandafter{\si@tempb}%
  \fi}
%    \end{macrocode}
%\end{macro}
%\begin{macro}{\si@num@posterr}
%\begin{macro}{\si@num@psterr}
% Life is more complex for an error in the decimal part.  This is
% found, then it may need zero-filling or the insertion of a decimal
% point.  This depends on whether the number of error digits is
% larger than the number of post-decimal digits.
%    \begin{macrocode}
\newcommand*{\si@num@posterr}{%
  \si@num@seperr{post}%
  \ifx\@empty\si@tempb\@empty\else
    \ifsi@seperr
      \expandafter\expandafter\expandafter\si@num@psterr
    \else
      \let\si@num@err\si@tempb
    \fi
  \fi}
\newcommand*{\si@num@psterr}{%
    \si@num@cntdigits{\si@tempb}%
    \si@tempcntb\si@tempcnta\relax
    \si@num@cntdigits{\si@num@postdec}%
    \ifnum\si@tempcnta<\si@tempcntb\relax
      \expandafter\si@num@largeerr
    \else
      \expandafter\si@num@smallerr
    \fi}
%    \end{macrocode}
%\end{macro}
%\end{macro}
%\begin{macro}{\si@num@seperr}
%\begin{macro}{\si@num@finderr}
%\darg{either \texttt{pre} or \texttt{post}}
% The usual hand-off is made for searching for an error.
%    \begin{macrocode}
\newcommand*{\si@num@seperr}[1]{%
  \si@switchfalse
  \renewcommand*{\si@tempa}{}%
  \renewcommand*{\si@tempb}{}%
  \expandafter\expandafter\expandafter\si@num@finderr
    \csname si@num@#1dec\endcsname\@empty\@empty
  \ifx\@empty\si@tempb\@empty\else
    \expandafter\let\csname si@num@#1dec\endcsname\si@tempa
  \fi}
\def\si@num@finderr#1#2\@empty{%
%    \end{macrocode}
% First a check for the opening character of an error.
%    \begin{macrocode}
  \si@str@ifchrstr{#1}{\si@numopenerr}
%    \end{macrocode}
% If the switch is set when an error is found, then something is
% wrong.
%    \begin{macrocode}
    {\ifsi@switch
       \si@log@err{Invalid error in number}
         {The numerical argument \si@num@arg\space has two (or
           more)\MessageBreak error-opening characters}%
     \else
       \expandafter\si@switchtrue
     \fi}
%    \end{macrocode}
% The end-of-error character has to be the last item of the input,
% and an error needs to start before it ends.
%    \begin{macrocode}
    {\si@str@ifchrstr{#1}{\si@numcloseerr}
       {\ifsi@switch
          \ifx\@empty#2\@empty\else
            \si@log@err{Invalid error in number}
              {The numerical argument \si@num@arg\space has an
              error-closing before the last character}%
          \fi
        \else
          \si@log@err{Invalid error in number}
           {The numerical argument \si@num@arg\space has an
              error-closing character\MessageBreak but no
              error-opening one}%
        \fi}
%    \end{macrocode}
% The input must be a digit.  It is stored in the appropriate area.
%    \begin{macrocode}
       {\ifsi@switch
          \expandafter\si@num@addtmpb
        \else
          \expandafter\si@num@addtmpa
        \fi
        {#1}}}%
  \ifx\@empty#2\@empty\else
    \si@num@finderr#2\@empty
  \fi}
%    \end{macrocode}
%\end{macro}
%\end{macro}
%\begin{macro}{\si@num@addtmpa}
%\begin{macro}{\si@num@addtmpb}
%\darg{number}
% Some quick methods for adding to the temporary macros with \cs{if}
% expansion.
%    \begin{macrocode}
\newcommand*{\si@num@addtmpa}[1]{\si@num@addtmp{a}{#1}}
\newcommand*{\si@num@addtmpb}[1]{\si@num@addtmp{b}{#1}}
%    \end{macrocode}
%\end{macro}
%\end{macro}
%\begin{macro}{\si@num@addtmp}
%\darg{either \texttt{a} or \texttt{b}}
%\darg{number}
% The addition is committed.
%    \begin{macrocode}
\newcommand*{\si@num@addtmp}[2]{%
  \expandafter\protected@edef\csname si@temp#1\endcsname{%
    \csname si@temp#1\endcsname#2}}
%    \end{macrocode}
%\end{macro}
%\begin{macro}{\si@num@cntdigits}
%\begin{macro}{\si@num@cntdgt}
%\darg{number}
% A recursive system for counting the number of characters in a given
% input; only used here to count digits in a number.
%    \begin{macrocode}
\newcommand*{\si@num@cntdigits}[1]{%
  \si@tempcnta\z@\relax
  \expandafter\si@num@cntdgt#1\@empty\@empty}
\def\si@num@cntdgt#1#2\@empty{%
  \ifx\@empty#1\@empty\else
    \advance\si@tempcnta\@ne\relax
  \fi
  \ifx\@empty#2\@empty\else
    \expandafter\si@num@cntdgt#2\@empty
  \fi}
%    \end{macrocode}
%\end{macro}
%\end{macro}
%\begin{macro}{\si@num@smallerr}
%\begin{macro}{\si@num@serr}
% For handling an error with no more digits than the post-decimal
% part of a number, zero-padding is undertaken before adding a
% decimal sign and (possibly) leading zero.
%    \begin{macrocode}
\newcommand*{\si@num@smallerr}{%
  \si@tempcntb\si@tempcnta\relax
  \si@num@serr
  \protected@edef\si@num@err{%
    \ifsi@num@padlead0\fi\expandafter\@car\si@numdecimal\@nil
    \si@tempb}}
\newcommand*{\si@num@serr}{%
  \si@num@cntdigits{\si@tempb}%
  \ifnum\si@tempcnta=\si@tempcntb\relax\else
    \protected@edef\si@tempb{0\si@tempb}%
    \expandafter\si@num@serr
  \fi}
%    \end{macrocode}
%\end{macro}
%\end{macro}
%\begin{macro}{\si@num@largeerr}
%\changes{v1.0m}{2008/09/16}{Fixed problem with error separation
%  crossing decimal boundary}
%\begin{macro}{\si@num@lerr}
%\begin{macro}{\si@num@movedigit}
% When the error has more digits than the decimal part, some
% shuffling is needed.
%    \begin{macrocode}
\newcommand*{\si@num@largeerr}{%
  \renewcommand*{\si@tempa}{}%
  \si@tempcntb\si@tempcnta\relax
  \si@num@lerr
  \protected@edef\si@num@err{%
    \si@tempa\expandafter\@car\si@numdecimal\@nil\si@tempb}}
\newcommand*{\si@num@lerr}{%
  \si@num@cntdigits{\si@tempb}%
  \ifnum\si@tempcnta=\si@tempcntb\relax\else
    \expandafter\si@num@movedigit\si@tempb\@empty\@empty
    \si@num@lerr
  \fi}
\def\si@num@movedigit#1#2\@empty{%
  \protected@edef\si@tempa{\si@tempa#1}%
  \protected@edef\si@tempb{#2}}
%    \end{macrocode}
%\end{macro}
%\end{macro}
%\end{macro}
%\begin{macro}{\si@num@fixlength}
% The checks for fixing number length are spun out here.
%    \begin{macrocode}
\newcommand*{\si@num@fixlength}{%
  \ifsi@fixdp
    \expandafter\si@num@fixdp
  \else
    \ifsi@fixsf
      \expandafter\expandafter\expandafter\si@num@fixsf
    \fi
  \fi}
%    \end{macrocode}
%\end{macro}
%\begin{macro}{\si@num@fixdp}
% The test for fixing decimal places starts by counting up the number
% of decimal digits.  The number is then padded, rounded or left
% alone.
%    \begin{macrocode}
\newcommand*{\si@num@fixdp}{%
  \si@num@cntdigits{\si@num@postdec}%
  \ifx\@empty\si@num@postdec\@empty
    \si@tempcnta\z@\relax
  \fi
  \ifnum\si@tempcnta>\si@num@dp\relax
    \expandafter\si@num@round
  \else
    \ifnum\si@tempcnta<\si@num@dp\relax
      \expandafter\expandafter\expandafter\si@num@pad
    \fi
  \fi}
%    \end{macrocode}
%\end{macro}
%\begin{macro}{\si@num@pad}
%\begin{macro}{\si@num@pd}
% Padding a number is a relatively easy matter.  The number of digits
% is increased by adding ``$0$'' recursively.
%    \begin{macrocode}
\newcommand*{\si@num@pad}{%
  \si@log@debug{Padding to \the\si@num@dp\space digits}%
  \loop\ifnum\si@tempcnta<\si@num@dp\si@num@pd\repeat}
\newcommand*{\si@num@pd}{%
  \advance\si@tempcnta\@ne\relax
  \protected@edef\si@num@postdec{\si@num@postdec0}}
%    \end{macrocode}
%\end{macro}
%\end{macro}
%\begin{macro}{\si@num@rndtarget}
%\changes{v1.1}{2008/08/25}{New counter}
% A scratch counter is created to allow some code sharing between
% fixing decimal places and significant figures.
%    \begin{macrocode}
\newcount\si@num@rndtarget
%    \end{macrocode}
%\end{macro}
%\begin{macro}{\si@num@round}
%\begin{macro}{\si@num@prernd}
%\begin{macro}{\si@num@postrnd}
%\begin{macro}{\si@num@rndstart}
%\changes{v1.1}{2008/08/25}{Separated out from \cs{si@num@round}}
% Rounding a number is more complicated.  The main macro here relies
% on several others, and is simply a set-up and hand-off system.
%    \begin{macrocode}
\newcommand*{\si@num@prernd}{}
\newcommand*{\si@num@postrnd}{}
\newcommand*{\si@num@round}{%
  \si@log@debug{Rounding to \the\si@num@dp\space digits}%
  \si@num@rndtarget\si@num@dp\relax
  \si@num@rndstart}
\newcommand*{\si@num@rndstart}{%
  \si@num@reverse{\si@num@postdec}%
  \si@num@reverse{\si@num@predec}%
  \let\si@num@prernd\si@num@predec
  \let\si@num@postrnd\si@num@postdec
  \renewcommand*{\si@num@predec}{}%
  \renewcommand*{\si@num@postdec}{}%
  \si@switchfalse
  \si@num@rnd}
%    \end{macrocode}
%\end{macro}
%\end{macro}
%\end{macro}
%\end{macro}
%\begin{macro}{\si@num@reverse}
%\begin{macro}{\si@num@rev}
%\darg{storage-macro}
% The initial stage is to reverse the entire number, to make life
% easier.  This is then undone by the other macros as the output is
% constructed.
%    \begin{macrocode}
\newcommand*{\si@num@reverse}[1]{%
  \renewcommand*{\si@tempa}{}%
  \expandafter\si@num@rev#1\@empty\@empty
  \let#1\si@tempa}
\def\si@num@rev#1#2\@empty{%
  \edef\si@tempa{#1\si@tempa}%
  \ifx\@empty#2\@empty\else
    \si@num@rev#2\@empty\@empty
  \fi}
%    \end{macrocode}
%\end{macro}
%\end{macro}
%\begin{macro}{\si@num@rnd}
%\changes{v1.1}{2008/08/25}{Altered check for ending to generalise
%  code}
%\begin{macro}{\si@num@rndpre}
%\begin{macro}{\si@num@rndpost}
%\changes{v1.0b}{2008/06/23}{Corrected bug in rounding code}
% The core looping macro of the system simply divides the flow
% between rounding before and after the decimal.  The two routines
% are quite similar, but have subtly different requirements.  Both
% take one character at a time from the input, increment if there is
% a carry digit, check its value, and add to the output string. This
% is very similar to the same routines used in other macro packages
% to achieve the same thing.
%    \begin{macrocode}
\newcommand*{\si@num@rnd}{%
  \ifx\@empty\si@num@postrnd\@empty
    \expandafter\si@num@rndpre
  \else
    \expandafter\si@num@rndpost
  \fi}
\newcommand*{\si@num@rndpre}{%
  \expandafter\edef\expandafter\si@tempa\expandafter{%
    \expandafter\@car\si@num@prernd\@nil}%
  \expandafter\edef\expandafter\si@num@prernd\expandafter{%
    \expandafter\@cdr\si@num@prernd\@nil}%
  \si@tempcntb\si@tempa\relax
  \ifsi@switch
    \advance\si@tempcntb\@ne\relax
  \fi
  \si@switchfalse
  \ifnum\si@tempcntb=10\relax
    \si@tempcntb\z@\relax
    \expandafter\expandafter\expandafter\si@switchtrue
  \fi
  \edef\si@num@predec{\the\si@tempcntb\si@num@predec}%
  \ifx\@empty\si@num@prernd\@empty
    \ifsi@switch
      \edef\si@num@predec{1\si@num@predec}%
    \fi
  \else
    \expandafter\si@num@rnd
  \fi}
\newcommand*{\si@num@rndpost}{%
  \expandafter\edef\expandafter\si@tempa\expandafter{%
    \expandafter\@car\si@num@postrnd\@nil}%
  \expandafter\edef\expandafter\si@num@postrnd\expandafter{%
    \expandafter\@cdr\si@num@postrnd\@nil}%
  \si@tempcntb\si@tempa\relax
  \ifsi@switch
    \advance\si@tempcntb\@ne\relax
  \fi
  \si@switchfalse
  \advance\si@num@rndtarget\@ne\relax
  \ifnum\si@tempcnta>\si@num@rndtarget\relax
    \advance\si@num@rndtarget\m@ne\relax
  \else
    \advance\si@num@rndtarget\m@ne\relax
    \ifnum\si@tempcnta>\si@num@rndtarget\relax
      \ifnum\si@tempcntb>4\relax
        \expandafter\expandafter\expandafter\si@switchtrue
      \fi
    \else
      \ifnum\si@tempcntb=10\relax
        \si@tempcntb\z@\relax
        \expandafter\expandafter\expandafter\si@switchtrue
      \fi
      \edef\si@num@postdec{\the\si@tempcntb\si@num@postdec}%
    \fi
  \fi
  \advance\si@tempcnta\m@ne\relax
%    \end{macrocode}
% If \cs{si@tempcnta} is negative, this must be the significant
% figures routine and one digit needs to be thrown away.
%    \begin{macrocode}
  \ifnum\si@tempcnta=\m@ne\relax
    \si@num@throw
  \fi
  \si@num@rnd}
%    \end{macrocode}
%\end{macro}
%\end{macro}
%\end{macro}
%\begin{macro}{\si@num@throw}
% Throw one digit away from the end of the post-decimal part.
%    \begin{macrocode}
\newcommand*{\si@num@throw}{%
  \si@num@reverse{\si@num@postdec}%
  \edef\si@num@postdec{\expandafter\@cdr\si@num@postdec\@nil}%
  \si@num@reverse{\si@num@postdec}}
%    \end{macrocode}
%\end{macro}
%\begin{macro}{\si@num@fixsf}
% Setting up a fixed number of significant figures is rather more
% complicated than fixing the decimal places.  There are a number of
% tests needed to see what needs to be done.
%    \begin{macrocode}
\newcommand*{\si@num@fixsf}{%
  \si@num@cntdigits{\si@num@predec}%
  \ifx\@empty\si@num@predec\@empty
    \si@tempcnta\z@\relax
  \fi
%    \end{macrocode}
% If the pre-decimal component has only a single digit, it could be
% ``0'': if so, only the significant figures in the decimal part are
% counted.
%    \begin{macrocode}
  \ifnum\si@tempcnta=\@ne\relax
    \renewcommand*{\si@tempa}{0}%
    \ifx\si@num@predec\si@tempa
%    \end{macrocode}
% The number of significant digits after the decimal point have to be
% counted up. If there are enough digits there, then some rounding
% will again be needed.
%    \begin{macrocode}
      \ifx\@empty\si@num@postdec\@empty\else
        \si@tempcnta\z@\relax
        \si@switchfalse
        \expandafter\si@num@cntpostsf\si@num@postdec
          \@empty\@empty
        \ifnum\si@tempcnta>\si@num@sf\relax
          \si@log@debug{Rounding to \the\si@num@sf\space
            figures}%
          \si@num@rndtarget\si@num@sf\relax
          \si@num@rndstart
        \fi
      \fi
    \else
      \si@num@cntprepost
    \fi
  \else
%    \end{macrocode}
% If there are too many digits in the pre-decimal part, a check is
% first made to see if there may be a need to round from the first
% post-decimal digit. If not, the post decimal part is thrown away.
%    \begin{macrocode}
    \ifnum\si@tempcnta<\si@num@sf\relax
      \si@num@cntprepost
    \else
      \advance\si@tempcnta\m@ne\relax
      \si@switchfalse
      \ifnum\si@tempcnta>\si@num@sf\relax\else
        \ifx\@empty\si@num@postdec\@empty\else
          \edef\si@num@postdec{\expandafter\@car
            \si@num@postdec\@nil}%
          \ifnum\si@num@postdec>4\relax
            \expandafter\si@switchtrue
          \fi
        \fi
      \fi
      \advance\si@tempcnta\@ne\relax
      \renewcommand*{\si@num@postdec}{}%
      \si@num@reverse{\si@num@predec}%
      \let\si@num@prernd\si@num@predec
      \renewcommand*{\si@num@predec}{}%
      \si@num@sfpre
      \ifsi@switch
        \edef\si@num@predec{1\si@num@predec}%
      \fi
    \fi
  \fi}
%    \end{macrocode}
%\end{macro}
%\begin{macro}{\si@num@cntpostsf}
% Counting up significant figures in the post-decimal part needs an
% auxiliary macro.
%    \begin{macrocode}
\def\si@num@cntpostsf#1#2\@empty{%
  \ifsi@switch
    \advance\si@tempcnta\@ne\relax
  \else
    \if 0#1\relax\else
      \si@tempcnta\@ne\relax
      \expandafter\expandafter\expandafter\si@switchtrue
    \fi
  \fi
  \ifx\@empty#2\@empty\else
    \si@num@cntpostsf#2\@empty\@empty
  \fi}
%    \end{macrocode}
%\end{macro}
%\begin{macro}{\si@num@cntprepost}
% There are two routes to find that the number of digits in both
% parts of the number are important.  In either case, rounding is
% only needed if the two parts have too many digits.
%    \begin{macrocode}
\newcommand*{\si@num@cntprepost}{%
  \si@tempcntb\si@tempcnta\relax
  \si@num@cntdigits{\si@num@postdec}%
  \advance\si@tempcnta\si@tempcntb\relax
  \ifnum\si@tempcnta>\si@num@sf
    \si@log@debug{Rounding to \the\si@num@sf\space figures}%
    \si@num@rndtarget\si@num@sf\relax
    \si@num@rndstart
    \ifsi@switch
      \si@num@throw
    \fi
  \fi}
%    \end{macrocode}
%\end{macro}
%\begin{macro}{\si@num@sfpre}
% The code here is very similar to that used to round to a fixed
% number of decimal places.
%    \begin{macrocode}
\newcommand*{\si@num@sfpre}{%
  \expandafter\edef\expandafter\si@tempa\expandafter{%
    \expandafter\@car\si@num@prernd\@nil}%
  \expandafter\edef\expandafter\si@num@prernd\expandafter{%
    \expandafter\@cdr\si@num@prernd\@nil}%
  \si@tempcntb\si@tempa\relax
  \ifsi@switch
    \advance\si@tempcntb\@ne\relax
  \fi
  \si@switchfalse
  \advance\si@num@sf\@ne\relax
  \ifnum\si@tempcnta>\si@num@sf\relax
    \advance\si@num@sf\m@ne\relax
    \edef\si@num@predec{0\si@num@predec}%
  \else
    \advance\si@num@sf\m@ne\relax
    \ifnum\si@tempcnta>\si@num@sf\relax
      \ifnum\si@tempcntb>4\relax
        \expandafter\expandafter\expandafter\si@switchtrue
      \fi
      \edef\si@num@predec{0\si@num@predec}%
    \else
      \ifnum\si@tempcntb=10\relax
        \si@tempcntb\z@\relax
        \expandafter\expandafter\expandafter\si@switchtrue
      \fi
      \edef\si@num@predec{\the\si@tempcntb\si@num@predec}%
    \fi
  \fi
  \advance\si@tempcnta\m@ne\relax
  \ifx\@empty\si@num@prernd\@empty\else
    \expandafter\si@num@sfpre
  \fi}
%    \end{macrocode}
%\end{macro}
%
%\begin{macro}{\si@num@int}
%\darg{integer-part}
% The formatting code for separating thousands is taken more-or-less
% directly from \SIstyle.  A few changes are made to fit the various
% conventions here.  Following on from the code above, \cs{si@tempa}
% is used to store the integer part of the number, and \cs{si@tempb}
% is used for the decimal part.
%    \begin{macrocode}
\newcommand*{\si@num@int}[1]{%
  \renewcommand*{\si@num@predec}{}%
  \ifsi@sepfour
    \si@num@intfmt{}#1\@empty\@empty\@empty
  \else
    \si@num@iffive{#1}
      {\si@num@intfmt{}#1\@empty\@empty\@empty}
      {\renewcommand*{\si@num@predec}{#1}}%
  \fi}
%    \end{macrocode}
%\end{macro}
%\begin{macro}{\si@num@iffive}
%\begin{macro}{\si@num@five}
%\darg{number}
% A test is needed for the presence of more than four characters.
%    \begin{macrocode}
\newcommand*{\si@num@iffive}[1]{%
  \si@num@five#1\@empty\@empty\@empty\@empty\@empty\end}
\def\si@num@five#1#2#3#4#5\end{%
   \ifx\@empty#5\@empty
      \expandafter\@secondoftwo
   \else
      \expandafter\@firstoftwo
   \fi}
%    \end{macrocode}
%\end{macro}
%\end{macro}
%\begin{macro}{\si@num@intfmt}
%\begin{macro}{\si@num@fiint}
% The business end of the integer formatter.
%    \begin{macrocode}
\newcommand*{\si@num@intfmt}[4]{%
  \ifx\@empty#2\@empty
    \si@num@intsep#1\relax
  \else
    \ifx\@empty#3\@empty
      \si@num@intsep\@empty\@empty#1#2\relax
    \else
      \ifx\@empty#4\@empty
        \si@num@intsep\@empty#1#2#3\relax
      \else
        \si@num@fiint{#1#2#3#4}%
      \fi
    \fi
  \fi}
\def\si@num@fiint#1\fi\fi\fi{\fi\fi\fi\si@num@intfmt{#1}}
%    \end{macrocode}
%\end{macro}
%\end{macro}
%\begin{macro}{\si@num@intsep}
% For adding separation to integers, an extra function is needed.
%    \begin{macrocode}
\newcommand*{\si@num@intsep}[4]{%
  \protected@edef\si@num@predec{\si@num@predec#1#2#3}%
  \if\relax#4\relax\else
    \protected@edef\si@num@predec{%
      \si@num@predec\ensuremath{\noexpand\si@digitsep}}%
    \expandafter\si@num@intsep\expandafter#4%
  \fi}
%    \end{macrocode}
%\end{macro}
%\begin{macro}{\si@num@dec}
%\begin{macro}{\si@num@decfmt}
%\darg{decimal-part}
% Formatting a decimal uses a similar mechanism, but with a few
% alterations needed.
%    \begin{macrocode}
\newcommand*{\si@num@dec}[1]{%
  \renewcommand*{\si@num@postdec}{}%
  \ifsi@sepfour
    \si@num@decfmt#1\@empty\@empty\@empty\@empty
  \else
    \si@num@iffive{#1}
      {\si@num@decfmt#1\@empty\@empty\@empty\@empty}
      {\protected@edef\si@num@postdec{\si@num@postdec#1}}%
  \fi}
\newcommand*{\si@num@decfmt}[4]{%
  \protected@edef\si@num@postdec{\si@num@postdec#1#2#3}%
  \ifx\@empty#4\@empty%
  \else
    \protected@edef\si@num@postdec{%
      \si@num@postdec\ensuremath{\noexpand\si@digitsep}}%
    \expandafter\si@num@decfmt\expandafter#4%
  \fi}
%    \end{macrocode}
%\end{macro}
%\end{macro}
%\begin{macro}{\si@num@procerr}
% Any error is recycled to to formatted correctly.  The $\pm$ sign,
% and any unit, are also added.
%    \begin{macrocode}
\newcommand*{\si@num@procerr}{%
  \si@num@addunit
  \ensuremath{\si@pm}%
  \expandafter\si@num\expandafter{\si@num@err}}
%    \end{macrocode}
%\end{macro}
%\begin{macro}{\si@num@sepxpart}
% A similar approach for numbers containing products, except the
% first token of the input has to be deleted.
%    \begin{macrocode}
\newcommand*{\si@num@sepxpart}{%
  \si@num@addunit
  \ensuremath{\times}%
  \expandafter\expandafter\expandafter\si@num\expandafter
    \expandafter\expandafter{%
      \expandafter\@cdr\si@num@xpart\@nil}}
%    \end{macrocode}
%\end{macro}
%\begin{macro}{\si@num@addunit}
% A short macro to add units if needed.
%    \begin{macrocode}
\newcommand*{\si@num@addunit}{%
  \si@unt@numtrue
  \ifx\@empty\si@unt@unitarg\@empty\else
    \ifsi@repeatunits
      \si@unt@printunit{\si@unt@unitarg}%
    \fi
  \fi}
%    \end{macrocode}
%\end{macro}
%
%\subsection{Formatting angles}
%\begin{macro}{\ang}
%\doarg{keyval options}
%\darg{angle, either in decimal or deg;min;sec format}
% The approach used here is similar to that in \SIstyle, but has been
% modified in a few ways.
%    \begin{macrocode}
\si@newrobustcmd*{\ang}[2][]{%
  \begingroup
    \sisetup{#1}%
    \si@fam@mode
    \si@log@debug{Processing \string\ang\space input `#2'}%
    \@makeother{\;}%
    \makeatletter
    \scantokens{\si@ang@parse#2;;;\@nil}}
%    \end{macrocode}
%\end{macro}
%\begin{macro}{\si@ang@parse}
% With the correct catcodes in place, processing can take place strip
% out the semi-colons.  Here, the input must either contain no
% semi-colons or two semi-colons.
%    \begin{macrocode}
\def\si@ang@parse#1;#2;#3;#4\@nil{%
  \let\ifsi@ang@fixdp\ifsi@fixdp
  \si@fixdpfalse
  \si@ifmtarg{#4}
    {\si@log@debug{Angle argument contains no
       semi-colons:\MessageBreak decimal angle}%
     \si@ang@dec{#1}{}{}}
    {\si@log@debug{Angle argument contains
       semi-colons:\MessageBreak degree-minute-second angle}%
     \renewcommand*{\si@tempa}{#4}%
     \renewcommand*{\si@tempb}{;;}%
     \ifx\si@tempa\si@tempb\else
       \ifsi@strictarc
         \renewcommand*{\si@tempb}{;}%
         \ifx\si@tempa\si@tempb
           \si@log@err{Insufficient semi-colons in argument
             of \string\ang}{The argument of
             \string\ang\space  must contain either no
             semi-colons or exactly two}%
         \else
           \si@log@err{Excess semi-colons in argument of
             \string\ang}{The argument of \string\ang\space
             must contain either no semi-colons or exactly
             two}%
         \fi
       \fi
     \fi
     \si@ang@arc{#1}{#2}{#3}}}
%    \end{macrocode}
%\end{macro}
%\begin{macro}{\si@ang@dec}
%\begin{macro}{\si@ang@arc}
% Two tests are needed, in case the input format requires conversion.
%    \begin{macrocode}
\newcommand*{\si@ang@dec}{%
  \let\si@ang@fix\@gobble
  \ifsi@ang@toarc
    \expandafter\si@ang@dectoarc
  \else
    \sisetup{padangle=none}\expandafter\si@ang@typeset
  \fi}
\newcommand*{\si@ang@arc}{%
  \let\si@ang@fix\si@ang@arcfix
  \ifsi@ang@todec
    \expandafter\si@ang@arctodec
  \else
    \expandafter\si@ang@typeset
  \fi}
%    \end{macrocode}
%\end{macro}
%\end{macro}
%\begin{macro}{\ifsi@ang@fixdp}
%\begin{macro}{\si@ang@fix}
%\begin{macro}{\si@ang@arcfix}
%\darg{number}
% A check is needed so that rounding and zero filling are only
% applied to the seconds of an arc angle.
%    \begin{macrocode}
\newif\ifsi@ang@fixdp
\newcommand*{\si@ang@fix}[1]{}
\newcommand*{\si@ang@arcfix}[1]{%
  \renewcommand*{\si@tempa}{second}%
  \renewcommand*{\si@tempb}{#1}%
  \ifx\si@tempa\si@tempb
    \ifsi@ang@fixdp
      \expandafter\expandafter\expandafter\si@fixdptrue
    \else
      \expandafter\expandafter\expandafter\si@fixdpfalse
    \fi
  \else
    \expandafter\si@fixdpfalse
  \fi}
%    \end{macrocode}
%\end{macro}
%\end{macro}
%\end{macro}
%\begin{macro}{\si@ang@ifnum}
%\darg{number}
% A test is required to check that the provided data consists of
% numbers which can actually be processed by \TeX.  This is achieved
% by using \cs{si@num@ifvalid} with a fixed list of valid
% characters.
%    \begin{macrocode}
\newcommand*{\si@ang@ifnum}[1]{%
  \begingroup
    \renewcommand*{\si@numvalid}{0123456789,.+-}%
    \ifx\@empty#1\@empty
      \aftergroup\@firstoftwo
    \else
      \si@num@ifvalid{#1}
        {\aftergroup\@firstoftwo}
        {\aftergroup\@secondoftwo}%
    \fi
  \endgroup}
%    \end{macrocode}
%\end{macro}
%\begin{macro}{\si@ang@arctodec}
%\darg{degrees}
%\darg{minutes}
%\darg{seconds}
% The business end of converting arcs to decimal angles is relatively
% straight-forward, as it only needs one calculation.  This has to be
% divided up, so that disaster does not strike if there are empty
% arguments; a check is also needed for the sign of the angle,
% so that the maths makes sense.
%    \begin{macrocode}
\newcommand*{\si@ang@arctodec}[3]{%
  \let\si@ang@fix\@gobble
  \ifnum\si@num@dp>\thr@@\relax
    \si@num@dp\thr@@\relax
  \fi
  \si@fixdptrue
  \si@ang@ifnum{#1}
    {\si@ang@ifnum{#2}
      {\si@ang@ifnum{#3}
        {\si@tempdima\z@\relax
         \renewcommand*{\si@tempa}{+}%
         \ifx\@empty#1\@empty\else
           \si@tempdima #1pt\relax
         \fi
         \ifdim\si@tempdima<\z@\relax
           \renewcommand*{\si@tempa}{-}%
         \fi
         \ifx\@empty#2\@empty\else
           \si@tempdima\dimexpr\si@tempdima\si@tempa
             #2pt/60\relax
         \fi
         \ifdim\si@tempdima<\z@\relax
           \renewcommand*{\si@tempa}{-}%
         \else
         \fi
         \ifx\@empty#3\@empty\else
           \si@tempdima\dimexpr\si@tempdima\si@tempa
             #3pt/3600\relax
         \fi
         \sisetup{numdecimal=.}%
         \expandafter\si@ang@typeset\expandafter{%
           \strip@pt\si@tempdima}{}{}}
        {\si@ang@notnum{#1}{#2}{#3}}}
      {\si@ang@notnum{#1}{#2}{#3}}}
    {\si@ang@notnum{#1}{#2}{#3}}}
%    \end{macrocode}
%\end{macro}
%\begin{macro}{\si@ang@dectoarc}
%\begin{macro}{\si@ang@arcdeg}
%\begin{macro}{\si@ang@arcmin}
%\begin{macro}{\si@ang@arcsec}
%\darg{number}
% The conversion macros for decimal to arc angles.  Life is more
% complex here than above, even without the need for checks on the
% input.  A number of separation steps are needed, each of which
% needs a separate macro.
%    \begin{macrocode}
\newcommand*{\si@ang@dectoarc}[1]{%
  \let\si@ang@fix\si@ang@arcfix
  \si@ang@fixdptrue
  \ifnum\si@num@dp>\@ne\relax
    \si@num@dp\@ne\relax
  \fi
  \si@ang@ifnum{#1}
    {\si@tempdima\z@\relax
     \ifx\@empty#1\@empty\else
       \si@tempdima #1pt\relax
     \fi
     \si@ang@sepint{deg}%
     \si@tempdima\dimexpr\si@tempdima *60\relax
     \si@ang@sepint{min}%
     \edef\si@tempa{\the\dimexpr\si@tempdima *60\relax}%
     \expandafter\newcommand\expandafter*\expandafter{%
       \expandafter\si@ang@arcsec\expandafter}\expandafter{%
       \expandafter\si@ang@strippt\si@tempa}%
%    \end{macrocode}
% A check is made for ``0.0'' seconds, which should be converted to
% simply ``0''.
%    \begin{macrocode}
     \si@tempdima\z@\relax
     \edef\si@tempa{\the\si@tempdima}%
     \expandafter\renewcommand\expandafter*\expandafter{%
       \expandafter\si@tempa\expandafter}\expandafter{%
       \expandafter\si@ang@strippt\si@tempa}%
     \ifx\si@tempa\si@ang@arcsec
       \renewcommand*{\si@ang@arcsec}{0}%
     \fi
%    \end{macrocode}
% To avoid adding zeros where they are not required, each part of the
% angle is now checked.
%    \begin{macrocode}
     \renewcommand*{\si@tempa}{0}%
     \ifx\si@ang@arcdeg\si@tempa
       \si@temptoks{{}}%
     \else
       \si@temptoks{{\si@ang@arcdeg}}%
     \fi
     \ifx\si@ang@arcmin\si@tempa
       \si@temptoks\expandafter{\the\si@temptoks{}}%
     \else
       \si@temptoks\expandafter{\the\si@temptoks{%
         \si@ang@arcmin}}%
     \fi
     \ifx\si@ang@arcsec\si@tempa
       \si@temptoks\expandafter{\the\si@temptoks{}}%
     \else
       \si@temptoks\expandafter{\the\si@temptoks{%
         \si@ang@arcsec}}%
     \fi
     \expandafter\si@ang@typeset\the\si@temptoks}
    {\si@ang@notnum{#1}{}{}}}
%    \end{macrocode}
%\end{macro}
%\end{macro}
%\end{macro}
%\end{macro}
%\begin{macro}{\si@ang@sepint}
%\begin{macro}{\si@ang@sint}
%\begin{macro}{\si@ang@strippt}
%\darg{either \texttt{deg} or \texttt{min}}
% Support macros for the conversion from decimal to arc angles.
%    \begin{macrocode}
\newcommand*{\si@ang@sepint}[1]{%
  \expandafter\si@ang@sint\the\si@tempdima\@empty
  \expandafter\let\csname si@ang@arc#1\endcsname\si@tempa}
\def\si@ang@sint#1.#2\@empty{%
  \renewcommand*{\si@tempa}{#1}%
  \si@tempdima 0.#2\relax}
\begingroup
  \catcode`P=12
  \catcode`T=12
  \lowercase{
    \renewcommand*{\si@tempa}{%
      \def\si@ang@strippt##1PT{##1}}}
  \expandafter\endgroup
\si@tempa
%    \end{macrocode}
%\end{macro}
%\end{macro}
%\end{macro}
%\begin{macro}{\si@ang@notnum}
%\darg{degrees}
%\darg{minutes}
%\darg{seconds}
% Not a \TeX\ number: complain.
%    \begin{macrocode}
\newcommand*{\si@ang@notnum}[3]{%
  \si@log@warn{Angle `#1;#2;#3' is not a pure
    number:\MessageBreak output will be as given}%
  \si@ang@typeset{#1}{#2}{#3}}
%    \end{macrocode}
%\end{macro}
%\begin{macro}{\ifsi@ang@sign}
% A flag is needed to leave signs along for angles, where
% a zero value may still need a sign.
%    \begin{macrocode}
\newif\ifsi@ang@sign
%    \end{macrocode}
%\end{macro}
%\begin{macro}{\si@ang@typeset}
%\darg{degrees}
%\darg{minutes}
%\darg{seconds}
% The \cs{si@ang@set} macro does the work of assigning the degrees,
% minutes and seconds, and actually typesetting the result.
%    \begin{macrocode}
\newcommand*{\si@ang@typeset}[3]{%
%    \end{macrocode}
%\begin{macro}{\si@ang@degs}
%\begin{macro}{\si@ang@mins}
%\begin{macro}{\si@ang@secs}
% First, the three macros that will contain the measures must exist.
%    \begin{macrocode}
  \ifsi@ang@padlarge
    \newcommand*{\si@ang@degs}{0\si@sym@degree}%
    \newcommand*{\si@ang@mins}{0\si@sym@minute}%
    \newcommand*{\si@ang@secs}{0\si@sym@second}%
  \else
    \newcommand*{\si@ang@degs}{}%
    \newcommand*{\si@ang@mins}{}%
    \newcommand*{\si@ang@secs}{}%
  \fi
%    \end{macrocode}
%\begin{macro}{\si@ang@decimalsymbol}
% The current definition of \cs{si@decimalsymbol} needs to be saved.
%    \begin{macrocode}
  \protected@edef\si@ang@decimalsymbol{\si@decimalsymbol}%
%    \end{macrocode}
%\end{macro}
%\begin{macro}{\si@ang@movesign}
% Either the signs need to be moved, or this needs to be killed off.
%    \begin{macrocode}
  \ifsi@astroang
    \let\si@ang@movesign\si@ang@astrosign
  \else
    \let\si@ang@movesign\@gobble
  \fi
%    \end{macrocode}
%\end{macro}
%\begin{macro}{\si@ang@secnum}
%\begin{macro}{\si@ang@minnum}
% The arguments are now examined in reverse order.  If they are
% empty, then nothing is done.  Otherwise, the larger measures are
% zero-filled, if this has been requested. Some steps are needed
% to allow for addition of signs to numbers.
%    \begin{macrocode}
  \newcommand*{\si@ang@secnum}{\si@ang@num{second}}%
  \newcommand*{\si@ang@minnum}{\si@ang@num{minute}}%
   \si@ifnotmtarg{#3}
     {\si@log@debug{Found seconds `#3'}%
      \si@ang@ifnum{#3}
        {\ifdim #3 pt=\z@\relax\else
           \si@ang@signtrue
         \fi}{}%
      \renewcommand*{\si@ang@secs}
       {\si@ang@secnum{#3}\si@sym@second}%
      \renewcommand*{\si@ang@mins}
        {\si@ang@pad{0\si@sym@minute}}%
      \renewcommand*{\si@ang@degs}
        {\si@ang@pad{0\si@sym@degree}}}%
   \si@ifnotmtarg{#2}
     {\si@log@debug{Found minutes `#2'}%
      \si@ang@ifnum{#2}
        {\ifdim #2 pt=\z@\relax\else
           \si@ang@signtrue
         \fi}{}%
      \renewcommand*{\si@ang@secnum}{%
        \si@ang@signlessnum{second}}%
      \renewcommand*{\si@ang@mins}
       {\si@ang@minnum{#2}\si@sym@minute}%
      \renewcommand*{\si@ang@degs}
        {\si@ang@pad{0\si@sym@degree}}}%
   \si@ifnotmtarg{#1}
     {\si@log@debug{Found degrees `#1'}%
      \renewcommand*{\si@ang@secnum}{%
        \si@ang@signlessnum{second}}%
      \renewcommand*{\si@ang@minnum}{%
        \si@ang@signlessnum{minute}}%
      \renewcommand*{\si@ang@degs}
%    \end{macrocode}
% The group here is needed to get the mechanism to move the symbol to
% work properly.
%    \begin{macrocode}
       {\si@ang@num{degree}{#1}%
        \si@sym@degree}}%
   \si@out@num
     {\si@ang@degs\si@anglesep\si@ang@mins\si@anglesep
        \si@ang@secs}%
%    \end{macrocode}
%\end{macro}
%\end{macro}
% The group opened by \cs{ang} is closed.
%    \begin{macrocode}
   \endgroup}
%    \end{macrocode}
%\end{macro}
%\end{macro}
%\end{macro}
%\end{macro}
%\begin{macro}{\si@ang@pad}
%\darg{number}
% Padding is only added if requested; the zero is a literal.
%    \begin{macrocode}
\newcommand*{\si@ang@pad}[1]{\ifsi@ang@padsmall #1\fi}
%    \end{macrocode}
%\end{macro}
%\begin{macro}{\si@ang@num}
%\begin{macro}{\si@ang@signlessnum}
%\darg{one of \texttt{degree}, \texttt{minute} or \texttt{second}}
%\darg{number}
% Modified versions of \cs{num}, one to typeset angles without a
% leading sign and the other with.
%    \begin{macrocode}
\newcommand*{\si@ang@num}[2]{%
  \begingroup
    \si@ang@fix{#1}%
    \si@ang@movesign{#1}%
    \si@num{#2}%
  \endgroup}
\newcommand*{\si@ang@signlessnum}[2]{%
  \begingroup
    \si@ang@fix{#1}%
    \si@ang@movesign{#1}%
    \sisetup{addsign=none}%
    \si@num{#2}%
  \endgroup}
%    \end{macrocode}
%\end{macro}
%\end{macro}
%\begin{macro}{\si@ang@killdegree}
%\begin{macro}{\si@ang@killminute}
%\begin{macro}{\si@ang@killsecond}
% A mechanism is needed to handle moving the angle unit signs for the
% \opt{astroang} option.  First, some support macros are needed.
%    \begin{macrocode}
\newcommand*{\si@ang@killdegree}{\let\si@sym@degree\relax}
\newcommand*{\si@ang@killminute}{\let\si@sym@minute\relax}
\newcommand*{\si@ang@killsecond}{\let\si@sym@second\relax}
%    \end{macrocode}
%\end{macro}
%\end{macro}
%\end{macro}
%\begin{macro}{\si@ang@astrosign}
%\darg{one of \texttt{degree}, \texttt{minute} or \texttt{second}}
% The method needs two steps, producing the sign over the decimal
% sign and preventing duplicate symbols appearing. This is based on a
% suggestion from Morten H{\o}gholm, but using \TeX\ internals as
% \cs{makebox} does not work here.  Note the need to correct for
% \cs{scriptspace} (thanks to Donald Arseneau for that).
%    \begin{macrocode}
\newcommand*{\si@ang@astrosign}[1]{%
  \renewcommand*{\si@decimalsymbol}{%
    \setbox\si@tempboxa=\hbox{%
      \ensuremath{{\si@ang@decimalsymbol}}}%
    \si@tempdima\wd\si@tempboxa\relax
    \setbox\si@tempboxb=\hbox to\z@{%
      \hss\unhbox\si@tempboxa\hss}%
    \setbox\si@tempboxa=\hbox{%
      \csname si@sym@#1\endcsname\hskip-\scriptspace}%
    \si@tempdimb\wd\si@tempboxa\relax
    \setbox\si@tempboxc=\hbox to\z@{%
      \hss\unhbox\si@tempboxa\hss}%
    \setbox\si@tempboxd=\hbox{%
      \usebox\si@tempboxb\usebox\si@tempboxc}%
    \ifdim\si@tempdima>\si@tempdimb\relax
      \setbox\si@tempboxa=\hbox to\si@tempdima{%
        \hss\unhbox\si@tempboxd\hss}%
    \else
      \setbox\si@tempboxa=\hbox to\si@tempdimb{%
        \hss\unhbox\si@tempboxd\hss}%
    \fi
    \usebox\si@tempboxa%
    \ifdim\si@tempdima>\si@tempdimb\relax\else
      \hskip\scriptspace
    \fi}%
  \renewcommand*{\si@num@decimalhook}{\expandafter\aftergroup
    \csname si@ang@kill#1\endcsname}}%
%    \end{macrocode}
%\end{macro}
%
%\subsection{Tabular material}
% The automatic formatting and alignment of numerical data in columns
% is handled here.  The various other packages that work in this area
% are basically ripped-off here.  The letters |D|, |N| and |R| are
% already taken by the other packages for numerical alignment, and so
% |S| ($=$ \currpkg) is chosen for the alignment of numerical
% material.  The package also provides a second column type, |s|, for
% units (the letter is taken from \cs{si}).
%
%\begin{macro}{\NC@list}
%\changes{v1.1e}{2008/11/18}{Fixed issue with \pkg{cellspace} package}
% The first part of the job is to create the basic apparatus for the
% columns using the \pkg{array} package.  To prevent any issues with
% the content of optional arguments to the new columns, so
% rearrangement is needed.  The \currpkg columns have to come
% \emph{before} any other column definitions.  This is achieved by
% saving \cs{NC@list}, creating the columns then restoring the list
% with the appropriate extra parts.
%    \begin{macrocode}
\@ifpackageloaded{cellspace}
  {\newcolumntype{C}[1]{>{\bcolumn #1\@nil}#1<{\ecolumn}}}
  {\AtBeginDocument{
    \@ifpackageloaded{cellspace}
     {\newcolumntype{C}[1]{>{\bcolumn #1\@nil}#1<{\ecolumn}}%
      \renewcommand*{\NC@rewrite@S}[1][]{%
        \edef\si@tempa{\the\@temptokena
        >{\noexpand\si@tab@begin@S[#1]}c%
        <{\noexpand\si@tab@end@S}}%
        \@temptokena\expandafter{\si@tempa}%
        \NC@find}}
     {}}}
\edef\si@tempa{%
  \noexpand\NC@do S\noexpand\NC@do s\the\NC@list}
\newcolumntype{S}{}
\newcolumntype{s}{}
\NC@list\expandafter{\si@tempa}
%    \end{macrocode}
%\end{macro}
%\changes{v1.1h}{2008/11/30}{Fixed problem with \pkg{colortbl} package}
%\changes{v1.1i}{2008/11/30}{Better solution to \pkg{colortbl} issue}
%\begin{macro}{\si@tab@corr@colortbl}
% A correction if \pkg{colortbl} is loaded.
%    \begin{macrocode}
\newcommand*{\si@tab@corr@colortbl}{\relax}
\AtBeginDocument{
  \@ifpackageloaded{colortbl}{%
    \renewcommand*{\si@tab@corr@colortbl}{%
      \hskip\stretch{-0.5}\kern\z@
    }
  }{}
}
%    \end{macrocode}
%\end{macro}
%\begin{macro}{\NC@rewrite@S}
%\begin{macro}{\NC@rewrite@s}
%\doarg{keyval options}
% Following the \numprint approach, the \cs{NC@rewrite@\ldots} macros
% are rewritten to collect the cell contents.  This means messing
% with the internal macros of another package, but there is no other
% way to do this.  As \pkg{array} is a standard package from the
% \pkg{tools} bundle, this should be reasonably safe. Here the begin
% and end code needed is added to the existing list if
% \cs{@temptokena}, with the start and end macros unexpanded.
% Argument |#1| contains any user setup options for this column.
% Passing an argument at this stage will cause issues, so each column
% type needs its own \texttt{begin} and \texttt{end} macros.
%    \begin{macrocode}
\renewcommand*{\NC@rewrite@S}[1][]{%
  \edef\si@tempa{\the\@temptokena
    >{\noexpand\si@tab@begin@S[#1]}c%
    <{\noexpand\si@tab@end@S}}%
  \@temptokena\expandafter{\si@tempa}%
  \NC@find}
\renewcommand*{\NC@rewrite@s}[1][]{%
  \edef\si@tempa{\the\@temptokena
    >{\noexpand\si@tab@begin@s[#1]}c%
    <{\noexpand\si@tab@end@s}}%
  \@temptokena\expandafter{\si@tempa}%
  \NC@find}
%    \end{macrocode}
%\end{macro}
%\end{macro}
%\begin{macro}{\si@tab@begin@S}
%\begin{macro}{\si@tab@begin@s}
%\begin{macro}{\si@tab@gettok}
%\doarg{keyval options}
% At this stage, the appropriate token gathering macro is activated,
% and the common starting macro is called.  For the |S| column, the
% \opt{seperr} is turned off, and an error is set to be raised by any
% ``x-part'' input.
%    \begin{macrocode}
\newcommand*{\si@tab@begin@S}[1][]{%
  \si@log@debug{Processing S column cell contents}%
  \let\si@tab@gettok\si@tab@gettok@S
  \si@seperrfalse
  \renewcommand*{\si@num@sepxpart}{%
    \si@log@err{Multiple numbers not allowed in
       tables\MessageBreak Only the first number used}
       \@ehb}%
  \si@tab@begin[#1]}
\newcommand*{\si@tab@begin@s}[1][]{%
  \si@log@debug{Processing s column cell contents}%
  \let\si@tab@gettok\si@tab@gettok@s
  \si@tab@begin[#1]}
%    \end{macrocode}
%\end{macro}
%\end{macro}
%\end{macro}
%\begin{macro}{\si@tab@toks}
%\begin{macro}{\si@tab@pretoks}
%\begin{macro}{\si@tab@posttoks}
% Some storage is needed for the data to build up.  In common with
% \pkg{rccol} and \numprint, token registers are used for this (thus
% leaving problematic input to be handled later).
%    \begin{macrocode}
\newtoks\si@tab@toks
\newtoks\si@tab@pretoks
\newtoks\si@tab@posttoks
%    \end{macrocode}
%\end{macro}
%\end{macro}
%\end{macro}
%\begin{macro}{\si@tab@begin}
%\doarg{keyval options}
% The macro for gathering up input is common to both column types. It
% uses the method for \pkg{rccol}; the cell contents are collected by
% a second macro, which then stores all of the data in an appropriate
% token store.
%    \begin{macrocode}
\newcommand*{\si@tab@begin}[1][]{%
  \begingroup
    \sisetup{#1}%
    \si@tab@toks{}%
    \si@tab@pretoks{}%
    \si@tab@posttoks{}%
    \si@switchfalse
    \si@tab@gettok}
%    \end{macrocode}
%\end{macro}
%\begin{macro}{\si@tab@next}
% A macro is needed for recursion when collecting cell contents.
%    \begin{macrocode}
\newcommand*{\si@tab@next}{}
%    \end{macrocode}
%\end{macro}
%\begin{macro}{\si@tab@gettok@S}
%\changes{v1.0j}{2008/08/21}{Token gathering code now checks for
%  expandable content}
%\darg{token}
% The cell-content collection system is inherited from \pkg{rccol}.
% First, the current input is compared with a fixed list of macros
% which need custom handling.
%    \begin{macrocode}
\newcommand*{\si@tab@gettok@S}[1]{%
  \ifx\tabularnewline#1\relax
    \let\si@tab@next\si@tab@newline@S
  \else
    \ifx\end#1\relax
      \let\si@tab@next\end
    \else
      \ifx\si@tab@end@S#1\relax
        \let\si@tab@next\si@tab@end@S
      \else
        \ifx\endtabular#1\relax
          \let\si@tab@next\endtabular
        \else
          \ifx\csname#1\relax
            \let\si@tab@next\csname
          \else
            \ifx\relax#1\relax
              \let\si@tab@next\relax
            \else
%    \end{macrocode}
% If the code gets to here, then recursion should occur.
%    \begin{macrocode}
              \let\si@tab@next\si@tab@gettok@S
%    \end{macrocode}
% Cells always contain \cs{ignorespaces} and \cs{unskip}. These can
% both be filtered out here and are saved to either the pre-number or
% post-number token store, as appropriate. This is done without
% fixing the destination in case the user adds addition copies to the
% input.
%    \begin{macrocode}
              \ifx\ignorespaces#1\relax
                \si@tab@othertok{#1}%
              \else
                \ifx\unskip#1\relax
                  \si@tab@othertok{#1}%
                \else
%    \end{macrocode}
% If the current token is a valid numerical character, then the flag
% is set and the input is saved into the ``processing'' token
% register.
%    \begin{macrocode}
                  \si@str@ifchrstr{#1}{\si@numvalid}
                     {\si@switchtrue
                      \si@log@debug{Found numerical cell
                        contents `#1'}%
                      \si@tab@toks=\expandafter{%
                        \the\si@tab@toks#1}}
%    \end{macrocode}
% For non-numerical content, some testing is needed. A check is made
% to see if the current ``token'' is a single item. If it was wrapped
% in one set of braces before being picked up here, they will have
% been removed. So the input may now have more than one character to
% it. If it is a single token, testing for expendability first
% requires a check for another set of braces.  This is handed off
% using \cs{futurelet}.
%    \begin{macrocode}
                     {\si@tab@ifonechar{#1}
                       {\futurelet\si@tempa\si@tab@bracetest#1}%
                       {\si@tab@othertok{#1}}}%
                \fi
              \fi
            \fi
          \fi
        \fi
      \fi
    \fi
  \fi
  \si@tab@next}
%    \end{macrocode}
%\end{macro}
%\begin{macro}{\si@tab@othertok}
%\darg{token}
% Tokens which are not part of a number can occur before or after a
% number. This is all sorted out here, with some debugging
% information for good measure.
%    \begin{macrocode}
\newcommand*{\si@tab@othertok}[1]{%
  \si@log@debug{Found other cell contents `\unexpanded{#1}'}%
  \ifsi@switch
    \si@log@debug{Adding to post-numeral part}%
    \si@tab@posttoks=\expandafter{\the\si@tab@posttoks#1}%
  \else
    \si@log@debug{Adding to pre-numeral part}%
    \si@tab@pretoks=\expandafter{\the\si@tab@pretoks#1}%
  \fi}
%    \end{macrocode}
%\end{macro}
%\begin{macro}{\si@tab@ifonechar}
%\darg{token}
%\begin{macro}{\si@tab@onechar}
% A quick check for a single character.
%    \begin{macrocode}
\newcommand*{\si@tab@ifonechar}[1]{\si@tab@onechar#1\@empty}
\def\si@tab@onechar#1#2\@empty{%
  \ifx\@empty#2\@empty
    \expandafter\@firstoftwo
  \else
    \expandafter\@secondoftwo
  \fi}
%    \end{macrocode}
%\end{macro}
%\end{macro}
%\begin{macro}{\si@tab@bracetest}
% For the \cs{futurelet} test, a separate macro is needed. If the
% current token starts with a brace, then it is added to the
% ``other'' stack.  If not, a test can be made on its definition.
%    \begin{macrocode}
\newcommand*{\si@tab@bracetest}{%
  \ifx\si@tempa\bgroup
    \expandafter\si@tab@othertok
  \else
    \expandafter\si@tab@exptest
  \fi}
%    \end{macrocode}
%\end{macro}
%\begin{macro}{\si@tab@exptest}
%\begin{macro}{\si@tab@meaning}
%\begin{macro}{\si@tab@strip}
%\darg{token}
% The meaning of the token is compared to that for a short macro. If
% the two are of the same type, expansion is attempted. In this case,
% the current pre-number tokens are also typeset.
%    \begin{macrocode}
\newcommand*{\si@tab@exptest}[1]{%
  \edef\si@tempa{\si@tab@meaning{#1}}%
  \ifx\si@tempa\si@tab@macro
    \si@log@debug{Expanding cell contents `\unexpanded{#1}'}%
    \the\si@tab@pretoks
    \si@tab@pretoks{}%
    \protected@edef\si@tab@next{\noexpand\si@tab@gettok@S#1}%
  \else
    \si@tab@othertok{#1}%
  \fi}
\newcommand*{\si@tab@meaning}[1]{%
  \expandafter\si@tab@strip\meaning#1:->:->\@empty}
\def\si@tab@strip#1:->#2:->#3\@empty{#1}
%    \end{macrocode}
%\end{macro}
%\end{macro}
%\end{macro}
%\begin{macro}{\si@tab@macro}
% A supporter macro for the above which always expands to ``macro''
% with everything with category code 12.
%    \begin{macrocode}
\newcommand*{\si@tab@macro}{}
\edef\si@tab@macro{\si@tab@meaning{\si@tab@macro}}
%    \end{macrocode}
%\end{macro}
%\begin{macro}{\si@tab@gettok@s}
%\changes{v1.0a}{2008/06/20}{Fixed issues with alignment of
%  \texttt{s} column contents}
%\darg{token}
% The situation for the |s| column is less complex, as there is only
% one store and expansion is not an issue.
%    \begin{macrocode}
\newcommand*{\si@tab@gettok@s}[1]{%
  \ifx\tabularnewline#1\relax
    \let\si@tab@next\si@tab@newline@s
  \else
    \ifx\end#1\relax
      \let\si@tab@next\end
    \else
      \ifx\si@tab@end@s#1\relax
        \let\si@tab@next\si@tab@end@s
      \else
        \ifx\endtabular#1\relax
          \let\si@tab@next\endtabular
        \else
          \ifx\csname#1\relax
            \let\si@tab@next\csname
          \else
            \ifx\relax#1\relax
              \let\si@tab@next\relax
            \else
              \let\si@tab@next\si@tab@gettok@s
              \ifx\ignorespaces#1\relax\else
                \ifx\unskip#1\relax\else
                  \si@tab@toks=\expandafter{%
                    \the\si@tab@toks#1}%
                  \si@log@debug{Found cell contents `#1'}%
                \fi
              \fi
            \fi
          \fi
        \fi
      \fi
    \fi
  \fi
  \si@tab@next}
%    \end{macrocode}
%\end{macro}
%\begin{macro}{\si@tab@end@S}
%\changes{v1.0a}{2008/06/20}{Fixed mixed number/text column
%  alignment}
%\changes{v1.0k}{2008/09/02}{Fixed problem with coloured cells}
%\begin{macro}{\si@tab@rfill}
%\begin{macro}{\si@tab@lfill}
%\begin{macro}{\si@tab@end@s}
% The end macros are similar, but with some minor differences. In
% both cases, the appropriate filling is carried out.  For the |S|
% column, this depends on the cell contents, whereas for the |s|
% column the fill is always the same.  Output of a number in an |S|
% column is only attempted if one was found, otherwise the cell
% contents will all be in \cs{si@tab@pretoks}.
%    \begin{macrocode}
\newcommand*{\si@tab@end@S}{%
    \ifsi@switch
      \let\si@tab@lfill\si@tab@lfill@S
      \let\si@tab@rfill\si@tab@rfill@S
    \else
      \let\si@tab@rfill\si@tab@rfill@t
      \let\si@tab@lfill\si@tab@lfill@t
    \fi
    \si@tab@corr@colortbl
    \hskip\stretch{\si@tab@lfill}\kern\z@
    \ifsi@switch
      \the\si@tab@pretoks
      \si@tab@numout
      \the\si@tab@posttoks
    \else
      \the\si@tab@pretoks
    \fi
    \hskip\stretch{\si@tab@rfill}\kern\z@
    \si@tab@corr@colortbl
  \endgroup}
\newcommand*{\si@tab@end@s}{%
    \si@tab@corr@colortbl
    \hskip\stretch{\si@tab@lfill@s}\kern\z@ 
    \ignorespaces
    \expandafter\si\expandafter{\the\si@tab@toks}%
    \unskip
    \hskip\stretch{\si@tab@rfill@s}\kern\z@ 
    \si@tab@corr@colortbl
  \endgroup}
%    \end{macrocode}
%\end{macro}
%\end{macro}
%\end{macro}
%\end{macro}
%\begin{macro}{\si@tab@newline@S}
%\begin{macro}{\si@tab@newline@s}
%\begin{macro}{\si@tab@end}
% If the column is the final one read, then some work is needed with
% the \cs{tabularnewline} macro.  Output has to happen \emph{before}
% the new line, then the ending macro is made safe before calling the
% \LaTeX\ line end.  If the user makes use of the primitive \cs{cr}
% then this problem does not arise as \cs{si@tab@end@\ldots} is
% called correctly.
%    \begin{macrocode}
\newcommand*{\si@tab@newline@S}{%
  \si@tab@end@S
  \hfil\relax
  \let\si@tab@end\si@tab@end@S
  \renewcommand*{\si@tab@end@S}{\let\si@tab@end@S\si@tab@end}%
  \tabularnewline}
\newcommand*{\si@tab@newline@s}{%
  \si@tab@end@s
  \hfil\relax
  \let\si@tab@end\si@tab@end@s
  \renewcommand*{\si@tab@end@s}{\let\si@tab@end@s\si@tab@end}%
  \tabularnewline}
%    \end{macrocode}
%\end{macro}
%\end{macro}
%\end{macro}
%
%\begin{macro}{\si@tempcnta}
%\begin{macro}{\si@tempcntb}
% The second part of the tabular code is concerned with typesetting
% numbers in |S| columns with the appropriate alignment. Counters are
% needed for the digit-counting system.
%    \begin{macrocode}
\newcount\si@tempcnta
\newcount\si@tempcntb
%    \end{macrocode}
%\end{macro}
%\end{macro}
%\begin{macro}{\si@tab@numout}
%\changes{v1.1}{2008/09/03}{Extended for non-aligned output}
% If a number is found, then some secondary processing is needed to
% format it correctly.
%    \begin{macrocode}
\newcommand*{\si@tab@numout}{%
  \si@num@intabtrue
  \ifsi@tabparseonly\else
    \ifsi@tab@fixed
      \ifsi@tabautofit
        \si@num@dp\si@tab@mantpostcnt\relax
        \expandafter\expandafter\expandafter\expandafter
          \expandafter\expandafter\expandafter\si@fixdptrue
      \fi
    \fi
  \fi
  \expandafter\si@num\expandafter{\the\si@tab@toks}%
  \si@tab@format}
%    \end{macrocode}
%\end{macro}
%\begin{macro}{\si@tab@prebox}
%\begin{macro}{\si@tab@postbox}
%\begin{macro}{\si@tab@midbox}
%\begin{macro}{\si@tab@expbox}
% The various boxes needed for the column centring are declared
% Unlike the \pkg{dcolumn} original, private boxes are used here.
% \cs{si@tempboxa} is used when a space to measure one of the
% constituents is needed; it is never used for output.
%    \begin{macrocode}
\newbox\si@tab@prebox
\newbox\si@tab@midbox
\newbox\si@tab@postbox
\newbox\si@tab@expbox
%    \end{macrocode}
%\end{macro}
%\end{macro}
%\end{macro}
%\end{macro}
%\begin{macro}{\si@tab@format}
%\changes{v1.1}{2008/09/03}{Extended for non-aligned output}
% The formatting set up is taken from \pkg{dcolumn}, but with
% control of the output form the stored information.  The choice
% of a variable (decimal-centred) column or fixed width boxes is
% made.
%    \begin{macrocode}
\newcommand*{\si@tab@format}{%
  \ifsi@tabparseonly
    \expandafter\si@tab@outonly
  \else
    \ifsi@tab@fixed
      \expandafter\expandafter\expandafter\si@tab@fixed
    \else
      \expandafter\expandafter\expandafter\si@tab@unfixed
    \fi
  \fi
%    \end{macrocode}
% A hack to get the correct colour everywhere without too much work.
%    \begin{macrocode}
  \ifsi@colourvalues
    \si@fam@colourcmd{\si@valuecolour}%
  \fi
  \box\si@tab@prebox\box\si@tab@midbox\box\si@tab@postbox
    \box\si@tab@expbox}
%    \end{macrocode}
%\end{macro}
%\begin{macro}{\si@tab@outonly}
% The non-alignment option means that no real processing is done: only font
% control is required.
%    \begin{macrocode}
\newcommand*{\si@tab@outonly}{%
  \si@log@debug{Using non-aligned S column}%
  \setbox\si@tab@prebox=\hbox{}%
  \setbox\si@tab@midbox=\hbox{}%
  \setbox\si@tab@postbox=\hbox{}%
  \setbox\si@tab@expbox=\hbox{}%
    \protected@edef\si@tempa{%
      \si@tab@out
    \ifx\@empty\si@num@out\@empty\else
      \ensuremath{{\si@decimalsymbol}}%
    \fi
    \si@num@out
    \si@tab@expout}%
  \expandafter\si@out@num\expandafter{\si@tempa}}
%    \end{macrocode}
%\end{macro}
%\begin{macro}{\si@tab@unfixed}
%\changes{v1.0l}{2008/09/03}{Fixed issue with decimal sign and
%  integers}
% When the width of the contents is not fixed, the system creates a
% block in which the decimal marker is always at the centre. This is
% achieved by placing the pre- and post-decimal parts of the number
% in boxes.  The wider one is then used to set up the column width,
% by resizing the other one.
%    \begin{macrocode}
\newcommand*{\si@tab@unfixed}{%
  \si@log@debug{Using variable width S column}%
  \ifx\@empty\si@num@out\@empty
    \setbox\si@tab@midbox=\hbox
      {\phantom{\ensuremath{{\si@decimalsymbol}}}}%
  \else
    \setbox\si@tab@midbox=\hbox
      {\ensuremath{{\si@decimalsymbol}}}%
  \fi
  \protected@edef\si@num@out{\si@num@out\si@tab@expout}%
  \setbox\si@tab@prebox=\hbox
    {\expandafter\si@out@num\expandafter{\si@tab@out}}%
  \setbox\si@tab@postbox=\hbox
    {\expandafter\si@out@num\expandafter{\si@num@out}}%
  \ifdim\wd\si@tab@prebox>\wd\si@tab@postbox\relax
    \setbox\si@tab@postbox=\hbox to\wd\si@tab@prebox%
      {\unhbox\si@tab@postbox\hfill}%
  \else
    \setbox\si@tab@prebox=\hbox to\wd\si@tab@postbox%
      {\hfill\unhbox\si@tab@prebox}%
  \fi
  \setbox\si@tab@expbox=\hbox{}}
%    \end{macrocode}
%\end{macro}
%\begin{macro}{\si@tab@predim}
%\begin{macro}{\si@tab@postdim}
%\begin{macro}{\si@tab@expdim}
%\begin{macro}{\si@tempdima}
%\begin{macro}{\si@tempdimb}
% Some storage dimensions are declared.
%    \begin{macrocode}
\newdimen\si@tab@predim
\newdimen\si@tab@postdim
\newdimen\si@tab@expdim
\newdimen\si@tempdima
\newdimen\si@tempdimb
%    \end{macrocode}
%\end{macro}
%\end{macro}
%\end{macro}
%\end{macro}
%\end{macro}
%\begin{macro}{\si@tab@sp}
% A short macro to control superscript.
%    \begin{macrocode}
\newcommand*{\si@tab@sp}{}
%    \end{macrocode}
%\end{macro}
%\begin{macro}{\si@tab@fixed}
%\changes{v1.0l}{2008/09/03}{Fixed issue with spacing when no digits
%  present after decimal marker}
% The column is not centred on the decimal marker; the user specifies
% how many characters on each side are allowed for.  First, the width
% of a character is measured, and stored.
%    \begin{macrocode}
\newcommand*{\si@tab@fixed}{%
  \si@log@debug{Using fixed-width S column}%
  \let\si@tab@sp\relax
  \setbox\si@tab@midbox=\hbox{}%
  \setbox\si@tab@expbox=\hbox{}%
  \setbox\si@tempboxa=\hbox{\si@out@num{1}}%
  \si@tempdima\wd\si@tempboxa\relax
%    \end{macrocode}
% The width for the two output boxes is set up.
%    \begin{macrocode}
  \si@tab@predim\the\si@tab@mantprecnt\si@tempdima\relax
  \si@tab@sepcorr{mantpre}{pre}%
  \si@tab@postdim\si@tab@mantpostcnt\si@tempdima\relax
  \setbox\si@tempboxa=\hbox{\ensuremath{{\si@decimalsymbol}}}%
  \ifnum\si@tab@mantpostcnt=\z@\relax\else
    \advance\si@tab@postdim\wd\si@tempboxa\relax
  \fi
  \si@tab@sepcorr{mantpost}{post}%
%    \end{macrocode}
% If space is needed for an exponent, it needs to be allowed for in
% the exponent box dimension.  First, the digits of the two parts are
% checked for; the width of a character is altered to be superscript.
%    \begin{macrocode}
  \setbox\si@tempboxa=\hbox{\si@out@num{^{1}}}%
  \si@tempdima\wd\si@tempboxa\relax
  \ifnum\si@tab@expprecnt>\z@\relax
    \si@tab@expdim\si@tab@expprecnt\si@tempdima\relax
    \si@tab@sepcorr{exppre}{exp}%
  \fi
  \let\si@tab@sp\sp
  \ifnum\si@tab@exppostcnt>\z@\relax
    \advance\si@tab@expdim\si@tab@exppostcnt\si@tempdima\relax
    \setbox\si@tempboxa=\hbox{%
      \ensuremath{^{\si@decimalsymbol}}}%
    \advance\si@tab@expdim\wd\si@tempboxa\relax
    \si@tab@sepcorr{exppost}{exp}%
  \fi
%    \end{macrocode}
% Space is reserved for signs.
%    \begin{macrocode}
  \setbox\si@tempboxa=\hbox{\ensuremath{-}}%
  \ifsi@tab@mantsign
    \advance\si@tab@predim\wd\si@tempboxa\relax
  \fi
  \setbox\si@tempboxa=\hbox{\ensuremath{^{-}}}%
  \ifsi@tab@expsign
    \advance\si@tab@expdim\wd\si@tempboxa\relax
  \fi
%    \end{macrocode}
% Now, if there is space to be saved for an exponent under any
% circumstances, the space for the ``$\times 10$'' part is needed.
% This is done by adding both counters together, then using this
% result for the logic.
%    \begin{macrocode}
  \si@tempcnta\si@tab@expprecnt\relax
  \advance\si@tempcnta\si@tab@exppostcnt\relax
  \ifnum\si@tempcnta>\z@\relax
    \setbox\si@tempboxa=\hbox{\ensuremath{%
      {}\si@expproduct{}\si@expbase}}%
    \advance\si@tab@expdim\wd\si@tempboxa\relax
  \fi
%    \end{macrocode}
% Finally for the box dimensions, if the exponent is not aligned,
% the space reserved for it is added to the post box.
%    \begin{macrocode}
  \ifsi@tabalignexp\else
    \advance\si@tab@postdim\si@tab@expdim\relax
  \fi
%    \end{macrocode}
% With the boxes set up, the contents can be sorted out.  A bit of
% shuffling may be needed, depending on the treatment of exponents.
%    \begin{macrocode}
  \setbox\si@tab@prebox=\hbox to\si@tab@predim{\hss\hfill
    \expandafter\si@out@num\expandafter{\si@tab@out}}%
  \ifx\@empty\si@num@out\@empty
    \setbox\si@tab@postbox=\hbox to\si@tab@postdim{\hfil}%
  \else
    \ifsi@tabalignexp\else
      \protected@edef\si@num@out{\si@num@out\si@tab@expout}%
    \fi
    \setbox\si@tab@postbox=\hbox to\si@tab@postdim
      {\ensuremath{{\si@decimalsymbol}}\expandafter\si@out@num
        \expandafter{\si@num@out}\hfil}%
  \fi
  \ifx\@empty\si@tab@expout\@empty
    \ifsi@tabalignexp
      \setbox\si@tab@expbox=\hbox to\si@tab@expdim{\hfil}%
    \fi
  \else
    \ifsi@tabalignexp
      \setbox\si@tab@expbox=\hbox to\si@tab@expdim
        {\expandafter\si@out@num\expandafter{\si@tab@expout}%
           \hfil}%
    \fi
  \fi}
%    \end{macrocode}
%\end{macro}
%\begin{macro}{\si@tab@sepcorr}
%\darg{one of \texttt{mantpre}, \texttt{mantpost}, \texttt{exppre} or
%  \texttt{exppost}}
%\darg{one of \texttt{pre}, \texttt{post} or \texttt{exp}}
% A spacing correction is needed \emph{if} the number of digits to be
% allowed for will lead to the introduction of a separator.  A counter
% and dimension are needed for the testing.
%    \begin{macrocode}
\newcommand*{\si@tab@sepcorr}[2]{%
  \expandafter\si@tempcnta\expandafter\the
    \csname si@tab@#1cnt\endcsname\relax
%    \end{macrocode}
% Calculate how many groups of three there are, then allow for not
% separating four characters if \cs{ifsi@sepfour} is false.
%    \begin{macrocode}
  \divide\si@tempcnta\thr@@\relax
  \ifsi@sepfour\else
    \expandafter\ifnum\expandafter\the
      \csname si@tab@#1cnt\endcsname=4\relax
      \si@tempcnta\z@\relax
    \fi
  \fi
%    \end{macrocode}
% The width of the separators is measured, and the correct number of
% separator widths are added to the box dimension.
%    \begin{macrocode}
  \setbox\si@tempboxa=\hbox{%
    \ensuremath{\si@tab@sp{\si@digitsep}}}%
  \expandafter\advance\csname si@tab@#2dim\endcsname
    \si@tempcnta\wd\si@tempboxa}
%    \end{macrocode}
%\end{macro}
%
%\subsection{Units}
%\begin{macro}{\SI}
%\doarg{keyval options}
%\darg{number}
% There are two types of user macros for the units system; those for
% defining new units, prefixes and powers, and those for using them.
% There are two macros for using units, \cs{SI} and \cs{si}, which
% work in very similar ways. argument to \cs{SI}
%    \begin{macrocode}
\si@newrobustcmd*{\SI}[2][]{%
  \@ifnextchar[%]
    {\si@SI[#1]{#2}}
    {\si@SI[#1]{#2}[]}}
%    \end{macrocode}
%\end{macro}
%\begin{macro}{\si}
%\doarg{keyval options}
%\darg{unit}
% \cs{si} is just an alias for \cs{SI} with no number; everything is
% handed off into an internal macro.  The internal macro also handles
% the optional prefix
%    \begin{macrocode}
\si@newrobustcmd*{\si}[2][]{\si@SI[#1]{}[]{#2}}
%    \end{macrocode}
%\end{macro}

%\begin{macro}{\newunit}
%\begin{macro}{\renewunit}
%\begin{macro}{\provideunit}
%\doarg{keyval options}
%\darg{unit}
%\darg{symbol}
% The \cs{newunit} and \cs{renewunit} macros create the new unit
% macros. To allow a mechanism for checking an existing definition,
% these macros simply carry out the appropriate tests, before handing
% off to the internal macro.  \cs{@ifdefinable} is not used here as a
% customised error is desirable. Other than that, the code here gives
% very similar results to \cs{newcommand} and \cs{renewcommand}.
% Finally, \cs{provideunit} adds the unit definition only if it does
% not already exist.
%    \begin{macrocode}
\newcommand*{\newunit}[3][]{%
  \si@ifdefinable{#2}
    {\si@unt@defunit[#1]{#2}{#3}}
    {\si@log@err{Unit \string#2 already defined!}\@eha}}
\newcommand*{\renewunit}[3][]{%
  \si@ifdefinable{#2}
    {\si@log@err{Unit \string#2 undefined}\@ehc
     \si@unt@defunit[#1]{#2}{#3}}
    {\si@log@inf{Redefining unit \string#2}%
     \si@unt@defunit[#1]{#2}{#3}}}
\newcommand*{\provideunit}[3][]{%
  \si@ifdefinable{#2}
    {\si@unt@defunit[#1]{#2}{#3}}
    {}}
%    \end{macrocode}
%\end{macro}
%\end{macro}
%\end{macro}
%\begin{macro}{\newprefix}
%\begin{macro}{\renewprefix}
%\begin{macro}{\provideprefix}
%\doarg{\texttt{binary}}
%\darg{multiple}
%\darg{powers-ten}
%\darg{symbol}
% The multiples of units are defined here; very similar code is used
% to the \cs{newunit}, \etc, macros.  The multiple prefixes cannot
% take an optional argument, and must represent some power.  Hence
% the arguments required are different.
%    \begin{macrocode}
\newcommand*{\newprefix}[4][]{%
  \si@ifdefinable{#2}
    {\si@unt@defprefix[#1]{#2}{#3}{#4}}
    {\si@log@err{Prefix \string#2 already defined!}\@eha}}
\newcommand*{\renewprefix}[4][]{%
  \si@ifdefinable{#2}
    {\si@log@err{Prefix \string#2 undefined}\@ehc
     \si@unt@defprefix[#1]{#2}{#3}{#4}}
    {\si@log@inf{Redefining prefix \string#2}%
     \si@unt@defprefix[#1]{#2}{#3}{#4}}}
\newcommand*{\provideprefix}[4][]{%
  \si@ifdefinable{#2}
    {\si@unt@defprefix[#1]{#2}{#3}{#4}}
    {}}
%    \end{macrocode}
%\end{macro}
%\end{macro}
%\end{macro}
%\begin{macro}{\newpower}
%\begin{macro}{\renewpower}
%\begin{macro}{\providepower}
%\doarg{\texttt{post}}
%\darg{number}
%\darg{power}
% Here power multiples for units are set up.  As with units and
% multiples, a layered approach is used to keep things easy to
% maintain.
%    \begin{macrocode}
\newcommand*{\newpower}[3][]{%
  \si@ifdefinable{#2}
    {\si@unt@defpower[#1]{#2}{#3}}
    {\si@log@err{Power \string#2 already defined!}\@eha}}
\newcommand*{\renewpower}[3][]{%
  \si@ifdefinable{#2}
    {\si@log@err{Power \string#2 undefined}\@ehc
     \si@unt@defpower[#1]{#2}{#3}}
    {\si@log@inf{Redefining power \string#2}%
     \si@unt@defpower[#1]{#2}{#3}}}
\newcommand*{\providepower}[3][]{%
  \si@ifdefinable{#2}
    {\si@unt@defpower[#1]{#2}{#3}}
    {}}
%    \end{macrocode}
%\end{macro}
%\end{macro}
%\end{macro}
%\begin{macro}{\newqualifier}
%\changes{v1.1}{2008/09/04}{New user macro}
%\begin{macro}{\renewqualifier}
%\changes{v1.1}{2008/09/04}{New user macro}
%\begin{macro}{\providequalifier}
%\changes{v1.1}{2008/09/04}{New user macro}
%\darg{qualifier}
%\darg{symbol}
% Creating unit qualifiers is much the same method.
%    \begin{macrocode}
\newcommand*{\newqualifier}[2]{%
  \si@ifdefinable{#1}
    {\si@unt@defqual{#1}{#2}}
    {\si@log@err{Qualifier \string#1 already defined!}\@eha}}
\newcommand*{\renewqualifier}[2]{%
  \si@ifdefinable{#1}
    {\si@log@err{Qualifier \string#1 undefined}\@ehc
     \si@unt@defqual{#1}{#2}}
    {\si@unt@defqual{#1}{#2}}}
\newcommand*{\providequalifier}[2]{%
  \si@ifdefinable{#1}
    {\si@unt@defqual{#1}{#2}}
    {}}
%    \end{macrocode}
%\end{macro}
%\end{macro}
%\end{macro}
%\begin{macro}{\ifsi@unt@num}
% A flag is needed to tell the processor whether there is a number,
% to get the correct spacing.
%    \begin{macrocode}
\newif\ifsi@unt@num
%    \end{macrocode}
%\end{macro}
%\begin{macro}{\si@unt@unitarg}
% A storage macro for the argument of the unit macro.
%    \begin{macrocode}
\newcommand*{\si@unt@unitarg}{}
%    \end{macrocode}
%\end{macro}
%\begin{macro}{\si@SI}
%\begin{macro}{\si@unt@SIopts}
%\doarg{keyval options}
%\darg{unit}
%\doarg{preunit}
%\darg{unit}
% The internal processing starts with \cs{si@SI}, which processes the
% second optional argument to \cs{SI} (which is empty for \cs{si}).
% Everything is set up in a group, and processing begins by handling
% the options.
%    \begin{macrocode}
\newcommand*{\si@unt@SIopts}{}
\def\si@SI[#1]#2[#3]#4{%
  \begingroup
    \let\fg\SIfg
    \si@ifnotmtarg{#1}
      {\sisetup{#1}%
       \renewcommand*{\si@unt@SIopts}{#1}}%
%    \end{macrocode}
%\end{macro}
% The prefix unit is handled before any processing of the number; the
% flags are set to get spacing correct.
%    \begin{macrocode}
    \si@unt@numfalse
    \si@xspacefalse
    \si@ifnotmtarg{#3}
      {\si@log@debug{Prefix unit found}%
       \si@unt@printunit{#3}}%
%    \end{macrocode}
% The numerical argument may be empty, in which case no extra space
% should be produced.
%    \begin{macrocode}
    \si@ifnotmtarg{#4}
      {\renewcommand*{\si@unt@unitarg}{#4}}%
    \si@ifnotmtarg{#2}
      {\si@log@debug{Number found in \string\SI\space
         argument}%
       \ifsi@repeatunits\else
         \ifsi@trapambigerr
           \expandafter\expandafter\expandafter
             \si@num@ambigerrtrue
         \fi
       \fi
       \num{#2}%
       \si@unt@numtrue}%
%    \end{macrocode}
% If there is a unit, a check is needed in case the units need to
% have a power added.
%    \begin{macrocode}
    \si@ifnotmtarg{#4}
      {\si@ifmtarg{#2}
         {\si@unt@printunit{#4}}
         {\si@tempcnta\z@\relax
         \ifsi@addunitpower
           \si@unt@countx{#2}%
         \fi
          \ifnum\si@tempcnta>\z@\relax
            \advance\si@tempcnta\@ne\relax
            \edef\si@tempa{\noexpand\tothe{\si@tempcnta}}%
            \renewcommand*{\si@tempb}{#4}%
            \expandafter\expandafter\expandafter
            \si@unt@printunit\expandafter\expandafter
            \expandafter{%
              \expandafter\si@tempb\si@tempa}%
          \else
            \si@unt@printunit{#4}%
          \fi}}%
  \endgroup}
%    \end{macrocode}
%\end{macro}
%\begin{macro}{\si@unt@countx}
% A short macro to count up any multiplication in numerical input.
%    \begin{macrocode}
\newcommand*{\si@unt@countx}[1]{%
  \si@tempcnta\z@\relax
  \expandafter\si@unt@cntx#1\@empty\@empty}
\def\si@unt@cntx#1#2\@empty{%
  \si@str@ifchrstr{#1}{\si@numprod}
    {\advance\si@tempcnta\@ne\relax}
    {}%
  \ifx\@empty#2\@empty\else
    \si@unt@cntx#2\@empty\@empty
  \fi}
%    \end{macrocode}
%\end{macro}
%\begin{macro}{\si@unt@ifliteral}
%\begin{macro}{\ifsi@unt@littest}
%\darg{unit}
% The next stage of the processor is to determine whether or not the
% argument of the unit macro is processable.  For literal arguments,
% this is not the case, and the argument is typeset ``as is''.  On
% the other hand, any units, \etc., declared by the package will work
% with the processor, and so need to be executed before typesetting
% the result.
%    \begin{macrocode}
\newif\ifsi@unt@littest
\newcommand*{\si@unt@ifliteral}[1]{%
  \begingroup
    \si@unt@littesttrue
%    \end{macrocode}
% The test relies on any non-processable test having some width;
% hopefully, this should be the case.
%    \begin{macrocode}
    \setbox\si@tempboxa=\hbox{\si@unt@out{#1}}%
    \ifdim\wd\si@tempboxa>\z@\relax
      \aftergroup\@firstoftwo
    \else
      \aftergroup\@secondoftwo
    \fi
  \endgroup}
%    \end{macrocode}
%\end{macro}
%\end{macro}
%\begin{macro}{\ifsi@unt@litout}
%\begin{macro}{\si@unt@printunit}
%\darg{unit}
% The printing macro uses the above test to determine how to act.  It
% then carries out the appropriate action: either typesetting or
% executing.  A flag is also provided so that any macro units inside
% a partially-literal argument will work (this is also needed to
% emulate \unitsdef).
%    \begin{macrocode}
\newif\ifsi@unt@litout
\newcommand*{\si@unt@printunit}[1]{%
  \si@unt@ifliteral{#1}
%    \end{macrocode}
% The unit includes one or more literal items; typeset using the
% unit typesetting macro.
%    \begin{macrocode}
    {\si@log@debug{Literal items found in unit
      argument:\MessageBreak outputting without further
      processing}%
     \si@unt@litouttrue
     \si@unt@addvaluesep
     \si@unt@out{#1}}
%    \end{macrocode}
% For processable output, the argument is executed; the macros are
% all designed for this.
%    \begin{macrocode}
    {\si@log@debug{Macro unit found:\MessageBreak
       processing to format output}%
     \si@unt@init
     \advance\si@unt@depthcnt\@ne\relax
     #1%
     \si@unt@final}}
%    \end{macrocode}
%\end{macro}
%\end{macro}
%\begin{macro}{\si@unt@addvaluesep}
%\begin{macro}{\si@unt@addvalsep}
%\begin{macro}{\si@unt@litvalsep}
%\begin{macro}{\si@unt@stackvalsep}
% To ensure no problems pop up with expansion, adding the value--unit
% space is handled by a macro.
%    \begin{macrocode}
\newcommand*{\si@unt@addvaluesep}{%
  \ifsi@unt@num
    \expandafter\si@unt@addvalsep
  \fi}
\newcommand*{\si@unt@addvalsep}{%
  \ifsi@unt@litout
    \expandafter\si@unt@litvalsep
  \else
    \expandafter\si@unt@stackvalsep
  \fi}
\newcommand*{\si@unt@stackvalsep}{%
  \protected@edef\si@unt@spstack{\si@valuesep}}
\newcommand*{\si@unt@litvalsep}{%
  \nobreak\ensuremath{\si@valuesep}\nobreak}
%    \end{macrocode}
%\end{macro}
%\end{macro}
%\end{macro}
%\end{macro}
%\begin{macro}{\si@unt@spstack}
%\begin{macro}{\si@unt@stacka}
%\begin{macro}{\si@unt@stackb}
%\begin{macro}{\si@unt@unitcnta}
%\begin{macro}{\si@unt@unitcntb}
%\begin{macro}{\si@unt@depthcnt}
%\begin{macro}{\ifsi@unt@first}
%\begin{macro}{\ifsi@unt@first}
%\begin{macro}{\si@unt@init}
% The initialisation macro sets up the various switches, and clears
% the storage areas for the formatted output.  There are two stacks,
% as when typesetting as fractions, the two parts of the number have
% to be stored separately.  The depth counter is used to tell when
% recursion ends in the processor.  The ``first'' switch is needed as
% the depth counter will not be at one for items processed by
% \cs{SI}.
%    \begin{macrocode}
\newcommand*{\si@unt@spstack}{}
\newcommand*{\si@unt@stacka}{}
\newcommand*{\si@unt@stackb}{}
\newcount\si@unt@unitcnta
\newcount\si@unt@unitcntb
\newcount\si@unt@depthcnt
\newif\ifsi@unt@first
\si@unt@depthcnt\m@ne\relax
\newcommand*{\si@unt@init}{%
  \begingroup
    \si@unt@litoutfalse
    \si@unt@litprefixfalse
    \si@unt@firsttrue
    \si@unt@perfalse
    \si@unt@perseenfalse
    \si@unt@prepowerfalse
    \si@unt@depthcnt\z@\relax
    \si@unt@powerdim\z@\relax
    \si@unt@unitcnta\z@\relax
    \si@unt@unitcntb\z@\relax
    \si@unt@prefixcnt\z@\relax
    \renewcommand*{\si@unt@spstack}{}%
    \renewcommand*{\si@unt@stacka}{}%
    \renewcommand*{\si@unt@stackb}{}%
    \renewcommand*{\si@unt@spacestacka}{}%
    \renewcommand*{\si@unt@spacestackb}{}%
    \renewcommand*{\si@unt@lastadda}{space}%
    \renewcommand*{\si@unt@lastaddb}{space}}
%    \end{macrocode}
%\end{macro}
%\end{macro}
%\end{macro}
%\end{macro}
%\end{macro}
%\end{macro}
%\end{macro}
%\end{macro}
%\end{macro}
%\begin{macro}{\si@unt@final}
% The finalisation macro finishes off the output and resets the flags.
%    \begin{macrocode}
\newcommand*{\si@unt@final}{%
    \si@unt@third
    \si@unt@stackout
  \endgroup
  \ifsi@xspace
    \expandafter\expandafter\expandafter\xspace
  \fi}
%    \end{macrocode}
%\end{macro}
%\begin{macro}{\si@unt@defunit}
%\doarg{keyval options}
%\darg{unit}
%\darg{symbol}
% The internal macro for defining a unit does not check for
% redefinition; that is done by the user macros.
%    \begin{macrocode}
\newcommand*{\si@unt@defunit}[3][]{%
  \si@log@debug{Declaring unit \string#2 with \MessageBreak
    meaning \string#3}%
%    \end{macrocode}
% The optional argument needs to be saved. The macro name is reversed
% so that life is easier with the expansions here.
%    \begin{macrocode}
  \si@ifnotmtarg{#1}
    {\expandafter\@namedef\expandafter{%
      \expandafter\@gobble\string#2@opt@unt@si}{#1}}%
%    \end{macrocode}
% The unit macro itself is now defined.  The definition simply
% selects the correct path for the rest of the processing to go down.
%    \begin{macrocode}
  \protected\def#2{%
    \ifsi@allowoptarg
      \expandafter\si@unt@withopt
    \else
      \expandafter\si@unt@noopt
    \fi
    {#2}{#3}}}
%    \end{macrocode}
%\end{macro}
%\begin{macro}{\si@unt@withopt}
%\begin{macro}{\si@unt@noopt}
%\darg{unit}
%\darg{symbol}
% To allow the correct expansion, the potential optional argument to
% a unit macro has to come last.  So \cs{@ifnextchar} is needed to
% detect it and pass data through.  To keep variation down, when the
% argument is not allowed, the empty |[]| is supplied.
%    \begin{macrocode}
\newcommand*{\si@unt@withopt}[2]{%
  \@ifnextchar[%]
    {\si@unt@opt{#1}{#2}}
    {\si@unt@opt{#1}{#2}[]}}
\newcommand*{\si@unt@noopt}[2]{\si@unt@opt{#1}{#2}[]}
%    \end{macrocode}
%\end{macro}
%\end{macro}
%\begin{macro}{\si@unt@opt}
%\darg{unit}
%\darg{symbol}
%\doarg{number}
% The optional argument to the unit macro (if present) is converted
% to a normal one for ease.  The correct route for processing is then
% picked.
%    \begin{macrocode}
\def\si@unt@opt#1#2[#3]{%
  \ifsi@unt@littest
    \expandafter\si@gobblethree
  \else
%    \end{macrocode}
% For literal output, the second argument is all that is needed.
%    \begin{macrocode}
    \ifsi@unt@litout
      \expandafter\expandafter\expandafter\@gobbletwo
    \else
      \expandafter\expandafter\expandafter\si@unt@unit
    \fi
  \fi
  {#3}{#1}{#2}}
%    \end{macrocode}
%\end{macro}
%\begin{macro}{\si@gobblethree}
% \LaTeX\ does not have a \cs{@gobblethree} macro, but one is needed.
%    \begin{macrocode}
\long\def\si@gobblethree #1#2#3{}
%    \end{macrocode}
%\end{macro}
%\begin{macro}{\si@unt@defprefix}
%\begin{macro}{\ifsi@unt@litprefix}
%\doarg{\texttt{binary}}
%\darg{multiple}
%\darg{powers-ten}
%\darg{symbol}
% As with units, multiples are defined by an internal macro.
%    \begin{macrocode}
\newif\ifsi@unt@litprefix
\si@unt@litprefixtrue
\newcommand*{\si@unt@defprefix}[4][]{%
  \si@log@debug{Declaring multiple \string#1 with\MessageBreak
    meaning \string#4}%
%    \end{macrocode}
% The optional argument is saved, using \cs{def} as no check is made
% on an existing definition of the storage macro.
%    \begin{macrocode}
  \expandafter\expandafter\expandafter\def\expandafter
    \csname\expandafter\@gobble\string#2@opt@si\endcsname{#1}%
  \protected\def#2{%
    \ifsi@unt@littest
      \expandafter\si@gobblethree
    \else
      \ifsi@unt@litout
        \expandafter\expandafter\expandafter\@gobbletwo
      \else
        \ifsi@unt@litprefix
          \expandafter\expandafter\expandafter\expandafter
            \expandafter\expandafter\expandafter\@gobbletwo
        \else
          \expandafter\expandafter\expandafter\expandafter
            \expandafter\expandafter\expandafter\si@unt@prefix
        \fi
      \fi
    \fi
    {#2}{#3}{#4}}}
%    \end{macrocode}
%\end{macro}
%\end{macro}
%\begin{macro}{\si@unt@defpower}
%\doarg{\texttt{post}}
%\darg{number}
%\darg{power}
% The definition of powers is complicated by the need to handle both
% those given before units (such as \cs{cubic}) and those given after
% (\eg \cs{cubed}).  This means that an optional argument is needed.
%    \begin{macrocode}
\newcommand*{\si@unt@defpower}[3][]{%
  \si@log@debug{Declaring power \string#2 with\MessageBreak
    meaning \string#3}%
%    \end{macrocode}
% Once again the optional argument is saved.
%    \begin{macrocode}
  \expandafter\expandafter\expandafter\def\expandafter
    \csname\expandafter\@gobble\string#2@opt@si\endcsname{#1}%
  \protected\def#2{%
    \ifsi@unt@littest
      \expandafter\@gobbletwo
    \else
%    \end{macrocode}
% The literal output here does not need to gobble anything.
%    \begin{macrocode}
      \ifsi@unt@litout
        \expandafter\expandafter\expandafter\si@unt@litpower
      \else
        \expandafter\expandafter\expandafter\si@unt@power
      \fi
    \fi
    {#2}{#3}}}
%    \end{macrocode}
%\end{macro}
%\begin{macro}{\si@unt@defqual}
%\darg{qualifier}
%\darg{symbol}
% Qualifiers are defined in a similar manner.
%    \begin{macrocode}
\newcommand*{\si@unt@defqual}[2]{%
  \si@log@debug{Declaring qualifier \string#1 with\MessageBreak
    meaning \string#2}%
  \protected\def#1{%
    \ifsi@unt@littest
      \expandafter\@gobble
    \else
      \ifsi@unt@litout
        \expandafter\expandafter\expandafter\si@unt@litqual
      \else
        \expandafter\expandafter\expandafter\si@unt@qual
      \fi
    \fi
    {#2}}}
%    \end{macrocode}
%\end{macro}
%
%\begin{macro}{\si@unt@unithook}
%\begin{macro}{\si@unt@unit}
%\darg{number}
%\darg{unit}
%\darg{symbol}
% The macro for units is actually a processor, rather than
% typesetting anything, which is handled elsewhere.  The first
% argument to the macro is optional, but does not have square
% brackets to keep things simple with gobbling.
%    \begin{macrocode}
\newcommand*{\si@unt@unithook}{}
\newcommand*{\si@unt@unit}[3]{%
%    \end{macrocode}
% When the count is minus one at the start of the processor, then the
% unit is begin used on its own: initialisation occurs.
%    \begin{macrocode}
  \ifnum\si@unt@depthcnt=\m@ne\relax
    \expandafter\si@unt@init
  \fi
  \advance\si@unt@depthcnt\@ne\relax
  \si@log@debug{Unit processing: level \the\si@unt@depthcnt,
    \MessageBreak unit \string#2}%
  \si@unt@firstorsecond{#1}{#2}%
%    \end{macrocode}
% The core of the \cs{si@unt@unit} macro is testing if the symbol for
% the unit is a literal value or another macro.  Depending on the
% result, the symbol is either used as a literal or executed.
%    \begin{macrocode}
  \si@unt@ifliteral{#3}
    {\si@unt@addtostack{unit}{#3}%
     \ifsi@unt@prepower
       \expandafter\si@unt@stkpower
     \fi}
    {#3}%
%    \end{macrocode}
% The counter is now stepped down, before checking if this is
% the end of a compound unit.
%    \begin{macrocode}
  \advance\si@unt@depthcnt\m@ne\relax
  \ifnum\si@unt@depthcnt=\z@\relax
    \expandafter\si@unt@final
  \fi}
%    \end{macrocode}
%\end{macro}
%\end{macro}
%\begin{macro}{\si@unt@firstorsecond}
%\darg{number}
%\darg{unit-macro}
% At this stage, the flag will be set for the first item to be
% processed whichever route the unit has been called by.
%    \begin{macrocode}
\newcommand*{\si@unt@firstorsecond}[2]{%
  \ifsi@unt@first
    \expandafter\si@unt@first
  \else
    \expandafter\si@unt@second
  \fi
  {#1}{#2}}%
%    \end{macrocode}
%\end{macro}
%\begin{macro}{\si@unt@first}
%\darg{number}
%\darg{unit-macro}
% For the first unit in the input, some extra tasks are needed.
% First, the optional argument for the unit macro needs to be
% tested.
%    \begin{macrocode}
\newcommand*{\si@unt@first}[2]{%
  \si@ifnotmtarg{#1}
    {\num{#1}%
     \si@unt@numtrue}%
  \si@unt@unithook
%    \end{macrocode}
% To avoid filling up the macro list with useless values, the \eTeX\
% primitive \cs{ifcsname} is employed here (it also avoids complex
% expansion issues).  If some options exist, they are set.
%    \begin{macrocode}
  \ifcsname\expandafter\@gobble\string#2@opt@unt@si\endcsname
    \expandafter\si@unt@setopts
  \else
    \expandafter\@gobble
  \fi
  {#2}%
  \si@unt@addvaluesep
  \si@unt@firstfalse}
%    \end{macrocode}
%\end{macro}
%\begin{macro}{\si@unt@setopts}
%\begin{macro}{\si@unt@setSIopts}
%\darg{unit}
% A rather long set of \cs{expandafter} commands to get the options
% to set safely.
%    \begin{macrocode}
\newcommand*{\si@unt@setopts}[1]{%
  \expandafter\expandafter\expandafter\expandafter\expandafter
    \expandafter\expandafter\si@temptoks\expandafter
    \expandafter\expandafter\expandafter\expandafter
    \expandafter\expandafter{\expandafter%
      \csname\expandafter\@gobble\string#1@opt@unt@si%
      \endcsname}%
  \expandafter\sisetup\expandafter{\the\si@temptoks}%
  \si@log@debug{Applying options `\the\si@temptoks'
    for\MessageBreak unit \string#1}%
%    \end{macrocode}
% The user options are reloaded, if defined, to ensure that they
% still work as expected.
%    \begin{macrocode}
  \@ifundefined{si@unt@SIopts}{}
    {\ifx\@empty\si@unt@SIopts\@empty\else
       \expandafter\expandafter\si@unt@setSIopts
     \fi}}
\newcommand*{\si@unt@setSIopts}{%
  \expandafter\si@temptoks\expandafter{\si@unt@SIopts}%
  \expandafter\sisetup\expandafter{\the\si@temptoks}}
%    \end{macrocode}
%\end{macro}
%\end{macro}
%\begin{macro}{\si@unt@second}
%\begin{macro}{\si@unt@third}
%\darg{number}
%\darg{unit}
% For everything apart from the first item to be processed, spacing
% may need to be added to separated different units.  The macro is
% divided into two, so that everything except the space can be added
% in finalisation.
%    \begin{macrocode}
\newcommand*{\si@unt@second}[2]{%
  \si@ifnotmtarg{#1}
    {\si@log@warn{Optional argument to unit macro\MessageBreak
       allowed only for outer unit}}%
  \si@unt@third
  \si@unt@addtostack{space}{\ensuremath{\si@unitsep}}}
\newcommand*{\si@unt@third}{%
  \ifsi@unt@prepower\else
    \expandafter\si@unt@stkpower
  \fi
%    \end{macrocode}
% A check is made to avoid adding $-1$ to prefixes.  If \opt{frac} is
% active, then the |b| stack will be in use, otherwise it will be
% |a|.
%    \begin{macrocode}
  \renewcommand*{\si@tempa}{prefix}%
  \expandafter\ifx
    \csname si@unt@lastadd\si@unt@checkstack\endcsname\si@tempa
  \else
    \expandafter\si@unt@spacecheck
  \fi
  \ifsi@unt@per
    \expandafter\si@unt@perseentrue
  \fi}
%    \end{macrocode}
%\end{macro}
%\end{macro}
%\begin{macro}{\si@unt@spacecheck}
% A check to prevent adding $-1$ at the very beginning of the unit,
% where there is a space on the stack.
%    \begin{macrocode}
\newcommand*{\si@unt@spacecheck}{%
  \renewcommand*{\si@tempa}{space}%
  \expandafter\ifx
    \csname si@unt@lastadd\si@unt@checkstack\endcsname\si@tempa
  \else
    \expandafter\si@unt@reciptest
  \fi}
%    \end{macrocode}
%\end{macro}
%\begin{macro}{\si@unt@prefix}
%\darg{multiple}
%\darg{power-ten}
%\darg{symbol}
% Actual output of the prefixes.
%    \begin{macrocode}
\newcommand*{\si@unt@prefix}[3]{%
  \si@unt@firstorsecond{}{#1}%
  \ifsi@prefixsymbolic
    \expandafter\si@unt@addprefix
  \else
    \expandafter\si@unt@countprefix
  \fi
  {#1}{#2}{#3}}
%    \end{macrocode}
%\end{macro}
%\begin{macro}{\si@unt@addprefix}
%\darg{gobbled}
%\darg{gobbled}
%\darg{symbol}
% To add the prefix, a little translation is needed.
%    \begin{macrocode}
\newcommand*{\si@unt@addprefix}[3]{%
  \si@unt@addtostack{prefix}{#3}}
%    \end{macrocode}
%\end{macro}
%\begin{macro}{\si@unt@prefixcnt}
% A storage area is created.
%    \begin{macrocode}
\newcount\si@unt@prefixcnt
%    \end{macrocode}
%\end{macro}
%\begin{macro}{\si@unt@countprefix}
%\begin{macro}{\si@unt@invprefix}
%\darg{multiple}
%\darg{powers-ten}
%\darg{gobbled}
% To count the prefix numeral, the symbol is thrown away.  First, a
% check is made for binary units.
%    \begin{macrocode}
\newcommand*{\si@unt@countprefix}[3]{%
  \renewcommand*{\si@tempa}{binary}%
  \expandafter\expandafter\expandafter\ifx\expandafter
    \csname\expandafter\@gobble\string#1@opt@si\endcsname
    \si@tempa
    \expandafter\sisetup
  \else
    \expandafter\@gobble
  \fi
  {prefixbase=two}%
  \si@tempcnta#2\relax
  \ifsi@unt@per
    \expandafter\si@unt@invprefix
  \fi
  \advance\si@unt@prefixcnt\si@tempcnta\relax}
\newcommand*{\si@unt@invprefix}{%
  \si@tempcntb\si@tempcnta\relax
  \si@tempcnta -\si@tempcntb\relax}
%    \end{macrocode}
%\end{macro}
%\end{macro}
%\begin{macro}{\si@unt@litpower}
%\darg{gobbled}
%\darg{number}
% For literal power output, the number can't simply be dumped, so
% a macro is needed.
%    \begin{macrocode}
\newcommand*{\si@unt@litpower}[2]{\textsuperscript{#2}}
%    \end{macrocode}
%\end{macro}
%\begin{macro}{\ifsi@unt@prepower}
% If a power is seen before a unit, tracking is needed.
%    \begin{macrocode}
\newif\ifsi@unt@prepower
%    \end{macrocode}
%\end{macro}
%\begin{macro}{\si@unt@power}
%\darg{power}
%\darg{number}
% The handling of powers starts by ensuring that ``pre'' powers
% follow \cs{per} cleanly.  Then a check is needed for inversion.
%    \begin{macrocode}
\newcommand*{\si@unt@power}[2]{%
  \renewcommand*{\si@tempa}{post}%
  \expandafter\expandafter\expandafter\ifx\expandafter
    \csname\expandafter\@gobble\string#1@opt@si\endcsname
    \si@tempa
    \expandafter\@gobbletwo
  \else
    \expandafter\si@unt@firstorsecond
  \fi
  {}{\power}%
  \si@unt@powerdim #2 pt\relax
  \ifsi@frac\else
    \ifsi@unt@per
      \expandafter\expandafter\expandafter\si@unt@invpower
    \fi
  \fi
  \renewcommand*{\si@tempa}{post}%
  \si@unt@prepowertrue
  \expandafter\expandafter\expandafter\ifx\expandafter
    \csname\expandafter\@gobble\string#1@opt@si\endcsname
    \si@tempa
    \expandafter\si@unt@stackpower
  \else
    \si@log@debug{Power \strip@pt\si@unt@powerdim\space saved
      to be added after\MessageBreak next unit}%
  \fi}
%    \end{macrocode}
%\end{macro}
%\begin{macro}{\si@unt@powerdim}
% To do sign-inversion on the power, a dimension is used (this allows
% non-integer values to be handled).
%    \begin{macrocode}
\newdimen\si@unt@powerdim
%    \end{macrocode}
%\end{macro}
%\begin{macro}{\si@unt@stackpower}
%\begin{macro}{\si@unt@stkpower}
%\begin{macro}{\si@unt@stkpwr}
%\begin{macro}{\si@unt@skpwr}
% Adding powers to the stack should also clear the power list.  If
% the number is already zero, then  of course nothing happens.
%    \begin{macrocode}
\newcommand*{\si@unt@stackpower}{%
  \si@unt@prepowerfalse
%    \end{macrocode}
% A trap is used for $-1$ added to the denominator of a fraction.
%    \begin{macrocode}
  \si@unt@stkpower
  \ifsi@stickyper\else
    \si@unt@perfalse
    \si@unt@perseenfalse
  \fi}
%    \end{macrocode}
% The \cs{si@unt@stkpower} macro needs to be called from a few
% places, so is spun out from the above.
%    \begin{macrocode}
\newcommand*{\si@unt@stkpower}{%
  \ifdim\si@unt@powerdim=\m@ne pt\relax
    \ifsi@frac\else
      \expandafter\expandafter\expandafter\si@unt@stkpwr
    \fi
  \else
    \expandafter\si@unt@stkpwr
  \fi}
%    \end{macrocode}
%\changes{v1.1}{2008/09/04}{Altered to account for qualifiers}
% Finally, the actual adding (set up to avoid problems with the
% \cs{if} above).
%    \begin{macrocode}
\newcommand*{\si@unt@stkpwr}{%
  \ifdim\si@unt@powerdim=\z@\relax\else
    \renewcommand*{\si@tempa}{unit}%
    \expandafter\ifx
      \csname si@unt@lastadd\si@unt@checkstack\endcsname
      \si@tempa
      \si@unt@skpwr
    \else
      \renewcommand*{\si@tempa}{qual}%
      \expandafter\ifx
        \csname si@unt@lastadd\si@unt@checkstack\endcsname
        \si@tempa
        \si@unt@skpwrqual
      \fi
    \fi
  \fi
  \si@unt@powerdim\z@\relax}
\newcommand*{\si@unt@skpwr}{%
  \si@log@debug{Adding power
    \strip@pt\si@unt@powerdim\space to output stack}%
  \si@unt@addtostack{power}{^{\num[fixdp=false]{%
    \strip@pt\si@unt@powerdim}}}}
%    \end{macrocode}
%\end{macro}
%\end{macro}
%\end{macro}
%\end{macro}
%\begin{macro}{\si@unt@invpower}
% A macro to change the sign of the current power.
%    \begin{macrocode}
\newcommand*{\si@unt@invpower}{%
  \si@tempdima\si@unt@powerdim\relax
  \si@unt@powerdim -\si@tempdima\relax
%    \end{macrocode}
% The power might end up as ``1'', which is not wanted.  So it is
% chucked away.
%    \begin{macrocode}
  \ifdim\si@unt@powerdim=\p@\relax
    \si@unt@powerdim\z@\relax
  \fi}
%    \end{macrocode}
%\end{macro}
%\begin{macro}{\si@unt@litqual}
%    \begin{macrocode}
\newcommand*{\si@unt@litqual}[1]{(#1)}
%    \end{macrocode}
%\end{macro}
%\begin{macro}{\si@unt@qual}
%    \begin{macrocode}
\newcommand*{\si@unt@qual}[1]{%
  \ifsi@unt@first
    \si@log@err{Qualifier before units}
      {A unit qualifier can only be given after\MessageBreak
       one or more units.  You tried to open a unit
       with\MessageBreak \string#1}%
  \fi
%    \end{macrocode}
% The space after the current unit is shuffled so it will come
% after the qualifier.
%    \begin{macrocode}
  \expandafter\let\expandafter\si@unt@spacestack
    \csname si@unt@spacestack\si@unt@checkstack\endcsname
  \expandafter\let\csname si@unt@spacestack\si@unt@checkstack
    \endcsname\@empty
%    \end{macrocode}
%\begin{macro}{\si@unt@skpwrqual}
% Some protection against trying to add two powers.
%    \begin{macrocode}
  \renewcommand*{\si@tempa}{power}%
    \expandafter\ifx
    \csname si@unt@lastadd\si@unt@checkstack\endcsname\si@tempa
    \let\si@unt@skpwrqual\relax
  \else
    \let\si@unt@skpwrqual\si@unt@skpwr
  \fi
  \si@unt@addtostack{qual}{_{#1}}%
  \expandafter\let\csname si@unt@spacestack\si@unt@checkstack
    \endcsname\si@unt@spacestack}
%    \end{macrocode}
%\end{macro}
%\end{macro}
%\begin{macro}{\ifsi@unt@per}
%\begin{macro}{\ifsi@unt@perseen}
%\begin{macro}{\per}
%\begin{macro}{\si@per}
%\begin{macro}{\si@unt@per}
% The \cs{per} macro sets the correct flags; almost everything else
% is done elsewhere.  There is always the case of two \cs{per}
% instructions; so the flag is flipped rather than set arbitrarily.
% The second flag is needed so that \cs{per} can give powers of $-1$
% properly.
%    \begin{macrocode}
\newif\ifsi@unt@per
\newif\ifsi@unt@perseen
\si@newrobustcmd*{\si@per}{%
  \ifsi@unt@littest\else
    \ifsi@unt@litout
      \expandafter\expandafter\expandafter /%
    \else
      \ifsi@unt@litprefix
        \expandafter\expandafter\expandafter\expandafter
          \expandafter\expandafter\expandafter /%
      \else
        \expandafter\expandafter\expandafter\expandafter
          \expandafter\expandafter\expandafter\si@unt@per
      \fi
    \fi
  \fi}
\newcommand*{\si@unt@per}{%
  \si@unt@firstorsecond{}{\per}%
  \ifsi@unt@per
    \ifsi@stickyper\else
      \expandafter\expandafter\expandafter\si@unt@perfalse
    \fi
  \else
    \expandafter\si@unt@pertrue
  \fi}
\let\per\si@per
%    \end{macrocode}
%\end{macro}
%\end{macro}
%\end{macro}
%\end{macro}
%\end{macro}
%\begin{macro}{\si@unt@reciptest}
%\begin{macro}{\si@unt@recip}
% A test is needed for adding $-1$ when needed. The second macro is
% fired only if the power should be reciprocal.
%    \begin{macrocode}
\newcommand*{\si@unt@reciptest}{%
  \ifsi@unt@per
    \ifsi@unt@perseen
      \expandafter\expandafter\expandafter\si@unt@recip
    \fi
  \fi}
\newcommand*{\si@unt@recip}{%
  \si@unt@powerdim\m@ne pt\relax
  \si@unt@stackpower}
%    \end{macrocode}
%\end{macro}
%\end{macro}
%\begin{macro}{\si@unt@lastadda}
%\begin{macro}{\si@unt@lastaddb}
% Items cannot be added directly to the stacks (except the spacing
% stack, a) as the fractional handling may need to add the item to
% either storage area. First, a track is kept of what has
% been added at each stage.
%    \begin{macrocode}
\newcommand*{\si@unt@lastadda}{}
\newcommand*{\si@unt@lastaddb}{}
%    \end{macrocode}
%\end{macro}
%\end{macro}
%\begin{macro}{\si@unt@addtostack}
%\changes{v1.1b}{2008/10/12}{Added warning for multiple prefixes}
%\darg{type}
%\darg{token}
% By indicating the type of data to be added to the stack, problems
% can be avoided.
%    \begin{macrocode}
\newcommand*{\si@unt@addtostack}[2]{%
  \renewcommand*{\si@tempa}{#1}%
%    \end{macrocode}
% Two consecutive items cannot be of the same type; there must be spaces
% between units, units between prefixes, \etc.
%    \begin{macrocode}
  \expandafter\ifx
    \csname si@unt@lastadd\si@unt@checkstack\endcsname\si@tempa
      \renewcommand*{\si@tempa}{prefix}%
      \expandafter\ifx\csname si@unt@lastadd\si@unt@checkstack
        \endcsname\si@tempa
        \si@log@warn{Two prefixes found: only first printed}%
      \fi
    \expandafter\@gobbletwo
  \else
    \expandafter\si@unt@preplussp
  \fi
  {#1}{#2}}
%    \end{macrocode}
%\end{macro}
%\begin{macro}{\si@unt@preplussp}
%\darg{type}
%\darg{stack}
%\darg{token}
%\darg{gobbled}
% The space added after a prefix needs to be ignored.
%    \begin{macrocode}
\newcommand*{\si@unt@preplussp}[2]{%
  \renewcommand*{\si@tempa}{prefix+space}%
  \edef\si@tempb{%
    \csname si@unt@lastadd\si@unt@checkstack\endcsname+#1}%
  \ifx\si@tempa\si@tempb
    \expandafter\@gobbletwo
  \else
    \expandafter\si@unt@stack
  \fi
  {#1}{#2}}
%    \end{macrocode}
%\end{macro}
%\begin{macro}{\si@unt@stack}
%\darg{type}
%\darg{token}
% The macro for actually doing the stacking up.
%    \begin{macrocode}
\newcommand*{\si@unt@stack}[2]{%
  \expandafter\renewcommand\expandafter*\expandafter{%
    \csname si@unt@lastadd\si@unt@checkstack\endcsname}{#1}%
%    \end{macrocode}
% A count is kept of the number of \emph{units} added to each stack.
%    \begin{macrocode}
  \renewcommand*{\si@tempa}{#1}%
  \renewcommand*{\si@tempb}{unit}%
  \ifx\si@tempa\si@tempb
    \expandafter\si@unt@inccnt
  \fi
%    \end{macrocode}
% If a space is added, it is actually held until the next add.
%    \begin{macrocode}
  \renewcommand*{\si@tempb}{space}%
  \ifx\si@tempa\si@tempb
    \expandafter\si@unt@holdspace
  \else
    \expandafter\si@unt@addstack
  \fi
  {#2}}
%    \end{macrocode}
%\end{macro}
%\begin{macro}{\si@unt@inccnt}
% The appropriate counter is added to.
%    \begin{macrocode}
\newcommand*{\si@unt@inccnt}{%
  \expandafter\advance
    \csname si@unt@unitcnt\si@unt@checkstack\endcsname
    \@ne\relax}
%    \end{macrocode}
%\end{macro}
%\begin{macro}{\si@unt@spacestack}
%\begin{macro}{\si@unt@spacestacka}
%\begin{macro}{\si@unt@spacestackb}
% The stacked material needs somewhere to live.
%    \begin{macrocode}
\newcommand*{\si@unt@spacestack}{}
\newcommand*{\si@unt@spacestacka}{}
\newcommand*{\si@unt@spacestackb}{}
%    \end{macrocode}
%\end{macro}
%\end{macro}
%\end{macro}
%\begin{macro}{\si@unt@holdspace}
%\begin{macro}{\si@unt@addstack}
%\darg{token}
% Depending on the nature of the addition, it is either held or added
% to the stack.  For the ``b'' space stack, a check is made to ensure
% that a space cannot be added before the first item.
%    \begin{macrocode}
\newcommand*{\si@unt@holdspace}[1]{%
  \renewcommand*{\si@tempa}{b}%
  \edef\si@tempb{\si@unt@checkstack}%
  \ifx\si@tempa\si@tempb
    \ifx\@empty\si@unt@stackb\@empty
    \else
      \protected@edef\si@unt@spacestackb{#1}%
    \fi
  \else
    \protected@edef\si@unt@spacestacka{#1}%
  \fi}
\newcommand*{\si@unt@addstack}[1]{%
  \expandafter\protected@edef
    \csname si@unt@stack\si@unt@checkstack\endcsname
    {\csname si@unt@stack\si@unt@checkstack\endcsname
     \csname si@unt@spacestack\si@unt@checkstack\endcsname#1}%
  \expandafter\renewcommand\expandafter*\expandafter{%
    \csname si@unt@spacestack\si@unt@checkstack\endcsname}{}}
%    \end{macrocode}
%\end{macro}
%\end{macro}
%\begin{macro}{\si@unt@stackout}
% The stack contents are actually typeset here.  First the spacing
% between units and values is added.
%    \begin{macrocode}
\newcommand*{\si@unt@stackout}{%
  \si@unt@litouttrue
  \ifsi@frac
    \expandafter\si@unt@fracout
  \else
    \expandafter\si@unt@normout
  \fi}
%    \end{macrocode}
%\end{macro}
%\begin{macro}{\si@unt@checkstack}
% Which stack is in use needs to be tested.
%    \begin{macrocode}
\newcommand*{\si@unt@checkstack}{%
  \ifsi@frac
    \ifsi@unt@per
      \expandafter\expandafter\expandafter b%
    \else
      \expandafter\expandafter\expandafter a%
    \fi
  \else
    \expandafter a%
  \fi}
%    \end{macrocode}
%\end{macro}
%\begin{macro}{\si@unt@spaceout}
% The space before a unit might not be needed, so it crops up a few
% times in the output routine.
%    \begin{macrocode}
\newcommand*{\si@unt@spaceout}{%
   \ensuremath{\si@unt@spstack}}
%    \end{macrocode}
%\end{macro}
%\begin{macro}{\si@unt@prefixout}
% If the prefix counter is not zero, then there is something to
% typeset.
%    \begin{macrocode}
\newcommand*{\si@unt@prefixout}{%
  \ifnum\si@unt@prefixcnt=\z@\relax\else
    \ifsi@unt@num
      \si@out{\ensuremath{{}\si@prefixproduct{}}}%
    \fi
    \si@unt@stackvalsep
    \let\si@expbase\si@prefixbase
    \num[fixdp=false]{e\the\si@unt@prefixcnt}%
  \fi}
%    \end{macrocode}
%\end{macro}
%\begin{macro}{\si@unt@normout}
% The normal output mode is set up here; nothing much needs to be
% done as there is no need for complex checks.
%    \begin{macrocode}
\newcommand*{\si@unt@normout}{%
  \si@unt@prefixout
  \si@unt@spaceout
  \expandafter\si@unt@out\expandafter{\si@unt@stacka}}
%    \end{macrocode}
%\end{macro}
%\begin{macro}{\si@unt@fracout}
% For fractions, some checks are needed.
%    \begin{macrocode}
\newcommand*{\si@unt@fracout}{%
  \si@unt@notambig
  \ifx\@empty\si@unt@stacka\@empty
    \ifx\@empty\si@unt@stackb\@empty
      \ifsi@unt@litout\else
        \si@log@err{Empty fractional unit}{The unit
          argument\MessageBreak given does not contain any
          symbols}%
      \fi
    \else
%    \end{macrocode}
% With an empty numerator, no space is added
%    \begin{macrocode}
      \ifsi@slash
        \si@unt@prefixout
        \si@frac{}{\si@unt@stackb}%
      \else
        \si@unt@prefixout
        \si@unt@spaceout
        \si@frac{1}{\si@unt@stackb}%
      \fi
    \fi
  \else
%    \end{macrocode}
% If the denominator is empty, then the usual output system can be
% used.
%    \begin{macrocode}
    \ifx\@empty\si@unt@stackb\@empty
      \si@unt@normout
    \else
      \si@unt@prefixout
      \si@unt@spaceout
      \si@frac{\si@unt@stacka}{\si@unt@stackb}%
    \fi
  \fi}
%    \end{macrocode}
%\end{macro}
%\begin{macro}{\si@unt@notambig}
%\begin{macro}{\si@unt@notabg}
% A trap is set for adding brackets to units using a slash, when more
% than one unit is in the denominator.
%    \begin{macrocode}
\newcommand*{\si@unt@notambig}{%
  \ifnum\si@unt@unitcntb>\@ne\relax
    \ifsi@slash
        \ifsi@trapambigfrac
          \expandafter\expandafter\expandafter\expandafter
            \expandafter\expandafter\expandafter\si@unt@notabg
        \fi
    \fi
  \fi}
\newcommand*{\si@unt@notabg}{%
  \protected@edef\si@unt@stackb{\si@openfrac\si@unt@stackb
    \si@closefrac}}
%    \end{macrocode}
%\end{macro}
%\end{macro}
%
%\begin{macro}{\si@unt@out}
%\changes{v1.0d}{2008/06/25}{Sanitise one level of unit input
%  only}
%\darg{unit}
% The final part of the units system is the output routine.  This has
% to cope with units not only as macros but also as direct input
% (\SIstyle-type input).  Non-Latin characters are also handled
% cleanly.  The \cs{scantokens} system deals with everything except
% full stops; these are left out so that a single level system can be
% used \latin{via} a token register.
%    \begin{macrocode}
\begingroup
  \catcode`\~=\active
  \catcode`\.=\active
  \gdef\si@unt@out#1{%
    \si@temptoks{#1}%
    \si@unt@fullstop
    \def.{\ensuremath{\si@unitsep}}%
    \def~{\ensuremath{\si@unitspace}}%
    \expandafter\protected@edef\expandafter\si@tempa
      \expandafter{\the\si@temptoks}%
    \begingroup
      \si@unt@nonlatin
      \makeatletter
      \endlinechar\m@ne
      \expandafter\si@out\expandafter{%
        \expandafter\scantokens\expandafter{\si@tempa}}%
    \endgroup}
\endgroup
%    \end{macrocode}
%\end{macro}
%\begin{macro}{\si@unt@fullstop}
%\begin{macro}{\si@unt@stp}
% Two macros modified from \pkg{kvsetkeys} to deal with a single
% level of active full stops only.
%    \begin{macrocode}
\begingroup
  \catcode`\.=\active
  \catcode`\&=12\relax
  \begingroup
    \lccode`\.=`\.\relax
    \lccode`\&=`\.\relax
  \lowercase{\endgroup
    \gdef\si@unt@fullstop{%
      \si@temptoks\expandafter{\expandafter}\expandafter
      \si@unt@stp\the\si@temptoks&\@nil}
  \gdef\si@unt@stp#1&#2\@nil{%
    \edef\si@tempa{\the\si@temptoks}%
    \ifx\si@tempa\@empty
      \expandafter\@firstoftwo
    \else
      \expandafter\@secondoftwo
    \fi
      {\si@temptoks{#1}}
      {\si@temptoks\expandafter{\the\si@temptoks.#1}}%
    \si@ifmtarg{#2}
      {}
      {\si@unt@stp#2\@nil}}}
\endgroup
%    \end{macrocode}
%\end{macro}
%\end{macro}
%\begin{macro}{\si@unt@nonlatin}
% To handle non-Latin symbols in the input, a single macro is
% provided.  If \XeTeX\ is in use, this can be detected immediately.
%    \begin{macrocode}
\newcommand*{\si@unt@nonlatin}{}
\ifdefined\XeTeXrevision
  \renewcommand*{\si@unt@nonlatin}{%
    \catcode176=\active
    \catcode181=\active
    \catcode197=\active
    \si@unt@sym{176}{\si@sym@degree}%
    \si@unt@sym{181}{\si@sym@mu}%
    \si@unt@sym{197}{\si@sym@ringA}}%
\fi
%    \end{macrocode}
% If \pkg{inputenc} has been loaded, then a check is made that the
% encoding is correct.  If all is well, the non-Latin symbols are
% handled.  The degree symbol is character 176, the micro
% symbol is character 181 and ring capital A is character 197 in
% latin1.
%    \begin{macrocode}
\AtBeginDocument{
  \@ifpackageloaded{inputenc}
    {\@for\si@tempa:=latin1,latin5,latin9\do{
       \ifx\inputencodingname\si@tempa
         \renewcommand*{\si@unt@nonlatin}{%
           \catcode176=\active
           \catcode181=\active
           \catcode197=\active
           \si@unt@sym{176}{\si@sym@degree}%
           \si@unt@sym{181}{\si@sym@mu}%
           \si@unt@sym{197}{\si@sym@ringA}}%
       \fi}}
    {}}
%    \end{macrocode}
%\end{macro}
%\begin{macro}{\si@unt@sym}
%\darg{charcode}
% A macro for declaring symbols: a copy of \cs{DeclareInputMath} from
% \pkg{inputenc}.
%    \begin{macrocode}
\newcommand*{\si@unt@sym}[1]{%
  \bgroup
    \uccode`\~#1%
    \uppercase{%
      \egroup
      \def~}}
%    \end{macrocode}
%\end{macro}
%
%\begin{macro}{\kilogram}
%\begin{macro}{\metre}
%\begin{macro}{\meter}
%\begin{macro}{\mole}
%\begin{macro}{\kelvin}
%\begin{macro}{\candela}
%\begin{macro}{\second}
%\begin{macro}{\ampere}
% With the system set up, the basic unit macros are implemented.  The
% only units defined whatever options are given are the base SI
% units.
%    \begin{macrocode}
\newunit{\kilogram}{kg}
\newunit{\metre}{m}
\newunit{\meter}{\metre}
\newunit{\mole}{mol}
\AtBeginDocument{\provideunit{\second}{s}}
\newunit{\ampere}{A}
\newunit{\kelvin}{K}
\newunit{\candela}{cd}
%    \end{macrocode}
%\end{macro}
%\end{macro}
%\end{macro}
%\end{macro}
%\end{macro}
%\end{macro}
%\end{macro}
%\end{macro}
%\begin{macro}{\Square}
%\changes{v1.1b}{2008/10/12}{Only defined if not otherwise available}
%\begin{macro}{\ssquare}
%\begin{macro}{\squared}
%\begin{macro}{\cubic}
%\begin{macro}{\cubed}
% Unlike multiples (which can be skipped if needed), the basic powers
% are also always defined.
%    \begin{macrocode}
\AtBeginDocument{
  \@ifundefined{Square}{%
    \newpower{\Square}{2}}{}}
\newpower{\ssquare}{2}
\newpower[post]{\squared}{2}
\newpower{\cubic}{3}
\newpower[post]{\cubed}{3}
%    \end{macrocode}
%\end{macro}
%\end{macro}
%\end{macro}
%\end{macro}
%\end{macro}
%\begin{macro}{\tothe}
%\begin{macro}{\raiseto}
% The user macros for arbitraty powers are simple
% calling macros to a common internal macro.
%    \begin{macrocode}
\newcommand*{\tothe}{\si@tothe{\tothe}}
\newcommand*{\raiseto}{\si@tothe{\raiseto}}
%    \end{macrocode}
%\end{macro}
%\end{macro}
%\begin{macro}{\si@tothe}
%\begin{macro}{\tothe@opt@si}
%\begin{macro}{\raiseto@opt@si}
%\darg{either \texttt{\cs{tothe}} or \texttt{\cs{raiseto}}}
%\darg{number}
% A macro for arbitrary powers, which comes after the unit and so
% needs to be marked as such.  Two fake option-storage macros are
% created so that everything works correctly.
%    \begin{macrocode}
\newcommand*{\si@tothe}[2]{%
  \ifsi@unt@littest
    \expandafter\@gobbletwo
  \else
    \ifsi@unt@litout
      \expandafter\expandafter\expandafter\si@unt@litpower
    \else
      \expandafter\expandafter\expandafter\si@unt@power
    \fi
  \fi
  {#1}{#2}}
\newcommand*{\tothe@opt@si}{post}
\newcommand*{\raiseto@opt@si}{}
%    \end{macrocode}
%\end{macro}
%\end{macro}
%\end{macro}
%
%\subsection{Locales}
%\begin{macro}{\si@loc@load}
%\begin{macro}{\si@loc@sisetup}
%\darg{locale}
% When loading a locale, the setup is saved rather than applied.
% Anything other than simple settings should be inside
% \cs{addtolocale}, which is already defined.
%    \begin{macrocode}
\newcommand*{\si@loc@load}[1]{%
  \let\si@loc@sisetup\sisetup
  \renewcommand*{\sisetup}[1]{%
    \expandafter\gdef\csname si@loc@#1\endcsname{##1}}%
  \si@loadfile{#1}%
  \let\sisetup\si@loc@sisetup}
%    \end{macrocode}
%\end{macro}
%\end{macro}
%\begin{macro}{\si@loc@set}
%\darg{locale}
% Setting the locale transfers the settings to \cs{sisetup}, and runs
% any extra macros.
%    \begin{macrocode}
\newcommand*{\si@loc@set}[1]{%
  \ifcsname si@loc@#1\endcsname
    \si@log@inf{Setting locale to `#1'}%
    \expandafter\expandafter\expandafter\expandafter
      \expandafter\expandafter\expandafter\si@temptoks
      \expandafter\expandafter\expandafter\expandafter
      \expandafter\expandafter\expandafter{%
        \expandafter\csname si@loc@#1\endcsname}%
    \expandafter\sisetup\expandafter{\the\si@temptoks}%
    \ifcsname si@loc@#1@extra\endcsname
      \csname si@loc@#1@extra\endcsname
    \fi
  \else
    \ifcsname si@loc@#1@extra\endcsname
      \si@log@inf{Setting locale to `#1'}%
      \csname si@loc@#1@extra\endcsname
    \else
      \si@log@warn{Unknown locale `#1'}%
    \fi
  \fi}
%    \end{macrocode}
%\end{macro}
%\begin{macro}{\si@loc@ltol}
%\changes{v1.0a}{2008/06/20}{\pkg{babel} support fixed for default
%  document language}
%\darg{list of locales and languages}
% The necessary loading for language modules occurs.
%    \begin{macrocode}
\newcommand*{\si@loc@ltol}[1]{%
  \def\si@tempa##1:##2\@nil{\si@loc@load{##1}}
  \@for\si@tempb:=#1\do{%
    \expandafter\si@tempa\si@tempb:\@nil}
  \AtBeginDocument{
    \@ifpackageloaded{babel}
      {\def\si@tempa##1:##2:##3\@nil{%
        \expandafter\addto\expandafter{%
          \csname extras##2\endcsname}%
          {\si@loc@set{##1}}}%
       \@for\si@tempb:=#1\do{%
         \expandafter\si@tempa\si@tempb::\@nil}%
       \expandafter\selectlanguage\expandafter{\languagename}}
      {\si@log@warn{babel not loaded \MessageBreak
        loctolang option ignored}}}}
\AtBeginDocument{
  \@ifpackageloaded{babel}
    {\renewcommand*{\si@loc@ltol}[1]{%
       \def\si@tempa##1:##2\@nil{\si@loc@load{##1}}%
       \@for\si@tempb:=#1\do{%
         \expandafter\si@tempa\si@tempb:\@nil}%
       \def\si@tempa##1:##2:##3\@nil{%
         \expandafter\addto\expandafter{%
           \csname extras##2\endcsname}%
           {\si@loc@set{##1}}}%
       \@for\si@tempb:=#1\do{%
         \expandafter\si@tempa\si@tempb::\@nil}}}
    {\renewcommand*{\si@loc@ltol}[1]{%
       \si@log@warn{babel not loaded \MessageBreak
         loctolang option ignored}}}}
%    \end{macrocode}
%\end{macro}
%\begin{macro}{\addtolocale}
%\darg{locale}
%\darg{commands}
% Arbitrary macros may need to be added to the locale.
%    \begin{macrocode}
\newcommand*{\addtolocale}[2]{%
  \si@addtocsname{si@loc@#1@extra}{#2}}
%    \end{macrocode}
%\end{macro}
%
%\subsection{Output routine}
%\begin{macro}{\si@out}
%\changes{v1.1}{2008/07/22}{Extended handling to subscripts}
%\darg{text}
% With all of the setup done, the text can finally be typeset.  This
% is done inside a \cs{text} block, which takes care of
% \cs{ensuremath}, \etc.  First of all, the various catcode settings
% needed to get maths-in-text mode are made.  \cs{makeatletter} is
% needed so that \cs{scantokens} still allows internal macros to
% work.
%    \begin{macrocode}
\begingroup
  \catcode`\^=\active
  \catcode`\-=\active
  \catcode`\_=\active
  \gdef\si@out#1{%
    \begingroup
      \catcode`\^=\active
      \catcode`\_=\active
      \makeatletter
      \endlinechar\m@ne
%    \end{macrocode}
%\changes{v1.0f}{2008/07/17}{Fixed bug with colour and spacing}
% The various font families can now be set up.  First a check is made
% in case there are nested calls to \cs{si@out@text}.
%    \begin{macrocode}
      \ifsi@fam@set\else
        \expandafter\si@fam@set
      \fi
      \text{%
        \si@colourcmd{\si@colour}%
        \si@fam@italic\si@fam@bold\si@fam@text
%    \end{macrocode}
%  The correct mode is selected, and the input is handed off for
%  typesetting.
%    \begin{macrocode}
        \ifsi@textmode
          \expandafter\si@out@text
        \else
          \expandafter\si@out@maths
        \fi
        {\scantokens{#1}}}%
    \endgroup
    \check@mathfonts}
%    \end{macrocode}
%\end{macro}
%\begin{macro}{\si@out@text}
%\begin{macro}{\si@out@maths}
%\darg{text}
% Output occurs with the correct changes to superscript behaviour.
%    \begin{macrocode}
  \gdef\si@out@text#1{%
    \let^\si@out@sp
    \let_\si@out@sb
    \let\textsuperscript\si@out@sp
    \let\textsubscript\si@out@sb
    \catcode`\-=\active\relax
    \let-\si@out@minus
    #1}
  \gdef\si@out@maths#1{%
    \let^\sp
    \let_\sb
    \let\textsuperscript\sp
    \let\textsubscript\sb
    $\si@fam@maths{#1}$}
\endgroup
%    \end{macrocode}
%\end{macro}
%\end{macro}
%\begin{macro}{\si@out@sp}
%\begin{macro}{\si@out@sb}
%\darg{text}
% \cs{textsuperscript} gives slightly different alignment of numbers
% to using |^| in text mode.  To avoid this, a slightly different
% definition is used.  Elsewhere \cs{textsuperscript} is used, as the
% code above sorts out the text/maths mode issues. The same is done
% for subscripts.
%    \begin{macrocode}
\newcommand*{\si@out@sp}[1]{\ensuremath{^{\text{#1}}}}
\newcommand*{\si@out@sb}[1]{\ensuremath{_{\text{#1}}}}
%    \end{macrocode}
%\end{macro}
%\end{macro}
%\begin{macro}{\si@out@minus}
%\changes{v1.1j}{2008/12/22}{Fixed infinite loop with \pkg{breqn}}
% An active minus sign is needed.
%    \begin{macrocode}
\newcommand*{\si@out@minus}{}
\DeclareTextSymbolDefault{\si@out@minus}{TS1}
\DeclareTextSymbol{\si@out@minus}{TS1}{61}
%    \end{macrocode}
%\end{macro}
%\begin{macro}{\ifsi@out@num}
% A flag is needed to control output settings.  This will be
% false unless inside \cs{si@out@num}.
%    \begin{macrocode}
\newif\ifsi@out@num
%    \end{macrocode}
%\end{macro}
%\begin{macro}{\si@out@num}
%\darg{number}
% For numerical output, the default fonts are controlled slightly
% differently to text output.
%    \begin{macrocode}
\newcommand*{\si@out@num}[1]{%
  \begingroup
    \si@out@numtrue
    \si@out{#1}%
  \endgroup}
%    \end{macrocode}
%\end{macro}
%
%\subsection{Finalisation}
%\begin{macro}{\si@extension}
%\begin{macro}{\si@fileprefix}
% To keep the code easy to maintain, the reusable filename components
% are macros rather than literals.
%    \begin{macrocode}
\newcommand*{\si@extension}{cfg}
\newcommand*{\si@fileprefix}{si-}
%    \end{macrocode}
%\end{macro}
%\end{macro}
%\begin{macro}{\si@ifl@aded}
%\begin{macro}{\si@ifloaded}
%\darg{package}
% A bit of borrowing from the \LaTeX\ kernel.  A copy of
% \cs{@ifl@aded} is needed as things aren't always done in the
% preamble by \currpkg.
%    \begin{macrocode}
\newcommand*{\si@ifl@aded}{}
\let\si@ifl@aded\@ifl@aded
\newcommand*{\si@ifloaded}[1]{%
  \si@ifl@aded\si@extension{\si@fileprefix#1}}
%    \end{macrocode}
%\end{macro}
%\end{macro}
%\begin{macro}{\si@loadfile}
%\darg{file}
% Loading configuration files is handled here.
%    \begin{macrocode}
\newcommand*{\si@loadfile}[1]{%
  \si@ifloaded{#1}{}
    {\si@InputIfFileExists{\si@fileprefix#1.\si@extension}
      {}
      {\si@log@err{Failed to load file
         \si@fileprefix#1.\si@extension}
         {The configuration file requested could not be
          found}}}}
%    \end{macrocode}
%\end{macro}
%\begin{macro}{\ifsi@outerinput}
% At various points, the package needs to know if the
% currently-read configuration file is the
% outer one.
%    \begin{macrocode}
\newif\ifsi@outerinput\si@outerinputtrue
%    \end{macrocode}
%\end{macro}
%\begin{macro}{\si@InputIfFileExists}
% To allow reading configuration files after the start of the
% document, a private copy of \cs{InputIfFileExists} is needed.  The
% main macro here simply checks if the modified version is needed.
%    \begin{macrocode}
\let\si@InputIfFileExists\InputIfFileExists
\AtBeginDocument{
  \renewcommand*{\si@InputIfFileExists}{%
    \ifsi@outerinput
      \expandafter\si@outerinput
    \else
      \expandafter\InputIfFileExists
    \fi}}
%    \end{macrocode}
%\end{macro}
%\begin{macro}{\si@outerinput}
%\darg{file}
%\darg{true-code}
%\darg{false-code}
% The outer call to load a file needs to set |@| to a letter and turn
% of printing using \cs{nullfont}.
%    \begin{macrocode}
\newcommand*{\si@outerinput}[3]{%
  \makeatletter
  \nullfont
  \si@outerinputfalse
  \IfFileExists{#1}%
    {#2\@addtofilelist{#1}\@@input \@filef@und\normalsize}
    {#3}%
  \normalsize
  \makeatother}
%    \end{macrocode}
%\end{macro}
%\begin{macro}{\requiresiconfigs}
%\darg{config-file}
% The configuration files depend on each other.
%    \begin{macrocode}
\newcommand*{\requiresiconfigs}[1]{%
  \@for\si@tempb:=#1\do{\si@loadfile{\si@tempb}}}
%    \end{macrocode}
%\end{macro}
%\begin{macro}{\si@loademfile}
%\darg{file}
% For emulation files, an additional check is made.
%    \begin{macrocode}
\newcommand*{\si@loademfile}[1]{%
  \@ifpackageloaded{#1}
    {\si@log@err{Emulation clash for package `#1'}
      {You have asked for emulation of package
       `#1'\MessageBreak
       (perhaps by giving siunitx a back-compatibility
       option)\MessageBreak but the package is already
       loaded!}}
    {\si@loadfile{#1}}}
%    \end{macrocode}
%\end{macro}
%\begin{macro}{\si@emclash}
%\darg{package}
%\darg{package}
% A macro for emulation clashes.
%    \begin{macrocode}
\newcommand*{\si@emclash}[2]{%
  \si@log@err{Emulation clash: `#1' and `#2'}
    {You have asked for emulation of package `#1'\MessageBreak
     but have already loaded emulation of `#2'}}
%    \end{macrocode}
%\end{macro}
%\begin{macro}{\si@emulating}
%\darg{package}
%\darg{version}
% For packages that are emulated, the \LaTeX\ mechanism to prevent
% re-loading is used.  The list of packages to check at the start
% of the document also has to be altered.
%    \begin{macrocode}
\newcommand*{\si@emulating}[2]{%
  \@namedef{ver@#1.sty}{#2 siunitx emulation of #1}%
  \let\si@tempa\si@blockpkgs
  \renewcommand*{\si@blockpkgs}{}%
  \@for\si@tempb:=\si@tempa\do{%
    \renewcommand*{\si@tempa}{#1}%
    \ifx\si@tempa\si@tempb\else
      \lowercase{\edef\si@tempa{#1}}%
      \lowercase{\edef\si@tempc{\si@tempb}}%
      \ifx\si@tempa\si@tempc
        \@namedef{ver@\si@tempc.sty}{#2 siunitx emulation of
          #1}%
      \else
        \si@addtolist{\si@blockpkgs}{\si@tempb}%
      \fi
    \fi}%
  \let\si@tempa\si@checkpkgs
  \renewcommand*{\si@checkpkgs}{}%
  \renewcommand*{\si@tempb}{#1}%
  \@for\si@tempc:=\si@tempa\do{%
    \ifx\si@tempb\si@tempc\else
      \si@addtolist{\si@checkpkgs}{\si@tempc}%
    \fi}}
%    \end{macrocode}
%\end{macro}
% With the \currpkg kernel macros defined, the package can now run
% through finalisation.  First, the default key values are set. The
% user options are then processed.
%    \begin{macrocode}
\sisetup{
  addsign=none,
  allowzeroexp=false,
  angformat=unchanged,
  astroang=false,%(
  closeerr=),%(
  closefrac=),
  colour=black,
  colourall=false,
  colourneg=false,
  decimalsymbol=fullstop,
  detectdisplay=true,
  digitsep=thin,
  dp=3,
  eVcorra=0.3ex,
  eVcorrb=0ex,
  errspace=none,
  fixdp=false,
  inlinebold=text,
  load=default,
  mathsrm=mathrm,
  mathssf=mathsf,
  mathstt=mathtt,
  mode=maths,
  negcolour=red,
  noload={},
  numaddn={},%(
  numcloseerr=),
  numdecimal={.,},
  numdigits=0123456789,
  numdiv=/,
  numexp=eEdD,
  numgobble={},
  numopenerr=(,%)
  numprod=x,
  numsign=+-\pm\mp,
  obeybold=false,
  obeyitalic=false,
  obeymode=false,
  openerr=(,%)
  openfrac=(,%)
  padangle=small,
  padnumber=lead,
  per=reciprocal,
  prefixbase=ten,
  prefixproduct=times,
  prefixsymbolic=true,
  prespace=false,
  redefsymbols=true,
  repeatunits=true,
  retainplus=false,
  seperr=false,
  sepfour=false,
  sign=plus,
  slash=slash,
  stickyper=false,
  strictarc=true,
  tabalignexp=true,
  tabautofit=false,
  tabformat=3.2,
  tabnumalign=centredecimal,
  tabparseonly=false,
  tabtextalign=centre,
  tabunitalign=left,
  textrm=rmfamily,
  textsf=sffamily,
  texttt=ttfamily,
  tightpm=false,
  trapambigerr=true,
  trapambigfrac=true,
  unitsep=thin,
  valuesep=thin,
  xspace=false}
\ProcessOptionsX[si]<key>
%    \end{macrocode}
% A check is now made so that emulation takes place one file at a
% time, and that each file is loaded only once.
%    \begin{macrocode}
\ifx\@empty\si@emulate\@empty\else
  \@for\si@tempa:=\si@emulate\do{%
    \si@loademfile{\si@tempa}}
\fi
%    \end{macrocode}
%\begin{macro}{\si@expanddefault}
% For turning the list of default choices into a loadable list.
%    \begin{macrocode}
\newcommand*{\si@expanddefault}[2]{%
  \expandafter\ifx\expandafter\@empty\csname si@#1\endcsname
    \@empty
  \else
    \renewcommand*{\si@tempb}{default}%
    \renewcommand*{\si@tempc}{}%
    \expandafter\@for\expandafter\si@tempa\expandafter
      :\expandafter=\csname si@#1\endcsname\do{%
      \ifx\si@tempa\si@tempb
        \si@addtolist{\si@tempc}{#2}%
      \else
        \si@addtolist{\si@tempc}{\si@tempa}%
      \fi}
    \expandafter\edef\csname si@#1\endcsname{\si@tempc}%
    \expandafter\si@addtolist\expandafter{%
      \csname si@no#1\endcsname}%
      {default}%
    \renewcommand*{\si@tempc}{}%
    \expandafter\@for\expandafter\si@tempa\expandafter
      :\expandafter=\csname si@#1\endcsname\do{%
      \si@switchfalse
      \expandafter\@for\expandafter\si@tempb\expandafter
        :\expandafter=\csname si@no#1\endcsname\do{%
        \ifx\si@tempa\si@tempb
          \si@switchtrue
        \fi
        \ifsi@switch\else
          \si@addtolist{\si@tempc}{\si@tempa}%
        \fi}}
    \@for\si@tempa:=\si@tempc\do{%
      \si@loadfile{\si@tempa}}%
  \fi}
%    \end{macrocode}
%\end{macro}
% The configuration and abbreviation files are loaded.
%    \begin{macrocode}
\si@expanddefault{load}{prefix,named,addn,prefixed,accepted,%
  physical,abbr}
%    \end{macrocode}
% The very last job is to load a local configuration file, if one
% exists, and restore catcodes.
%    \begin{macrocode}
\IfFileExists{siunitx.cfg}
  {\si@log@inf{Local configuration file found}%
   \InputIfFileExists{siunitx.cfg}{}{}}
  {}
\si@catcodes
%    \end{macrocode}
%
%\iffalse
%</package>
%\fi
%
%\section{Loadable modules}
% To keep the package relatively clear, and to make maintenance
% easier, the only units defined in the package itself are the base
% units. Everything else is a loadable module (similar to the
% approach in \unitsdef).
%
%\iffalse
%<*prefix>
%\fi
%\subsection{Multiple prefixes}
%\begin{macro}{\yocto}
%\begin{macro}{\zepto}
%\begin{macro}{\atto}
%\begin{macro}{\femto}
%\begin{macro}{\pico}
%\begin{macro}{\nano}
%\begin{macro}{\micro}
%\begin{macro}{\Micro}
%\begin{macro}{\milli}
%\begin{macro}{\centi}
%\begin{macro}{\deci}
% The various SI multiple prefixes are defined here.  First the small
% powers.
%    \begin{macrocode}
\ProvidesFile{si-prefix.cfg}
  [2008/12/22 v1.1j siunitx: SI Multiple prefixes]
\newprefix{\yocto}{-24}{y}
\newprefix{\zepto}{-21}{z}
\newprefix{\atto}{-18}{a}
\newprefix{\femto}{-15}{f}
\newprefix{\pico}{-12}{p}
\newprefix{\nano}{-9}{n}
%    \end{macrocode}
% Some testing is needed for \unitsdef compatibility.
%    \begin{macrocode}
\ifsi@old@OHM
  \newprefix{\Micro}{-6}{\si@sym@mu}
\else
  \ifsi@gensymb\else
    \newprefix{\micro}{-6}{\si@sym@mu}
  \fi
\fi
\newprefix{\milli}{-3}{m}
\newprefix{\centi}{-2}{c}
\newprefix{\deci}{-1}{d}
%    \end{macrocode}
%\end{macro}
%\end{macro}
%\end{macro}
%\end{macro}
%\end{macro}
%\end{macro}
%\end{macro}
%\end{macro}
%\end{macro}
%\end{macro}
%\end{macro}
%\begin{macro}{\deca}
%\begin{macro}{\hecto}
%\begin{macro}{\kilo}
%\begin{macro}{\mega}
%\begin{macro}{\giga}
%\begin{macro}{\tera}
%\begin{macro}{\peta}
%\begin{macro}{\exa}
%\begin{macro}{\zetta}
%\begin{macro}{\yotta}
% Now the large ones.
%    \begin{macrocode}
\newprefix{\deca}{1}{da}
\newprefix{\hecto}{2}{h}
\newprefix{\kilo}{3}{k}
\newprefix{\mega}{6}{M}
\newprefix{\giga}{9}{G}
\newprefix{\tera}{12}{T}
\newprefix{\peta}{15}{P}
\newprefix{\exa}{18}{E}
\newprefix{\zetta}{21}{Z}
\newprefix{\yotta}{24}{Y}
%    \end{macrocode}
%\end{macro}
%\end{macro}
%\end{macro}
%\end{macro}
%\end{macro}
%\end{macro}
%\end{macro}
%\end{macro}
%\end{macro}
%\end{macro}
%\begin{macro}{\deka}
% Apparently, ``deka'' is common in the US for deca.
%    \begin{macrocode}
\newprefix{\deka}{1}{da}
%    \end{macrocode}
%\end{macro}
%\begin{macro}{\gram}
%\begin{macro}{\kilogram}
% As the base unit of mass is the kilogram, rather than the gram, a
% bit of extra work is needed; by default the package only defines
% \cs{kilogram}, but with the prefixes available, this is altered to
% be \cs{kilo}\cs{gram}.  For that, the \cs{gram} must be defined first.
%    \begin{macrocode}
\newunit{\gram}{g}
\renewunit{\kilogram}{\kilo\gram}
%    \end{macrocode}
%\end{macro}
%\end{macro}
%\iffalse
%</prefix>
%<*named>
%\fi
%\subsection{Derived units with specific names}
%\begin{macro}{\becquerel}
%\begin{macro}{\coulomb}
%\begin{macro}{\farad}
%\begin{macro}{\Gray}
%\begin{macro}{\ggray}
%\begin{macro}{\hertz}
%\begin{macro}{\henry}
%\begin{macro}{\joule}
%\begin{macro}{\katal}
%\begin{macro}{\lumen}
%\begin{macro}{\lux}
%\begin{macro}{\newton}
% Derived units with specific names and symbols are defined.  Litre
% is an awkward one, but here the UK standard is used.
%    \begin{macrocode}
\ProvidesFile{si-named.cfg}
  [2008/12/22 v1.1j siunitx: SI Named units]
\newunit{\becquerel}{Bq}
\newunit{\coulomb}{C}
\newunit{\farad}{F}
\newunit{\Gray}{Gy}
\newunit{\ggray}{Gy}
\newunit{\hertz}{Hz}
\newunit{\henry}{H}
\newunit{\joule}{J}
\newunit{\katal}{kat}
\newunit{\lumen}{lm}
\newunit{\lux}{lx}
\newunit{\newton}{N}
%    \end{macrocode}
%\end{macro}
%\end{macro}
%\end{macro}
%\end{macro}
%\end{macro}
%\end{macro}
%\end{macro}
%\end{macro}
%\end{macro}
%\end{macro}
%\end{macro}
%\end{macro}
%\begin{macro}{\ohm}
%\begin{macro}{\Ohm}
%\begin{macro}{\pascal}
%\begin{macro}{\siemens}
%\begin{macro}{\sievert}
%\begin{macro}{\tesla}
%\begin{macro}{\volt}
%\begin{macro}{\watt}
%\begin{macro}{\weber}
% Some testing is needed for \unitsdef compatibility.
%    \begin{macrocode}
\ifsi@old@OHM
  \newunit{\Ohm}{\si@sym@Omega}
\else
  \ifsi@gensymb\else
%    \end{macrocode}
% To be on the safe side, use \cs{provideunit}.
%    \begin{macrocode}
    \provideunit{\ohm}{\si@sym@Omega}
  \fi
\fi
\newunit{\pascal}{Pa}
\newunit{\siemens}{S}
\newunit{\sievert}{Sv}
\newunit{\tesla}{T}
\newunit{\volt}{V}
\newunit{\watt}{W}
\newunit{\weber}{Wb}
%    \end{macrocode}
%\end{macro}
%\end{macro}
%\end{macro}
%\end{macro}
%\end{macro}
%\end{macro}
%\end{macro}
%\end{macro}
%\end{macro}
%\begin{macro}{\celsius}
%\begin{macro}{\Celsius}
% The degree celsius is a named unit.
%    \begin{macrocode}
\ifsi@old@OHM
  \newunit{\Celsius}{\si@sym@celsius}
\else
  \ifsi@gensymb\else
    \newunit{\celsius}{\si@sym@celsius}
  \fi
\fi
%    \end{macrocode}
%\end{macro}
%\end{macro}
%\begin{macro}{\radian}
%\begin{macro}{\steradian}
% The radian and steradian are officially derived units.
%    \begin{macrocode}
\newunit{\radian}{rad}
\newunit{\steradian}{sr}
%    \end{macrocode}
%\end{macro}
%\end{macro}
%\iffalse
%</named>
%<*prefixed>
%\fi
%\subsection{Units with prefixes}
% As in \unitsdef, some commonly used prefixed units are set up.
% This requires \file{si-prefix.cfg} and \file{si-named.cfg}.
%    \begin{macrocode}
\ProvidesFile{si-prefixed.cfg}
  [2008/12/22 v1.1j siunitx: SI Prefixed units]
\requiresiconfigs{prefix,named,accepted,physical}
%    \end{macrocode}
%\begin{macro}{\picometre}
%\begin{macro}{\nanometre}
%\begin{macro}{\micrometre}
%\begin{macro}{\millimetre}
%\begin{macro}{\centimetre}
%\begin{macro}{\decimetre}
%\begin{macro}{\kilometre}
% Extra distances.
%    \begin{macrocode}
\newunit{\picometre}{\pico\metre}
\newunit{\nanometre}{\nano\metre}
\newunit{\micrometre}{\micro\metre}
\newunit{\millimetre}{\milli\metre}
\newunit{\centimetre}{\centi\metre}
\newunit{\decimetre}{\deci\metre}
\newunit{\kilometre}{\kilo\metre}
%    \end{macrocode}
%\end{macro}
%\end{macro}
%\end{macro}
%\end{macro}
%\end{macro}
%\end{macro}
%\end{macro}
%\begin{macro}{\femtogram}
%\begin{macro}{\picogram}
%\begin{macro}{\nanogram}
%\begin{macro}{\microgram}
%\begin{macro}{\milligram}
% Extra masses.
%    \begin{macrocode}
\newunit{\femtogram}{\femto\gram}
\newunit{\picogram}{\pico\gram}
\newunit{\nanogram}{\nano\gram}
\newunit{\microgram}{\micro\gram}
\newunit{\milligram}{\milli\gram}
%    \end{macrocode}
%\end{macro}
%\end{macro}
%\end{macro}
%\end{macro}
%\end{macro}
%\begin{macro}{\femtomole}
%\begin{macro}{\picomole}
%\begin{macro}{\nanomole}
%\begin{macro}{\micromole}
%\begin{macro}{\millimole}
%\begin{macro}{\kilomole}
%\changes{v1.1e}{2008/11/18}{New unit}
% Now some moles.
%    \begin{macrocode}
\newunit{\femtomole}{\femto\mole}
\newunit{\picomole}{\pico\mole}
\newunit{\nanomole}{\nano\mole}
\newunit{\micromole}{\micro\mole}
\newunit{\millimole}{\milli\mole}
\newunit{\kilomole}{\kilo\mole}
%    \end{macrocode}
%\end{macro}
%\end{macro}
%\end{macro}
%\end{macro}
%\end{macro}
%\end{macro}
%\begin{macro}{\attosecond}
%\begin{macro}{\femtosecond}
%\begin{macro}{\picosecond}
%\begin{macro}{\nanosecond}
%\begin{macro}{\microsecond}
%\begin{macro}{\millisecond}
% Prefixed seconds.
%    \begin{macrocode}
\newunit{\attosecond}{\atto\second}
\newunit{\femtosecond}{\femto\second}
\newunit{\picosecond}{\pico\second}
\newunit{\nanosecond}{\nano\second}
\newunit{\microsecond}{\micro\second}
\newunit{\millisecond}{\milli\second}
%    \end{macrocode}
%\end{macro}
%\end{macro}
%\end{macro}
%\end{macro}
%\end{macro}
%\end{macro}
%\begin{macro}{\picoampere}
%\begin{macro}{\nanoampere}
%\begin{macro}{\microampere}
%\begin{macro}{\milliampere}
%\begin{macro}{\kiloampere}
% The last prefixed base units are amperes.
%    \begin{macrocode}
\newunit{\picoampere}{\pico\ampere}
\newunit{\nanoampere}{\nano\ampere}
\newunit{\microampere}{\micro\ampere}
\newunit{\milliampere}{\milli\ampere}
\newunit{\kiloampere}{\kilo\ampere}
%    \end{macrocode}
%\end{macro}
%\end{macro}
%\end{macro}
%\end{macro}
%\end{macro}
%\begin{macro}{\millivolt}
%\begin{macro}{\kilovolt}
%\begin{macro}{\milliwatt}
%\begin{macro}{\kilowatt}
%\begin{macro}{\megawatt}
%\begin{macro}{\femtofarad}
%\begin{macro}{\picofarad}
%\begin{macro}{\nanofarad}
%\begin{macro}{\microfarad}
%\begin{macro}{\millifarad}
%\begin{macro}{\millisiemens}
% More electricity-related units.
%    \begin{macrocode}
\newunit{\millivolt}{\milli\volt}
\newunit{\kilovolt}{\kilo\volt}
\newunit{\milliwatt}{\milli\watt}
\newunit{\kilowatt}{\kilo\watt}
\newunit{\megawatt}{\mega\watt}
\newunit{\femtofarad}{\femto\farad}
\newunit{\picofarad}{\pico\farad}
\newunit{\nanofarad}{\nano\farad}
\newunit{\microfarad}{\micro\farad}
\newunit{\millifarad}{\milli\farad}
\newunit{\millisiemens}{\milli\siemens}
%    \end{macrocode}
%\end{macro}
%\end{macro}
%\end{macro}
%\end{macro}
%\end{macro}
%\end{macro}
%\end{macro}
%\end{macro}
%\end{macro}
%\end{macro}
%\end{macro}
%\begin{macro}{\kilohm}
%\changes{v1.1f}{2008/11/25}{Renamed from \cs{kiloohm}}
%\begin{macro}{\megohm}
%\changes{v1.1f}{2008/11/25}{Renamed from \cs{megaohm}}
%\begin{macro}{\gigaohm}
% For resistance, checks are needed again for the definition of
% \cs{ohm}.
%    \begin{macrocode}
\ifsi@old@OHM
  \newunit{\kilohm}{\kilo\Ohm}
  \newunit{\megohm}{\mega\Ohm}
  \newunit{\gigaohm}{\giga\Ohm}
\else
  \ifsi@gensymb\else
  \newunit{\kilohm}{\kilo\ohm}
  \newunit{\megohm}{\mega\ohm}
  \newunit{\gigaohm}{\giga\ohm}
  \fi
\fi
%    \end{macrocode}
%\end{macro}
%\end{macro}
%\end{macro}
%\begin{macro}{\microlitre}
%\begin{macro}{\millilitre}
%\begin{macro}{\microliter}
%\begin{macro}{\milliliter}
%\begin{macro}{\cubicmetre}
%\begin{macro}{\cubiccentimetre}
%\begin{macro}{\centimetrecubed}
%\begin{macro}{\cubicmicrometre}
%\begin{macro}{\cubicmillimetre}
%\begin{macro}{\cubicdecimetre}
% Volumes (unlike \unitsdef, with litre and metre spelled
% correctly).  \cs{millilitre} and \cs{microlitre} are defined
% as they are the two officially-sanctioned prefixes for the litre.
%    \begin{macrocode}
\newunit{\microlitre}{\micro\litre}
\newunit{\millilitre}{\milli\litre}
\newunit{\microliter}{\micro\liter}
\newunit{\milliliter}{\milli\liter}
\newunit{\cubicmetre}{\metre\cubed}
\newunit{\cubiccentimetre}{\centi\metre\cubed}
\newunit{\centimetrecubed}{\centi\metre\cubed}
\newunit{\cubicmicrometre}{\micro\metre\cubed}
\newunit{\cubicmillimetre}{\milli\metre\cubed}
\newunit{\cubicdecimetre}{\cubic\deci\metre}
%    \end{macrocode}
%\end{macro}
%\end{macro}
%\end{macro}
%\end{macro}
%\end{macro}
%\end{macro}
%\end{macro}
%\end{macro}
%\end{macro}
%\end{macro}
%\begin{macro}{\squaremetre}
%\begin{macro}{\squarecentimetre}
%\begin{macro}{\centimetresquared}
%\begin{macro}{\squarekilometre}
% Areas, with metre spelled corrected; \cs{are} and \cs{hectare} are
% in the ``temporarily accepted'' file.
%    \begin{macrocode}
\newunit{\squaremetre}{\Square\metre}
\newunit{\squarecentimetre}{\Square\centi\metre}
\newunit{\centimetresquared}{\centi\metre\squared}
\newunit{\squarekilometre}{\Square\kilo\metre}
%    \end{macrocode}
%\end{macro}
%\end{macro}
%\end{macro}
%\end{macro}
%\begin{macro}{\millijoule}
%\begin{macro}{\kilojoule}
%\begin{macro}{\megajoule}
%\begin{macro}{\millielectronvolt}
%\begin{macro}{\kiloelectronvolt}
%\begin{macro}{\megaelectronvolt}
%\begin{macro}{\gigaelectronvolt}
%\begin{macro}{\teraelectronvolt}
%\begin{macro}{\kilowatthour}
% Some energy is needed by now!  Notice the definition of
% \cs{kilowatthour}
%    \begin{macrocode}
\newunit{\millijoule}{\milli\joule}
\newunit{\kilojoule}{\kilo\joule}
\newunit{\megajoule}{\mega\joule}
\newunit{\millielectronvolt}{\milli\electronvolt}
\newunit{\kiloelectronvolt}{\kilo\electronvolt}
\newunit{\megaelectronvolt}{\mega\electronvolt}
\newunit{\gigaelectronvolt}{\giga\electronvolt}
\newunit{\teraelectronvolt}{\tera\electronvolt}
\newunit[unitsep=none]{\kilowatthour}{\kilo\watt\hour}
%    \end{macrocode}
%\end{macro}
%\end{macro}
%\end{macro}
%\end{macro}
%\end{macro}
%\end{macro}
%\end{macro}
%\end{macro}
%\end{macro}
%\begin{macro}{\millihertz}
%\begin{macro}{\kilohertz}
%\begin{macro}{\megahertz}
%\begin{macro}{\gigahertz}
%\begin{macro}{\terahertz}
% Frequencies.
%    \begin{macrocode}
\newunit{\millihertz}{\milli\hertz}
\newunit{\kilohertz}{\kilo\hertz}
\newunit{\megahertz}{\mega\hertz}
\newunit{\gigahertz}{\giga\hertz}
\newunit{\terahertz}{\tera\hertz}
%    \end{macrocode}
%\end{macro}
%\end{macro}
%\end{macro}
%\end{macro}
%\end{macro}
%\begin{macro}{\millinewton}
%\begin{macro}{\kilonewton}
%\begin{macro}{\hectopascal}
%\begin{macro}{\megabecquerel}
%\begin{macro}{\millisievert}
% A few more from various areas.
%    \begin{macrocode}
\newunit{\millinewton}{\milli\newton}
\newunit{\kilonewton}{\kilo\newton}
\newunit{\hectopascal}{\hecto\pascal}
\newunit{\megabecquerel}{\mega\becquerel}
\newunit{\millisievert}{\milli\sievert}
%    \end{macrocode}
%\end{macro}
%\end{macro}
%\end{macro}
%\end{macro}
%\end{macro}
%\iffalse
%</prefixed>
%<*abbr>
%\fi
%\subsection{Abbreviated units}
%\begin{macro}{\pA}
%\begin{macro}{\nA}
%\begin{macro}{\micA}
%\begin{macro}{\mA}
%\begin{macro}{\kA}
% The abbreviated units are sorted in one file.  To allow
% back-compatibility with \unitsdef, each one is inside an \cs{if}
% block for the appropriate option.  First currents.
%    \begin{macrocode}
\ProvidesFile{si-abbr.cfg}
  [2008/12/22 v1.1j siunitx: Abbreviated units]
\requiresiconfigs{prefix,named,accepted,physical}
\newunit{\pA}{\pico\ampere}
\newunit{\nA}{\nano\ampere}
\newunit{\micA}{\micro\ampere}
\newunit{\mA}{\milli\ampere}
\newunit{\kA}{\kilo\ampere}
%    \end{macrocode}
%\end{macro}
%\end{macro}
%\end{macro}
%\end{macro}
%\end{macro}
%\begin{macro}{\Hz}
%\begin{macro}{\mHz}
%\begin{macro}{\kHz}
%\begin{macro}{\MHz}
%\begin{macro}{\GHz}
%\begin{macro}{\THz}
% Then frequencies.
%    \begin{macrocode}
\newunit{\Hz}{\hertz}
\newunit{\mHz}{\milli\hertz}
\newunit{\kHz}{\kilo\hertz}
\newunit{\MHz}{\mega\hertz}
\newunit{\GHz}{\giga\hertz}
\newunit{\THz}{\tera\hertz}
%    \end{macrocode}
%\end{macro}
%\end{macro}
%\end{macro}
%\end{macro}
%\end{macro}
%\end{macro}
%\begin{macro}{\fmol}
%\begin{macro}{\pmol}
%\begin{macro}{\nmol}
%\begin{macro}{\micmol}
%\begin{macro}{\mmol}
%\begin{macro}{\kmol}
%\changes{v1.1e}{2008/11/18}{New unit}
% Amounts of substance.
%    \begin{macrocode}
\newunit{\fmol}{\femto\mole}
\newunit{\pmol}{\pico\mole}
\newunit{\nmol}{\nano\mole}
\newunit{\micmol}{\micro\mole}
\newunit{\mmol}{\milli\mole}
\newunit{\kmol}{\kilo\mole}
%    \end{macrocode}
%\end{macro}
%\end{macro}
%\end{macro}
%\end{macro}
%\end{macro}
%\end{macro}
%\begin{macro}{\kV}
%\begin{macro}{\mV}
% Potentials.
%    \begin{macrocode}
\newunit{\kV}{\kilo\volt}
\newunit{\mV}{\milli\volt}
%    \end{macrocode}
%\end{macro}
%\end{macro}
%\begin{macro}{\ml}
%\begin{macro}{\micl}
%\begin{macro}{\mL}
%\begin{macro}{\micL}
%\begin{macro}{\cmc}
%\begin{macro}{\dmc}
%\begin{macro}{\cms}
% Volumes and areas
%    \begin{macrocode}
\provideunit{\ml}{\milli\litre}
\provideunit{\micl}{\micro\litre}
\provideunit{\mL}{\milli\liter}
\provideunit{\micL}{\micro\liter}
\newunit{\cmc}{\centi\metre\cubed}
\newunit{\dmc}{\deci\metre\cubed}
\newunit{\cms}{\centi\metre\squared}
%    \end{macrocode}
%\end{macro}
%\end{macro}
%\end{macro}
%\end{macro}
%\end{macro}
%\end{macro}
%\end{macro}
%\begin{macro}{\fg}
%\begin{macro}{\SIfg}
%\begin{macro}{\kg}
%\begin{macro}{\fg}
%\begin{macro}{\pg}
%\begin{macro}{\nanog}
%\begin{macro}{\micg}
%\begin{macro}{\mg}
%\begin{macro}{\amu}
% Masses.
%    \begin{macrocode}
\newunit{\kg}{\kilo\gram}
%    \end{macrocode}
% There is a name clash with \pkg{babel} here in French; hopefully
% there will not be too many complaints.
%    \begin{macrocode}
\AtBeginDocument{\provideunit{\fg}{\femto\gram}}
\newunit{\SIfg}{\femto\gram}
\newunit{\pg}{\pico\gram}
\newunit{\nanog}{\nano\gram}
\newunit{\micg}{\micro\gram}
\newunit{\mg}{\milli\gram}
\newunit{\amu}{\atomicmass}
%    \end{macrocode}
%\end{macro}
%\end{macro}
%\end{macro}
%\end{macro}
%\end{macro}
%\end{macro}
%\end{macro}
%\end{macro}
%\end{macro}
%\begin{macro}{\kJ}
%\begin{macro}{\eV}
%\begin{macro}{\meV}
%\begin{macro}{\keV}
%\begin{macro}{\MeV}
%\begin{macro}{\GeV}
%\begin{macro}{\TeV}
%\begin{macro}{\kWh}
% Energies.
%    \begin{macrocode}
\newunit{\kJ}{\kilo\joule}
\newunit{\eV}{\electronvolt}
\newunit{\meV}{\milli\electronvolt}
\newunit{\keV}{\kilo\electronvolt}
\newunit{\MeV}{\mega\electronvolt}
\newunit{\GeV}{\giga\electronvolt}
\newunit{\TeV}{\tera\electronvolt}
\newunit[unitsep=none]{\kWh}{\kilo\watt\hour}
%    \end{macrocode}
%\end{macro}
%\end{macro}
%\end{macro}
%\end{macro}
%\end{macro}
%\end{macro}
%\end{macro}
%\end{macro}
%\begin{macro}{\picom}
%\begin{macro}{\nm}
%\begin{macro}{\micm}
%\begin{macro}{\mm}
%\begin{macro}{\cm}
%\begin{macro}{\dm}
%\begin{macro}{\km}
% Lengths.
%    \begin{macrocode}
\newunit{\picom}{\pico\metre}
\newunit{\nm}{\nano\metre}
\newunit{\micm}{\micro\metre}
\newunit{\mm}{\milli\metre}
\newunit{\cm}{\centi\metre}
\newunit{\dm}{\deci\metre}
\newunit{\km}{\kilo\metre}
%    \end{macrocode}
%\end{macro}
%\end{macro}
%\end{macro}
%\end{macro}
%\end{macro}
%\end{macro}
%\end{macro}
%\begin{macro}{\Sec}
%\begin{macro}{\as}
%\begin{macro}{\fs}
%\begin{macro}{\ps}
%\begin{macro}{\ns}
%\begin{macro}{\mics}
%\begin{macro}{\ms}
% Finally, times.
%    \begin{macrocode}
\newunit{\Sec}{\second}
\newunit{\as}{\atto\second}
\newunit{\fs}{\femto\second}
%    \end{macrocode}
% The letter class (and others) define \cs{ps} for postscripts, so
% \cs{provideunit} is best here.
%    \begin{macrocode}
\provideunit{\ps}{\pico\second}
\newunit{\ns}{\nano\second}
\newunit{\mics}{\micro\second}
\newunit{\ms}{\milli\second}
%    \end{macrocode}
%\end{macro}
%\end{macro}
%\end{macro}
%\end{macro}
%\end{macro}
%\end{macro}
%\end{macro}
%\iffalse
%</abbr>
%<*addn>
%\fi
%\subsection{Additional (temporary) SI units}
%\begin{macro}{\angstrom}
%\begin{macro}{\are}
%\begin{macro}{\hectare}
%\begin{macro}{\barn}
%\begin{macro}{\BAR}
%\begin{macro}{\bbar}
%\begin{macro}{\millibar}
%\begin{macro}{\gal}
%\begin{macro}{\curie}
%\begin{macro}{\roentgen}
%\begin{macro}{\rad}
%\begin{macro}{\rem}
%\changes{v1.1c}{2008/10/22}{Uses \cs{provideunit}}
% Some units are ``temporarily'' acceptable for use in the SI system.
% These are defined here, although some are in very general use.
%    \begin{macrocode}
\ProvidesFile{si-addn.cfg}
  [2008/12/22 v1.1j siunitx: SI Additional units]
\newunit{\angstrom}{\si@sym@ringA}
\newunit{\are}{a}
\newunit{\hectare}{\hecto\are}
\newunit{\barn}{b}
\newunit{\BAR}{bar}
\newunit{\bbar}{bar}
\newunit{\millibar}{\milli\BAR}
\newunit{\gal}{Gal}
\newunit{\curie}{Ci}
\newunit{\roentgen}{R}
\newunit{\rad}{rad}
\provideunit{\rem}{rem}
%    \end{macrocode}
%\end{macro}
%\end{macro}
%\end{macro}
%\end{macro}
%\end{macro}
%\end{macro}
%\end{macro}
%\end{macro}
%\end{macro}
%\end{macro}
%\end{macro}
%\end{macro}
%\iffalse
%</addn>
%<*accepted>
%\fi
%\subsection{Units accepted for use with SI}
% The units which are accepted but do not fit in the above, plus
% \cs{percent} which seems to fit into this category.
%\begin{macro}{\minute}
%\begin{macro}{\hour}
%\begin{macro}{\Day}
%\begin{macro}{\dday}
%\begin{macro}{\degree}
%\changes{v1.1c}{2008/10/22}{Uses \cs{provideunit}}
%\begin{macro}{\Degree}
%\begin{macro}{\arcmin}
%\begin{macro}{\arcsec}
%\begin{macro}{\litre}
%\begin{macro}{\liter}
%\begin{macro}{\tonne}
%\begin{macro}{\neper}
%\begin{macro}{\bel}
%\begin{macro}{\percent}
%    \begin{macrocode}
\ProvidesFile{si-accepted.cfg}
  [2008/12/22 v1.1j siunitx: SI Accepted units]
\newunit{\minute}{min}
\newunit{\hour}{h}
\newunit{\Day}{d}
\newunit{\dday}{d}
\ifsi@old@OHM
  \AtBeginDocument{
    \provideunit[valuesep=none]{\Degree}{\si@sym@degree}
  }
\else
  \ifsi@gensymb\else
    \AtBeginDocument{
      \provideunit[valuesep=none]{\degree}{\si@sym@degree}
    }
  \fi
\fi
\newunit[valuesep=none]{\arcmin}{\si@sym@minute}
\newunit[valuesep=none]{\arcsec}{\si@sym@second}
\newunit{\litre}{l}
\newunit{\liter}{L}
\newunit{\tonne}{t}
\newunit{\neper}{Np}
\newunit{\bel}{B}
\newunit{\percent}{\%}
%    \end{macrocode}
%\end{macro}
%\end{macro}
%\end{macro}
%\end{macro}
%\end{macro}
%\end{macro}
%\end{macro}
%\end{macro}
%\end{macro}
%\end{macro}
%\end{macro}
%\end{macro}
%\end{macro}
%\end{macro}
%\iffalse
%</accepted>
%<*physical>
%\fi
%\subsection{Units based on physical measurements}
%\begin{macro}{\si@eVspacea}
%\begin{macro}{\si@eVspaceb}
%\begin{macro}{\electronvolt}
%\begin{macro}{\atomicmassunit}
%\begin{macro}{\atomicmass}
% A few units based on physical measurements exist.  For \cs{eV}, the need
% for a negative kern does make things a bit complicated.
%    \begin{macrocode}
\ProvidesFile{si-physical.cfg}
  [2008/12/22 v1.1j siunitx: SI Physically-measured units]
\newcommand*{\si@eVspacea}{\text{\kern-\si@eVcorra}}%
\newcommand*{\si@eVspaceb}{\text{\kern-\si@eVcorrb}}%
\newunit{\electronvolt}{e\protect\si@eVspacea V\protect%
  \si@eVspaceb}
\newunit{\atomicmass}{u}
\newunit{\atomicmassunit}{u}
%    \end{macrocode}
%\end{macro}
%\end{macro}
%\end{macro}
%\end{macro}
%\end{macro}
%\iffalse
%</physical>
%\fi
%
%\section{Additional configurations}
% To provide flexibility for people in specific areas, specialised
% units can be set up.  These are then stored separately to ease use.
%
%\iffalse
%<*synchem>
%\fi
%\subsection{Synthetic chemistry}
%\begin{macro}{\mmHg}
%\begin{macro}{\molar}
%\begin{macro}{\Molar}
%\begin{macro}{\torr}
%\begin{macro}{\dalton}
% Some useful units for synthetic chemists; although \cs{mmHg} and
% \cs{Molar} are outside of the SI system, they are used a lot.
% These are set up using \cs{provideunit} as people may have their
% own definitions.
%    \begin{macrocode}
\ProvidesFile{si-synchem.cfg}
  [2008/12/22 v1.1j siunitx: Units for synthetic chemists]
\requiresiconfigs{prefix}
\newunit{\mmHg}{mmHg}
\newunit{\molar}{\mole\per\cubic\deci\metre}
\newunit{\Molar}{\textsc{m}}
\newunit{\torr}{Torr}
\newunit{\dalton}{Da}
%    \end{macrocode}
%\end{macro}
%\end{macro}
%\end{macro}
%\end{macro}
%\end{macro}
%\iffalse
%</synchem>
%<*hep>
%\fi
%\subsection{High-energy physics}
% Some units inspired by \pkg{hepunits}.
%    \begin{macrocode}
\ProvidesFile{si-hep.cfg}
  [2008/12/22 v1.1j siunitx: Units for high-energy physics]
\requiresiconfigs{prefix,named}
%    \end{macrocode}
%\begin{macro}{\micron}
%\begin{macro}{\mrad}
%\begin{macro}{\gauss}
% These specific to high-energy physics, but are not defined
% elsewhere in \currpkg.\footnote{It is not clear from
% \pkg{hepunits}, but the assumption is that \cs{mrad} represents
% millirad and not milliradian.}
%    \begin{macrocode}
\provideunit{\micron}{\micro\metre}
\newunit{\mrad}{\milli\rad}
\newunit{\gauss}{G}
%    \end{macrocode}
%\end{macro}
%\end{macro}
%\end{macro}
%\begin{macro}{\clight}
%\begin{macro}{\eVperc}
% The speed of light is used in units for the area, although of
% course it is not strictly a unit.  However, this makes it easier to
% use in other units.
%    \begin{macrocode}
\newunit{\clight}{\ensuremath{\mathnormal{c}}}
\newunit{\eVperc}{\eV\per\clight}
%    \end{macrocode}
%\end{macro}
%\end{macro}
%\begin{macro}{\nanobarn}
%\begin{macro}{\picobarn}
%\begin{macro}{\femtobarn}
%\begin{macro}{\attobarn}
%\begin{macro}{\zeptobarn}
%\begin{macro}{\yoctobarn}
% Various prefixed barns.
%    \begin{macrocode}
\newunit{\nanobarn}{\nano\barn}
\newunit{\picobarn}{\pico\barn}
\newunit{\femtobarn}{\femto\barn}
\newunit{\attobarn}{\atto\barn}
\newunit{\zeptobarn}{\zepto\barn}
\newunit{\yoctobarn}{\yocto\barn}
%    \end{macrocode}
%\end{macro}
%\end{macro}
%\end{macro}
%\end{macro}
%\end{macro}
%\end{macro}
%\begin{macro}{\nb}
%\begin{macro}{\pb}
%\begin{macro}{\fb}
%\begin{macro}{\ab}
%\begin{macro}{\zb}
%\begin{macro}{\yb}
% Abbreviations for the same, but using \cs{provideunit} to avoid any
% clashes.
%    \begin{macrocode}
\provideunit{\nb}{\nano\barn}
\provideunit{\pb}{\pico\barn}
\provideunit{\fb}{\femto\barn}
\provideunit{\ab}{\atto\barn}
\provideunit{\zb}{\zepto\barn}
\provideunit{\yb}{\yocto\barn}
%    \end{macrocode}
%\end{macro}
%\end{macro}
%\end{macro}
%\end{macro}
%\end{macro}
%\end{macro}
%\iffalse
%</hep>
%<*astro>
%\fi
%\subsection{Astronomy}
%\begin{macro}{\parsec}
%\begin{macro}{\lightyear}
% For astronomy, the \cs{parsec} unit is needed.
%    \begin{macrocode}
\ProvidesFile{si-astro.cfg}
  [2008/12/22 v1.1j siunitx: Units for astronomy]
\newunit{\parsec}{pc}
\newunit{\lightyear}{ly}
%    \end{macrocode}
%\end{macro}
%\end{macro}
%\iffalse
%</astro>
%<*geophys>
%\fi
%\subsection{Geophysics}
%\begin{macro}{\lightyear}
% The unit \cs{gon} is useful here.
%    \begin{macrocode}
\ProvidesFile{si-geophys.cfg}
  [2008/12/22 v1.1j siunitx: Units for astronomy]
\newunit{\gon}{gon}
%    \end{macrocode}
%\end{macro}
%\iffalse
%</geophys>
%<*binary>
%\fi
%\subsection{Binary units}
%\begin{macro}{\kibi}
%\begin{macro}{\mebi}
%\begin{macro}{\gibi}
%\begin{macro}{\tebi}
%\begin{macro}{\pebi}
%\begin{macro}{\exbi}
%\begin{macro}{\zebi}
%\begin{macro}{\yobi}
% The binary units, as specified by the IEC and made available by
% \SIunits. First, the binary prefixes.
%    \begin{macrocode}
\ProvidesFile{si-binary.cfg}
  [2008/12/22 v1.1j siunitx: Binary units]
\newprefix[binary]{\kibi}{10}{Ki}
\newprefix[binary]{\mebi}{20}{Mi}
\newprefix[binary]{\gibi}{30}{Gi}
\newprefix[binary]{\tebi}{40}{Ti}
\newprefix[binary]{\pebi}{50}{Pi}
\newprefix[binary]{\exbi}{60}{Ei}
\newprefix[binary]{\zebi}{70}{Zi}
\newprefix[binary]{\yobi}{80}{Yi}
%    \end{macrocode}
%\end{macro}
%\end{macro}
%\end{macro}
%\end{macro}
%\end{macro}
%\end{macro}
%\end{macro}
%\end{macro}
%\begin{macro}{\bit}
%\begin{macro}{\byte}
% Now the units.
%    \begin{macrocode}
\newunit{\bit}{bit}
\newunit{\byte}{B}
%    \end{macrocode}
%\end{macro}
%\end{macro}
%
%\iffalse
%</binary>
%\fi
%
%\section{Loadable locales}
% Some short files to provide the correct settings for various places.
%
%\iffalse
%<*UK>
%\fi
%\subsection{United Kingdom}
% This is identical to the USA locale, and contains the package
% default values.\footnote{The package author is in the UK.}
%    \begin{macrocode}
\ProvidesFile{si-UK.cfg}
  [2008/12/22 v1.1j siunitx: UK locale]
\sisetup{
  unitsep=thin,
  expproduct=times,
  valuesep=thin,
  decimalsymbol=fullstop,
  digitsep=thin,
  sepfour=false}
%    \end{macrocode}
%\iffalse
%</UK>
%<*USA>
%\fi
%\subsection{United States}
% The same as for the UK.
%    \begin{macrocode}
\ProvidesFile{si-USA.cfg}
  [2008/12/22 v1.1j siunitx: USA locale]
\sisetup{
  unitsep=thin,
  expproduct=times,
  valuesep=thin,
  decimalsymbol=fullstop,
  digitsep=thin,
  sepfour=false}
%    \end{macrocode}
%\iffalse
%</USA>
%<*DE>
%\fi
%\subsection{Germany}
% Germany, hopefully.
%    \begin{macrocode}
\ProvidesFile{si-DE.cfg}
  [2008/12/22 v1.1j siunitx: Germany locale]
\sisetup{
  unitsep=cdot,
  valuesep=thin,
  decimalsymbol=comma,
  expproduct=cdot,
  digitsep=thin,
  sepfour=false}
%    \end{macrocode}
%\iffalse
%</DE>
%<*ZA>
%\fi
%\subsection{South Africa}
% Design decisions taken from the same section of \SIstyle.
%    \begin{macrocode}
\ProvidesFile{si-ZA.cfg}
  [2008/12/22 v1.1j siunitx: South Africa locale]
\sisetup{
  unitsep=cdot,
  valuesep=thin,
  expproduct=times,
  decimalsymbol=comma,
  digitsep=thin,
  sepfour=false}
%    \end{macrocode}
%\iffalse
%</ZA>
%\fi
%
%\section{Emulation code}
% Each emulation mode loads an appropriate definition file.  This
% then alters the package defaults, and defines new macros provided
% by the emulated package.
%
%\iffalse
%<*units>
%\fi
%\subsection{\units}
% The very first thing to do here is a reload check, as things
% could go wrong with \unitsdef emulation.
%    \begin{macrocode}
\si@ifloaded{units}{\endinput}{}
%    \end{macrocode}
% The \units package is quite easy to emulate, as it only has a
% few options and macros. There is also no error checking in
% \units for conflicting options, so users probably expect none.
%    \begin{macrocode}
\ProvidesFile{si-units.cfg}
  [2008/12/22 v1.1j siunitx: Emulation of units]
\si@emulating{units}{1998/08/04 v0.9b}
\si@ifloaded{SIunits}
  {\si@emclash{units}{SIunits}\endinput}{}
\si@ifloaded{sistyle}
  {\si@emclash{units}{sistyle}\endinput}{}
%    \end{macrocode}
% To emulate \units, \cs{per} must give fractions.
%    \begin{macrocode}
\sisetup{per=fraction,fraction=nice,obeybold,inlinebold=maths,
  ,obeymode}
\ifsi@old@tight
  \sisetup{valuesep=thin}
\fi
\ifsi@old@loose
  \sisetup{valuesep=space}
\fi
\ifsi@old@ugly
  \sisetup{fraction=ugly}
\fi
%    \end{macrocode}
%\begin{macro}{\unit}
%\doarg{number}
%\darg{unit}
% The \units version of \cs{unit} is similar to \cs{SI}.  Here and
% in \cs{unitfrac} the \cs{num} macro is used; thus the number given
% really has to be a number.  However, if users are using \currpkg
% rather than \units they should expect more checking of input.  As
% the \units package uses the current mode, this has to be
% detected.
%    \begin{macrocode}
\si@newrobustcmd*{\unit}[2][]{%
  \ifmmode
    \SI{#1}{#2}%
  \else
    \SI[obeyfamily,obeyitalic]{#1}{#2}%
  \fi}
%    \end{macrocode}
%\end{macro}
%\begin{macro}{\unitfrac}
%\doarg{number}
%\darg{numerator}
%\darg{denominator}
% \cs{unitfrac} is a bit more of a hack.
%    \begin{macrocode}
\si@newrobustcmd*{\unitfrac}[3][]{%
  \begingroup
    \si@fam@mode%
    \ifmmode\else
      \sisetup{obeyfamily,obeyitalic}%
    \fi
    \si@ifnotmtarg{#1}
      {\num{#1}\ensuremath{\si@valuesep}}%
    \si@frac{#2}{#3}
  \endgroup}
%    \end{macrocode}
%\end{macro}
%\iffalse
%</units>
%\fi
%\iffalse
%<*unitsdef>
%\fi
%\subsection{\unitsdef}
% The package begins with the usual identification of what is
% happening.  Although \file{si-units.cfg} makes the same checks,
% the error will make more sense if it comes here, in the event of a
% clash.
%    \begin{macrocode}
\ProvidesFile{si-unitsdef.cfg}
  [2008/12/22 v1.1j siunitx:
   Emulation of unitsdef]
\si@emulating{unitsdef}{2005/01/04 v0.2}
\si@ifloaded{SIunits}
  {\si@emclash{unitsdef}{SIunits}\endinput}{}
\si@ifloaded{sistyle}
  {\si@emclash{unitsdef}{sistyle}\endinput}{}
%    \end{macrocode}
% Emulation of \units is needed for \unitsdef to work.
%    \begin{macrocode}
\requiresiconfigs{units}
%    \end{macrocode}
% The \unitsdef package loads some packages that \currpkg does not.
% In particular, it loads \pkg{textcomp} and \pkg{fontenc}.  This
% could be important for output, and so the same is done here.
%    \begin{macrocode}
\RequirePackage{textcomp}
\RequirePackage[T1]{fontenc}
%    \end{macrocode}
% The multitude of package options for \unitsdef need to be handled.
%    \begin{macrocode}
\sisetup{mode=text,allowoptarg,prespace}
\ifsi@old@noxspace
  \sisetup{xspace=false}
\fi
%    \end{macrocode}
% The various options for loading unit abbreviations have to be
% handled. Here, any request to avoid abbreviations prevents any
% loading.
%    \begin{macrocode}
\ifsi@old@noabbr
  \sisetup{noload=abbr}
\fi
\ifsi@old@nofrequncyabbr
  \sisetup{noload=abbr}
\fi
\ifsi@old@nomolabbr
  \sisetup{noload=abbr}
\fi
\ifsi@old@novoltageabbr
  \sisetup{noload=abbr}
\fi
\ifsi@old@novolumeabbr
  \sisetup{noload=abbr}
\fi
\ifsi@old@noweightabbr
  \sisetup{noload=abbr}
\fi
\ifsi@old@noenergyabbr
  \sisetup{noload=abbr}
\fi
\ifsi@old@nolengthabbr
  \sisetup{noload=abbr}
\fi
\ifsi@old@notimeabbr
  \sisetup{noload=abbr}
\fi
%    \end{macrocode}
%\begin{macro}{\unitvaluesep}
% To emulate the \cs{unitvaluesep} macro, a hack is needed of the original
% \pkg{xkeyval} macro for \opt{valuesep}, as well of course as a definition
% of the macro itself.
%    \begin{macrocode}
\newcommand*{\unitvaluesep}{\,}
\renewcommand*{\si@valuesep}{\text{\unitvaluesep}}
\define@choicekey*+[si]{key}{valuesep}[\si@tempa]
  {space,thin,med,medium,thick,none}
  {\renewcommand*\unitvaluesep\@nameuse{si@fix@##1}%
   \si@log@debug{Option valuesep set to ##1}}
  {\si@log@debug{Option valuesep set to ##1}%
   \renewcommand*\unitvaluesep{##1}}
%    \end{macrocode}
%\end{macro}
%\begin{macro}{\unitsignonly}
%\begin{macro}{\ilu}
%\begin{macro}{\arc}
% Some rather straight-forward definitions, with just a bit of fun to
% get the spacing correct.
%    \begin{macrocode}
\let\unitsignonly\si
\si@newrobustcmd*{\ilu}[2][]{%
  \begingroup
    #1\unitvaluesep%
    \unit{#2}%
  \endgroup}
\let\arc\ang
%    \end{macrocode}
%\end{macro}
%\end{macro}
%\end{macro}
%\begin{macro}{\unitSIdef}
%\begin{macro}{\si@unitSIdef}
% The \unitsdef package uses a different approach to setting the font
% inside its version of \cs{SI}.  The problem is the same as for
% \cs{unitvaluesep}, but with the added problem that \currpkg uses
% \cs{csname} \ldots \cs{endcsname}.
%    \begin{macrocode}
\newcommand*{\unitSIdef}{\upshape}
\newcommand*{\si@unitSIdef}{\unitSIdef\selectfont}
\sisetup{textrm=si@unitSIdef}
%    \end{macrocode}
%\end{macro}
%\end{macro}
%\begin{macro}{\per}
% Rather awkwardly, \unitsdef uses \cs{per} in a different way to
% \currpkg.
%    \begin{macrocode}
\let\per\relax
\si@newrobustcmd*{\per}[2]{%
  \begingroup
    \si@xspacefalse
    \renewcommand*{\unitvaluesep}{}%
    \unitfrac{#1}{#2}%
  \endgroup}
%    \end{macrocode}
%\end{macro}
%\begin{macro}{\unittimes}
%\begin{macro}{\unitsep}
%\begin{macro}{\unitsuperscript}
% Some pretty straight-forward stuff again; notice that the automatic
% analyser for units has to be turned off for this to work.
%    \begin{macrocode}
\newcommand*{\unittimes}{\ensuremath{\cdot}}
\newcommand*{\unitsep}{\,}
\renewcommand*{\si@unt@unithook}{\si@unt@litouttrue}
\sisetup{unitsep=none}
\newcommand*{\unitsuperscript}{\tothe}
%    \end{macrocode}
%\end{macro}
%\end{macro}
%\end{macro}
%\begin{macro}{\newnosepunit}
%\begin{macro}{\renewnosepunit}
% Simple aliases.
%    \begin{macrocode}
\newcommand*{\newnosepunit}{\newunit[valuesep=none]}
\newcommand*{\renewnosepunit}{\renewunit[valuesep=none]}
%    \end{macrocode}
%\end{macro}
%\end{macro}
%\begin{macro}{\setTextOmega}
%\begin{macro}{\setMathOmega}
%\begin{macro}{\setTextmu}
%\begin{macro}{\setMathmu}
%\begin{macro}{\setTextCelsius}
%\begin{macro}{\setMathCelsius}
%\begin{macro}{\setMathDegree}
%\begin{macro}{\setTextDegree}
% Controlling symbols is a simple translation job; as only one
% setting is used by \currpkg in text mode, a bit of extra work is
% needed.
%    \begin{macrocode}
\newcommand*{\setTextOmega}[2]{%
  \renewcommand*{\si@textOmega}{%
      \begingroup
      \edef\si@tempa{\sfdefault}%
      \ifx\f@family\si@tempa
        \expandafter#2%
      \else
        \expandafter#1%
      \fi
    \endgroup}}
\newcommand*{\setMathOmega}[1]{\sisetup{mathsOmega=#1}}
\newcommand*{\setTextmu}[2]{%
  \renewcommand*{\si@textmu}{%
    \begingroup
      \edef\si@tempa{\sfdefault}%
      \ifx\f@family\si@tempa
        \expandafter#2%
      \else
        \expandafter#1%
      \fi
    \endgroup}}
\newcommand*{\setMathmu}[1]{\sisetup{mathsmu=#1}}
\newcommand*{\setTextCelsius}[2]{%
  \renewcommand*{\si@textcelsius}{%
    \begingroup
      \edef\si@tempa{\sfdefault}%
      \ifx\f@family\si@tempa
        \expandafter#2%
      \else
        \expandafter#1%
      \fi
    \endgroup}}
\newcommand*{\setMathCelsius}[1]{\sisetup{mathscelsius=#1}}
\newcommand*{\setMathDegree}[2]{%
  \renewcommand*{\si@textdegree}{%
    \begingroup%
      \edef\si@tempa{\sfdefault}%
      \ifx\f@family\si@tempa
        \expandafter#2%
      \else
        \expandafter#1%
      \fi
    \endgroup}}
\newcommand*{\setTextDegree}[1]{\sisetup{textdegree=#1}}
%    \end{macrocode}
%\end{macro}
%\end{macro}
%\end{macro}
%\end{macro}
%\end{macro}
%\end{macro}
%\end{macro}
%\end{macro}
% The \opt{ohm} and \opt{OHM} options are checked, and some sanity is
% ensured.  This needs to happen before loading the configuration
% files.
%    \begin{macrocode}
\ifsi@old@OHM
  \ifsi@old@ohm
    \si@log@inf{Both `ohm' and `OHM' options given\MessageBreak
      Using default behaviour for unitsdef}
    \expandafter\expandafter\expandafter\si@old@OHMfalse
  \fi
\fi
%    \end{macrocode}
%\begin{macro}{\liter}
%\begin{macro}{\ton}
%\begin{macro}{\days}
% Tonne is spelled as ``ton'' by \unitsdef, which is wrong in the UK
% at least (\SI{1}{ton} $=$ \SI{40}{cwt} $=$ \SI{2240}{lb}!).
%    \begin{macrocode}
\ifsi@old@liter
  \ifsi@old@LITER
    \si@log@inf{Both `liter' and `LITER' options
      given\MessageBreak Using default behaviour for unitsdef}
  \else
    \renewunit{\liter}{l}
  \fi
\fi
\newunit{\ton}{t}
\newunit{\days}{d}
%    \end{macrocode}
%\end{macro}
%\end{macro}
%\end{macro}
%\begin{macro}{\picometer}
%\begin{macro}{\nanometer}
%\begin{macro}{\micrometer}
%\begin{macro}{\millimeter}
%\begin{macro}{\centimeter}
%\begin{macro}{\decimeter}
%\begin{macro}{\kilometer}
% Extra distances.
%    \begin{macrocode}
\newunit{\picometer}{\pico\meter}
\newunit{\nanometer}{\nano\meter}
\newunit{\micrometer}{\micro\meter}
\newunit{\millimeter}{\milli\meter}
\newunit{\centimeter}{\centi\meter}
\newunit{\decimeter}{\deci\meter}
\newunit{\kilometer}{\kilo\meter}
%    \end{macrocode}
%\end{macro}
%\end{macro}
%\end{macro}
%\end{macro}
%\end{macro}
%\end{macro}
%\end{macro}
%\begin{macro}{\femtoliter}
%\begin{macro}{\picoliter}
%\begin{macro}{\nanoliter}
%\begin{macro}{\microliter}
%\begin{macro}{\milliliter}
%\begin{macro}{\centiliter}
%\begin{macro}{\deciliter}
%\begin{macro}{\hectoliter}
%\begin{macro}{\cubicmeter}
%\begin{macro}{\cubicmicrometer}
%\begin{macro}{\cubicmillimeter}
% Volumes with US spellings.
%    \begin{macrocode}
\newunit{\femtoliter}{\femto\liter}
\newunit{\picoliter}{\pico\liter}
\newunit{\nanoliter}{\nano\liter}
\newunit{\microliter}{\micro\liter}
\newunit{\milliliter}{\milli\liter}
\newunit{\centiliter}{\centi\liter}
\newunit{\deciliter}{\deci\liter}
\newunit{\hectoliter}{\hecto\liter}
\newunit{\cubicmeter}{\meter\cubed}
\newunit{\cubicmicrometer}{\micro\meter\cubed}
\newunit{\cubicmillimeter}{\milli\meter\cubed}
%    \end{macrocode}
%\end{macro}
%\end{macro}
%\end{macro}
%\end{macro}
%\end{macro}
%\end{macro}
%\end{macro}
%\end{macro}
%\end{macro}
%\end{macro}
%\end{macro}
%\begin{macro}{\squaremeter}
%\begin{macro}{\squarecentimeter}
%\begin{macro}{\squarekilometer}
%\begin{macro}{\ar}
%\begin{macro}{\hectar}
% Areas, including the mis-spellings for \cs{are} and
% \cs{hectare}.
%    \begin{macrocode}
\newunit{\squaremeter}{\Square\meter}
\newunit{\squarecentimeter}{\Square\centi\meter}
\newunit{\squarekilometer}{\Square\kilo\meter}
\newunit{\ar}{a}
\newunit{\hectar}{\hecto\ar}
%    \end{macrocode}
%\end{macro}
%\end{macro}
%\end{macro}
%\end{macro}
%\end{macro}
%\begin{macro}{\kv}
%\begin{macro}{\mv}
% The code for \unitsdef has the capitalisation wrong for \cs{kV} and
% \cs{mV}.
%    \begin{macrocode}
\ifsi@old@noabbr
\else
  \ifsi@old@novoltageabbr\else
    \newunit{\kv}{\kilo\volt}
    \newunit{\mv}{\milli\volt}
  \fi
\fi
%    \end{macrocode}
%\end{macro}
%\end{macro}
%\begin{macro}{\sek}
%\begin{macro}{\fg}
%\begin{macro}{\fl}
%\begin{macro}{\pl}
%\begin{macro}{\nl}
%\begin{macro}{\micl}
%\begin{macro}{\ml}
%\begin{macro}{\cl}
%\begin{macro}{\dl}
%\begin{macro}{\hl}
% There are some slightly different abbreviations, plus some which
% are not officially allowed.
%    \begin{macrocode}
\ifsi@old@noabbr\else
  \ifsi@old@notimeabbr\else
    \newunit{\sek}{\second}
  \fi
  \ifsi@old@noweightabbr\else
    \newunit{\fg}{\femto\gram}
  \fi
  \ifsi@old@novolumeabbr\else
    \newunit{\fl}{\femto\liter}
    \newunit{\pl}{\pico\liter}
    \newunit{\nl}{\nano\liter}
    \newunit{\micl}{\micro\liter}
    \newunit{\ml}{\milli\liter}
    \newunit{\cl}{\centi\liter}
    \newunit{\dl}{\deci\liter}
    \newunit{\hl}{\hecto\liter}
  \fi
\fi
%    \end{macrocode}
%\end{macro}
%\end{macro}
%\end{macro}
%\end{macro}
%\end{macro}
%\end{macro}
%\end{macro}
%\end{macro}
%\end{macro}
%\end{macro}
%\begin{macro}{\calory}
%\begin{macro}{\kilocalory}
% \unitsdef spells calorie incorrectly, and it is also not an SI unit.
%    \begin{macrocode}
\newunit{\calory}{cal}
\newunit{\kilocalory}{\kilo\calory}
%    \end{macrocode}
%\end{macro}
%\end{macro}
%\begin{macro}{\uBar}
% \unitsdef uses \cs{ubar} for bar.
%    \begin{macrocode}
\newunit{\uBar}{ba}
%    \end{macrocode}
%\end{macro}
%\begin{macro}{\gensymbohm}
%\begin{macro}{\gensymbcelsius}
%\begin{macro}{\gensymbmicro}
%\begin{macro}{\gensymbdegree}
% If the options relating to \pkg{gensymb} are given, then the
% package \emph{has} to be loaded.  The definitions are then renamed;
% a slight awkward feature is that the hyphen character needs to be a
% letter.  To avoid needing to worry about this again, a second
% switch is set up.
%    \begin{macrocode}
\catcode`\-=11\relax
\ifsi@old@redef-gensymb
  \expandafter\si@gensymbtrue
\fi
\catcode`\-=12\relax
\ifsi@gensymb
  \RequirePackage{gensymb}
  \AtBeginDocument{
    \let\gensymbohm\ohm
    \let\gensymbcelsius\celsius
    \let\gensymbmicro\micro
    \let\gensymbdegree\degree
    \let\ohm\@undefined
    \let\celsius\@undefined
    \let\micro\@undefined
    \let\degree\@undefined
    \ifsi@old@OHM\else
      \newunit{\ohm}{\si@sym@Omega}
      \newunit{\celsius}{\si@sym@celsius}
      \newprefix{\micro}{\si@sym@mu}{-6}
      \newunit{\degree}{\si@sym@degree}
    \fi}
\fi
%    \end{macrocode}
%\end{macro}
%\end{macro}
%\end{macro}
%\end{macro}
% The configuration files can now be loaded.
%    \begin{macrocode}
\requiresiconfigs{prefix,named,addn,accepted}
%    \end{macrocode}
% The \opt{noconfig} option could be ignored, but it costs little to
% let is be used.
%    \begin{macrocode}
\ifsi@old@noconfig\else
  \InputIfFileExists{unitsdef.cfg}
    {\si@log@inf{unitsdef config file loaded}}
    {\si@log@inf{unitsdef config file not found}}
\fi
%    \end{macrocode}
%
%\iffalse
%</unitsdef>
%<*sistyle>
%\fi
%\subsection{\SIstyle}
% After setting the necessary defaults, the emulation code defines
% the macros in \SIstyle as given in the manual for that package.
%    \begin{macrocode}
\ProvidesFile{si-sistyle.cfg}
  [2008/12/22 v1.1j siunitx: Emulation of
     SIstyle]
\si@emulating{sistyle}{2006/12/20 v2.3}
\sisetup{%
  sepfour=true,
  obeyfamily,
  obeyitalic=true,
  numsign=+-,
  strictarc=false,
  unitsep=cdot}
%    \end{macrocode}
%\begin{macro}{\SIobeyboldtrue}
%\begin{macro}{\SIobeyboldfalse}
% Some simple switches, but not using \cs{newif}.
%    \begin{macrocode}
\newcommand*{\SIobeyboldtrue}{\sisetup{obeybold=true}}
\newcommand*{\SIobeyboldfalse}{\sisetup{obeybold=false}}
%    \end{macrocode}
%\end{macro}
%\end{macro}
%\begin{macro}{\num}
%\begin{macro}{\si@sis@num}
%\begin{macro}{\si@sis@numstar}
% To get the correct behaviour for \cs{num}, some redefinitions are
% needed to handle to optional |*|.
%    \begin{macrocode}
\let\num\relax
\si@newrobustcmd*{\num}{%
  \@ifstar
    {\si@sis@numstar}
    {\si@sis@num}}
\newcommand*{\si@sis@num}[2][]{%
  \begingroup%
    \sisetup{#1}%
    \expandafter\si@out@num\expandafter{\si@num{#2}}%
  \endgroup}
\newcommand*{\si@sis@numstar}[2][]{%
  \begingroup%
    \sisetup{mode=text,obeybold}%
    \sisetup{#1}%
    \expandafter\si@out@num\expandafter{\si@num{#2}}%
  \endgroup}
%    \end{macrocode}
%\end{macro}
%\end{macro}
%\end{macro}
%\begin{macro}{\pnt}
% The \cs{pnt} macro is needed as |.| is active inside \cs{SI}.  The
% name is exactly the same as in \SIstyle, but the implementation is
% different.  This is not defined by the main package as there are
% better ways of including numbers in the output than this.
%    \begin{macrocode}
\newcommand*{\pnt}{\ensuremath{\si@decimalsymbol}}
%    \end{macrocode}
%\end{macro}
%\begin{macro}{\SIgroupfourtrue}
%\begin{macro}{\SIgroupfourfalse}
% Switches for grouping four characters.
%    \begin{macrocode}
\newcommand*{\SIgroupfourtrue}{\sisetup{sepfour=true}}
\newcommand*{\SIgroupfourfalse}{\sisetup{sepfour=false}}
%    \end{macrocode}
%\end{macro}
%\end{macro}
%\begin{macro}{\SIunitsep}
%\begin{macro}{\SIunitspace}
%\begin{macro}{\SIunitdot}
% Whatever is given here is passed through to \cs{sisetup}.
%    \begin{macrocode}
\newcommand*{\SIunitsep}[1]{\sisetup{valuesep={#1}}}
\newcommand*{\SIunitspace}[1]{\sisetup{unitspace={#1}}}
\newcommand*{\SIunitdot}[1]{\sisetup{unitsep={#1}}}
%    \end{macrocode}
%\end{macro}
%\end{macro}
%\end{macro}
%\begin{macro}{\SIdecimalsymbol}
%\begin{macro}{\SIthousandsep}
%\begin{macro}{\SIproductsign}
%\begin{macro}{\SIdecimalsign}
% The same is true here, with the appropriate translation.
%    \begin{macrocode}
\newcommand*{\SIdecimalsymbol}[1]{\sisetup{decimalsymbol={#1}}}
\newcommand*{\SIthousandsep}[1]{\sisetup{digitsep={#1}}}
\newcommand*{\SIproductsign}[1]{\sisetup{expproduct={#1}}}
\newcommand*{\SIdecimalsign}[1]{\sisetup{decimalsymbol={#1}}}
%    \end{macrocode}
%\end{macro}
%\end{macro}
%\end{macro}
%\end{macro}
%\begin{macro}{\si@sis@savefont}
%\darg{setting}
%\darg{argument}
% The font definitions need a bit of extra work doing.  As both
% settings here have |@| as a letter, all should be fine.
%    \begin{macrocode}
\newcommand*{\si@sis@savefont}[2]{%
  \@namedef{si@sis@#1}{#2}%
  \sisetup{#1=si@sis@#1}}
%    \end{macrocode}
%\end{macro}
%\begin{macro}{\SImathrm}
%\begin{macro}{\SImathsf}
%\begin{macro}{\SImathtt}
% The font control macros have to ensure that a macro name is passed
% to \cs{sisetup}.
%    \begin{macrocode}
\newcommand*{\SImathrm}[1]{\si@sis@savefont{mathrm}{#1}}
\newcommand*{\SImathsf}[1]{\si@sis@savefont{mathsf}{#1}}
\newcommand*{\SImathtt}[1]{\si@sis@savefont{mathtt}{#1}}
%    \end{macrocode}
%\end{macro}
%\end{macro}
%\end{macro}
%\begin{macro}{\SIdefaultMfam}
%\begin{macro}{\SIdefaultNfam}
%\begin{macro}{\SIdefaultTfam}
% The same for the default keys.
%    \begin{macrocode}
\newcommand*{\SIdefaultMfam}[1]{\si@sis@savefont{mathrm}{#1}}
\newcommand*{\SIdefaultNfam}[1]{\si@sis@savefont{mathnumrm}{#1}}
\newcommand*{\SIdefaultTfam}[1]{\si@sis@savefont{textrm}{#1}}
%    \end{macrocode}
%\end{macro}
%\end{macro}
%\end{macro}
%\begin{macro}{\ensureupmath}
% The \cs{ensureupmath} command guarantees processing by the
% font-matching system. The argument cannot be processed here, so
% care is needed.
%    \begin{macrocode}
\si@newrobustcmd*{\ensureupmath}[1]{%
  \begingroup
    \sisetup{mode=maths,obeyitalic=false}%
    \si@out{#1}%
  \endgroup}
%    \end{macrocode}
%\end{macro}
%\begin{macro}{\degC}
%\begin{macro}{\degF}
%\begin{macro}{\arcdeg}
% A few extra symbol names are needed.
%    \begin{macrocode}
\newcommand*{\degC}{\si@sym@celsius}
\newcommand*{\arcdeg}{\si@sym@degree}
\newcommand*{\degF}{\si@sym@degree F}
%    \end{macrocode}
%\end{macro}
%\end{macro}
%\end{macro}
%\begin{macro}{\AddToSIstyle}
%\begin{macro}{\SIstyle}
%\begin{macro}{\SIstyleToLang}
%\begin{macro}{\si@sis@addtolocale}
% Finally, the locale control.
%    \begin{macrocode}
\newcommand*{\SIstyle}[1]{\sisetup{locale=#1}}
\newcommand*{\SIstyleToLang}[2]{\sisetup{loctolang=#1:#2}}
\newcommand*{\AddToSIstyle}{%
  \si@switchfalse
  \@ifstar
    {\si@switchtrue
     \si@sis@addtolocale}
    {\si@sis@addtolocale}}
\newcommand*{\si@sis@addtolocale}[2]{%
  \ifsi@switch
    \expandafter\let\csname si@loc@#1@extra\endcsname\relax
  \fi
  \addtolocale{#1}{#2}}
%    \end{macrocode}
%\end{macro}
%\end{macro}
%\end{macro}
%\end{macro}
%
%\iffalse
%</sistyle>
%<*SIunits>
%\fi
%\subsection{\SIunits}
% \SIunits emulation starts in much the same way.
%    \begin{macrocode}
\ProvidesFile{si-SIunits.cfg}
  [2008/12/22 v1.1j siunitx: Emulation of
     SIunits]
\si@emulating{SIunits}{2007/12/02 v1.36}
\sisetup{
  unitsep=thick,
  valuesep=thick,
  prefixproduct=\si@valuesep,
  trapambigfrac=false,
  stickyper}
\requiresiconfigs{prefix,named,accepted,physical}
%    \end{macrocode}
%\begin{macro}{\reciprocal}
%\begin{macro}{\rp}
%\begin{macro}{\per}
%\begin{macro}{\usk}
%\begin{macro}{\power}
%\begin{macro}{\rpsquare}
%\begin{macro}{\rpcubic}
%\begin{macro}{\fourth}
%\begin{macro}{\rpfourth}
% A few very simple translations, using the internal version of
% \cs{per} to allow changes of output style.
%    \begin{macrocode}
\newcommand*{\reciprocal}{\sisetup{per=reciprocal}\si@per}
\let\rp\reciprocal
\renewcommand*{\per}{\sisetup{per=slash}\si@per}
\newcommand*{\usk}{}
\newcommand*{\power}[1]{#1\tothe}
\newcommand*{\rpsquare}{\sisetup{per=reciprocal}\si@per\Square}
\newcommand*{\rpcubic}{\sisetup{per=reciprocal}\si@per\cubic}
\newpower{\fourth}{4}
\newcommand*{\rpfourth}{\sisetup{per=reciprocal}\si@per\fourth}
%    \end{macrocode}
%\end{macro}
%\end{macro}
%\end{macro}
%\end{macro}
%\end{macro}
%\end{macro}
%\end{macro}
%\end{macro}
%\end{macro}
%\begin{macro}{\rpsquared}
%\begin{macro}{\rpcubed}
% Here, some low-level switch changing is needed.
%    \begin{macrocode}
\newcommand*{\rpsquared}{%
  \sisetup{per=reciprocal}\si@unt@pertrue\si@unt@perseentrue%
    \squared}
\newcommand*{\rpcubed}{%
  \sisetup{per=reciprocal}\si@unt@pertrue\cubed}
%    \end{macrocode}
%\end{macro}
%\end{macro}
%\begin{macro}{\SIsetup}
%\begin{macro}{\si@siu@setup}
% The various package spacing options are processed.  They also have
% to be correctly handled by the \cs{SIsetup} macro.
%    \begin{macrocode}
\newcommand*{\SIsetup}[1]{%
  \@for\si@tempa:=#1\do{%
    \@ifundefined{ifsi@old@#1}
      {\si@log@warn{Unknown SIunits option `#1'}}
      {\csname si@old@#1true\endcsname}}
  \si@siu@setup}
\newcommand*{\si@siu@setup}{%
  \ifsi@old@cdot
    \sisetup{unitsep=cdot}%
  \fi
  \ifsi@old@thickspace
    \sisetup{unitsep=thick}%
  \fi
  \ifsi@old@mediumspace
    \sisetup{unitsep=medium}%
  \fi
  \ifsi@old@thinspace
    \sisetup{unitsep=thin}%
  \fi
  \ifsi@old@thickqspace
    \sisetup{valuesep=thick}%
  \fi
  \ifsi@old@mediumqspace
    \sisetup{valuesep=medium}%
  \fi
  \ifsi@old@thinqspace
    \sisetup{valuesep=thin}%
  \fi}
\si@siu@setup
%    \end{macrocode}
%\end{macro}
%\end{macro}
%\begin{macro}{\square}
%\begin{macro}{\squaren}
% \SIunits does slightly different things about the clash with
% \cs{square}, and either redefines this macro or provides
% \cs{squaren}.
%    \begin{macrocode}
\ifsi@old@squaren
  \newpower{\squaren}{2}
\fi
\AtBeginDocument{%
  \@ifundefined{square}
    {\newpower{\square}{2}}
    {\ifsi@old@amssymb
       \renewpower{\square}{2}
     \else
       \ifsi@old@squaren\else
         \si@log@warn{\string\square\space already
           defined\MessageBreak SIunits mode may cause
           errors}%
       \fi
     \fi}}
%    \end{macrocode}
%\end{macro}
%\end{macro}
%\begin{macro}{\gray}
% The potential clash with \pkg{PStricks} is also handled
% differently; here, \cs{Gray} will already be defined by the
%  \currpkg kernel.
%    \begin{macrocode}
\AtBeginDocument{
  \@ifundefined{gray}
    {\newunit{\gray}{Gy}}
    {\ifsi@old@pstricks
       \renewunit{\gray}{Gy}
     \else
       \ifsi@old@Gray\else
         \si@log@warn{\string\gray\space already
           defined\MessageBreak SIunits mode may cause
           errors}%
       \fi
     \fi}}
%    \end{macrocode}
%\end{macro}
%\begin{macro}{\unit}
%\begin{macro}{\unita}
% The \cs{unit} macro is defined.
%    \begin{macrocode}
\ifsi@old@italian
  \let\unita\SI
\else
  \let\unit\SI
\fi
%    \end{macrocode}
%\end{macro}
%\end{macro}
% The miscellaneous options are moped up.
%    \begin{macrocode}
\ifsi@old@textstyle
  \sisetup{mode=text}
\fi
\ifsi@old@binary
  \sisetup{alsoload=binary}
\fi
\ifsi@old@noams
  \AtBeginDocument{%
    \renewcommand*{\si@textmu}{\ensuremath\si@mathsmu}}
\fi
%    \end{macrocode}
%\begin{macro}{\arcminute}
%\begin{macro}{\arcsecond}
%\begin{macro}{\rperminute}
%\begin{macro}{\ton}
%\begin{macro}{\degreecelsius}
% The unit macros defined by \SIunits that are not defined by
% \currpkg (by default).
%    \begin{macrocode}
\newunit[valuesep=none]{\arcminute}{\si@sym@minute}
\newunit[valuesep=none]{\arcsecond}{\si@sym@second}
\newunit{\rperminute}{r/min}
\newunit{\ton}{t}
\newunit{\degreecelsius}{\celsius}
%    \end{macrocode}
%\end{macro}
%\end{macro}
%\end{macro}
%\end{macro}
%\end{macro}
%\begin{macro}{\addunit}
% This is an alias for \cs{newunit}.
%    \begin{macrocode}
\let\addunit\newunit
%    \end{macrocode}
%\end{macro}
%\begin{macro}{\addprefix}
% A little more work for \cs{addprefix}.
%    \begin{macrocode}
\newcommand*{\addprefix}[2]{\newprefix{#1}{#2}}
%    \end{macrocode}
%\end{macro}
%\begin{macro}{\si@siu@newunit}
%\changes{v1.1a}{2008/09/30}{Corrected very minor bug in definition}
%\begin{macro}{\si@siu@power}
%\begin{macro}{\si@siu@newunithook}
%\doarg{power}
%\darg{numerator}
%\darg{denominator}
% To save some code, making new units which need a \texttt{\ldots np}
% variant is handled by a dedicated macro.
%    \begin{macrocode}
\newcommand*{\si@siu@newunit}[3][]{%
%    \end{macrocode}
% A test is needed to sort out \cs{square}.
%    \begin{macrocode}
  \renewcommand*{\si@tempa}{#1}%
  \renewcommand*{\si@tempb}{square}%
  \def\si@siu@power{}%
  \ifx\@empty\si@tempa\@empty\else
    \ifx\si@tempa\si@tempb
      \renewcommand*{\si@siu@power}{\ssquare}%
    \else
      \edef\si@siu@power{%
        \expandafter\noexpand\csname #1\endcsname}%
    \fi
  \fi
%    \end{macrocode}
% The necessary information is now stored in temporary macros.
%    \begin{macrocode}
  \edef\si@tempa{%
    \expandafter\noexpand\csname #2per#1#3\endcsname}%
  \edef\si@tempb{%
    \expandafter\noexpand\csname #2\endcsname\noexpand\per
    \expandafter\noexpand\si@siu@power
    \expandafter\noexpand\csname #3\endcsname}%
  \expandafter\expandafter\expandafter\newunit\expandafter%
    \expandafter\expandafter{\expandafter\si@tempa\expandafter}%
      \expandafter{\si@tempb}
  \edef\si@tempa{%
    \expandafter\noexpand\csname #2per#1#3np\endcsname}%
  \edef\si@tempb{%
    \expandafter\noexpand\csname #2\endcsname\noexpand
    \reciprocal\expandafter\noexpand\si@siu@power
    \expandafter\noexpand\csname #3\endcsname}%
  \expandafter\expandafter\expandafter\newunit\expandafter
    \expandafter\expandafter{\expandafter\si@tempa\expandafter}%
      \expandafter{\si@tempb}%
  \si@siu@newunithook[#1]{#2}{#3}}
\providecommand*{\si@siu@newunithook}[3][]{}
%    \end{macrocode}
%\end{macro}
%\end{macro}
%\end{macro}
% The basic units are now defined; these only have a single csname in
% each of the numerator and denominator.
%    \begin{macrocode}
\si@siu@newunit{gray}{second}
\si@siu@newunit[square]{metre}{second}
\si@siu@newunit{joule}{mole}
\si@siu@newunit[cubic]{mole}{metre}
\si@siu@newunit[square]{radian}{second}
\si@siu@newunit{radian}{second}
\si@siu@newunit[cubic]{squaremetre}{metre}
\si@siu@newunit[cubic]{katal}{metre}
\si@siu@newunit{coulomb}{mol}
\si@siu@newunit[square]{ampere}{metre}
\si@siu@newunit[cubic]{kilogram}{metre}
\si@siu@newunit[cubic]{coulomb}{metre}
\si@siu@newunit{volt}{metre}
\si@siu@newunit[square]{coulomb}{squaremetre}
\si@siu@newunit{farad}{metre}
\si@siu@newunit[square]{watt}{metre}
\si@siu@newunit[square]{joule}{metre}
\si@siu@newunit[cubic]{newton}{metre}
\si@siu@newunit{newton}{kilogram}
\si@siu@newunit{joule}{kelvin}
\si@siu@newunit{joule}{kilogram}
\si@siu@newunit{coulomb}{kilogram}
\si@siu@newunit{squaremetre}{second}
\si@siu@newunit[square]{squaremetre}{second}
\si@siu@newunit[square]{candela}{metre}
\si@siu@newunit{ampere}{metre}
\si@siu@newunit{joule}{tesla}
\si@siu@newunit{henry}{metre}
\si@siu@newunit{kilogram}{second}
\si@siu@newunit[square]{kilogram}{metre}
\si@siu@newunit{kilogram}{metre}
\si@siu@newunit[square]{newton}{metre}
\si@siu@newunit{watt}{kilogram}
\si@siu@newunit[cubic]{watt}{metre}
\si@siu@newunit{squaremetre}{kilogram}
\si@siu@newunit{cubicmetre}{kilogram}
\si@siu@newunit{newton}{metre}
\si@siu@newunit[cubic]{squaremetre}{second}
\si@siu@newunit{metre}{second}
\si@siu@newunit[cubic]{joule}{metre}
\si@siu@newunit{cubicmetre}{second}
%    \end{macrocode}
%\begin{macro}{\si@siu@newunitx}
%\begin{macro}{\si@siu@newunitxhook}
% For the more complex units, a slightly different approach is used;
% four arguments are required, and have to cover everything.
%    \begin{macrocode}
\newcommand*{\si@siu@newunitx}[4]{%
  \expandafter\newunit\expandafter{\csname #1per#2\endcsname}
    {#3\per#4}%
  \expandafter\newunit\expandafter{\csname #1per#2np\endcsname}
    {#3\reciprocal#4}
  \si@siu@newunitxhook{#1}{#2}{#3}{#4}}
\providecommand*{\si@siu@newunitxhook}[4]{}
%    \end{macrocode}
%\end{macro}
%\end{macro}
% The units are defined.
%    \begin{macrocode}
\si@siu@newunitx{kilogramsquaremetre}{second}
  {\kilogram\squaremetre}{\second}
\si@siu@newunitx{squaremetre}{newtonsecond}{\squaremetre}
  {\newton\second}
\si@siu@newunitx{kilogrammetre}{second}{\kilogram\metre}
  {\second}
\si@siu@newunitx{kilogram}{squaremetresecond}{\kilogram}
  {\squaremetre\second}
\si@siu@newunitx{joule}{molekelvin}{\joule}{\mole\kelvin}
\si@siu@newunitx{kilogram}{kilomole}{\kilogram}{\kilo\mole}
\si@siu@newunitx{kilogrammetre}{squaresecond}{\kilogram\metre}
  {\second\squared}
\si@siu@newunitx{watt}{squaremetresteradian}{\watt}
  {\squaremetre\steradian}
\si@siu@newunitx{joule}{kilogramkelvin}{\joule}
  {\kilogram\kelvin}
\si@siu@newunitx{watt}{metrekelvin}{\watt}{\metre\kelvin}
\si@siu@newunitx{kilogram}{cubicmetrecoulomb}{\kilogram}
  {\cubic\metre\coulomb}
\si@siu@newunitx{kilogram}{secondcubicmetre}{\kilogram}
  {\second\cubicmetre}
%    \end{macrocode}
%\begin{macro}{\si@siu@unity}
% A bit of cleverness to get the ``1'' correct; to avoid any clash, the unit
% is given an internal name.
%    \begin{macrocode}
\newunit{\si@siu@unity}{1}
\si@siu@newunitx{}{squaremetresecond}{\si@siu@unity}
  {\squaremetre\second}
%    \end{macrocode}
%\end{macro}
% A few compound units that are best defined directly.
%    \begin{macrocode}
\newunit{\pascalsecond}{\pascal\second}
\newunit{\amperemetresecond}{\ampere\metre\second}
\newunit{\ohmmetre}{\ohm\metre}
\newunit{\newtonmetre}{\newton\metre}
\let\newtonmetrenp\newtonmetre
\newunit{\kilogramsquaremetre}{\kilogram\squaremetre}
\let\kilogramsquaremetrenp\kilogramsquaremetre
%    \end{macrocode}
%\begin{macro}{\si@siu@newprefix}
%\darg{prefix}
% To generate the prefixes correctly, a small saving in repetition.
%    \begin{macrocode}
\newcommand*{\si@siu@newprefix}[1]{%
  \edef\si@tempa{\expandafter\noexpand\csname #1d\endcsname}%
  \edef\si@tempb{\expandafter\noexpand\csname #1\endcsname}%
  \expandafter\expandafter\expandafter\newcommand\expandafter
    \expandafter\expandafter*\expandafter\expandafter
    \expandafter{\expandafter\si@tempa\expandafter}\expandafter
      {\expandafter\si@prefixsymbolicfalse\si@tempb}}
%    \end{macrocode}
%\end{macro}
% This is now implemented.
%    \begin{macrocode}
\si@siu@newprefix{yocto}
\si@siu@newprefix{zepto}
\si@siu@newprefix{atto}
\si@siu@newprefix{femto}
\si@siu@newprefix{pico}
\si@siu@newprefix{nano}
\si@siu@newprefix{micro}
\si@siu@newprefix{milli}
\si@siu@newprefix{centi}
\si@siu@newprefix{deca}
\si@siu@newprefix{deka}
\si@siu@newprefix{hecto}
\si@siu@newprefix{kilo}
\si@siu@newprefix{mega}
\si@siu@newprefix{giga}
\si@siu@newprefix{tera}
\si@siu@newprefix{peta}
\si@siu@newprefix{exa}
\si@siu@newprefix{zetta}
\si@siu@newprefix{yotta}
\ifsi@old@binary
  \si@siu@newprefix{kibi}
  \si@siu@newprefix{mebi}
  \si@siu@newprefix{gibi}
  \si@siu@newprefix{tebi}
  \si@siu@newprefix{pebi}
  \si@siu@newprefix{exbi}
\fi
%    \end{macrocode}
%\begin{macro}{\derradian}
%\begin{macro}{\dersteradian}
%\begin{macro}{\derhertz}
%\begin{macro}{\dernewton}
%\begin{macro}{\derpascal}
%\begin{macro}{\derjoule}
%\begin{macro}{\derwatt}
%\begin{macro}{\dercoulomb}
%\begin{macro}{\dervolt}
%\begin{macro}{\derfarad}
%\begin{macro}{\derohm}
% The derived units may need to be defined.
%    \begin{macrocode}
\ifsi@old@derived
  \newunit{\derradian}{\metre\reciprocal\metre}
  \newunit{\dersteradian}{\squaremetre\rpsquare\metre}
  \newunit{\derhertz}{\reciprocal\second}
  \newunit{\dernewton}{\metre\kilogram\second\rpsquared}
  \newunit{\derpascal}{\newton\rpsquare\metre}
  \newunit{\derjoule}{\newton\metre}
  \newunit{\derwatt}{\joule\reciprocal\second}
  \newunit{\dercoulomb}{\ampere\second}
  \newunit{\dervolt}{\watt\reciprocal\ampere}
  \newunit{\derfarad}{\coulomb\reciprocal\volt}
  \newunit{\derohm}{\volt\reciprocal\ampere}
%    \end{macrocode}
%\end{macro}
%\end{macro}
%\end{macro}
%\end{macro}
%\end{macro}
%\end{macro}
%\end{macro}
%\end{macro}
%\end{macro}
%\end{macro}
%\end{macro}
%\begin{macro}{\dersiemens}
%\begin{macro}{\derweber}
%\begin{macro}{\dertesla}
%\begin{macro}{\derhenry}
%\begin{macro}{\dercelsius}
%\begin{macro}{\derlumen}
%\begin{macro}{\derlux}
%\begin{macro}{\derbecquerel}
%\begin{macro}{\dergray}
%\begin{macro}{\dersievert}
%\begin{macro}{\derkatal}
% In two blocks!
%    \begin{macrocode}
  \newunit{\dersiemens}{\ampere\reciprocal\volt}
  \newunit{\derweber}
    {\squaremetre\kilogram\second\rpsquared\reciprocal\ampere}
  \newunit{\dertesla}{\weber\rpsquare\metre}
  \newunit{\derhenry}{\weber\reciprocal\ampere}
  \newunit{\dercelsius}{\kelvin}
  \newunit{\derlumen}{\candela\steradian}
  \newunit{\derlux}{\lumen\rpsquare\metre}
  \newunit{\derbecquerel}{\derhertz}
  \newunit{\dergray}{\joule\reciprocal\kilogram}
  \newunit{\dersievert}{\dergray}
  \newunit{\derkatal}{\rp\second\usk\mole}
\fi
%    \end{macrocode}
%\end{macro}
%\end{macro}
%\end{macro}
%\end{macro}
%\end{macro}
%\end{macro}
%\end{macro}
%\end{macro}
%\end{macro}
%\end{macro}
%\end{macro}
%\begin{macro}{\radianbase}
%\begin{macro}{\steradianbase}
%\begin{macro}{\hertzbase}
%\begin{macro}{\newtonbase}
%\begin{macro}{\pascalbase}
%\begin{macro}{\joulebase}
%\begin{macro}{\wattbase}
%\begin{macro}{\coulombbase}
%\begin{macro}{\voltbase}
%\begin{macro}{\faradbase}
%\begin{macro}{\ohmbase}
% Also the ``derived-in-base''.
%    \begin{macrocode}
\ifsi@old@derivedinbase
  \newunit{\radianbase}{\metre\reciprocal\metre}
  \newunit{\steradianbase}{\squaremetre\rpsquare\metre}
  \newunit{\hertzbase}{\reciprocal\second}
  \newunit{\newtonbase}{\metre\kilogram\second\rpsquared}
  \newunit{\pascalbase}{\reciprocal\metre\kilogram\second%
    \rpsquared}
  \newunit{\joulebase}{\squaremetre\kilogram\second\rpsquared}
  \newunit{\wattbase}{\squaremetre\kilogram\rpcubic\second}
  \newunit{\coulombbase}{\ampere\second}
  \newunit{\voltbase}
    {\squaremetre\kilogram\rpcubic\second\reciprocal\ampere}
  \newunit{\faradbase}
    {\rpsquare\metre\reciprocal\kilogram\fourth\second\ampere%
      \squared}
  \newunit{\ohmbase}
    {\squaremetre\kilogram\rpcubic\second\rpsquare\ampere}
%    \end{macrocode}
%\end{macro}
%\end{macro}
%\end{macro}
%\end{macro}
%\end{macro}
%\end{macro}
%\end{macro}
%\end{macro}
%\end{macro}
%\end{macro}
%\end{macro}
%\begin{macro}{\siemensbase}
%\begin{macro}{\weberbase}
%\begin{macro}{\teslabase}
%\begin{macro}{\henrybase}
%\begin{macro}{\celsiusbase}
%\begin{macro}{\lumenbase}
%\begin{macro}{\luxbase}
%\begin{macro}{\becquerelbase}
%\begin{macro}{\graybase}
%\begin{macro}{\sievertbase}
%\begin{macro}{\katalbase}
% Also in two blocks.
%    \begin{macrocode}
  \newunit{\siemensbase}
    {\rpsquare\metre\reciprocal\kilogram\cubic\second\ampere%
       \squared}
  \newunit{\weberbase}
    {\squaremetre\kilogram\second\rpsquared\reciprocal\ampere}
  \newunit{\teslabase}{\kilogram\second\rpsquared\reciprocal%
     \ampere}
  \newunit{\henrybase}
    {\squaremetre\kilogram\second\rpsquared\rpsquare\ampere}
  \newunit{\celsiusbase}{\kelvin}
  \newunit{\lumenbase}{\candela\squaremetre\rpsquare\metre}
  \newunit{\luxbase}{\candela\squaremetre\rpfourth\metre}
  \newunit{\becquerelbase}{\hertzbase}
  \newunit{\graybase}{\squaremetre\second\rpsquared}
  \newunit{\sievertbase}{\graybase}
  \newunit{\katalbase}{\rp\second\mole}
\fi
%    \end{macrocode}
%\end{macro}
%\end{macro}
%\end{macro}
%\end{macro}
%\end{macro}
%\end{macro}
%\end{macro}
%\end{macro}
%\end{macro}
%\end{macro}
%\end{macro}
% Any configuration file is used if found.
%    \begin{macrocode}
\InputIfFileExists{SIunits.cfg}
  {\si@log@inf{SIunits config file loaded}}
  {\si@log@inf{SIunits config file not found}}
%    \end{macrocode}
%\iffalse
%</SIunits>
%<*hepunits>
%\fi
%\subsection{\pkg{hepunits}}
% The \pkg{hepunits} package provides some rather odd unit names,
% which are not really to be encouraged.
%    \begin{macrocode}
\ProvidesFile{si-hepunits.cfg}
  [2008/12/22 v1.1j siunitx: Emulation of
     hepunits]
\si@emulating{hepunits}{2007/09/27}
\requiresiconfigs{SIunits,accepted,prefix,hep}
%    \end{macrocode}
%\begin{macro}{\invbarn}
%\begin{macro}{\invnanobarn}
%\begin{macro}{\invpicobarn}
%\begin{macro}{\invfemtobarn}
%\begin{macro}{\invattobarn}
%\begin{macro}{\invzeptobarn}
%\begin{macro}{\invyoctobarn}
% Inverses barn units.
%    \begin{macrocode}
\ifsi@old@noprefixcmds\else
  \newunit{\invbarn}{\per\barn}
  \newunit{\invnanobarn}{\per\nano\barn}
  \newunit{\invpicobarn}{\per\pico\barn}
  \newunit{\invfemtobarn}{\per\femto\barn}
  \newunit{\invattobarn}{\per\atto\barn}
  \newunit{\invzeptobarn}{\per\zepto\barn}
  \newunit{\invyoctobarn}{\per\yocto\barn}
%    \end{macrocode}
%\end{macro}
%\end{macro}
%\end{macro}
%\end{macro}
%\end{macro}
%\end{macro}
%\begin{macro}{\invnb}
%\begin{macro}{\invpb}
%\begin{macro}{\invfb}
%\begin{macro}{\invab}
%\begin{macro}{\invzb}
%\begin{macro}{\invyb}
% Also available abbreviated.
%    \begin{macrocode}
  \newunit{\invnb}{\per\nano\barn}
  \newunit{\invpb}{\per\pico\barn}
  \newunit{\invfb}{\per\femto\barn}
  \newunit{\invab}{\per\atto\barn}
  \newunit{\invzb}{\per\zepto\barn}
  \newunit{\invyb}{\per\yocto\barn}
\fi
%    \end{macrocode}
%\end{macro}
%\end{macro}
%\end{macro}
%\end{macro}
%\end{macro}
%\end{macro}
%\end{macro}
%\begin{macro}{\invcmsqpersecond}
%\begin{macro}{\invcmsqpersec}
%\begin{macro}{\lumiunits}
% Luminosity.
%    \begin{macrocode}
\newunit{\invcmsqpersecond}{\per\Square\centi\metre\per\second}
\newunit{\invcmsqpersec}{\per\Square\centi\metre\per\second}
\newunit{\lumiunits}{\per\Square\centi\metre\per\second}
%    \end{macrocode}
%\end{macro}
%\end{macro}
%\end{macro}
%\begin{macro}{\inveV}
%\begin{macro}{\minveV}
%\begin{macro}{\minveV}
%\begin{macro}{\kinveV}
%\begin{macro}{\MinveV}
%\begin{macro}{\GinveV}
%\begin{macro}{\TinveV}
% The inverse of an electron-volt, plus prefixes.
%    \begin{macrocode}
\newunit{\inveV}{\per\electronvolt}
\newunit{\minveV}{\per\milli\electronvolt}
\newunit{\kinveV}{\per\kilo\electronvolt}
\newunit{\MinveV}{\per\mega\electronvolt}
\newunit{\GinveV}{\per\giga\electronvolt}
\newunit{\TinveV}{\per\tera\electronvolt}
%    \end{macrocode}
%\end{macro}
%\end{macro}
%\end{macro}
%\end{macro}
%\end{macro}
%\end{macro}
%\end{macro}
%\begin{macro}{\eVoverc}
%\begin{macro}{\eVovercsq}
% Some combinations of electron-volts and the speed of light.  As
% these are called |over|, they are set with a slash.  The
% \opt{eVcorrb} values have been set for Computer Modern.
%    \begin{macrocode}
\newunit[per=slash,eVcorrb=0.6ex]{\eVoverc}
  {\electronvolt\per\clight}
\newunit[per=slash,eVcorrb=0.6ex]{\eVovercsq}
  {\electronvolt\per\Square\clight}
%    \end{macrocode}
%\end{macro}
%\end{macro}
%\begin{macro}{\meVoverc}
%\begin{macro}{\keVoverc}
%\begin{macro}{\MeVoverc}
%\begin{macro}{\GeVoverc}
%\begin{macro}{\TeVoverc}
% Prefixed combinations, first of the speed of light.
%    \begin{macrocode}
\newunit[per=slash,eVcorrb=0.6ex]{\meVoverc}
  {\milli\electronvolt\per\clight}
\newunit[per=slash,eVcorrb=0.6ex]{\keVoverc}
  {\kilo\electronvolt\per\clight}
\newunit[per=slash,eVcorrb=0.6ex]{\MeVoverc}
  {\mega\electronvolt\per\clight}
\newunit[per=slash,eVcorrb=0.6ex]{\GeVoverc}
  {\giga\electronvolt\per\clight}
\newunit[per=slash,eVcorrb=0.6ex]{\TeVoverc}
  {\tera\electronvolt\per\clight}
%    \end{macrocode}
%\end{macro}
%\end{macro}
%\end{macro}
%\end{macro}
%\end{macro}
%\begin{macro}{\meVovercsq}
%\begin{macro}{\keVovercsq}
%\begin{macro}{\MeVovercsq}
%\begin{macro}{\GeVovercsq}
%\begin{macro}{\TeVovercsq}
% Then of the square.
%    \begin{macrocode}
\newunit[per=slash,eVcorrb=0.6ex]{\meVovercsq}
  {\milli\electronvolt\per\Square\clight}
\newunit[per=slash,eVcorrb=0.6ex]{\keVovercsq}
  {\kilo\electronvolt\per\Square\clight}
\newunit[per=slash,eVcorrb=0.6ex]{\MeVovercsq}
  {\mega\electronvolt\per\Square\clight}
\newunit[per=slash,eVcorrb=0.6ex]{\GeVovercsq}
  {\giga\electronvolt\per\Square\clight}
\newunit[per=slash,eVcorrb=0.6ex]{\TeVovercsq}
  {\tera\electronvolt\per\Square\clight}
%    \end{macrocode}
%\end{macro}
%\end{macro}
%\end{macro}
%\end{macro}
%\end{macro}
%\iffalse
%</hepunits>
%<*fancynum>
%\fi
%\subsection{\pkg{fancynum}}
% \pkg{fancynum} only does things with numbers, so there is only
% a little emulation and a few macros needed.
%    \begin{macrocode}
\ProvidesFile{si-fancynum.cfg}
  [2008/12/22 v1.1j siunitx: Emulation of
     fancynum]
\si@emulating{fancynum}{2000/08/08 0.92}
\sisetup{decimalsymbol=cdot,digitsep=comma}
%    \end{macrocode}
%\begin{macro}{\fnum}
% The \cs{fnum} macro is rather restricted, but this is not
% reproduced.  Instead, it is \cs{let} as an alias to the \cs{num}
% macro.
%    \begin{macrocode}
\let\fnum\num
%    \end{macrocode}
%\end{macro}
%\begin{macro}{\setfnumdsym}
%\begin{macro}{\setfnumgsym}
%\begin{macro}{\setfnummsym}
% The control macros are defined.
%    \begin{macrocode}
\newcommand*{\setfnumdsym}[1]{\sisetup{decimalsymbol={#1}}}
\newcommand*{\setfnumgsym}[1]{\sisetup{digitsep={#1}}}
\newcommand*{\setfnummsym}[1]{\sisetup{expproduct={#1}}}
%    \end{macrocode}
%\end{macro}
%\end{macro}
%\end{macro}
% The various package options are now processed if necessary.
%    \begin{macrocode}
\ifsi@old@english
  \sisetup{decimalsymbol=cdot,digitsep=comma}
\fi
\ifsi@old@french
  \sisetup{decimalsymbol=comma,digitsep=fullstop}
\fi
\ifsi@old@tight
  \sisetup{expproduct=tighttimes}
\fi
\ifsi@old@loose
  \sisetup{expproduct=times}
\fi
\ifsi@old@thinspaces
  \sisetup{digitsep=thin}
\fi
\ifsi@old@commas
  \sisetup{digitsep=comma}
\fi
\ifsi@old@plain
  \sisetup{digitsep=none}
\fi
%    \end{macrocode}
%
%\iffalse
%</fancynum>
%<*fancyunits>
%\fi
%\subsection{\pkg{fancyunits}}
% The \pkg{fancyunits} package is not available on \textsc{ctan}, but
% is available from its authors homepage \cite{fancyunits}.  It is
% similar to \SIunits, and so most of the code is shared here.
% However, a few bits of set up occur first, and an emulation-clash
% test is needed.
%    \begin{macrocode}
\ProvidesFile{si-fancyunits.cfg}
  [2008/12/22 v1.1j siunitx: Emulation of
     fancyunits]
\si@emulating{fancyunits}{2007/02/01 v1.0.1}
\si@ifloaded{SIunits}
  {\si@log@err{SIunits emulation loaded\MessageBreak before
    fancyunits emulation}{You need to load the fancyunits
    emualtion\MessageBreak code before that for
    SIunits.\MessageBreak Try emulate=fancyunits as the first
    option when\MessageBreak loading siunitx}}{}
%    \end{macrocode}
%\begin{macro}{\si@siu@newunithook}
%\begin{macro}{\si@siu@newunitxhook}
% To create the extra macros provided by \pkg{fancyunits}, the
% \SIunits emulation code is changed to add the ``\texttt{uf}''
% variants.
%    \begin{macrocode}
\newcommand*{\si@siu@newunithook}[3][]{%
  \edef\si@tempa{%
    \expandafter\noexpand\csname #2per#1#3uf\endcsname}%
  \renewcommand*{\si@tempb}{stickyper,per=fraction,
    fraction=nice}%
  \edef\si@tempc{%
    \noexpand\sisetup{\si@tempb}%
    \expandafter\noexpand\csname #2\endcsname\noexpand\si@per%
    \expandafter\noexpand\si@siu@power%
    \expandafter\noexpand\csname #3\endcsname}%
  \expandafter\expandafter\expandafter\newunit\expandafter
    \expandafter\expandafter{\expandafter\si@tempa\expandafter}%
      \expandafter{\si@tempc}%
  \edef\si@tempa{%
    \expandafter\noexpand\csname #2per#1#3Uf\endcsname}%
  \renewcommand*{\si@tempb}{stickyper,per=fraction,
    fraction=frac}%
  \edef\si@tempc{%
    \noexpand\sisetup{\si@tempb}%
    \noexpand\def\noexpand\si@frc@hook{\noexpand\textstyle}%
    \expandafter\noexpand\csname #2\endcsname\noexpand\si@per%
    \expandafter\noexpand\si@siu@power%
    \expandafter\noexpand\csname #3\endcsname}%
  \expandafter\expandafter\expandafter\newunit\expandafter
    \expandafter\expandafter{\expandafter\si@tempa\expandafter}%
      \expandafter{\si@tempc}%
  \edef\si@tempa{%
    \expandafter\noexpand\csname #2per#1#3UF\endcsname}%
  \edef\si@tempc{%
    \noexpand\sisetup{\si@tempb}%
    \noexpand\def\noexpand\si@frc@hook{\noexpand\displaystyle}%
    \expandafter\noexpand\csname #2\endcsname\noexpand\si@per%
    \expandafter\noexpand\si@siu@power%
    \expandafter\noexpand\csname #3\endcsname}%
  \expandafter\expandafter\expandafter\newunit\expandafter
    \expandafter\expandafter{\expandafter\si@tempa\expandafter}%
      \expandafter{\si@tempc}}
\newcommand*{\si@siu@newunitxhook}[4]{%
  \expandafter\newunit\expandafter{\csname #1per#2uf\endcsname}
    {\sisetup{stickyper,per=fraction,fraction=nice}%
     #3\si@per#4}%
  \expandafter\newunit\expandafter{\csname #1per#2Uf\endcsname}
    {\sisetup{stickyper,per=fraction,fraction=frac}%
     \renewcommand*{\si@frc@hook}{\textstyle}%
     #3\si@per#4}%
  \expandafter\newunit\expandafter{\csname #1per#2UF\endcsname}
    {\sisetup{stickyper,per=fraction,fraction=frac}%
     \renewcommand*{\si@frc@hook}{\displaystyle}%
     #3\si@per#4}}
%    \end{macrocode}
%\end{macro}
%\end{macro}
% With that done, the emulation modules can be loaded.
%    \begin{macrocode}
\requiresiconfigs{SIunits,addn,astro}
\sisetup{obeyall}
%    \end{macrocode}
% There is one \pkg{fancyunits}-specific option to handle.  The other
% options all get sent through to the \SIunits system.\footnote{As
% \SIunits is rather more widely known than \pkg{fancyunits}, any
% other options which could be for either are assumed to be for
% \SIunits.}
%    \begin{macrocode}
\ifsi@old@spaceqspace
  \sisetup{valuesep=space}
\fi
%    \end{macrocode}
%\begin{macro}{\paminute}
%\begin{macro}{\parsecond}
%\begin{macro}{\AstroE}
%\begin{macro}{\oersted}
%\begin{macro}{\ton}
% \pkg{fancyunits} provides some extra units, plus tonne spelled
% incorrectly (again).
%    \begin{macrocode}
\newunit{\paminute}{'}
\newunit{\parsecond}{''}
\newunit{\AstroE}{AE}
\newunit{\oersted}{OE}
\provideunit{\ton}{t}
%    \end{macrocode}
%\end{macro}
%\end{macro}
%\end{macro}
%\end{macro}
%\end{macro}
%\begin{macro}{\decaD}
% An additional prefix.
%    \begin{macrocode}
\let\decaD\decad
%    \end{macrocode}
%\end{macro}
%\begin{macro}{\ufrac}
%\begin{macro}{\Ufrac}
%\begin{macro}{\UFrac}
% The fractional unit macros need to be reproduced.
%    \begin{macrocode}
\newcommand*{\ufrac}[2]{%
  \si[stickyper,per=fraction,fraction=nice]{#1\si@per#2}}
\newcommand*{\Ufrac}[2]{%
  \ensuremath{\textstyle{%
    \si[stickyper,per=fraction,fraction=frac]{#1\si@per#2}}}}
\newcommand*{\UFrac}[2]{%
  \ensuremath{\displaystyle{%
    \si[stickyper,per=fraction,fraction=frac]{#1\si@per#2}}}}
%    \end{macrocode}
%\end{macro}
%\end{macro}
%\end{macro}
%\begin{macro}{\pow}
% A slightly-shorted named \cs{power}.
%    \begin{macrocode}
\let\pow\power
%    \end{macrocode}
%\end{macro}
%\begin{macro}{\Squaremetre}
% An alias for \cs{squaremetre}.
%    \begin{macrocode}
\let\Squaremetre\squaremetre
%    \end{macrocode}
%\end{macro}
% As with \SIunits, there is now a list of compound units to add.
% Only a few are not covered by the \SIunits emulation.
%    \begin{macrocode}
\si@siu@newunit{Gray}{second}
\si@siu@newunit[square]{Squaremetre}{metre}
\si@siu@newunitx{Squaremetre}{newtonsecond}{\Square\metre}
  {\newton\second}
\si@siu@newunit{Squaremetre}{second}
\si@siu@newunit[square]{Squaremetre}{squaresecond}
\si@siu@newunit{Squaremetre}{kilogram}
\si@siu@newunit[cubic]{Squaremetre}{second}
%    \end{macrocode}
%\iffalse
%</fancyunits>
%\fi
%
%\Finale
%\iffalse
%<*refs>
@misc{BIPM:EN,
  note = {\url{http://www.bipm.org/en/si/}},
}

@misc{BIPM:FR,
  note = {\url{http://www.bipm.org/fr/si/}},
}

@misc{BIPM:PDF,
  note = {\url{http://www.bipm.org/en/si/si_brochure/}},
}

@misc{Eijkhout,
  author =       {Victor Eijkhout},
  note =         {\url{http://www.eijkhout.net/tbt/}},
  title =        {{\TeX} by {T}opic},
  year =         {2007},
}

@misc{fancyunits,
  author = {Heiko Bauke},
  note = {\url{http://www.mpi-hd.mpg.de/personalhomes/bauke/LaTeX/Tips_und_Tricks/fancyunits/index.php}},
  title = {\pkg{fancyunits}},
  year = 2007,
}

@misc{IUPAC,
  note = {\url{http://old.iupac.org/publications/books/gbook/green_book_2ed.pdf}},
  title = {The \textsc{iupac} {G}reen {B}ook},
  year = 1993,
}

@misc{NIST,
  note = {\url{http://physics.nist.gov/cuu/Units/index.html}},
}
%</refs>
%<*jawltxdoc>
\NeedsTeXFormat{LaTeX2e}
\ProvidesPackage{jawltxdoc}
\usepackage[T1]{fontenc}
\usepackage{lmodern}
\usepackage[final]{listings,graphicx,microtype}
\usepackage[scaled=0.95]{helvet}
\usepackage[version=3]{mhchem}
\usepackage[osf]{mathpazo}
\usepackage{booktabs,array,url,courier,xspace,varioref}
\usepackage{upgreek,ifpdf,float,caption,longtable,babel}
\begingroup
  \@ifundefined{eTeXversion}
    {\aftergroup\@gobble}
    {\aftergroup\@firstofone}
\endgroup
{\usepackage{etoolbox}}
\floatstyle{plaintop}
\restylefloat{table}
\labelformat{figure}{\figurename~#1}
\labelformat{table}{\tablename~#1}
\IfFileExists{\jobname.sty}
  {\usepackage{\jobname}}{}
\usepackage[numbered]{hypdoc}
\setcounter{IndexColumns}{2}
\newlength\LaTeXwidth
\newlength\LaTeXoutdent
\newlength\LaTeXgap
\setlength\LaTeXgap{1em}
\setlength\LaTeXoutdent{-0.15\textwidth}
\newbox\lst@samplebox
\edef\LaTeXexamplefile{\jobname.tmp}
\lst@RequireAspects{writefile}
\lstnewenvironment{LaTeXexample}[1][example]{%
  \global\let\lst@intname\@empty
  \ifcsname LaTeXcode#1\endcsname
    \expandafter\let\expandafter\LaTeXcode
      \csname LaTeXcode#1\endcsname
    \expandafter\let\expandafter\LaTeXcodeend
      \csname LaTeXcode#1end\endcsname
  \else
    \PackageError{jawltxdoc}
      {Undefined example type `#1'}
      \@ehd
    \let\LaTeXcode\relax
    \let\LaTeXcodeend\relax
  \fi
  \LaTeXcode}
  {\lst@EndWriteFile
   \LaTeXcodeend}
\newcommand*{\LaTeXcodeexample}{%
  \setbox\lst@samplebox=\hbox\bgroup
  \LaTeXcodefloat}
\let\LaTeXcoderesultonly\LaTeXcodeexample
\newcommand*{\LaTeXcodeexampleend}{%
  \egroup
  \setlength\LaTeXwidth{\wd\lst@samplebox}%
  \begin{list}{}{%
    \setlength\itemindent{0pt}
    \setlength\leftmargin\LaTeXoutdent
    \setlength\rightmargin{0pt}}%
    \item
      \setlength\LaTeXoutdent{-0.15\textwidth}
      \begin{minipage}[c]{%
        \textwidth-\LaTeXwidth-\LaTeXoutdent-\LaTeXgap}
        \LaTeXcodefloatend
      \end{minipage}%
      \hfill
      \begin{minipage}[c]{\LaTeXwidth}%
        \hbox to\linewidth{\box\lst@samplebox\hss}%
      \end{minipage}%
  \end{list}}
\newcommand*{\LaTeXcodefloat}{%
  \setkeys{lst}{tabsize=4,gobble=3,breakindent=0pt,
    basicstyle=\small\ttfamily,basewidth=0.51em,
    keywordstyle=\color{blue}}%
  \lst@BeginAlsoWriteFile{\LaTeXexamplefile}}
\let\LaTeXcodenoexample\LaTeXcodefloat
\let\LaTeXcodenoexampleend\@empty
\newcommand*{\LaTeXcodefloatend}{%
  \MakePercentComment\catcode`\^^M=10\relax
  \small
  {\setkeys{lst}{SelectCharTable=\lst@ReplaceInput{\^\^I}%
    {\lst@ProcessTabulator}}%
    \leavevmode \input{\LaTeXexamplefile}}%
  \MakePercentIgnore}
\newcommand*{\LaTeXcoderesultonlyend}{\egroup\LaTeXcodefloatend}
\lstnewenvironment{BibTeXexample}{%
  \global\let\lst@intname\@empty
  \setbox\lst@samplebox=\hbox\bgroup
  \setkeys{lst}{tabsize=4,gobble=3,breakindent=0pt,
    basicstyle=\small\ttfamily,basewidth=0.51em,
    keywordstyle=\color{black}}
  \lst@BeginAlsoWriteFile{\LaTeXexamplefile}}
 {\lst@EndWriteFile
   \LaTeXcodeexampleend}
\newcommand*{\DescribeOption}{%
  \leavevmode\@bsphack\begingroup\MakePrivateLetters
  \Describe@Option}
\newcommand*{\Describe@Option}[1]{\endgroup
              \marginpar{\raggedleft\PrintDescribeEnv{#1}}%
              \SpecialOptionIndex{#1}\@esphack\ignorespaces}
\newcommand*{\SpecialOptionIndex}[1]{\@bsphack
    \index{#1\actualchar{\protect\ttfamily#1}
           (option)\encapchar usage}%
    \index{options:\levelchar#1\actualchar{\protect\ttfamily#1}%
      \encapchar usage}\@esphack}
\newcommand*{\indexopt}[1]{\DescribeOption{#1}\opt{#1}}
\newcommand*{\DescribeOptionInfo}[2]{%
  \DescribeOption{#1}%
  \opt{#1=\meta{#2}}\xspace}
\newcommand*{\ofixarg}[1]{%
  {\ttfamily[}%
  \ifmmode \expandafter \nfss@text \fi
  {%
    \meta@font@select
    \edef\meta@hyphen@restore{%
      \hyphenchar\the\font\the\hyphenchar\font}%
    \hyphenchar\font\m@ne
    \language\l@nohyphenation
    #1\/%
    \meta@hyphen@restore
    }%
    {\ttfamily]}}
\newcommand*{\pkg}[1]{\textsf{#1}}
\newcommand*{\currpkg}{\pkg{\jobname}\xspace}
\newcommand*{\opt}[1]{\texttt{#1}}
\newcommand*{\defaultopt}[1]{\opt{\textbf{#1}}}
\newcommand*{\file}[1]{\texttt{#1}}
\newcommand*{\ext}[1]{\file{.#1}}
\newcommand*{\latin}[1]{\emph{#1}}
\newcommand*{\etc}{%
  \@ifnextchar.
    {\latin{etc}}
    {\latin{etc}.\xspace}}
\newcommand*{\eg}{%
  \@ifnextchar.
    {\latin{e.g}}
    {\latin{e.g}.\xspace}}
\newcommand*{\ie}{%
  \@ifnextchar.
    {\latin{i.e}}
    {\latin{i.e}.\xspace}}
\newcommand*{\etal}{%
  \@ifnextchar.
    {\latin{et~al.}}
    {\latin{et~al}.\xspace}}
\newcommand*{\AMS}{{\protect\usefont{OMS}{cmsy}{m}{n}%
  A\kern-.1667em\lower.5ex\hbox{M}\kern-.125emS}}
\providecommand*{\eTeX}{\ensuremath{\varepsilon}-\TeX}
\DeclareRobustCommand*{\XeTeX}
  {X\kern-.125em\lower.5ex\hbox{\reflectbox{E}}\kern-.1667em\TeX}
\providecommand*{\CTAN}{\textsc{ctan}}
\@ifpackageloaded{etoolbox}
  {\patchcmd{\@addmarginpar}
    {\@latex@warning@no@line {Marginpar on page \thepage\space moved}}
    {\relax}{}{}}
  {}
\newcounter{argument}
\g@addto@macro\endmacro{\setcounter{argument}{0}}
\newcommand*\darg[1]{%
  \stepcounter{argument}%
  {\ttfamily\char`\#\theargument~:~}#1\par\noindent\ignorespaces}
\newcommand*\doarg[1]{%
  \stepcounter{argument}%
  {\ttfamily\makebox[0pt][r]{[}%
   \char`\#\theargument]:~}#1\par\noindent\ignorespaces}
%</jawltxdoc>
%\fi

%
% Documentation:
%    (a) Without write18 enabled:
%          pdflatex siunitx.dtx
%          bibtex8 --wolfgang siunitx
%          makeindex -s gind.ist siunitx.idx
%          makeindex -s gglo.ist -o siunitx.gls siunitx.glo
%          pdflatex siunitx.dtx
%          pdflatex siunitx.dtx
%    (b) With write18 enabled:
%          pdflatex siunitx.dtx
%          pdflatex siunitx.dtx
%          pdflatex siunitx.dtx
%
% Installation:
%     Copy siunitx.sty and the various .cfg files to a location
%     searched by TeX, and if required by your TeX installation,
%     run the appropriate command to build a hash of files
%     (texhash, initexmf --update-fndb, etc.)
%
% Note:
%     The jawltxdoc.sty file is not needed for installation,
%     only for building the documentation; it may be deleted
%     after producing the documentation (if necessary).
%
%<*ignore>
% This is all taken verbatim from Heiko Oberdiek's packages
\begingroup
  \def\x{LaTeX2e}%
\expandafter\endgroup
\ifcase 0\ifx\install y1\fi\expandafter
         \ifx\csname processbatchFile\endcsname\relax\else1\fi
         \ifx\fmtname\x\else 1\fi\relax
\else\csname fi\endcsname
%</ignore>
%<*install>
\input docstrip.tex
\keepsilent
\askforoverwritefalse
\preamble
 ----------------------------------------------------------------
 The siunitx package --- A comprehensive (SI) units package
 Maintained by Joseph Wright
 E-mail: joseph.wright@morningstar2.co.uk
 Released under the LaTeX Project Public License v1.3c or later
 See http://www.latex-project.org/lppl.txt
 ----------------------------------------------------------------

\endpreamble
\Msg{Generating siunitx files:}
\generate{\file{jawltxdoc.sty}{\from{\jobname.dtx}{jawltxdoc}}
}
\usedir{tex/latex/siunitx}
\generate{\file{\jobname.sty}{\from{\jobname.dtx}{package}}
}
\usedir{source/latex/siunitx}
\generate{\file{\jobname.ins}{\from{\jobname.dtx}{install}}
}
\usedir{doc/latex/siunitx}
\generate{\file{si-bug.ltx}{\from{\jobname.dtx}{bug}}
}
\usedir{tex/latex/siunitx/config}
\generate{\file{si-SIunits.cfg}{\from{\jobname.dtx}{SIunits}}
          \file{si-sistyle.cfg}{\from{\jobname.dtx}{sistyle}}
          \file{si-unitsdef.cfg}{\from{\jobname.dtx}{unitsdef}}
          \file{si-units.cfg}{\from{\jobname.dtx}{units}}
          \file{si-hepunits.cfg}{\from{\jobname.dtx}{hepunits}}
          \file{si-fancynum.cfg}{\from{\jobname.dtx}{fancynum}}
          \file{si-fancyunits.cfg}{\from{\jobname.dtx}{fancyunits}}
}
\generate{\file{si-prefix.cfg}{\from{\jobname.dtx}{prefix}}
          \file{si-named.cfg}{\from{\jobname.dtx}{named}}
          \file{si-prefixed.cfg}{\from{\jobname.dtx}{prefixed}}
          \file{si-addn.cfg}{\from{\jobname.dtx}{addn}}
          \file{si-accepted.cfg}{\from{\jobname.dtx}{accepted}}
          \file{si-physical.cfg}{\from{\jobname.dtx}{physical}}
          \file{si-abbr.cfg}{\from{\jobname.dtx}{abbr}}
}
\generate{\file{si-synchem.cfg}{\from{\jobname.dtx}{synchem}}
          \file{si-hep.cfg}{\from{\jobname.dtx}{hep}}
          \file{si-astro.cfg}{\from{\jobname.dtx}{astro}}
          \file{si-geophys.cfg}{\from{\jobname.dtx}{geophys}}
          \file{si-binary.cfg}{\from{\jobname.dtx}{binary}}
}
\generate{\file{si-UK.cfg}{\from{\jobname.dtx}{UK}}
          \file{si-USA.cfg}{\from{\jobname.dtx}{USA}}
          \file{si-DE.cfg}{\from{\jobname.dtx}{DE}}
          \file{si-ZA.cfg}{\from{\jobname.dtx}{ZA}}
}
\nopreamble\nopostamble
\usedir{doc/latex/siunitx}
\generate{\file{README.txt}{\from{\jobname.dtx}{readme}}
          \file{\jobname.bib}{\from{\jobname.dtx}{refs}}
}
\endbatchfile
%</install>
%<*readme>
----------------------------------------------------------------
The siunitx package --- A comprehensive (SI) units package
Maintained by Joseph Wright
E-mail: joseph.wright@morningstar2.co.uk
Released under the LaTeX Project Public License v1.3c or later
See http://www.latex-project.org/lppl.txt
----------------------------------------------------------------

Typesetting values with units requires care to ensure that the
combined mathematical meaning of the value plus unit combination
is clear.  In particular, the SI units system lays down a
consistent set of units with rules on how these are to be used.
However, different countries and publishers have differing
conventions on the exact appearance of numbers (and units).

The siunitx provides a set of tools for authors to typeset
numbers and units in a consistent way.  The package has an
extended set of configuration options which make it possible to
follow varying typographic conventions with the same input
syntax.  The package includes automated processing of numbers
and units, and the ability to control tabular alignment of
numbers.

A number of LaTeX packages have been developed in the past for
formatting units: SIunits, sistyle, unitsdef, unitspkg,
fancyunits and fancynum.  Support for users of all of these
packages is available as emulation modules in siunitx.  In
addition, siunitx can carry out many of the functions of the
dcolumn, rccol and numprint packages.
%</readme>
%<*ignore>
\fi
% Will Robertson's trick
\immediate\write18{bibtex8 --wolfgang \jobname}
\immediate\write18{makeindex -s gind.ist -o \jobname.ind  \jobname.idx}
\immediate\write18{makeindex -s gglo.ist -o \jobname.gls  \jobname.glo}
%</ignore>
%<*driver>
\PassOptionsToClass{a4paper}{article}
\documentclass[german,english,UKenglish]{ltxdoc}
\EnableCrossrefs
\CodelineIndex
\RecordChanges
%\OnlyDescription
% Various adjustments are needed to the packages loaded.  xfrac is
% only loaded if it is available.
\usepackage[final]{graphicx}
% Currently, xfrac is not working with expl3!
%\IfFileExists{xfrac.sty}
%  {\usepackage{xfrac}}
%  {}
\usepackage{jawltxdoc}
\usepackage{textcomp,tikz,pgfplots,tocloft,datatool}
\usepackage[latin1]{inputenc}
\sisetup{
  eVcorra=0.1ex,
  unitsep=cdot,
  alsoload={synchem,hep,binary,astro,geophys},
  loctolang={UK:UKenglish,DE:german}}
\setlength{\cftsubsecnumwidth}{2.5em}
\begin{document}
  \DocInput{\jobname.dtx}
\end{document}
%</driver>
% \fi
%
%\CheckSum{10177}
%
% \CharacterTable
%  {Upper-case    \A\B\C\D\E\F\G\H\I\J\K\L\M\N\O\P\Q\R\S\T\U\V\W\X\Y\Z
%   Lower-case    \a\b\c\d\e\f\g\h\i\j\k\l\m\n\o\p\q\r\s\t\u\v\w\x\y\z
%   Digits        \0\1\2\3\4\5\6\7\8\9
%   Exclamation   \!     Double quote  \"     Hash (number) \#
%   Dollar        \$     Percent       \%     Ampersand     \&
%   Acute accent  \'     Left paren    \(     Right paren   \)
%   Asterisk      \*     Plus          \+     Comma         \,
%   Minus         \-     Point         \.     Solidus       \/
%   Colon         \:     Semicolon     \;     Less than     \<
%   Equals        \=     Greater than  \>     Question mark \?
%   Commercial at \@     Left bracket  \[     Backslash     \\
%   Right bracket \]     Circumflex    \^     Underscore    \_
%   Grave accent  \`     Left brace    \{     Vertical bar  \|
%   Right brace   \}     Tilde         \~}
%
%\GetFileInfo{\jobname.sty}
%
%\DoNotIndex{\@car,\@eha,\@ehc,\@empty,\@firstofone,\@firstoftwo}
%\DoNotIndex{\@for,\@gobble,\@gobbletwo,\@ifl@aded,\@ifnextchar}
%\DoNotIndex{\@ifpackagelater,\@ifpackageloaded,\@ifstar}
%\DoNotIndex{\@ifundefined,\@makeother,\@namedef,\@nameuse,\@ne}
%\DoNotIndex{\@nil,\@secondoftwo,\@temptokena,\@undefined,\AA}
%\DoNotIndex{\active,\addto,\addtolength,\advance,\afterassignment}
%\DoNotIndex{\aftergroup,\AtBeginDocument,\AtEndOfPackage}
%\DoNotIndex{\begingroup,\bfseries,\bgroup,\boldmath,\box}
%\DoNotIndex{\capitalring,\catcode,\cdot,\chardef,\check@mathfonts}
%\DoNotIndex{\circ,\csname,\DeclareFontEncoding,\DeclareFontFamily}
%\DoNotIndex{\DeclareFontShape,\DeclareFontSubstitution}
%\DoNotIndex{\DeclareMathSymbol,\DeclareRobustCommand}
%\DoNotIndex{\DeclareSymbolFont,\DeclareTextSymbol}
%\DoNotIndex{\DeclareTextSymbolDefault,\def,\define@boolkey}
%\DoNotIndex{\define@choicekey,\define@cmdkey,\define@cmdkeys}
%\DoNotIndex{\define@key,\displaystyle,\displaywidth,\divide,\do}
%\DoNotIndex{\edef,\egroup,\else,\encodingdefault,\end,\endcsname}
%\DoNotIndex{\endgroup,\endinput,\endtabular,\ensuremath}
%\DoNotIndex{\expandafter,\f@family,\f@series,\fam,\fi,\font}
%\DoNotIndex{\fontsize,\frac,\gdef,\global,\hbox,\hfil,\hfill,\hskip}
%\DoNotIndex{\hspace,\hss,\ifcsname,\ifdim,\iffalse,\IfFileExists}
%\DoNotIndex{\ifmmode,\ifnum,\ifx,\ignorespaces,\inputencodingname}
%\DoNotIndex{\InputIfFileExists,\key@ifundefine,\let,\long,\m@ne}
%\DoNotIndex{\makeatletter,\math@version,\mathchoice,\mathnormal}
%\DoNotIndex{\mathord,\mathrm,\mathsf,\mathtt,\mdseries}
%\DoNotIndex{\MessageBreak,\mkern,\mp,\NC@find,\NeedsTeXFormat}
%\DoNotIndex{\newbox,\newcolumntype,\newcommand,\newcount,\newdimen}
%\DoNotIndex{\newif,\newlength,\newtoks,\nobreak,\noexpand,\p@}
%\DoNotIndex{\PackageError,\PackageInfo,\PackageWarning,\phantom}
%\DoNotIndex{\ProcessOptionsX,\protect,\protected@edef,\ProvidesFile}
%\DoNotIndex{\ProvidesPackage,\Q,\raisebox,\relax,\renewcommand}
%\DoNotIndex{\RequirePackage,\rlap,\sbox,\scantokens}
%\DoNotIndex{\scriptscriptstyle,\scriptstyle,\selectfont,\setbox}
%\DoNotIndex{\setkeys,\setlength,\settoheight,\sf@size,\sfdefault}
%\DoNotIndex{\sfrac,\skewchar,\sp,\space,\string,\strip@pt}
%\DoNotIndex{\tabularnewline,\text,\textdegree,\textmu,\textohm}
%\DoNotIndex{\textsc,\textstyle,\textsuperscript,\the,\thr@@,\times}
%\DoNotIndex{\ttdefault,\uccode,\unboldmath,\unhbox,\upmu,\Upomega}
%\DoNotIndex{\uppercase,\upshape,\wd,\xdef,\xspace,\z@}
%
%\DoNotIndex{\if,\lowercase,\+,\,,\-,\.,\:,\;,\@argdef,\@cdr}
%\DoNotIndex{\@ifdefinable,\@testopt,\@yargdef,\color,\endlinechar}
%\DoNotIndex{\key@ifundefined,\newrobustcmd,\protected,\scriptspace}
%\DoNotIndex{\usebox,\tw@,\ifinner,\usepackage,\texttt,\textsf}
%\DoNotIndex{\NC@do,\loop,\l@ngrel@x,\g@addto@macro,\documentclass}
%\DoNotIndex{\document\,\begin,\=,\@ehb,\\,\listfiles,\nullfont}
%\DoNotIndex{\normalsize,\futurelet}
%
%\DoNotIndex{\pascalsecond,\amperemetresecond,\ohmmetre}
%\DoNotIndex{\newtonmetre,\newtonmetrenp}
%\DoNotIndex{\kilogramsquaremetre,\kilogramsquaremetrenp}
%
%\changes{v1.0}{2008/06/12}{First official release}
%
%\setkeys{lst}{language=[LaTeX]{TeX},moretexcs={sisetup,num,ang,SI,
%  si,newunit,newpower,newprefix,metre,candela,kelvin,ampere,eV,
%  second,mole,per,kilogram,Square,squared,cubed,cubic,kilo,centi,
%  gram,tothe,toprule,midrule,bottomrule,newton,micro,joule,mega,
%  raiseto,deci,mathnormal,numprint,electronvolt,selectlanguage,
%  kibi,byte,molar,Molar,mmHg,mebi,torr,parsec,lightyear,clight,
%  eVperc,addplot,color,milli,cm,Hz,kg,DeclareRobustCommand,
%  pascal,dalton,addtolocale,DTLnewdb,DTLnewrow,DTLnewdbentry,
%  DTLforeach,DTLiffirstrow,DTLmul,newqualifier,hour,renewunit,
%  degree}}
%
%\title{\currpkg\ --- A comprehensive (SI) units package^^A
%  \thanks{This file describes version \fileversion, last revised
%    \filedate.}}
%\author{Joseph Wright^^A
%  \thanks{E-mail: joseph.wright@morningstar2.co.uk}}
%\date{Released \filedate}
%
%\maketitle
%
%\newcommand*{\SIunits}{\pkg{SIunits}\xspace}
%\newcommand*{\SIstyle}{\pkg{SIstyle}\xspace}
%\newcommand*{\unitsdef}{\pkg{unitsdef}\xspace}
%\newcommand*{\units}{\pkg{units}\xspace}
%\newcommand*{\numprint}{\pkg{numprint}\xspace}
%\begin{abstract}
% Typesetting values with units requires care to ensure that the
% combined mathematical meaning of the value plus unit combination is
% clear.  In particular, the SI units system lays down a consistent
% set of units with rules on how these are to be used.  However,
% different countries and publishers have differing conventions on
% the exact appearance of numbers (and units).
%
% The \currpkg provides a set of tools for authors to typeset numbers
% and units in a consistent way.  The package has an extended set of
% configuration options which make it possible to follow varying
% typographic conventions with the same input syntax.  The package
% includes automated processing of numbers and units, and the ability
% to control tabular alignment of numbers.
%
% A number of \LaTeX\ packages have been developed in the past for
% formatting units: \SIunits, \SIstyle, \unitsdef, \units,
% \pkg{fancyunits} and \pkg{fancynum}. Support for users of all of
% these packages is available as emulation modules in \currpkg.
% In addition, \currpkg can carry out many of the functions of the
% \pkg{dcolumn}, \pkg{rccol} and \numprint packages.
%\end{abstract}
%
%\begin{multicols}{2}
%  \tableofcontents
%\end{multicols}
%
%\clearpage
%\part{Introduction}
% The correct application of units of measurement is very important
% in technical applications.  For this reason, carefully-crafted
% definitions of a coherent units system have been laid down by the
% \latin{Conf\'erence G\'en\'erale des Poids et
% Mesures}\footnote{General Conference on Weights and Measures.}
% (CGPM): this has resulted in the \latin{Syst\`eme International
% d'Unit\'es}\footnote{International System of Units.} (SI).  At the
% same time, typographic conventions for correctly displaying both
% numbers and units exist to ensure that no loss of meaning occurs in
% printed matter.
%
% \currpkg aims to provide a unified method for \LaTeX\ users to
% typeset units and values correctly and easily.  The design
% philosophy of \currpkg is to follow the agreed rules by default,
% but to allow variation through option settings.  In this way,
% users can use \currpkg to follow the requirements of publishers,
% co-authors, universities, \etc without needing to alter the input
% at all.
%
% \currpkg is intended as a complete replacement for \SIunits,
% \SIstyle, \unitsdef, \units, \pkg{fancyunits} and
% \pkg{fancynum}. As such, emulation modes are provided for all of
% these packages.  Where possible, conventions from the existing
% solutions have been used here.  For example, the macros \cs{num},
% \cs{ang} and \cs{SI} act in a very similar fashion to those in
% existing packages.
%
%\clearpage
%\part{Using the \currpkg package}
%
%\section{For the impatient}
% \currpkg provides the user macros:
%\begin{itemize}
%  \item \cs{SI}\oarg{options}\marg{value}\oarg{pre-unit}\marg{unit}
%  \item \cs{si}\oarg{options}\marg{unit}
%  \item \cs{num}\oarg{options}\marg{number}
%  \item \cs{ang}\oarg{options}\marg{angle}
%  \item \cs{sisetup}\marg{options}
%\end{itemize}
% plus the |S| and |s| column types for decimal alignments and units
% in tables.  These macros are designed for typesetting units and
% values with control of appearance and with intelligent processing.
%
% By default, all text is typeset in the current upright, serif maths
% font.  This can be changed by setting the appropriate package
% options: \opt{obeyall} will use the current font for typesetting.
%
% The package includes a ``unit processor'', which allows the use of
% named units or literal values.  Named units are processed to
% correctly include powers.
%\begin{LaTeXexample}
%  \SI{10}{\gram}\\
%  \SI{23.4}{g.cm^3}\\
%  \num{1e34}\\
%  \ang{1;2;3}\\
%  \emph{\SI{16,7}{\metre\per\second}}\\
%  \textbf{\SI{30e3}{\Hz}}\\
%  \SI{1.2 x 3.56 x 9.2}{\milli\metre}\\
%  \sisetup{obeyall}
%  \textbf{\SI{-4.5}{\cm}}\\
%  \si{\joule\per\mole\per\kelvin}\\
%  \si[per=frac]{\joule\per\mole\per\kelvin}\\
%  \num[dp=4]{1.23456}\\
%  \num[dp=4]{9.8}\\
%  \begin{tabular}{S[tabformat=3.2]}
%    \toprule
%    {Heading}\\
%    \midrule
%    1.3 \\
%    134.2 \\
%    3.56 \\
%    74,7 \\
%    \bottomrule
%  \end{tabular}
%\end{LaTeXexample}
%
%\section{Requirements}
% \currpkg requires a reasonably up to date \TeX\ system.  The
% package requires \eTeX-extensions, which should be available on
% most systems.\footnote{If you have an old \LaTeX\, try
% ``\texttt{elatex}'' rather than ``\texttt{latex}''.}  The following
% packages are also needed:
%\begin{itemize}
%  \item \pkg{array} and \pkg{xspace}: from the \pkg{tools} bundle,
%    which should be available to everyone;
%  \item \pkg{xkeyval}: this processes the option handling, and needs
%    to be at least v2.5;
%  \item \pkg{amstext}: from the \AMS\TeX\ support bundle (the
%    \AMS\ fonts are also needed to provide the default upright
%    \si{\micro}).
%\end{itemize}
% Hopefully most people using the package will have access to all of
% those items.
%
% To use the \opt{fraction=sfrac} option, the \pkg{xfrac} package is
% needed.  This needs various experimental \LaTeX3 packages.  As a
% result, \currpkg does not load \pkg{xfrac}.  If you want to use
% \opt{fraction=sfrac}, \emph{you} need to load \pkg{xfrac} in your
% preamble before \currpkg.\footnote{This document has been compiled
% in this way.  You have to load \pkg{xfrac} first as otherwise very
% nasty things happen with \pkg{xkeyval}.  MiK\TeX\ users should note
% that the packaged versions of \pkg{expl3}, \pkg{template} and
% \pkg{xparse} will not work with \pkg{xfrac}: download copies from
% \textsc{ctan}!} If the package is not loaded, \opt{fraction=sfrac}
% falls back on a \pkg{nicefrac}-like method.  The interested user
% should look at the \pkg{xfrac} documentation for reasons this might
% not be ideal.\footnote{On the other hand, some fractional units
% will look really bad with \cs{sfrac}.  Use this option with
% caution.}
%
%\section{Loading the package}
% \currpkg is loaded by the usual \LaTeX\ method.
%\begin{LaTeXexample}[noexample]
%  \usepackage[<options>]{siunitx}
%\end{LaTeXexample}
% As is shown in the example, the package can be loaded with one or
% more options, using the key--value system.  The full range of
% package options are described in Section~\ref{sec:options}; some
% options are described in the along with the appropriate user
% macros.  Most of the user macros accept the same key--value
% settings as an optional argument.
%
%\section{Numbers}
%\DescribeMacro{\num}
% Numbers are automatically formatted by the \cs{num} macro.  This
% takes one optional and one mandatory argument:
% \cs{num}\oarg{options}\marg{number}.  The contents of \meta{number}
% are automatically formatted, in a similar method to that used by
% \numprint.  The formatter removes ``hard'' spaces (\cs{,} and |~|),
% automatically identifies exponents (by default marked using |e| or
% |d|) and adds the appropriate spacing of large numbers.  A leading
% zero is added before a decimal marker, if needed: both ``|.|'' and
% ``|,|'' are recognised as decimal marker.
%\begin{LaTeXexample}
%  \num{1}    \num{123}     \num{1234}  \num{12345}\\
%  \num{0.1}  \num{0.123}   \num{0,1234} \num{.12345}\\
%  \num{1e10} \num{3.45d-4} \num{-e10}
%\end{LaTeXexample}
% Various error-checking systems are built into the package, so that
% if \meta{number} does not contain any numeric characters, a warning
% is issued.  Isolated signs are also detected.  The package
% recognises |(| and |)| as ``extra'' characters, which can be used
% to indicate the error in a number.\footnote{This is common in
% chemical crystallography, for example.}  The \opt{seperr} causes
% this data to be given as a separate error value.  If the number
% also contains an exponent, then brackets are re-added after the
% separation to ensure that meaning is not lost.
%\begin{LaTeXexample}
%  $\num{1.234(5)} = \num[seperr]{1.234(5)}$\\
%  $\num{1.234(5)e6} = \num[seperr]{1.234(5)e6}$
%\end{LaTeXexample}
% The same applies to the unit and value macro \cs{SI}, described
% later, for example the rest mass of an electron \cite{IUPAC}:
%\begin{LaTeXexample}
%   $ m_{\mathrm{e}}
%     = \SI{9.1093897(54)e-31}{\kg} $\\
%   $ m_{\mathrm{e}}
%     = \SI[seperr]{9.1093897(54)e-31}{\kg} $
%\end{LaTeXexample}
%
% A number of effects are available as options.  These are fully
% explained in Section~\ref{sec:options}.  Some of the more useful
% options are illustrated here.  By default, the output of the
% package is typeset in maths mode.  However, the use of the current
% text font can be forced.\footnote{This document is typeset using
% lowercase numbers in text mode, which emphasises the effect here.}
%\begin{LaTeXexample}
%  \num{1234567890} \num[mode=text]{1234567890}
%\end{LaTeXexample}
% \currpkg can automatically add zeros and signs to numbers.  This
% can be altered as desired.
%\begin{LaTeXexample}
%  \num{1.}  \num[padnumber=all]{1.}\\
%  \num{2}   \num[addsign=all]{2}\\
%  \num{3e4} \num[addsign=mant]{3e4} \num[addsign=all]{3e4}\\
%  \num{.5}  \num[padnumber=none]{.5}
%\end{LaTeXexample}
% The separation of digits can be turned on and off, and the output
% changed.
%\begin{LaTeXexample}
%  \num{1234}  \num[sepfour=true]{1234}\\
%  \num{12345} \num[digitsep=comma]{12345}\\
%  \num[digitsep=none]{12345}
%\end{LaTeXexample}
% The formatting of exponents is also customisable.
%\begin{LaTeXexample}
%  \num{1e10} \num[expproduct=cdot]{1e10}\\
%  \num{2e20} \num[expbase=5]{2e20}\\
%  \num{3e30} \num[expproduct=tighttimes]{3e30}
%\end{LaTeXexample}
%
% \currpkg can automatically add colour to negative numbers, which is
% often useful for highlighting purposes.  This is turned on with the
% \opt{colourneg} option; the colour used is set by \opt{negcolour}.
% Both of these are available with the US spellings: \opt{colorneg}
% and \opt{negcolor}.
%\begin{LaTeXexample}
%  \num{-1}\\
%  \sisetup{colourneg} \num{-2}\\
%  \num{3e-3}\\
%  \num[negcolour=blue]{-4}
%\end{LaTeXexample}
%
% \currpkg can automatically zero-fill and round to a fixed number of
% decimal places.  This is controlled by two options \opt{fixdp} and
% \opt{dp}.  The later is an integer which specifies how many places
% to fix to; setting this option automatically sets \opt{fixdp} to
% \opt{true}.  The place-fixing system will only alter pure numbers: for
% example, any error component will result in the input being left
% unchanged.
%\begin{LaTeXexample}
%  \num{1.23456}\\
%  \sisetup{dp=2}
%  \num{1.23456}\\
%  \num{9.8}\\
%  \num{-10.432}\\
%  \num{44.3221(2)}
%\end{LaTeXexample}
%
%\section{Tabular material}
%\subsection{Aligning numbers}
% Centring numbers in tabular content is handled by a new column
% type, the |S| column.  This is based closely on the \pkg{dcolumn}
% method for centring numbers in columns, but adds the functionality
% of the \cs{num} macro.\footnote{The approach used is actually a
% combination of \pkg{dcolumn} for centring the material and
% \numprint for processing it.  It will therefore give rather
% different results than the \texttt{n} and \texttt{N} column types
% in \numprint.}
%
% By default, the decimal marker of the number is placed at the
% centre of the column, which then resizes to accommodate the width
% of the contents (\ref{tab:default}).  This behaviour is set by the
% \opt{tabnumalign=centredecimal} option.  By setting the
% \opt{tabnumalign} option to \opt{centre}, the centre of the space
% reserved for the number is placed at the centre of the column.  The
% space reserved is stored in \opt{tabformat}, which is of the form
% \texttt{\meta{before}\meta{dec}\meta{after}}, where \meta{before}
% is the number of characters before the decimal marker and
% \meta{after} is the number after. Thus in the example,
% \opt{tabformat=2.4} provides space for two digits before the
% decimal marker and four after.  \opt{tabnumalign} can also be set
% to \opt{left} and \opt{right}, with the expected results.
%\begin{LaTeXexample}[float]
%  \begin{table}
%    \caption{Behaviour of \texttt{S} column type}
%    \label{tab:default}
%    \centering
%    \begin{tabular}{%
%      S%
%      S[tabnumalign=centre,tabformat=2.4]%
%      S[tabnumalign=right,tabformat=2.4]%
%      S[tabnumalign=centre,tabformat=2.4,decimalsymbol=comma]}
%    \toprule
%    {Some Values} & {Some Values} & {Some Values} & {Some Values} \\
%    \midrule
%     2.3456  &  2.3456  &  2.3456 &  2.3456 \\
%    34.2345  & 34.2345  & 34.2345 & 34.2345 \\
%    56.7835  & 56.7835  & 56.7835 & 56.7835 \\
%    90.473   & 90.473   & 90.473  & 90.473  \\
%    \bottomrule
%    \end{tabular}
%  \end{table}
%\end{LaTeXexample}
% The \opt{tabformat} setting can also be used to reserve space for
% numbers containing exponents.  This is given in the same format as
% above, but with a mantissa and exponent part (\ref{tab:exptab}).
% Notice that this is designed to expect that numbers will contain a
% mantissa.  Exponents can either be aligned so that the ``$\times$''
% symbols match up vertically, or the exponent part can be allowed to
% move across as needed. Space for signs is added by using any sign
% in the \opt{tabformat}, so for example \opt{tabformat=+2.2} and
% \opt{tabformat=-2.2} have exactly the same effect. Setting
% \opt{tabformat} will automatically switch \opt{tabnumalign} from
% \opt{centredecimal} to \opt{centre}, if the former is
% currently set. In other cases, the current alignment option is
% retained.
%\begin{LaTeXexample}[float]
%  \begin{table}
%    \caption{Exponents in tables}
%    \label{tab:exptab}
%    \centering
%    \begin{tabular}{%
%      S[tabnumalign=right,tabformat=2.2e2]%
%      S[tabnumalign=centre,tabformat=2.2e1.1]%
%      S[tabnumalign=centre,tabformat=2.2e1.1,tabalignexp=false]%
%      S[tabnumalign=centre,tabformat=+2.2]}
%    \toprule
%        {Longer values}
%      & {Longer values}
%      & {Longer values}
%      & {Values} \\
%    \midrule
%    2.3e1    &  2.34e1  &  2.34e1  & +2.31 \\
%    34.23e45 & 34.23e45 & 34.23e45 & 34.23 \\
%    56.78    & 56.78    & 56.78    & -56.78 \\
%    1.0e34   & 1.0e34   & 1.0e34   & +-1.0 \\
%    \bottomrule
%    \end{tabular}
%  \end{table}
%\end{LaTeXexample}
%
% Data not to be processed as a number should be protected by
% wrapping it in braces: this is most likely to be true for column
% headers (again as illustrated).  By default, the contents of
% non-numeric cells are centred.  This can be altered by setting
% \opt{tabtextalign}, which can be set to \opt{left}, \opt{right} or
% \opt{centre}.  The use of digit separators in table columns is
% accounted for: extra space is reserved if digit separators will be
% added.
%
%\subsection{Columns of units}
% As a complement to the |S| column, \currpkg also provides a second
% column type, |s|.  This is intended for producing columns of units.
% The letters chosen are intended to be similar to \cs{SI} and
% \cs{si}, respectively.  The alignment of material in |s| columns is
% governed by the \opt{tabunitalign} option.
%\begin{LaTeXexample}[float]
%  \begin{table}
%  \centering
%  \caption{Number and units in tables}
%  \label{tab:num-unit}
%    \begin{tabular}{%
%      S[tabformat=1.2e-1,tabnumalign=centre]%
%      s[tabunitalign=left]}
%    \toprule
%      {Value} & \multicolumn{1}{c}{Unit} \\
%    \midrule
%      2.16e-5 & \metre\squared\per\second \\
%      2.83e-6 & \metre\squared\per\second \\
%      7.39e3  & \pascal\cubic\metre\per\mole \\
%      1.0e5   & \pascal \\
%    \bottomrule
%    \end{tabular}
%  \end{table}
%\end{LaTeXexample}
%
% As the \cs{si} macro can take literal or macro input, the |s|
% column cannot validate the input.  \emph{Everything} in an |s|
% column is therefore passed to the \cs{si} macro for processing.  To
% prevent this, you have to use \cs{multicolumn}, as is shown in
% \ref{tab:s-limits}.  Notice that the braces do not prevent
% processing and colouring of the cell contents.
%\begin{LaTeXexample}[float]
%  \begin{table}
%  \centering
%  \caption{The \texttt{s} column processes everything}
%  \label{tab:s-limits}
%    \begin{tabular}{%
%      s[colourall,colour=orange]%
%      s[colourall,colour=orange]}
%    \toprule
%      {Unit}    & \multicolumn{1}{c}{Unit}\\
%    \midrule
%      {m^3}     & \multicolumn{1}{c}{\si{m^3}} \\
%      \kilogram & \kilogram \\
%    \bottomrule
%    \end{tabular}
%  \end{table}
%\end{LaTeXexample}
%
%\section{Angles}
%\DescribeMacro{\ang}
% Angles can be typeset using the \cs{ang} command.  This takes two
% arguments, \cs{ang}\oarg{options}\marg{angle}, where \meta{options}
% can be any of the package options to apply only to this value.
% \meta{angle} can be given either as a decimal number or as a
% semi-colon separated list of degrees, minutes and seconds, \ie\
% \cs{ang}\marg{decimal angle} or
% \cs{ang}|{|\meta{degrees}|;|\meta{minutes}|;|\meta{seconds}|}|. By
% default, no space is introduced between angles and the degrees,
% minutes and seconds markers.
%\begin{LaTeXexample}
%  \ang{10} \ang{12.3} \ang{4,5}\\
%  \ang{1;2;3} \ang{;;1}\\
%  \ang{+10;;} \ang{-0;1;}
%\end{LaTeXexample}
%
% By default, angles with no degrees (or minutes) are zero-filled;
% angles with degrees but no minutes or seconds are not filled.  This
% behaviour can be altered using the package options.
%\begin{LaTeXexample}
%  \ang{;;1} \ang[padangle=none]{;;1}\\
%  \ang{2;;} \ang[padangle=all]{2;;}\\
%  \sisetup{padangle=all} \ang{;3;} \ang{4;;} \ang{;;5}
%\end{LaTeXexample}
% The \cs{num} macro is used to typeset each number of the angle, so
% the options for \cs{num} also apply here.  The \opt{anglesep} value
% can be used to separate degrees, minutes and seconds.
%\begin{LaTeXexample}
%  \ang{1.05}    \ang[decimalsymbol=comma]{1.05}\\
%  \ang{3.67890} \ang[digitsep=comma]{3.67890}\\
%  \ang{9;8;7}   \ang[anglesep=thin]{9;8;7}
%\end{LaTeXexample}
% The degrees, minutes and seconds signs can be placed over the
% decimal sign using the \opt{astroang} option.  This is designed on
% the assumption that only the last number given has a decimal
% part.
%\begin{LaTeXexample}
%  \ang{1.2} \ang[astroang]{1.2}\\
%  \ang{1;2.3;} \ang[astroang]{1;2.3;}\\
%  \ang{1;2;3.4} \ang[astroang]{1;2;3.4}
%\end{LaTeXexample}
%
%\section{Units and values}
%\DescribeMacro{\SI}
% The core aim of \currpkg is correctly typesetting values which have
% units. The main output macro here is \cs{SI}, which has the same
% syntax as the macros with the same name in \SIstyle and \unitsdef
% packages. The \cs{SI} macro takes two mandatory arguments, in
% addition to the optional set up argument, and a second optional
% argument:
% \cs{SI}\oarg{options}\marg{number}\oarg{preunit}\marg{unit}.  The
% \meta{number} argument operates in exactly the same manner as the
% equivalent argument of the \cs{num} macro.  \meta{unit} will be
% typeset with a non-breakable space between it and the preceding
% number, with font control as outlined earlier.  Finally,
% \meta{preunit} is a unit to be typeset \emph{before} the numerical
% value (most likely to be a currency).  Some examples illustrate the
% general power of the macro.
%\begin{LaTeXexample}
%  \SI[mode=text]{1.23}{J.mol^{-1}.K^{-1}}\\
%  \SI{.23e7}{\candela}\\
%  \SI[per=slash]{1.99}[\pounds]{\per\kilogram}\\
%  \SI{70}{\metre\per\second}\\
%  \SI[per=frac,fraction=nice]{1,345}{\ampere\per\mole}
%\end{LaTeXexample}
% The use of unit macros outside of the \cs{SI} macro is described
% later.
%
%\subsection{Literal units}
%\changes{v1.1}{2008/07/23}{Unit processor extended to interpret
%  \texttt{\textunderscore} correctly}
% Units can be input in two ways, inspired by \SIstyle and \SIunits.
% The \SIstyle-like method uses literal input.  Five characters have
% a special meaning:
%\begin{itemize}
%  \item ``|^|'' and ``|_|''  The superscript and subscript
%    characters can be used without the usual
%    need for surrounding maths characters (|$|);^^A$
%  \item ``|.|'' and ``|,|'': the full stop (point) symbol and comma
%    are made active, and produce the current contents of the
%    \opt{unitsep} option;
%  \item ``|~|'' The contents of the \opt{unitspace} option are typeset
%    by a tilde.
%\end{itemize}
% This allows ready input of units.
%\begin{LaTeXexample}
%  \SI{10}{kg.m.s^{-2}}\\
%  \SI{1.453}{g/cm^3}\\
%  \SI{6.345}{kg_{pol}.mmol_{cat}^{-1}}\\
%  \SI{33.562}{cd~s}\\
%  \SI[unitsep=medium]{100}{m.s^{-2}}
%\end{LaTeXexample}
% The literal unit system will correctly typeset input containing the
% symbols \si{\micro} (micro), \si{\degree} (degree) and
% \si{\angstrom} (ring-A).\footnote{Currently this works with \XeTeX\
% and \pkg{inputenc} using the \texttt{latin1}, \texttt{latin5} and
% \texttt{latin9} encodings.}
% \begingroup\makeatletter\si@unt@nonlatin
%\begin{LaTeXexample}
%  \SI{10}{�m}\\
%  \SI{20}{�C}\\
%  \SI{30}{�}
%\end{LaTeXexample}
%\endgroup
%
%\subsection{The unit interpreter}
% The second operation mode for the \cs{SI} macro is based on the
% behaviour of \SIunits.  Here, each unit, SI multiple prefix and
% power is given a macro name.  These are entered in a method very
% similar to the reading of the unit name in English.
%\begin{LaTeXexample}
%  \SI{10}{\kilo\gram\metre\per\second\squared}\\
%  \SI{1.453}{\gram\per\cubic\centi\metre}\\
%  \SI{33.562}{\candela\second}\\
%  \SI[unitsep=medium]{100}{\metre\per\Square\second}\\
%  \SI[prefixsymbolic=false]{4.56}{\kilo\metre\per\second}
%\end{LaTeXexample}
% On its own, this is very similar to \SIunits, and is less
% convenient than the direct input method.\footnote{Users of \SIunits
% should note the lack of need for a \cs{usk}-type macro.}  However,
% the package allows you to define new unit macros; a large number of
% pre-defined abbreviations are also supplied.  More importantly, by
% defining macros for units, instead of literal values, new
% functionality is made available.  Units may be re-defined to give
% different output, and handling of reciprocal values can be altered.
%\begin{LaTeXexample}
%  \SI[per=frac,fraction=frac]{10}{\gram\metre\per\second\squared}\\
%  \SI[per=slash]{1.453}{\gram\per\cubic\centi\metre}\\
%  \SI{33.562}{\candela\second}\\
%  \SI[per=frac,fraction=nice]{100}{\metre\per\Square\second}
%\end{LaTeXexample}
% The unit processor will trap \emph{some} errors in the input and
% give the ``best guess'' result.  However, it is down to the user to
% check the output.
%
%\subsection{Powers of units}
%\DescribeMacro{\Square}
%\DescribeMacro{\ssquare}
%\DescribeMacro{\squared}
%\DescribeMacro{\cubic}
%\DescribeMacro{\cubed}
% Including powers in units is handled using a ``natural language''
% method.  Thus preceding a unit by \cs{Square} or \cs{cubic} which
% raise the unit to the appropriate power, while \cs{squared} or
% \cs{cubed} follow the unit they apply to. The \cs{Square} macro is
% capitalised to avoid a name clash with \pkg{pstricks}; the
% alternative \cs{ssquare} is also provided.
%\begin{LaTeXexample}
%  \SI{10}{\metre\squared}\\
%  \SI{20}{\Square\metre}\\
%  \SI{30}{\metre\cubed}\\
%  \SI{40}{\cubic\metre}
%\end{LaTeXexample}
%
%\DescribeMacro{\per}
% The \cs{per} macro intelligently creates reciprocal powers, and
% also adds the power $-1$ when appropriate.
%\begin{LaTeXexample}
%  \SI{10}{\per\second\squared}\\
%  \SI{20}{\per\Square\second}\\
%  \SI[per=frac,fraction=nice]{30}{\per\second\cubed}\\
%  \SI[per=slash]{40}{\per\cubic\second}\\
%  \SI{50}{\per\second}\\
%  \SI{60}{\per\metre\Square\candela}
%\end{LaTeXexample}
%
%\DescribeMacro{\tothe}
%\DescribeMacro{\raiseto}
% For powers not defined above or with \cs{newpower}, the \cs{tothe}
% macro can be used ``in line'' to produce a power.  As follows from
% standard English usage, this comes after the unit. \cs{raiseto}
% achieves the same, but is used \emph{before} a unit to add a power
% \emph{after}.\footnote{\cs{raiseto} acts in the same way as
% \cs{tothe} when used in a literal context: the power will be
% produced where the macro is, rather than moving after the next
% item.}
%\begin{LaTeXexample}
%  \SI{16.86}{\metre\tothe{4}}\\
%  \SI{7.895}{\raiseto{-6}\newton}\\
%  \SI{1.34}{\per\kelvin\tothe{7}}
%\end{LaTeXexample}
%
%\subsection{Units with no values}
%\DescribeMacro{\si}
% For typesetting the symbol for a unit on its own, with the full
% font control and without extra spaces, the \cs{si} macro is
% provided.\footnote{The same effect can be achieved using the
% \cs{SI} macro with an empty numerical argument.}  The macro name
% avoids a clash with the functionality of the earlier packages, but
% is similar to \cs{ilu} from the \unitsdef package.
%\begin{LaTeXexample}
%  \SI{}{kg.m/s^2}\\
%  \si{kg.m/s^2}\\
%  \si[mode=text,unitsep=thin]{\mole\per\cubic\deci\metre}
%\end{LaTeXexample}
%
%\subsection{Free-standing units}
% Users of the \unitsdef package will be accustomed to using unit
% macros on their own (following a value) or with an optional
% argument containing a number.  In both cases, only a single unit
% macro could be used. \currpkg supports both operation modes, with
% the limitation that units trailing values loose font control of the
% value.  When used in this way, the units \emph{do not} take an
% optional \pkg{keyval} argument.
%\begin{LaTeXexample}
%  \sisetup{prespace,allowoptarg}
%  123\metre\\
%  \kelvin[123]\\
%  \sisetup{mode=text} \ampere[234]\\
%  6\second
%\end{LaTeXexample}
%
%\subsection{Pre-defined units, prefixes and powers}
%\newcommand*{\csindex}[1]{^^A
%  \cs{#1}^^A
%  \expandafter\SpecialUsageIndex\csname#1\endcsname}
%\newcommand*{\unitinfo}[1]{^^A
%  #1 & \csindex{#1} &
%  \expandafter\si\expandafter{\csname#1\endcsname}}
%\begin{table}
%  \caption{The seven base SI units}
%  \label{tab:baseunits}
%  \centering
%  \begin{tabular}{lll}
%    \toprule
%      \multicolumn{1}{c}{Unit} &
%      \multicolumn{1}{c}{Macro} &
%      \multicolumn{1}{c}{Symbol} \\
%    \midrule
%    \unitinfo{kilogram}\\
%    \unitinfo{metre}\\
%    \unitinfo{second}\\
%    \unitinfo{mole}\\
%    \unitinfo{kelvin}\\
%    \unitinfo{ampere}\\
%    \unitinfo{candela}\\
%    \bottomrule
%  \end{tabular}
%\end{table}
%\DescribeMacro{\metre}
%\DescribeMacro{\meter}
% The package always defines the seven base SI units, irrespective of
% any package options given (\ref{tab:baseunits}). The kilogram is
% notable as by default it is a \emph{base} unit with a prefix. Thus,
% when the package is loaded with the option \opt{load=\{\}},
% \cs{kilo} and \cs{gram} \emph{are not defined}.  As metre is often
% spelled as ``meter'' in the US, the macro \cs{meter} is provided in
% addition to the \cs{metre} macro.\footnote{The official SI spelling
% for the unit is ``metre''.}
%
% By default, a number of additional definitions are created by the
% package.  These are controlled by the \opt{load} and \opt{noload}
% options. Unless specifically requested with the option
% \opt{noload=prefix}, \currpkg defines the standard prefixes for
% powers of ten (\ref{tab:prefix}).  This leads to the redefinition
% of \cs{kilogram} as \cs{kilo}\cs{gram}.  The macro \cs{deka} is
% provided, as this is used as an alias for \cs{deca} in some places.
% The package also defines a number of derived SI units which have
% assigned names and symbols (\ref{tab:derived}). Note that \cs{Gray}
% is capitalised to avoid a name clash with the \pkg{pstricks}
% package.\footnote{The macros \cs{ohm} and \cs{celsius} are not
% defined by \currpkg if the \pkg{gensymb} package is loaded.}
%\newcommand*{\prefixunitsym}[1]{\si[prefixsymbolic=false]{#1}}
%\newcommand*{\prefixinfo}[1]{^^A
%  #1 & \csindex{#1} &
%    \expandafter\prefixunitsym\expandafter{\csname#1\endcsname} &
%    \expandafter\si\expandafter{\csname#1\endcsname}}
%\begin{table}
%  \caption{The SI prefixes (\opt{load=prefix})}
%  \label{tab:prefix}
%  \centering
%  \begin{tabular}{lllclllc}
%    \toprule
%      \multicolumn{1}{c}{Prefix} &
%      \multicolumn{1}{c}{Macro} &
%      \multicolumn{1}{c}{Power} &
%      Symbol &
%      \multicolumn{1}{c}{Prefix} &
%      \multicolumn{1}{c}{Macro} &
%      \multicolumn{1}{c}{Power} &
%      Symbol \\
%    \midrule
%    \prefixinfo{yocto} &
%    \prefixinfo{deca} \\
%    \prefixinfo{zepto} &
%    \prefixinfo{hecto} \\
%    \prefixinfo{atto} &
%    \prefixinfo{kilo} \\
%    \prefixinfo{femto} &
%    \prefixinfo{mega} \\
%    \prefixinfo{pico} &
%    \prefixinfo{giga} \\
%    \prefixinfo{nano} &
%    \prefixinfo{tera} \\
%    \prefixinfo{micro} &
%    \prefixinfo{peta} \\
%    \prefixinfo{milli} &
%    \prefixinfo{exa} \\
%    \prefixinfo{centi} &
%    \prefixinfo{zetta} \\
%    \prefixinfo{deci} &
%    \prefixinfo{yotta} \\
%    \bottomrule
%  \end{tabular}
%\end{table}
%\begin{table}
%  \caption{The derived SI units with defined names
%    (\opt{load=derived})}
%  \label{tab:derived}
%  \centering
%  \begin{tabular}{llllll}
%    \toprule
%      \multicolumn{1}{c}{Unit} &
%      \multicolumn{1}{c}{Macro} &
%      \multicolumn{1}{c}{Symbol} &
%      \multicolumn{1}{c}{Unit} &
%      \multicolumn{1}{c}{Macro} &
%      \multicolumn{1}{c}{Symbol} \\
%    \midrule
%    \unitinfo{becquerel} &
%    \unitinfo{newton} \\
%    \unitinfo{celsius} &
%    \unitinfo{ohm} \\
%    \unitinfo{coulomb} &
%    \unitinfo{pascal} \\
%    \unitinfo{farad} &
%    \unitinfo{radian} \\
%    \unitinfo{Gray} &
%    \unitinfo{siemens} \\
%      & \csindex{ggray} & \ggray &
%    \unitinfo{sievert} \\
%    \unitinfo{hertz} &
%    \unitinfo{steradian} \\
%    \unitinfo{henry} &
%    \unitinfo{tesla} \\
%    \unitinfo{joule} &
%    \unitinfo{volt} \\
%    \unitinfo{katal} &
%    \unitinfo{watt} \\
%    \unitinfo{lumen} &
%    \unitinfo{weber} \\
%    \unitinfo{lux} \\
%    \bottomrule
%  \end{tabular}
%\end{table}
%
% In addition to these units, there are three other groups of units
% for use with the SI system which do not fit into the above.  These
% are those derived from physical measurements
% (\ref{tab:experimental}), those considered ``accepted''
% (\ref{tab:accepted}), and those accepted temporarily
% (\ref{tab:addn}).\footnote{These are supposed to be replaced over
% time by SI units.}
%\DescribeMacro{\litre}
%\DescribeMacro{\liter}
% The unit ``litre'' is often spelled ``liter'' in the US; both
% spellings are provided by \currpkg, with \cs{liter} giving
% \si{\liter} and \cs{litre} producing \si{\litre}.
%\begin{table}
%  \caption{Units derived from experiments
%    (\opt{load=physical})}
%  \label{tab:experimental}
%  \centering
%  \begin{tabular}{lll}
%    \toprule
%      \multicolumn{1}{c}{Unit} &
%      \multicolumn{1}{c}{Macro} &
%      \multicolumn{1}{c}{Symbol} \\
%    \midrule
%    electron volt & \csindex{electronvolt} & \electronvolt \\
%    unified atomic mass unit
%      & \csindex{atomicmassunit} & \atomicmassunit \\
%      & \csindex{atomicmass} & \atomicmass \\
%    \bottomrule
%  \end{tabular}
%\end{table}
%\begin{table}
%  \caption{Units accepted for use with SI
%    (\opt{load=accepted})}
%  \label{tab:accepted}
%  \centering
%  \begin{tabular}{lll}
%    \toprule
%      \multicolumn{1}{c}{Unit} &
%      \multicolumn{1}{c}{Macro} &
%      \multicolumn{1}{c}{Symbol} \\
%    \midrule
%    \unitinfo{bel} \\
%    day & \csindex{Day} & \Day \\
%     & \csindex{dday} & \dday \\
%    \unitinfo{degree} \\
%    \unitinfo{hour} \\
%    \unitinfo{litre} \\
%     & \csindex{liter} & \liter \\
%    \unitinfo{minute} \\
%    minute (arc) & \csindex{arcmin} & \arcmin \\
%    \unitinfo{neper} \\
%    \unitinfo{percent} \\
%    second (arc) & \csindex{arcsec} & \arcsec \\
%    \unitinfo{tonne} \\
%    \bottomrule
%  \end{tabular}
%\end{table}
%\begin{table}
%  \centering
%  \caption{Additional (temporary) SI units
%    (\opt{load=addn})}
%  \label{tab:addn}
%  \begin{tabular}{lll}
%    \toprule
%      \multicolumn{1}{c}{Unit} &
%      \multicolumn{1}{c}{Macro} &
%      \multicolumn{1}{c}{Symbol} \\
%    \midrule
%    {\aa}ngstr{\"o}m & \csindex{angstrom} & \angstrom \\
%    \unitinfo{are} \\
%    \unitinfo{curie} \\
%    bar & \csindex{BAR} & \BAR \\
%     & \csindex{bbar} & \bbar \\
%    \unitinfo{barn} \\
%    \unitinfo{gal} \\
%    \unitinfo{hectare} \\
%    \unitinfo{millibar} \\
%    \unitinfo{rad} \\
%    \unitinfo{rem} \\
%    \unitinfo{roentgen} \\
%    \bottomrule
%  \end{tabular}
%\end{table}
%
%\subsection{Prefixed and abbreviated units}
% Many basic units have prefixes which are commonly used with the
% unit, such as centimetre or megahertz.  The package therefore
% defines a number of common prefixed units (\opt{load=prefixed}).
% Several of these also have obvious abbreviations (such as \cs{MHz}
% for \cs{megahertz}), which are made available by \opt{load=abbr}.
% In common with the units discussed above, the prefixed and
% abbreviated unit definitions are loaded by default.
%\newcommand*{\tablesubhead}[1]{^^A
%  \addlinespace \emph{#1} \\*}
%\begin{longtable}{llll}
%    \caption{Prefixed (\opt{load=prefixed}) and abbreviated
%      (\opt{load=abbr}) units\label{tab:abbr}}\\
%    \toprule
%    Unit & Macro & Symbol & Abbreviation \\
%    \midrule
%  \endfirsthead
%    \toprule
%    Unit & Macro & Symbol & Abbreviation \\
%    \midrule
%  \endhead
%    \bottomrule
%    \multicolumn{4}{r}{\emph{Continued on next page}}
%  \endfoot
%    \bottomrule
%  \endlastfoot
%
%  \tablesubhead{Masses}
%  \unitinfo{kilogram} & \csindex{kg} \\
%  \unitinfo{femtogram} & \csindex{fg} \\
%  \unitinfo{picogram} & \csindex{pg} \\
%  \unitinfo{nanogram} & \csindex{nanog} \\
%  \unitinfo{microgram} & \csindex{micg} \\
%  \unitinfo{milligram} & \csindex{mg} \\
%  atomic mass & \csindex{atomicmass} & \amu & \csindex{amu} \\
%
%  \tablesubhead{Lengths}
%  \unitinfo{picometre} & \csindex{picom} \\
%  \unitinfo{nanometre} & \csindex{nm} \\
%  \unitinfo{micrometre} & \csindex{micm} \\
%  \unitinfo{millimetre} & \csindex{mm} \\
%  \unitinfo{centimetre} & \csindex{cm} \\
%  \unitinfo{decimetre} & \csindex{dm} \\
%  \unitinfo{kilometre} & \csindex{km} \\
%
%  \tablesubhead{Times}
%  \unitinfo{second} & \csindex{Sec} \\
%  \unitinfo{attosecond} & \csindex{as} \\
%  \unitinfo{femtosecond} & \csindex{fs} \\
%  \unitinfo{picosecond} & \csindex{ps} \\
%  \unitinfo{nanosecond} & \csindex{ns} \\
%  \unitinfo{microsecond} & \csindex{mics} \\
%  \unitinfo{millisecond} & \csindex{ms} \\
%
%  \tablesubhead{Moles}
%  \unitinfo{femtomole} & \csindex{fmol} \\
%  \unitinfo{picomole} & \csindex{pmol} \\
%  \unitinfo{nanomole} & \csindex{nmol} \\
%  \unitinfo{micromole} & \csindex{micmol} \\
%  \unitinfo{millimole} & \csindex{mmol} \\
%  \unitinfo{kilomole} & \csindex{kmol} \\
%
%  \tablesubhead{Currents}
%  \unitinfo{picoampere} & \csindex{pA} \\
%  \unitinfo{nanoampere} & \csindex{nA} \\
%  \unitinfo{microampere} & \csindex{micA} \\
%  \unitinfo{milliampere} & \csindex{mA} \\
%  \unitinfo{kiloampere} & \csindex{kA} \\
%
%  \tablesubhead{Areas}
%  square centimetre & \csindex{squarecentimetre} &
%    \squarecentimetre & \csindex{cms} \\*
%    & \csindex{centimetresquared} & \centimetresquared \\
%  square metre & \csindex{squaremetre} & \squaremetre \\
%  square kilometre & \csindex{squarekilometre} & \squarekilometre \\
%
%  \tablesubhead{Volumes}
%\changes{v1.1d}{2008/10/29}{New units \cs{microliter}, \cs{milliliter},
%  \cs{micL} and \cs{mL}}
%  \unitinfo{microlitre} & \csindex{micl} \\
%  \unitinfo{millilitre} & \csindex{ml} \\
%  \unitinfo{microliter} & \csindex{micL} \\
%  \unitinfo{milliliter} & \csindex{mL} \\
%  cubic centimetre & \csindex{cubiccentimetre} &
%    \cubiccentimetre & \csindex{cmc} \\*
%    & \csindex{centimetrecubed} & \centimetrecubed \\
%  cubic decimetre & \csindex{cubicdecimetre} &
%    \cubicdecimetre & \csindex{dmc} \\
%
%  \tablesubhead{Frequencies}
%  \unitinfo{hertz} & \csindex{Hz} \\
%  \unitinfo{millihertz} & \csindex{mHz} \\
%  \unitinfo{kilohertz} & \csindex{kHz} \\
%  \unitinfo{megahertz} & \csindex{MHz} \\
%  \unitinfo{gigahertz} & \csindex{GHz} \\
%  \unitinfo{terahertz} & \csindex{THz} \\
%
%  \tablesubhead{Potentials}
%  \unitinfo{millivolt} & \csindex{mV} \\
%  \unitinfo{kilovolt} & \csindex{kV} \\
%
%  \tablesubhead{Energies}
%  \unitinfo{kilojoule}  & \csindex{kJ}\\
%  \unitinfo{electronvolt} & \csindex{eV} \\
%  \unitinfo{millielectronvolt} & \csindex{meV} \\
%  \unitinfo{kiloelectronvolt} & \csindex{keV} \\
%  \unitinfo{megaelectronvolt} & \csindex{MeV} \\
%  \unitinfo{gigaelectronvolt} & \csindex{GeV} \\
%  \unitinfo{teraelectronvolt} & \csindex{TeV} \\
%  \unitinfo{kilowatthour} & \cs{kWh} \\
%
%  \tablesubhead{Powers}
%  \unitinfo{milliwatt} \\
%  \unitinfo{kilowatt} \\
%  \unitinfo{megawatt} \\
%
%  \tablesubhead{Capacitances}
%  \unitinfo{femtofarad} \\
%  \unitinfo{picofarad} \\
%  \unitinfo{nanofarad} \\
%  \unitinfo{microfarad} \\
%  \unitinfo{millifarad} \\
%
%  \tablesubhead{Resistances}
%  \unitinfo{kilohm} \\
%  \unitinfo{megohm} \\
%  \unitinfo{gigaohm} \\
%  \unitinfo{millisiemens} \\
%
%  \tablesubhead{Forces}
%  \unitinfo{millinewton} \\
%  \unitinfo{kilonewton} \\
%
%  \tablesubhead{Other units}
%  \unitinfo{hectopascal} \\
%  \unitinfo{megabecquerel} \\
%  \unitinfo{millisievert} \\
%\end{longtable}
%
%\subsection{Defining new units}
%\DescribeMacro{\newunit}
%\DescribeMacro{\renewunit}
%\DescribeMacro{\provideunit}
% New units are produced using the \cs{newunit} macro.  This works as
% might be expected:
% \cs{newunit}\oarg{options}\marg{unit}\marg{symbol}, where
% \meta{symbol} can contain literal values, other units, multiple
% prefixes, powers and \cs{per}.  The \meta{options} argument can be
% any suitable options, and applies the specific unit macro only. The
% most obvious example for using this macro is the \cs{degree}
% unit.\footnote{Although the \cs{ang} macro is preferred for this
% job.}  The (first) optional argument to \cs{SI} and \cs{si} can be
% used to override the settings for the unit.  The \cs{renewunit} and
% \cs{provideunit} macros take the same arguments.
%\begin{LaTeXexample}
%  \SI{3.1415}{\degree}\\
%  \newunit[valuesep=none]{\oddunit}{XXX}
%  \SI{12345}{\oddunit}
%  \SI[valuesep=thick]{67890}{\oddunit}
%\end{LaTeXexample}
%
% As with the \LaTeX\ commands \cs{newcommand}, \etc, the choice of
% \cs{newunit}, \cs{renewunit} or \cs{provideunit} depends on the
% presence of an existing definition.  While \cs{newunit} should be
% used when a unit has not been previously defined, \cs{renewunit}
% will issue a warning if the named unit does not already exist.
% \cs{provideunit} defines the unit if it does not exist, and
% otherwise does nothing at all.  The same behaviour is seen with
% \cs{providepower} and \cs{provideprefix} (\latin{vide infra}).
%
% Output that is only valid in maths mode requires \cs{ensuremath},
% text-only input requires \cs{text}.  In the example below,
% \cs{mathnormal} is used to force the font choice only for the
% single character.\footnote{The \cs{mathrm} font used for this
% document has an ``\ss'' at the $\pi$ position.}
%\begin{LaTeXexample}
%  \newunit{\SIpi}{\ensuremath{\mathnormal{\pi}}}
%  \SI{10}{\metre\per\SIpi\squared}
%\end{LaTeXexample}
%
%\DescribeMacro{\newpower}
%\DescribeMacro{\renewpower}
%\DescribeMacro{\providepower}
% Powers are defined:
% \cs{newpower}\ofixarg{post}\marg{power}\marg{num}. Here,
% \meta{power} is the name of the power macro and \meta{num} is the
% (positive) number it represents.  The later argument is always
% processed internally by \cs{num}, but \emph{must} be a number.
% Giving the optional argument \opt{post} indicates to the package
% that the power will come after the unit it applies to; by default
% it is assumed that it will come before.
%\begin{LaTeXexample}
%  \newpower{\quartic}{4}
%  \newpower[post]{\totheforth}{4}\\
%  \si{\kilogram\totheforth}\\
%  \si{\quartic\metre}
%\end{LaTeXexample}
%
%\DescribeMacro{\newprefix}
%\DescribeMacro{\renewprefix}
%\DescribeMacro{\provideprefix}
% The standard SI powers of ten are defined by the package, and are
% described above.  However, the user can define new prefixes with
% \cs{newprefix}.  This has syntax
% \cs{newunit}\ofixarg{binary}\marg{prefix}\marg{symbol}\marg{powers-ten},
% where \meta{powers-ten} is the number of powers of ten the prefix
% represents.  When the \opt{binary} option is given, the prefix is a
% power of two.  For example, \cs{kilo} and \cs{kibi} are defined:
%\begin{LaTeXexample}[noexample]
%  \newprefix{\kilo}{k}{3}
%  \newprefix[binary]{\kibi}{Ki}{10}
%\end{LaTeXexample}
%
%\changes{v1.1}{2008/09/22}{Qualifiers introduced}
%\DescribeMacro{\newqualifier}
%\DescribeMacro{\renewqualifier}
%\DescribeMacro{\providequalifier}
% It is possible to create unit ``qualifiers'', which add subscript
% descriptions to units.  Although not encouraged, this is sometimes
% necessary to make the meaning clear.  No qualifiers are pre-defined
% by \currpkg, and so the user needs to declare them using
% \cs{newqualifier}. Currently, only subscript qualifiers can be
% created; a more extended set of options is planned for the next
% release of \currpkg.
%\begin{LaTeXexample}
%  \newqualifier{\polymer}{pol}
%  \newqualifier{\catalyst}{cat}
%  \si{\kg\polymer\per\mole\catalyst\per\hour}
%\end{LaTeXexample}
%
%\section{Specialist units}
% In some subject area, there are units which are in common use even
% though they are outside of the SI system.  Unlike the units
% discussed earlier, these specialist units are not loaded by
% default.  In each case, they should be requested with the option
% \opt{alsoload=\meta{name}}.
%
%\subsection{Binary units (\opt{binary})\label{sec:binary}}
%\DescribeMacro{\bit}
%\DescribeMacro{\byte}
% The binary prefixes, \cs{bit}, \cs{byte} (\ref{tab:binary}) are not
% formally part of the SI system. They are available by giving the
% \opt{alsoload=binary} option.
%\begin{LaTeXexample}
%  \SI{100}{\mebi\byte}
%\end{LaTeXexample}
%\begin{table}
%  \caption{Binary prefixes
%    (\opt{alsoload=binary})}
%  \label{tab:binary}
%  \centering
%  \begin{tabular}{lllc}
%    \toprule
%      \multicolumn{1}{c}{Prefix} &
%      \multicolumn{1}{c}{Macro} &
%      \multicolumn{1}{c}{Power} &
%      Symbol \\
%    \midrule
%      \prefixinfo{kibi} \\
%      \prefixinfo{mebi} \\
%      \prefixinfo{gibi} \\
%      \prefixinfo{tebi} \\
%      \prefixinfo{pebi} \\
%      \prefixinfo{exbi} \\
%    \bottomrule
%  \end{tabular}
%\end{table}
%
%\subsection{Synthetic chemistry (\opt{synchem})}
%\DescribeMacro{\mmHg}
%\DescribeMacro{\molar}
%\DescribeMacro{\Molar}
%\DescribeMacro{\torr}
%\DescribeMacro{\dalton}
% The \opt{synchem} file adds the common chemistry units \cs{mmHg},
% \cs{molar}, \cs{Molar}, \cs{torr} and \cs{dalton} to \currpkg.  The
% \cs{Molar} macro is somewhat awkward, as it can be given as either
% ``\textsc{m}'' or ``M''.  The later is obviously easily confused
% with the sign for the prefix mega.  By default, \currpkg uses the
% UK default of a small-caps symbol.  The \cs{dalton} unit is defined
% here as this name is not recognised by the various international
% bodies: the symbol \si{\amu} is preferred.
%\begin{LaTeXexample}
%  \SI{1}{\Molar} HCl\\
%  \SI{760}{\torr}\\
%  \SI{0.01}{\mmHg}\\
%  \SI{3.0}{\molar}\\
%  \SI{106.42}{\dalton}
%\end{LaTeXexample}
%
%\subsection{High-energy physics (\opt{hep})}
% In contrast to \pkg{hepunits}, \currpkg does not define a long list
% of compound units for high-energy physics.\footnote{Using the
% \opt{emulate=hepunits} option will load a file defining those.}
% Instead, a small selection of new units are defined
% (\ref{tab:hep}).  The mechanisms provided by \currpkg should avoid
% the need for large numbers of abbreviations.  For example, the
% \pkg{hepunits} \cs{MinveV} can be given as |\per\MeV| in \currpkg,
% which requires only one more character.
%\begin{table}
%  \caption{High-energy physics units (\opt{alsoload=hep})}
%  \label{tab:hep}
%  \centering
%  \begin{tabular}{lllc}
%    \toprule
%    Unit & Macro & Symbol & Abbreviation \\
%    \midrule
%    \addlinespace \emph{Areas} \\
%    \unitinfo{yoctobarn} & \csindex{yb} \\
%    \unitinfo{zeptobarn} & \csindex{zb} \\
%    \unitinfo{attobarn} & \csindex{ab} \\
%    \unitinfo{femtobarn} & \csindex{fb} \\
%    \unitinfo{picobarn} & \csindex{pb} \\
%    \unitinfo{nanobarn} & \csindex{nb} \\
%    \addlinespace \emph{Other units} \\
%    \unitinfo{micron} \\
%    millirad & \csindex{mrad} & \mrad \\
%    \unitinfo{gauss} \\
%    \bottomrule
%  \end{tabular}
%\end{table}
%
%\DescribeMacro{\clight}
%\DescribeMacro{\eVperc}
% The \opt{hep} option defines two units which are slightly unusual.
% \cs{clight} gives \clight, which is recognised as a unit when used
% in the appropriate circumstances. The second unit provided is
% \cs{eVperc}, which is commonly-used and clear enough for a compound
% definition. Notice that the value of \opt{eVcorrb} will need to be
% adjusted when using this unit.
%\begin{LaTeXexample}
%  \SI[per=slash,eVcorrb=0.4ex]{4.657}{\mega\eVperc\squared}
%\end{LaTeXexample}
%
%\subsection{Astronomy (\opt{astro})}
%\DescribeMacro{\parsec}
%\DescribeMacro{\lightyear}
% For astronomers, the \cs{parsec} and \cs{lightyear} units are
% available, and give the obvious results.
%\begin{LaTeXexample}
%  \SI{12}{\parsec}\\
%  \SI{1}{\lightyear}
%\end{LaTeXexample}
%
%
%\subsection{Geophysics (\opt{geophys})}
%\changes{v1.1j}{2008/12/22}{New module \opt{geophys} containing \cs{gon}
%  unit}
%\DescribeMacro{\gon}
% For geophysics, the unit \cs{gon} is available.
%\begin{LaTeXexample}
%  \SI{12}{\gon}
%\end{LaTeXexample}
%
%\section{Font control}
% Following the lead of \SIstyle, \currpkg provides control over the
% font used to typeset output.  By default, all text is typeset using
% the current upright serif maths font, whether the macros are given
% in text or maths mode.  Some examples will show the effect.
%\begin{LaTeXexample}
%  \num{10} $\num{10}$\\
%  \sffamily \ang{20} $\ang{20}$\\
%  \textbf{\SI{30}{\kilo\gram}}\\
%  \boldmath $\SI{40}{\kilo\gram}$
%  \[ \num{50} \]
%\end{LaTeXexample}
%
% In contrast, by setting \opt{obeyall}, the current font is used:
% this may be maths or text, depending on the context.
%\begin{LaTeXexample}
%  \sisetup{obeyall}\\
%  \ang{1;1;1} $\ang{1;1;1}$\\
%  \sffamily \ang{2;2;2} $\ang{2;2;2}$\\
%  \textbf{\ang{3;3;3}} \boldmath $\ang{3;3;3}$\\
%  \emph{\ang{4;4;4}} \emph{$\ang{4;4;4}$}
%  \[ \ang{5;5;5} \]
%\end{LaTeXexample}
%
% Fine control of which elements of the local font are used is
% available with the \opt{obeyfamily}, \opt{obeybold},
% \opt{obeyitalic} and \opt{obeymode} options.
%\begin{LaTeXexample}
%  \sisetup{obeyfamily}\\
%  \ang{1;1;1} $\ang{1;1;1}$\\
%  \sffamily \ang{2;2;2} $\ang{2;2;2}$\\
%  \sisetup{obeybold}
%  \textbf{\ang{3;3;3}} \boldmath $\ang{3;3;3}$\\
%  \emph{\ang{4;4;4}} \emph{$\ang{4;4;4}$}
%  \[ \ang{5;5;5} \]
%\end{LaTeXexample}
%
%\section{Package options\label{sec:options}}
%\DescribeMacro{\sisetup}
% The ``native'' options for the package are all given using the
% key--value method. Most of the package options can be given both
% when loading the package and at any point in the document.  This is
% achieved using the \cs{sisetup} macro.
%
% The package options take a number of different forms.
%\begin{itemize}
%  \item \opt{option=\meta{bool}} Simple true/false values.  These
%    macros all default to \opt{option=true}, meaning that giving the
%    option name along will set the appropriate flag.
%  \item \opt{option=\meta{choice}} Take a single item from a
%    pre-determined list. Depending on the value, one or more
%    internal states will be altered.  Values not on the list are
%    ignored (with a warning).
%  \item \opt{option=\meta{choice,literal}} If the given value is a
%    \meta{choice}, then the internal settings for that choice are
%    used.  Any other value is used directly.
%  \item \opt{option=\meta{literal}} The given value is used as a
%    literal by the package.
%  \item \opt{option=\meta{csname}} These options expect a command
%    sequence as a value.
%  \item \opt{option=\meta{length}} Requires a \TeX\ length, for
%    example \opt{0.5ex}.
%  \item \opt{option=\meta{list}} Takes a list of one or more items,
%    which are not determined in advance.
%  \item \opt{option=\meta{number}} Takes a number (possibly
%    including an exponent part).
%\end{itemize}
%
% The package has a large range of options, to allow full control of
% the various features of the package.  These control differing
% aspects of the package, and are given below in groups based on
% function.  Where the key has a default value, it is given in bold.
%
%\subsection{Font family and style}
% The font used when typesetting material can be tightly controlled
% using \currpkg.  A number of options affect how the package matches
% the surrounding font, and the font families used to achieve this.
% The default is to use the current upright maths serif font with no
% variation.
%
%\DescribeOption{mode}
%\DescribeOption{textmode}
%\DescribeOption{valuemode}
%\DescribeOption{unitmode}
%\DescribeOption{obeymode}
% The output of \currpkg can occur using either text or maths mode.
% The package option \opt{mode} determines which is used: valid
% options are \defaultopt{maths} and \opt{text}.\footnote{Here and in
% all other cases, either UK or US spelling may be used.  Thus
% \opt{mode=maths} or \opt{mode=math} have exactly the same effect.}
% The shortcut \opt{textmode} is provided for setting \opt{mode=text}
% quickly. Further refinement is possible using the \opt{valuemode}
% and \opt{unitmode} options.  These apply to numbers (the output of
% \cs{num} and the first mandatory argument of \cs{SI}) and units
% (all other output), respectively. By setting the \opt{obeymode}
% flag, the package will use the local typesetting mode (maths or
% text).
%
%\DescribeOption{obeyall}
%\DescribeOption{obeyfamily}
%\DescribeOption{obeybold}
%\DescribeOption{obeyitalic}
% The detection and matching of surrounding text can be controlled
% using a number of Boolean package options. \opt{obeyall} turns on
% all of the detection.  Thus output with \opt{obeyall} in force will
% always match the local text appearance.  \opt{obeyfamily} instructs
% the package on detecting the surrounding font family (Roman, sans
% serif, fixed width), but does not detect bold or italic.
% \opt{obeybold} detects the local bold setting, whilst
% \opt{obeyitalic} picks up italic fonts.
%
% Bold detection is influenced by the value of \opt{inlinebold},
% which takes values \defaultopt{text} and \opt{maths}.  The package
% can detect the local value of bold for either the surrounding text,
% or the surrounding inline (|$|\ldots|$|) maths.  The
% \opt{obeyitalic} option does \emph{not} have the same facility
% (maths is italic anyway).
%
%\DescribeOption{mathsrm}
%\DescribeOption{mathssf}
%\DescribeOption{mathstt}
%\SpecialOptionIndex{mathrm}
%\SpecialOptionIndex{mathsf}
%\SpecialOptionIndex{mathtt}
%\DescribeOption{textrm}
%\DescribeOption{textsf}
%\DescribeOption{texttt}
% The font commands used by the package to achieve the above are all
% available for user modification.  The options \opt{mathsrm},
% \opt{mathssf} and \opt{mathstt} hold the command sequences used in
% maths mode,\footnote{These can also be set using \opt{mathrm},
% \opt{mathsf} and \opt{mathtt}} while \opt{textrm}, \opt{textsf} and
% \opt{texttt} do the same for text mode.  By default, these contain
% the obvious command names, for example \opt{mathsrm=mathrm} and
% \opt{texttt=ttfamily}.  However, they can be set at will: the macro
% names indicate the nature of the surrounding text detected.  For
% example, the value of \opt{mathssf} is used in maths mode when the
% surrounding text is sans serif.
%
%\SpecialOptionIndex{valuemathsrm}
%\SpecialOptionIndex{valuemathssf}
%\SpecialOptionIndex{valuemathstt}
%\SpecialOptionIndex{valuemathrm}
%\SpecialOptionIndex{valuemathsf}
%\SpecialOptionIndex{valuemathtt}
%\SpecialOptionIndex{unitmathsrm}
%\SpecialOptionIndex{unitmathssf}
%\SpecialOptionIndex{unitmathstt}
%\SpecialOptionIndex{valuetextrm}
%\SpecialOptionIndex{valuetextsf}
%\SpecialOptionIndex{valuetexttt}
%\SpecialOptionIndex{valuetextrm}
%\SpecialOptionIndex{valuetextsf}
%\SpecialOptionIndex{valuetexttt}
%\SpecialOptionIndex{unittextrm}
%\SpecialOptionIndex{unittextsf}
%\SpecialOptionIndex{unittexttt}
% Each of the font options can be given separately for the contents
% of numbers and units.  The option names include \opt{value} or
% \opt{unit} before the mode name.  For example, the \opt{mathsrm}
% option may be split into \opt{valuemathsrm} and \opt{unitmathsrm}.
%
%\DescribeOption{detectdisplay}
% The font detection system can treat displayed mathematical content
% in two ways.  This is controlled by the \opt{detectdisplay} option.
% When set to \defaultopt{true}, display mathematics is treated
% independently from the body of the document.  Thus the local
% \emph{maths} font is checked for matching.  In contrast, when set
% to \opt{false}, display material is treated with the current
% running text font.
%\begin{LaTeXexample}
%  \sffamily
%  Some text
%  \sisetup{obeyall}
%  \[ x = \SI{1.2e3}{\kg\kelvin\candela} \]
%  More text
%  \sisetup{detectdisplay=false}
%  \[ y = \SI{3}{\metre\second\mole} \]
%\end{LaTeXexample}
%
%\subsection{Spacing and separators}
%\DescribeOption{unitsep}
%\DescribeOption{valuesep}
%\DescribeOption{digitsep}
%\DescribeOption{anglesep}
% The separators between items can all be set using options taking a
% list of pre-defined items or a literal value.  The ``\opt{sep}''
% options (\opt{unitsep}, \opt{valuesep}, \opt{digitsep} and
% \opt{anglesep}) all recognise \opt{thin}, \opt{medium}, \opt{med},
% \opt{thick}, \opt{space}, \opt{cdot}, \opt{times}, \opt{tightcdot},
% \opt{tighttimes}, \opt{fullstop}, \opt{stop}, \opt{period} and
% \opt{none}.  The named spaces are the normal maths separations,
% with \opt{space} representing a full (non-breakable text) space,
% and with the obvious meanings for \opt{cdot} and \opt{times}.  The
% \opt{tight} variants reduce the spacing available. Three possible
% values are provided for ``|.|'', and \opt{none} yields no space at
% all. In all cases, other values are treated literally and are
% typeset in maths mode.  The default value is \defaultopt{thin} for
% all separations except \opt{anglesep}, which is set to
% \defaultopt{none}.
%
%\DescribeOption{unitspace}
%\DescribeOption{errspace}
% The \opt{unitspace} and \opt{errspace} options again take a list
% or literal value, but only the ``real'' spaces \opt{thin},
% \opt{medium}, \opt{med}, \opt{thick}, \opt{space} and \opt{none}
% are recognised in the list. The \opt{unitspace} option controls the
% output generated by an explicit space (|~|) inside a unit macro,
% while \opt{errspace} is used to separate a bracketed error from the
% main number.
%
%\subsection{Number formatting}
%\DescribeOption{numdigits}
%\DescribeOption{numdecimal}
%\DescribeOption{numdiv}
%\DescribeOption{numexp}
%\DescribeOption{numsign}
%\DescribeOption{numaddn}
%\DescribeOption{numgobble}
% There are two groups of options for formatting numbers.  The first
% group all begin with ``\opt{num}'', and take literal values used by
% the package to parse numbers.  \opt{numdigits} contains the valid
% number symbols (\defaultopt{0123456789}), with \opt{numdecimal}
% containing the decimal markers (\defaultopt{.,}).  As in the
% \numprint package, \opt{numexp} (the list of exponent markers)
% recognises \defaultopt{deDE} as valid by default.  \opt{numsign}
% contains the sign markers for numbers
% (\defaultopt{+-\cs{pm}\cs{mp}}).  \opt{numdiv} holds the division
% marker in numbers (\defaultopt{/}). \opt{numaddn} and \opt{numgobble}
% both control which other characters do not give an error when present
% in a number.  \opt{numaddn} contains valid characters which should be
% included in the final output ``as is'', whereas \opt{numgobble} lists
% the characters that are completely ignored.  In all cases, the
% content of the options is a simple string, for example
% \opt{numdigits=1234567890}.
%
%\DescribeOption{decimalsymbol}
%\DescribeOption{digitsep}
%\DescribeOption{sepfour}
% The second group of number options control the output of numbers
% after parsing. The symbol used by \currpkg as a decimal marker is
% set by the \opt{decimalsymbol} option, which can take a list of
% choices or a literal.  The valid choices here are
% \defaultopt{fullstop}, \opt{comma}, \opt{cdot} and
% \opt{tightcdot}.\footnote{\opt{fullstop} also has aliases
% \opt{stop} and \opt{period}.}  Notice that this does not have to
% agree with the input marker.  The other separator for numerical
% output is the division of digits into groups of three.  The result
% is dependent on two options.  The previously-described
% \opt{digitsep} option controls the spacing added between groups of
% three numbers.  For numbers consisting of exactly four digits, the
% \opt{sepfour} Boolean option controls whether separation occurs in
% these cases.  The default is \defaultopt{false}.
%\begin{LaTeXexample}
%  \num{1234}\\
%  \num[sepfour]{1234}
%\end{LaTeXexample}
%
%\DescribeOption{seperr}
%\DescribeOption{numopenerr}
%\DescribeOption{numcloseerr}
% For numbers given with an error [\eg \num{1.23(4)}], the package
% can separate out the error part, to give for example
% \num[seperr]{1.23(4)}.  This behaviour is activated by the
% \opt{seperr} option, and requires that \opt{numopenerr} and
% \opt{numcloseerr} contain the left- and right-hand delimiters for
% the error [defaults \opt{numopenerr=(} and \opt{numcloseerr=)}].
%\begin{LaTeXexample}
%  \sisetup{seperr}\\
%  \num{1.234(5)}\\
%  \num{1.234(5)e6}
%\end{LaTeXexample}
%\DescribeOption{trapambigerr}
%\DescribeOption{openerr}
%\DescribeOption{closeerr}
%\DescribeOption{tightpm}
% If the number has an exponent, or if units are not repeated, then
% the result can be considered ambiguous.  By default, the package
% adds the markers stored in \opt{openerr} and \opt{closeerr} to
% remove the ambiguity; the options have the same default values as
% the input error markers.  Detection of a potentially-ambiguous
% error is controlled by the \opt{trapambigerr} option, although for
% numbers with units the \opt{repeatunits} option is also important.
% The spacing around the $\pm$ sign is normally set by \TeX. However,
% using the \opt{tightpm} option will cause this to be reduced to a
% minimum.
%\begin{LaTeXexample}
%  \sisetup{seperr}\\
%  \num[trapambigerr=false]{1.234(5)e6}\\
%  \SI{1.234(5)}{\metre}\\
%  \sisetup{repeatunits=false}
%  \SI{1.234(5)}{\metre}\\
%  \SI[trapambigerr=false]{1.234(5)}{\metre}\\
%  \num[tightpm]{1.234(5)}
%\end{LaTeXexample}
%
%\DescribeOption{numprod}
%\DescribeOption{repeatunits}
% The number processor can recognise products in the numerical input;
% the symbols used for products are stored in \opt{numprod}, with the
% default value of ``\opt{x}''.  By default, the \cs{SI} macro will
% repeat the units for a number given in this way.  This behaviour is
% altered  by the \opt{repeatunits} option, which takes the values
% \defaultopt{true}, \opt{false} and \opt{power}.  The later applies
% only when providing multiplied numbers with units, and converts the
% unit to the appropriate power rather than repeating the
% units.\footnote{This is a very simply option: do not expect it to
% work with anything except areas and volumes.} Notice that when
% applied to errors, \opt{repeatunits} takes priority over
% \opt{trapambigerr}.
%\begin{LaTeXexample}
%  \num{1 x 2 x 3}\\
%  \SI{4 x 5 x 6}{\metre}\\
%  \sisetup{repeatunits=false}
%  \SI{1.2 x 3.4 x 5.6}{\milli\metre\cubed}\\
%  \SI[seperr]{1.234(5)}{\metre}\\
%  \SI[seperr,
%    trapambigerr=false]
%    {9.1093897(54)e-31}{\kilo\gram}\\
%  \SI[seperr,repeatunits,
%    trapambigerr=false]
%    {9.1093897(54)e-31}{\kilo\gram}\\
%  \SI[repeatunits=power]{1 x 2 x 3}{\metre}
%\end{LaTeXexample}
%
%\DescribeOption{expproduct}
%\DescribeOption{expbase}
%\DescribeOption{allowzeroexp}
% The formatting of exponents is controlled by \opt{expproduct} and
% \opt{expbase}.  \opt{expproduct} sets the symbol used to indicate a
% product for exponents (\eg the $\times$ in \num{2e2}), while the
% value of \opt{expbase} sets the power used (the \num{10} in the
% example).  Both options accept a very short list of options:
% \defaultopt{times}, \opt{tighttimes}, \opt{cdot} and
% \opt{tightcdot} for the product, and \defaultopt{ten} and \opt{two}
% for the power.\footnote{The \opt{tighttimes} and \opt{tightcdot}
% options give the rather questionable results:
% \num[expproduct=tighttimes]{1e2} and
% \num[expproduct=tightcdot]{1e2}, as opposed to \num{1e2} and
% \num[expproduct=cdot]{1e2}.} Other choices are used literally. Also
% relevant to exponent processing is the \opt{allowzeroexp} option.
% By default, the package will suppress a zero exponent, but setting
% the flag will allow the output of \num[allowzeroexp]{e0}.
%
%\DescribeOption{addsign}
%\DescribeOption{sign}
%\DescribeOption{retainplus}
% Additions to the input can take the form of implicit signs and
% padded zeros.  The \opt{addsign} option takes a list of potential
% sites to add a sign: \defaultopt{none}, \opt{mantissa},
% \opt{exponent} and \opt{both}.\footnote{Aliases are provided:
% \opt{mant} = \opt{mantissa}, \opt{exp} = \opt{exponent}, \opt{all}
% = \opt{true} = \opt{both}, \opt{false} = \opt{none}.}  If no sign
% is given in the input, the setting here determines if one is added.
% The sign to add is stored in \opt{sign}, which takes the list of
% choices \defaultopt{plus}, \opt{minus}, \opt{pm} and \opt{mp}, or
% uses the input literally (in maths mode).  For positive numbers,
% the \opt{retainplus} option causes a $+$ sign explicitly in the
% input to be retained.  By default, the package will remove such
% signs.
%
%\DescribeOption{padnumber}
% The \opt{padnumber} option controls the addition of zeros to the
% input, to ``pad'' the result.  The option takes a list of choices:
% \defaultopt{leading}, \opt{trailing}, \opt{both} and
% \opt{none}.\footnote{Aliases: \opt{all} = \opt{true} = \opt{both},
% \opt{false} = \opt{none}.}  No additional precision is added by
% this option; integer input will not add a decimal point.
%\begin{LaTeXexample}
%  \num[padnumber=leading]{.1}\\
%  \num[padnumber=leading]{2.}\\
%  \num[padnumber=trailing]{3.}\\
%  \num[padnumber=both]{4.}\\
%  \num[padnumber=both]{.5}\\
%  \num[padnumber=both]{6}\\
%  \num[padnumber=none]{7.}\\
%  \num[padnumber=none]{.8}
%\end{LaTeXexample}
%
%\DescribeOption{fixdp}
%\DescribeOption{dp}
% In contrast to the \opt{padnumber} option, the package can alter
% the precision of the input number if the \opt{fixdp} option is set.
% The \opt{fixdp} option will fix the decimal places of the output to
% the number stored in the \opt{dp} option.  The later should be a 
% positive integer or zero.
%\begin{LaTeXexample}
%  \sisetup{fixdp,dp=3}
%  \num{1}\\
%  \num{53.9}\\
%  \num{4.56783}\\
%  \num{-1.2942}\\
%  \num{-1.2959}\\
%  \num{9.9999}\\
%  \num{2.1264e90}\\
%  \num[dp=0]{12.345}
%\end{LaTeXexample}
%
%\DescribeOption{fixsf}
%\changes{v1.1}{2008/08/22}{New \opt{fixsf} option}
%\DescribeOption{sf}
%\changes{v1.1}{2008/08/22}{New \opt{sf} option}
% In contrast to the \opt{dp} option, the \opt{fixsf} and \opt{sf}
% options control the number of significant figures in the output.
% This never adds extra digits to the output, as significant figures
% are concerned with accuracy.
%\begin{LaTeXexample}
%  \sisetup{fixsf,sf=3}
%  \num{1}\\
%  \num{53.9}\\
%  \num{4.56783}\\
%  \num{-1.2942}\\
%  \num{-1.2959}\\
%  \num{9.9999}\\
%  \num{2.1264e90}\\
%  \num{773322}
%\end{LaTeXexample}
%
%\subsection{Angle formatting}
%\DescribeOption{padangle}
%\DescribeOption{strictarc}
% The angle formatter uses \cs{num} to format numbers: any options
% for numbers are therefore applicable here.  The \opt{padangle}
% option takes choices \defaultopt{small}, \opt{large}, \opt{all} and
% \opt{none}, and controls how angles are padded when given in
% degrees, minutes and seconds.\footnote{Aliases: \opt{all} =
% \opt{true} = \opt{both}, \opt{false} = \opt{none}.}  When giving
% angles as arcs (in degrees, minutes and seconds), the package can
% detect if the correct number of semi-colons have been given. This
% is controlled by the \opt{strictarc} option, which is a Boolean
% switch with a \defaultopt{true} default.  With \opt{strictarc} set
% to \opt{false}, an incomplete arc is interpreted as degrees and
% minute, while an over-complete one will drop excess input.
%\begin{LaTeXexample}
%  \ang[padangle=none]{1;;}\\
%  \ang[padangle=large]{2;;}\\
%  \ang[padangle=small]{3;;}\\
%  \ang[padangle=both]{;4;}\\
%  \ang[padangle=none]{;5;5}\\
%  \ang[strictarc=false]{1;2}\\
%  \ang[strictarc=false]{1;2;3;4;}
%\end{LaTeXexample}
%
%\DescribeOption{angformat}
% The angle formatting system can convert between decimal angles and
% those given as degrees, minutes and seconds.  This is controlled by
% the \opt{angformat} option, which takes choices
% \defaultopt{unchanged}, \opt{decimal} and
% \opt{arc}.\footnote{Aliases: \opt{decimal} = \opt{dec}, \opt{arc} =
% \opt{dms}, \opt{unchanged} = \opt{none}.}  When set to
% \opt{unchanged}, nothing is done to the input.  The conversion is
% based on \TeX\ dimensions, and is therefore limited in accuracy.
% For this reason, the output is automatically rounded: output as a
% decimal angle is limited to three places, and that as an arc is
% given to a single decimal place for the seconds component.
%\begin{LaTeXexample}
%  $\ang{1;2;3} = \ang[angformat=dec]{1;2;3}$\\
%  $\ang{4.56} = \ang[angformat=arc]{4.56}$
%\end{LaTeXexample}
%
%\DescribeOption{astroang}
% For astronomers, the \opt{astroang} option is provided.  This
% moves the degrees, minutes or seconds symbol (as appropriate)
% over the decimal marker rather than after the number.
%\begin{LaTeXexample}
%  \ang{1;2;3.4}\\
%  \ang[astroang]{5;6;7.8}
%\end{LaTeXexample}
%
%\subsection{Tabular material}
%\DescribeOption{tabnumalign}
% Material typeset in |S| columns is processed internally by the
% \cs{num} macro.  Thus, as with angles, the number options also
% apply here.  The positioning of tabular material is controlled by
% the two options \opt{tabnumalign} and \opt{tabformat}.
% \opt{tabnumalign} takes values \defaultopt{centredecimal},
% \opt{centre}, \opt{left} and \opt{right}.\footnote{Aliases
% \opt{centerdecimal} = \opt{centredecimal}; \opt{center} =
% \opt{centre}.}  When using \opt{centredecimal}, the package places
% the decimal marker of the mantissa at the centre of the column,
% which then grows to accommodate the widest number given.  For equal
% numbers of digits before and after the decimal sign, this is the
% easiest option.  The other choices use a fixed-width box to store
% the number; the box is then aligned with the edges of the column.
%
%\DescribeOption{tabformat}
% The \opt{tabformat} option sets the amount of space reserved by
% \currpkg for the alignment box when not using the
% \opt{centredecimal} setting of \opt{tabnumalign}.  The numerical
% parts of \opt{tabformat} are interpreted as
% \texttt{\meta{pre}\meta{dec}\meta{post}}; \meta{pre} and
% \meta{post} are the number of digits before and after the decimal
% sign, respectively.  Both signs and exponents can be included in
% \opt{tabformat}, resulting in appropriate space being reserved. The
% entire \opt{tabformat} input is processed using the \cs{num} macro
% internally.  Thus the decimal and exponent signs used in
% \opt{tabformat} are checked against \opt{numdecimal} and
% \opt{numexp}, respectively.
%
%\DescribeOption{tabalignexp}
%\DescribeOption{tabexpalign}
%\changes{v1.1d}{2008/10/29}{Added \opt{tabexpalign} as an alias for
%  \opt{tabalignexp}}
% When \opt{tabformat} contains exponents, two possibilities are
% available for alignment.  The first method is to place the exponent
% parts so that the ``$\times 10$'' parts form a column, with
% whitespace after shorter mantissa components.  In the second
% method, no additional space is added after the mantissa, and the
% exponents do no line up (\ref{tab:alignexp}).  This is controlled
% by the \opt{tabalignexp} option, which can be set to
% \defaultopt{true} or \opt{false}. The alias \opt{tabexpalign} is
% available for this option.
%\begin{LaTeXexample}[float]
%  \begin{table}
%    \centering
%    \caption{The \opt{tabalignexp} option}
%    \label{tab:alignexp}
%    \sisetup{tabformat=1.3e2,tabnumalign=centre}
%    \begin{tabular}{SS[tabalignexp=false]}
%      \toprule
%      {Header} & {Header} \\
%      \midrule
%      1.2e3    & 1.2e3    \\
%      1.234e56 & 1.234e56 \\
%      \bottomrule
%    \end{tabular}
%  \end{table}
%\end{LaTeXexample}
%
%\DescribeOption{tabtextalign}
%\DescribeOption{tabunitalign}
%\DescribeOption{tabalign}
% Cells containing no numbers are handled by \currpkg in a manner
% similar to \cs{multicolumn}.  The setting of \opt{tabtextalign} is
% taken from the list \defaultopt{centre}, \opt{right} and
% \opt{left}.\footnote{Alias \opt{centre} = \opt{center}.}  As would
% be expected, these settings centre, right- or left-align the cell
% contents. In |s| columns, all content is treated as input to the
% \cs{si} macro.  The alignment of the contents relative to the cell
% is controlled by the \opt{tabunitalign} option, which takes options
% \defaultopt{left}, \opt{right} and \opt{centre}.  The settings for
% \opt{tabnumalign}, \opt{tabtextalign} and \opt{tabunitalign} can be
% set to the same value in one go with the \opt{tabalign} option.
%
%\DescribeOption{tabautofit}
% The contents of table cells can automatically be rounded or
% zero-filled to the number of decimal places given in
% \opt{tabformat}.  This is activated by the \opt{tabautofit} Boolean
% option. As \opt{tabformat} does not apply to columns with alignment
% \opt{centredecimal}, \opt{tabautofit} is also inactive for these
% columns (\ref{tab:autofit}).
%\begin{LaTeXexample}[float]
%  \begin{table}
%    \centering
%    \caption{The \opt{tabautofit} option}
%    \label{tab:autofit}
%    \sisetup{tabformat=1.3,tabnumalign=centre}
%    % Notice the overfull hbox which results with
%    % the first column
%    \begin{tabular}{%
%      S%
%      S[tabautofit]%
%      S[tabautofit,tabnumalign=centredecimal]}
%      \toprule
%      {Header} & {Header} & {Header} \\
%      \midrule
%      1.2    & 1.2    & 1.2    \\
%      1.2345 & 1.2345 & 1.2345 \\
%      \bottomrule
%    \end{tabular}
%  \end{table}
%\end{LaTeXexample}
%
%\DescribeOption{tabparseonly}
% It is possible to turn off the automatic alignment of numbers
% while retaining the ability to parse the |S| column input.
% Setting \opt{tabparseonly} to \opt{true} will still parse the
% column contents, but will not align the decimal markers.  The
% alignment of the cell is governed by \opt{tabnumalign}
% (\ref{tab:parse}).
%\begin{LaTeXexample}[float]
%  \begin{table}
%    \centering
%    \caption{The \opt{tabparseonly} option}
%    \label{tab:parse}
%    \begin{tabular}{%
%      S[tabparseonly]%
%      S[tabparseonly,tabnumalign=left]%
%      S[tabparseonly,tabnumalign=right]}
%      \toprule
%      {Header} & {Header} & {Header} \\
%      \midrule
%      14.2    & 14.2    & 14.2    \\
%      1.23456 & 1.23456 & 1.23456 \\
%      1.2e3   & 1.2e3   & 1.2e3   \\
%      \bottomrule
%    \end{tabular}
%  \end{table}
%\end{LaTeXexample}
%
%\subsection{Units}
%\DescribeOption{per}
% Most of the unit options are concerned with the processing of named
% units.  The processor for units given as macro names can be
% influenced to give a variety of output formats.  The \opt{per}
% option defines how the keyword macro \cs{per} is handled.  This
% option takes a choice from the list \defaultopt{reciprocal},
% \opt{slash} and \opt{fraction}.\footnote{Aliases \opt{reciprocal} =
% \opt{rp} = \opt{power}, \opt{fraction} = \opt{frac}.}  The default
% option uses \cs{per} to indicate reciprocal powers, whereas
% \opt{slash} causes the package to use ``$/$'' to show division.
%
%\DescribeOption{fraction}
%\DescribeOption{slash}
% The \opt{fraction} option defines how \opt{per=fraction} is
% interpreted.  The list of applicable values here is
% \defaultopt{frac}, \opt{nice}, \opt{ugly} and \opt{sfrac}.  In each
% case, the unit is typeset as a fraction, but the macro use to
% achieve this varies. \opt{frac} uses the \TeX\ \cs{frac}, macro,
% while \opt{nice} makes use of a \cs{nicefrac}-like method.  The
% \opt{ugly} option uses a slash in text mode and \cs{frac} in maths
% mode.\footnote{Similar to the \pkg{ugly} option of the
% \pkg{nicefrac} package.}  Finally, the setting \opt{fraction=sfrac}
% uses the \cs{sfrac} macro from the \pkg{xfrac} package, when
% available.\footnote{\pkg{xfrac} is part of the experimental system
% for \LaTeX3.  As it requires a number of additional packages to
% work, \currpkg does not load \pkg{xfrac}. If it is unavailable, the
% \opt{sfrac} setting will fall back to using \cs{nicefrac}.  See the
% \pkg{xfrac} documentation for reasons to prefer \cs{sfrac} to
% \cs{nicefrac}.}  The \opt{slash} option sets the symbol used when
% \opt{per=slash} is in force.  This recognises the single keyword
% \opt{slash}; anything else is used literally.
%\begin{LaTeXexample}
%  \sisetup{per=fraction}
%  \si[fraction=frac]{\metre\per\second}\\
%  \si[fraction=nice]{\metre\per\second}\\
%  \si[fraction=sfrac]{\metre\per\second}\\
%  \si[per=slash]{\metre\per\second}
%\end{LaTeXexample}
%
%\DescribeOption{stickyper}
% By default, \cs{per} applies only to the next unit
% given.\footnote{This is the standard method of reading units in
% English: for example, \si{\joule\per\mole\per\kelvin} is pronounced
% ``joules per mole per kelvin''.}  By setting the \opt{stickyper}
% flag, this behaviour is changed so that \cs{per} applies to all
% subsequent units.\footnote{This is the behaviour in \SIunits.}
%\begin{LaTeXexample}
%  \si{\kilogram\per\metre\ampere}\\
%  \si[stickyper]{\kilogram\per\metre\ampere}
%\end{LaTeXexample}
%
%\DescribeOption{trapambigfrac}
%\DescribeOption{openfrac}
%\DescribeOption{closefrac}
% When using \opt{per=slash}, multiple units in the denominator will
% yield a potentially ambiguous result.  The \opt{trapambigfrac}
% determines whether the package checks for this: this takes
% \defaultopt{true} and \opt{false}.  When set, the contents of
% \opt{openfrac} are inserted before the denominator, and
% \opt{closefrac} is inserted after.
%\begin{LaTeXexample}
%  \sisetup{trapambigfrac,openfrac=(,closefrac=),per=slash}\\
%  \si{\metre\per\second\per\kilogram}\\
%  \si[trapambigfrac=false]{\metre\per\second\per\kilogram}
%\end{LaTeXexample}
%
%\DescribeOption{prefixsymbolic}
%\DescribeOption{prefixbase}
%\DescribeOption{prefixproduct}
% The unit prefixes (\cs{kilo}, \etc) are normally given as letters.
% However, the package can convert these into numerical
% powers.\footnote{Provided things are not too complex!}  This is
% controlled by the \opt{prefixsymbolic} Boolean option, which by
% default is \defaultopt{true}.  If \opt{prefixsymbolic} is set to
% \opt{false}, the format of the prefix is controlled by
% \opt{prefixbase} and \opt{prefixproduct}, which work in the same
% way as \opt{expbase} and \opt{expproduct}.
%
%\DescribeOption{prespace}
%\DescribeOption{xspace}
%\DescribeOption{allowoptarg}
% By default, the single unit macros (\eg \cs{metre}) add no space
% either before or after the unit.  Setting the \opt{xspace} flag to
% true means that the single macros are followed by the \cs{xspace}
% command (when used outside of \cs{SI}/\cs{si}).  For users of
% \unitsdef, the \opt{prespace} macro changes the behaviour of the
% unit macros, so that they can immediately follow a number.  As a
% result, the unit macros will \emph{always} be preceded by a fixed
% space when the \opt{prespace} flag is true: this will be in
% addition to any other space. Also relevant to users moving from
% \unitsdef is the \opt{allowoptarg} option.  This allows single unit
% macros to take an optional numerical argument, in the same way that
% occurs in that package.
%\begin{LaTeXexample}
%  \metre is the symbol for metres\\
%  \sisetup{xspace}
%  No, \metre is the correct symbol\\
%  \sisetup{prespace}
%  30\metre\\
%  Do not use \metre in running text\\
%  \sisetup{allowoptarg}
%  \metre[40]
%\end{LaTeXexample}
%
%\subsection{Symbols}
% User access to control the symbols used for \si{\ohm}, \si{\micro},
% \si{\degree}, \si{\arcmin}, \si{\arcsec}, \si{\angstrom} and
% \si{\celsius} is provided here. These are all literal options,
% which are available in text and maths mode variants.  For example,
% \opt{textmicro} is the code used for the \si{\micro} symbol in text
% mode.  The text mode macros should be safe when forced into text,
% and the maths ones when forced into maths.  The symbols defined in
% this way are:
%\newcommand*{\listsymbopt}[1]{\item \opt{#1}\SpecialOptionIndex{#1}}
%\begin{itemize}
%  \listsymbopt{textOmega};
%  \listsymbopt{mathsOmega};
%  \listsymbopt{textmu};
%  \listsymbopt{mathsmu};
%  \listsymbopt{textdegree};
%  \listsymbopt{mathsdegree};
%  \listsymbopt{textminute};
%  \listsymbopt{mathsminute};
%  \listsymbopt{textsecond};
%  \listsymbopt{mathssecond};
%  \listsymbopt{textringA};
%  \listsymbopt{mathsringA};
%  \listsymbopt{textcelsius};
%  \listsymbopt{mathscelsius}.
%\end{itemize}
%
%\DescribeOption{redefsymbols}
% When \currpkg is loaded, it can check for the presence of the
% \pkg{textcomp} and \pkg{upgreek} packages, to provide better
% symbols for certain items. To prevent this, set the
% \opt{redefsymbols} option \opt{false} (the default is
% \defaultopt{true}).
%
%\DescribeOption{eVcorra}
%\DescribeOption{eVcorrb}
% The \si{\electronvolt} symbol requires some fine-tuning, and so has
% two options of its own, both \TeX\ lengths.  \opt{eVcorra} is the
% correction applied to the gap between ``e'' and ``V'' of the unit:
% the default is \defaultopt{0.3ex}. \opt{eVcorrb} is the correction
% applied to the gap between ``V'' of the unit and whatever follows;
% the default is \defaultopt{0ex}.  The optimal value for these
% options will depend on the current font settings.\footnote{This
% document uses \opt{eVcorra=0.1ex}.}
%\begin{LaTeXexample}
%  \si[per=slash]{\electronvolt\per\metre}\\
%  \si[per=slash,eVcorrb=0.7ex]{\electronvolt\per\metre}
%\end{LaTeXexample}
%
%\subsection{Colour}
%\DescribeOption{colourall}
%\DescribeOption{colourunits}
%\DescribeOption{colourvalues}
%\SpecialOptionIndex{colorall}
%\SpecialOptionIndex{colorunit}
%\SpecialOptionIndex{colorvalue}
% The package provides internal hooks for applying colour to part or
% all of the output.  This requires the user to load the \pkg{color}
% or \pkg{xcolor} package to support colour in the output; \currpkg
% will ignore a colour request if support is unavailable. The Boolean
% options \opt{colourall}, \opt{colourunits} and \opt{colourvalues}
% are used to turn application of a given colour on or off for all
% output, only units and only values, respectively.  All three
% switches are available with US spelling, \eg \opt{colorall} and
% \opt{colourall} behave in the same way.  With colour turned off, no
% \cs{color} command is issued internally, and output follows the
% surrounding text.
%
%\DescribeOption{colour}
%\DescribeOption{unitcolour}
%\DescribeOption{valuecolour}
%\SpecialOptionIndex{color}
%\SpecialOptionIndex{unitcolor}
%\SpecialOptionIndex{valuecolor}
% The colour names to use for colouring output are set by the
% \opt{colour}, \opt{unitcolour} and \opt{valuecolour} options (all
% also available with US spelling).  The \opt{colour} option
% internally sets both \opt{unitcolour} and \opt{valuecolour}.
%\begin{LaTeXexample}
%  {\color{brown} Text \num{50}}\\
%  \sisetup{colourall,colour=green}
%  \SI{10}{\metre}\\
%  {\color{purple} Text \num{70}}\\
%  \sisetup{colourunits=true,colourvalues=false}
%  \SI{40}{\second}
%\end{LaTeXexample}
%
%\DescribeOption{colourneg}
%\DescribeOption{negcolour}
%\SpecialOptionIndex{colorneg}
%\SpecialOptionIndex{negcolor}
% \currpkg can automatically add a colour to negative numbers.  This
% is turned on using the \opt{colourneg} switch.  The colour used is
% set by the \opt{negcolour} option; both options are available using
% US spellings.
%
%\subsection{International support}
%\DescribeOption{locale}
% \currpkg allows the user to switch between the typographic
% conventions of different (geographical) areas by using
% \latin{locales}.  Currently, the package is supplied with
% configurations for locales \opt{UK}, \opt{USA}, \opt{DE} (Germany)
% and \opt{ZA} (South Africa). The \opt{locale} option is used to
% switch to a particular locale.
%\begin{LaTeXexample}
%  \SI{1.234}{\metre}\\
%  \SI[locale=DE]{6.789}{\metre}
%\end{LaTeXexample}
%
%\DescribeOption{loctolang}
% Locales are distinct from \pkg{babel} languages, as typographic
% conventions are not tightly integrated with language.  However, it
% is useful to be able to associate a particular locale with a
% \pkg{babel} language.  The option \opt{loctolang} handles this, and
% expects pairs of values:
% \opt{loctolang=\meta{locale}:\meta{language}}.
%\begin{LaTeXexample}
%  \sisetup{loctolang={UK:UKenglish,DE:german}}\\
%  \SI{6.022e23}{\per\mole}\\
%  \selectlanguage{german}\\
%  \SI{6.022e23}{\per\mole}
%\end{LaTeXexample}
%
%\subsection{Package control}
%\DescribeOption{load}
%\DescribeOption{noload}
%\DescribeOption{alsoload}
% The package keeps most of the unit and abbreviations definitions in
% files separate from \file{siunitx.sty}.  To control what is loaded,
% three complementary options are provided, all of which take a list
% of one or more choices.  \opt{load} and \opt{alsoload} define which
% support configuration files are loaded.  The list in \opt{load}
% recognises the value \opt{default}, which is expanded to the normal
% list before loading. The difference between \opt{load} and
% \opt{alsoload} is that \opt{load} specifies the \emph{complete}
% list of files to load, whereas \opt{alsoload} adds to the existing
% list. To use the \opt{load} option successfully requires knowing
% everything that is needed. The \opt{noload} option can be used to
% delete one or more items from the \opt{load} list, without needing
% to know what is on it.\footnote{\opt{noload} does not prevent the
% loading of a file needed by one which is loaded.  Thus the package
% may internally override a \opt{noload} value if needed.}
%
%\DescribeOption{log}
%\DescribeOption{debug}
% To control data written to the \ext{log} file, the \opt{log} option
% is provided.  This takes a value from the list \defaultopt{normal},
% \opt{none}, \opt{minimal}, \opt{errors} and \opt{debug}.  As would
% be expected, these indicate the amount of detail written to the log
% file. As a shortcut to \opt{log=debug}, the package also recognises
% the \opt{debug} option directly.
%
%\DescribeOption{strict}
% Some users will want to stick closely to the official rules for
% typesetting units.  This could be made complicated if the options
% for non-standards behaviour could not be turned off.  The load-time
% option \opt{strict} resets package behaviour to follow the rules
% closely, and disables options which deviate from this.  If the
% package is loaded with the \opt{strict} option, all output is made
% in maths mode using the upright serif font.
%
%\subsection{Back-compatibility options}
%\DescribeOption{emulate}
% The package can emulate \SIunits, \SIstyle, \unitsdef, \units,
% \pkg{hepunits}, \pkg{fancyunits} and \pkg{fancynum}.  Giving the
% \opt{emulate=\meta{package}} option will give the desired
% emulation, and combinations which would be possible with the real
% packages will also work here. The package will recognise the
% options of the emulated packages.  This will automatically cause
% emulation to be switched on.
%
%\subsection{Summary of all options}
% \ref{tab:allopt} lists a summary of the package options (excluding
% those for backward-compatibility).  A reminder or the input format
% is also provided.
%\begin{longtable}{>{\ttfamily}lll}
%    \caption{All package options\label{tab:allopt}}\\
%    \toprule
%    Option & Type & Description \\
%    \midrule
%  \endfirsthead
%    \toprule
%    Option & Type & Description \\
%    \midrule
%  \endhead
%    \bottomrule
%    \multicolumn{3}{r}{\emph{Continued on next page}}
%  \endfoot
%    \bottomrule
%  \endlastfoot
%  addsign & List & Add sign to number \\
%  allowzeroexp & Boolean & Allow \num[allowzeroexp]{e0} \\
%  angformat & List & Conversion of angle format \\
%  anglesep & List or literal & Space between angle components \\
%  astroang & Boolean & Astronomy-style angles \\
%  closeerr & Literal & Closes potential-ambiguous error \\
%  closefrac & Literal & Closes potential-ambiguous fraction \\
%  color & Literal & Colour used for units and values \\
%  colour & Literal & Colour used for units and values \\
%  colorall & Boolean & Switch for colouring all output \\
%  colourall & Boolean & Switch for colouring all output \\
%  colorneg & Boolean & Colour negative numbers \\
%  colourneg & Boolean & Colour negative numbers \\
%  colorunits & Boolean & Switch for colouring units \\
%  colourunits & Boolean & Switch for colouring units \\
%  colorvalues & Boolean & Switch for colouring values \\
%  colourvalues & Boolean & Switch for colouring values \\
%  decimalsymbol & List or literal & Decimal symbol \\
%  debug & Boolean & Write debugging data to log \\
%  detectdisplay & Boolean & Treat display maths separately \\
%  digitsep & List & Separation of digits in large numbers \\
%  dp & Integer & Number of decimal places to \\
%   & & output numbers to \\
%  emulate & Modules & Emulation modules to load \\
%  errspace & List & Spacing of bracketed error \\
%  eVcorra & Length & Spacing correction in \eV \\
%  eVcorrb & Length & Spacing correction after \eV \\
%  expbase & List or literal & Base used for exponents \\
%  expproduct & List or literal & Product sign for exponents \\
%  fixdp & Boolean & Switch for fixing decimal \\
%   & & places of numbers \\
%  fixsf & Boolean & Switch for fixing significant \\
%    & & figures of numbers \\
%  fraction & List & Method used when setting \opt{per=frac} \\
%  inlinebold & List & Select how inline bold is tested \\
%  load & Modules & Modules to load \\
%  locale & Modules & Locale to follow \\
%  loctolang & Special & Associate locale with \pkg{babel}
%    language \\
%  log & List & Amount of data added to log \\
%  mathOmega & Literal & ``$\makeatletter\si@mathsOmega\makeatother$''
%    symbol in maths mode \\
%  mathcelsius & Literal & ``$\makeatletter\si@mathscelsius\makeatother$''
%    symbol in maths mode \\
%  mathdegree & Literal & ``$\makeatletter\si@mathsdegree\makeatother$''
%    symbol in maths mode \\
%  mathminute & Literal & ``$\makeatletter\si@mathsminute\makeatother$''
%    symbol in maths mode \\
%  mathmu & Literal & ``$\makeatletter\si@mathsmu\makeatother$''
%    symbol in maths mode \\
%  mathringA & Literal & ``$\makeatletter\si@mathsringA\makeatother$''
%    symbol in maths mode \\
%  mathrm & Csname & Roman maths font \\
%  mathsOmega & Literal & ``$\makeatletter\si@mathsOmega\makeatother$''
%    symbol in maths mode \\
%  mathscelsius & Literal & ``$\makeatletter\si@mathscelsius\makeatother$''
%    symbol in maths mode \\
%  mathsdegree & Literal & ``$\makeatletter\si@mathsdegree\makeatother$''
%    symbol in maths mode \\
%  mathsecond & Literal & ``$\makeatletter\si@mathssecond\makeatother$''
%    symbol in maths mode \\
%  mathsf & Csname & Sans serif maths font \\
%  mathsminute & Literal & ``$\makeatletter\si@mathsminute\makeatother$''
%    symbol in maths mode \\
%  mathsmu & Literal & ``$\makeatletter\si@mathsmu\makeatother$''
%    symbol in maths mode \\
%  mathsringA & Literal & ``$\makeatletter\si@mathsringA\makeatother$''
%    symbol in maths mode \\
%  mathsrm & Csname & Roman maths font \\
%  mathssecond & Literal & ``$\makeatletter\si@mathssecond\makeatother$''
%    symbol in maths mode \\
%  mathssf & Csname & Sans serif maths font \\
%  mathstt & Csname & Fixed-width maths font \\
%  mathtt & Csname & Fixed-width maths font \\
%  mode & List & Use text or maths mode for \\
%   & & typesetting \\
%  negcolor & Literal & Colour used for negative numbers \\
%  negcolour & Literal & Colour used for negative numbers \\
%  noload & Modules & Modules not to load \\
%  numaddn & Literal & Additional input allowed in numbers \\
%  numcloseerr & Literal & Character indicating end of \\
%   & & numerical error\\
%  numdecimal & Literal & Decimal symbols in numbers \\
%  numdigits & Literal & Digit characters in numbers \\
%  numdiv & Literal & Division characters in numbers \\
%  numexp & Literal & Exponent characters in numbers \\
%  numgobble & Literal & Characters to ignore in numbers \\
%  numopenerr & Literal & Character indicating start of \\
%   & & numerical error\\
%  numprod & Literal & Characters used for a product \\
%  numsign & Literal & Sign characters in numbers \\
%  obeyall & Boolean & Combination of \opt{obeybold}, \\
%   & & \opt{obeyitalic} and \opt{obeymode} \\
%  obeybold & Boolean & Check local bold setting \\
%  obeyitalic & Boolean & Check local italic setting \\
%  obeymode & Boolean & Check local mode (text/maths) \\
%  openerr & Literal & Opens potentially-ambiguous error \\
%  openfrac & Literal & Opens potentially-ambiguous fraction \\
%  padangle & List & Add zeros to blank parts of angles \\
%  padnumber & List & Add zeros to blank parts of numbers \\
%  per & List & Behaviour of \cs{per} \\
%  prefixbase & List or literal & Base used when making prefixes \\
%   & & numerical \\
%  prefixproduct & List or literal & Product sign for prefixes \\
%  prefixsymbolic & Boolean & Behaviour of unit prefixes\\
%  prespace & Boolean & Add space before units \\
%  redefsymbols & Boolean & Use better symbols if available \\
%  repeatunits & List & Repeat units with separated errors \\
%   & &  and products \\
%  retainplus & Boolean & Retain explicit plus sign\\
%  seperr & Boolean & Separate number and error \\
%  sepfour & Boolean & Separate four-digit numbers \\
%  sf & Integer & Number of significant figures \\
%   & & to output numbers to \\
%  sign & List or literal & Sign to add to numbers \\
%  slash & List or literal & Symbol used for ``/'' \\
%  stickyper & Boolean & Require \cs{per} only once \\
%  strict & Boolean & Obey the rules strictly \\
%  strictarc & Boolean & Require exactly zero or \\
%   & & two semi-colons \\
%  tabalign & List & Positioning of all column data \\
%  tabalignexp & Boolean & Alignment of exponents in tables \\
%  tabautofit & Boolean & Switch for rounding numbers to \\
%   & & length given by \opt{tabformat} \\
%  tabformat & Number & Space reserved in table for numbers \\
%  tabnumalign & List & Alignment of |S| column numbers \\
%  tabparseonly & Boolean & Do not align |S| columns \\
%   & & on decimal marker \\
%  tabtextalign & List & Positioning of text in |S| columns \\
%  tabunitalign & List & Positioning of units in |s| columns \\
%  textcelsius & Literal & ``$^{\circ}$C'' symbol in text mode \\
%  textdegree & Literal & ``${\circ}$'' symbol in text mode \\
%  textminute & Literal & ``$'$'' symbol in text mode \\
%  textmu & Literal & ``$\upmu$'' symbol in text mode \\
%  textomega & Literal & ``$\Omega$'' symbol in text mode \\
%  textringa & Literal & ``\AA'' symbol in maths mode \\
%  textrm & Csname & Roman text font \\
%  textsecond & Literal & ``$''$'' symbol in text mode \\
%  textsf & Csname & Sans serif text font \\
%  texttt & Csname & Fixed-width text font \\
%  tightpm & Boolean & Reduce space around $\pm$ \\
%  trapambigerr & Boolean & Check for ambiguous errors \\
%  trapambigfrac & Boolean & Check for ambiguous fractions \\
%  unitcolor & Literal & Switch for colouring units \\
%  unitcolour & Literal & Switch for colouring units \\
%  unitmathrm & Csname & Roman maths font for units \\
%  unitmathsf & Csname & Sans serif maths font for units \\
%  unitmathsrm & Csname & Roman maths font for units \\
%  unitmathssf & Csname & Sans serif maths font for units \\
%  unitmathstt & Csname & Fixed-width maths font for units \\
%  unitmathtt & Csname & Fixed-width maths font for units \\
%  unitmode & List &  As \opt{mode}, for units only \\
%  unitsep & List or literal & Separator for units \\
%  unitspace & List or literal & Space used for ``|~|'' in units \\
%  valuecolor & Literal & Switch for colouring value \\
%  valuecolour & Literal & Switch for colouring value \\
%  valuemathrm & Csname & Roman maths font for values \\
%  valuemathsf & Csname & Sans serif maths font for values \\
%  valuemathsrm & Csname & Roman maths font for values \\
%  valuemathssf & Csname & Sans serif maths font for values \\
%  valuemathstt & Csname & Fixed-width maths font for values \\
%  valuemathtt & Csname & Fixed-width maths font for values \\
%  valuemode & List &  As \opt{mode}, for values only \\
%  valuesep & List or literal & Separator between value and unit \\
%  xspace & Boolean & Use \pkg{xspace} after units \\
%\end{longtable}
%
%\section{Emulation of other packages}
% \currpkg has been designed as a replacement for \SIunits, \SIstyle,
% \unitsdef, \units, \pkg{hepunits}, \pkg{fancyunits} and
% \pkg{fancynum}. It therefore provides options reproduce the
% functions of all of these packages.  In this way, \currpkg should
% be usable as a straight replacement for the older
% packages.\footnote{User macros means that they are described in the
% package documentation; simply not containing an \texttt{@} does not
% mean they will have been emulated.}  This means for example that
% the \cs{num} macro takes an optional star when emulating \SIstyle.
% However, there are some points that should be remembered.  In
% particular, \currpkg validates numerical input, meaning that places
% where a number is expected in the older packages \emph{require} a
% number when emulated by \currpkg.
%
% The \numprint package has provided many useful ideas for the code
% used here for number formatting. The basic use of the \cs{numprint}
% (or \cs{np}) macro can be reproduced using \currpkg.  However,
% \numprint is large and complex, with its own backward-compatibility
% options.  As a result, emulation of \numprint is not provided here.
% To use a \numprint document with \currpkg, the \cs{numprint} macro
% could be provided using the following code.
%^^A Need to deal with \cs{numprint} being defined.
%\let\numprint\relax
%\begin{LaTeXexample}
%  \newcommand*{\numprint}[2][]{\SI[obeymode]{#2}{#1}}\\
%  \numprint{-123456} \\
%  \numprint[N/mm^2]{-123456}
%\end{LaTeXexample}
%^^A Back to the normal - with \cs{newcommand}
%\newcommand*{\numprint}{\pkg{numprint}\xspace}
%
% \currpkg can be used more-or-less directly to replace both
% \pkg{dcolumn} and \pkg{rccol}.  As is explained in the code
% section, much of the column-alignment system here is taken from
% \pkg{dcolumn}, while \pkg{rccol} provided a model for a
% customisable system.  However, neither package has been directly
% emulated here.  The |S| column type can be used to replace both |D|
% and |R| columns by setting the appropriate package options.
%
%\section{Configuration files}
% \currpkg is a modular package.  The unit definitions, abbreviations
% and locales are all stored in configuration files.  These all take
% names of the form \file{si-\meta{name}.cfg}, where \meta{name} is
% the part of the filename used as an option in \cs{sisetup} or when
% loading \currpkg.  Producing new configuration files therefore
% consists of making a suitably-named file and adding it to the path
% searched by \TeX.  The files should normally consist of settings
% (in \cs{sisetup}) and unit definitions, \etc.
%
%\DescribeMacro{\addtolocale}
% To allow arbitrary macros to be stored in locales, the
% \cs{addtolocale} macro is provided.  This ensures that arbitrary
% text is only executed when using a locale, not when loading it. The
% macro takes two arguments, \marg{locale} and \marg{code}.
%\begin{LaTeXexample}
%  \addtolocale{DE}{TEST}
%  Some text as filler \\
%  \sisetup{locale=DE}
%  This example is rather trivial!
%\end{LaTeXexample}
%
%\DescribeMacro{\requiresiconfigs}
% To load one or more configuration files from inside another
% configuration file, the \cs{requiresiconfigs} macro is provided.
% This accepts a comma-separated list of configuration names, in the
% same way as \opt{load} or \opt{noload}.
%
% In addition to the various configuration files provided with the
% package, a local file \file{siunitx.cfg} may be provided.  This is
% read at the end of loading \currpkg, and allows the user to include
% any local definitions or settings easily.
%
%\section{Common questions}
%\subsection{Why do I need \cs{per} more than once?}
% The unit engine of \currpkg is based around the English method for
% reading units out loud. Thus \si{\joule\per\mole\per\kelvin} is
% pronounced ``joules per mole per kelvin''.  Hence by default you
% need to put \cs{per} before each item in the denominator of a unit.
% The behaviour can be altered by setting the \opt{stickyper} option.
%
%\subsection{Why is the order of my units changed?}
% Then using \opt{per=reciprocal}, the units are typeset as given.
% However, when using \opt{per=slash} or \opt{per=fraction}, the
% package needs to find which ones are in the denominator.  It then
% prints the numerator and denominator separately.  So if you give a
% unit in the denominator \emph{before} one in the numerator, they
% have to be re-ordered.\footnote{``Double division''
% (\SI{1}{s/m/kg}) is mathematically incorrect.}
%\begin{LaTeXexample}
%  \SI{88}{\per\kilogram\metre} \\
%  \SI[per=slash]{66}{\per\kilogram\metre}
%\end{LaTeXexample}
%
%\subsection{Why are compound units not recommended outside of
%  \cs{SI}/\cs{si}?}
% To fully process units made up of several parts, the processor has
% to know where the end of the unit is.  When the unit macros are
% used outside of \cs{SI}/\cs{si}, this is not the case.  The package
% therefore does its best, but results may be sub-optimal.  To get
% consistent results, either define a new \emph{single} unit, or keep
% compound units inside \cs{SI} and \cs{si}.
%\begin{LaTeXexample}
%  \metre\per\micro\second \\
%  \si{\metre\per\micro\second} \\
%  \newunit{\myunit}{\metre\per\micro\second}
%  \myunit
%  \newunit{\myunittwo}{\kilogram\myunit} \\
%  \myunittwo \\
%  \kilogram\myunit
%\end{LaTeXexample}
% Notice the difference in behaviour of \cs{per} in the first two
% lines, and the spacing error on the last line in the example.
%
%\subsection{How do I set superscripts to use lining numbers?}
% Lowercase (``old style'') numbers are favoured by many people for
% use in running text.  However, this does not necessarily look good
% in superscripts. The mode used to typeset data can be varied, so
% that maths mode numbers are used for the unit part of the output.
%\begin{LaTeXexample}
%  \SI[mode=text]{1234}{\metre\per\second}\\
%  \SI[valuemode=text,unitmode=maths]{1234}{\metre\per\second}
%\end{LaTeXexample}
%
%\subsection{Why do most of the examples use
%  \texorpdfstring{\si{\joule\per\mole\per\kelvin}}{J mol-1 K-1}?}
% The package author is a chemist, and this is the unit of entropy
% (disorder).  It nicely demonstrates the use of the \cs{per} macro,
% and so it crops up a lot.  It is also in the subsection heading
% here to act as a test with \pkg{hyperref} and moving arguments!
%
%\subsection{What can \numprint do that \currpkg cannot?}
% \currpkg uses a lot of ideas from \numprint: a reasonable amount of
% the number-processing code here is based on that in \numprint.
% However, the two packages have somewhat different aims, and as a
% result there are things that \numprint can do that \currpkg does
% not implement.  The main features of \numprint not available here
% are:
%\begin{itemize}
%  \item General support for numbers with base other than $10$ (see
%    \pkg{nbaseprt});
%  \item Alignment of the decimal marker of powers in tables;
%  \item Alignment of numbers in running text;
%  \item Specific formatting commands for \TeX\ counters and lengths.
%\end{itemize}
%
%\section{Tricks and known issues}
%
%\subsection{Ensuring maths mode}
% Due to the possibility of output in either maths or text mode, any
% input which requires a particular mode needs to be protected.  You
% cannot use |$|\ldots|$|, as this can get ``caught out'', but also
% as it may give hard-to-follow errors.  Always use \cs{ensuremath}
% to force maths processing, and \cs{text} (from the \AMS\TeX\
% bundle) to ensure text mode.
%
%\subsection{Using \texttt{.} and fixed spaces in units}
% To use a literal |.| in a unit, it has to be within an extra set of
% braces.  This does not need any extra protection, unlike the
% situation with \SIstyle (for example, no \cs{text} macro is
% needed).  The fixed space (|~|) is more problematic: set
% \opt{unitspace=space} to get a full space here.
%\begin{LaTeXexample}
%  \newunit[unitspace=space]{\myunit}{V~vs{.}~NHE}%
%  \SI{10}{\myunit}
%\end{LaTeXexample}
%
%\subsection{Passing unprocessed digits through an \texttt{S}
%  column}
% The method used to detect numbers in an |S| column will pick up
% material wrapped inside braces if there is more than a single
% character inside the braces.  If you want to pass a \emph{single}
% numerical character without processing it, you need two sets of
% braces (\ref{tab:S-limits}).
%\begin{LaTeXexample}[float]
%  \begin{table}
%    \centering
%    \caption{Passing single digit characters}
%    \label{tab:S-limits}
%    \begin{tabular}{S[colourall,colour=orange]}
%      \toprule
%        {Heading} \\
%      \midrule
%        {1-2} \\
%        {1}   \\
%        {{1}} \\
%        {{-}} \\ % Using {-} gives an error!
%      \bottomrule
%    \end{tabular}
%  \end{table}
%\end{LaTeXexample}
%
%\subsection{Limitations of \cs{mathrm}}
% The package uses the \cs{mathrm} font family by default to typeset
% output in maths mode.  This however has a few side-effects.  For
% example, the Greek alphabet can give odd results.\footnote{This
% depends on your font setup; this document uses T1 encoding, which
% shows the issue, whereas using OT1 does not.}  The use of the
% \cs{mathnormal} font \emph{may} get around this issue.
%\begin{LaTeXexample}
%  \num[numaddn=\pi]{4\pi e-7}\\
%  \num[numaddn=\pi,mathsrm=mathnormal]{4\pi e-7}
%\end{LaTeXexample}
% On the other hand, you may want to use text mode, in which case
% \cs{ensuremath} is needed.  Depending on the exact circumstances,
% the \LaTeX\ protection mechanism (\cs{DeclareRobustCommand}) may be
% sufficient; in some cases, this will fail and the \eTeX\
% \cs{protected} system may be required. There are several potential
% pitfalls in this area; experimentation may well be needed.
%\begin{LaTeXexample}
%  \DeclareRobustCommand*{\numpi}{\ensuremath{\pi}}
%  \num[numaddn=\numpi,mode=text]{4\numpi e-7}
%\end{LaTeXexample}
%
%\subsection{Entire document in sans serif font}
% If your entire document is not in a Roman font, using
% the font detection system is not the most efficient method
% for setting the \currpkg output.  Instead, the \opt{mathrm}
% and \opt{textrm} package options can be redefined.
%\begin{LaTeXexample}
%  \sffamily
%  Some text \\
%  \sisetup{obeyfamily=false,mathrm=mathsf,textrm=sffamily}
%  \num{1e2} \\
%  \SI{3}{\newton}
%  \[ \SI{4e5}{\pascal} \]
%\end{LaTeXexample}
%
%\subsection{Effects of emulation}
% The package has been designed so that almost everything can be set
% using the options.  In the emulation code, some internal macros are
% redefined.  This is because the legacy packages do odd things,
% which are deliberately not implemented by \currpkg.  Using an
% emulation file will prevent subsequent loading of the real package.
% This is to prevent errors or, worse, difficult to diagnose changes
% to output.
%
%\subsection{Centring columns on non-decimal input}
% The \pkg{dcolumn} manual suggests using that package to align a
% column on a $\pm$ sign. The same type of output is possible using
% \currpkg, but some care is needed (\ref{tab:dcolumn}).  Odd things
% may happen: use with care!
%\begin{LaTeXexample}[float]
%  \begin{table}
%    \centering
%    \caption{Non-standard \texttt{S} column}
%    \label{tab:dcolumn}
%    \begin{tabular}{%
%      S[digitsep=none,decimalsymbol={\,\pm\,},
%        numdigits={0123456789.},numdecimal=+]}
%    \toprule
%    {Some Values} \\
%    \midrule
%    2.3456  + 0.02  \\
%    34.2345 + 0.001 \\
%    56.7835 + 0.067 \\
%    90.473  + 0.021 \\
%    \bottomrule
%    \end{tabular}
%  \end{table}
%\end{LaTeXexample}
%
%\subsection{Expanding content in tables}
% When processing |S| columns, \currpkg works hard to expand any
% items which may give numbers.  So for example you can define values
% as macros (\ref{tab:vmacros}). To avoid the expansion of single
% macros, you must either wrap them in two sets of braces or make
% them robust (using either \cs{DeclareRobustCommand} or
% \cs{protected}).
%\begin{LaTeXexample}[float]
%  \begin{table}
%    \centering
%    \caption{Values as macros}
%    \label{tab:vmacros}
%    \newcommand*{\myvala}{1.234}%
%    \newcommand*{\myvalb}{20.345}%
%    \newcommand*{\myvalc}{0.987654}%
%    \DeclareRobustCommand*{\myvald}{88.88}%
%    \begin{tabular}{S[tabtextalign=left]}
%    \toprule
%    {Some Values} \\
%    \midrule
%    \myvala  \\
%    \myvalb \\
%    \myvalc \\
%    {{\myvala}} \\
%    \myvald \\
%    \bottomrule
%    \end{tabular}
%  \end{table}
%\end{LaTeXexample}
%
% It is possible to use calculated values in tables. For this to
% work, the calculation must take place before attempting to parse
% the number. This means that any calculation code should be wrapped
% in braces and appear before the use of the number. This is
% illustrated by using the \pkg{datatool} package to set up a
% miniature database, and then output the values doubled
% (\ref{tab:calc}). This example also shows that macros such as
% \cs{DLTforeach}, which construct a table while \TeX\ is running,
% cannot be placed inside an |S| column (or an |s| one). Instead, an
% additional dummy column has been added with no inter-column space.
% This is used to contain the tale-building macro.
%\begin{LaTeXexample}[float]
%  \DTLnewdb{data}
%  \DTLnewrow{data}\DTLnewdbentry{data}{value}{66.7012}
%  \DTLnewrow{data}\DTLnewdbentry{data}{value}{66.0212}
%  \DTLnewrow{data}\DTLnewdbentry{data}{value}{64.9026}
%  \begin{table}
%    \caption{Calculated values}
%    \label{tab:calc}
%    \centering
%    \sisetup{tabformat=2.4}
%    \begin{tabular}{SS@{}l}
%      \toprule
%      {Value} & {Doubled}  &
%      \DTLforeach{data}{\myvalue=value}{%
%        \DTLiffirstrow {\\ \midrule}{\\}%
%        \myvalue & % First column
%        {\DTLmul{\myvalue}{\myvalue}{2}} \myvalue % second column
%        & }\\
%      \bottomrule
%    \end{tabular}
%  \end{table}
%\end{LaTeXexample}
%
%\subsection{Adding items after the last column of a tabular}
% If you use an |S| or |s| column as the last one in a tabular, and
% you use the \pkg{array} ``|<|'' construction to add items after it,
% the spacing may be wrong.  This will occur if the column contents
% are of differing widths.  Changing the \LaTeX\ |\\| line ending for
% the plain \TeX\ \cs{cr} will give the correct spacing, but does not
% allow adjustment of inter-row distance (\ref{tab:cr}).\footnote{For
% the \TeX\ experts, the issue here is that the system to gather up
% cell contents is added in using the \texttt{<} construction.
% Normally, this comes after the cell contents and any other
% \texttt{<} arguments, so collects the user additions.  However, in
% the last cell the contents include \texttt{\bslash\bslash}, which
% is converted to \cs{cr} before gathering can occur.  By using
% \cs{cr} directly, the gathering process receives all of the cell
% contents as normal.}  In most cases, this should not be a serious
% issue. Notice that an extra set of braces is needed here, compared
% to usual \LaTeX\ use; this is to prevent any expansion of the
% material by \currpkg.
%\begin{LaTeXexample}[float]
%  \begin{table}
%    \caption{Correcting spacing in last \texttt{S} column}
%    \label{tab:cr}
%    \hfil
%    \begin{tabular}{S<{{\,\si{\kg}}}S<{{\,\si{\kg}}}}
%      \toprule
%      \multicolumn{1}{c}{Long header} &
%        \multicolumn{1}{c}{Long header} \\
%      \midrule
%      1.23 & 1.23 \\
%      4.56 & 4.56 \\
%      7.8  & 7.8  \\
%      \bottomrule
%    \end{tabular}
%    \hfil
%    \begin{tabular}{S<{{\,\si{\kg}}}S<{{\,\si{\kg}}}}
%      \toprule
%      \multicolumn{1}{c}{Long header} &
%        \multicolumn{1}{c}{Long header} \\
%      \midrule
%      1.23 & 1.23 \cr
%      4.56 & 4.56 \cr
%      7.8  & 7.8  \cr
%      \bottomrule
%    \end{tabular}
%    \hfil
%  \end{table}
%\end{LaTeXexample}
%
%\subsection{Using \currpkg\ with the \pkg{cellspace} package}
% Both \currpkg\ and \pkg{cellspace} use the letter \texttt{S} for a 
% new column type. This obviously leads to a problem. If both are 
% loaded, \currpkg\ will retain the \texttt{S} column, and moves the
% functionality of \pkg{cellspace} to the letter \texttt{C}. This 
% allows the normal use of \pkg{cellspace} with standard column types:
% it does \emph{not} work with the \currpkg\ \texttt{S} or \texttt{s}
% columns.
% 
%\subsection{Using \currpkg\ with the \pkg{mathabx} package}
%\changes{v1.1g}{2008/11/26}{Respect loading of \pkg{mathabx}}
% The \pkg{mathabx} package defines its own \cs{second} and 
% \cs{degree} macros.  This is respected by \currpkg: if you want the
% \currpkg\ definitions instead, put the lines
%\begin{LaTeXexample}[noexample]
%  \makeatletter
%  \renewunit{\second}{s}
%  \renewunit[valuesep=none]{\degree}{\si@sym@degree}
%  \makeatother
%\end{LaTeXexample}
% after |\begin{document}|.
%
%\iffalse
%<*bug>
%\fi
%\section{Reporting a problem}
% \currpkg is quite long and complicated, and works hard to cover all
% possible eventualities.  However, there will be bugs in the code
% and unexpected interactions with other packages.  If you think you
% have found a bug, please report it.  A short test-case
% demonstrating the problem would be very welcome.  The following is
% a suitable template, and is available as \file{si-bug.ltx} in the
% \texttt{doc/latex/siunitx} directory or by running the \ext{dtx} or
% \ext{ins} file through \TeX.
%    \begin{macrocode}
\listfiles
\documentclass{article}
%    \end{macrocode}
% Load other packages needed here.
%    \begin{macrocode}
\usepackage{siunitx}
%    \end{macrocode}
% Normally, debugging the load procedure will not be needed; the
% debug option here means that all run-time information is logged.
%    \begin{macrocode}
\sisetup{debug}
\begin{document}
This is the bug test-case document for the \textsf{siunitx}
package.\\
Please put your demonstration here, and e-mail to the package
author.
\begin{center}
  \texttt{joseph.wright@morningtar2.co.uk}
\end{center}
\end{document}
%    \end{macrocode}
%\iffalse
%</bug>
%\fi
%
%\section{Feature requests}
% Feature requests for \currpkg are welcome.  The package maintainer
% will consider any ideas within the remit of the package (units and
% values).  If suggesting a new feature, an example of how it should
% work would be appreciated.  If new controls are needed, some
% suggestions for option names would be welcome.
%
%\section{Acknowledgements}
% Many thanks indeed to Stefan Pinnow, who has made a very large
% number of suggestions and found numerous bugs in the package.  His
% contribution to the package has been vital. The package author has
% learned \LaTeX\ tricks from far too many people to thank all of
% them.  However, for this package specific thanks must go to the
% authors of the existing ``unit'' packages: Danie Els (\SIstyle),
% Marcel Heldoorn (\SIunits), Patrick Happel (\unitsdef), Axel
% Reichert (\units) and Harald Harders (\numprint).  Will Robertson
% and Heiko Oberdiek deserve much credit for demonstrating \LaTeX\
% coding best practice. Victor Eijkhout's excellent (and free)
% \emph{\TeX\ by Topic} has provided some useful coding hints
% \cite{Eijkhout}.  The idea for combining and extending unit
% provision in \LaTeX\ was heavily inspired by Philip Lehmann's
% \pkg{biblatex}. Thanks to the various contributors of ideas for the
% package: Donald Arseneau, Michele Dondi, Paul Gans, Ben Morrow, Lan
% Thuy Pham, Alan Ristow, Patrick Heinze, Andrea Blomenhofer, Morten
% H{\o}gholm, Burkhard Moddemann  and Patrick Steegstra.
%
%\clearpage
%\part{Correct application of (SI) units}
%\section{Background}
% Consistent and logical units are a necessity for scientific work,
% and have applicability everywhere.  Historically, a number of
% systems have been used for physical units.  SI units were
% introduced by the \latin{Conf\'erence G\'en\'erale des Poids et
% Mesures} (CGPM) in 1960.  SI units are a coherent system based on
% seven base units, from which all other units may be derived.
%
% At the same time, physical quantities with units are mathematical
% entities, and as such way that they are typeset is important.  In
% mathematics, changes of type (such as using bold, italic, sans
% serif typeface and so on) convey information.  This means that
% rules exist not only for the type of units to be used under the SI
% system, but also the way they should appear in print.  Advice on
% best practice has been made available by the \emph{National
% Institute of Standards and Technology} (NIST) \cite{NIST}.
%
% As befits an agreed international standard, the full rules are
% detailed. It is not appropriate to reproduce these in totality
% here; instead, a useful summary of the key points is provided. The
% full details are available from the \latin{Bureau International des
% Poids et Mesures} (BIPM) in French \cite{BIPM:FR} and English
% \cite{BIPM:EN}.  They also publish a very useful and detailed guide
% to using units, values and so on, available online in a number of
% different formats \cite{BIPM:PDF}.
%
% \currpkg takes account of the information given here, so far as is
% possible.  Thus the package defaults follow the recommendations
% made for typesetting units and values.  Spacing and so forth is
% handled in such a way as to make implementing the rules
% (relatively) easy.
%
%\section{Units}
%\subsection{SI base units}
% There are seven base SI units, listed in \ref{tab:baseunits}. Apart
% from the kilogram, these are defined in terms of a measurable
% physical quantity needing the definition alone.\footnote{Some base
% units need others defined first; there is therefore a required
% order of definition.}  The base units have been chosen such that
% all physical quantities can be expressed using an appropriative
% combination of these units, needing no others and with no
% redundancy.  The kilogram is slightly different from the other base
% units as it is still defined in terms of a ``prototype'' held in
% Paris.\footnote{At the time of writing, this is under review and
% will be altered.}
%
%\subsection{SI derived units}
% All other units within the SI system are regarded as ``derived''
% from the seven base units.  At the most basic, all other SI units
% can be expressed as combinations of the base units.  However, many
% units (listed in \ref{tab:derived}, \ref{tab:experimental} and
% \ref{tab:accepted}) have a special name and symbol.\footnote{The
% nautical mile has a given name but no agreed symbol, and although
% accepted by the SI is not provided by \currpkg as a unit macro.}
% Most of these units are simple combinations of one or more base
% units (raised to powers as appropriate).  A small number of units
% derived from experimental data are allowed as SI units
% (\ref{tab:experimental}).
%
% Some of these units (in \ref{tab:accepted}) are regarded as only
% ``temporarily'' accepted, as the use of only the base and
% fully-consistent derived units in \ref{tab:derived} is encouraged.
% They are accepted as they are in common use in one or more
% disciplines; some are still very widespread in the appropriate
% areas.  These units are mainly multiples of base units (for
% example, a tonne is \SI{1000}{\kilo\gram}).
%
% One point to note is that ``unitless ratios'' are regarded as
% having base units which cancel out.  For example, the radian is
% regarded as having base unit \si{\metre\per\metre}.  The result of
% this division (``$1$'') is therefore regarded as a derived SI unit
% in this context.
%
%\subsection{SI prefixes}
% A series of SI prefixes for decimal multiples and submultiples are
% provided, and can be used as modifiers for any SI unit (either base
% or derived units) with the exception of the kilogram.  The prefixes
% are listed in \ref{tab:prefix}.  No space should be used between a
% prefix and the unit, and only a single prefix should be used.  Even
% the degree Celsius can be given a prefix, for example
% \SI{1}{\milli\celsius}.  The only exception to this rule is for
% degrees, minutes and seconds of an arc: \ang{1;2;3}.
%
% It is important to note that the kilogram is the only SI unit with
% a prefix as part of its name and symbol.  Only  single prefix may
% be used, and so in the case of the kilogram prefix names are used
% with the unit name ``gram'' and the prefix symbols are used with
% the unit symbol \si{\gram}.  For example $\SI{1e-6}{\kilogram} =
% \SI{1}{\milli\gram}$.
% 
% Three unit-prefix combinations deserve special note. The units
% hectare, kilohm and megohm elide the additional vowel (these
% would otherwise be hectoare, kiloohm and megaohm).  All other
% vowel combinations are retained.\footnote{Thanks to Charles
% Cameron for raising this issue with the package author.}
%
%\subsection{Other units}
% The application of SI units is meant to provide a single set of
% units which ensure consistency and clarity across all areas.
% However, other units are common is many areas, and are not without
% merit. The units provided by \currpkg by default do not included
% any of these; only units which are part of the SI set or are
% accepted for use with SI units are defined.  However, several other
% sets of units can be loaded as optional modules. The binary
% prefixes and units (Section~\ref{sec:binary} and \ref{tab:binary})
% are the most obvious example.  These are \emph{not} part of the SI
% specifications, but the prefix names are derived from those in
% \ref{tab:prefix}.
%
% Other units (such as those provided by the modules \opt{synchem},
% \opt{hep} and \opt{astro}) are normally to be avoided where
% possible.  SI units should, in the main, be preferred due to the
% advantages of clear definition and self-consistency this brings.
% However, there will probably always be a place for specialist or
% non-standard units.  This is particulary true of units derived from
% basic physical constants; for example reason, the \opt{hep} module
% defines the speed of light, \si{\clight}, as a unit.  For work in
% basic science, a small number of physical constants are recognised
% as units provided the results for comparison with experiment are
% given in SI units.
%
% There are also many areas where non-standard units are used so
% commonly that to do otherwise is difficult or impossible.  For
% example, most synthetic chemists measure the pressure inside vacuum
% apparatus in \si{\mmHg}, partly because the most common gauge for
% the task still uses a column of mercury metal.  For these reasons,
% \currpkg does define non-SI units.
%
%\section{Units and values in print}
%\subsection{Mathematical meaning}
% As explained earlier, a unit--value combination is a single
% mathematical entity.  This has implications for how both the number
% and the unit should be printed.  Firstly, the two parts should not
% be separated.  With the exception of the symbols for plane angles
% (\si{\degree}, \si{\arcmin} and \si{\arcsec}), it is usual to have
% a space between the unit and the value.  This should therefore be a
% non-breaking space between the two.  Different geographical areas
% have different conventions on the size of this space; a ``small''
% space (|\,|) is the \currpkg default.
%\begin{LaTeXexample}
%  A space for \SI{10}{\percent}\\
%  and also for \SI{100}{\celsius}\\
%  but not for \ang{1.23}.
%\end{LaTeXexample}
%
% The mathematical meaning of units also means that the shape, weight
% and family are important.  Units are supposed to be typeset in an
% upright, medium weight serif font.  Italic, bold and sans serif are
% all used mathematically to convey other meanings.  \currpkg package
% defaults again follow this convention: any local settings are
% ignored, and uses the current upright serif maths font.  However,
% there are occasions where this may not be the most desirable
% behaviour.  A classic example would be in an all-bold section
% heading.  As the surrounding text is bold, some people feel that
% any units should follow this.
%\begin{LaTeXexample}
%  Units should \textbf{not be bold: \SI{54}{\farad}}\\
%  \textbf{But perhaps in a running block,\\
%  it might look better: \SI[obeybold]{54}{\farad}}
%\end{LaTeXexample}
%
%\subsection{Unit multiplication and division}
% Symbols for units formed from other units by multiplication are
% indicated by means of either a half-height (that is, centred) dot
% or a (thin) space.  This document uses a half-height dot as
% (\emph{i}) this is the recommendation of \textsc{nist}, amongst
% others and (\emph{ii}) it avoids potential confusion between unit
% prefixes and multiplied units.
%\begin{LaTeXexample}
%  $\si{\metre\second} = \mbox{metre second}$ \\
%  $\si{\milli\second} = \mbox{millisecond}$ \\
%  \sisetup{unitsep=thin}
%  $\si{\metre\second} = \mbox{metre second}$ \\
%  $\si{\milli\second} = \mbox{millisecond}$
%\end{LaTeXexample}
% There are some circumstances under which it is permissable to omit
% any spaces.  The classic example is \si{\kWh}, where
% ``\si[unitsep=thin]{\kWh}'' does not add any useful information.
% If using such a unit repeatedly, users of \currpkg are advised to
% create a custom unit to ensure consistency.\footnote{\cs{kWh} and
% \cs{kilowatthour} are defined by \currpkg in this way.}
%
% Symbols for units formed from other units by division are indicated
% by means of a virgule (oblique stroke, slash, |/|), a horizontal
% line, or negative exponents.\footnote{Notice that a virgule and a
% solidus are not the same symbol.}  However, to avoid ambiguity, the
% virgule must not be repeated on the same line unless parentheses
% are used. This is ensured when using named unit macros in \currpkg,
% which will ``trap'' repeated division and format it correctly.  In
% complicated cases, negative exponents are to be preferred over
% other formats.
%\begin{LaTeXexample}
%  \si{\joule\per\mole\per\kelvin}\\
%  \si[per=fraction]{\joule\per\mole\per\kelvin}\\
%  \si[per=slash]{\joule\per\mole\per\kelvin}
%\end{LaTeXexample}
%
%\subsection{Repeating units}
% Products and errors should show what unit applies to each value
% given.  Thus \SI{2x3}{\metre} is an ordered set of lengths of a
% geometric area, whereas \SI[repeatunits=false]{2x3}{\metre} is a
% length (and equal to \SI{6}{\metre}). Thus, $\times$ is not a
% product but is a mathematical operator; in the same way, a $2
% \times 3$ matrix is not a $6$ matrix!  In some areas, areas and
% volumes are given with separated units but a unit raised to the
% appropriate power: \SI[repeatunits=power]{2 x 3}{\metre}. Although
% this does display the correct overall units, it is
% potentially-confusing and is not encouraged.
%
%\subsection{Clarity in writing values of quantities}
% Care must be taken when writing ranges of numbers.  For purely
% numerical values, it is common to use a en-dash between values, for
% example ``see pages 1--5''.  On the other hand, values with units
% could be misinterpret as negative values if written in this way. As
% the unit--value combination is a single mathematic entity, writing
% the values with an en-dash followed by a single unit is also
% incorrect.  As a result, using the word ``to'' is strongly
% recommended.
%\begin{LaTeXexample}
%  \SI{1}{\metre} to \SI{5}{\metre} long.
%\end{LaTeXexample}
%
%\subsection{Graphs and tables}
% In graphs and tables, repetition of the units following each entry
% or axis mark is confusing and repetitive.  It is therefore best to
% place the unit in the label part of the information.  Placing the
% unit in square brackets is common but mathematically
% poor.\footnote{For example, for an acceleration $a$, the expression
% $[a]$ is the dimensions of $a$, \ie length per time squared in this
% case.}  Much better is to show division of all values by the unit,
% which leaves the entries as unitless ratios.  This is illustrated
% in \ref{tab:label} and \ref{fig:label}.
%\begin{LaTeXexample}[float]
%  \begin{table}
%    \centering
%    \caption{An example of table labelling}
%    \label{tab:label}
%    \begin{tabular}{cS[tabformat=1.4,tabnumalign=centre]}
%      \toprule
%      {Entry} & {Length/\si{\metre}} \\
%      \midrule
%      1 & 1.1234 \\
%      2 & 1.1425 \\
%      3 & 1.7578 \\
%      4 & 1.9560 \\
%      \bottomrule
%    \end{tabular}
%  \end{table}
%\end{LaTeXexample}
%\begin{LaTeXexample}[float]
%  \begin{figure}
%    \centering
%    \begin{tikzpicture}
%      \begin{axis}[xlabel=$t/\si{\second}$,ylabel=$d/\si{\metre}$]
%        \addplot[smooth,mark=x]
%          plot coordinates {
%            (0,0)
%            (1,5)
%            (2,8)
%            (3,9)
%            (4,8)
%            (5,5)
%            (6,0)
%           };
%      \end{axis}
%    \end{tikzpicture}
%    \caption{An example of graph labelling}
%    \label{fig:label}
%  \end{figure}
%\end{LaTeXexample}
%
% In most cases, adding exponent values in the body of a table is
% less desirable than adding a fixed exponent to column headers.  An
% example is shown in \ref{tab:exp}.  The use of \cs{multicolumn} is
% needed here due to the ``|<|''; without \cs{multicolumn}, the
% titles are followed by ``\si{\kilo\gram}''!
%\begin{LaTeXexample}[float]
%  \begin{table}
%    \centering
%    \caption{Good and bad columns}
%    \label{tab:exp}
%    \sisetup{tabnumalign=centre}
%    \begin{tabular}{%
%      c%
%      S[tabformat=1.3e1]<{\,\si{\kilogram}}%
%      S[tabformat=2.2]}
%      \toprule
%      Entry & \multicolumn{1}{c}{Mass} &
%        {Mass/\SI{e3}{\kilogram}} \\
%      \midrule
%      1 & 4.56e3  & 4.56 \\
%      2 & 2.40e3  & 2.40 \\
%      3 & 1.345e4 & 13.45 \\
%      4 & 4.5e2   & 0.45 \\
%      \bottomrule
%    \end{tabular}
%  \end{table}
%\end{LaTeXexample}
%
%\bibliographystyle{unsrt}
%\StopEventually{%
%  \clearpage
%  \part{Notes}
%  \PrintChanges
%  \PrintIndex
%  \bibliography{siunitx}}
%
% \iffalse
%<*package>
% \fi
%
%\clearpage
%\part{Implementation}
%\section{Main package}
% Much of the code here is taken, with little or no modification,
% from the existing packages.  These are all released under the LPPL,
% and so this use is entirely allowed.  Rather than confuse the
% source here with repeated references, note that code here could be
% copied from \SIstyle, \SIunits, \numprint, \unitsdef or \units.
% Some ideas have also been borrowed from \pkg{biblatex}; again these
% will not be specifically noted. Code from other packages will be
% marked when used.
%
% User-space commands (those not containing |@|) defined here should
% give the same result as macros with the same name in the older
% packages.\footnote{Although extra optional arguments may be added.}
% However, internal package macros will behave differently; if the
% user has redefined internal macros, then compatibility will be
% impaired.
%
% The code used here uses \LaTeX\ rather than \TeX\ commands where
% possible.\footnote{This applies to \LaTeX\ kernel commands only;
% for example, \pkg{ifthenelse} is not used.}  For example,
% \cs{newcommand*} is used in place of \cs{def}, unless custom
% parameters are needed.  Hopefully, this will aid future
% maintenance.  Grouping is used where possible to limit the scope of
% temporary assignments.
%
%\subsection{Setup code}
% As always, the package starts with identification.
%    \begin{macrocode}
\NeedsTeXFormat{LaTeX2e}
\ProvidesPackage{siunitx}
  [2008/12/22 v1.1j A comprehensive (SI) units package]
%    \end{macrocode}
% The package requires \eTeX, so the usual test is made.
%    \begin{macrocode}
\begingroup
  \@ifundefined{eTeXversion}
    {\PackageError{siunitx}
       {Not running under e-TeX}
       {This package requires e-TeX. Try compiling the document
        with\MessageBreak `elatex' instead of `latex'. When
        using pdfTeX, try `pdfelatex'\MessageBreak instead of
        `pdflatex'}%
     \endgroup\endinput}
    {\endgroup}
%    \end{macrocode}
%\begin{macro}{\si@catcodes}
% There are circumstances under which some odd category codes might
% be in place.  The category codes for |{|, |}|, |[|, |]| and |#| are
% assumed to be okay; if issues arise, this can be altered.  \LaTeX\
% will have set |@| as a letter on loading \currpkg.
%    \begin{macrocode}
\edef\si@catcodes{%
  \catcode\string`\string ` \the\catcode\string`\`\relax
  \catcode\string`\string = \the\catcode\string`\=\relax
  \catcode\string`\string ^ \the\catcode\string`\^\relax
  \catcode\string`\string _ \the\catcode\string`\_\relax
  \catcode\string`\string ~ \the\catcode\string`\~\relax
  \catcode\string`\string : \the\catcode\string`\:\relax
  \catcode\string`\string - \the\catcode\string`\-\relax
  \catcode\string`\string + \the\catcode\string`\+\relax
  \catcode\string`\string ; \the\catcode\string`\;\relax
  \catcode\string`\string , \the\catcode\string`\,\relax
  \catcode\string`\string . \the\catcode\string`\.\relax}
\catcode\string`\` 12\relax
\catcode`\= 12\relax
\catcode`\^ 7\relax
\catcode`\_ 8\relax
\catcode`\~ \active\relax
\@makeother{\:}
\@makeother{\-}
\@makeother{\+}
\@makeother{\;}
\@makeother{\,}
\@makeother{\.}
%    \end{macrocode}
%\end{macro}
% Packages needed for functionality are loaded.  \pkg{xkeyval}
% handles the package options, while \pkg{amstext} from the \AMS\
% bundle is needed for \cs{text}.  \pkg{array} is needed for the new
% column type for tabular material.  \pkg{xspace} provides ``magic''
% spacing after macros, if requested.
%    \begin{macrocode}
\RequirePackage{xkeyval}[2005/05/07]
\RequirePackage{amstext,array,xspace}
%    \end{macrocode}
%\begin{macro}{\si@tempa}
%\begin{macro}{\si@tempb}
%\begin{macro}{\si@tempc}
% Some scratch commands are defined; apart from where a known value
% is carried through, these could contain anything.
%    \begin{macrocode}
\newcommand*{\si@tempa}{}
\newcommand*{\si@tempb}{}
\newcommand*{\si@tempc}{}
%    \end{macrocode}
%\end{macro}
%\end{macro}
%\end{macro}
%\begin{macro}{\ifsi@switch}
% Various items will need a switch.  To avoid name pollution, a
% single switch is defined here; grouping will keep the definition
% local.
%    \begin{macrocode}
\newif\ifsi@switch
%    \end{macrocode}
%\end{macro}
%\begin{macro}{\si@tempboxa}
%\begin{macro}{\si@tempboxb}
%\begin{macro}{\si@tempboxc}
%\begin{macro}{\si@tempboxd}
% Some boxes are also needed.
%    \begin{macrocode}
\newbox\si@tempboxa
\newbox\si@tempboxb
\newbox\si@tempboxc
\newbox\si@tempboxd
%    \end{macrocode}
%\end{macro}
%\end{macro}
%\end{macro}
%\end{macro}
%\begin{macro}{\si@temptoks}
% A token register is also handy.
%    \begin{macrocode}
\newtoks{\si@temptoks}
%    \end{macrocode}
%\end{macro}
%\begin{macro}{\si@packagecheck}
%\begin{macro}{\si@blockpkgs}
%\begin{macro}{\si@checkpkgs}
% As \currpkg is intended to replace the other unit-management
% packages, these are tested for before any further processing.  If
% any are loaded, the package halts compilation; name clashes or
% unexpected results could occur if this is not tested.  Notice that
% \SIunits and \SIstyle could be loaded with variable capitalisation
% (at least on Windows); both possibilities are tested.
%    \begin{macrocode}
\newcommand*{\si@blockpkgs}{SIunits,sistyle,siunits,SIstyle,%
  unitsdef,fancyunits}
\newcommand*{\si@checkpkgs}{units,hepunits,fancynum}
\newcommand*{\si@packagecheck}{%
  \begingroup
  \@for\si@tempa:=\si@blockpkgs\do{
    \@ifpackageloaded{\si@tempa}
      {\PackageError{siunitx}
         {Package `\si@tempa' incompatible}
         {The \si@tempa\space package and siunitx are
          incompatible.\MessageBreak Use the
          `emulate=\si@tempa' package option when loading
          siunitx}}
      {}}
%    \end{macrocode}
% Some packages should not cause a clash, but are emulated and would
% be better handled that way.
%    \begin{macrocode}
  \@for\si@tempa:=\si@checkpkgs\do{%
    \@ifpackageloaded{\si@tempa}
      {\PackageWarning{siunitx}
         {Consider loading the siunitx package
          with\MessageBreak option `emulate=\si@tempa', rather
          than\MessageBreak loading both \si@tempa\space and
          siunitx}}
      {}}
  \endgroup}
%    \end{macrocode}
% The check is carried out on loading and at the beginning of the
% document, so that packages loaded both before and after \currpkg
% are caught.
%    \begin{macrocode}
\si@packagecheck
\AtBeginDocument{\si@packagecheck}
%    \end{macrocode}
%\end{macro}
%\end{macro}
%\end{macro}
%\begin{macro}{\si@ifdefinable}
%\darg{macro}
% Using \cs{@ifdefinable} to check macro definitions gives a generic
% error.  To give something more helpful, \cs{@ifundefined} is used,
% but this needs some \cs{expandafter} work.  This way it can also be
% used as a form of \cs{@ifundefined} for macro names.
%    \begin{macrocode}
\newcommand*{\si@ifdefinable}[1]{%
  \expandafter\expandafter\expandafter\@ifundefined%
    \expandafter\expandafter\expandafter%
      {\expandafter\@gobble\string#1}}
%    \end{macrocode}
%\end{macro}
%\begin{macro}{\si@addtolist}
%\darg{macro}
%\darg{items}
% It is quite useful to be able to add to a comma-separated list of
% expandable items.
%    \begin{macrocode}
\newcommand*{\si@addtolist}[2]{%
  \ifx\@empty#1\@empty
    \edef#1{#2}%
  \else
    \edef#1{#1,#2}%
  \fi}
%    \end{macrocode}
%\end{macro}
%\begin{macro}{\si@addtocsname}
%\darg{csname}
%\darg{tokens}
% A second item to add to a command sequence.
%    \begin{macrocode}
\newcommand*{\si@addtocsname}[2]{%
  \@ifundefined{#1}
    {\expandafter\gdef\csname #1\endcsname{#2}}
    {\si@temptoks\expandafter\expandafter\expandafter{%
       \csname #1\endcsname#2}%
     \expandafter\xdef\csname #1\endcsname{\the\si@temptoks}}}
%    \end{macrocode}
%\end{macro}
%\begin{macro}{\si@ifmtarg}
%\begin{macro}{\si@xifmtarg}
%\begin{macro}{\si@ifnotmtarg}
% To keep down dependance on other packages, the very short code
% block from \pkg{ifmtarg} is copied here with an internal name.
%    \begin{macrocode}
\begingroup
  \catcode`\Q=3
  \long\gdef\si@ifmtarg#1{%
    \si@xifmtarg#1QQ\@secondoftwo\@firstoftwo\@nil}
  \long\gdef\si@xifmtarg#1#2Q#3#4#5\@nil{#4}
  \long\gdef\si@ifnotmtarg#1{%
    \si@xifmtarg#1QQ\@firstofone\@gobble\@nil}
\endgroup
%    \end{macrocode}
%\end{macro}
%\end{macro}
%\end{macro}
%\begin{macro}{\si@newrobustcmd}
%\begin{macro}{\si@newcommand}
%\begin{macro}{\si@newcmd}
%\begin{macro}{\si@xargdef}
%\darg{macro}
%\doarg{num-args}
%\doarg{default}
%\darg{def}
% Some more copying, this time from \pkg{etoolbox}.  Various macros
% need to be \emph{really} robust.  This is achieved using the \eTeX\
% \cs{protected} primitive in various places.  However, it would be
% nice to have a \cs{protected} version of \cs{newcommand}.
% \pkg{etoolbox} has code for that, but to avoid needing to load it,
% the necessary stuff is copied here.  The only changes form the
% original are names, and the use of \cs{newcommand*} for the
% \cs{si@newcommand} macro.
%    \begin{macrocode}
\@ifpackageloaded{etoolbox}
  {\let\si@newrobustcmd\newrobustcmd}
  {\protected\def\si@newrobustcmd{%
     \@ifstar
       {\let\l@ngrel@x\protected\si@newcommand}
       {\def\l@ngrel@x{\protected\long}\si@newcommand}}
   \newcommand*{\si@newcommand}[1]{\@testopt{\si@newcmd#1}0}
   \def\si@newcmd#1[#2]{%
     \@ifnextchar[%]
       {\si@xargdef#1[#2]}
       {\@argdef#1[#2]}}
   \long\def\si@xargdef#1[#2][#3]#4{%
     \@ifdefinable#1{%
       \expandafter\protected
       \expandafter\def
       \expandafter#1%
       \expandafter{%
         \expandafter\@testopt
         \csname\string#1\endcsname{#3}}%
       \expandafter\@yargdef
       \csname\string#1\endcsname\tw@{#2}{#4}}}}
%    \end{macrocode}
%\end{macro}
%\end{macro}
%\end{macro}
%\end{macro}
%
%\subsection{Logging}
%\begin{macro}{\ifsi@debug}
%\begin{macro}{\ifsi@logmin}
%\begin{macro}{\ifsi@lognone}
% To control logging, some new switches are declared.
%    \begin{macrocode}
\newif\ifsi@debug
\newif\ifsi@logmin
\newif\ifsi@lognone
%    \end{macrocode}
%\end{macro}
%\end{macro}
%\end{macro}
%\begin{macro}{\si@log@err}
%\darg{error}
%\darg{explanation}
% Some handy re-usable macros are defined here.  These all take names
% beginning  These pop up in various places.  First errors are
% handled.
%    \begin{macrocode}
\newcommand*{\si@log@err}[2]{%
  \ifsi@lognone\else
    \ifsi@logmin
      \PackageWarning{siunitx}{#1}%
    \else
      \PackageError{siunitx}{#1}{#2}%
    \fi
  \fi}
%    \end{macrocode}
%\end{macro}
%\begin{macro}{\si@log@warn}
%\begin{macro}{\si@log@inf}
%\darg{text}
% Then warnings and information.
%    \begin{macrocode}
\newcommand*{\si@log@warn}[1]{%
  \ifsi@lognone\else
    \ifsi@logmin\else
      \PackageWarning{siunitx}{#1}%
    \fi
  \fi}
\newcommand*{\si@log@inf}[1]{%
  \ifsi@lognone\else
    \ifsi@logmin\else
      \PackageInfo{siunitx}{#1}%
    \fi
  \fi}
%    \end{macrocode}
%\end{macro}
%\end{macro}
%\begin{macro}{\si@log@debug}
%\darg{debug-info}
% The debug macro only gives output if the appropriate package option
% is set.
%    \begin{macrocode}
\newcommand*{\si@log@debug}[1]{%
  \ifsi@lognone\else
    \ifsi@debug
      \PackageInfo{siunitx}{#1}%
    \fi
  \fi}
%    \end{macrocode}
%\end{macro}
%
%\subsection{String comparison}
%\begin{macro}{\si@str@ifchrstr}
%\begin{macro}{\si@str@chrstr}
%\darg{char}
%\darg{string}
% At various points, the package needs to compare two strings, to
% find if one occurs in the other.  The first test is if a single
% character is part of a second string; this is used, for example, to
% check that a character is valid as input.  The first argument is
% not expanded further, but the second is two allow division into
% individual units.
%    \begin{macrocode}
\newcommand*{\si@str@ifchrstr}[2]{%
  \begingroup
    \si@switchfalse
    \renewcommand*{\si@tempa}{#1}%
    \protected@edef\si@tempb{#2}%
    \expandafter\si@str@chrstr\si@tempb\@empty\@empty\@empty
    \ifsi@switch
      \aftergroup\@firstoftwo
    \else
      \aftergroup\@secondoftwo
    \fi
  \endgroup}
\def\si@str@chrstr#1#2\@empty{%
  \renewcommand*{\si@tempc}{#1}%
  \ifx\si@tempa\si@tempc
    \expandafter\si@switchtrue
  \else
    \ifx\@empty#2\@empty\else
      \si@str@chrstr#2\@empty\@empty
    \fi
  \fi}
%    \end{macrocode}
%\end{macro}
%\end{macro}
%\begin{macro}{\si@str@ifonlychrs}
%\begin{macro}{\si@str@onlychrs}
%\darg{string}
%\darg{chars}
% The second test builds on the first.  Here, a check is made to see
% if the first string is made up only of characters from the second
% string.  In this case, the first string is expanded before
% testing.  The second string will be expanded by the internal
% character by character test.
%    \begin{macrocode}
\newcommand*{\si@str@ifonlychrs}[2]{%
  \begingroup
    \si@switchtrue
    \protected@edef\si@tempa{#1}%
    \renewcommand*{\si@tempb}{#2}%
    \expandafter\si@str@onlychrs\si@tempa\@empty\@empty\@empty
    \ifsi@switch
      \aftergroup\@firstoftwo
    \else
      \aftergroup\@secondoftwo
    \fi
  \endgroup}
\def\si@str@onlychrs#1#2\@empty{%
  \si@str@ifchrstr{#1}{\si@tempb}
    {}{\si@switchfalse}%
  \ifx\@empty#2\@empty\else
    \si@str@onlychrs#2\@empty\@empty
  \fi}
%    \end{macrocode}
%\end{macro}
%\end{macro}
%
%\subsection{Option handling}
%\begin{macro}{\sisetup}
%\darg{options}
% To allow modification of options at run time, a setup macro is
% provided.
%    \begin{macrocode}
\newcommand*{\sisetup}{\setkeys[si]{key}}
%    \end{macrocode}
%\end{macro}
%\begin{macro}{\si@opt@key}
%\darg{keyname}
%\darg{code}
% To aid maintenance, some shortcuts are defined for generating keys.
% These also allow the debugging messages to be added automatically
% to every key.  First of all the basic key definition.
%    \begin{macrocode}
\newcommand*{\si@opt@key}[2]{%
  \define@key[si]{key}{#1}
  {#2\si@log@debug{Option #1 set to ##1}}}
%    \end{macrocode}
%\end{macro}
%\begin{macro}{\si@opt@cmdkey}
%\doarg{default}
%\darg{keyname}
%\darg{function}
% For a single command key, a function must be specified.
%    \begin{macrocode}
\newcommand*{\si@opt@cmdkey}[3][]{%
  \define@cmdkey[si]{key}[si@]{#2}[#1]{#3}}
%    \end{macrocode}
%\end{macro}
%\begin{macro}{\si@opt@cmdkeys}
%\doarg{default}
%\darg{keyname}
% Whereas multiple definitions do not take a function.
%    \begin{macrocode}
\newcommand*{\si@opt@cmdkeys}[2][]{%
  \define@cmdkeys[si]{key}[si@]{#2}[#1]}
%    \end{macrocode}
%\end{macro}
%\begin{macro}{\si@opt@boolkey}
%\doarg{function}
%\darg{keyname}
% Keys which only take switch values; anything other than \opt{true}
% or \opt{false} will generate a warning from \pkg{xkeyval}.
%    \begin{macrocode}
\newcommand*{\si@opt@boolkey}[2][]{%
  \define@boolkey[si]{key}[si@]{#2}[true]
  {#1\si@log@debug{Option #2 set to ##1}}}
%    \end{macrocode}
%\end{macro}
%\begin{macro}{\si@opt@choicekey}
%\doarg{default}
%\darg{keyname}
%\darg{choices}
%\darg{function}
% A ``fill in the blanks'' choice key.  In all cases, \cs{si@tempa}
% is used to hold the value given to the key, so that \cs{ifx}
% testing can occur.
%    \begin{macrocode}
\newcommand*{\si@opt@choicekey}[4][]{%
  \define@choicekey*+[si]{key}{#2}[\si@tempa]{#3}[#1]
    {#4\si@log@debug{Option #2 set to ##1}}
    {\si@log@warn{Unknown value `##1' for option #2}}}
%    \end{macrocode}
%\end{macro}
%\begin{macro}{\si@opt@xchoicekey}
%\darg{keyname}
%\darg{choices}
%\darg{initial-value}
% Several of the package options can take either a choice from a list
% of known options, or a value to be interpreted literally.  To aid
% maintenance, the necessary code can be set up here.  These keys all
% define a new macro, which must exist.  The \cs{si@opt@xchoicekey}
% macro therefore ensures that this is defined, as well as setting up
% the \pkg{xkeyval} key.
%    \begin{macrocode}
\newcommand*{\si@opt@xchoicekey}[3]{%
  \define@choicekey*+[si]{key}{#1}[\si@tempa]{#2}[#1]
%    \end{macrocode}
% This code will execute if the option is on the list.  There will be
% a ``fixed'' macro with a matching name, which is used for this.
%    \begin{macrocode}
    {\si@log@debug{Option #1 set to ##1}%
     \expandafter\renewcommand\expandafter*\expandafter{%
      \csname si@#1\endcsname}{\@nameuse{si@fix@##1}}}
%    \end{macrocode}
% The user has given something that is not on the list as an
% argument.  It is used literally.
%    \begin{macrocode}
    {\si@log@debug{Option #1 set to ##1}%
     \expandafter\renewcommand\expandafter*\expandafter{%
      \csname si@#1\endcsname}{##1}}
%    \end{macrocode}
% Finally, the initial value of the macro is set up.
%    \begin{macrocode}
  \expandafter\newcommand\expandafter*\expandafter{%
    \csname si@#1\endcsname}%
    {\@nameuse{si@fix@#3}}}
%    \end{macrocode}
%\end{macro}
%\begin{macro}{\si@opt@compatkey}
%\darg{package}
%\darg{keyname}
% An all-in-one definition for a back-compatibility key.  These
% should only be used at load time, so are automatically disabled
% once the package is loaded.  Emulation is also automatically turned
% on.
%    \begin{macrocode}
\newcommand*{\si@opt@compatkey}[2]{%
  \define@boolkey[si]{key}[si@old@]{#2}[true]
  {\si@log@debug{Emulating #1 package option\MessageBreak #2}%
   \sisetup{emulate=#1}%
   \si@log@debug{Option #2 set to ##1}}
   \AtEndOfPackage{\si@opt@disablekey{#2}
     {Compatibility option #2 only\MessageBreak
        available when loading siunitx package}}}
%    \end{macrocode}
%\end{macro}
%\begin{macro}{\si@opt@disablekey}
%\darg{keyname}
%\darg{text}
% The ability to disable a key with a meaningful message is a must;
% the warning will come from \currpkg, and not from \pkg{xkeyval}.
%    \begin{macrocode}
\newcommand*{\si@opt@disablekey}[2]{%
  \key@ifundefined[si]{key}{#1}
    {}
    {\si@log@debug{Disabling key #1}%
     \si@opt@key{#1}{\si@log@warn{#2}}}}
%    \end{macrocode}
%\end{macro}
%
% The \pkg{xkeyval} package option for logging is declared. This is
% then processed to set the switches correctly.
%    \begin{macrocode}
\si@opt@choicekey[normal]{log}{debug,verbose,normal,errors,none}
%    \end{macrocode}
% A series of comparisons are made to assign the logging mode.  The
% \opt{normal} option is not tested, as executing the option sets the
% switches appropriately.
%    \begin{macrocode}
  {\si@debugfalse
   \si@logminfalse
   \si@lognonefalse
   \renewcommand*{\si@tempb}{none}%
   \ifx\si@tempa\si@tempb
     \si@lognonetrue
   \fi
   \renewcommand*{\si@tempb}{minimal}%
   \ifx\si@tempa\si@tempb
     \si@logmintrue
   \fi
   \renewcommand*{\si@tempb}{debug}%
   \ifx\si@tempa\si@tempb
     \si@debugtrue
   \fi
   \renewcommand*{\si@tempb}{verbose}%
   \ifx\si@tempa\si@tempb
     \si@debugtrue
   \fi}
%    \end{macrocode}
% A quick method to set \opt{log=debug}.
%    \begin{macrocode}
\si@opt@boolkey{debug}
%    \end{macrocode}
%\begin{macro}{\ifsi@strict}
% It would be useful to be able to disable some keys, when strict
% interpretation of the rules is desired.  This is a load-time
% option, and has to disable various options.
%    \begin{macrocode}
\si@opt@boolkey[%
  \ifsi@strict
    \sisetup{
      obeymode=false,
      obeybold=false,
      obeyitalic=false,
      mode=maths,
      repeatunits=true,
      trapambigerr=true,
      trapambigfrac=true}
    \@for\si@tempa:=obeyall,obeymode,obeyitalic,mode,unitmode,%
      valuemode,textmode,obeybold,repeatunits,trapambigerr,%
      trapambigfrac\do{%
      \begingroup
        \edef\si@tempb{\endgroup
          \noexpand\si@opt@disablekey{\si@tempa}
          {Option `\si@tempa' forbidden in strict mode}}%
      \si@tempb}
  \fi]{strict}
\AtEndOfPackage{
  \si@opt@disablekey{strict}
    {Option `strict' only available when\MessageBreak
     loading package}}
%    \end{macrocode}
%\end{macro}
%\begin{macro}{\si@emulate}
% The \opt{emulate} option is used for back-compatibility mode; the
% option is only valid when loading \currpkg.
%    \begin{macrocode}
\newcommand*{\si@emulate}{}
\si@opt@key{emulate}{\si@addtolist{\si@emulate}{#1}}
\AtEndOfPackage{
  \si@opt@disablekey{emulate}
    {Option `emulate' only available when\MessageBreak
     loading package}}
%    \end{macrocode}
%\end{macro}
%\begin{macro}{\si@unitsep}
%\begin{macro}{\si@unitspace}
%\begin{macro}{\si@errspace}
%\begin{macro}{\si@valuesep}
% The two \opt{\dots sep} options control the size of spaces between
% the number and the unit (\cs{si@valuesep}), and that used to
% represent a product (\cs{si@unitsep}).  Known values here are
% \opt{thin}, \opt{med}, \opt{medium}, \opt{thick}, \opt{cdot},
% \opt{tightcdot}\footnote{Both \cs{cdot}-based options only valid
% for \opt{unitsep}.} and \opt{none};\footnote{Only valid for
% \opt{valuesep}.} other entries will be treated as custom spaces.
%    \begin{macrocode}
\si@opt@xchoicekey{unitsep}
  {thin,med,medium,thick,none,comma,stop,fullstop,period,
    times,tighttimes,cdot,tightcdot}{thin}
\si@opt@xchoicekey{unitspace}{space,thin,med,medium,thick,
  none}{thin}
\si@opt@xchoicekey{errspace}{space,thin,med,medium,thick,
  none}{none}
\si@opt@xchoicekey{valuesep}
  {thin,med,medium,thick,none,comma,stop,fullstop,period,
    times,tighttimes,cdot,tightcdot}{thin}
%    \end{macrocode}
%\end{macro}
%\end{macro}
%\end{macro}
%\end{macro}
%\begin{macro}{\si@digitsep}
% Separation of digits in large numbers is controlled by the
% \opt{digitsep} option.  As with the other \opt{sep} values, this
% one has a choice of possible values.  The list is quite long, so
% that a range of options are handled automatically.  Notice that
% \opt{digitsep=none} will be used for no separation at all.
%    \begin{macrocode}
\si@opt@xchoicekey{digitsep}
  {thin,med,medium,thick,none,comma,stop,fullstop,period,
    times,tighttimes,cdot,tightcdot}{thin}
%    \end{macrocode}
%\end{macro}
%\begin{macro}{\si@decimalsymbol}
% The symbol used for the decimal position is varied here.  There are
% only two real options, but options are given for the name of a full
% stop.
%    \begin{macrocode}
\si@opt@xchoicekey{decimalsymbol}{comma,stop,fullstop,period,
  cdot,tightcdot}{fullstop}
%    \end{macrocode}
%\end{macro}
%\begin{macro}{\si@anglesep}
% The separator between degrees and minutes, and between minutes and
% seconds, when using \cs{ang}.
%    \begin{macrocode}
\si@opt@xchoicekey{anglesep}
    {thin,med,medium,thick,none,comma,stop,fullstop,period,
      times,tighttimes,cdot,tightcdot}{none}
%    \end{macrocode}
%\end{macro}
%\begin{macro}{\ifsi@obeymode}
% The first test for the font control is whether to respect the
% surrounding maths or text mode.
%    \begin{macrocode}
\si@opt@boolkey{obeymode}
%    \end{macrocode}
%\end{macro}
%\begin{macro}{\ifsi@numtextmode}
%\begin{macro}{\ifsi@unittextmode}
% The output of the package can be typeset using either text or maths
% mode fonts. Two switches are needed, for numbers and units.
%    \begin{macrocode}
\newif\ifsi@numtextmode
\newif\ifsi@unittextmode
%    \end{macrocode}
%\end{macro}
%\end{macro}
% The \opt{textmode} option has to set both flags.
%    \begin{macrocode}
\si@opt@choicekey[true]{textmode}{true,false}
  {\si@numtextmodefalse
   \si@unittextmodefalse
   \renewcommand*{\si@tempb}{true}%
   \ifx\si@tempa\si@tempb
     \si@numtextmodetrue
     \si@unittextmodetrue
   \fi}
%    \end{macrocode}
% The \opt{mode} option applies to numbers and units.
%    \begin{macrocode}
\si@opt@choicekey{mode}{math,maths,text}
  {\si@numtextmodefalse
   \si@unittextmodefalse
   \renewcommand*{\si@tempb}{text}%
   \ifx\si@tempa\si@tempb
     \si@numtextmodetrue
     \si@unittextmodetrue
   \fi}
%    \end{macrocode}
% Now the two options for numbers or units alone.
%    \begin{macrocode}
\si@opt@choicekey{valuemode}{math,maths,text}
  {\si@numtextmodefalse
   \renewcommand*{\si@tempb}{text}%
   \ifx\si@tempa\si@tempb
     \si@numtextmodetrue
   \fi}
\si@opt@choicekey{unitmode}{math,maths,text}
  {\si@unittextmodefalse
   \renewcommand*{\si@tempb}{text}%
   \ifx\si@tempa\si@tempb
     \si@unittextmodetrue
   \fi}
%    \end{macrocode}
%\begin{macro}{\ifsi@obeyfamily}
% The package can work to match the font family (serif, sans serif,
% typewriter) of the surrounding text.  This is controlled by a
% Boolean option.
%    \begin{macrocode}
\si@opt@boolkey{obeyfamily}
%    \end{macrocode}
%\end{macro}
%\begin{macro}{\ifsi@obeybold}
% The package can attempt to respect bold, or may ignore it.
%    \begin{macrocode}
\si@opt@boolkey{obeybold}
%    \end{macrocode}
%\end{macro}
%\begin{macro}{\ifsi@inlinebtext}
% For inline maths, two options for checking what is bold are
% available, the maths environment (\ie \cs{boldmath}) and the
% surrounding text (\cs{textbf} or \cs{bffamily}).
%    \begin{macrocode}
\newif\ifsi@inlinebtext
\si@opt@choicekey{inlinebold}{text,maths,math}
  {\si@inlinebtextfalse
   \renewcommand*{\si@tempb}{text}%
   \ifx\si@tempa\si@tempb
     \si@inlinebtexttrue
   \fi}
%    \end{macrocode}
%\end{macro}
%\begin{macro}{\ifsi@obeyitalic}
% Italic is slightly different to bold, as there is no convenient
% switch for maths.  Thus a choice key is used, with the appropriate
% check.
%    \begin{macrocode}
\si@opt@boolkey{obeyitalic}
%    \end{macrocode}
%\end{macro}
%\begin{macro}{\ifsi@detectdisplay}
% For handling display mathematics, a setting is needed for whether to
% treat it differently from other maths.
%    \begin{macrocode}
\si@opt@boolkey{detectdisplay}
%    \end{macrocode}
%\end{macro}
% The option to obey all font switching commands is Boolean-like but
% needs alternative setup.
%    \begin{macrocode}
\si@opt@choicekey[true]{obeyall}{true,false}
  {\si@obeyboldfalse
   \si@obeyitalicfalse
   \si@obeymodefalse
   \si@obeyfamilyfalse
   \renewcommand*{\si@tempb}{true}%
   \ifx\si@tempa\si@tempb
     \si@obeyboldtrue
     \si@obeyitalictrue
     \si@obeymodetrue
     \si@obeyfamilytrue
   \fi}
%    \end{macrocode}
%\begin{macro}{\si@valuemathsrm}
%\begin{macro}{\si@valuemathssf}
%\begin{macro}{\si@valuemathstt}
%\begin{macro}{\si@unitmathsrm}
%\begin{macro}{\si@unitmathssf}
%\begin{macro}{\si@unitmathstt}
% The fonts used by the package default to the obvious \LaTeX\ ones;
% however, this needs to be exposed to user modification.  First the
% maths mode fonts are sorted out.
%    \begin{macrocode}
\si@opt@cmdkeys{valuemathsrm,valuemathssf,valuemathstt}
\si@opt@cmdkeys{unitmathsrm,unitmathssf,unitmathstt}
%    \end{macrocode}
%\end{macro}
%\end{macro}
%\end{macro}
%\end{macro}
%\end{macro}
%\end{macro}
% To make life easier for the user, US spellings are provided for the
% maths keys.
%    \begin{macrocode}
\si@opt@key{valuemathrm}{\sisetup{valuemathsrm=#1}}
\si@opt@key{valuemathsf}{\sisetup{valuemathssf=#1}}
\si@opt@key{valuemathtt}{\sisetup{valuemathstt=#1}}
\si@opt@key{unitmathrm}{\sisetup{unitmathsrm=#1}}
\si@opt@key{unitmathsf}{\sisetup{unitmathssf=#1}}
\si@opt@key{unitmathtt}{\sisetup{unitmathstt=#1}}
%    \end{macrocode}
% The combined options are given, for setting numbers and units
% at the same time.
%    \begin{macrocode}
\si@opt@key{mathsrm}{\sisetup{valuemathsrm=#1,unitmathsrm=#1}}
\si@opt@key{mathssf}{\sisetup{valuemathssf=#1,unitmathssf=#1}}
\si@opt@key{mathstt}{\sisetup{valuemathstt=#1,unitmathstt=#1}}
\si@opt@key{mathrm}{\sisetup{valuemathsrm=#1,unitmathsrm=#1}}
\si@opt@key{mathsf}{\sisetup{valuemathssf=#1,unitmathssf=#1}}
\si@opt@key{mathtt}{\sisetup{valuemathstt=#1,unitmathstt=#1}}
%    \end{macrocode}
%\begin{macro}{\si@valuetextrm}
%\begin{macro}{\si@valuetextsf}
%\begin{macro}{\si@valuetexttt}
%\begin{macro}{\si@unittextrm}
%\begin{macro}{\si@unittextsf}
%\begin{macro}{\si@unittexttt}
% The same thing for text mode fonts.  Once again the default values
% are pretty obvious.
%    \begin{macrocode}
\si@opt@cmdkeys{valuetextrm,valuetextsf,valuetexttt}
\si@opt@cmdkeys{unittextrm,unittextsf,unittexttt}
\si@opt@key{textrm}{\sisetup{unittextrm=#1,valuetextrm=#1}}
\si@opt@key{textsf}{\sisetup{unittextsf=#1,valuetextsf=#1}}
\si@opt@key{texttt}{\sisetup{unittexttt=#1,valuetexttt=#1}}
%    \end{macrocode}
%\end{macro}
%\end{macro}
%\end{macro}
%\end{macro}
%\end{macro}
%\end{macro}
%\changes{v1.1}{2008/09/22}{Added \opt{numdiv} option}
%\begin{macro}{\si@numdigits}
%\begin{macro}{\si@numdiv}
%\begin{macro}{\si@numdecimal}
%\begin{macro}{\si@numexp}
%\begin{macro}{\si@numprod}
%\begin{macro}{\si@numgobble}
%\begin{macro}{\si@numsign}
%\begin{macro}{\si@numcloseerr}
%\begin{macro}{\si@numopenerr}
%\begin{macro}{\si@numaddn}
% The list of possible valid characters for parsing numbers is set
% up.  This is similar to \numprint, but with the \opt{extra} class,
% and with characters ignored with no output renamed as \opt{gobble}.
%    \begin{macrocode}
\si@opt@cmdkeys{numdigits,numdecimal,numexp,numgobble,numsign,%
  numcloseerr,numopenerr,numaddn,numprod,numdiv}
%    \end{macrocode}
%\end{macro}
%\end{macro}
%\end{macro}
%\end{macro}
%\end{macro}
%\end{macro}
%\end{macro}
%\end{macro}
%\end{macro}
%\end{macro}
%\begin{macro}{\si@numvalid}
%\begin{macro}{\si@numextra}
% The various valid characters are collected together in a single
% macro for later.  In common with the above macros, this one starts
% \cs{si@num\ldots}.  The order here is the order the values are
% tested later on.
%    \begin{macrocode}
\newcommand*{\si@numextra}{%
  \si@numopenerr\si@numcloseerr\si@numaddn}
\newcommand*{\si@numvalid}{\si@numgobble\si@numexp\si@numsign
  \si@numdecimal\si@numdigits\si@numextra\si@numprod\si@numdiv}
%    \end{macrocode}
%\end{macro}
%\end{macro}
%\begin{macro}{\ifsi@seperr}
%\begin{macro}{\ifsi@trapambigerr}
%\begin{macro}{\si@openerr}
%\begin{macro}{\si@closeerr}
% An option to control whether numerical error values are printed with
% or separate from the number.
%    \begin{macrocode}
\si@opt@boolkey{seperr}
\si@opt@boolkey{trapambigerr}
\si@opt@cmdkeys{openerr,closeerr}
%    \end{macrocode}
%\end{macro}
%\end{macro}
%\end{macro}
%\end{macro}
%\begin{macro}{\ifsi@sepfour}
% With four digits in a number, separating may or may not be desired.
% Note that this option is the same as one for \numprint.
%    \begin{macrocode}
\si@opt@boolkey{sepfour}
%    \end{macrocode}
%\end{macro}
%\begin{macro}{\ifsi@retainplus}
% An option to keep an explicit positive sign.
%    \begin{macrocode}
\si@opt@boolkey{retainplus}
%    \end{macrocode}
%\end{macro}
%\begin{macro}{\si@expbase}
%\begin{macro}{\si@expproduct}
% The options for exponents are set up.
%    \begin{macrocode}
\si@opt@xchoicekey{expproduct}{times,tighttimes,
  cdot,tightcdot}{times}
\si@opt@xchoicekey{expbase}{ten}{ten}
%    \end{macrocode}
%\end{macro}
%\end{macro}
%\begin{macro}{\ifsi@allowzeroexp}
% The package normally prevents \num[allowzeroexp]{e0}.
%    \begin{macrocode}
\si@opt@boolkey{allowzeroexp}
%    \end{macrocode}
%\end{macro}
%\begin{macro}{\si@prefixproduct}
% The marker for multiplication in prefixes.
%    \begin{macrocode}
\si@opt@xchoicekey{prefixproduct}{times,tighttimes,cdot,
  tightcdot,none}{times}
%    \end{macrocode}
%\end{macro}
%\begin{macro}{\si@prefixbase}
% In the same area, the power for prefixes is variable.  Here, two
% choices ar needed.
%    \begin{macrocode}
\si@opt@xchoicekey{prefixbase}{ten,two}{ten}
%    \end{macrocode}
%\end{macro}
%\begin{macro}{\ifsi@prefixsymbolic}
% Unit prefixes can be given as either symbols or numerically.
%    \begin{macrocode}
\si@opt@boolkey{prefixsymbolic}
%    \end{macrocode}
%\end{macro}
%\begin{macro}{\ifsi@num@padlead}
%\begin{macro}{\ifsi@num@padtrail}
% A setting is needed to indicate when to add zeros to decimal
% numbers, either before the decimal marker (|.1| giving
% ``\num[padnumber=all]{.1}'') or after (|1.| giving
% ``\num[padnumber=all]{1.}'').
%    \begin{macrocode}
\newif\ifsi@num@padlead
\newif\ifsi@num@padtrail
\si@opt@choicekey[all]{padnumber}
  {leading,lead,trailing,trail,all,both,true,none,false}
  {\si@num@padleadfalse
   \si@num@padtrailfalse
   \renewcommand*{\si@tempb}{leading}%
   \ifx\si@tempa\si@tempb
     \si@num@padleadtrue
   \fi
   \renewcommand*{\si@tempb}{lead}%
   \ifx\si@tempa\si@tempb
     \si@num@padleadtrue
   \fi
   \renewcommand*{\si@tempb}{trailing}%
   \ifx\si@tempa\si@tempb
     \si@num@padtrailtrue
   \fi
   \renewcommand*{\si@tempb}{trail}%
   \ifx\si@tempa\si@tempb
     \si@num@padtrailtrue
   \fi
   \renewcommand*{\si@tempb}{all}%
   \ifx\si@tempa\si@tempb
     \si@num@padleadtrue
     \si@num@padtrailtrue
   \fi
   \renewcommand*{\si@tempb}{true}%
   \ifx\si@tempa\si@tempb
     \si@num@padleadtrue
     \si@num@padtrailtrue
   \fi
   \renewcommand*{\si@tempb}{both}%
   \ifx\si@tempa\si@tempb
     \si@num@padleadtrue
     \si@num@padtrailtrue
   \fi}
%    \end{macrocode}
%\end{macro}
%\end{macro}
%\begin{macro}{\si@sign}
%\begin{macro}{\ifsi@num@signmant}
%\begin{macro}{\ifsi@num@signexp}
% Some new switches for adding signs to numbers
%    \begin{macrocode}
\newif\ifsi@num@signmant
\newif\ifsi@num@signexp
%    \end{macrocode}
%\end{macro}
%\end{macro}
% Signs can be added to numbers by default. Two options are needed
% here; whether to add a sign by default, and what the sign is.
%    \begin{macrocode}
\si@opt@xchoicekey{sign}{plus,minus,pm,mp}{plus}
\si@opt@choicekey[all]{addsign}
  {mantissa,exponent,mant,exp,all,both,true,none,false}
%    \end{macrocode}
%\end{macro}
% The option is now processed.
%    \begin{macrocode}
  {\si@num@signmantfalse
   \si@num@signexpfalse
   \renewcommand*{\si@tempb}{mantissa}%
   \ifx\si@tempa\si@tempb
     \si@num@signmanttrue
   \fi
   \renewcommand*{\si@tempb}{mant}%
   \ifx\si@tempa\si@tempb
     \si@num@signmanttrue
   \fi
   \renewcommand*{\si@tempb}{exponent}%
   \ifx\si@tempa\si@tempb
     \si@num@signexptrue
   \fi
   \renewcommand*{\si@tempb}{exp}%
   \ifx\si@tempa\si@tempb
     \si@num@signexptrue
   \fi
   \renewcommand*{\si@tempb}{all}%
   \ifx\si@tempa\si@tempb
     \si@num@signmanttrue
     \si@num@signexptrue
   \fi
   \renewcommand*{\si@tempb}{true}%
   \ifx\si@tempa\si@tempb
     \si@num@signmanttrue
     \si@num@signexptrue
   \fi
   \renewcommand*{\si@tempb}{both}%
   \ifx\si@tempa\si@tempb
     \si@num@signmanttrue
     \si@num@signexptrue
   \fi}
%    \end{macrocode}
%\begin{macro}{\ifsi@tightpm}
%\begin{macro}{\si@pm}
% To reduce spacing, it might be necessary to use a ``tight'' $\pm$
% sign.
%    \begin{macrocode}
\si@opt@boolkey{tightpm}
\newcommand*{\si@pm}{%
  \ifsi@tightpm
    \si@fix@tightpm
  \else
    \si@fix@pm
  \fi}
%    \end{macrocode}
%\end{macro}
%\end{macro}
%\begin{macro}{\ifsi@ang@padsmall}
%\begin{macro}{\ifsi@ang@padlarge}
% A switch for determining whether to typeset |\ang{;;1}| as
% \ang[padangle=both]{;;1} or \ang[padangle=none]{;;1}.  First, two
% new Boolean switches are needed to indicate padding.
%    \begin{macrocode}
\newif\ifsi@ang@padsmall
\newif\ifsi@ang@padlarge
\si@opt@choicekey[all]{padangle}
  {small,large,all,both,true,none,false}
  {\si@ang@padsmallfalse
   \si@ang@padlargefalse
   \renewcommand*{\si@tempb}{small}%
   \ifx\si@tempa\si@tempb
     \si@ang@padsmalltrue
   \fi
   \renewcommand*{\si@tempb}{large}%
   \ifx\si@tempa\si@tempb
     \si@ang@padlargetrue
   \fi
   \renewcommand*{\si@tempb}{all}%
   \ifx\si@tempa\si@tempb
     \si@ang@padsmalltrue
     \si@ang@padlargetrue
   \fi
   \renewcommand*{\si@tempb}{true}%
   \ifx\si@tempa\si@tempb
     \si@ang@padsmalltrue
     \si@ang@padlargetrue
   \fi
   \renewcommand*{\si@tempb}{both}%
   \ifx\si@tempa\si@tempb
     \si@ang@padsmalltrue
     \si@ang@padlargetrue
   \fi}
%    \end{macrocode}
%\end{macro}
%\end{macro}
%\begin{macro}{\ifsi@ang@toarc}
%\begin{macro}{\ifsi@ang@todec}
% To control whether angles are formatted as decimals or
% degrees--minutes--seconds, a package option plus two switches are
% needed.  The later format is referred to as ``\opt{arc}'' most
% readily.  An option to leave the input unchanged is also provided.
%    \begin{macrocode}
\newif\ifsi@ang@toarc
\newif\ifsi@ang@todec
\si@opt@choicekey[all]{angformat}
  {dec,decimal,arc,dms,unchanged,none}
  {\si@ang@toarcfalse
   \si@ang@todecfalse
   \renewcommand*{\si@tempb}{dec}%
   \ifx\si@tempa\si@tempb
     \si@ang@todectrue
   \fi
   \renewcommand*{\si@tempb}{decimal}%
   \ifx\si@tempa\si@tempb
     \si@ang@todectrue
   \fi
   \renewcommand*{\si@tempb}{arc}%
   \ifx\si@tempa\si@tempb
     \si@ang@toarctrue
   \fi
   \renewcommand*{\si@tempb}{dms}%
   \ifx\si@tempa\si@tempb
     \si@ang@toarctrue
   \fi}
%    \end{macrocode}
%\end{macro}
%\end{macro}
%\begin{macro}{\ifsi@astroang}
% A slightly odd option to allow the method used by astronomers for
% angles.
%    \begin{macrocode}
\si@opt@boolkey{astroang}
%    \end{macrocode}
%\end{macro}
%\begin{macro}{\ifsi@strictarc}
% For the \cs{ang} macro, the default is to require two semi-colons
% in the input for arc angles.  This is controlled here.
%    \begin{macrocode}
\si@opt@boolkey{strictarc}
%    \end{macrocode}
%\end{macro}
%\changes{v1.1}{2008/09/03}{Added \opt{tabparseonly} option}
%\begin{macro}{\ifsi@tab@fixed}
%\begin{macro}{\ifsi@tabparseonly}
%\begin{macro}{\si@tabnumalign}
%\begin{macro}{\si@tab@rfill@S}
%\begin{macro}{\si@tab@lfill@S}
% To determine the control of table alignment, two options are
% provided.  The \opt{tabnumalign} option controls which centring method
% is used, and the fills used for achieving this.
%    \begin{macrocode}
\newif\ifsi@tab@fixed
\si@opt@boolkey{tabparseonly}
\si@opt@choicekey{tabnumalign}
  {centredecimal,centerdecimal,right,left,centre,center}
  {\si@tab@fixedtrue
   \def\si@tab@rfill@S{0.5}%
   \def\si@tab@lfill@S{0.5}%
   \renewcommand*{\si@tempb}{right}%
   \ifx\si@tempa\si@tempb
     \def\si@tab@rfill@S{0}%
     \def\si@tab@lfill@S{1}%
   \fi
   \renewcommand*{\si@tempb}{left}%
   \ifx\si@tempa\si@tempb
     \def\si@tab@rfill@S{1}%
     \def\si@tab@lfill@S{0}%
   \fi
   \renewcommand*{\si@tempb}{centredecimal}%
   \ifx\si@tempa\si@tempb
     \expandafter\si@tab@fixedfalse
   \fi
   \renewcommand*{\si@tempb}{centerdecimal}%
   \ifx\si@tempa\si@tempb
     \expandafter\si@tab@fixedfalse
   \fi}
\si@opt@key{tabalign}{\sisetup{tabnumalign=#1,tabtextalign=#1,
  tabunitalign=#1}}
%    \end{macrocode}
%\end{macro}
%\end{macro}
%\end{macro}
%\end{macro}
%\end{macro}
%\begin{macro}{\ifsi@tabalignexp}
% A switch for aligning exponents.
%    \begin{macrocode}
\si@opt@boolkey{tabalignexp}
\si@opt@boolkey[%
  \si@tabalignexpfalse
  \ifsi@tabexpalign
    \expandafter\si@tabalignexptrue
  \fi
  ]{tabexpalign}
%    \end{macrocode}
%\end{macro}
%\begin{macro}{\si@tab@mantprecnt}
%\begin{macro}{\si@tab@mantpostcnt}
%\begin{macro}{\si@tab@expprecnt}
%\begin{macro}{\si@tab@exppostcnt}
%\begin{macro}{\ifsi@tab@mantsign}
%\begin{macro}{\ifsi@tab@expsign}
% To process the format information, various internal
% number-processing macros are used.  First, some
% storage areas are created.
%    \begin{macrocode}
\newcount\si@tab@mantprecnt
\newcount\si@tab@mantpostcnt
\newcount\si@tab@expprecnt
\newcount\si@tab@exppostcnt
\newif\ifsi@tab@mantsign
\newif\ifsi@tab@expsign
%    \end{macrocode}
%\end{macro}
%\end{macro}
%\end{macro}
%\end{macro}
%\end{macro}
%\end{macro}
% The input is split into a mantissa and exponent, then passed to a
% re-useable macro for further processing.
%    \begin{macrocode}
\si@opt@cmdkey{tabformat}
  {\si@num@fixpm
   \renewcommand*{\si@num@arg}{tabformat data}%
   \renewcommand*{\si@num@exp}{}%
   \renewcommand*{\si@num@mant}{}%
   \si@tab@mantsignfalse
   \si@tab@expsignfalse
   \si@switchfalse
   \si@num@sepmantexp{#1}%
%    \end{macrocode}
% When checking for a sign, the internal flag for finding but
% deleting a plus sign is used.
%    \begin{macrocode}
   \si@num@sepsign{mant}%
   \ifx\@empty\si@num@mantsign\@empty
     \ifsi@num@delplus
       \expandafter\expandafter\expandafter\si@tab@mantsigntrue
     \fi
   \else
     \expandafter\si@tab@mantsigntrue
   \fi
   \si@num@sepsign{exp}%
   \ifx\@empty\si@num@expsign\@empty
     \ifsi@num@delplus
       \expandafter\expandafter\expandafter\si@tab@expsigntrue
     \fi
   \else
     \expandafter\si@tab@expsigntrue
   \fi
   \si@opt@proctform{mant}%
   \si@opt@proctform{exp}%
%    \end{macrocode}
%\changes{v1.0h}{2008/08/02}{Simplified warnings for \opt{tabformat}}
% If both the integer and decimal parts are empty, then something has
% probably gone wrong.
%    \begin{macrocode}
   \ifnum\si@tab@mantpostcnt=\z@\relax
     \ifnum\si@tab@mantprecnt=\z@\relax
       \si@log@err{Empty mantissa argument for tabformat}
         {The argument `#1' contains no valid entry for
          a mantissa\MessageBreak It should be of the
          form `m.n', where  m and n are integers}%
     \fi
   \fi
%    \end{macrocode}
% If \opt{tabformat} has been given with
% \opt{tabnumalign=centredecimal} active, then the alignment is
% changed to centred.
%    \begin{macrocode}
   \ifsi@tab@fixed\else
     \sisetup{tabnumalign=centre}%
   \fi
   \let\pm\si@num@pm
   \let\mp\si@num@mp}
%    \end{macrocode}
%\begin{macro}{\si@opt@proctform}
%\darg{either \texttt{mant} or \texttt{exp}}
% Processing the number further uses the \cs{si@num@digits} macro.
% The results are stored in the appropriate counter.
%    \begin{macrocode}
\newcommand*{\si@opt@proctform}[1]{%
  \renewcommand*{\si@num@predec}{}%
  \renewcommand*{\si@num@postdec}{}%
  \si@switchfalse
  \expandafter\si@ifnotmtarg\expandafter{%
    \csname si@num@#1\endcsname}
     {\expandafter\expandafter\expandafter\si@num@digits
        \csname si@num@#1\endcsname\@empty\@empty}%
   \csname si@tab@#1precnt\endcsname\z@\relax
   \csname si@tab@#1postcnt\endcsname\z@\relax
   \ifx\@empty\si@num@predec\@empty\else
     \csname si@tab@#1precnt\endcsname\si@num@predec\relax
   \fi
   \ifx\@empty\si@num@postdec\@empty\else
     \csname si@tab@#1postcnt\endcsname\si@num@postdec\relax
   \fi}
%    \end{macrocode}
%\end{macro}
% The alignment of tabular material when not processed by \cs{num}
% needs to be available.
%    \begin{macrocode}
\si@opt@choicekey{tabtextalign}{left,right,centre,center}
%    \end{macrocode}
%\begin{macro}{\si@tab@rfill@t}
%\begin{macro}{\si@tab@lfill@t}
% By default, centring happens on both sides of the content of
% tabular material.
%    \begin{macrocode}
  {\def\si@tab@rfill@t{0.5}%
   \def\si@tab@lfill@t{0.5}%
   \renewcommand*{\si@tempb}{right}%
   \ifx\si@tempa\si@tempb
     \def\si@tab@rfill@t{0}%
     \def\si@tab@lfill@t{1}%
   \fi
   \renewcommand*{\si@tempb}{left}%
   \ifx\si@tempa\si@tempb
     \def\si@tab@rfill@t{1}%
     \def\si@tab@lfill@t{0}%
   \fi}
%    \end{macrocode}
%\end{macro}
%\end{macro}
% The alignment of unit columns for tabular material has a similar
% control option.
%    \begin{macrocode}
\si@opt@choicekey{tabunitalign}{left,right,centre,center}
%    \end{macrocode}
%\begin{macro}{\si@tab@rfill@s}
%\begin{macro}{\si@tab@lfill@s}
% By default, centring happens on both sides of the content of
% tabular material.
%    \begin{macrocode}
  {\def\si@tab@rfill@s{0.5}%
   \def\si@tab@lfill@s{0.5}%
   \renewcommand*{\si@tempb}{right}%
   \ifx\si@tempa\si@tempb
     \def\si@tab@rfill@s{0}%
     \def\si@tab@lfill@s{1}%
   \fi
   \renewcommand*{\si@tempb}{left}%
   \ifx\si@tempa\si@tempb
     \def\si@tab@rfill@s{1}%
     \def\si@tab@lfill@s{0}%
   \fi}
%    \end{macrocode}
%\end{macro}
%\end{macro}
%\begin{macro}{\ifsi@fixdp}
%\changes{v1.1}{2008/08/22}{Modified to work with \opt{fixsf} option}
%\begin{macro}{\si@num@dp}
%\changes{v1.1}{2008/08/22}{Modified to work with \opt{fixsf} option}
% To allow control of rounding, two options are needed.  One sets how
% many fixed digits to use, the second turns this function on and
% off.
%    \begin{macrocode}
\si@opt@boolkey[%
  \ifsi@fixdp
    \expandafter\si@fixsffalse
  \fi]{fixdp}
\newcount\si@num@dp
\si@opt@key{dp}{%
  \si@str@ifonlychrs{#1}{0123456789}
    {\si@num@dp#1\relax
     \si@fixdptrue
     \si@fixsffalse}
    {\si@log@err{Invalid input for dp option}
      {The dp option must be given a positive integer}}}
%    \end{macrocode}
%\end{macro}
%\end{macro}
%\begin{macro}{\ifsi@fixsf}
%\begin{macro}{\si@num@sf}
% A similar approach for fixing significant figures.
%    \begin{macrocode}
\si@opt@boolkey[%
  \ifsi@fixsf
    \expandafter\si@fixdpfalse
  \fi]{fixsf}
\newcount\si@num@sf
\si@opt@key{sf}{%
  \si@str@ifonlychrs{#1}{0123456789}
    {\si@num@sf#1\relax
     \ifnum\si@num@sf=\z@\relax
       \si@log@warn{Setting sf=0 disables function}%
       \si@fixsffalse
     \else
       \si@fixsftrue
       \si@fixdpfalse
     \fi}
    {\si@log@err{Invalid input for sf option}
      {The sf option must be given a positive integer}}}
%    \end{macrocode}
%\end{macro}
%\end{macro}
%\begin{macro}{\ifsi@tabautofit}
% To apply rounding automatically in a table, a separate option is
% used.
%    \begin{macrocode}
\si@opt@boolkey{tabautofit}
%    \end{macrocode}
%\end{macro}
%\begin{macro}{\ifsi@xspace}
% Unit macros on their own may need \pkg{xpsace}.
%    \begin{macrocode}
\si@opt@boolkey{xspace}
%    \end{macrocode}
%\end{macro}
%\begin{macro}{\ifsi@prespace}
%    \begin{macrocode}
\si@opt@boolkey
  [\si@unt@numfalse
   \ifsi@prespace
     \si@unt@numtrue
  \fi]
  {prespace}
%    \end{macrocode}
%\end{macro}
%\begin{macro}{\ifsi@allowoptarg}
% For \unitsdef users, a method to absorb optional arguments is
% needed.
%    \begin{macrocode}
\si@opt@boolkey{allowoptarg}
%    \end{macrocode}
%\end{macro}
%\changes{v1.1d}{2008/10/29}{Minor bug with \opt{per} option fixed}
%\begin{macro}{\ifsi@frac}
%\begin{macro}{\ifsi@slash}
%\begin{macro}{\ifsi@stickyper}
% The option processing for formatting units with \cs{per} in them
% needs two switches.
%    \begin{macrocode}
\newif\ifsi@slash
\newif\ifsi@frac
\si@opt@boolkey{stickyper}
\si@opt@choicekey[reciprocal]{per}
  {reciprocal,rp,power,slash,frac,fraction}
  {\si@slashfalse
   \si@fracfalse
   \let\si@frac\si@frc@frac
   \renewcommand*{\si@tempb}{slash}%
   \ifx\si@tempa\si@tempb
     \si@fractrue
     \si@slashtrue
     \let\si@frac\si@frc@slash
   \fi
   \renewcommand*{\si@tempb}{frac}%
   \ifx\si@tempa\si@tempb
     \si@fractrue
   \fi
   \renewcommand*{\si@tempb}{fraction}%
   \ifx\si@tempa\si@tempb
     \si@fractrue
   \fi}
%    \end{macrocode}
%\end{macro}
%\end{macro}
%\end{macro}
%\begin{macro}{\si@slash}
% For the \opt{slash} option, the separator can be customised.
%    \begin{macrocode}
\si@opt@xchoicekey{slash}{slash}{slash}
%    \end{macrocode}
%\end{macro}
%\begin{macro}{\ifsi@repeatunits}
%\begin{macro}{\ifsi@addunitpower}
% An option is needed for cases where units should be repeated.
%    \begin{macrocode}
\newif\ifsi@repeatunits
\newif\ifsi@addunitpower
\si@opt@choicekey[true]{repeatunits}{true,false,power}
  {\si@repeatunitsfalse
   \si@addunitpowerfalse
   \renewcommand*{\si@tempb}{true}%
   \ifx\si@tempa\si@tempb
     \si@repeatunitstrue
   \fi
   \renewcommand*{\si@tempb}{power}%
   \ifx\si@tempa\si@tempb
     \si@addunitpowertrue
   \fi}
%    \end{macrocode}
%\end{macro}
%\end{macro}
%\begin{macro}{\ifsi@trapambigfrac}
%\begin{macro}{\si@closefrac}
%\begin{macro}{\si@openfrac}
% Macros for the right and left brackets added to potentially
% ambiguous denominators.
%    \begin{macrocode}
\si@opt@boolkey{trapambigfrac}
\si@opt@cmdkeys{closefrac,openfrac}
%    \end{macrocode}
%\end{macro}
%\end{macro}
%\end{macro}
% In the case of fractional handling of the \cs{per} operator,
% further refinement is available.
%    \begin{macrocode}
\si@opt@choicekey[frac]{fraction}
  {frac,nicefrac,nice,sfrac,xfrac,uglyfrac,ugly}
  {\let\si@frac\si@frc@frac
   \renewcommand*{\si@tempb}{nicefrac}%
   \ifx\si@tempa\si@tempb
     \let\si@frac\si@frc@nice
   \fi
   \renewcommand*{\si@tempb}{uglyfrac}%
   \ifx\si@tempa\si@tempb
     \let\si@frac\si@frc@ugly
   \fi
   \renewcommand*{\si@tempb}{nice}%
   \ifx\si@tempa\si@tempb
     \let\si@frac\si@frc@nice
   \fi
   \renewcommand*{\si@tempb}{sfrac}%
   \ifx\si@tempa\si@tempb
     \let\si@frac\si@frc@sfrac
   \fi
   \renewcommand*{\si@tempb}{xfrac}%
   \ifx\si@tempa\si@tempb
     \let\si@frac\si@frc@sfrac
   \fi
   \renewcommand*{\si@tempb}{ugly}%
   \ifx\si@tempa\si@tempb
     \let\si@frac\si@frc@ugly
   \fi}
%    \end{macrocode}
%\begin{macro}{\si@load}
%\begin{macro}{\si@noload}
% Loading of support files is controlled by two keys.  The first
% defines a list of files that may be loaded, the second a list that
% will not.  This makes it easy to exclude a single file from a long
% list.
%    \begin{macrocode}
\si@opt@cmdkeys{load,noload}
\si@opt@key{alsoload}{\si@addtolist{\si@load}{#1}}
\AtEndOfPackage{
  \si@opt@disablekey{load}
    {Configuration files can only be used\MessageBreak
     when loading package}
  \si@opt@disablekey{noload}
    {Configuration files can only be used\MessageBreak
     when loading package}}
\AtEndOfPackage{
  \si@opt@key{alsoload}{%
    \@for\si@tempa:=#1\do{\si@loadfile{\si@tempa}}}}
%    \end{macrocode}
%\end{macro}
%\end{macro}
%\begin{macro}{\ifsi@colourunits}
%\begin{macro}{\ifsi@colourvalues}
%\begin{macro}{\si@unitcolour}
%\begin{macro}{\si@valuecolour}
% Colour is turned on and off using two switches and the appropriate
% options.  US spellings are also provided.
%    \begin{macrocode}
\si@opt@boolkey{colourunits}
\si@opt@boolkey{colourvalues}
\si@opt@choicekey[true]{colorunits}
  {true,false}
  {\si@colourunitsfalse
   \renewcommand*{\si@tempb}{true}%
   \ifx\si@tempa\si@tempb
     \si@colourunitstrue
   \fi}
\si@opt@choicekey[true]{colorvalues}
  {true,false}
  {\si@colourvaluesfalse
   \renewcommand*{\si@tempb}{true}%
   \ifx\si@tempa\si@tempb
     \si@colourvaluestrue
   \fi}
\si@opt@choicekey[true]{colorall}
  {true,false}
  {\si@colourvaluesfalse
   \si@colourunitsfalse
   \renewcommand*{\si@tempb}{true}%
   \ifx\si@tempa\si@tempb
     \si@colourunitstrue
     \si@colourvaluestrue
   \fi}
\si@opt@choicekey[true]{colourall}
  {true,false}
  {\si@colourvaluesfalse
   \si@colourunitsfalse
   \renewcommand*{\si@tempb}{true}%
   \ifx\si@tempa\si@tempb
     \si@colourunitstrue
     \si@colourvaluestrue
   \fi}
\si@opt@cmdkeys{unitcolour,valuecolour}
\si@opt@key{unitcolor}{\sisetup{unitcolour=#1}}
\si@opt@key{valuecolor}{\sisetup{valuecolour=#1}}
\si@opt@key{colour}{\sisetup{unitcolour=#1,valuecolour=#1}}
\si@opt@key{color}{\sisetup{unitcolour=#1,valuecolour=#1}}
%    \end{macrocode}
%\end{macro}
%\end{macro}
%\end{macro}
%\end{macro}
%\begin{macro}{\ifsi@colourneg}
%\begin{macro}{\si@negcolour}
% The set up for colouring negative numbers is similar.
%    \begin{macrocode}
\si@opt@boolkey{colourneg}
\si@opt@choicekey[true]{colorneg}
  {true,false}
  {\si@colournegfalse
   \renewcommand*{\si@tempb}{true}%
   \ifx\si@tempa\si@tempb
     \si@colournegtrue
   \fi}
\si@opt@cmdkeys{negcolour}
\si@opt@key{negcolor}{\sisetup{negcolour=#1}}
%    \end{macrocode}
%\end{macro}
%\end{macro}
%\begin{macro}{\si@textOmega}
%\begin{macro}{\si@mathsOmega}
% The various non-Latin symbols need to be handled, and given user interfaces.
% Some definitions are more complex than others; for \si{\ohm}
% things are easy.
%    \begin{macrocode}
\si@opt@cmdkeys{textOmega,mathsOmega}
\si@opt@key{mathOmega}{\sisetup{mathsOmega=#1}}
\newcommand*{\si@mathsOmega}{\text{\ensuremath{\Omega}}}
\newcommand*{\si@textOmega}{\ensuremath{\Omega}}
%    \end{macrocode}
%\end{macro}
%\end{macro}
%\changes{v1.0c}{2008/06/25}{Fixed excess loading of maths fonts}
%\begin{macro}{\si@textmu}
%\begin{macro}{\si@mathsmu}
% For the \si{\micro} symbol, some direct loading of symbols is
% needed as the maths mu sign ($\mu$) is wrong.
%    \begin{macrocode}
\si@opt@cmdkeys{textmu,mathsmu}
\si@opt@key{mathmu}{\sisetup{mathsmu=#1}}
\DeclareFontEncoding{TS1}{}{}
\DeclareFontSubstitution{TS1}{cmr}{m}{n}
\DeclareTextSymbol{\si@textmu}{TS1}{181}
\DeclareTextSymbolDefault{\si@textmu}{TS1}
\@ifpackageloaded{upgreek}
  {}
  {\DeclareFontFamily{OML}{eur}{\skewchar\font'177}
   \DeclareFontShape{OML}{eur}{m}{n}{%
    <-6> eurm5 <6-8> eurm7 <8-> eurm10}{}}
\AtBeginDocument{
  \@ifpackageloaded{upgreek}
    {\let\si@mathsmu\upmu}
    {\DeclareSymbolFont{si@greek}{OML}{eur}{m}{n}
     \DeclareMathSymbol{\si@mathsmu}{\mathord}{si@greek}{"16}}}
%    \end{macrocode}
%\end{macro}
%\end{macro}
%\begin{macro}{\si@textdegree}
%\begin{macro}{\si@mathsdegree}
%\begin{macro}{\si@textminute}
%\begin{macro}{\si@mathsminute}
%\begin{macro}{\si@textsecond}
%\begin{macro}{\si@mathssecond}
% The angle signs.
%    \begin{macrocode}
\si@opt@cmdkeys{textdegree,mathsdegree,textminute,mathsminute,
  textsecond,mathssecond}
\si@opt@key{mathdegree}{\sisetup{mathsdegree=#1}}
\si@opt@key{mathminute}{\sisetup{mathsminute=#1}}
\si@opt@key{mathsecond}{\sisetup{mathssecond=#1}}
\newcommand*{\si@textdegree}{\ensuremath{{}^{\circ}}}
\newcommand*{\si@mathsdegree}{{}^{\circ}}
\newcommand*{\si@textminute}{\ensuremath{{}^{\prime}}}
\newcommand*{\si@mathsminute}{{}^{\prime}}
\newcommand*{\si@textsecond}{\ensuremath{{}^{\prime\prime}}}
\newcommand*{\si@mathssecond}{{}^{\prime\prime}}
%    \end{macrocode}
%\end{macro}
%\end{macro}
%\end{macro}
%\end{macro}
%\end{macro}
%\end{macro}
%\begin{macro}{\si@textcelsius}
%\changes{v1.1e}{2008/11/18}{Bug fix in appearance}
%\begin{macro}{\si@mathscelsius}
% Finally, degrees Celsius, which may need the degree symbol.
%    \begin{macrocode}
\si@opt@cmdkeys{textcelsius,mathscelsius}
\si@opt@key{mathcelsius}{\sisetup{mathscelsius=#1}}
\newcommand*{\si@textcelsius}{%
  \si@textdegree\kern-\scriptspace C}
\newcommand*{\si@mathscelsius}{%
  \si@mathsdegree\kern-\scriptspace C}
%    \end{macrocode}
%\end{macro}
%\end{macro}
%\begin{macro}{\si@textringA}
%\begin{macro}{\si@mathsringA}
% The \si{\angstrom} sign.
%    \begin{macrocode}
\si@opt@cmdkeys{textringA,mathsringA}
\si@opt@key{mathringA}{\sisetup{mathsringA=#1}}
\newcommand*{\si@textringA}{\AA}
\newcommand*{\si@mathsringA}{\text{\AA}}
%    \end{macrocode}
%\end{macro}
%\end{macro}
%\begin{macro}{\ifsi@redefsymbols}
% A flag for using \pkg{textcomp} and \pkg{upgreek} to provide better
% symbols.
%    \begin{macrocode}
\si@opt@boolkey{redefsymbols}
\AtBeginDocument{
  \si@opt@disablekey{redefsymbols}
    {Symbols can only be redefined\MessageBreak
     when loading siunitx}}
%    \end{macrocode}
%\end{macro}
%\begin{macro}{\si@eVcorra}
%\begin{macro}{\si@eVcorrb}
%    \begin{macrocode}
\newlength\si@eVcorra
\newlength\si@eVcorrb
\si@opt@key{eVcorra}{\setlength\si@eVcorra{#1}}
\si@opt@key{eVcorrb}{\setlength\si@eVcorrb{#1}}
%    \end{macrocode}
%\end{macro}
%\end{macro}
%\begin{macro}{\si@locale}
%\begin{macro}{\si@loctolang}
% Handling typographic conventions needs three keys.  \opt{locale} is
% used to set the locale, \opt{loctolang} to bind to \pkg{babel}.
%    \begin{macrocode}
\si@opt@key{locale}{%
  \si@loc@load{#1}%
  \si@loc@set{#1}}%
\si@opt@key{loctolang}{\si@loc@ltol{#1}}
%    \end{macrocode}
%\end{macro}
%\end{macro}
%
%\subsection{Compatibility options}
%\begin{macro}{\ifsi@old@ugly}
%\begin{macro}{\ifsi@old@nice}
%\begin{macro}{\ifsi@old@loose}
%\begin{macro}{\ifsi@old@tight}
% With the options for the package set up, the next stage is to
% provide support for users of the older packages.  These all set up
% switches, but do not do anything.  That is left to the emulation
% files, loaded at the end of the package. First of all, the
% \units options are dealt with; there are not many.
%    \begin{macrocode}
\si@opt@compatkey{units}{ugly}
\si@opt@compatkey{units}{nice}
\si@opt@compatkey{units}{loose}
\si@opt@compatkey{units}{tight}
%    \end{macrocode}
%\end{macro}
%\end{macro}
%\end{macro}
%\end{macro}
%\begin{macro}{\ifsi@old@OHM}
%\begin{macro}{\ifsi@old@ohm}
%\begin{macro}{\ifsi@old@redef-gensymb}
%\begin{macro}{\ifsi@gensymb}
% The \unitsdef package is unfortunately much more profligate with
% options. The first set are to do with support for \pkg{gensymb}.
%    \begin{macrocode}
\si@opt@compatkey{unitsdef}{OHM}
\si@opt@compatkey{unitsdef}{ohm}
\si@opt@compatkey{unitsdef}{redef-gensymb}
\newif\ifsi@gensymb
%    \end{macrocode}
%\end{macro}
%\end{macro}
%\end{macro}
%\end{macro}
%\begin{macro}{\ifsi@old@LITER}
%\begin{macro}{\ifsi@old@liter}
%\begin{macro}{\ifsi@old@noxspace}
%\begin{macro}{\ifsi@old@noconfig}
% The second set are more general functionality.
%    \begin{macrocode}
\si@opt@compatkey{unitsdef}{LITER}
\si@opt@compatkey{unitsdef}{liter}
\si@opt@compatkey{unitsdef}{noxspace}
\si@opt@compatkey{unitsdef}{noconfig}
%    \end{macrocode}
%\end{macro}
%\end{macro}
%\end{macro}
%\end{macro}
%\begin{macro}{\ifsi@old@noabbr}
%\begin{macro}{\ifsi@old@noamperageabbr}
%\begin{macro}{\ifsi@old@nofrequncyabbr}
%\begin{macro}{\ifsi@old@nomolabbr}
%\begin{macro}{\ifsi@old@novoltageabbr}
%\begin{macro}{\ifsi@old@novolumeabbr}
%\begin{macro}{\ifsi@old@noweightabbr}
%\begin{macro}{\ifsi@old@noenergyabbr}
%\begin{macro}{\ifsi@old@nolengthabbr}
%\begin{macro}{\ifsi@old@notimeabbr}
% The final set are for control of abbreviations, and are a good
% demonstration of why to use \pkg{xkeyval}!
%    \begin{macrocode}
\si@opt@compatkey{unitsdef}{noabbr}
\si@opt@compatkey{unitsdef}{noampereageabbr}
\si@opt@compatkey{unitsdef}{nofrequncyabbr}
\si@opt@compatkey{unitsdef}{nomolabbr}
\si@opt@compatkey{unitsdef}{novoltageabbr}
\si@opt@compatkey{unitsdef}{novolumeabbr}
\si@opt@compatkey{unitsdef}{noweightabbr}
\si@opt@compatkey{unitsdef}{noenergyabbr}
\si@opt@compatkey{unitsdef}{nolengthabbr}
\si@opt@compatkey{unitsdef}{notimeabbr}
%    \end{macrocode}
%\end{macro}
%\end{macro}
%\end{macro}
%\end{macro}
%\end{macro}
%\end{macro}
%\end{macro}
%\end{macro}
%\end{macro}
%\end{macro}
%\begin{macro}{\ifsi@old@cdot}
%\begin{macro}{\ifsi@old@thickspace}
%\begin{macro}{\ifsi@old@mediumspace}
%\begin{macro}{\ifsi@old@thinspace}
%\begin{macro}{\ifsi@old@thickqspace}
%\begin{macro}{\ifsi@old@mediumqspace}
%\begin{macro}{\ifsi@old@thinqspace}
% The \SIunits package has lots of options.  These ones are all
% related to spacing.
%    \begin{macrocode}
\si@opt@compatkey{SIunits}{cdot}
\si@opt@compatkey{SIunits}{thickspace}
\si@opt@compatkey{SIunits}{mediumspace}
\si@opt@compatkey{SIunits}{thinspace}
\si@opt@compatkey{SIunits}{thickqspace}
\si@opt@compatkey{SIunits}{mediumqspace}
\si@opt@compatkey{SIunits}{thinqspace}
%    \end{macrocode}
%\end{macro}
%\end{macro}
%\end{macro}
%\end{macro}
%\end{macro}
%\end{macro}
%\end{macro}
%\begin{macro}{\ifsi@old@amssymb}
%\begin{macro}{\ifsi@old@squaren}
%\begin{macro}{\ifsi@old@pstricks}
%\begin{macro}{\ifsi@old@Gray}
%\begin{macro}{\ifsi@old@italian}
% These options are used by \SIunits to control clashes with other
% packages.
%    \begin{macrocode}
\si@opt@compatkey{SIunits}{amssymb}
\si@opt@compatkey{SIunits}{squaren}
\si@opt@compatkey{SIunits}{pstricks}
\si@opt@compatkey{SIunits}{Gray}
\si@opt@compatkey{SIunits}{italian}
%    \end{macrocode}
%\end{macro}
%\end{macro}
%\end{macro}
%\end{macro}
%\end{macro}
%\begin{macro}{\ifsi@old@textstyle}
%\begin{macro}{\ifsi@old@binary}
%\begin{macro}{\ifsi@old@noams}
%\begin{macro}{\ifsi@old@derivedinbase}
%\begin{macro}{\ifsi@old@derived}
% The miscellaneous options.
%    \begin{macrocode}
\si@opt@compatkey{SIunits}{textstyle}
\si@opt@compatkey{SIunits}{binary}
\si@opt@compatkey{SIunits}{noams}
\si@opt@compatkey{SIunits}{derivedinbase}
\si@opt@compatkey{SIunits}{derived}
%    \end{macrocode}
%\end{macro}
%\end{macro}
%\end{macro}
%\end{macro}
%\end{macro}
%\begin{macro}{\ifsi@old@noprefixcmds}
% The \pkg{hepunits} package only has one option.
%    \begin{macrocode}
\si@opt@compatkey{hepunits}{noprefixcmds}
%    \end{macrocode}
%\end{macro}
%\begin{macro}{\ifsi@old@english}
%\begin{macro}{\ifsi@old@french}
% \pkg{fancynum} provides a few options.  First the rather oddly
% named \opt{english} and \opt{french} ones.
%    \begin{macrocode}
\si@opt@compatkey{fancynum}{english}
\si@opt@compatkey{fancynum}{french}
%    \end{macrocode}
%\end{macro}
%\end{macro}
%\begin{macro}{\ifsi@old@tight}
%\begin{macro}{\ifsi@old@loose}
% A couple for spacing multiplication.
%    \begin{macrocode}
\si@opt@compatkey{fancynum}{tight}
\si@opt@compatkey{fancynum}{loose}
%    \end{macrocode}
%\end{macro}
%\end{macro}
%\begin{macro}{\ifsi@old@thinspaces}
%\begin{macro}{\ifsi@old@commas}
%\begin{macro}{\ifsi@old@plain}
% Three for digit separation.
%    \begin{macrocode}
\si@opt@compatkey{fancynum}{thinspaces}
\si@opt@compatkey{fancynum}{commas}
\si@opt@compatkey{fancynum}{plain}
%    \end{macrocode}
%\end{macro}
%\end{macro}
%\end{macro}
%\begin{macro}{\ifsi@old@spaceqspace}
% The \pkg{fancyunits} package provides one option not available with
% \SIunits.
%    \begin{macrocode}
\si@opt@compatkey{fancyunits}{spaceqspace}
%    \end{macrocode}
%\end{macro}
%
%\subsection{Constants}
% A number of macros are needed by the package that provide a
% non-changing output.  These are defined here; the intention is that
% these should not be macros that the user is likely to need to
% alter.  All of these macros have preface \cs{si@fix@}, to flag that
% that are intended as constants.  The package may rely on the
% contents of these macros for functionality.
%
%\begin{macro}{\si@fix@thin}
%\begin{macro}{\si@fix@med}
%\begin{macro}{\si@fix@medium}
%\begin{macro}{\si@fix@thick}
%\begin{macro}{\si@fix@space}
%\changes{v1.0a}{2008/06/20}{Fixed problem with \opt{space} option}
% First, there are the various space macros.  To allow both \opt{med}
% and \opt{medium} to be used as a space description, two macros are
% needed for the same output.
%    \begin{macrocode}
\newcommand*{\si@fix@thin}{\,}
\newcommand*{\si@fix@med}{\:}
\newcommand*{\si@fix@medium}{\:}
\newcommand*{\si@fix@thick}{\;}
\newcommand*{\si@fix@space}{\text{ }}
%    \end{macrocode}
%\end{macro}
%\end{macro}
%\end{macro}
%\end{macro}
%\end{macro}
%\begin{macro}{\si@fix@cdot}
%\begin{macro}{\si@fix@comma}
%\begin{macro}{\si@fix@stop}
%\begin{macro}{\si@fix@fullstop}
%\begin{macro}{\si@fix@period}
%\begin{macro}{\si@fix@times}
%\begin{macro}{\si@fix@tighttimes}
%\begin{macro}{\si@fix@tightcdot}
% Next there are macros for material that is not simply whitespace.
% To allow several options, the full-stop gets lots of names.
%    \begin{macrocode}
\newcommand*{\si@fix@cdot}{{}\cdot{}}
\newcommand*{\si@fix@comma}{{,}}
\newcommand*{\si@fix@stop}{{.}}
\newcommand*{\si@fix@fullstop}{{.}}
\newcommand*{\si@fix@period}{{.}}
\newcommand*{\si@fix@times}{\times}
\newcommand*{\si@fix@tighttimes}{\bgroup\times\egroup}
\newcommand*{\si@fix@tightcdot}{\bgroup\cdot\egroup}
%    \end{macrocode}
%\end{macro}
%\end{macro}
%\end{macro}
%\end{macro}
%\end{macro}
%\end{macro}
%\end{macro}
%\end{macro}
%\begin{macro}{\si@fix@plus}
%\begin{macro}{\si@fix@minus}
%\begin{macro}{\si@fix@pm}
%\begin{macro}{\si@fix@tightpm}
%\begin{macro}{\si@fix@mp}
% Signs for numbers are needed.
%    \begin{macrocode}
\newcommand*{\si@fix@plus}{+}
\newcommand*{\si@fix@minus}{-}
\newcommand*{\si@fix@pm}{\pm}
\newcommand*{\si@fix@tightpm}{\bgroup\pm\egroup}
\newcommand*{\si@fix@mp}{\mp}
%    \end{macrocode}
%\end{macro}
%\end{macro}
%\end{macro}
%\end{macro}
%\end{macro}
%\begin{macro}{\si@fix@two}
%\begin{macro}{\si@fix@ten}
% The literals ``2'' and ``10'' are needed for exponents.
%    \begin{macrocode}
\newcommand*{\si@fix@two}{2}
\newcommand*{\si@fix@ten}{10}
%    \end{macrocode}
%\end{macro}
%\end{macro}
%\begin{macro}{\si@fix@slash}
% Another optional component that will probably not be used by many
% people.
%    \begin{macrocode}
\newcommand*{\si@fix@slash}{/}
%    \end{macrocode}
%\end{macro}
%\begin{macro}{\si@fix@none}
% Finally for spacing, there is the possibility of nothing at all
%    \begin{macrocode}
\newcommand*{\si@fix@none}{}
%    \end{macrocode}
%\end{macro}
%
%\subsection{Symbols}
%\begin{macro}{\si@symbol}
% Each of the symbol macros needs to be set up; the options give a
% maths and text mode sign, but internally a single macro is needed
% for each.
%    \begin{macrocode}
\newcommand*{\si@symbol}[1]{%
  \expandafter\protected\expandafter\def
    \csname si@sym@#1\endcsname{%
    \ifmmode
      \expandafter\csname si@maths#1\expandafter\endcsname
    \else
      \expandafter\csname si@text#1\expandafter\endcsname
    \fi}}
%    \end{macrocode}
%\end{macro}
%\begin{macro}{\si@sym@Omega}
%\begin{macro}{\si@sym@ringA}
%\begin{macro}{\si@sym@mu}
%\begin{macro}{\si@sym@degree}
%\begin{macro}{\si@sym@minute}
%\begin{macro}{\si@sym@second}
%\begin{macro}{\si@sym@celsius}
% The various symbols are now declared.
%    \begin{macrocode}
\si@symbol{Omega}
\si@symbol{ringA}
\si@symbol{mu}
\si@symbol{degree}
\si@symbol{minute}
\si@symbol{second}
\si@symbol{celsius}
%    \end{macrocode}
%\end{macro}
%\end{macro}
%\end{macro}
%\end{macro}
%\end{macro}
%\end{macro}
%\end{macro}
% The issue of redefinition of symbols now arises.  \currpkg can
% check for the loading of a number of support package, and can then
% redefine the appropriate symbols.
%    \begin{macrocode}
\AtBeginDocument{%
  \ifsi@redefsymbols
    \@ifpackageloaded{textcomp}
      {\si@log@debug{Redefining symbols using textcomp}%
       \renewcommand*{\si@textdegree}{\textdegree}%
       \renewcommand*{\si@mathsdegree}{\text{\textdegree}}%
%    \end{macrocode}
% \pkg{mathptmx} will give issues with \pkg{textcomp} and the
% \si{\ohm} sign.
%    \begin{macrocode}
       \@ifpackageloaded{mathptmx}{}
         {\renewcommand*{\si@textmu}{\textmu}%
          \renewcommand*{\si@textOmega}{\textohm}}%
%    \end{macrocode}
% The \si{\angstrom} symbol is only redefined if the encoding is
% OT1; other encodings should have a proper glyph used for \cs{AA}.
% The \cs{encodingdefault} macro is \cs{long} for some reason.
%    \begin{macrocode}
       \renewcommand*{\si@tempa}{OT1}%
       \ifx\si@tempa\encodingdefault
         \renewcommand*{\si@mathsringA}{%
           \text{\capitalring{A}}}%
         \renewcommand*{\si@textringA}{\capitalring{A}}
       \fi}{}
    \@ifpackageloaded{upgreek}
      {\si@log@debug{Redefining symbols using upgreek}%
       \renewcommand*{\si@mathsOmega}{\Upomega}}{}
  \fi}
%    \end{macrocode}
%
%\subsection{Handling fractions}
%\begin{macro}{\si@frac}
%\begin{macro}{\si@frc@hook}
%\begin{macro}{\si@frc@frac}
%\begin{macro}{\si@frc@slash}
%\begin{macro}{\si@frc@nice}
%\begin{macro}{\si@frc@sfrac}
% Various methods of handling fractions are provided.
%    \begin{macrocode}
\newcommand*{\si@frc@frac}[2]{%
  \ensuremath{\si@frc@hook\frac{%
    \expandafter\si@unt@out\expandafter{#1}}%
    {\expandafter\si@unt@out\expandafter{#2}}}}
\let\si@frac\si@frc@frac
\newcommand*{\si@frc@hook}{}
\newcommand*{\si@frc@slash}[2]{%
    \expandafter\si@unt@out\expandafter{#1}%
    \si@out{\ensuremath{\si@slash}}%
    \expandafter\si@unt@out\expandafter{#2}}
\newcommand*{\si@frc@nice}[2]{%
  \ensuremath{\si@frc@nicefrac{\expandafter\si@unt@out%
    \expandafter{#1}}{\expandafter\si@unt@out\expandafter
      {#2}}}}
\newcommand*{\si@frc@sfrac}[2]{%
    \sfrac{\expandafter\si@unt@out\expandafter{#1}}%
      {\expandafter\si@unt@out\expandafter{#2}}}
\AtBeginDocument{
  \@ifpackageloaded{xfrac}
    {}
    {\si@log@inf{xfrac package unavailable\MessageBreak
       using `fraction=sfrac' will fall back on\MessageBreak
       nicefrac-like method}%
     \renewcommand*{\si@frc@sfrac}[2]{%
      \si@log@warn{xfrac package unavailable}%
      \si@frc@nice{#1}{#2}}}}
%    \end{macrocode}
%\end{macro}
%\end{macro}
%\end{macro}
%\end{macro}
%\end{macro}
%\end{macro}
%\begin{macro}{\si@frc@nicefrac}
%\begin{macro}{\si@frc@displen}
%\begin{macro}{\si@frc@textlen}
%\begin{macro}{\si@frc@suplen}
%\begin{macro}{\si@frc@ssuplen}
% To avoid needing \units installed, the \cs{nicefrac} macro needs
% to be emulated here. The code is taken (with permission) from
% \kg{nicefrac}.\footnote{The original is licensed under the GPL;
% thanks to the author Axel Reichert for permission to copy the code
% here.}
%    \begin{macrocode}
\newlength\si@frc@displen
\newlength\si@frc@textlen
\newlength\si@frc@suplen
\newlength\si@frc@ssuplen
\newcommand*{\si@frc@nicefrac}{%
  \ifmmode
    \expandafter\si@frc@mathsnf
  \else
    \expandafter\si@frc@textnf
  \fi}
%    \end{macrocode}
%\end{macro}
%\end{macro}
%\end{macro}
%\end{macro}
%\end{macro}
%\begin{macro}{\si@frc@mathsnf}
%\darg{numerator}
%\darg{denominator}
% The maths mode system.
%    \begin{macrocode}
\newcommand*{\si@frc@mathsnf}[2]{%
  \begingroup
    \settoheight{\si@frc@displen}{\ensuremath{%
      \displaystyle{M}}}%
    \settoheight{\si@frc@textlen}{\ensuremath{%
      \textstyle{M}}}%
    \settoheight{\si@frc@suplen}{\ensuremath{%
      \scriptstyle{M}}}%
    \settoheight{\si@frc@ssuplen}{%
      \ensuremath{\scriptscriptstyle{M}}}%
    \addtolength{\si@frc@displen}{-\si@frc@ssuplen}%
    \addtolength{\si@frc@textlen}{-\si@frc@ssuplen}%
    \addtolength{\si@frc@suplen}{-\si@frc@ssuplen}%
    \mathchoice
      {\raisebox{\si@frc@displen}{\ensuremath{%
        \scriptstyle{#1}}}}%
      {\raisebox{\si@frc@textlen}{\ensuremath{%
         \scriptstyle{#1}}}}%
      {\raisebox{\si@frc@suplen}%
        {\ensuremath{\scriptscriptstyle{#1}}}}%
      {\raisebox{\si@frc@ssuplen}%
        {\ensuremath{\scriptscriptstyle{#1}}}}%
    \mkern-2mu\relax/\mkern-1mu\relax
    \bgroup
      \mathchoice
        {\scriptstyle}%
        {\scriptstyle}%
        {\scriptscriptstyle}%
        {\scriptscriptstyle}%
      {#2}%
    \egroup
  \endgroup}
%    \end{macrocode}
%\end{macro}
%\begin{macro}{\si@frc@textnf}
%\darg{numerator}
%\darg{denominator}
% A stripped down version of the \pkg{nicefrac} system for text mode.
%    \begin{macrocode}
\newcommand*{\si@frc@textnf}[2]{%
  \begingroup
    \settoheight{\si@frc@textlen}{M}%
    \settoheight{\si@frc@ssuplen}{\fontsize\sf@size\z@\relax
      \selectfont{M}}%
    \addtolength{\si@frc@textlen}{-\si@frc@ssuplen}%
    \raisebox{\si@frc@textlen}{\fontsize\sf@size\z@\relax
      \selectfont{#1}}%
    \hspace{-0.25ex}/\hspace{-0.25ex}%
    \hbox{\fontsize\sf@size\z@\selectfont{#2}}%
  \endgroup}
%    \end{macrocode}
%\end{macro}
%\begin{macro}{\si@frc@ugly}
%\darg{numerator}
% The \cs{si@frc@ugly} macro is needed to emulate the \opt{ugly}
% option in \units, where output depends on the current mode.
%    \begin{macrocode}
\newcommand*{\si@frc@ugly}[1]{%
  \renewcommand*{\si@tempa}{#1}%
  \ifmmode
    \expandafter\si@frc@frac
  \else
    \renewcommand*{\si@tempb}{1}%
    \ifx\si@tempa\si@tempb
%    \end{macrocode}
% The slash switch cannot be used, so the possibility of the
% numerator being one is handled here.
%    \begin{macrocode}
      \setbox\si@tempboxa=\hbox{\ensuremath{\si@valuesep}}%
      \hskip-\wd\si@tempboxa\relax
      \renewcommand*{\si@tempa}{}%
    \fi
    \expandafter\si@frc@slash
  \fi
  {\si@tempa}}
%    \end{macrocode}
%\end{macro}
%
%\subsection{Font control}
% A number of controls and tests are needed to control the font used
% for output.  Underlying all of this is the \AMS\ package
% \pkg{amstext} package, providing the \cs{text} command.  Much of
% the font control system here is taken more or less verbatim from
% \SIstyle; modifications have been made to fit the \currpkg
% interface.
%
%\begin{macro}{\si@fam@sf}
%\begin{macro}{\si@fam@tt}
% The package needs to know the maths font families in use.  This is
% set right at the start of the document, after any changes can have
% been made by, for example, \pkg{fontspec}.
%    \begin{macrocode}
\g@addto@macro{\document}{%
  \ifdefined\mathsf
    \setbox\si@tempboxa=\hbox{%
      $\mathsf{\global\chardef\si@fam@sf=\fam}$}%
  \else
    \si@log@inf{\string\mathsf not found}%
    \global\chardef\si@fam@sf=99\relax
  \fi
  \ifdefined\mathtt
    \setbox\si@tempboxa=\hbox{%
      $\mathtt{\global\chardef\si@fam@tt=\fam}$}%
  \else
    \si@log@inf{\string\mathtt not found}%
    \global\chardef\si@fam@tt=99\relax
  \fi}
%    \end{macrocode}
%\end{macro}
%\end{macro}
%\begin{macro}{\si@fam@ifbtext}
%\begin{macro}{\si@fam@ifbmaths}
%\darg{code}
% These tests check for bold in text and maths mode, respectively.
%    \begin{macrocode}
\newcommand*{\si@fam@ifbtext}[1]{%
  \if b\expandafter\@car\f@series\@nil
    #1\fi}
\newcommand*{\si@fam@ifbmaths}[1]{%
  \renewcommand*{\si@tempa}{bold}%
  \ifx\math@version\si@tempa
    #1\fi}
%    \end{macrocode}
%\end{macro}
%\end{macro}
%\begin{macro}{\si@fam@ifbinline}
% For compatibility with \units, a method to change the behaviour
% when in inline maths is needed for the bold detector.
%    \begin{macrocode}
\newcommand*{\si@fam@ifbinline}{%
  \ifsi@inlinebtext
    \expandafter\si@fam@ifbtext
  \else
    \expandafter\si@fam@ifbmaths
  \fi}
%    \end{macrocode}
%\end{macro}
%\begin{macro}{\si@fam@ifitext}
%\darg{code}
% This test check for italic or slanted text in text mode, by
% negation (upright text is |n|).
%    \begin{macrocode}
\newcommand*{\si@fam@ifitext}[1]{%
  \if n\expandafter\@car\f@series\@nil\else
    #1\fi}
%    \end{macrocode}
%\end{macro}
%\begin{macro}{\si@fam@mode}
% Detection of the current mode needs to happen``early'' (before any
% change of \cs{ensuremath}).  So a short macro is provided to do the
% job.
%    \begin{macrocode}
\newcommand*{\si@fam@mode}{%
  \ifsi@obeymode
    \ifmmode
      \sisetup{mode=maths}%
    \else
      \sisetup{mode=text}%
    \fi
  \fi}
%    \end{macrocode}
%\end{macro}
%\begin{macro}{\si@fam@colourcmd}
% The colour command is set up.
%    \begin{macrocode}
\AtBeginDocument{
  \@ifpackageloaded{color}
    {\let\si@fam@colourcmd\color}
    {\let\si@fam@colourcmd\@gobble}}
%    \end{macrocode}
%\end{macro}
%
%\begin{macro}{\ifsi@fam@set}
%\begin{macro}{\ifsi@textmode}
% A marker is set up to check if font-matching has been taken place.
% A second flag is used to track the use of text mode.
%    \begin{macrocode}
\newif\ifsi@fam@set
\newif\ifsi@textmode
%    \end{macrocode}
%\end{macro}
%\end{macro}
%\begin{macro}{\si@fam@set}
% Using the code from \SIstyle as a base, a set of tests are used to
% set the current font families and weights.  To begin with, the
% mode to use is set up.
%    \begin{macrocode}
\newcommand*{\si@fam@set}{%
  \ifsi@out@num
    \ifsi@numtextmode
      \expandafter\expandafter\expandafter\si@textmodetrue
    \else
      \expandafter\expandafter\expandafter\si@textmodefalse
    \fi
  \else
    \ifsi@unittextmode
      \expandafter\expandafter\expandafter\si@textmodetrue
    \else
      \expandafter\expandafter\expandafter\si@textmodefalse
    \fi
  \fi
%    \end{macrocode}
%\begin{macro}{\si@mathsrm}
%\begin{macro}{\si@mathssf}
%\begin{macro}{\si@mathstt}
%\begin{macro}{\si@textrm}
%\begin{macro}{\si@textsf}
%\begin{macro}{\si@texttt}
%\begin{macro}{\si@colourcmd}
%\begin{macro}{\si@colour}
% The appropriate font macros are now established, if necessary.
%    \begin{macrocode}
  \ifsi@fam@set\else
    \let\si@colourcmd\@gobble
    \ifsi@out@num
      \let\si@mathsrm\si@valuemathsrm
      \let\si@mathssf\si@valuemathssf
      \let\si@mathstt\si@valuemathstt
      \let\si@textrm\si@valuetextrm
      \let\si@textsf\si@valuetextsf
      \let\si@texttt\si@valuetexttt
      \ifsi@colourvalues
        \let\si@colourcmd\si@fam@colourcmd
      \fi
      \let\si@colour\si@valuecolour
    \else
      \let\si@mathsrm\si@unitmathsrm
      \let\si@mathssf\si@unitmathssf
      \let\si@mathstt\si@unitmathstt
      \let\si@textrm\si@unittextrm
      \let\si@textsf\si@unittextsf
      \let\si@texttt\si@unittexttt
      \ifsi@colourunits
        \let\si@colourcmd\si@fam@colourcmd
      \fi
      \let\si@colour\si@unitcolour
    \fi
  \fi
  \si@fam@settrue
%    \end{macrocode}
%\end{macro}
%\end{macro}
%\end{macro}
%\end{macro}
%\end{macro}
%\end{macro}
%\end{macro}
%\end{macro}
% The temporary macros are needed for the \cs{ifx} tests, which need to
% be expanded once.
%    \begin{macrocode}
  \edef\si@tempa{\sfdefault}%
  \edef\si@tempb{\ttdefault}%
%    \end{macrocode}
%\begin{macro}{\si@fam@maths}
%\begin{macro}{\si@fam@text}
% The surrounding font family is only tested if matching is
% requested.  First, the defaults are set up assuming no detection
% takes place.
%    \begin{macrocode}
  \expandafter\let\expandafter\si@fam@maths
    \csname\si@mathsrm\endcsname
  \expandafter\let\expandafter\si@fam@text
    \csname\si@textrm\endcsname
  \ifsi@obeyfamily
    \si@log@debug{Font detection: checking font}%
%    \end{macrocode}
%\end{macro}
%\end{macro}
% The detection code has to check the mode currently in operation.
% Display mathematics can be handled in two ways, so this means some
% code is repeated: it is spun out to separate routines.
%    \begin{macrocode}
    \ifmmode
      \ifinner
        \si@log@debug{Font detection: inline maths}%
        \si@fam@dettext
      \else
        \si@log@debug{Font detection: display maths}%
        \ifsi@detectdisplay
          \si@fam@detmaths
        \else
          \si@fam@dettext
        \fi
      \fi
    \else
      \si@log@debug{Font detection: text}%
      \si@fam@dettext
    \fi
  \else
    \si@log@debug{Font detection: inactive}%
  \fi
%    \end{macrocode}
%\begin{macro}{\si@fam@bold}
% With the font family set, the next check is for bold text.  This
% again needs to examine the current mode. Things are a bit more
% complex than in \SIstyle as it is possible to be typesetting in
% either text or maths mode.  The bold command is set up with
% \cs{def}, as nested calls can occur.
%    \begin{macrocode}
  \def\si@fam@bold{\unboldmath\mdseries}%
  \ifsi@obeybold
    \si@log@debug{Weight detection: checking weight}%
    \ifmmode
      \ifdim\displaywidth>0pt\relax
        \ifsi@detectdisplay
          \expandafter\si@fam@ifbmaths
        \else
          \expandafter\si@fam@ifbtext
        \fi
        \si@fam@setbold
      \else
        \si@fam@ifbinline\si@fam@setbold
      \fi
    \else
      \si@fam@ifbtext\si@fam@setbold
    \fi
  \fi
%    \end{macrocode}
%\end{macro}
%\begin{macro}{\si@fam@italic}
% The value of \opt{obeyitalic} is now tested; as this does nothing
% in maths mode, a reminder is added to the log.
%    \begin{macrocode}
  \let\si@fam@italic\upshape
  \ifsi@obeyitalic
    \si@log@debug{Italic detection: checking italic}%
    \si@fam@ifitext
      {\let\si@fam@italic\relax
       \si@log@debug{Italic detection: italic}}%
  \fi}
%    \end{macrocode}
%\end{macro}
%\end{macro}
%\begin{macro}{\si@fam@detmaths}
%\begin{macro}{\si@fam@dettext}
% Two detection macros are needed for maths and text mode. This
% allows handling of the various combinations without needing too
% many code lines.
%    \begin{macrocode}
\newcommand*{\si@fam@detmaths}{%
  \ifnum\the\fam=\si@fam@sf
    \si@log@debug{Font detection: sf}%
    \expandafter\let\expandafter\si@fam@maths
      \csname\si@mathssf\endcsname
    \expandafter\let\expandafter\si@fam@text
      \csname\si@textsf\endcsname
  \else
    \ifnum\the\fam=\si@fam@tt
      \si@log@debug{Font detection: tt}%
      \expandafter\let\expandafter\si@fam@maths
        \csname\si@mathstt\endcsname
      \expandafter\let\expandafter\si@fam@text
        \csname\si@texttt\endcsname
    \else
      \si@log@debug{Font detection: rm}%
      \expandafter\let\expandafter\si@fam@maths
        \csname\si@mathsrm\endcsname
      \expandafter\let\expandafter\si@fam@text
        \csname\si@textrm\endcsname
    \fi
  \fi}
\newcommand*{\si@fam@dettext}{%
  \ifx\f@family\si@tempa
    \si@log@debug{Font detection: sf}%
    \expandafter\let\expandafter\si@fam@maths
      \csname\si@mathssf\endcsname
    \expandafter\let\expandafter\si@fam@text
      \csname\si@textsf\endcsname
  \else
    \ifx\f@family\si@tempb
      \si@log@debug{Font detection: tt}%
      \expandafter\let\expandafter\si@fam@maths
        \csname\si@mathstt\endcsname
      \expandafter\let\expandafter\si@fam@text
        \csname\si@texttt\endcsname
    \else
      \si@log@debug{Font detection: rm}%
      \expandafter\let\expandafter\si@fam@maths
        \csname\si@mathsrm\endcsname
      \expandafter\let\expandafter\si@fam@text
        \csname\si@textrm\endcsname
    \fi
  \fi}
%    \end{macrocode}
%\end{macro}
%\end{macro}
%\begin{macro}{\si@fam@setbold}
%\begin{macro}{\si@fam@boldify}
% For setting bold, a couple of control macros are needed.
%    \begin{macrocode}
\newcommand*{\si@fam@setbold}{%
  \si@log@debug{Weight detection: bold weight}%
  \let\si@fam@bold\si@fam@boldify}
\newcommand*{\si@fam@boldify}{\boldmath\bfseries}
%    \end{macrocode}
%\end{macro}
%\end{macro}
%
%\subsection{Formatting numbers}
%\begin{macro}{\num}
%\doarg{keyval options}
%\darg{number}
% The system used here is modelled on that in \numprint; the
% input is broken down into single tokens, each one is examined and
% the result is re-assembled into an output number.  However, various
% changes have been made to the system used, and so the macros here
% are not simply renamed copies of those in \numprint.  The user
% macro \cs{num} sets any local keys, then calls the number
% formatting macro on the processed number.
%    \begin{macrocode}
\si@newrobustcmd*{\num}[2][]{%
  \begingroup
    \sisetup{#1}%
    \si@fam@mode
    \si@num@intabfalse
    \si@log@debug{Processing \string\num\space input `#2'}%
    \expandafter\si@out@num\expandafter{\si@num{#2}}%
  \endgroup}
%    \end{macrocode}
%\end{macro}
%\begin{macro}{\ifsi@num@intab}
% A flag for processing inside a table is needed.
%    \begin{macrocode}
\newif\ifsi@num@intab
%    \end{macrocode}
%\end{macro}
%\begin{macro}{\si@num}
%\darg{number}
% This is the main processing macro.  Unlike the related macro in
% \numprint, the output of this macro is not subjected to any font
% changes.  That is left to one of the \cs{si@out@\ldots} macros.  No
% grouping is applied here; any call to \cs{si@num} (or any of the
% sub-macros) must be within a group as the definitions used rely on
% this.  Grouping is not applied here so that other macros can get
% the various separated parts of the input.
%    \begin{macrocode}
\newcommand*{\si@num}[1]{%
%    \end{macrocode}
% The argument of the macro is fully expanded before any processing.
% By using \cs{scantokens}, any odd problems from packages with
% active characters can be avoided.
%    \begin{macrocode}
  \si@num@fixpm
  \begingroup
    \makeatletter
    \@makeother{\,}%
    \@makeother{\.}%
    \@makeother{\+}%
    \@makeother{\-}%
    \def~{}%
    \def\,{}%
    \catcode`\~=\active\relax
    \catcode`\^=\active\relax
    \catcode`\_=\active\relax
    \everyeof{\noexpand}%
    \endlinechar\m@ne
    \protected@xdef\si@tempa{\scantokens{#1}}%
  \endgroup
%    \end{macrocode}
% Processing only takes place if there is actually something in the
% argument.  This is tested once ``hard'' spaces have been stripped
% out; if there is input other than spaces, the processor first
% checks the validity of the input, then moves on to format it.
%    \begin{macrocode}
  \si@ifnotmtarg{\si@tempa}
    {\si@num@ifvalid{\si@tempa}
      {\si@num@format{\si@tempa}}
%    \end{macrocode}
% The parser has to bailed-out, and so no further processing of
% the input is done.  Instead, whatever was passed to the macro is
% returned as supplied.
%    \begin{macrocode}
      {\si@log@err{Invalid character `#1' in numerical input}%
         {Only characters from the list
          `\si@numvalid'\MessageBreak should be present in the
          argument of the \string\num\space  macro\MessageBreak
         (or derivative such as an `s' column)}%
       {#1}}}}
%    \end{macrocode}
%\end{macro}
%\begin{macro}{\si@num@fixpm}
%\begin{macro}{\si@num@pm}
%\begin{macro}{\si@num@mp}
%\begin{macro}{\pm}
%\begin{macro}{\mp}
% With certain packages loaded, there can be issues with \cs{pm} and
% \cs{mp}.  To avoid this, the \eTeX\ \cs{protected} system is
% employed; this is only used within local groups.
%    \begin{macrocode}
\newcommand*{\si@num@fixpm}{%
  \let\si@num@pm\pm
  \let\si@num@mp\mp
  \protected\def\pm{\si@num@pm}%
  \protected\def\mp{\si@num@mp}}
%    \end{macrocode}
%\end{macro}
%\end{macro}
%\end{macro}
%\end{macro}
%\end{macro}
%\begin{macro}{\si@num@ifvalid}
%\begin{macro}{\si@num@valid}
%\darg{chars}
% Assuming that there is a non-space argument to \cs{si@num}, every
% character is checked to ensure it is valid in the context, so that
% further processing can occur without sanity checks.  If the
% character is valid, recursion occurs.
%    \begin{macrocode}
\newcommand*{\si@num@ifvalid}[1]{%
  \begingroup
    \si@switchtrue
    \expandafter\si@num@valid#1\@empty\@empty
    \ifsi@switch
      \aftergroup\@firstoftwo
    \else
      \aftergroup\@secondoftwo
    \fi
  \endgroup}
\def\si@num@valid#1#2\@empty{%
  \si@str@ifchrstr{#1}{\si@numvalid}
    {\ifx\@empty#2\@empty\else
       \si@num@valid#2\@empty\@empty\@empty
     \fi}
    {\si@switchfalse}}
%    \end{macrocode}
%\end{macro}
%\end{macro}
%\begin{macro}{\si@num@in}
%\begin{macro}{\si@num@out}
%\begin{macro}{\si@num@exp}
%\begin{macro}{\si@num@expsign}
%\begin{macro}{\si@num@mant}
%\begin{macro}{\si@num@mantsign}
%\begin{macro}{\si@num@err}
%\begin{macro}{\si@num@xpart}
%\begin{macro}{\si@num@ambig}
%\begin{macro}{\si@tab@out}
%\begin{macro}{\si@tab@expout}
% Various storage macros are needed.
%    \begin{macrocode}
\newcommand*{\si@num@in}{}
\newcommand*{\si@num@out}{}
\newcommand*{\si@num@exp}{}
\newcommand*{\si@num@expsign}{}
\newcommand*{\si@num@mant}{}
\newcommand*{\si@num@mantsign}{}
\newcommand*{\si@num@err}{}
\newcommand*{\si@num@xpart}{}
\newcommand*{\si@num@ambig}{}
\newcommand*{\si@tab@out}{}
\newcommand*{\si@tab@expout}{}
%    \end{macrocode}
%\end{macro}
%\end{macro}
%\end{macro}
%\end{macro}
%\end{macro}
%\end{macro}
%\end{macro}
%\end{macro}
%\end{macro}
%\end{macro}
%\end{macro}
%\begin{macro}{\ifsi@num@erropen}
% A flag is set up for tracking unclosed errors.
%    \begin{macrocode}
\newif\ifsi@num@erropen
%    \end{macrocode}
%\end{macro}
%\begin{macro}{\si@num@format}
%\begin{macro}{\si@num@arg}
%\darg{number}
% The number processor starts by saving |#1| (odd things happen
% otherwise).  A hook is also provided to allow modifications by
% other macros.
%    \begin{macrocode}
\newcommand*{\si@num@arg}{}
\newcommand*{\si@num@format}[1]{%
  \protected@edef\si@num@arg{#1}%
  \si@log@debug{Formatting number `\si@num@arg'}%
%    \end{macrocode}
%\end{macro}
% The storage areas are emptied.
%    \begin{macrocode}
  \renewcommand*{\si@num@in}{}%
  \renewcommand*{\si@num@exp}{}%
  \renewcommand*{\si@num@expsign}{}%
  \renewcommand*{\si@num@mant}{}%
  \renewcommand*{\si@num@mantsign}{}%
  \renewcommand*{\si@num@err}{}%
  \renewcommand*{\si@num@xpart}{}%
%    \end{macrocode}
% Any ``x-part'' is now found, leaving the first number in
% \cs{si@num@in} and anything else in \cs{si@num@xpart}.
%    \begin{macrocode}
  \si@switchfalse
  \expandafter\si@num@findxpart\si@num@arg\@empty\@empty
%    \end{macrocode}
% The input is split into an mantissa and an exponent; the flag is
% used here to indicate if an exponent is found.  The mantissa will
% end up in \cs{si@num@mant}, and the exponent in \cs{si@num@exp}.
%    \begin{macrocode}
  \si@switchfalse
  \si@num@sepmantexp{\si@num@in}%
%    \end{macrocode}
% The sign and value of the mantissa and exponent are separated; the
% mantissa is done after the exponent as this makes life easier when
% using the table-formatting fork. Checks are then needed, as a sign
% with no value is potentially-valid for the mantissa (for example
% \num{-e10}).
%    \begin{macrocode}
  \si@num@sepsign{exp}%
  \si@num@sepsign{mant}%
  \ifx\@empty\si@num@exp\@empty
    \ifx\@empty\si@num@expsign\@empty\else
      \si@log@warn{Sign but no number for `\si@num@arg'}%
    \fi
    \let\si@num@expsign\@empty
  \fi
  \ifx\@empty\si@num@mant\@empty
    \ifx\@empty\si@num@mantsign\@empty\else
      \ifx\@empty\si@num@exp\@empty
        \si@log@warn{Sign but no number for `\si@num@arg'}%
        \let\si@num@mantsign\@empty
      \fi
    \fi
  \fi
%    \end{macrocode}
% A check for negative mantissa values is made, to allow some
% colour-based trickery.
%    \begin{macrocode}
  \renewcommand*{\si@tempa}{{-}}%
  \ifx\si@num@mantsign\si@tempa
    \ifsi@colourneg
      \expandafter\expandafter\expandafter\si@fam@colourcmd
    \else
      \expandafter\expandafter\expandafter\@gobble
    \fi
  \else
    \expandafter\@gobble
  \fi
  {\si@negcolour}%
%    \end{macrocode}
% The next stage is to process the remaining data, to find the
% decimal marker and reformat correctly.  These two macros control
% this entire process.  The exponent is processed first as this makes
% life easier when the system is used to typeset tabular
% material.\footnote{The contents of \cs{si@num@predec} and
% \cs{si@num@postdec} are needed for the mantissa.}
%    \begin{macrocode}
  \si@num@procnum{exp}%
  \si@num@procnum{mant}%
%    \end{macrocode}
% If the exponent is zero, then if might need to be deleted.
%    \begin{macrocode}
  \si@str@ifonlychrs{\si@num@exp}{0\si@numdecimal}
    {\ifsi@allowzeroexp\else
       \renewcommand*{\si@num@exp}{}%
       \ifx\@empty\si@num@mant\@empty
         \renewcommand*{\si@num@mant}{1}%
       \fi
     \fi}{}%
%    \end{macrocode}
% To build up the number for typesetting, a rather complex series
% of tests is needed.  First, the ``ambiguous error'' flag is
% set if there is an exponent and the package has been asked to check
% this.  The same flag will already be true if a unit is present and
% checking is active.
%    \begin{macrocode}
  \ifx\@empty\si@num@exp\@empty\else
    \ifsi@trapambigerr
      \expandafter\expandafter\expandafter\si@num@ambigerrtrue
    \fi
  \fi
%    \end{macrocode}
%\begin{macro}{\si@num@out}
%\begin{macro}{\si@tab@out}
%\begin{macro}{\si@tab@expout}
% Processing now divides, as when used with the |S| column some extra
% steps are needed.  Inside a table, the macro \cs{si@tab@out} holds
% the part of the mantissa before the decimal sign.  The post-decimal
% part then ends up in \cs{i@num@out}.  On the other hand, under
% normal circumstances the entire mantissa should be copied to
% \cs{si@num@out}.
%    \begin{macrocode}
  \protected@edef\si@num@out{%
    \ensuremath{{\si@num@mantsign}}\si@num@mant}%
  \renewcommand*{\si@tempa}{num}%
  \ifsi@num@intab
    \protected@edef\si@tab@out{%
      \ensuremath{{\si@num@mantsign}}\si@num@predec}%
    \protected@edef\si@num@out{\si@num@postdec}%
    \renewcommand*{\si@tempa}{tab}%
  \fi
%    \end{macrocode}
% Now there is the error part to handle.  For tables, a check has to
% be made so the error ends up in the correct part of the number. The
% value of \cs{si@tempa} is used to track this., and so is set up
% here.
%    \begin{macrocode}
  \ifx\@empty\si@num@postdec\@empty\else
    \renewcommand*{\si@tempa}{num}%
  \fi
  \ifx\@empty\si@num@err\@empty\else
%    \end{macrocode}
% If there is an error, and it is begin separated, the problem arises
% of the potential for ambiguous values.  This can only apply outside
% of a table, as \opt{seperr} is disabled in tabular material.
%    \begin{macrocode}
    \ifsi@seperr
      \ifsi@num@ambigerr
        \protected@edef\si@num@out{%
          \ensuremath{\si@openerr}\si@num@out}%
        \si@repeatunitsfalse
        \expandafter\si@num@erropentrue
      \else
%    \end{macrocode}
% If there is an exponential part and an error, errors are being
% separated and ambiguous errors are not trapped, then there is work
% to do.  The exponent part is added to the \emph{error}, and deleted
% from the mantissa if necessary.
%    \begin{macrocode}
        \ifsi@trapambigerr\else
          \ifx\@empty\si@num@exp\@empty\else
            \protected@edef\si@num@err{%
              \si@num@err\expandafter\@car\si@numexp\@nil
              \si@num@expsign\si@num@exp}%
            \ifsi@repeatunits\else
              \renewcommand*{\si@num@exp}{}%
              \renewcommand*{\si@num@expsign}{}%
            \fi
          \fi
        \fi
      \fi
%    \end{macrocode}
% If \opt{seperr} is not in force, the error and mantissa have to be
% recombined.  This is handled so that the same macro deals with
% tables and normal processing.
%    \begin{macrocode}
    \else
      \expandafter\protected@edef\csname
        si@\si@tempa @out\endcsname{%
          \si@num@out\ensuremath{\si@errspace}\ensuremath
          {\si@openerr}\si@num@err\ensuremath{\si@closeerr}}%
      \renewcommand*{\si@num@err}{}%
    \fi
  \fi
%    \end{macrocode}
% The main reconstruction can now occur.  This performs various
% checks on the validity of the input, and adds the necessary
% ``filler'' between the supplied data.
%    \begin{macrocode}
  \renewcommand*{\si@tempa}{num@out}%
  \ifsi@num@erropen
    \renewcommand*{\si@tempa}{num@ambig}%
  \fi
  \ifsi@num@intab
    \renewcommand*{\si@tempa}{tab@expout}%
  \fi
  \ifx\@empty\si@num@exp\@empty
    \ifx\@empty\si@num@mant\@empty
      \si@log@err{Invalid number format `\si@num@arg'}
        {Something is wrong with the number format; does it
         contain \MessageBreak any numbers (from the list
         `\si@numdigits')?}%
      \renewcommand*\si@num@out{}%
    \fi
  \else
    \ifx\@empty\si@num@mant\@empty\else
      \expandafter\protected@edef\csname
        si@\si@tempa\endcsname{%
          \csname si@\si@tempa\endcsname\ensuremath{{}%
          \si@expproduct{}}}%
    \fi
    \expandafter\protected@edef\csname
      si@\si@tempa\endcsname{%
        \csname si@\si@tempa\endcsname\si@expbase
        \textsuperscript{\ensuremath{\si@num@expsign}%
        \si@num@exp}}%
  \fi
%    \end{macrocode}
%\end{macro}
%\end{macro}
%\end{macro}
% With everything done, the result is output.
%    \begin{macrocode}
  \ifsi@num@intab\else
    \expandafter\si@num@out
  \fi
%    \end{macrocode}
% If there is anything inside the ``x-part'', then there is now
% more work to do.
%    \begin{macrocode}
  \ifx\@empty\si@num@err\@empty\else
    \expandafter\si@num@procerr
  \fi
  \ifsi@num@erropen
    \expandafter\si@out@num\expandafter{%
      \ensuremath{\si@closeerr}}%
      \ifx\@empty\si@num@ambig\@empty\else
        \expandafter\si@out@num\expandafter{\si@num@ambig}%
        \renewcommand*{\si@num@ambig}{}%
    \fi
  \fi
  \si@num@erropenfalse
  \ifx\@empty\si@num@xpart\@empty\else
    \expandafter\si@num@sepxpart
  \fi}
%    \end{macrocode}
%\end{macro}
%\begin{macro}{\si@num@findxpart}
% Before processing the number, any multiplied parts have to be found
% and removed.  As there can be more than one product sign, the
% building process here has no error checking.
%    \begin{macrocode}
\def\si@num@findxpart#1#2\@empty{%
  \si@str@ifchrstr{#1}{\si@numprod}
    {\si@switchtrue\si@seperrfalse}{}%
  \ifsi@switch
    \protected@edef\si@num@xpart{\si@num@xpart#1}%
  \else
    \protected@edef\si@num@in{\si@num@in#1}%
  \fi
  \ifx\@empty#2\@empty\else
    \si@num@findxpart#2\@empty
  \fi}
%    \end{macrocode}
%\end{macro}
%\begin{macro}{\si@num@sepmantexp}
%\begin{macro}{\si@num@mantexp}
%\darg{number}
% Splitting the mantissa and exponent first checks for characters to
% gobble, which are simply thrown away.  For any other input, there
% are two possibilities.  If the character is an exponent marker,
% then the package switches from collecting the mantissa to
% collecting the exponent (after a sanity check).  All other
% characters are added to either the mantissa or the exponent, as
% appropriate.
%    \begin{macrocode}
\newcommand*{\si@num@sepmantexp}[1]{%
  \expandafter\si@num@mantexp#1\@empty\@empty}
\def\si@num@mantexp#1#2\@empty{%
  \si@str@ifchrstr{#1}{\si@numgobble}
    {\si@log@debug{Gobbling `#1' in \si@num@arg}}
    {\si@str@ifchrstr{#1}{\si@numexp}
       {\ifsi@switch
          \si@log@err{Duplicate exponent marker found}
            {Only a single exponent character \MessageBreak
            (from the list `\si@numexp')\MessageBreak may
            occur in a numerical argument}%
        \else
          \si@log@debug{Exponent marker `#1' found in
            `\si@num@arg'}%
        \fi
        \si@switchtrue}%
       {\ifsi@switch
          \expandafter\si@num@addexp
        \else
          \expandafter\si@num@addmnt
        \fi
        {#1}}}%
%    \end{macrocode}
% If the recursion has not bottomed out, another loop occurs.
%    \begin{macrocode}
  \ifx\@empty#2\@empty
    \expandafter\@gobble
  \else
    \expandafter\si@num@sepmantexp
  \fi
  {#2}}
%    \end{macrocode}
%\end{macro}
%\end{macro}
%\begin{macro}{\si@num@addmnt}
%\begin{macro}{\si@num@addexp}
%\darg{char}
% To allow \cs{expandafter} use in the above, the actual addition to
% the appropriate macro is handled here.  Two helper macros are
% needed.
%    \begin{macrocode}
\newcommand*{\si@num@addmnt}[1]{%
  \si@num@addmntexp{#1}{mant}{mantissa}}
\newcommand*{\si@num@addexp}[1]{%
  \si@num@addmntexp{#1}{exp}{exponent}}
%    \end{macrocode}
%\end{macro}
%\end{macro}
%\begin{macro}{\si@num@addmntexp}
%\darg{char}
%\darg{either \texttt{mant} or \texttt{exp}}
%\darg{info-text}
% Then the business end.
%    \begin{macrocode}
\newcommand*{\si@num@addmntexp}[3]{%
  \si@log@debug{Adding `#1' to #3 for `\si@num@arg'}%
  \expandafter\protected@edef\csname si@num@#2\endcsname{%
    \csname si@num@#2\endcsname#1}}
%    \end{macrocode}
%\end{macro}
%\begin{macro}{\si@num@sepsign}
%\darg{either \texttt{mant} or \texttt{exp}}
% The input is now tested for a sign.  If one exists, it is
% transferred into the \cs{si@num@}|#1sign| storage macro, with the
% remained of the number in \cs{si@num@}|#1|.  The only check made
% directly here is that there is something to process.
%    \begin{macrocode}
\newcommand*{\si@num@sepsign}[1]{%
  \expandafter\ifx\expandafter\@empty
    \csname si@num@#1\endcsname\@empty
    \expandafter\@gobble
  \else
    \expandafter\si@num@gensign
  \fi
  {#1}}
%    \end{macrocode}
%\end{macro}
%\begin{macro}{\si@num@gensign}
%\darg{either \texttt{mant} or \texttt{exp}}
% The sign generator starts by calling the procedure to check if the
% input contains a valid one- or two-digit sign.  The results are
% returned as \cs{si@num@sign} and \cs{si@num@value}.
%    \begin{macrocode}
\newcommand*{\si@num@gensign}[1]{%
  \expandafter\expandafter\expandafter\si@num@findsign
    \csname si@num@#1\endcsname\@empty\@empty
%    \end{macrocode}
% If no sign has been found, then there may be a need to add one
% anyway.
%    \begin{macrocode}
  \ifx\@empty\si@num@sign\@empty
    \ifx\@empty\si@num@value\@empty
      \expandafter\expandafter\expandafter\@gobble
    \else
      \expandafter\expandafter\expandafter\si@num@addsign
    \fi
  \else
    \expandafter\@gobble
  \fi
  {#1}%
%    \end{macrocode}
% The appropriate storage areas are now assigned.
%    \begin{macrocode}
  \expandafter\let\csname si@num@#1sign\endcsname\si@num@sign
  \expandafter\let\csname si@num@#1\endcsname\si@num@value}
%    \end{macrocode}
%\end{macro}
%\begin{macro}{\si@num@findsign}
%\begin{macro}{\si@num@sign}
%\begin{macro}{\si@num@value}
% The first one or two characters of the mantissa or exponent may
% contain a sign.  To test for this, the first two characters of the
% number are split off, and examined.  Two characters are used so
% that \cs{pm} and \cs{mp} can be represented by |+-| and |-+|,
% respectively. To allow the user to alter the valid signs, but
% retain this conversion, the generic character test is used before
% checking specific matches.
%    \begin{macrocode}
\newcommand*{\si@num@sign}{}
\def\si@num@findsign#1#2#3\@empty{%
  \si@num@delplusfalse
  \si@str@ifchrstr{#1}{\si@numsign}{%
    \si@str@ifchrstr{#2}{\si@numsign}{%
      \if +#1%
        \if -#2%
          \si@log@debug{Found sign combination +- for
            `\si@num@arg'}%
          \renewcommand*{\si@num@sign}{{\si@pm}}%
        \else
          \si@log@inf{Unknown sign combination `#1#2'}%
          \renewcommand*{\si@num@sign}{{#1#2}}%
        \fi
      \else
        \if -#1%
          \if +#2%
            \si@log@debug{Found sign combination -+ for
              `\si@num@arg'}%
            \renewcommand*{\si@num@sign}{{\mp}}%
          \else
            \si@log@inf{Unknown sign combination `#1#2'}%
            \renewcommand*{\si@num@sign}{{#1#2}}%
          \fi
        \else
          \si@log@inf{Unknown sign combination `#1#2'}%
          \renewcommand*{\si@num@sign}{{#1#2}}%
        \fi
      \fi
      \protected@edef\si@num@value{#3}}%
%    \end{macrocode}
% Only one valid sign character.
%    \begin{macrocode}
      {\si@log@debug{Found single sign character `#1' for
         `\si@num@arg'}%
       \renewcommand*{\si@num@sign}{{#1}}%
       \if +#1%
         \ifsi@retainplus\else
           \expandafter\expandafter\expandafter\si@num@killsign
         \fi
       \fi
       \protected@edef\si@num@value{#2#3}}}%
%    \end{macrocode}
% No valid sign, so \cs{@empty} is returned for the sign .
%    \begin{macrocode}
    {\si@log@debug{No sign found for `\si@num@arg'}%
     \renewcommand*{\si@num@sign}{}%
     \protected@edef\si@num@value{#1#2#3}}}
%    \end{macrocode}
%\end{macro}
%\end{macro}
%\end{macro}
%\begin{macro}{\ifsi@num@delplus}
%\begin{macro}{\si@num@killsign}
% A simple spin-out to remove a plus sign.  A flag is set as it might
% be useful to know this.
%    \begin{macrocode}
\newif\ifsi@num@delplus
\newcommand*{\si@num@killsign}{%
  \si@num@delplustrue
  \renewcommand*{\si@num@sign}{}}
%    \end{macrocode}
%\end{macro}
%\end{macro}
%\begin{macro}{\si@num@addsign}
%\begin{macro}{\si@num@asign}
%\darg{either \texttt{mant} or \texttt{exp}}
% The macro to add a sign to an unsigned number has to check whether
% this is a mantissa or an exponent.  The result is still placed in
% \cs{si@num@sign} for ease of processing later.
%    \begin{macrocode}
\newcommand*{\si@num@addsign}[1]{%
  \begingroup
    \renewcommand*{\si@tempa}{#1}%
    \renewcommand*{\si@tempb}{mant}%
    \ifx\si@tempa\si@tempb
      \aftergroup\@firstoftwo
    \else
      \aftergroup\@secondoftwo
    \fi
  \endgroup
  {\ifsi@num@signmant
     \expandafter\si@num@asign
   \else
     \expandafter\@gobble
   \fi
   {mantissa}}
  {\ifsi@num@signexp
     \expandafter\si@num@asign
   \else
     \expandafter\@gobble
   \fi
   {exponent}}}
\newcommand*{\si@num@asign}[1]{%
  \let\si@num@sign\si@sign
  \si@log@debug{Adding sign \si@sign\space to #1 for
    `\si@num@arg'}}
%    \end{macrocode}
%\end{macro}
%\end{macro}
%\begin{macro}{\si@num@procnum}
%\darg{either \texttt{mant} or \texttt{exp}}
% The control macro for processing the number (plus any extra
% characters).
%    \begin{macrocode}
\newcommand*{\si@num@procnum}[1]{%
  \expandafter\ifx\expandafter\@empty
    \csname si@num@#1\endcsname\@empty
    \expandafter\@gobble
  \else
    \expandafter\si@num@finddigits
  \fi
  {#1}}
%    \end{macrocode}
%\end{macro}
%\begin{macro}{\si@num@predec}
%\begin{macro}{\si@num@postdec}
% Two new storage areas are defined.
%    \begin{macrocode}
\newcommand*{\si@num@predec}{}
\newcommand*{\si@num@postdec}{}
%    \end{macrocode}
%\end{macro}
%\end{macro}
%\begin{macro}{\si@num@finddigits}
%\darg{either \texttt{mant} or \texttt{exp}}
% The core digit processor divides the number into the parts before
% and after the decimal point marker.  The temporary switch is used
% to indicate finding a decimal marker.
%    \begin{macrocode}
\newcommand*{\si@num@finddigits}[1]{%
  \renewcommand*{\si@num@predec}{}%
  \renewcommand*{\si@num@postdec}{}%
  \si@switchfalse
  \expandafter\expandafter\expandafter\si@num@digits
    \csname si@num@#1\endcsname\@empty\@empty
%    \end{macrocode}
% Tests are now made to see if padding zeros are needed.  The
% trailing test needs to verify if a decimal marker was found, as
% well as if a zero is needed.
%    \begin{macrocode}
  \ifx\@empty\si@num@predec\@empty
    \ifsi@num@padlead
      \expandafter\expandafter\expandafter\si@num@addprezero
    \fi
  \fi
  \ifx\@empty\si@num@postdec\@empty
    \ifsi@num@padtrail
      \ifsi@switch
        \expandafter\expandafter\expandafter\expandafter
          \expandafter\expandafter\expandafter
          \si@num@addpostzero
      \fi
    \fi
  \fi
%    \end{macrocode}
% The next checks to make concern input validity in a more
% mathematical sense.  First, if the number is zero, then no sign
% should be given under any circumstances.  Then leading zeros need
% to be removed from the input.  This is slightly complicated by the
% potential presence of ``extra'' characters.
%    \begin{macrocode}
  \si@num@unsign{#1}%
  \ifx\@empty\si@num@predec\@empty
  \else
    \expandafter\si@num@nozero
  \fi
%    \end{macrocode}
% A sanity check is made to ensure that the supplied number consisted
% of more than a decimal marker.
%    \begin{macrocode}
  \ifx\@empty\si@num@predec\@empty
    \ifx\@empty\si@num@postdec\@empty
      \expandafter\expandafter\expandafter\@gobble
    \else
      \expandafter\expandafter\expandafter\si@num@sepdigits
    \fi
  \else
    \expandafter\si@num@sepdigits
  \fi
  {#1}}
%    \end{macrocode}
%\end{macro}
%\begin{macro}{\si@num@digits}
% The \cs{si@num@digits} macro compares each character in the input
% against the list of characters valid at this stage: numbers,
% decimal markers and ``extra'' characters.
%    \begin{macrocode}
\def\si@num@digits#1#2\@empty{%
  \si@str@ifchrstr{#1}{\si@numdecimal}
    {\ifsi@switch
       \si@log@err{Duplicate decimal marker in `\si@num@arg'}
         {Only a single decimal marker (from the list
            `\si@numdecimal')\MessageBreak may occur in a
            numerical argument}%
     \else
       \si@log@debug{Found decimal marker `#1' in
         `\si@num@arg'}%
       \expandafter\si@switchtrue
     \fi}
%    \end{macrocode}
% The earlier code only checks for a sign at the start of the text. A
% check is therefore needed for a sign after the first two
% characters; if one is found, it is ignored.
%    \begin{macrocode}
    {\si@str@ifchrstr{#1}{\si@numsign}
       {\si@log@err{Misplaced sign character
          `#1' in `\si@num@arg'}
          {Sign characters `\si@numsign' can only
            occur\MessageBreak at the start of a number}}
%    \end{macrocode}
% The current character is added to the appropriate stack; this is
% ``spun out'' to avoid problems with expansion of the switch code.
%    \begin{macrocode}
       {\ifsi@switch
          \expandafter\si@num@post
        \else
          \expandafter\si@num@pre
        \fi
        {#1}}}%
  \ifx\@empty#2\@empty\else
    \si@num@digits#2\@empty\@empty
  \fi}
%    \end{macrocode}
%\end{macro}
%\begin{macro}{\si@num@pre}
%\begin{macro}{\si@num@post}
%\darg{char}
% Storage of the result is spun out.
%    \begin{macrocode}
\newcommand*{\si@num@pre}[1]{%
  \si@num@prepost{#1}{pre}{integer}}
\newcommand*{\si@num@post}[1]{%
  \si@num@prepost{#1}{post}{decimal}}
%    \end{macrocode}
%\end{macro}
%\end{macro}
%\begin{macro}{\si@num@prepost}
%\darg{char}
%\darg{either \texttt{pre} or \texttt{post}}
%\darg{info-text}
% Then actually stored here.
%    \begin{macrocode}
\newcommand*{\si@num@prepost}[3]{%
  \expandafter\protected@edef\csname si@num@#2dec\endcsname{%
    \csname si@num@#2dec\endcsname#1}%
  \si@log@debug{Adding `#1' to #3 part for `\si@num@arg'}}
%    \end{macrocode}
%\end{macro}
%\begin{macro}{\si@num@addprezero}
%\begin{macro}{\si@num@addpostzero}
% A similar set of macros are used for the padding zeros.
%    \begin{macrocode}
\newcommand*{\si@num@addprezero}{%
  \si@num@addpzero{pre}{leading}}
\newcommand*{\si@num@addpostzero}{%
  \si@num@addpzero{post}{trailing}}
%    \end{macrocode}
%\end{macro}
%\end{macro}
%\begin{macro}{\si@num@addpzero}
%\darg{either \texttt{pre} or \texttt{post}}
% In this case, there is no argument to be passed along.
%    \begin{macrocode}
\newcommand*{\si@num@addpzero}[2]{%
  \si@log@debug{Adding #2 zero for `\si@num@arg'}%
  \@namedef{si@num@#1dec}{0}}
%    \end{macrocode}
%\end{macro}
%\begin{macro}{\si@num@unsign}
%\begin{macro}{\si@num@nosign}
%\darg{either \texttt{mant} or \texttt{exp}}
% The trap for a sign with zero numerical input is made.  First, a
% check is made to see if there is a sign to worry about.  The pre-
% and post-decimal parts are then examined, to see if they contain
% something other than ``0'' or an extra character.
%    \begin{macrocode}
\newcommand*{\si@num@unsign}[1]{%
  \expandafter\ifx\expandafter\@empty
    \csname si@num@#1sign\endcsname\@empty
    \expandafter\@gobble
  \else
    \expandafter\si@num@nosign
  \fi
  {#1}}
\newcommand*{\si@num@nosign}[1]{%
  \begingroup
    \si@switchtrue
    \si@str@ifonlychrs{\si@num@predec\si@num@postdec}{0}
      {\si@switchfalse}{}%
    \ifsi@switch
      \aftergroup\@gobble
    \else
      \aftergroup\@firstofone
    \fi
  \endgroup
  {\si@log@debug{Zero value: removing any sign}%
   \ifsi@ang@sign\else
     \@namedef{si@num@#1sign}{}%
   \fi}}
%    \end{macrocode}
%\end{macro}
%\end{macro}
%\begin{macro}{\si@num@nozero}
%  A very short test for a totally zero pre-decimal component.
%    \begin{macrocode}
\newcommand*{\si@num@nozero}{%
  \si@str@ifonlychrs{\si@num@predec}{0}
    {\renewcommand*{\si@num@predec}{0}}{}}
%    \end{macrocode}
%\end{macro}
%\begin{macro}{\si@num@decimalhook}
% A hook is needed to attach things inside the group to happen
% afterwards, if the number is a decimal.
%    \begin{macrocode}
\newcommand*{\si@num@decimalhook}{}
%    \end{macrocode}
%\end{macro}
%\begin{macro}{\si@num@sepdigits}
%\darg{either \texttt{mant} or \texttt{exp}}
% The \cs{si@num@sepdigits} macro is only called if at least one of
% the mantissa and exponent contain something to output.
%    \begin{macrocode}
\newcommand*{\si@num@sepdigits}[1]{%
%    \end{macrocode}
% First an overall check is needed for additional characters.  By
% altering the contents of \cs{si@numextra}, the same code can be
% shared by two different checks.
%    \begin{macrocode}
  \begingroup
    \let\si@numextra\si@numaddn
    \protected@edef\si@tempa{\si@num@predec\si@num@postdec}%
    \si@num@ifextra{\si@tempa}
      {\aftergroup\@gobble}
      {\aftergroup\@firstofone}%
  \endgroup
%    \end{macrocode}
% Separation of the error in a number from the number itself is only
% attempted for the mantissa.
%    \begin{macrocode}
  {\renewcommand*{\si@tempb}{mant}%
   \renewcommand*{\si@tempc}{#1}%
   \ifx\si@tempb\si@tempc
     \expandafter\si@num@checkerr
   \fi}%
%    \end{macrocode}
%\changes{v1.1}{2008/08/22}{Fixed-length code generalised}
%\changes{v1.1}{2008/08/26}{Length-fixing only applies to
%  manitssa}
% If both parts of the number contain only digits, then any rounding
% can be attempted if there is no error part.  Otherwise, the input
% must be left alone.
%    \begin{macrocode}
  \protected@edef\si@tempa{\si@num@predec\si@num@postdec}%
  \expandafter\si@str@ifonlychrs\expandafter{\si@tempa}
    {0123456789}
    {\ifx\@empty\si@num@err\@empty
       \renewcommand*{\si@tempb}{mant}%
       \renewcommand*{\si@tempc}{#1}%
       \ifx\si@tempb\si@tempc
         \expandafter\expandafter\expandafter\si@num@fixlength
       \fi
     \fi}{}%
%    \end{macrocode}
% If the pre-decimal part contains nothing except numbers, then
% digit separation is carried out.
%    \begin{macrocode}
  \si@num@ifextra{\si@num@predec}{}
    {\expandafter\si@num@int\expandafter{\si@num@predec}}%
%    \end{macrocode}
% A decision is made about the decimal sign, then digit separation
% occurs on the post-decimal part of the number.
%    \begin{macrocode}
  \renewcommand*{\si@tempc}{}%
  \ifx\@empty\si@num@postdec\@empty\else
    \si@num@decimalhook
    \renewcommand*{\si@tempc}{%
      \ensuremath{{\si@decimalsymbol}}}%
    \si@num@ifextra{\si@num@postdec}{}
      {\expandafter\si@num@dec\expandafter{\si@num@postdec}}%
  \fi
%    \end{macrocode}
% The construction is finalised by re-combining the number.
%    \begin{macrocode}
  \expandafter\protected@edef\csname si@num@#1\endcsname
    {\si@num@predec\si@tempc\si@num@postdec}}
%    \end{macrocode}
%\end{macro}
%\begin{macro}{\si@num@ifextra}
%\begin{macro}{\si@num@extra}
%\darg{number}
% A relatively simple test for ``extra'' characters.  Once again, a
% bit of group trickery is used.
%    \begin{macrocode}
\newcommand*{\si@num@ifextra}[1]{%
  \begingroup
    \si@switchfalse
    \expandafter\si@num@extra#1\@empty\@empty
    \ifsi@switch
      \si@log@debug{Found `extra' characters in `#1'}%
      \aftergroup\@firstoftwo
    \else
      \aftergroup\@secondoftwo
    \fi
  \endgroup}
\def\si@num@extra#1#2\@empty{%
  \ifx\@empty#1\@empty\else
    \si@str@ifchrstr{#1}{\si@numextra}{\si@switchtrue}{}%
    \ifx\@empty#2\@empty\else
      \si@num@extra#2\@empty\@empty
    \fi
  \fi}
%    \end{macrocode}
%\end{macro}
%\end{macro}
%\begin{macro}{\ifsi@num@ambigerr}
%\begin{macro}{\si@num@checkerr}
% When separating out an error from a number, the first step is to
% see if there is a decimal part to the number.  If so, any error
% must be in that part of the decimal part; it is not possible to
% have an error starting in the integer part and continuing into the
% decimal part.
%    \begin{macrocode}
\newif\ifsi@num@ambigerr
\newcommand*{\si@num@checkerr}{%
  \ifx\@empty\si@num@postdec\@empty
    \expandafter\si@num@preerr
  \else
    \expandafter\si@num@posterr
  \fi}
%    \end{macrocode}
%\end{macro}
%\end{macro}
%\begin{macro}{\si@num@preerr}
% Errors in integers are easy to handle.  After finding the error,
% the result is simply stored in the error macro \cs{si@num@err}.
%    \begin{macrocode}
\newcommand*{\si@num@preerr}{%
  \si@num@seperr{pre}%
  \ifx\@empty\si@tempb\@empty\else
    \expandafter\renewcommand\expandafter*\expandafter
      \si@num@err\expandafter{\si@tempb}%
  \fi}
%    \end{macrocode}
%\end{macro}
%\begin{macro}{\si@num@posterr}
%\begin{macro}{\si@num@psterr}
% Life is more complex for an error in the decimal part.  This is
% found, then it may need zero-filling or the insertion of a decimal
% point.  This depends on whether the number of error digits is
% larger than the number of post-decimal digits.
%    \begin{macrocode}
\newcommand*{\si@num@posterr}{%
  \si@num@seperr{post}%
  \ifx\@empty\si@tempb\@empty\else
    \ifsi@seperr
      \expandafter\expandafter\expandafter\si@num@psterr
    \else
      \let\si@num@err\si@tempb
    \fi
  \fi}
\newcommand*{\si@num@psterr}{%
    \si@num@cntdigits{\si@tempb}%
    \si@tempcntb\si@tempcnta\relax
    \si@num@cntdigits{\si@num@postdec}%
    \ifnum\si@tempcnta<\si@tempcntb\relax
      \expandafter\si@num@largeerr
    \else
      \expandafter\si@num@smallerr
    \fi}
%    \end{macrocode}
%\end{macro}
%\end{macro}
%\begin{macro}{\si@num@seperr}
%\begin{macro}{\si@num@finderr}
%\darg{either \texttt{pre} or \texttt{post}}
% The usual hand-off is made for searching for an error.
%    \begin{macrocode}
\newcommand*{\si@num@seperr}[1]{%
  \si@switchfalse
  \renewcommand*{\si@tempa}{}%
  \renewcommand*{\si@tempb}{}%
  \expandafter\expandafter\expandafter\si@num@finderr
    \csname si@num@#1dec\endcsname\@empty\@empty
  \ifx\@empty\si@tempb\@empty\else
    \expandafter\let\csname si@num@#1dec\endcsname\si@tempa
  \fi}
\def\si@num@finderr#1#2\@empty{%
%    \end{macrocode}
% First a check for the opening character of an error.
%    \begin{macrocode}
  \si@str@ifchrstr{#1}{\si@numopenerr}
%    \end{macrocode}
% If the switch is set when an error is found, then something is
% wrong.
%    \begin{macrocode}
    {\ifsi@switch
       \si@log@err{Invalid error in number}
         {The numerical argument \si@num@arg\space has two (or
           more)\MessageBreak error-opening characters}%
     \else
       \expandafter\si@switchtrue
     \fi}
%    \end{macrocode}
% The end-of-error character has to be the last item of the input,
% and an error needs to start before it ends.
%    \begin{macrocode}
    {\si@str@ifchrstr{#1}{\si@numcloseerr}
       {\ifsi@switch
          \ifx\@empty#2\@empty\else
            \si@log@err{Invalid error in number}
              {The numerical argument \si@num@arg\space has an
              error-closing before the last character}%
          \fi
        \else
          \si@log@err{Invalid error in number}
           {The numerical argument \si@num@arg\space has an
              error-closing character\MessageBreak but no
              error-opening one}%
        \fi}
%    \end{macrocode}
% The input must be a digit.  It is stored in the appropriate area.
%    \begin{macrocode}
       {\ifsi@switch
          \expandafter\si@num@addtmpb
        \else
          \expandafter\si@num@addtmpa
        \fi
        {#1}}}%
  \ifx\@empty#2\@empty\else
    \si@num@finderr#2\@empty
  \fi}
%    \end{macrocode}
%\end{macro}
%\end{macro}
%\begin{macro}{\si@num@addtmpa}
%\begin{macro}{\si@num@addtmpb}
%\darg{number}
% Some quick methods for adding to the temporary macros with \cs{if}
% expansion.
%    \begin{macrocode}
\newcommand*{\si@num@addtmpa}[1]{\si@num@addtmp{a}{#1}}
\newcommand*{\si@num@addtmpb}[1]{\si@num@addtmp{b}{#1}}
%    \end{macrocode}
%\end{macro}
%\end{macro}
%\begin{macro}{\si@num@addtmp}
%\darg{either \texttt{a} or \texttt{b}}
%\darg{number}
% The addition is committed.
%    \begin{macrocode}
\newcommand*{\si@num@addtmp}[2]{%
  \expandafter\protected@edef\csname si@temp#1\endcsname{%
    \csname si@temp#1\endcsname#2}}
%    \end{macrocode}
%\end{macro}
%\begin{macro}{\si@num@cntdigits}
%\begin{macro}{\si@num@cntdgt}
%\darg{number}
% A recursive system for counting the number of characters in a given
% input; only used here to count digits in a number.
%    \begin{macrocode}
\newcommand*{\si@num@cntdigits}[1]{%
  \si@tempcnta\z@\relax
  \expandafter\si@num@cntdgt#1\@empty\@empty}
\def\si@num@cntdgt#1#2\@empty{%
  \ifx\@empty#1\@empty\else
    \advance\si@tempcnta\@ne\relax
  \fi
  \ifx\@empty#2\@empty\else
    \expandafter\si@num@cntdgt#2\@empty
  \fi}
%    \end{macrocode}
%\end{macro}
%\end{macro}
%\begin{macro}{\si@num@smallerr}
%\begin{macro}{\si@num@serr}
% For handling an error with no more digits than the post-decimal
% part of a number, zero-padding is undertaken before adding a
% decimal sign and (possibly) leading zero.
%    \begin{macrocode}
\newcommand*{\si@num@smallerr}{%
  \si@tempcntb\si@tempcnta\relax
  \si@num@serr
  \protected@edef\si@num@err{%
    \ifsi@num@padlead0\fi\expandafter\@car\si@numdecimal\@nil
    \si@tempb}}
\newcommand*{\si@num@serr}{%
  \si@num@cntdigits{\si@tempb}%
  \ifnum\si@tempcnta=\si@tempcntb\relax\else
    \protected@edef\si@tempb{0\si@tempb}%
    \expandafter\si@num@serr
  \fi}
%    \end{macrocode}
%\end{macro}
%\end{macro}
%\begin{macro}{\si@num@largeerr}
%\changes{v1.0m}{2008/09/16}{Fixed problem with error separation
%  crossing decimal boundary}
%\begin{macro}{\si@num@lerr}
%\begin{macro}{\si@num@movedigit}
% When the error has more digits than the decimal part, some
% shuffling is needed.
%    \begin{macrocode}
\newcommand*{\si@num@largeerr}{%
  \renewcommand*{\si@tempa}{}%
  \si@tempcntb\si@tempcnta\relax
  \si@num@lerr
  \protected@edef\si@num@err{%
    \si@tempa\expandafter\@car\si@numdecimal\@nil\si@tempb}}
\newcommand*{\si@num@lerr}{%
  \si@num@cntdigits{\si@tempb}%
  \ifnum\si@tempcnta=\si@tempcntb\relax\else
    \expandafter\si@num@movedigit\si@tempb\@empty\@empty
    \si@num@lerr
  \fi}
\def\si@num@movedigit#1#2\@empty{%
  \protected@edef\si@tempa{\si@tempa#1}%
  \protected@edef\si@tempb{#2}}
%    \end{macrocode}
%\end{macro}
%\end{macro}
%\end{macro}
%\begin{macro}{\si@num@fixlength}
% The checks for fixing number length are spun out here.
%    \begin{macrocode}
\newcommand*{\si@num@fixlength}{%
  \ifsi@fixdp
    \expandafter\si@num@fixdp
  \else
    \ifsi@fixsf
      \expandafter\expandafter\expandafter\si@num@fixsf
    \fi
  \fi}
%    \end{macrocode}
%\end{macro}
%\begin{macro}{\si@num@fixdp}
% The test for fixing decimal places starts by counting up the number
% of decimal digits.  The number is then padded, rounded or left
% alone.
%    \begin{macrocode}
\newcommand*{\si@num@fixdp}{%
  \si@num@cntdigits{\si@num@postdec}%
  \ifx\@empty\si@num@postdec\@empty
    \si@tempcnta\z@\relax
  \fi
  \ifnum\si@tempcnta>\si@num@dp\relax
    \expandafter\si@num@round
  \else
    \ifnum\si@tempcnta<\si@num@dp\relax
      \expandafter\expandafter\expandafter\si@num@pad
    \fi
  \fi}
%    \end{macrocode}
%\end{macro}
%\begin{macro}{\si@num@pad}
%\begin{macro}{\si@num@pd}
% Padding a number is a relatively easy matter.  The number of digits
% is increased by adding ``$0$'' recursively.
%    \begin{macrocode}
\newcommand*{\si@num@pad}{%
  \si@log@debug{Padding to \the\si@num@dp\space digits}%
  \loop\ifnum\si@tempcnta<\si@num@dp\si@num@pd\repeat}
\newcommand*{\si@num@pd}{%
  \advance\si@tempcnta\@ne\relax
  \protected@edef\si@num@postdec{\si@num@postdec0}}
%    \end{macrocode}
%\end{macro}
%\end{macro}
%\begin{macro}{\si@num@rndtarget}
%\changes{v1.1}{2008/08/25}{New counter}
% A scratch counter is created to allow some code sharing between
% fixing decimal places and significant figures.
%    \begin{macrocode}
\newcount\si@num@rndtarget
%    \end{macrocode}
%\end{macro}
%\begin{macro}{\si@num@round}
%\begin{macro}{\si@num@prernd}
%\begin{macro}{\si@num@postrnd}
%\begin{macro}{\si@num@rndstart}
%\changes{v1.1}{2008/08/25}{Separated out from \cs{si@num@round}}
% Rounding a number is more complicated.  The main macro here relies
% on several others, and is simply a set-up and hand-off system.
%    \begin{macrocode}
\newcommand*{\si@num@prernd}{}
\newcommand*{\si@num@postrnd}{}
\newcommand*{\si@num@round}{%
  \si@log@debug{Rounding to \the\si@num@dp\space digits}%
  \si@num@rndtarget\si@num@dp\relax
  \si@num@rndstart}
\newcommand*{\si@num@rndstart}{%
  \si@num@reverse{\si@num@postdec}%
  \si@num@reverse{\si@num@predec}%
  \let\si@num@prernd\si@num@predec
  \let\si@num@postrnd\si@num@postdec
  \renewcommand*{\si@num@predec}{}%
  \renewcommand*{\si@num@postdec}{}%
  \si@switchfalse
  \si@num@rnd}
%    \end{macrocode}
%\end{macro}
%\end{macro}
%\end{macro}
%\end{macro}
%\begin{macro}{\si@num@reverse}
%\begin{macro}{\si@num@rev}
%\darg{storage-macro}
% The initial stage is to reverse the entire number, to make life
% easier.  This is then undone by the other macros as the output is
% constructed.
%    \begin{macrocode}
\newcommand*{\si@num@reverse}[1]{%
  \renewcommand*{\si@tempa}{}%
  \expandafter\si@num@rev#1\@empty\@empty
  \let#1\si@tempa}
\def\si@num@rev#1#2\@empty{%
  \edef\si@tempa{#1\si@tempa}%
  \ifx\@empty#2\@empty\else
    \si@num@rev#2\@empty\@empty
  \fi}
%    \end{macrocode}
%\end{macro}
%\end{macro}
%\begin{macro}{\si@num@rnd}
%\changes{v1.1}{2008/08/25}{Altered check for ending to generalise
%  code}
%\begin{macro}{\si@num@rndpre}
%\begin{macro}{\si@num@rndpost}
%\changes{v1.0b}{2008/06/23}{Corrected bug in rounding code}
% The core looping macro of the system simply divides the flow
% between rounding before and after the decimal.  The two routines
% are quite similar, but have subtly different requirements.  Both
% take one character at a time from the input, increment if there is
% a carry digit, check its value, and add to the output string. This
% is very similar to the same routines used in other macro packages
% to achieve the same thing.
%    \begin{macrocode}
\newcommand*{\si@num@rnd}{%
  \ifx\@empty\si@num@postrnd\@empty
    \expandafter\si@num@rndpre
  \else
    \expandafter\si@num@rndpost
  \fi}
\newcommand*{\si@num@rndpre}{%
  \expandafter\edef\expandafter\si@tempa\expandafter{%
    \expandafter\@car\si@num@prernd\@nil}%
  \expandafter\edef\expandafter\si@num@prernd\expandafter{%
    \expandafter\@cdr\si@num@prernd\@nil}%
  \si@tempcntb\si@tempa\relax
  \ifsi@switch
    \advance\si@tempcntb\@ne\relax
  \fi
  \si@switchfalse
  \ifnum\si@tempcntb=10\relax
    \si@tempcntb\z@\relax
    \expandafter\expandafter\expandafter\si@switchtrue
  \fi
  \edef\si@num@predec{\the\si@tempcntb\si@num@predec}%
  \ifx\@empty\si@num@prernd\@empty
    \ifsi@switch
      \edef\si@num@predec{1\si@num@predec}%
    \fi
  \else
    \expandafter\si@num@rnd
  \fi}
\newcommand*{\si@num@rndpost}{%
  \expandafter\edef\expandafter\si@tempa\expandafter{%
    \expandafter\@car\si@num@postrnd\@nil}%
  \expandafter\edef\expandafter\si@num@postrnd\expandafter{%
    \expandafter\@cdr\si@num@postrnd\@nil}%
  \si@tempcntb\si@tempa\relax
  \ifsi@switch
    \advance\si@tempcntb\@ne\relax
  \fi
  \si@switchfalse
  \advance\si@num@rndtarget\@ne\relax
  \ifnum\si@tempcnta>\si@num@rndtarget\relax
    \advance\si@num@rndtarget\m@ne\relax
  \else
    \advance\si@num@rndtarget\m@ne\relax
    \ifnum\si@tempcnta>\si@num@rndtarget\relax
      \ifnum\si@tempcntb>4\relax
        \expandafter\expandafter\expandafter\si@switchtrue
      \fi
    \else
      \ifnum\si@tempcntb=10\relax
        \si@tempcntb\z@\relax
        \expandafter\expandafter\expandafter\si@switchtrue
      \fi
      \edef\si@num@postdec{\the\si@tempcntb\si@num@postdec}%
    \fi
  \fi
  \advance\si@tempcnta\m@ne\relax
%    \end{macrocode}
% If \cs{si@tempcnta} is negative, this must be the significant
% figures routine and one digit needs to be thrown away.
%    \begin{macrocode}
  \ifnum\si@tempcnta=\m@ne\relax
    \si@num@throw
  \fi
  \si@num@rnd}
%    \end{macrocode}
%\end{macro}
%\end{macro}
%\end{macro}
%\begin{macro}{\si@num@throw}
% Throw one digit away from the end of the post-decimal part.
%    \begin{macrocode}
\newcommand*{\si@num@throw}{%
  \si@num@reverse{\si@num@postdec}%
  \edef\si@num@postdec{\expandafter\@cdr\si@num@postdec\@nil}%
  \si@num@reverse{\si@num@postdec}}
%    \end{macrocode}
%\end{macro}
%\begin{macro}{\si@num@fixsf}
% Setting up a fixed number of significant figures is rather more
% complicated than fixing the decimal places.  There are a number of
% tests needed to see what needs to be done.
%    \begin{macrocode}
\newcommand*{\si@num@fixsf}{%
  \si@num@cntdigits{\si@num@predec}%
  \ifx\@empty\si@num@predec\@empty
    \si@tempcnta\z@\relax
  \fi
%    \end{macrocode}
% If the pre-decimal component has only a single digit, it could be
% ``0'': if so, only the significant figures in the decimal part are
% counted.
%    \begin{macrocode}
  \ifnum\si@tempcnta=\@ne\relax
    \renewcommand*{\si@tempa}{0}%
    \ifx\si@num@predec\si@tempa
%    \end{macrocode}
% The number of significant digits after the decimal point have to be
% counted up. If there are enough digits there, then some rounding
% will again be needed.
%    \begin{macrocode}
      \ifx\@empty\si@num@postdec\@empty\else
        \si@tempcnta\z@\relax
        \si@switchfalse
        \expandafter\si@num@cntpostsf\si@num@postdec
          \@empty\@empty
        \ifnum\si@tempcnta>\si@num@sf\relax
          \si@log@debug{Rounding to \the\si@num@sf\space
            figures}%
          \si@num@rndtarget\si@num@sf\relax
          \si@num@rndstart
        \fi
      \fi
    \else
      \si@num@cntprepost
    \fi
  \else
%    \end{macrocode}
% If there are too many digits in the pre-decimal part, a check is
% first made to see if there may be a need to round from the first
% post-decimal digit. If not, the post decimal part is thrown away.
%    \begin{macrocode}
    \ifnum\si@tempcnta<\si@num@sf\relax
      \si@num@cntprepost
    \else
      \advance\si@tempcnta\m@ne\relax
      \si@switchfalse
      \ifnum\si@tempcnta>\si@num@sf\relax\else
        \ifx\@empty\si@num@postdec\@empty\else
          \edef\si@num@postdec{\expandafter\@car
            \si@num@postdec\@nil}%
          \ifnum\si@num@postdec>4\relax
            \expandafter\si@switchtrue
          \fi
        \fi
      \fi
      \advance\si@tempcnta\@ne\relax
      \renewcommand*{\si@num@postdec}{}%
      \si@num@reverse{\si@num@predec}%
      \let\si@num@prernd\si@num@predec
      \renewcommand*{\si@num@predec}{}%
      \si@num@sfpre
      \ifsi@switch
        \edef\si@num@predec{1\si@num@predec}%
      \fi
    \fi
  \fi}
%    \end{macrocode}
%\end{macro}
%\begin{macro}{\si@num@cntpostsf}
% Counting up significant figures in the post-decimal part needs an
% auxiliary macro.
%    \begin{macrocode}
\def\si@num@cntpostsf#1#2\@empty{%
  \ifsi@switch
    \advance\si@tempcnta\@ne\relax
  \else
    \if 0#1\relax\else
      \si@tempcnta\@ne\relax
      \expandafter\expandafter\expandafter\si@switchtrue
    \fi
  \fi
  \ifx\@empty#2\@empty\else
    \si@num@cntpostsf#2\@empty\@empty
  \fi}
%    \end{macrocode}
%\end{macro}
%\begin{macro}{\si@num@cntprepost}
% There are two routes to find that the number of digits in both
% parts of the number are important.  In either case, rounding is
% only needed if the two parts have too many digits.
%    \begin{macrocode}
\newcommand*{\si@num@cntprepost}{%
  \si@tempcntb\si@tempcnta\relax
  \si@num@cntdigits{\si@num@postdec}%
  \advance\si@tempcnta\si@tempcntb\relax
  \ifnum\si@tempcnta>\si@num@sf
    \si@log@debug{Rounding to \the\si@num@sf\space figures}%
    \si@num@rndtarget\si@num@sf\relax
    \si@num@rndstart
    \ifsi@switch
      \si@num@throw
    \fi
  \fi}
%    \end{macrocode}
%\end{macro}
%\begin{macro}{\si@num@sfpre}
% The code here is very similar to that used to round to a fixed
% number of decimal places.
%    \begin{macrocode}
\newcommand*{\si@num@sfpre}{%
  \expandafter\edef\expandafter\si@tempa\expandafter{%
    \expandafter\@car\si@num@prernd\@nil}%
  \expandafter\edef\expandafter\si@num@prernd\expandafter{%
    \expandafter\@cdr\si@num@prernd\@nil}%
  \si@tempcntb\si@tempa\relax
  \ifsi@switch
    \advance\si@tempcntb\@ne\relax
  \fi
  \si@switchfalse
  \advance\si@num@sf\@ne\relax
  \ifnum\si@tempcnta>\si@num@sf\relax
    \advance\si@num@sf\m@ne\relax
    \edef\si@num@predec{0\si@num@predec}%
  \else
    \advance\si@num@sf\m@ne\relax
    \ifnum\si@tempcnta>\si@num@sf\relax
      \ifnum\si@tempcntb>4\relax
        \expandafter\expandafter\expandafter\si@switchtrue
      \fi
      \edef\si@num@predec{0\si@num@predec}%
    \else
      \ifnum\si@tempcntb=10\relax
        \si@tempcntb\z@\relax
        \expandafter\expandafter\expandafter\si@switchtrue
      \fi
      \edef\si@num@predec{\the\si@tempcntb\si@num@predec}%
    \fi
  \fi
  \advance\si@tempcnta\m@ne\relax
  \ifx\@empty\si@num@prernd\@empty\else
    \expandafter\si@num@sfpre
  \fi}
%    \end{macrocode}
%\end{macro}
%
%\begin{macro}{\si@num@int}
%\darg{integer-part}
% The formatting code for separating thousands is taken more-or-less
% directly from \SIstyle.  A few changes are made to fit the various
% conventions here.  Following on from the code above, \cs{si@tempa}
% is used to store the integer part of the number, and \cs{si@tempb}
% is used for the decimal part.
%    \begin{macrocode}
\newcommand*{\si@num@int}[1]{%
  \renewcommand*{\si@num@predec}{}%
  \ifsi@sepfour
    \si@num@intfmt{}#1\@empty\@empty\@empty
  \else
    \si@num@iffive{#1}
      {\si@num@intfmt{}#1\@empty\@empty\@empty}
      {\renewcommand*{\si@num@predec}{#1}}%
  \fi}
%    \end{macrocode}
%\end{macro}
%\begin{macro}{\si@num@iffive}
%\begin{macro}{\si@num@five}
%\darg{number}
% A test is needed for the presence of more than four characters.
%    \begin{macrocode}
\newcommand*{\si@num@iffive}[1]{%
  \si@num@five#1\@empty\@empty\@empty\@empty\@empty\end}
\def\si@num@five#1#2#3#4#5\end{%
   \ifx\@empty#5\@empty
      \expandafter\@secondoftwo
   \else
      \expandafter\@firstoftwo
   \fi}
%    \end{macrocode}
%\end{macro}
%\end{macro}
%\begin{macro}{\si@num@intfmt}
%\begin{macro}{\si@num@fiint}
% The business end of the integer formatter.
%    \begin{macrocode}
\newcommand*{\si@num@intfmt}[4]{%
  \ifx\@empty#2\@empty
    \si@num@intsep#1\relax
  \else
    \ifx\@empty#3\@empty
      \si@num@intsep\@empty\@empty#1#2\relax
    \else
      \ifx\@empty#4\@empty
        \si@num@intsep\@empty#1#2#3\relax
      \else
        \si@num@fiint{#1#2#3#4}%
      \fi
    \fi
  \fi}
\def\si@num@fiint#1\fi\fi\fi{\fi\fi\fi\si@num@intfmt{#1}}
%    \end{macrocode}
%\end{macro}
%\end{macro}
%\begin{macro}{\si@num@intsep}
% For adding separation to integers, an extra function is needed.
%    \begin{macrocode}
\newcommand*{\si@num@intsep}[4]{%
  \protected@edef\si@num@predec{\si@num@predec#1#2#3}%
  \if\relax#4\relax\else
    \protected@edef\si@num@predec{%
      \si@num@predec\ensuremath{\noexpand\si@digitsep}}%
    \expandafter\si@num@intsep\expandafter#4%
  \fi}
%    \end{macrocode}
%\end{macro}
%\begin{macro}{\si@num@dec}
%\begin{macro}{\si@num@decfmt}
%\darg{decimal-part}
% Formatting a decimal uses a similar mechanism, but with a few
% alterations needed.
%    \begin{macrocode}
\newcommand*{\si@num@dec}[1]{%
  \renewcommand*{\si@num@postdec}{}%
  \ifsi@sepfour
    \si@num@decfmt#1\@empty\@empty\@empty\@empty
  \else
    \si@num@iffive{#1}
      {\si@num@decfmt#1\@empty\@empty\@empty\@empty}
      {\protected@edef\si@num@postdec{\si@num@postdec#1}}%
  \fi}
\newcommand*{\si@num@decfmt}[4]{%
  \protected@edef\si@num@postdec{\si@num@postdec#1#2#3}%
  \ifx\@empty#4\@empty%
  \else
    \protected@edef\si@num@postdec{%
      \si@num@postdec\ensuremath{\noexpand\si@digitsep}}%
    \expandafter\si@num@decfmt\expandafter#4%
  \fi}
%    \end{macrocode}
%\end{macro}
%\end{macro}
%\begin{macro}{\si@num@procerr}
% Any error is recycled to to formatted correctly.  The $\pm$ sign,
% and any unit, are also added.
%    \begin{macrocode}
\newcommand*{\si@num@procerr}{%
  \si@num@addunit
  \ensuremath{\si@pm}%
  \expandafter\si@num\expandafter{\si@num@err}}
%    \end{macrocode}
%\end{macro}
%\begin{macro}{\si@num@sepxpart}
% A similar approach for numbers containing products, except the
% first token of the input has to be deleted.
%    \begin{macrocode}
\newcommand*{\si@num@sepxpart}{%
  \si@num@addunit
  \ensuremath{\times}%
  \expandafter\expandafter\expandafter\si@num\expandafter
    \expandafter\expandafter{%
      \expandafter\@cdr\si@num@xpart\@nil}}
%    \end{macrocode}
%\end{macro}
%\begin{macro}{\si@num@addunit}
% A short macro to add units if needed.
%    \begin{macrocode}
\newcommand*{\si@num@addunit}{%
  \si@unt@numtrue
  \ifx\@empty\si@unt@unitarg\@empty\else
    \ifsi@repeatunits
      \si@unt@printunit{\si@unt@unitarg}%
    \fi
  \fi}
%    \end{macrocode}
%\end{macro}
%
%\subsection{Formatting angles}
%\begin{macro}{\ang}
%\doarg{keyval options}
%\darg{angle, either in decimal or deg;min;sec format}
% The approach used here is similar to that in \SIstyle, but has been
% modified in a few ways.
%    \begin{macrocode}
\si@newrobustcmd*{\ang}[2][]{%
  \begingroup
    \sisetup{#1}%
    \si@fam@mode
    \si@log@debug{Processing \string\ang\space input `#2'}%
    \@makeother{\;}%
    \makeatletter
    \scantokens{\si@ang@parse#2;;;\@nil}}
%    \end{macrocode}
%\end{macro}
%\begin{macro}{\si@ang@parse}
% With the correct catcodes in place, processing can take place strip
% out the semi-colons.  Here, the input must either contain no
% semi-colons or two semi-colons.
%    \begin{macrocode}
\def\si@ang@parse#1;#2;#3;#4\@nil{%
  \let\ifsi@ang@fixdp\ifsi@fixdp
  \si@fixdpfalse
  \si@ifmtarg{#4}
    {\si@log@debug{Angle argument contains no
       semi-colons:\MessageBreak decimal angle}%
     \si@ang@dec{#1}{}{}}
    {\si@log@debug{Angle argument contains
       semi-colons:\MessageBreak degree-minute-second angle}%
     \renewcommand*{\si@tempa}{#4}%
     \renewcommand*{\si@tempb}{;;}%
     \ifx\si@tempa\si@tempb\else
       \ifsi@strictarc
         \renewcommand*{\si@tempb}{;}%
         \ifx\si@tempa\si@tempb
           \si@log@err{Insufficient semi-colons in argument
             of \string\ang}{The argument of
             \string\ang\space  must contain either no
             semi-colons or exactly two}%
         \else
           \si@log@err{Excess semi-colons in argument of
             \string\ang}{The argument of \string\ang\space
             must contain either no semi-colons or exactly
             two}%
         \fi
       \fi
     \fi
     \si@ang@arc{#1}{#2}{#3}}}
%    \end{macrocode}
%\end{macro}
%\begin{macro}{\si@ang@dec}
%\begin{macro}{\si@ang@arc}
% Two tests are needed, in case the input format requires conversion.
%    \begin{macrocode}
\newcommand*{\si@ang@dec}{%
  \let\si@ang@fix\@gobble
  \ifsi@ang@toarc
    \expandafter\si@ang@dectoarc
  \else
    \sisetup{padangle=none}\expandafter\si@ang@typeset
  \fi}
\newcommand*{\si@ang@arc}{%
  \let\si@ang@fix\si@ang@arcfix
  \ifsi@ang@todec
    \expandafter\si@ang@arctodec
  \else
    \expandafter\si@ang@typeset
  \fi}
%    \end{macrocode}
%\end{macro}
%\end{macro}
%\begin{macro}{\ifsi@ang@fixdp}
%\begin{macro}{\si@ang@fix}
%\begin{macro}{\si@ang@arcfix}
%\darg{number}
% A check is needed so that rounding and zero filling are only
% applied to the seconds of an arc angle.
%    \begin{macrocode}
\newif\ifsi@ang@fixdp
\newcommand*{\si@ang@fix}[1]{}
\newcommand*{\si@ang@arcfix}[1]{%
  \renewcommand*{\si@tempa}{second}%
  \renewcommand*{\si@tempb}{#1}%
  \ifx\si@tempa\si@tempb
    \ifsi@ang@fixdp
      \expandafter\expandafter\expandafter\si@fixdptrue
    \else
      \expandafter\expandafter\expandafter\si@fixdpfalse
    \fi
  \else
    \expandafter\si@fixdpfalse
  \fi}
%    \end{macrocode}
%\end{macro}
%\end{macro}
%\end{macro}
%\begin{macro}{\si@ang@ifnum}
%\darg{number}
% A test is required to check that the provided data consists of
% numbers which can actually be processed by \TeX.  This is achieved
% by using \cs{si@num@ifvalid} with a fixed list of valid
% characters.
%    \begin{macrocode}
\newcommand*{\si@ang@ifnum}[1]{%
  \begingroup
    \renewcommand*{\si@numvalid}{0123456789,.+-}%
    \ifx\@empty#1\@empty
      \aftergroup\@firstoftwo
    \else
      \si@num@ifvalid{#1}
        {\aftergroup\@firstoftwo}
        {\aftergroup\@secondoftwo}%
    \fi
  \endgroup}
%    \end{macrocode}
%\end{macro}
%\begin{macro}{\si@ang@arctodec}
%\darg{degrees}
%\darg{minutes}
%\darg{seconds}
% The business end of converting arcs to decimal angles is relatively
% straight-forward, as it only needs one calculation.  This has to be
% divided up, so that disaster does not strike if there are empty
% arguments; a check is also needed for the sign of the angle,
% so that the maths makes sense.
%    \begin{macrocode}
\newcommand*{\si@ang@arctodec}[3]{%
  \let\si@ang@fix\@gobble
  \ifnum\si@num@dp>\thr@@\relax
    \si@num@dp\thr@@\relax
  \fi
  \si@fixdptrue
  \si@ang@ifnum{#1}
    {\si@ang@ifnum{#2}
      {\si@ang@ifnum{#3}
        {\si@tempdima\z@\relax
         \renewcommand*{\si@tempa}{+}%
         \ifx\@empty#1\@empty\else
           \si@tempdima #1pt\relax
         \fi
         \ifdim\si@tempdima<\z@\relax
           \renewcommand*{\si@tempa}{-}%
         \fi
         \ifx\@empty#2\@empty\else
           \si@tempdima\dimexpr\si@tempdima\si@tempa
             #2pt/60\relax
         \fi
         \ifdim\si@tempdima<\z@\relax
           \renewcommand*{\si@tempa}{-}%
         \else
         \fi
         \ifx\@empty#3\@empty\else
           \si@tempdima\dimexpr\si@tempdima\si@tempa
             #3pt/3600\relax
         \fi
         \sisetup{numdecimal=.}%
         \expandafter\si@ang@typeset\expandafter{%
           \strip@pt\si@tempdima}{}{}}
        {\si@ang@notnum{#1}{#2}{#3}}}
      {\si@ang@notnum{#1}{#2}{#3}}}
    {\si@ang@notnum{#1}{#2}{#3}}}
%    \end{macrocode}
%\end{macro}
%\begin{macro}{\si@ang@dectoarc}
%\begin{macro}{\si@ang@arcdeg}
%\begin{macro}{\si@ang@arcmin}
%\begin{macro}{\si@ang@arcsec}
%\darg{number}
% The conversion macros for decimal to arc angles.  Life is more
% complex here than above, even without the need for checks on the
% input.  A number of separation steps are needed, each of which
% needs a separate macro.
%    \begin{macrocode}
\newcommand*{\si@ang@dectoarc}[1]{%
  \let\si@ang@fix\si@ang@arcfix
  \si@ang@fixdptrue
  \ifnum\si@num@dp>\@ne\relax
    \si@num@dp\@ne\relax
  \fi
  \si@ang@ifnum{#1}
    {\si@tempdima\z@\relax
     \ifx\@empty#1\@empty\else
       \si@tempdima #1pt\relax
     \fi
     \si@ang@sepint{deg}%
     \si@tempdima\dimexpr\si@tempdima *60\relax
     \si@ang@sepint{min}%
     \edef\si@tempa{\the\dimexpr\si@tempdima *60\relax}%
     \expandafter\newcommand\expandafter*\expandafter{%
       \expandafter\si@ang@arcsec\expandafter}\expandafter{%
       \expandafter\si@ang@strippt\si@tempa}%
%    \end{macrocode}
% A check is made for ``0.0'' seconds, which should be converted to
% simply ``0''.
%    \begin{macrocode}
     \si@tempdima\z@\relax
     \edef\si@tempa{\the\si@tempdima}%
     \expandafter\renewcommand\expandafter*\expandafter{%
       \expandafter\si@tempa\expandafter}\expandafter{%
       \expandafter\si@ang@strippt\si@tempa}%
     \ifx\si@tempa\si@ang@arcsec
       \renewcommand*{\si@ang@arcsec}{0}%
     \fi
%    \end{macrocode}
% To avoid adding zeros where they are not required, each part of the
% angle is now checked.
%    \begin{macrocode}
     \renewcommand*{\si@tempa}{0}%
     \ifx\si@ang@arcdeg\si@tempa
       \si@temptoks{{}}%
     \else
       \si@temptoks{{\si@ang@arcdeg}}%
     \fi
     \ifx\si@ang@arcmin\si@tempa
       \si@temptoks\expandafter{\the\si@temptoks{}}%
     \else
       \si@temptoks\expandafter{\the\si@temptoks{%
         \si@ang@arcmin}}%
     \fi
     \ifx\si@ang@arcsec\si@tempa
       \si@temptoks\expandafter{\the\si@temptoks{}}%
     \else
       \si@temptoks\expandafter{\the\si@temptoks{%
         \si@ang@arcsec}}%
     \fi
     \expandafter\si@ang@typeset\the\si@temptoks}
    {\si@ang@notnum{#1}{}{}}}
%    \end{macrocode}
%\end{macro}
%\end{macro}
%\end{macro}
%\end{macro}
%\begin{macro}{\si@ang@sepint}
%\begin{macro}{\si@ang@sint}
%\begin{macro}{\si@ang@strippt}
%\darg{either \texttt{deg} or \texttt{min}}
% Support macros for the conversion from decimal to arc angles.
%    \begin{macrocode}
\newcommand*{\si@ang@sepint}[1]{%
  \expandafter\si@ang@sint\the\si@tempdima\@empty
  \expandafter\let\csname si@ang@arc#1\endcsname\si@tempa}
\def\si@ang@sint#1.#2\@empty{%
  \renewcommand*{\si@tempa}{#1}%
  \si@tempdima 0.#2\relax}
\begingroup
  \catcode`P=12
  \catcode`T=12
  \lowercase{
    \renewcommand*{\si@tempa}{%
      \def\si@ang@strippt##1PT{##1}}}
  \expandafter\endgroup
\si@tempa
%    \end{macrocode}
%\end{macro}
%\end{macro}
%\end{macro}
%\begin{macro}{\si@ang@notnum}
%\darg{degrees}
%\darg{minutes}
%\darg{seconds}
% Not a \TeX\ number: complain.
%    \begin{macrocode}
\newcommand*{\si@ang@notnum}[3]{%
  \si@log@warn{Angle `#1;#2;#3' is not a pure
    number:\MessageBreak output will be as given}%
  \si@ang@typeset{#1}{#2}{#3}}
%    \end{macrocode}
%\end{macro}
%\begin{macro}{\ifsi@ang@sign}
% A flag is needed to leave signs along for angles, where
% a zero value may still need a sign.
%    \begin{macrocode}
\newif\ifsi@ang@sign
%    \end{macrocode}
%\end{macro}
%\begin{macro}{\si@ang@typeset}
%\darg{degrees}
%\darg{minutes}
%\darg{seconds}
% The \cs{si@ang@set} macro does the work of assigning the degrees,
% minutes and seconds, and actually typesetting the result.
%    \begin{macrocode}
\newcommand*{\si@ang@typeset}[3]{%
%    \end{macrocode}
%\begin{macro}{\si@ang@degs}
%\begin{macro}{\si@ang@mins}
%\begin{macro}{\si@ang@secs}
% First, the three macros that will contain the measures must exist.
%    \begin{macrocode}
  \ifsi@ang@padlarge
    \newcommand*{\si@ang@degs}{0\si@sym@degree}%
    \newcommand*{\si@ang@mins}{0\si@sym@minute}%
    \newcommand*{\si@ang@secs}{0\si@sym@second}%
  \else
    \newcommand*{\si@ang@degs}{}%
    \newcommand*{\si@ang@mins}{}%
    \newcommand*{\si@ang@secs}{}%
  \fi
%    \end{macrocode}
%\begin{macro}{\si@ang@decimalsymbol}
% The current definition of \cs{si@decimalsymbol} needs to be saved.
%    \begin{macrocode}
  \protected@edef\si@ang@decimalsymbol{\si@decimalsymbol}%
%    \end{macrocode}
%\end{macro}
%\begin{macro}{\si@ang@movesign}
% Either the signs need to be moved, or this needs to be killed off.
%    \begin{macrocode}
  \ifsi@astroang
    \let\si@ang@movesign\si@ang@astrosign
  \else
    \let\si@ang@movesign\@gobble
  \fi
%    \end{macrocode}
%\end{macro}
%\begin{macro}{\si@ang@secnum}
%\begin{macro}{\si@ang@minnum}
% The arguments are now examined in reverse order.  If they are
% empty, then nothing is done.  Otherwise, the larger measures are
% zero-filled, if this has been requested. Some steps are needed
% to allow for addition of signs to numbers.
%    \begin{macrocode}
  \newcommand*{\si@ang@secnum}{\si@ang@num{second}}%
  \newcommand*{\si@ang@minnum}{\si@ang@num{minute}}%
   \si@ifnotmtarg{#3}
     {\si@log@debug{Found seconds `#3'}%
      \si@ang@ifnum{#3}
        {\ifdim #3 pt=\z@\relax\else
           \si@ang@signtrue
         \fi}{}%
      \renewcommand*{\si@ang@secs}
       {\si@ang@secnum{#3}\si@sym@second}%
      \renewcommand*{\si@ang@mins}
        {\si@ang@pad{0\si@sym@minute}}%
      \renewcommand*{\si@ang@degs}
        {\si@ang@pad{0\si@sym@degree}}}%
   \si@ifnotmtarg{#2}
     {\si@log@debug{Found minutes `#2'}%
      \si@ang@ifnum{#2}
        {\ifdim #2 pt=\z@\relax\else
           \si@ang@signtrue
         \fi}{}%
      \renewcommand*{\si@ang@secnum}{%
        \si@ang@signlessnum{second}}%
      \renewcommand*{\si@ang@mins}
       {\si@ang@minnum{#2}\si@sym@minute}%
      \renewcommand*{\si@ang@degs}
        {\si@ang@pad{0\si@sym@degree}}}%
   \si@ifnotmtarg{#1}
     {\si@log@debug{Found degrees `#1'}%
      \renewcommand*{\si@ang@secnum}{%
        \si@ang@signlessnum{second}}%
      \renewcommand*{\si@ang@minnum}{%
        \si@ang@signlessnum{minute}}%
      \renewcommand*{\si@ang@degs}
%    \end{macrocode}
% The group here is needed to get the mechanism to move the symbol to
% work properly.
%    \begin{macrocode}
       {\si@ang@num{degree}{#1}%
        \si@sym@degree}}%
   \si@out@num
     {\si@ang@degs\si@anglesep\si@ang@mins\si@anglesep
        \si@ang@secs}%
%    \end{macrocode}
%\end{macro}
%\end{macro}
% The group opened by \cs{ang} is closed.
%    \begin{macrocode}
   \endgroup}
%    \end{macrocode}
%\end{macro}
%\end{macro}
%\end{macro}
%\end{macro}
%\begin{macro}{\si@ang@pad}
%\darg{number}
% Padding is only added if requested; the zero is a literal.
%    \begin{macrocode}
\newcommand*{\si@ang@pad}[1]{\ifsi@ang@padsmall #1\fi}
%    \end{macrocode}
%\end{macro}
%\begin{macro}{\si@ang@num}
%\begin{macro}{\si@ang@signlessnum}
%\darg{one of \texttt{degree}, \texttt{minute} or \texttt{second}}
%\darg{number}
% Modified versions of \cs{num}, one to typeset angles without a
% leading sign and the other with.
%    \begin{macrocode}
\newcommand*{\si@ang@num}[2]{%
  \begingroup
    \si@ang@fix{#1}%
    \si@ang@movesign{#1}%
    \si@num{#2}%
  \endgroup}
\newcommand*{\si@ang@signlessnum}[2]{%
  \begingroup
    \si@ang@fix{#1}%
    \si@ang@movesign{#1}%
    \sisetup{addsign=none}%
    \si@num{#2}%
  \endgroup}
%    \end{macrocode}
%\end{macro}
%\end{macro}
%\begin{macro}{\si@ang@killdegree}
%\begin{macro}{\si@ang@killminute}
%\begin{macro}{\si@ang@killsecond}
% A mechanism is needed to handle moving the angle unit signs for the
% \opt{astroang} option.  First, some support macros are needed.
%    \begin{macrocode}
\newcommand*{\si@ang@killdegree}{\let\si@sym@degree\relax}
\newcommand*{\si@ang@killminute}{\let\si@sym@minute\relax}
\newcommand*{\si@ang@killsecond}{\let\si@sym@second\relax}
%    \end{macrocode}
%\end{macro}
%\end{macro}
%\end{macro}
%\begin{macro}{\si@ang@astrosign}
%\darg{one of \texttt{degree}, \texttt{minute} or \texttt{second}}
% The method needs two steps, producing the sign over the decimal
% sign and preventing duplicate symbols appearing. This is based on a
% suggestion from Morten H{\o}gholm, but using \TeX\ internals as
% \cs{makebox} does not work here.  Note the need to correct for
% \cs{scriptspace} (thanks to Donald Arseneau for that).
%    \begin{macrocode}
\newcommand*{\si@ang@astrosign}[1]{%
  \renewcommand*{\si@decimalsymbol}{%
    \setbox\si@tempboxa=\hbox{%
      \ensuremath{{\si@ang@decimalsymbol}}}%
    \si@tempdima\wd\si@tempboxa\relax
    \setbox\si@tempboxb=\hbox to\z@{%
      \hss\unhbox\si@tempboxa\hss}%
    \setbox\si@tempboxa=\hbox{%
      \csname si@sym@#1\endcsname\hskip-\scriptspace}%
    \si@tempdimb\wd\si@tempboxa\relax
    \setbox\si@tempboxc=\hbox to\z@{%
      \hss\unhbox\si@tempboxa\hss}%
    \setbox\si@tempboxd=\hbox{%
      \usebox\si@tempboxb\usebox\si@tempboxc}%
    \ifdim\si@tempdima>\si@tempdimb\relax
      \setbox\si@tempboxa=\hbox to\si@tempdima{%
        \hss\unhbox\si@tempboxd\hss}%
    \else
      \setbox\si@tempboxa=\hbox to\si@tempdimb{%
        \hss\unhbox\si@tempboxd\hss}%
    \fi
    \usebox\si@tempboxa%
    \ifdim\si@tempdima>\si@tempdimb\relax\else
      \hskip\scriptspace
    \fi}%
  \renewcommand*{\si@num@decimalhook}{\expandafter\aftergroup
    \csname si@ang@kill#1\endcsname}}%
%    \end{macrocode}
%\end{macro}
%
%\subsection{Tabular material}
% The automatic formatting and alignment of numerical data in columns
% is handled here.  The various other packages that work in this area
% are basically ripped-off here.  The letters |D|, |N| and |R| are
% already taken by the other packages for numerical alignment, and so
% |S| ($=$ \currpkg) is chosen for the alignment of numerical
% material.  The package also provides a second column type, |s|, for
% units (the letter is taken from \cs{si}).
%
%\begin{macro}{\NC@list}
%\changes{v1.1e}{2008/11/18}{Fixed issue with \pkg{cellspace} package}
% The first part of the job is to create the basic apparatus for the
% columns using the \pkg{array} package.  To prevent any issues with
% the content of optional arguments to the new columns, so
% rearrangement is needed.  The \currpkg columns have to come
% \emph{before} any other column definitions.  This is achieved by
% saving \cs{NC@list}, creating the columns then restoring the list
% with the appropriate extra parts.
%    \begin{macrocode}
\@ifpackageloaded{cellspace}
  {\newcolumntype{C}[1]{>{\bcolumn #1\@nil}#1<{\ecolumn}}}
  {\AtBeginDocument{
    \@ifpackageloaded{cellspace}
     {\newcolumntype{C}[1]{>{\bcolumn #1\@nil}#1<{\ecolumn}}%
      \renewcommand*{\NC@rewrite@S}[1][]{%
        \edef\si@tempa{\the\@temptokena
        >{\noexpand\si@tab@begin@S[#1]}c%
        <{\noexpand\si@tab@end@S}}%
        \@temptokena\expandafter{\si@tempa}%
        \NC@find}}
     {}}}
\edef\si@tempa{%
  \noexpand\NC@do S\noexpand\NC@do s\the\NC@list}
\newcolumntype{S}{}
\newcolumntype{s}{}
\NC@list\expandafter{\si@tempa}
%    \end{macrocode}
%\end{macro}
%\changes{v1.1h}{2008/11/30}{Fixed problem with \pkg{colortbl} package}
%\changes{v1.1i}{2008/11/30}{Better solution to \pkg{colortbl} issue}
%\begin{macro}{\si@tab@corr@colortbl}
% A correction if \pkg{colortbl} is loaded.
%    \begin{macrocode}
\newcommand*{\si@tab@corr@colortbl}{\relax}
\AtBeginDocument{
  \@ifpackageloaded{colortbl}{%
    \renewcommand*{\si@tab@corr@colortbl}{%
      \hskip\stretch{-0.5}\kern\z@
    }
  }{}
}
%    \end{macrocode}
%\end{macro}
%\begin{macro}{\NC@rewrite@S}
%\begin{macro}{\NC@rewrite@s}
%\doarg{keyval options}
% Following the \numprint approach, the \cs{NC@rewrite@\ldots} macros
% are rewritten to collect the cell contents.  This means messing
% with the internal macros of another package, but there is no other
% way to do this.  As \pkg{array} is a standard package from the
% \pkg{tools} bundle, this should be reasonably safe. Here the begin
% and end code needed is added to the existing list if
% \cs{@temptokena}, with the start and end macros unexpanded.
% Argument |#1| contains any user setup options for this column.
% Passing an argument at this stage will cause issues, so each column
% type needs its own \texttt{begin} and \texttt{end} macros.
%    \begin{macrocode}
\renewcommand*{\NC@rewrite@S}[1][]{%
  \edef\si@tempa{\the\@temptokena
    >{\noexpand\si@tab@begin@S[#1]}c%
    <{\noexpand\si@tab@end@S}}%
  \@temptokena\expandafter{\si@tempa}%
  \NC@find}
\renewcommand*{\NC@rewrite@s}[1][]{%
  \edef\si@tempa{\the\@temptokena
    >{\noexpand\si@tab@begin@s[#1]}c%
    <{\noexpand\si@tab@end@s}}%
  \@temptokena\expandafter{\si@tempa}%
  \NC@find}
%    \end{macrocode}
%\end{macro}
%\end{macro}
%\begin{macro}{\si@tab@begin@S}
%\begin{macro}{\si@tab@begin@s}
%\begin{macro}{\si@tab@gettok}
%\doarg{keyval options}
% At this stage, the appropriate token gathering macro is activated,
% and the common starting macro is called.  For the |S| column, the
% \opt{seperr} is turned off, and an error is set to be raised by any
% ``x-part'' input.
%    \begin{macrocode}
\newcommand*{\si@tab@begin@S}[1][]{%
  \si@log@debug{Processing S column cell contents}%
  \let\si@tab@gettok\si@tab@gettok@S
  \si@seperrfalse
  \renewcommand*{\si@num@sepxpart}{%
    \si@log@err{Multiple numbers not allowed in
       tables\MessageBreak Only the first number used}
       \@ehb}%
  \si@tab@begin[#1]}
\newcommand*{\si@tab@begin@s}[1][]{%
  \si@log@debug{Processing s column cell contents}%
  \let\si@tab@gettok\si@tab@gettok@s
  \si@tab@begin[#1]}
%    \end{macrocode}
%\end{macro}
%\end{macro}
%\end{macro}
%\begin{macro}{\si@tab@toks}
%\begin{macro}{\si@tab@pretoks}
%\begin{macro}{\si@tab@posttoks}
% Some storage is needed for the data to build up.  In common with
% \pkg{rccol} and \numprint, token registers are used for this (thus
% leaving problematic input to be handled later).
%    \begin{macrocode}
\newtoks\si@tab@toks
\newtoks\si@tab@pretoks
\newtoks\si@tab@posttoks
%    \end{macrocode}
%\end{macro}
%\end{macro}
%\end{macro}
%\begin{macro}{\si@tab@begin}
%\doarg{keyval options}
% The macro for gathering up input is common to both column types. It
% uses the method for \pkg{rccol}; the cell contents are collected by
% a second macro, which then stores all of the data in an appropriate
% token store.
%    \begin{macrocode}
\newcommand*{\si@tab@begin}[1][]{%
  \begingroup
    \sisetup{#1}%
    \si@tab@toks{}%
    \si@tab@pretoks{}%
    \si@tab@posttoks{}%
    \si@switchfalse
    \si@tab@gettok}
%    \end{macrocode}
%\end{macro}
%\begin{macro}{\si@tab@next}
% A macro is needed for recursion when collecting cell contents.
%    \begin{macrocode}
\newcommand*{\si@tab@next}{}
%    \end{macrocode}
%\end{macro}
%\begin{macro}{\si@tab@gettok@S}
%\changes{v1.0j}{2008/08/21}{Token gathering code now checks for
%  expandable content}
%\darg{token}
% The cell-content collection system is inherited from \pkg{rccol}.
% First, the current input is compared with a fixed list of macros
% which need custom handling.
%    \begin{macrocode}
\newcommand*{\si@tab@gettok@S}[1]{%
  \ifx\tabularnewline#1\relax
    \let\si@tab@next\si@tab@newline@S
  \else
    \ifx\end#1\relax
      \let\si@tab@next\end
    \else
      \ifx\si@tab@end@S#1\relax
        \let\si@tab@next\si@tab@end@S
      \else
        \ifx\endtabular#1\relax
          \let\si@tab@next\endtabular
        \else
          \ifx\csname#1\relax
            \let\si@tab@next\csname
          \else
            \ifx\relax#1\relax
              \let\si@tab@next\relax
            \else
%    \end{macrocode}
% If the code gets to here, then recursion should occur.
%    \begin{macrocode}
              \let\si@tab@next\si@tab@gettok@S
%    \end{macrocode}
% Cells always contain \cs{ignorespaces} and \cs{unskip}. These can
% both be filtered out here and are saved to either the pre-number or
% post-number token store, as appropriate. This is done without
% fixing the destination in case the user adds addition copies to the
% input.
%    \begin{macrocode}
              \ifx\ignorespaces#1\relax
                \si@tab@othertok{#1}%
              \else
                \ifx\unskip#1\relax
                  \si@tab@othertok{#1}%
                \else
%    \end{macrocode}
% If the current token is a valid numerical character, then the flag
% is set and the input is saved into the ``processing'' token
% register.
%    \begin{macrocode}
                  \si@str@ifchrstr{#1}{\si@numvalid}
                     {\si@switchtrue
                      \si@log@debug{Found numerical cell
                        contents `#1'}%
                      \si@tab@toks=\expandafter{%
                        \the\si@tab@toks#1}}
%    \end{macrocode}
% For non-numerical content, some testing is needed. A check is made
% to see if the current ``token'' is a single item. If it was wrapped
% in one set of braces before being picked up here, they will have
% been removed. So the input may now have more than one character to
% it. If it is a single token, testing for expendability first
% requires a check for another set of braces.  This is handed off
% using \cs{futurelet}.
%    \begin{macrocode}
                     {\si@tab@ifonechar{#1}
                       {\futurelet\si@tempa\si@tab@bracetest#1}%
                       {\si@tab@othertok{#1}}}%
                \fi
              \fi
            \fi
          \fi
        \fi
      \fi
    \fi
  \fi
  \si@tab@next}
%    \end{macrocode}
%\end{macro}
%\begin{macro}{\si@tab@othertok}
%\darg{token}
% Tokens which are not part of a number can occur before or after a
% number. This is all sorted out here, with some debugging
% information for good measure.
%    \begin{macrocode}
\newcommand*{\si@tab@othertok}[1]{%
  \si@log@debug{Found other cell contents `\unexpanded{#1}'}%
  \ifsi@switch
    \si@log@debug{Adding to post-numeral part}%
    \si@tab@posttoks=\expandafter{\the\si@tab@posttoks#1}%
  \else
    \si@log@debug{Adding to pre-numeral part}%
    \si@tab@pretoks=\expandafter{\the\si@tab@pretoks#1}%
  \fi}
%    \end{macrocode}
%\end{macro}
%\begin{macro}{\si@tab@ifonechar}
%\darg{token}
%\begin{macro}{\si@tab@onechar}
% A quick check for a single character.
%    \begin{macrocode}
\newcommand*{\si@tab@ifonechar}[1]{\si@tab@onechar#1\@empty}
\def\si@tab@onechar#1#2\@empty{%
  \ifx\@empty#2\@empty
    \expandafter\@firstoftwo
  \else
    \expandafter\@secondoftwo
  \fi}
%    \end{macrocode}
%\end{macro}
%\end{macro}
%\begin{macro}{\si@tab@bracetest}
% For the \cs{futurelet} test, a separate macro is needed. If the
% current token starts with a brace, then it is added to the
% ``other'' stack.  If not, a test can be made on its definition.
%    \begin{macrocode}
\newcommand*{\si@tab@bracetest}{%
  \ifx\si@tempa\bgroup
    \expandafter\si@tab@othertok
  \else
    \expandafter\si@tab@exptest
  \fi}
%    \end{macrocode}
%\end{macro}
%\begin{macro}{\si@tab@exptest}
%\begin{macro}{\si@tab@meaning}
%\begin{macro}{\si@tab@strip}
%\darg{token}
% The meaning of the token is compared to that for a short macro. If
% the two are of the same type, expansion is attempted. In this case,
% the current pre-number tokens are also typeset.
%    \begin{macrocode}
\newcommand*{\si@tab@exptest}[1]{%
  \edef\si@tempa{\si@tab@meaning{#1}}%
  \ifx\si@tempa\si@tab@macro
    \si@log@debug{Expanding cell contents `\unexpanded{#1}'}%
    \the\si@tab@pretoks
    \si@tab@pretoks{}%
    \protected@edef\si@tab@next{\noexpand\si@tab@gettok@S#1}%
  \else
    \si@tab@othertok{#1}%
  \fi}
\newcommand*{\si@tab@meaning}[1]{%
  \expandafter\si@tab@strip\meaning#1:->:->\@empty}
\def\si@tab@strip#1:->#2:->#3\@empty{#1}
%    \end{macrocode}
%\end{macro}
%\end{macro}
%\end{macro}
%\begin{macro}{\si@tab@macro}
% A supporter macro for the above which always expands to ``macro''
% with everything with category code 12.
%    \begin{macrocode}
\newcommand*{\si@tab@macro}{}
\edef\si@tab@macro{\si@tab@meaning{\si@tab@macro}}
%    \end{macrocode}
%\end{macro}
%\begin{macro}{\si@tab@gettok@s}
%\changes{v1.0a}{2008/06/20}{Fixed issues with alignment of
%  \texttt{s} column contents}
%\darg{token}
% The situation for the |s| column is less complex, as there is only
% one store and expansion is not an issue.
%    \begin{macrocode}
\newcommand*{\si@tab@gettok@s}[1]{%
  \ifx\tabularnewline#1\relax
    \let\si@tab@next\si@tab@newline@s
  \else
    \ifx\end#1\relax
      \let\si@tab@next\end
    \else
      \ifx\si@tab@end@s#1\relax
        \let\si@tab@next\si@tab@end@s
      \else
        \ifx\endtabular#1\relax
          \let\si@tab@next\endtabular
        \else
          \ifx\csname#1\relax
            \let\si@tab@next\csname
          \else
            \ifx\relax#1\relax
              \let\si@tab@next\relax
            \else
              \let\si@tab@next\si@tab@gettok@s
              \ifx\ignorespaces#1\relax\else
                \ifx\unskip#1\relax\else
                  \si@tab@toks=\expandafter{%
                    \the\si@tab@toks#1}%
                  \si@log@debug{Found cell contents `#1'}%
                \fi
              \fi
            \fi
          \fi
        \fi
      \fi
    \fi
  \fi
  \si@tab@next}
%    \end{macrocode}
%\end{macro}
%\begin{macro}{\si@tab@end@S}
%\changes{v1.0a}{2008/06/20}{Fixed mixed number/text column
%  alignment}
%\changes{v1.0k}{2008/09/02}{Fixed problem with coloured cells}
%\begin{macro}{\si@tab@rfill}
%\begin{macro}{\si@tab@lfill}
%\begin{macro}{\si@tab@end@s}
% The end macros are similar, but with some minor differences. In
% both cases, the appropriate filling is carried out.  For the |S|
% column, this depends on the cell contents, whereas for the |s|
% column the fill is always the same.  Output of a number in an |S|
% column is only attempted if one was found, otherwise the cell
% contents will all be in \cs{si@tab@pretoks}.
%    \begin{macrocode}
\newcommand*{\si@tab@end@S}{%
    \ifsi@switch
      \let\si@tab@lfill\si@tab@lfill@S
      \let\si@tab@rfill\si@tab@rfill@S
    \else
      \let\si@tab@rfill\si@tab@rfill@t
      \let\si@tab@lfill\si@tab@lfill@t
    \fi
    \si@tab@corr@colortbl
    \hskip\stretch{\si@tab@lfill}\kern\z@
    \ifsi@switch
      \the\si@tab@pretoks
      \si@tab@numout
      \the\si@tab@posttoks
    \else
      \the\si@tab@pretoks
    \fi
    \hskip\stretch{\si@tab@rfill}\kern\z@
    \si@tab@corr@colortbl
  \endgroup}
\newcommand*{\si@tab@end@s}{%
    \si@tab@corr@colortbl
    \hskip\stretch{\si@tab@lfill@s}\kern\z@ 
    \ignorespaces
    \expandafter\si\expandafter{\the\si@tab@toks}%
    \unskip
    \hskip\stretch{\si@tab@rfill@s}\kern\z@ 
    \si@tab@corr@colortbl
  \endgroup}
%    \end{macrocode}
%\end{macro}
%\end{macro}
%\end{macro}
%\end{macro}
%\begin{macro}{\si@tab@newline@S}
%\begin{macro}{\si@tab@newline@s}
%\begin{macro}{\si@tab@end}
% If the column is the final one read, then some work is needed with
% the \cs{tabularnewline} macro.  Output has to happen \emph{before}
% the new line, then the ending macro is made safe before calling the
% \LaTeX\ line end.  If the user makes use of the primitive \cs{cr}
% then this problem does not arise as \cs{si@tab@end@\ldots} is
% called correctly.
%    \begin{macrocode}
\newcommand*{\si@tab@newline@S}{%
  \si@tab@end@S
  \hfil\relax
  \let\si@tab@end\si@tab@end@S
  \renewcommand*{\si@tab@end@S}{\let\si@tab@end@S\si@tab@end}%
  \tabularnewline}
\newcommand*{\si@tab@newline@s}{%
  \si@tab@end@s
  \hfil\relax
  \let\si@tab@end\si@tab@end@s
  \renewcommand*{\si@tab@end@s}{\let\si@tab@end@s\si@tab@end}%
  \tabularnewline}
%    \end{macrocode}
%\end{macro}
%\end{macro}
%\end{macro}
%
%\begin{macro}{\si@tempcnta}
%\begin{macro}{\si@tempcntb}
% The second part of the tabular code is concerned with typesetting
% numbers in |S| columns with the appropriate alignment. Counters are
% needed for the digit-counting system.
%    \begin{macrocode}
\newcount\si@tempcnta
\newcount\si@tempcntb
%    \end{macrocode}
%\end{macro}
%\end{macro}
%\begin{macro}{\si@tab@numout}
%\changes{v1.1}{2008/09/03}{Extended for non-aligned output}
% If a number is found, then some secondary processing is needed to
% format it correctly.
%    \begin{macrocode}
\newcommand*{\si@tab@numout}{%
  \si@num@intabtrue
  \ifsi@tabparseonly\else
    \ifsi@tab@fixed
      \ifsi@tabautofit
        \si@num@dp\si@tab@mantpostcnt\relax
        \expandafter\expandafter\expandafter\expandafter
          \expandafter\expandafter\expandafter\si@fixdptrue
      \fi
    \fi
  \fi
  \expandafter\si@num\expandafter{\the\si@tab@toks}%
  \si@tab@format}
%    \end{macrocode}
%\end{macro}
%\begin{macro}{\si@tab@prebox}
%\begin{macro}{\si@tab@postbox}
%\begin{macro}{\si@tab@midbox}
%\begin{macro}{\si@tab@expbox}
% The various boxes needed for the column centring are declared
% Unlike the \pkg{dcolumn} original, private boxes are used here.
% \cs{si@tempboxa} is used when a space to measure one of the
% constituents is needed; it is never used for output.
%    \begin{macrocode}
\newbox\si@tab@prebox
\newbox\si@tab@midbox
\newbox\si@tab@postbox
\newbox\si@tab@expbox
%    \end{macrocode}
%\end{macro}
%\end{macro}
%\end{macro}
%\end{macro}
%\begin{macro}{\si@tab@format}
%\changes{v1.1}{2008/09/03}{Extended for non-aligned output}
% The formatting set up is taken from \pkg{dcolumn}, but with
% control of the output form the stored information.  The choice
% of a variable (decimal-centred) column or fixed width boxes is
% made.
%    \begin{macrocode}
\newcommand*{\si@tab@format}{%
  \ifsi@tabparseonly
    \expandafter\si@tab@outonly
  \else
    \ifsi@tab@fixed
      \expandafter\expandafter\expandafter\si@tab@fixed
    \else
      \expandafter\expandafter\expandafter\si@tab@unfixed
    \fi
  \fi
%    \end{macrocode}
% A hack to get the correct colour everywhere without too much work.
%    \begin{macrocode}
  \ifsi@colourvalues
    \si@fam@colourcmd{\si@valuecolour}%
  \fi
  \box\si@tab@prebox\box\si@tab@midbox\box\si@tab@postbox
    \box\si@tab@expbox}
%    \end{macrocode}
%\end{macro}
%\begin{macro}{\si@tab@outonly}
% The non-alignment option means that no real processing is done: only font
% control is required.
%    \begin{macrocode}
\newcommand*{\si@tab@outonly}{%
  \si@log@debug{Using non-aligned S column}%
  \setbox\si@tab@prebox=\hbox{}%
  \setbox\si@tab@midbox=\hbox{}%
  \setbox\si@tab@postbox=\hbox{}%
  \setbox\si@tab@expbox=\hbox{}%
    \protected@edef\si@tempa{%
      \si@tab@out
    \ifx\@empty\si@num@out\@empty\else
      \ensuremath{{\si@decimalsymbol}}%
    \fi
    \si@num@out
    \si@tab@expout}%
  \expandafter\si@out@num\expandafter{\si@tempa}}
%    \end{macrocode}
%\end{macro}
%\begin{macro}{\si@tab@unfixed}
%\changes{v1.0l}{2008/09/03}{Fixed issue with decimal sign and
%  integers}
% When the width of the contents is not fixed, the system creates a
% block in which the decimal marker is always at the centre. This is
% achieved by placing the pre- and post-decimal parts of the number
% in boxes.  The wider one is then used to set up the column width,
% by resizing the other one.
%    \begin{macrocode}
\newcommand*{\si@tab@unfixed}{%
  \si@log@debug{Using variable width S column}%
  \ifx\@empty\si@num@out\@empty
    \setbox\si@tab@midbox=\hbox
      {\phantom{\ensuremath{{\si@decimalsymbol}}}}%
  \else
    \setbox\si@tab@midbox=\hbox
      {\ensuremath{{\si@decimalsymbol}}}%
  \fi
  \protected@edef\si@num@out{\si@num@out\si@tab@expout}%
  \setbox\si@tab@prebox=\hbox
    {\expandafter\si@out@num\expandafter{\si@tab@out}}%
  \setbox\si@tab@postbox=\hbox
    {\expandafter\si@out@num\expandafter{\si@num@out}}%
  \ifdim\wd\si@tab@prebox>\wd\si@tab@postbox\relax
    \setbox\si@tab@postbox=\hbox to\wd\si@tab@prebox%
      {\unhbox\si@tab@postbox\hfill}%
  \else
    \setbox\si@tab@prebox=\hbox to\wd\si@tab@postbox%
      {\hfill\unhbox\si@tab@prebox}%
  \fi
  \setbox\si@tab@expbox=\hbox{}}
%    \end{macrocode}
%\end{macro}
%\begin{macro}{\si@tab@predim}
%\begin{macro}{\si@tab@postdim}
%\begin{macro}{\si@tab@expdim}
%\begin{macro}{\si@tempdima}
%\begin{macro}{\si@tempdimb}
% Some storage dimensions are declared.
%    \begin{macrocode}
\newdimen\si@tab@predim
\newdimen\si@tab@postdim
\newdimen\si@tab@expdim
\newdimen\si@tempdima
\newdimen\si@tempdimb
%    \end{macrocode}
%\end{macro}
%\end{macro}
%\end{macro}
%\end{macro}
%\end{macro}
%\begin{macro}{\si@tab@sp}
% A short macro to control superscript.
%    \begin{macrocode}
\newcommand*{\si@tab@sp}{}
%    \end{macrocode}
%\end{macro}
%\begin{macro}{\si@tab@fixed}
%\changes{v1.0l}{2008/09/03}{Fixed issue with spacing when no digits
%  present after decimal marker}
% The column is not centred on the decimal marker; the user specifies
% how many characters on each side are allowed for.  First, the width
% of a character is measured, and stored.
%    \begin{macrocode}
\newcommand*{\si@tab@fixed}{%
  \si@log@debug{Using fixed-width S column}%
  \let\si@tab@sp\relax
  \setbox\si@tab@midbox=\hbox{}%
  \setbox\si@tab@expbox=\hbox{}%
  \setbox\si@tempboxa=\hbox{\si@out@num{1}}%
  \si@tempdima\wd\si@tempboxa\relax
%    \end{macrocode}
% The width for the two output boxes is set up.
%    \begin{macrocode}
  \si@tab@predim\the\si@tab@mantprecnt\si@tempdima\relax
  \si@tab@sepcorr{mantpre}{pre}%
  \si@tab@postdim\si@tab@mantpostcnt\si@tempdima\relax
  \setbox\si@tempboxa=\hbox{\ensuremath{{\si@decimalsymbol}}}%
  \ifnum\si@tab@mantpostcnt=\z@\relax\else
    \advance\si@tab@postdim\wd\si@tempboxa\relax
  \fi
  \si@tab@sepcorr{mantpost}{post}%
%    \end{macrocode}
% If space is needed for an exponent, it needs to be allowed for in
% the exponent box dimension.  First, the digits of the two parts are
% checked for; the width of a character is altered to be superscript.
%    \begin{macrocode}
  \setbox\si@tempboxa=\hbox{\si@out@num{^{1}}}%
  \si@tempdima\wd\si@tempboxa\relax
  \ifnum\si@tab@expprecnt>\z@\relax
    \si@tab@expdim\si@tab@expprecnt\si@tempdima\relax
    \si@tab@sepcorr{exppre}{exp}%
  \fi
  \let\si@tab@sp\sp
  \ifnum\si@tab@exppostcnt>\z@\relax
    \advance\si@tab@expdim\si@tab@exppostcnt\si@tempdima\relax
    \setbox\si@tempboxa=\hbox{%
      \ensuremath{^{\si@decimalsymbol}}}%
    \advance\si@tab@expdim\wd\si@tempboxa\relax
    \si@tab@sepcorr{exppost}{exp}%
  \fi
%    \end{macrocode}
% Space is reserved for signs.
%    \begin{macrocode}
  \setbox\si@tempboxa=\hbox{\ensuremath{-}}%
  \ifsi@tab@mantsign
    \advance\si@tab@predim\wd\si@tempboxa\relax
  \fi
  \setbox\si@tempboxa=\hbox{\ensuremath{^{-}}}%
  \ifsi@tab@expsign
    \advance\si@tab@expdim\wd\si@tempboxa\relax
  \fi
%    \end{macrocode}
% Now, if there is space to be saved for an exponent under any
% circumstances, the space for the ``$\times 10$'' part is needed.
% This is done by adding both counters together, then using this
% result for the logic.
%    \begin{macrocode}
  \si@tempcnta\si@tab@expprecnt\relax
  \advance\si@tempcnta\si@tab@exppostcnt\relax
  \ifnum\si@tempcnta>\z@\relax
    \setbox\si@tempboxa=\hbox{\ensuremath{%
      {}\si@expproduct{}\si@expbase}}%
    \advance\si@tab@expdim\wd\si@tempboxa\relax
  \fi
%    \end{macrocode}
% Finally for the box dimensions, if the exponent is not aligned,
% the space reserved for it is added to the post box.
%    \begin{macrocode}
  \ifsi@tabalignexp\else
    \advance\si@tab@postdim\si@tab@expdim\relax
  \fi
%    \end{macrocode}
% With the boxes set up, the contents can be sorted out.  A bit of
% shuffling may be needed, depending on the treatment of exponents.
%    \begin{macrocode}
  \setbox\si@tab@prebox=\hbox to\si@tab@predim{\hss\hfill
    \expandafter\si@out@num\expandafter{\si@tab@out}}%
  \ifx\@empty\si@num@out\@empty
    \setbox\si@tab@postbox=\hbox to\si@tab@postdim{\hfil}%
  \else
    \ifsi@tabalignexp\else
      \protected@edef\si@num@out{\si@num@out\si@tab@expout}%
    \fi
    \setbox\si@tab@postbox=\hbox to\si@tab@postdim
      {\ensuremath{{\si@decimalsymbol}}\expandafter\si@out@num
        \expandafter{\si@num@out}\hfil}%
  \fi
  \ifx\@empty\si@tab@expout\@empty
    \ifsi@tabalignexp
      \setbox\si@tab@expbox=\hbox to\si@tab@expdim{\hfil}%
    \fi
  \else
    \ifsi@tabalignexp
      \setbox\si@tab@expbox=\hbox to\si@tab@expdim
        {\expandafter\si@out@num\expandafter{\si@tab@expout}%
           \hfil}%
    \fi
  \fi}
%    \end{macrocode}
%\end{macro}
%\begin{macro}{\si@tab@sepcorr}
%\darg{one of \texttt{mantpre}, \texttt{mantpost}, \texttt{exppre} or
%  \texttt{exppost}}
%\darg{one of \texttt{pre}, \texttt{post} or \texttt{exp}}
% A spacing correction is needed \emph{if} the number of digits to be
% allowed for will lead to the introduction of a separator.  A counter
% and dimension are needed for the testing.
%    \begin{macrocode}
\newcommand*{\si@tab@sepcorr}[2]{%
  \expandafter\si@tempcnta\expandafter\the
    \csname si@tab@#1cnt\endcsname\relax
%    \end{macrocode}
% Calculate how many groups of three there are, then allow for not
% separating four characters if \cs{ifsi@sepfour} is false.
%    \begin{macrocode}
  \divide\si@tempcnta\thr@@\relax
  \ifsi@sepfour\else
    \expandafter\ifnum\expandafter\the
      \csname si@tab@#1cnt\endcsname=4\relax
      \si@tempcnta\z@\relax
    \fi
  \fi
%    \end{macrocode}
% The width of the separators is measured, and the correct number of
% separator widths are added to the box dimension.
%    \begin{macrocode}
  \setbox\si@tempboxa=\hbox{%
    \ensuremath{\si@tab@sp{\si@digitsep}}}%
  \expandafter\advance\csname si@tab@#2dim\endcsname
    \si@tempcnta\wd\si@tempboxa}
%    \end{macrocode}
%\end{macro}
%
%\subsection{Units}
%\begin{macro}{\SI}
%\doarg{keyval options}
%\darg{number}
% There are two types of user macros for the units system; those for
% defining new units, prefixes and powers, and those for using them.
% There are two macros for using units, \cs{SI} and \cs{si}, which
% work in very similar ways. argument to \cs{SI}
%    \begin{macrocode}
\si@newrobustcmd*{\SI}[2][]{%
  \@ifnextchar[%]
    {\si@SI[#1]{#2}}
    {\si@SI[#1]{#2}[]}}
%    \end{macrocode}
%\end{macro}
%\begin{macro}{\si}
%\doarg{keyval options}
%\darg{unit}
% \cs{si} is just an alias for \cs{SI} with no number; everything is
% handed off into an internal macro.  The internal macro also handles
% the optional prefix
%    \begin{macrocode}
\si@newrobustcmd*{\si}[2][]{\si@SI[#1]{}[]{#2}}
%    \end{macrocode}
%\end{macro}

%\begin{macro}{\newunit}
%\begin{macro}{\renewunit}
%\begin{macro}{\provideunit}
%\doarg{keyval options}
%\darg{unit}
%\darg{symbol}
% The \cs{newunit} and \cs{renewunit} macros create the new unit
% macros. To allow a mechanism for checking an existing definition,
% these macros simply carry out the appropriate tests, before handing
% off to the internal macro.  \cs{@ifdefinable} is not used here as a
% customised error is desirable. Other than that, the code here gives
% very similar results to \cs{newcommand} and \cs{renewcommand}.
% Finally, \cs{provideunit} adds the unit definition only if it does
% not already exist.
%    \begin{macrocode}
\newcommand*{\newunit}[3][]{%
  \si@ifdefinable{#2}
    {\si@unt@defunit[#1]{#2}{#3}}
    {\si@log@err{Unit \string#2 already defined!}\@eha}}
\newcommand*{\renewunit}[3][]{%
  \si@ifdefinable{#2}
    {\si@log@err{Unit \string#2 undefined}\@ehc
     \si@unt@defunit[#1]{#2}{#3}}
    {\si@log@inf{Redefining unit \string#2}%
     \si@unt@defunit[#1]{#2}{#3}}}
\newcommand*{\provideunit}[3][]{%
  \si@ifdefinable{#2}
    {\si@unt@defunit[#1]{#2}{#3}}
    {}}
%    \end{macrocode}
%\end{macro}
%\end{macro}
%\end{macro}
%\begin{macro}{\newprefix}
%\begin{macro}{\renewprefix}
%\begin{macro}{\provideprefix}
%\doarg{\texttt{binary}}
%\darg{multiple}
%\darg{powers-ten}
%\darg{symbol}
% The multiples of units are defined here; very similar code is used
% to the \cs{newunit}, \etc, macros.  The multiple prefixes cannot
% take an optional argument, and must represent some power.  Hence
% the arguments required are different.
%    \begin{macrocode}
\newcommand*{\newprefix}[4][]{%
  \si@ifdefinable{#2}
    {\si@unt@defprefix[#1]{#2}{#3}{#4}}
    {\si@log@err{Prefix \string#2 already defined!}\@eha}}
\newcommand*{\renewprefix}[4][]{%
  \si@ifdefinable{#2}
    {\si@log@err{Prefix \string#2 undefined}\@ehc
     \si@unt@defprefix[#1]{#2}{#3}{#4}}
    {\si@log@inf{Redefining prefix \string#2}%
     \si@unt@defprefix[#1]{#2}{#3}{#4}}}
\newcommand*{\provideprefix}[4][]{%
  \si@ifdefinable{#2}
    {\si@unt@defprefix[#1]{#2}{#3}{#4}}
    {}}
%    \end{macrocode}
%\end{macro}
%\end{macro}
%\end{macro}
%\begin{macro}{\newpower}
%\begin{macro}{\renewpower}
%\begin{macro}{\providepower}
%\doarg{\texttt{post}}
%\darg{number}
%\darg{power}
% Here power multiples for units are set up.  As with units and
% multiples, a layered approach is used to keep things easy to
% maintain.
%    \begin{macrocode}
\newcommand*{\newpower}[3][]{%
  \si@ifdefinable{#2}
    {\si@unt@defpower[#1]{#2}{#3}}
    {\si@log@err{Power \string#2 already defined!}\@eha}}
\newcommand*{\renewpower}[3][]{%
  \si@ifdefinable{#2}
    {\si@log@err{Power \string#2 undefined}\@ehc
     \si@unt@defpower[#1]{#2}{#3}}
    {\si@log@inf{Redefining power \string#2}%
     \si@unt@defpower[#1]{#2}{#3}}}
\newcommand*{\providepower}[3][]{%
  \si@ifdefinable{#2}
    {\si@unt@defpower[#1]{#2}{#3}}
    {}}
%    \end{macrocode}
%\end{macro}
%\end{macro}
%\end{macro}
%\begin{macro}{\newqualifier}
%\changes{v1.1}{2008/09/04}{New user macro}
%\begin{macro}{\renewqualifier}
%\changes{v1.1}{2008/09/04}{New user macro}
%\begin{macro}{\providequalifier}
%\changes{v1.1}{2008/09/04}{New user macro}
%\darg{qualifier}
%\darg{symbol}
% Creating unit qualifiers is much the same method.
%    \begin{macrocode}
\newcommand*{\newqualifier}[2]{%
  \si@ifdefinable{#1}
    {\si@unt@defqual{#1}{#2}}
    {\si@log@err{Qualifier \string#1 already defined!}\@eha}}
\newcommand*{\renewqualifier}[2]{%
  \si@ifdefinable{#1}
    {\si@log@err{Qualifier \string#1 undefined}\@ehc
     \si@unt@defqual{#1}{#2}}
    {\si@unt@defqual{#1}{#2}}}
\newcommand*{\providequalifier}[2]{%
  \si@ifdefinable{#1}
    {\si@unt@defqual{#1}{#2}}
    {}}
%    \end{macrocode}
%\end{macro}
%\end{macro}
%\end{macro}
%\begin{macro}{\ifsi@unt@num}
% A flag is needed to tell the processor whether there is a number,
% to get the correct spacing.
%    \begin{macrocode}
\newif\ifsi@unt@num
%    \end{macrocode}
%\end{macro}
%\begin{macro}{\si@unt@unitarg}
% A storage macro for the argument of the unit macro.
%    \begin{macrocode}
\newcommand*{\si@unt@unitarg}{}
%    \end{macrocode}
%\end{macro}
%\begin{macro}{\si@SI}
%\begin{macro}{\si@unt@SIopts}
%\doarg{keyval options}
%\darg{unit}
%\doarg{preunit}
%\darg{unit}
% The internal processing starts with \cs{si@SI}, which processes the
% second optional argument to \cs{SI} (which is empty for \cs{si}).
% Everything is set up in a group, and processing begins by handling
% the options.
%    \begin{macrocode}
\newcommand*{\si@unt@SIopts}{}
\def\si@SI[#1]#2[#3]#4{%
  \begingroup
    \let\fg\SIfg
    \si@ifnotmtarg{#1}
      {\sisetup{#1}%
       \renewcommand*{\si@unt@SIopts}{#1}}%
%    \end{macrocode}
%\end{macro}
% The prefix unit is handled before any processing of the number; the
% flags are set to get spacing correct.
%    \begin{macrocode}
    \si@unt@numfalse
    \si@xspacefalse
    \si@ifnotmtarg{#3}
      {\si@log@debug{Prefix unit found}%
       \si@unt@printunit{#3}}%
%    \end{macrocode}
% The numerical argument may be empty, in which case no extra space
% should be produced.
%    \begin{macrocode}
    \si@ifnotmtarg{#4}
      {\renewcommand*{\si@unt@unitarg}{#4}}%
    \si@ifnotmtarg{#2}
      {\si@log@debug{Number found in \string\SI\space
         argument}%
       \ifsi@repeatunits\else
         \ifsi@trapambigerr
           \expandafter\expandafter\expandafter
             \si@num@ambigerrtrue
         \fi
       \fi
       \num{#2}%
       \si@unt@numtrue}%
%    \end{macrocode}
% If there is a unit, a check is needed in case the units need to
% have a power added.
%    \begin{macrocode}
    \si@ifnotmtarg{#4}
      {\si@ifmtarg{#2}
         {\si@unt@printunit{#4}}
         {\si@tempcnta\z@\relax
         \ifsi@addunitpower
           \si@unt@countx{#2}%
         \fi
          \ifnum\si@tempcnta>\z@\relax
            \advance\si@tempcnta\@ne\relax
            \edef\si@tempa{\noexpand\tothe{\si@tempcnta}}%
            \renewcommand*{\si@tempb}{#4}%
            \expandafter\expandafter\expandafter
            \si@unt@printunit\expandafter\expandafter
            \expandafter{%
              \expandafter\si@tempb\si@tempa}%
          \else
            \si@unt@printunit{#4}%
          \fi}}%
  \endgroup}
%    \end{macrocode}
%\end{macro}
%\begin{macro}{\si@unt@countx}
% A short macro to count up any multiplication in numerical input.
%    \begin{macrocode}
\newcommand*{\si@unt@countx}[1]{%
  \si@tempcnta\z@\relax
  \expandafter\si@unt@cntx#1\@empty\@empty}
\def\si@unt@cntx#1#2\@empty{%
  \si@str@ifchrstr{#1}{\si@numprod}
    {\advance\si@tempcnta\@ne\relax}
    {}%
  \ifx\@empty#2\@empty\else
    \si@unt@cntx#2\@empty\@empty
  \fi}
%    \end{macrocode}
%\end{macro}
%\begin{macro}{\si@unt@ifliteral}
%\begin{macro}{\ifsi@unt@littest}
%\darg{unit}
% The next stage of the processor is to determine whether or not the
% argument of the unit macro is processable.  For literal arguments,
% this is not the case, and the argument is typeset ``as is''.  On
% the other hand, any units, \etc., declared by the package will work
% with the processor, and so need to be executed before typesetting
% the result.
%    \begin{macrocode}
\newif\ifsi@unt@littest
\newcommand*{\si@unt@ifliteral}[1]{%
  \begingroup
    \si@unt@littesttrue
%    \end{macrocode}
% The test relies on any non-processable test having some width;
% hopefully, this should be the case.
%    \begin{macrocode}
    \setbox\si@tempboxa=\hbox{\si@unt@out{#1}}%
    \ifdim\wd\si@tempboxa>\z@\relax
      \aftergroup\@firstoftwo
    \else
      \aftergroup\@secondoftwo
    \fi
  \endgroup}
%    \end{macrocode}
%\end{macro}
%\end{macro}
%\begin{macro}{\ifsi@unt@litout}
%\begin{macro}{\si@unt@printunit}
%\darg{unit}
% The printing macro uses the above test to determine how to act.  It
% then carries out the appropriate action: either typesetting or
% executing.  A flag is also provided so that any macro units inside
% a partially-literal argument will work (this is also needed to
% emulate \unitsdef).
%    \begin{macrocode}
\newif\ifsi@unt@litout
\newcommand*{\si@unt@printunit}[1]{%
  \si@unt@ifliteral{#1}
%    \end{macrocode}
% The unit includes one or more literal items; typeset using the
% unit typesetting macro.
%    \begin{macrocode}
    {\si@log@debug{Literal items found in unit
      argument:\MessageBreak outputting without further
      processing}%
     \si@unt@litouttrue
     \si@unt@addvaluesep
     \si@unt@out{#1}}
%    \end{macrocode}
% For processable output, the argument is executed; the macros are
% all designed for this.
%    \begin{macrocode}
    {\si@log@debug{Macro unit found:\MessageBreak
       processing to format output}%
     \si@unt@init
     \advance\si@unt@depthcnt\@ne\relax
     #1%
     \si@unt@final}}
%    \end{macrocode}
%\end{macro}
%\end{macro}
%\begin{macro}{\si@unt@addvaluesep}
%\begin{macro}{\si@unt@addvalsep}
%\begin{macro}{\si@unt@litvalsep}
%\begin{macro}{\si@unt@stackvalsep}
% To ensure no problems pop up with expansion, adding the value--unit
% space is handled by a macro.
%    \begin{macrocode}
\newcommand*{\si@unt@addvaluesep}{%
  \ifsi@unt@num
    \expandafter\si@unt@addvalsep
  \fi}
\newcommand*{\si@unt@addvalsep}{%
  \ifsi@unt@litout
    \expandafter\si@unt@litvalsep
  \else
    \expandafter\si@unt@stackvalsep
  \fi}
\newcommand*{\si@unt@stackvalsep}{%
  \protected@edef\si@unt@spstack{\si@valuesep}}
\newcommand*{\si@unt@litvalsep}{%
  \nobreak\ensuremath{\si@valuesep}\nobreak}
%    \end{macrocode}
%\end{macro}
%\end{macro}
%\end{macro}
%\end{macro}
%\begin{macro}{\si@unt@spstack}
%\begin{macro}{\si@unt@stacka}
%\begin{macro}{\si@unt@stackb}
%\begin{macro}{\si@unt@unitcnta}
%\begin{macro}{\si@unt@unitcntb}
%\begin{macro}{\si@unt@depthcnt}
%\begin{macro}{\ifsi@unt@first}
%\begin{macro}{\ifsi@unt@first}
%\begin{macro}{\si@unt@init}
% The initialisation macro sets up the various switches, and clears
% the storage areas for the formatted output.  There are two stacks,
% as when typesetting as fractions, the two parts of the number have
% to be stored separately.  The depth counter is used to tell when
% recursion ends in the processor.  The ``first'' switch is needed as
% the depth counter will not be at one for items processed by
% \cs{SI}.
%    \begin{macrocode}
\newcommand*{\si@unt@spstack}{}
\newcommand*{\si@unt@stacka}{}
\newcommand*{\si@unt@stackb}{}
\newcount\si@unt@unitcnta
\newcount\si@unt@unitcntb
\newcount\si@unt@depthcnt
\newif\ifsi@unt@first
\si@unt@depthcnt\m@ne\relax
\newcommand*{\si@unt@init}{%
  \begingroup
    \si@unt@litoutfalse
    \si@unt@litprefixfalse
    \si@unt@firsttrue
    \si@unt@perfalse
    \si@unt@perseenfalse
    \si@unt@prepowerfalse
    \si@unt@depthcnt\z@\relax
    \si@unt@powerdim\z@\relax
    \si@unt@unitcnta\z@\relax
    \si@unt@unitcntb\z@\relax
    \si@unt@prefixcnt\z@\relax
    \renewcommand*{\si@unt@spstack}{}%
    \renewcommand*{\si@unt@stacka}{}%
    \renewcommand*{\si@unt@stackb}{}%
    \renewcommand*{\si@unt@spacestacka}{}%
    \renewcommand*{\si@unt@spacestackb}{}%
    \renewcommand*{\si@unt@lastadda}{space}%
    \renewcommand*{\si@unt@lastaddb}{space}}
%    \end{macrocode}
%\end{macro}
%\end{macro}
%\end{macro}
%\end{macro}
%\end{macro}
%\end{macro}
%\end{macro}
%\end{macro}
%\end{macro}
%\begin{macro}{\si@unt@final}
% The finalisation macro finishes off the output and resets the flags.
%    \begin{macrocode}
\newcommand*{\si@unt@final}{%
    \si@unt@third
    \si@unt@stackout
  \endgroup
  \ifsi@xspace
    \expandafter\expandafter\expandafter\xspace
  \fi}
%    \end{macrocode}
%\end{macro}
%\begin{macro}{\si@unt@defunit}
%\doarg{keyval options}
%\darg{unit}
%\darg{symbol}
% The internal macro for defining a unit does not check for
% redefinition; that is done by the user macros.
%    \begin{macrocode}
\newcommand*{\si@unt@defunit}[3][]{%
  \si@log@debug{Declaring unit \string#2 with \MessageBreak
    meaning \string#3}%
%    \end{macrocode}
% The optional argument needs to be saved. The macro name is reversed
% so that life is easier with the expansions here.
%    \begin{macrocode}
  \si@ifnotmtarg{#1}
    {\expandafter\@namedef\expandafter{%
      \expandafter\@gobble\string#2@opt@unt@si}{#1}}%
%    \end{macrocode}
% The unit macro itself is now defined.  The definition simply
% selects the correct path for the rest of the processing to go down.
%    \begin{macrocode}
  \protected\def#2{%
    \ifsi@allowoptarg
      \expandafter\si@unt@withopt
    \else
      \expandafter\si@unt@noopt
    \fi
    {#2}{#3}}}
%    \end{macrocode}
%\end{macro}
%\begin{macro}{\si@unt@withopt}
%\begin{macro}{\si@unt@noopt}
%\darg{unit}
%\darg{symbol}
% To allow the correct expansion, the potential optional argument to
% a unit macro has to come last.  So \cs{@ifnextchar} is needed to
% detect it and pass data through.  To keep variation down, when the
% argument is not allowed, the empty |[]| is supplied.
%    \begin{macrocode}
\newcommand*{\si@unt@withopt}[2]{%
  \@ifnextchar[%]
    {\si@unt@opt{#1}{#2}}
    {\si@unt@opt{#1}{#2}[]}}
\newcommand*{\si@unt@noopt}[2]{\si@unt@opt{#1}{#2}[]}
%    \end{macrocode}
%\end{macro}
%\end{macro}
%\begin{macro}{\si@unt@opt}
%\darg{unit}
%\darg{symbol}
%\doarg{number}
% The optional argument to the unit macro (if present) is converted
% to a normal one for ease.  The correct route for processing is then
% picked.
%    \begin{macrocode}
\def\si@unt@opt#1#2[#3]{%
  \ifsi@unt@littest
    \expandafter\si@gobblethree
  \else
%    \end{macrocode}
% For literal output, the second argument is all that is needed.
%    \begin{macrocode}
    \ifsi@unt@litout
      \expandafter\expandafter\expandafter\@gobbletwo
    \else
      \expandafter\expandafter\expandafter\si@unt@unit
    \fi
  \fi
  {#3}{#1}{#2}}
%    \end{macrocode}
%\end{macro}
%\begin{macro}{\si@gobblethree}
% \LaTeX\ does not have a \cs{@gobblethree} macro, but one is needed.
%    \begin{macrocode}
\long\def\si@gobblethree #1#2#3{}
%    \end{macrocode}
%\end{macro}
%\begin{macro}{\si@unt@defprefix}
%\begin{macro}{\ifsi@unt@litprefix}
%\doarg{\texttt{binary}}
%\darg{multiple}
%\darg{powers-ten}
%\darg{symbol}
% As with units, multiples are defined by an internal macro.
%    \begin{macrocode}
\newif\ifsi@unt@litprefix
\si@unt@litprefixtrue
\newcommand*{\si@unt@defprefix}[4][]{%
  \si@log@debug{Declaring multiple \string#1 with\MessageBreak
    meaning \string#4}%
%    \end{macrocode}
% The optional argument is saved, using \cs{def} as no check is made
% on an existing definition of the storage macro.
%    \begin{macrocode}
  \expandafter\expandafter\expandafter\def\expandafter
    \csname\expandafter\@gobble\string#2@opt@si\endcsname{#1}%
  \protected\def#2{%
    \ifsi@unt@littest
      \expandafter\si@gobblethree
    \else
      \ifsi@unt@litout
        \expandafter\expandafter\expandafter\@gobbletwo
      \else
        \ifsi@unt@litprefix
          \expandafter\expandafter\expandafter\expandafter
            \expandafter\expandafter\expandafter\@gobbletwo
        \else
          \expandafter\expandafter\expandafter\expandafter
            \expandafter\expandafter\expandafter\si@unt@prefix
        \fi
      \fi
    \fi
    {#2}{#3}{#4}}}
%    \end{macrocode}
%\end{macro}
%\end{macro}
%\begin{macro}{\si@unt@defpower}
%\doarg{\texttt{post}}
%\darg{number}
%\darg{power}
% The definition of powers is complicated by the need to handle both
% those given before units (such as \cs{cubic}) and those given after
% (\eg \cs{cubed}).  This means that an optional argument is needed.
%    \begin{macrocode}
\newcommand*{\si@unt@defpower}[3][]{%
  \si@log@debug{Declaring power \string#2 with\MessageBreak
    meaning \string#3}%
%    \end{macrocode}
% Once again the optional argument is saved.
%    \begin{macrocode}
  \expandafter\expandafter\expandafter\def\expandafter
    \csname\expandafter\@gobble\string#2@opt@si\endcsname{#1}%
  \protected\def#2{%
    \ifsi@unt@littest
      \expandafter\@gobbletwo
    \else
%    \end{macrocode}
% The literal output here does not need to gobble anything.
%    \begin{macrocode}
      \ifsi@unt@litout
        \expandafter\expandafter\expandafter\si@unt@litpower
      \else
        \expandafter\expandafter\expandafter\si@unt@power
      \fi
    \fi
    {#2}{#3}}}
%    \end{macrocode}
%\end{macro}
%\begin{macro}{\si@unt@defqual}
%\darg{qualifier}
%\darg{symbol}
% Qualifiers are defined in a similar manner.
%    \begin{macrocode}
\newcommand*{\si@unt@defqual}[2]{%
  \si@log@debug{Declaring qualifier \string#1 with\MessageBreak
    meaning \string#2}%
  \protected\def#1{%
    \ifsi@unt@littest
      \expandafter\@gobble
    \else
      \ifsi@unt@litout
        \expandafter\expandafter\expandafter\si@unt@litqual
      \else
        \expandafter\expandafter\expandafter\si@unt@qual
      \fi
    \fi
    {#2}}}
%    \end{macrocode}
%\end{macro}
%
%\begin{macro}{\si@unt@unithook}
%\begin{macro}{\si@unt@unit}
%\darg{number}
%\darg{unit}
%\darg{symbol}
% The macro for units is actually a processor, rather than
% typesetting anything, which is handled elsewhere.  The first
% argument to the macro is optional, but does not have square
% brackets to keep things simple with gobbling.
%    \begin{macrocode}
\newcommand*{\si@unt@unithook}{}
\newcommand*{\si@unt@unit}[3]{%
%    \end{macrocode}
% When the count is minus one at the start of the processor, then the
% unit is begin used on its own: initialisation occurs.
%    \begin{macrocode}
  \ifnum\si@unt@depthcnt=\m@ne\relax
    \expandafter\si@unt@init
  \fi
  \advance\si@unt@depthcnt\@ne\relax
  \si@log@debug{Unit processing: level \the\si@unt@depthcnt,
    \MessageBreak unit \string#2}%
  \si@unt@firstorsecond{#1}{#2}%
%    \end{macrocode}
% The core of the \cs{si@unt@unit} macro is testing if the symbol for
% the unit is a literal value or another macro.  Depending on the
% result, the symbol is either used as a literal or executed.
%    \begin{macrocode}
  \si@unt@ifliteral{#3}
    {\si@unt@addtostack{unit}{#3}%
     \ifsi@unt@prepower
       \expandafter\si@unt@stkpower
     \fi}
    {#3}%
%    \end{macrocode}
% The counter is now stepped down, before checking if this is
% the end of a compound unit.
%    \begin{macrocode}
  \advance\si@unt@depthcnt\m@ne\relax
  \ifnum\si@unt@depthcnt=\z@\relax
    \expandafter\si@unt@final
  \fi}
%    \end{macrocode}
%\end{macro}
%\end{macro}
%\begin{macro}{\si@unt@firstorsecond}
%\darg{number}
%\darg{unit-macro}
% At this stage, the flag will be set for the first item to be
% processed whichever route the unit has been called by.
%    \begin{macrocode}
\newcommand*{\si@unt@firstorsecond}[2]{%
  \ifsi@unt@first
    \expandafter\si@unt@first
  \else
    \expandafter\si@unt@second
  \fi
  {#1}{#2}}%
%    \end{macrocode}
%\end{macro}
%\begin{macro}{\si@unt@first}
%\darg{number}
%\darg{unit-macro}
% For the first unit in the input, some extra tasks are needed.
% First, the optional argument for the unit macro needs to be
% tested.
%    \begin{macrocode}
\newcommand*{\si@unt@first}[2]{%
  \si@ifnotmtarg{#1}
    {\num{#1}%
     \si@unt@numtrue}%
  \si@unt@unithook
%    \end{macrocode}
% To avoid filling up the macro list with useless values, the \eTeX\
% primitive \cs{ifcsname} is employed here (it also avoids complex
% expansion issues).  If some options exist, they are set.
%    \begin{macrocode}
  \ifcsname\expandafter\@gobble\string#2@opt@unt@si\endcsname
    \expandafter\si@unt@setopts
  \else
    \expandafter\@gobble
  \fi
  {#2}%
  \si@unt@addvaluesep
  \si@unt@firstfalse}
%    \end{macrocode}
%\end{macro}
%\begin{macro}{\si@unt@setopts}
%\begin{macro}{\si@unt@setSIopts}
%\darg{unit}
% A rather long set of \cs{expandafter} commands to get the options
% to set safely.
%    \begin{macrocode}
\newcommand*{\si@unt@setopts}[1]{%
  \expandafter\expandafter\expandafter\expandafter\expandafter
    \expandafter\expandafter\si@temptoks\expandafter
    \expandafter\expandafter\expandafter\expandafter
    \expandafter\expandafter{\expandafter%
      \csname\expandafter\@gobble\string#1@opt@unt@si%
      \endcsname}%
  \expandafter\sisetup\expandafter{\the\si@temptoks}%
  \si@log@debug{Applying options `\the\si@temptoks'
    for\MessageBreak unit \string#1}%
%    \end{macrocode}
% The user options are reloaded, if defined, to ensure that they
% still work as expected.
%    \begin{macrocode}
  \@ifundefined{si@unt@SIopts}{}
    {\ifx\@empty\si@unt@SIopts\@empty\else
       \expandafter\expandafter\si@unt@setSIopts
     \fi}}
\newcommand*{\si@unt@setSIopts}{%
  \expandafter\si@temptoks\expandafter{\si@unt@SIopts}%
  \expandafter\sisetup\expandafter{\the\si@temptoks}}
%    \end{macrocode}
%\end{macro}
%\end{macro}
%\begin{macro}{\si@unt@second}
%\begin{macro}{\si@unt@third}
%\darg{number}
%\darg{unit}
% For everything apart from the first item to be processed, spacing
% may need to be added to separated different units.  The macro is
% divided into two, so that everything except the space can be added
% in finalisation.
%    \begin{macrocode}
\newcommand*{\si@unt@second}[2]{%
  \si@ifnotmtarg{#1}
    {\si@log@warn{Optional argument to unit macro\MessageBreak
       allowed only for outer unit}}%
  \si@unt@third
  \si@unt@addtostack{space}{\ensuremath{\si@unitsep}}}
\newcommand*{\si@unt@third}{%
  \ifsi@unt@prepower\else
    \expandafter\si@unt@stkpower
  \fi
%    \end{macrocode}
% A check is made to avoid adding $-1$ to prefixes.  If \opt{frac} is
% active, then the |b| stack will be in use, otherwise it will be
% |a|.
%    \begin{macrocode}
  \renewcommand*{\si@tempa}{prefix}%
  \expandafter\ifx
    \csname si@unt@lastadd\si@unt@checkstack\endcsname\si@tempa
  \else
    \expandafter\si@unt@spacecheck
  \fi
  \ifsi@unt@per
    \expandafter\si@unt@perseentrue
  \fi}
%    \end{macrocode}
%\end{macro}
%\end{macro}
%\begin{macro}{\si@unt@spacecheck}
% A check to prevent adding $-1$ at the very beginning of the unit,
% where there is a space on the stack.
%    \begin{macrocode}
\newcommand*{\si@unt@spacecheck}{%
  \renewcommand*{\si@tempa}{space}%
  \expandafter\ifx
    \csname si@unt@lastadd\si@unt@checkstack\endcsname\si@tempa
  \else
    \expandafter\si@unt@reciptest
  \fi}
%    \end{macrocode}
%\end{macro}
%\begin{macro}{\si@unt@prefix}
%\darg{multiple}
%\darg{power-ten}
%\darg{symbol}
% Actual output of the prefixes.
%    \begin{macrocode}
\newcommand*{\si@unt@prefix}[3]{%
  \si@unt@firstorsecond{}{#1}%
  \ifsi@prefixsymbolic
    \expandafter\si@unt@addprefix
  \else
    \expandafter\si@unt@countprefix
  \fi
  {#1}{#2}{#3}}
%    \end{macrocode}
%\end{macro}
%\begin{macro}{\si@unt@addprefix}
%\darg{gobbled}
%\darg{gobbled}
%\darg{symbol}
% To add the prefix, a little translation is needed.
%    \begin{macrocode}
\newcommand*{\si@unt@addprefix}[3]{%
  \si@unt@addtostack{prefix}{#3}}
%    \end{macrocode}
%\end{macro}
%\begin{macro}{\si@unt@prefixcnt}
% A storage area is created.
%    \begin{macrocode}
\newcount\si@unt@prefixcnt
%    \end{macrocode}
%\end{macro}
%\begin{macro}{\si@unt@countprefix}
%\begin{macro}{\si@unt@invprefix}
%\darg{multiple}
%\darg{powers-ten}
%\darg{gobbled}
% To count the prefix numeral, the symbol is thrown away.  First, a
% check is made for binary units.
%    \begin{macrocode}
\newcommand*{\si@unt@countprefix}[3]{%
  \renewcommand*{\si@tempa}{binary}%
  \expandafter\expandafter\expandafter\ifx\expandafter
    \csname\expandafter\@gobble\string#1@opt@si\endcsname
    \si@tempa
    \expandafter\sisetup
  \else
    \expandafter\@gobble
  \fi
  {prefixbase=two}%
  \si@tempcnta#2\relax
  \ifsi@unt@per
    \expandafter\si@unt@invprefix
  \fi
  \advance\si@unt@prefixcnt\si@tempcnta\relax}
\newcommand*{\si@unt@invprefix}{%
  \si@tempcntb\si@tempcnta\relax
  \si@tempcnta -\si@tempcntb\relax}
%    \end{macrocode}
%\end{macro}
%\end{macro}
%\begin{macro}{\si@unt@litpower}
%\darg{gobbled}
%\darg{number}
% For literal power output, the number can't simply be dumped, so
% a macro is needed.
%    \begin{macrocode}
\newcommand*{\si@unt@litpower}[2]{\textsuperscript{#2}}
%    \end{macrocode}
%\end{macro}
%\begin{macro}{\ifsi@unt@prepower}
% If a power is seen before a unit, tracking is needed.
%    \begin{macrocode}
\newif\ifsi@unt@prepower
%    \end{macrocode}
%\end{macro}
%\begin{macro}{\si@unt@power}
%\darg{power}
%\darg{number}
% The handling of powers starts by ensuring that ``pre'' powers
% follow \cs{per} cleanly.  Then a check is needed for inversion.
%    \begin{macrocode}
\newcommand*{\si@unt@power}[2]{%
  \renewcommand*{\si@tempa}{post}%
  \expandafter\expandafter\expandafter\ifx\expandafter
    \csname\expandafter\@gobble\string#1@opt@si\endcsname
    \si@tempa
    \expandafter\@gobbletwo
  \else
    \expandafter\si@unt@firstorsecond
  \fi
  {}{\power}%
  \si@unt@powerdim #2 pt\relax
  \ifsi@frac\else
    \ifsi@unt@per
      \expandafter\expandafter\expandafter\si@unt@invpower
    \fi
  \fi
  \renewcommand*{\si@tempa}{post}%
  \si@unt@prepowertrue
  \expandafter\expandafter\expandafter\ifx\expandafter
    \csname\expandafter\@gobble\string#1@opt@si\endcsname
    \si@tempa
    \expandafter\si@unt@stackpower
  \else
    \si@log@debug{Power \strip@pt\si@unt@powerdim\space saved
      to be added after\MessageBreak next unit}%
  \fi}
%    \end{macrocode}
%\end{macro}
%\begin{macro}{\si@unt@powerdim}
% To do sign-inversion on the power, a dimension is used (this allows
% non-integer values to be handled).
%    \begin{macrocode}
\newdimen\si@unt@powerdim
%    \end{macrocode}
%\end{macro}
%\begin{macro}{\si@unt@stackpower}
%\begin{macro}{\si@unt@stkpower}
%\begin{macro}{\si@unt@stkpwr}
%\begin{macro}{\si@unt@skpwr}
% Adding powers to the stack should also clear the power list.  If
% the number is already zero, then  of course nothing happens.
%    \begin{macrocode}
\newcommand*{\si@unt@stackpower}{%
  \si@unt@prepowerfalse
%    \end{macrocode}
% A trap is used for $-1$ added to the denominator of a fraction.
%    \begin{macrocode}
  \si@unt@stkpower
  \ifsi@stickyper\else
    \si@unt@perfalse
    \si@unt@perseenfalse
  \fi}
%    \end{macrocode}
% The \cs{si@unt@stkpower} macro needs to be called from a few
% places, so is spun out from the above.
%    \begin{macrocode}
\newcommand*{\si@unt@stkpower}{%
  \ifdim\si@unt@powerdim=\m@ne pt\relax
    \ifsi@frac\else
      \expandafter\expandafter\expandafter\si@unt@stkpwr
    \fi
  \else
    \expandafter\si@unt@stkpwr
  \fi}
%    \end{macrocode}
%\changes{v1.1}{2008/09/04}{Altered to account for qualifiers}
% Finally, the actual adding (set up to avoid problems with the
% \cs{if} above).
%    \begin{macrocode}
\newcommand*{\si@unt@stkpwr}{%
  \ifdim\si@unt@powerdim=\z@\relax\else
    \renewcommand*{\si@tempa}{unit}%
    \expandafter\ifx
      \csname si@unt@lastadd\si@unt@checkstack\endcsname
      \si@tempa
      \si@unt@skpwr
    \else
      \renewcommand*{\si@tempa}{qual}%
      \expandafter\ifx
        \csname si@unt@lastadd\si@unt@checkstack\endcsname
        \si@tempa
        \si@unt@skpwrqual
      \fi
    \fi
  \fi
  \si@unt@powerdim\z@\relax}
\newcommand*{\si@unt@skpwr}{%
  \si@log@debug{Adding power
    \strip@pt\si@unt@powerdim\space to output stack}%
  \si@unt@addtostack{power}{^{\num[fixdp=false]{%
    \strip@pt\si@unt@powerdim}}}}
%    \end{macrocode}
%\end{macro}
%\end{macro}
%\end{macro}
%\end{macro}
%\begin{macro}{\si@unt@invpower}
% A macro to change the sign of the current power.
%    \begin{macrocode}
\newcommand*{\si@unt@invpower}{%
  \si@tempdima\si@unt@powerdim\relax
  \si@unt@powerdim -\si@tempdima\relax
%    \end{macrocode}
% The power might end up as ``1'', which is not wanted.  So it is
% chucked away.
%    \begin{macrocode}
  \ifdim\si@unt@powerdim=\p@\relax
    \si@unt@powerdim\z@\relax
  \fi}
%    \end{macrocode}
%\end{macro}
%\begin{macro}{\si@unt@litqual}
%    \begin{macrocode}
\newcommand*{\si@unt@litqual}[1]{(#1)}
%    \end{macrocode}
%\end{macro}
%\begin{macro}{\si@unt@qual}
%    \begin{macrocode}
\newcommand*{\si@unt@qual}[1]{%
  \ifsi@unt@first
    \si@log@err{Qualifier before units}
      {A unit qualifier can only be given after\MessageBreak
       one or more units.  You tried to open a unit
       with\MessageBreak \string#1}%
  \fi
%    \end{macrocode}
% The space after the current unit is shuffled so it will come
% after the qualifier.
%    \begin{macrocode}
  \expandafter\let\expandafter\si@unt@spacestack
    \csname si@unt@spacestack\si@unt@checkstack\endcsname
  \expandafter\let\csname si@unt@spacestack\si@unt@checkstack
    \endcsname\@empty
%    \end{macrocode}
%\begin{macro}{\si@unt@skpwrqual}
% Some protection against trying to add two powers.
%    \begin{macrocode}
  \renewcommand*{\si@tempa}{power}%
    \expandafter\ifx
    \csname si@unt@lastadd\si@unt@checkstack\endcsname\si@tempa
    \let\si@unt@skpwrqual\relax
  \else
    \let\si@unt@skpwrqual\si@unt@skpwr
  \fi
  \si@unt@addtostack{qual}{_{#1}}%
  \expandafter\let\csname si@unt@spacestack\si@unt@checkstack
    \endcsname\si@unt@spacestack}
%    \end{macrocode}
%\end{macro}
%\end{macro}
%\begin{macro}{\ifsi@unt@per}
%\begin{macro}{\ifsi@unt@perseen}
%\begin{macro}{\per}
%\begin{macro}{\si@per}
%\begin{macro}{\si@unt@per}
% The \cs{per} macro sets the correct flags; almost everything else
% is done elsewhere.  There is always the case of two \cs{per}
% instructions; so the flag is flipped rather than set arbitrarily.
% The second flag is needed so that \cs{per} can give powers of $-1$
% properly.
%    \begin{macrocode}
\newif\ifsi@unt@per
\newif\ifsi@unt@perseen
\si@newrobustcmd*{\si@per}{%
  \ifsi@unt@littest\else
    \ifsi@unt@litout
      \expandafter\expandafter\expandafter /%
    \else
      \ifsi@unt@litprefix
        \expandafter\expandafter\expandafter\expandafter
          \expandafter\expandafter\expandafter /%
      \else
        \expandafter\expandafter\expandafter\expandafter
          \expandafter\expandafter\expandafter\si@unt@per
      \fi
    \fi
  \fi}
\newcommand*{\si@unt@per}{%
  \si@unt@firstorsecond{}{\per}%
  \ifsi@unt@per
    \ifsi@stickyper\else
      \expandafter\expandafter\expandafter\si@unt@perfalse
    \fi
  \else
    \expandafter\si@unt@pertrue
  \fi}
\let\per\si@per
%    \end{macrocode}
%\end{macro}
%\end{macro}
%\end{macro}
%\end{macro}
%\end{macro}
%\begin{macro}{\si@unt@reciptest}
%\begin{macro}{\si@unt@recip}
% A test is needed for adding $-1$ when needed. The second macro is
% fired only if the power should be reciprocal.
%    \begin{macrocode}
\newcommand*{\si@unt@reciptest}{%
  \ifsi@unt@per
    \ifsi@unt@perseen
      \expandafter\expandafter\expandafter\si@unt@recip
    \fi
  \fi}
\newcommand*{\si@unt@recip}{%
  \si@unt@powerdim\m@ne pt\relax
  \si@unt@stackpower}
%    \end{macrocode}
%\end{macro}
%\end{macro}
%\begin{macro}{\si@unt@lastadda}
%\begin{macro}{\si@unt@lastaddb}
% Items cannot be added directly to the stacks (except the spacing
% stack, a) as the fractional handling may need to add the item to
% either storage area. First, a track is kept of what has
% been added at each stage.
%    \begin{macrocode}
\newcommand*{\si@unt@lastadda}{}
\newcommand*{\si@unt@lastaddb}{}
%    \end{macrocode}
%\end{macro}
%\end{macro}
%\begin{macro}{\si@unt@addtostack}
%\changes{v1.1b}{2008/10/12}{Added warning for multiple prefixes}
%\darg{type}
%\darg{token}
% By indicating the type of data to be added to the stack, problems
% can be avoided.
%    \begin{macrocode}
\newcommand*{\si@unt@addtostack}[2]{%
  \renewcommand*{\si@tempa}{#1}%
%    \end{macrocode}
% Two consecutive items cannot be of the same type; there must be spaces
% between units, units between prefixes, \etc.
%    \begin{macrocode}
  \expandafter\ifx
    \csname si@unt@lastadd\si@unt@checkstack\endcsname\si@tempa
      \renewcommand*{\si@tempa}{prefix}%
      \expandafter\ifx\csname si@unt@lastadd\si@unt@checkstack
        \endcsname\si@tempa
        \si@log@warn{Two prefixes found: only first printed}%
      \fi
    \expandafter\@gobbletwo
  \else
    \expandafter\si@unt@preplussp
  \fi
  {#1}{#2}}
%    \end{macrocode}
%\end{macro}
%\begin{macro}{\si@unt@preplussp}
%\darg{type}
%\darg{stack}
%\darg{token}
%\darg{gobbled}
% The space added after a prefix needs to be ignored.
%    \begin{macrocode}
\newcommand*{\si@unt@preplussp}[2]{%
  \renewcommand*{\si@tempa}{prefix+space}%
  \edef\si@tempb{%
    \csname si@unt@lastadd\si@unt@checkstack\endcsname+#1}%
  \ifx\si@tempa\si@tempb
    \expandafter\@gobbletwo
  \else
    \expandafter\si@unt@stack
  \fi
  {#1}{#2}}
%    \end{macrocode}
%\end{macro}
%\begin{macro}{\si@unt@stack}
%\darg{type}
%\darg{token}
% The macro for actually doing the stacking up.
%    \begin{macrocode}
\newcommand*{\si@unt@stack}[2]{%
  \expandafter\renewcommand\expandafter*\expandafter{%
    \csname si@unt@lastadd\si@unt@checkstack\endcsname}{#1}%
%    \end{macrocode}
% A count is kept of the number of \emph{units} added to each stack.
%    \begin{macrocode}
  \renewcommand*{\si@tempa}{#1}%
  \renewcommand*{\si@tempb}{unit}%
  \ifx\si@tempa\si@tempb
    \expandafter\si@unt@inccnt
  \fi
%    \end{macrocode}
% If a space is added, it is actually held until the next add.
%    \begin{macrocode}
  \renewcommand*{\si@tempb}{space}%
  \ifx\si@tempa\si@tempb
    \expandafter\si@unt@holdspace
  \else
    \expandafter\si@unt@addstack
  \fi
  {#2}}
%    \end{macrocode}
%\end{macro}
%\begin{macro}{\si@unt@inccnt}
% The appropriate counter is added to.
%    \begin{macrocode}
\newcommand*{\si@unt@inccnt}{%
  \expandafter\advance
    \csname si@unt@unitcnt\si@unt@checkstack\endcsname
    \@ne\relax}
%    \end{macrocode}
%\end{macro}
%\begin{macro}{\si@unt@spacestack}
%\begin{macro}{\si@unt@spacestacka}
%\begin{macro}{\si@unt@spacestackb}
% The stacked material needs somewhere to live.
%    \begin{macrocode}
\newcommand*{\si@unt@spacestack}{}
\newcommand*{\si@unt@spacestacka}{}
\newcommand*{\si@unt@spacestackb}{}
%    \end{macrocode}
%\end{macro}
%\end{macro}
%\end{macro}
%\begin{macro}{\si@unt@holdspace}
%\begin{macro}{\si@unt@addstack}
%\darg{token}
% Depending on the nature of the addition, it is either held or added
% to the stack.  For the ``b'' space stack, a check is made to ensure
% that a space cannot be added before the first item.
%    \begin{macrocode}
\newcommand*{\si@unt@holdspace}[1]{%
  \renewcommand*{\si@tempa}{b}%
  \edef\si@tempb{\si@unt@checkstack}%
  \ifx\si@tempa\si@tempb
    \ifx\@empty\si@unt@stackb\@empty
    \else
      \protected@edef\si@unt@spacestackb{#1}%
    \fi
  \else
    \protected@edef\si@unt@spacestacka{#1}%
  \fi}
\newcommand*{\si@unt@addstack}[1]{%
  \expandafter\protected@edef
    \csname si@unt@stack\si@unt@checkstack\endcsname
    {\csname si@unt@stack\si@unt@checkstack\endcsname
     \csname si@unt@spacestack\si@unt@checkstack\endcsname#1}%
  \expandafter\renewcommand\expandafter*\expandafter{%
    \csname si@unt@spacestack\si@unt@checkstack\endcsname}{}}
%    \end{macrocode}
%\end{macro}
%\end{macro}
%\begin{macro}{\si@unt@stackout}
% The stack contents are actually typeset here.  First the spacing
% between units and values is added.
%    \begin{macrocode}
\newcommand*{\si@unt@stackout}{%
  \si@unt@litouttrue
  \ifsi@frac
    \expandafter\si@unt@fracout
  \else
    \expandafter\si@unt@normout
  \fi}
%    \end{macrocode}
%\end{macro}
%\begin{macro}{\si@unt@checkstack}
% Which stack is in use needs to be tested.
%    \begin{macrocode}
\newcommand*{\si@unt@checkstack}{%
  \ifsi@frac
    \ifsi@unt@per
      \expandafter\expandafter\expandafter b%
    \else
      \expandafter\expandafter\expandafter a%
    \fi
  \else
    \expandafter a%
  \fi}
%    \end{macrocode}
%\end{macro}
%\begin{macro}{\si@unt@spaceout}
% The space before a unit might not be needed, so it crops up a few
% times in the output routine.
%    \begin{macrocode}
\newcommand*{\si@unt@spaceout}{%
   \ensuremath{\si@unt@spstack}}
%    \end{macrocode}
%\end{macro}
%\begin{macro}{\si@unt@prefixout}
% If the prefix counter is not zero, then there is something to
% typeset.
%    \begin{macrocode}
\newcommand*{\si@unt@prefixout}{%
  \ifnum\si@unt@prefixcnt=\z@\relax\else
    \ifsi@unt@num
      \si@out{\ensuremath{{}\si@prefixproduct{}}}%
    \fi
    \si@unt@stackvalsep
    \let\si@expbase\si@prefixbase
    \num[fixdp=false]{e\the\si@unt@prefixcnt}%
  \fi}
%    \end{macrocode}
%\end{macro}
%\begin{macro}{\si@unt@normout}
% The normal output mode is set up here; nothing much needs to be
% done as there is no need for complex checks.
%    \begin{macrocode}
\newcommand*{\si@unt@normout}{%
  \si@unt@prefixout
  \si@unt@spaceout
  \expandafter\si@unt@out\expandafter{\si@unt@stacka}}
%    \end{macrocode}
%\end{macro}
%\begin{macro}{\si@unt@fracout}
% For fractions, some checks are needed.
%    \begin{macrocode}
\newcommand*{\si@unt@fracout}{%
  \si@unt@notambig
  \ifx\@empty\si@unt@stacka\@empty
    \ifx\@empty\si@unt@stackb\@empty
      \ifsi@unt@litout\else
        \si@log@err{Empty fractional unit}{The unit
          argument\MessageBreak given does not contain any
          symbols}%
      \fi
    \else
%    \end{macrocode}
% With an empty numerator, no space is added
%    \begin{macrocode}
      \ifsi@slash
        \si@unt@prefixout
        \si@frac{}{\si@unt@stackb}%
      \else
        \si@unt@prefixout
        \si@unt@spaceout
        \si@frac{1}{\si@unt@stackb}%
      \fi
    \fi
  \else
%    \end{macrocode}
% If the denominator is empty, then the usual output system can be
% used.
%    \begin{macrocode}
    \ifx\@empty\si@unt@stackb\@empty
      \si@unt@normout
    \else
      \si@unt@prefixout
      \si@unt@spaceout
      \si@frac{\si@unt@stacka}{\si@unt@stackb}%
    \fi
  \fi}
%    \end{macrocode}
%\end{macro}
%\begin{macro}{\si@unt@notambig}
%\begin{macro}{\si@unt@notabg}
% A trap is set for adding brackets to units using a slash, when more
% than one unit is in the denominator.
%    \begin{macrocode}
\newcommand*{\si@unt@notambig}{%
  \ifnum\si@unt@unitcntb>\@ne\relax
    \ifsi@slash
        \ifsi@trapambigfrac
          \expandafter\expandafter\expandafter\expandafter
            \expandafter\expandafter\expandafter\si@unt@notabg
        \fi
    \fi
  \fi}
\newcommand*{\si@unt@notabg}{%
  \protected@edef\si@unt@stackb{\si@openfrac\si@unt@stackb
    \si@closefrac}}
%    \end{macrocode}
%\end{macro}
%\end{macro}
%
%\begin{macro}{\si@unt@out}
%\changes{v1.0d}{2008/06/25}{Sanitise one level of unit input
%  only}
%\darg{unit}
% The final part of the units system is the output routine.  This has
% to cope with units not only as macros but also as direct input
% (\SIstyle-type input).  Non-Latin characters are also handled
% cleanly.  The \cs{scantokens} system deals with everything except
% full stops; these are left out so that a single level system can be
% used \latin{via} a token register.
%    \begin{macrocode}
\begingroup
  \catcode`\~=\active
  \catcode`\.=\active
  \gdef\si@unt@out#1{%
    \si@temptoks{#1}%
    \si@unt@fullstop
    \def.{\ensuremath{\si@unitsep}}%
    \def~{\ensuremath{\si@unitspace}}%
    \expandafter\protected@edef\expandafter\si@tempa
      \expandafter{\the\si@temptoks}%
    \begingroup
      \si@unt@nonlatin
      \makeatletter
      \endlinechar\m@ne
      \expandafter\si@out\expandafter{%
        \expandafter\scantokens\expandafter{\si@tempa}}%
    \endgroup}
\endgroup
%    \end{macrocode}
%\end{macro}
%\begin{macro}{\si@unt@fullstop}
%\begin{macro}{\si@unt@stp}
% Two macros modified from \pkg{kvsetkeys} to deal with a single
% level of active full stops only.
%    \begin{macrocode}
\begingroup
  \catcode`\.=\active
  \catcode`\&=12\relax
  \begingroup
    \lccode`\.=`\.\relax
    \lccode`\&=`\.\relax
  \lowercase{\endgroup
    \gdef\si@unt@fullstop{%
      \si@temptoks\expandafter{\expandafter}\expandafter
      \si@unt@stp\the\si@temptoks&\@nil}
  \gdef\si@unt@stp#1&#2\@nil{%
    \edef\si@tempa{\the\si@temptoks}%
    \ifx\si@tempa\@empty
      \expandafter\@firstoftwo
    \else
      \expandafter\@secondoftwo
    \fi
      {\si@temptoks{#1}}
      {\si@temptoks\expandafter{\the\si@temptoks.#1}}%
    \si@ifmtarg{#2}
      {}
      {\si@unt@stp#2\@nil}}}
\endgroup
%    \end{macrocode}
%\end{macro}
%\end{macro}
%\begin{macro}{\si@unt@nonlatin}
% To handle non-Latin symbols in the input, a single macro is
% provided.  If \XeTeX\ is in use, this can be detected immediately.
%    \begin{macrocode}
\newcommand*{\si@unt@nonlatin}{}
\ifdefined\XeTeXrevision
  \renewcommand*{\si@unt@nonlatin}{%
    \catcode176=\active
    \catcode181=\active
    \catcode197=\active
    \si@unt@sym{176}{\si@sym@degree}%
    \si@unt@sym{181}{\si@sym@mu}%
    \si@unt@sym{197}{\si@sym@ringA}}%
\fi
%    \end{macrocode}
% If \pkg{inputenc} has been loaded, then a check is made that the
% encoding is correct.  If all is well, the non-Latin symbols are
% handled.  The degree symbol is character 176, the micro
% symbol is character 181 and ring capital A is character 197 in
% latin1.
%    \begin{macrocode}
\AtBeginDocument{
  \@ifpackageloaded{inputenc}
    {\@for\si@tempa:=latin1,latin5,latin9\do{
       \ifx\inputencodingname\si@tempa
         \renewcommand*{\si@unt@nonlatin}{%
           \catcode176=\active
           \catcode181=\active
           \catcode197=\active
           \si@unt@sym{176}{\si@sym@degree}%
           \si@unt@sym{181}{\si@sym@mu}%
           \si@unt@sym{197}{\si@sym@ringA}}%
       \fi}}
    {}}
%    \end{macrocode}
%\end{macro}
%\begin{macro}{\si@unt@sym}
%\darg{charcode}
% A macro for declaring symbols: a copy of \cs{DeclareInputMath} from
% \pkg{inputenc}.
%    \begin{macrocode}
\newcommand*{\si@unt@sym}[1]{%
  \bgroup
    \uccode`\~#1%
    \uppercase{%
      \egroup
      \def~}}
%    \end{macrocode}
%\end{macro}
%
%\begin{macro}{\kilogram}
%\begin{macro}{\metre}
%\begin{macro}{\meter}
%\begin{macro}{\mole}
%\begin{macro}{\kelvin}
%\begin{macro}{\candela}
%\begin{macro}{\second}
%\begin{macro}{\ampere}
% With the system set up, the basic unit macros are implemented.  The
% only units defined whatever options are given are the base SI
% units.
%    \begin{macrocode}
\newunit{\kilogram}{kg}
\newunit{\metre}{m}
\newunit{\meter}{\metre}
\newunit{\mole}{mol}
\AtBeginDocument{\provideunit{\second}{s}}
\newunit{\ampere}{A}
\newunit{\kelvin}{K}
\newunit{\candela}{cd}
%    \end{macrocode}
%\end{macro}
%\end{macro}
%\end{macro}
%\end{macro}
%\end{macro}
%\end{macro}
%\end{macro}
%\end{macro}
%\begin{macro}{\Square}
%\changes{v1.1b}{2008/10/12}{Only defined if not otherwise available}
%\begin{macro}{\ssquare}
%\begin{macro}{\squared}
%\begin{macro}{\cubic}
%\begin{macro}{\cubed}
% Unlike multiples (which can be skipped if needed), the basic powers
% are also always defined.
%    \begin{macrocode}
\AtBeginDocument{
  \@ifundefined{Square}{%
    \newpower{\Square}{2}}{}}
\newpower{\ssquare}{2}
\newpower[post]{\squared}{2}
\newpower{\cubic}{3}
\newpower[post]{\cubed}{3}
%    \end{macrocode}
%\end{macro}
%\end{macro}
%\end{macro}
%\end{macro}
%\end{macro}
%\begin{macro}{\tothe}
%\begin{macro}{\raiseto}
% The user macros for arbitraty powers are simple
% calling macros to a common internal macro.
%    \begin{macrocode}
\newcommand*{\tothe}{\si@tothe{\tothe}}
\newcommand*{\raiseto}{\si@tothe{\raiseto}}
%    \end{macrocode}
%\end{macro}
%\end{macro}
%\begin{macro}{\si@tothe}
%\begin{macro}{\tothe@opt@si}
%\begin{macro}{\raiseto@opt@si}
%\darg{either \texttt{\cs{tothe}} or \texttt{\cs{raiseto}}}
%\darg{number}
% A macro for arbitrary powers, which comes after the unit and so
% needs to be marked as such.  Two fake option-storage macros are
% created so that everything works correctly.
%    \begin{macrocode}
\newcommand*{\si@tothe}[2]{%
  \ifsi@unt@littest
    \expandafter\@gobbletwo
  \else
    \ifsi@unt@litout
      \expandafter\expandafter\expandafter\si@unt@litpower
    \else
      \expandafter\expandafter\expandafter\si@unt@power
    \fi
  \fi
  {#1}{#2}}
\newcommand*{\tothe@opt@si}{post}
\newcommand*{\raiseto@opt@si}{}
%    \end{macrocode}
%\end{macro}
%\end{macro}
%\end{macro}
%
%\subsection{Locales}
%\begin{macro}{\si@loc@load}
%\begin{macro}{\si@loc@sisetup}
%\darg{locale}
% When loading a locale, the setup is saved rather than applied.
% Anything other than simple settings should be inside
% \cs{addtolocale}, which is already defined.
%    \begin{macrocode}
\newcommand*{\si@loc@load}[1]{%
  \let\si@loc@sisetup\sisetup
  \renewcommand*{\sisetup}[1]{%
    \expandafter\gdef\csname si@loc@#1\endcsname{##1}}%
  \si@loadfile{#1}%
  \let\sisetup\si@loc@sisetup}
%    \end{macrocode}
%\end{macro}
%\end{macro}
%\begin{macro}{\si@loc@set}
%\darg{locale}
% Setting the locale transfers the settings to \cs{sisetup}, and runs
% any extra macros.
%    \begin{macrocode}
\newcommand*{\si@loc@set}[1]{%
  \ifcsname si@loc@#1\endcsname
    \si@log@inf{Setting locale to `#1'}%
    \expandafter\expandafter\expandafter\expandafter
      \expandafter\expandafter\expandafter\si@temptoks
      \expandafter\expandafter\expandafter\expandafter
      \expandafter\expandafter\expandafter{%
        \expandafter\csname si@loc@#1\endcsname}%
    \expandafter\sisetup\expandafter{\the\si@temptoks}%
    \ifcsname si@loc@#1@extra\endcsname
      \csname si@loc@#1@extra\endcsname
    \fi
  \else
    \ifcsname si@loc@#1@extra\endcsname
      \si@log@inf{Setting locale to `#1'}%
      \csname si@loc@#1@extra\endcsname
    \else
      \si@log@warn{Unknown locale `#1'}%
    \fi
  \fi}
%    \end{macrocode}
%\end{macro}
%\begin{macro}{\si@loc@ltol}
%\changes{v1.0a}{2008/06/20}{\pkg{babel} support fixed for default
%  document language}
%\darg{list of locales and languages}
% The necessary loading for language modules occurs.
%    \begin{macrocode}
\newcommand*{\si@loc@ltol}[1]{%
  \def\si@tempa##1:##2\@nil{\si@loc@load{##1}}
  \@for\si@tempb:=#1\do{%
    \expandafter\si@tempa\si@tempb:\@nil}
  \AtBeginDocument{
    \@ifpackageloaded{babel}
      {\def\si@tempa##1:##2:##3\@nil{%
        \expandafter\addto\expandafter{%
          \csname extras##2\endcsname}%
          {\si@loc@set{##1}}}%
       \@for\si@tempb:=#1\do{%
         \expandafter\si@tempa\si@tempb::\@nil}%
       \expandafter\selectlanguage\expandafter{\languagename}}
      {\si@log@warn{babel not loaded \MessageBreak
        loctolang option ignored}}}}
\AtBeginDocument{
  \@ifpackageloaded{babel}
    {\renewcommand*{\si@loc@ltol}[1]{%
       \def\si@tempa##1:##2\@nil{\si@loc@load{##1}}%
       \@for\si@tempb:=#1\do{%
         \expandafter\si@tempa\si@tempb:\@nil}%
       \def\si@tempa##1:##2:##3\@nil{%
         \expandafter\addto\expandafter{%
           \csname extras##2\endcsname}%
           {\si@loc@set{##1}}}%
       \@for\si@tempb:=#1\do{%
         \expandafter\si@tempa\si@tempb::\@nil}}}
    {\renewcommand*{\si@loc@ltol}[1]{%
       \si@log@warn{babel not loaded \MessageBreak
         loctolang option ignored}}}}
%    \end{macrocode}
%\end{macro}
%\begin{macro}{\addtolocale}
%\darg{locale}
%\darg{commands}
% Arbitrary macros may need to be added to the locale.
%    \begin{macrocode}
\newcommand*{\addtolocale}[2]{%
  \si@addtocsname{si@loc@#1@extra}{#2}}
%    \end{macrocode}
%\end{macro}
%
%\subsection{Output routine}
%\begin{macro}{\si@out}
%\changes{v1.1}{2008/07/22}{Extended handling to subscripts}
%\darg{text}
% With all of the setup done, the text can finally be typeset.  This
% is done inside a \cs{text} block, which takes care of
% \cs{ensuremath}, \etc.  First of all, the various catcode settings
% needed to get maths-in-text mode are made.  \cs{makeatletter} is
% needed so that \cs{scantokens} still allows internal macros to
% work.
%    \begin{macrocode}
\begingroup
  \catcode`\^=\active
  \catcode`\-=\active
  \catcode`\_=\active
  \gdef\si@out#1{%
    \begingroup
      \catcode`\^=\active
      \catcode`\_=\active
      \makeatletter
      \endlinechar\m@ne
%    \end{macrocode}
%\changes{v1.0f}{2008/07/17}{Fixed bug with colour and spacing}
% The various font families can now be set up.  First a check is made
% in case there are nested calls to \cs{si@out@text}.
%    \begin{macrocode}
      \ifsi@fam@set\else
        \expandafter\si@fam@set
      \fi
      \text{%
        \si@colourcmd{\si@colour}%
        \si@fam@italic\si@fam@bold\si@fam@text
%    \end{macrocode}
%  The correct mode is selected, and the input is handed off for
%  typesetting.
%    \begin{macrocode}
        \ifsi@textmode
          \expandafter\si@out@text
        \else
          \expandafter\si@out@maths
        \fi
        {\scantokens{#1}}}%
    \endgroup
    \check@mathfonts}
%    \end{macrocode}
%\end{macro}
%\begin{macro}{\si@out@text}
%\begin{macro}{\si@out@maths}
%\darg{text}
% Output occurs with the correct changes to superscript behaviour.
%    \begin{macrocode}
  \gdef\si@out@text#1{%
    \let^\si@out@sp
    \let_\si@out@sb
    \let\textsuperscript\si@out@sp
    \let\textsubscript\si@out@sb
    \catcode`\-=\active\relax
    \let-\si@out@minus
    #1}
  \gdef\si@out@maths#1{%
    \let^\sp
    \let_\sb
    \let\textsuperscript\sp
    \let\textsubscript\sb
    $\si@fam@maths{#1}$}
\endgroup
%    \end{macrocode}
%\end{macro}
%\end{macro}
%\begin{macro}{\si@out@sp}
%\begin{macro}{\si@out@sb}
%\darg{text}
% \cs{textsuperscript} gives slightly different alignment of numbers
% to using |^| in text mode.  To avoid this, a slightly different
% definition is used.  Elsewhere \cs{textsuperscript} is used, as the
% code above sorts out the text/maths mode issues. The same is done
% for subscripts.
%    \begin{macrocode}
\newcommand*{\si@out@sp}[1]{\ensuremath{^{\text{#1}}}}
\newcommand*{\si@out@sb}[1]{\ensuremath{_{\text{#1}}}}
%    \end{macrocode}
%\end{macro}
%\end{macro}
%\begin{macro}{\si@out@minus}
%\changes{v1.1j}{2008/12/22}{Fixed infinite loop with \pkg{breqn}}
% An active minus sign is needed.
%    \begin{macrocode}
\newcommand*{\si@out@minus}{}
\DeclareTextSymbolDefault{\si@out@minus}{TS1}
\DeclareTextSymbol{\si@out@minus}{TS1}{61}
%    \end{macrocode}
%\end{macro}
%\begin{macro}{\ifsi@out@num}
% A flag is needed to control output settings.  This will be
% false unless inside \cs{si@out@num}.
%    \begin{macrocode}
\newif\ifsi@out@num
%    \end{macrocode}
%\end{macro}
%\begin{macro}{\si@out@num}
%\darg{number}
% For numerical output, the default fonts are controlled slightly
% differently to text output.
%    \begin{macrocode}
\newcommand*{\si@out@num}[1]{%
  \begingroup
    \si@out@numtrue
    \si@out{#1}%
  \endgroup}
%    \end{macrocode}
%\end{macro}
%
%\subsection{Finalisation}
%\begin{macro}{\si@extension}
%\begin{macro}{\si@fileprefix}
% To keep the code easy to maintain, the reusable filename components
% are macros rather than literals.
%    \begin{macrocode}
\newcommand*{\si@extension}{cfg}
\newcommand*{\si@fileprefix}{si-}
%    \end{macrocode}
%\end{macro}
%\end{macro}
%\begin{macro}{\si@ifl@aded}
%\begin{macro}{\si@ifloaded}
%\darg{package}
% A bit of borrowing from the \LaTeX\ kernel.  A copy of
% \cs{@ifl@aded} is needed as things aren't always done in the
% preamble by \currpkg.
%    \begin{macrocode}
\newcommand*{\si@ifl@aded}{}
\let\si@ifl@aded\@ifl@aded
\newcommand*{\si@ifloaded}[1]{%
  \si@ifl@aded\si@extension{\si@fileprefix#1}}
%    \end{macrocode}
%\end{macro}
%\end{macro}
%\begin{macro}{\si@loadfile}
%\darg{file}
% Loading configuration files is handled here.
%    \begin{macrocode}
\newcommand*{\si@loadfile}[1]{%
  \si@ifloaded{#1}{}
    {\si@InputIfFileExists{\si@fileprefix#1.\si@extension}
      {}
      {\si@log@err{Failed to load file
         \si@fileprefix#1.\si@extension}
         {The configuration file requested could not be
          found}}}}
%    \end{macrocode}
%\end{macro}
%\begin{macro}{\ifsi@outerinput}
% At various points, the package needs to know if the
% currently-read configuration file is the
% outer one.
%    \begin{macrocode}
\newif\ifsi@outerinput\si@outerinputtrue
%    \end{macrocode}
%\end{macro}
%\begin{macro}{\si@InputIfFileExists}
% To allow reading configuration files after the start of the
% document, a private copy of \cs{InputIfFileExists} is needed.  The
% main macro here simply checks if the modified version is needed.
%    \begin{macrocode}
\let\si@InputIfFileExists\InputIfFileExists
\AtBeginDocument{
  \renewcommand*{\si@InputIfFileExists}{%
    \ifsi@outerinput
      \expandafter\si@outerinput
    \else
      \expandafter\InputIfFileExists
    \fi}}
%    \end{macrocode}
%\end{macro}
%\begin{macro}{\si@outerinput}
%\darg{file}
%\darg{true-code}
%\darg{false-code}
% The outer call to load a file needs to set |@| to a letter and turn
% of printing using \cs{nullfont}.
%    \begin{macrocode}
\newcommand*{\si@outerinput}[3]{%
  \makeatletter
  \nullfont
  \si@outerinputfalse
  \IfFileExists{#1}%
    {#2\@addtofilelist{#1}\@@input \@filef@und\normalsize}
    {#3}%
  \normalsize
  \makeatother}
%    \end{macrocode}
%\end{macro}
%\begin{macro}{\requiresiconfigs}
%\darg{config-file}
% The configuration files depend on each other.
%    \begin{macrocode}
\newcommand*{\requiresiconfigs}[1]{%
  \@for\si@tempb:=#1\do{\si@loadfile{\si@tempb}}}
%    \end{macrocode}
%\end{macro}
%\begin{macro}{\si@loademfile}
%\darg{file}
% For emulation files, an additional check is made.
%    \begin{macrocode}
\newcommand*{\si@loademfile}[1]{%
  \@ifpackageloaded{#1}
    {\si@log@err{Emulation clash for package `#1'}
      {You have asked for emulation of package
       `#1'\MessageBreak
       (perhaps by giving siunitx a back-compatibility
       option)\MessageBreak but the package is already
       loaded!}}
    {\si@loadfile{#1}}}
%    \end{macrocode}
%\end{macro}
%\begin{macro}{\si@emclash}
%\darg{package}
%\darg{package}
% A macro for emulation clashes.
%    \begin{macrocode}
\newcommand*{\si@emclash}[2]{%
  \si@log@err{Emulation clash: `#1' and `#2'}
    {You have asked for emulation of package `#1'\MessageBreak
     but have already loaded emulation of `#2'}}
%    \end{macrocode}
%\end{macro}
%\begin{macro}{\si@emulating}
%\darg{package}
%\darg{version}
% For packages that are emulated, the \LaTeX\ mechanism to prevent
% re-loading is used.  The list of packages to check at the start
% of the document also has to be altered.
%    \begin{macrocode}
\newcommand*{\si@emulating}[2]{%
  \@namedef{ver@#1.sty}{#2 siunitx emulation of #1}%
  \let\si@tempa\si@blockpkgs
  \renewcommand*{\si@blockpkgs}{}%
  \@for\si@tempb:=\si@tempa\do{%
    \renewcommand*{\si@tempa}{#1}%
    \ifx\si@tempa\si@tempb\else
      \lowercase{\edef\si@tempa{#1}}%
      \lowercase{\edef\si@tempc{\si@tempb}}%
      \ifx\si@tempa\si@tempc
        \@namedef{ver@\si@tempc.sty}{#2 siunitx emulation of
          #1}%
      \else
        \si@addtolist{\si@blockpkgs}{\si@tempb}%
      \fi
    \fi}%
  \let\si@tempa\si@checkpkgs
  \renewcommand*{\si@checkpkgs}{}%
  \renewcommand*{\si@tempb}{#1}%
  \@for\si@tempc:=\si@tempa\do{%
    \ifx\si@tempb\si@tempc\else
      \si@addtolist{\si@checkpkgs}{\si@tempc}%
    \fi}}
%    \end{macrocode}
%\end{macro}
% With the \currpkg kernel macros defined, the package can now run
% through finalisation.  First, the default key values are set. The
% user options are then processed.
%    \begin{macrocode}
\sisetup{
  addsign=none,
  allowzeroexp=false,
  angformat=unchanged,
  astroang=false,%(
  closeerr=),%(
  closefrac=),
  colour=black,
  colourall=false,
  colourneg=false,
  decimalsymbol=fullstop,
  detectdisplay=true,
  digitsep=thin,
  dp=3,
  eVcorra=0.3ex,
  eVcorrb=0ex,
  errspace=none,
  fixdp=false,
  inlinebold=text,
  load=default,
  mathsrm=mathrm,
  mathssf=mathsf,
  mathstt=mathtt,
  mode=maths,
  negcolour=red,
  noload={},
  numaddn={},%(
  numcloseerr=),
  numdecimal={.,},
  numdigits=0123456789,
  numdiv=/,
  numexp=eEdD,
  numgobble={},
  numopenerr=(,%)
  numprod=x,
  numsign=+-\pm\mp,
  obeybold=false,
  obeyitalic=false,
  obeymode=false,
  openerr=(,%)
  openfrac=(,%)
  padangle=small,
  padnumber=lead,
  per=reciprocal,
  prefixbase=ten,
  prefixproduct=times,
  prefixsymbolic=true,
  prespace=false,
  redefsymbols=true,
  repeatunits=true,
  retainplus=false,
  seperr=false,
  sepfour=false,
  sign=plus,
  slash=slash,
  stickyper=false,
  strictarc=true,
  tabalignexp=true,
  tabautofit=false,
  tabformat=3.2,
  tabnumalign=centredecimal,
  tabparseonly=false,
  tabtextalign=centre,
  tabunitalign=left,
  textrm=rmfamily,
  textsf=sffamily,
  texttt=ttfamily,
  tightpm=false,
  trapambigerr=true,
  trapambigfrac=true,
  unitsep=thin,
  valuesep=thin,
  xspace=false}
\ProcessOptionsX[si]<key>
%    \end{macrocode}
% A check is now made so that emulation takes place one file at a
% time, and that each file is loaded only once.
%    \begin{macrocode}
\ifx\@empty\si@emulate\@empty\else
  \@for\si@tempa:=\si@emulate\do{%
    \si@loademfile{\si@tempa}}
\fi
%    \end{macrocode}
%\begin{macro}{\si@expanddefault}
% For turning the list of default choices into a loadable list.
%    \begin{macrocode}
\newcommand*{\si@expanddefault}[2]{%
  \expandafter\ifx\expandafter\@empty\csname si@#1\endcsname
    \@empty
  \else
    \renewcommand*{\si@tempb}{default}%
    \renewcommand*{\si@tempc}{}%
    \expandafter\@for\expandafter\si@tempa\expandafter
      :\expandafter=\csname si@#1\endcsname\do{%
      \ifx\si@tempa\si@tempb
        \si@addtolist{\si@tempc}{#2}%
      \else
        \si@addtolist{\si@tempc}{\si@tempa}%
      \fi}
    \expandafter\edef\csname si@#1\endcsname{\si@tempc}%
    \expandafter\si@addtolist\expandafter{%
      \csname si@no#1\endcsname}%
      {default}%
    \renewcommand*{\si@tempc}{}%
    \expandafter\@for\expandafter\si@tempa\expandafter
      :\expandafter=\csname si@#1\endcsname\do{%
      \si@switchfalse
      \expandafter\@for\expandafter\si@tempb\expandafter
        :\expandafter=\csname si@no#1\endcsname\do{%
        \ifx\si@tempa\si@tempb
          \si@switchtrue
        \fi
        \ifsi@switch\else
          \si@addtolist{\si@tempc}{\si@tempa}%
        \fi}}
    \@for\si@tempa:=\si@tempc\do{%
      \si@loadfile{\si@tempa}}%
  \fi}
%    \end{macrocode}
%\end{macro}
% The configuration and abbreviation files are loaded.
%    \begin{macrocode}
\si@expanddefault{load}{prefix,named,addn,prefixed,accepted,%
  physical,abbr}
%    \end{macrocode}
% The very last job is to load a local configuration file, if one
% exists, and restore catcodes.
%    \begin{macrocode}
\IfFileExists{siunitx.cfg}
  {\si@log@inf{Local configuration file found}%
   \InputIfFileExists{siunitx.cfg}{}{}}
  {}
\si@catcodes
%    \end{macrocode}
%
%\iffalse
%</package>
%\fi
%
%\section{Loadable modules}
% To keep the package relatively clear, and to make maintenance
% easier, the only units defined in the package itself are the base
% units. Everything else is a loadable module (similar to the
% approach in \unitsdef).
%
%\iffalse
%<*prefix>
%\fi
%\subsection{Multiple prefixes}
%\begin{macro}{\yocto}
%\begin{macro}{\zepto}
%\begin{macro}{\atto}
%\begin{macro}{\femto}
%\begin{macro}{\pico}
%\begin{macro}{\nano}
%\begin{macro}{\micro}
%\begin{macro}{\Micro}
%\begin{macro}{\milli}
%\begin{macro}{\centi}
%\begin{macro}{\deci}
% The various SI multiple prefixes are defined here.  First the small
% powers.
%    \begin{macrocode}
\ProvidesFile{si-prefix.cfg}
  [2008/12/22 v1.1j siunitx: SI Multiple prefixes]
\newprefix{\yocto}{-24}{y}
\newprefix{\zepto}{-21}{z}
\newprefix{\atto}{-18}{a}
\newprefix{\femto}{-15}{f}
\newprefix{\pico}{-12}{p}
\newprefix{\nano}{-9}{n}
%    \end{macrocode}
% Some testing is needed for \unitsdef compatibility.
%    \begin{macrocode}
\ifsi@old@OHM
  \newprefix{\Micro}{-6}{\si@sym@mu}
\else
  \ifsi@gensymb\else
    \newprefix{\micro}{-6}{\si@sym@mu}
  \fi
\fi
\newprefix{\milli}{-3}{m}
\newprefix{\centi}{-2}{c}
\newprefix{\deci}{-1}{d}
%    \end{macrocode}
%\end{macro}
%\end{macro}
%\end{macro}
%\end{macro}
%\end{macro}
%\end{macro}
%\end{macro}
%\end{macro}
%\end{macro}
%\end{macro}
%\end{macro}
%\begin{macro}{\deca}
%\begin{macro}{\hecto}
%\begin{macro}{\kilo}
%\begin{macro}{\mega}
%\begin{macro}{\giga}
%\begin{macro}{\tera}
%\begin{macro}{\peta}
%\begin{macro}{\exa}
%\begin{macro}{\zetta}
%\begin{macro}{\yotta}
% Now the large ones.
%    \begin{macrocode}
\newprefix{\deca}{1}{da}
\newprefix{\hecto}{2}{h}
\newprefix{\kilo}{3}{k}
\newprefix{\mega}{6}{M}
\newprefix{\giga}{9}{G}
\newprefix{\tera}{12}{T}
\newprefix{\peta}{15}{P}
\newprefix{\exa}{18}{E}
\newprefix{\zetta}{21}{Z}
\newprefix{\yotta}{24}{Y}
%    \end{macrocode}
%\end{macro}
%\end{macro}
%\end{macro}
%\end{macro}
%\end{macro}
%\end{macro}
%\end{macro}
%\end{macro}
%\end{macro}
%\end{macro}
%\begin{macro}{\deka}
% Apparently, ``deka'' is common in the US for deca.
%    \begin{macrocode}
\newprefix{\deka}{1}{da}
%    \end{macrocode}
%\end{macro}
%\begin{macro}{\gram}
%\begin{macro}{\kilogram}
% As the base unit of mass is the kilogram, rather than the gram, a
% bit of extra work is needed; by default the package only defines
% \cs{kilogram}, but with the prefixes available, this is altered to
% be \cs{kilo}\cs{gram}.  For that, the \cs{gram} must be defined first.
%    \begin{macrocode}
\newunit{\gram}{g}
\renewunit{\kilogram}{\kilo\gram}
%    \end{macrocode}
%\end{macro}
%\end{macro}
%\iffalse
%</prefix>
%<*named>
%\fi
%\subsection{Derived units with specific names}
%\begin{macro}{\becquerel}
%\begin{macro}{\coulomb}
%\begin{macro}{\farad}
%\begin{macro}{\Gray}
%\begin{macro}{\ggray}
%\begin{macro}{\hertz}
%\begin{macro}{\henry}
%\begin{macro}{\joule}
%\begin{macro}{\katal}
%\begin{macro}{\lumen}
%\begin{macro}{\lux}
%\begin{macro}{\newton}
% Derived units with specific names and symbols are defined.  Litre
% is an awkward one, but here the UK standard is used.
%    \begin{macrocode}
\ProvidesFile{si-named.cfg}
  [2008/12/22 v1.1j siunitx: SI Named units]
\newunit{\becquerel}{Bq}
\newunit{\coulomb}{C}
\newunit{\farad}{F}
\newunit{\Gray}{Gy}
\newunit{\ggray}{Gy}
\newunit{\hertz}{Hz}
\newunit{\henry}{H}
\newunit{\joule}{J}
\newunit{\katal}{kat}
\newunit{\lumen}{lm}
\newunit{\lux}{lx}
\newunit{\newton}{N}
%    \end{macrocode}
%\end{macro}
%\end{macro}
%\end{macro}
%\end{macro}
%\end{macro}
%\end{macro}
%\end{macro}
%\end{macro}
%\end{macro}
%\end{macro}
%\end{macro}
%\end{macro}
%\begin{macro}{\ohm}
%\begin{macro}{\Ohm}
%\begin{macro}{\pascal}
%\begin{macro}{\siemens}
%\begin{macro}{\sievert}
%\begin{macro}{\tesla}
%\begin{macro}{\volt}
%\begin{macro}{\watt}
%\begin{macro}{\weber}
% Some testing is needed for \unitsdef compatibility.
%    \begin{macrocode}
\ifsi@old@OHM
  \newunit{\Ohm}{\si@sym@Omega}
\else
  \ifsi@gensymb\else
%    \end{macrocode}
% To be on the safe side, use \cs{provideunit}.
%    \begin{macrocode}
    \provideunit{\ohm}{\si@sym@Omega}
  \fi
\fi
\newunit{\pascal}{Pa}
\newunit{\siemens}{S}
\newunit{\sievert}{Sv}
\newunit{\tesla}{T}
\newunit{\volt}{V}
\newunit{\watt}{W}
\newunit{\weber}{Wb}
%    \end{macrocode}
%\end{macro}
%\end{macro}
%\end{macro}
%\end{macro}
%\end{macro}
%\end{macro}
%\end{macro}
%\end{macro}
%\end{macro}
%\begin{macro}{\celsius}
%\begin{macro}{\Celsius}
% The degree celsius is a named unit.
%    \begin{macrocode}
\ifsi@old@OHM
  \newunit{\Celsius}{\si@sym@celsius}
\else
  \ifsi@gensymb\else
    \newunit{\celsius}{\si@sym@celsius}
  \fi
\fi
%    \end{macrocode}
%\end{macro}
%\end{macro}
%\begin{macro}{\radian}
%\begin{macro}{\steradian}
% The radian and steradian are officially derived units.
%    \begin{macrocode}
\newunit{\radian}{rad}
\newunit{\steradian}{sr}
%    \end{macrocode}
%\end{macro}
%\end{macro}
%\iffalse
%</named>
%<*prefixed>
%\fi
%\subsection{Units with prefixes}
% As in \unitsdef, some commonly used prefixed units are set up.
% This requires \file{si-prefix.cfg} and \file{si-named.cfg}.
%    \begin{macrocode}
\ProvidesFile{si-prefixed.cfg}
  [2008/12/22 v1.1j siunitx: SI Prefixed units]
\requiresiconfigs{prefix,named,accepted,physical}
%    \end{macrocode}
%\begin{macro}{\picometre}
%\begin{macro}{\nanometre}
%\begin{macro}{\micrometre}
%\begin{macro}{\millimetre}
%\begin{macro}{\centimetre}
%\begin{macro}{\decimetre}
%\begin{macro}{\kilometre}
% Extra distances.
%    \begin{macrocode}
\newunit{\picometre}{\pico\metre}
\newunit{\nanometre}{\nano\metre}
\newunit{\micrometre}{\micro\metre}
\newunit{\millimetre}{\milli\metre}
\newunit{\centimetre}{\centi\metre}
\newunit{\decimetre}{\deci\metre}
\newunit{\kilometre}{\kilo\metre}
%    \end{macrocode}
%\end{macro}
%\end{macro}
%\end{macro}
%\end{macro}
%\end{macro}
%\end{macro}
%\end{macro}
%\begin{macro}{\femtogram}
%\begin{macro}{\picogram}
%\begin{macro}{\nanogram}
%\begin{macro}{\microgram}
%\begin{macro}{\milligram}
% Extra masses.
%    \begin{macrocode}
\newunit{\femtogram}{\femto\gram}
\newunit{\picogram}{\pico\gram}
\newunit{\nanogram}{\nano\gram}
\newunit{\microgram}{\micro\gram}
\newunit{\milligram}{\milli\gram}
%    \end{macrocode}
%\end{macro}
%\end{macro}
%\end{macro}
%\end{macro}
%\end{macro}
%\begin{macro}{\femtomole}
%\begin{macro}{\picomole}
%\begin{macro}{\nanomole}
%\begin{macro}{\micromole}
%\begin{macro}{\millimole}
%\begin{macro}{\kilomole}
%\changes{v1.1e}{2008/11/18}{New unit}
% Now some moles.
%    \begin{macrocode}
\newunit{\femtomole}{\femto\mole}
\newunit{\picomole}{\pico\mole}
\newunit{\nanomole}{\nano\mole}
\newunit{\micromole}{\micro\mole}
\newunit{\millimole}{\milli\mole}
\newunit{\kilomole}{\kilo\mole}
%    \end{macrocode}
%\end{macro}
%\end{macro}
%\end{macro}
%\end{macro}
%\end{macro}
%\end{macro}
%\begin{macro}{\attosecond}
%\begin{macro}{\femtosecond}
%\begin{macro}{\picosecond}
%\begin{macro}{\nanosecond}
%\begin{macro}{\microsecond}
%\begin{macro}{\millisecond}
% Prefixed seconds.
%    \begin{macrocode}
\newunit{\attosecond}{\atto\second}
\newunit{\femtosecond}{\femto\second}
\newunit{\picosecond}{\pico\second}
\newunit{\nanosecond}{\nano\second}
\newunit{\microsecond}{\micro\second}
\newunit{\millisecond}{\milli\second}
%    \end{macrocode}
%\end{macro}
%\end{macro}
%\end{macro}
%\end{macro}
%\end{macro}
%\end{macro}
%\begin{macro}{\picoampere}
%\begin{macro}{\nanoampere}
%\begin{macro}{\microampere}
%\begin{macro}{\milliampere}
%\begin{macro}{\kiloampere}
% The last prefixed base units are amperes.
%    \begin{macrocode}
\newunit{\picoampere}{\pico\ampere}
\newunit{\nanoampere}{\nano\ampere}
\newunit{\microampere}{\micro\ampere}
\newunit{\milliampere}{\milli\ampere}
\newunit{\kiloampere}{\kilo\ampere}
%    \end{macrocode}
%\end{macro}
%\end{macro}
%\end{macro}
%\end{macro}
%\end{macro}
%\begin{macro}{\millivolt}
%\begin{macro}{\kilovolt}
%\begin{macro}{\milliwatt}
%\begin{macro}{\kilowatt}
%\begin{macro}{\megawatt}
%\begin{macro}{\femtofarad}
%\begin{macro}{\picofarad}
%\begin{macro}{\nanofarad}
%\begin{macro}{\microfarad}
%\begin{macro}{\millifarad}
%\begin{macro}{\millisiemens}
% More electricity-related units.
%    \begin{macrocode}
\newunit{\millivolt}{\milli\volt}
\newunit{\kilovolt}{\kilo\volt}
\newunit{\milliwatt}{\milli\watt}
\newunit{\kilowatt}{\kilo\watt}
\newunit{\megawatt}{\mega\watt}
\newunit{\femtofarad}{\femto\farad}
\newunit{\picofarad}{\pico\farad}
\newunit{\nanofarad}{\nano\farad}
\newunit{\microfarad}{\micro\farad}
\newunit{\millifarad}{\milli\farad}
\newunit{\millisiemens}{\milli\siemens}
%    \end{macrocode}
%\end{macro}
%\end{macro}
%\end{macro}
%\end{macro}
%\end{macro}
%\end{macro}
%\end{macro}
%\end{macro}
%\end{macro}
%\end{macro}
%\end{macro}
%\begin{macro}{\kilohm}
%\changes{v1.1f}{2008/11/25}{Renamed from \cs{kiloohm}}
%\begin{macro}{\megohm}
%\changes{v1.1f}{2008/11/25}{Renamed from \cs{megaohm}}
%\begin{macro}{\gigaohm}
% For resistance, checks are needed again for the definition of
% \cs{ohm}.
%    \begin{macrocode}
\ifsi@old@OHM
  \newunit{\kilohm}{\kilo\Ohm}
  \newunit{\megohm}{\mega\Ohm}
  \newunit{\gigaohm}{\giga\Ohm}
\else
  \ifsi@gensymb\else
  \newunit{\kilohm}{\kilo\ohm}
  \newunit{\megohm}{\mega\ohm}
  \newunit{\gigaohm}{\giga\ohm}
  \fi
\fi
%    \end{macrocode}
%\end{macro}
%\end{macro}
%\end{macro}
%\begin{macro}{\microlitre}
%\begin{macro}{\millilitre}
%\begin{macro}{\microliter}
%\begin{macro}{\milliliter}
%\begin{macro}{\cubicmetre}
%\begin{macro}{\cubiccentimetre}
%\begin{macro}{\centimetrecubed}
%\begin{macro}{\cubicmicrometre}
%\begin{macro}{\cubicmillimetre}
%\begin{macro}{\cubicdecimetre}
% Volumes (unlike \unitsdef, with litre and metre spelled
% correctly).  \cs{millilitre} and \cs{microlitre} are defined
% as they are the two officially-sanctioned prefixes for the litre.
%    \begin{macrocode}
\newunit{\microlitre}{\micro\litre}
\newunit{\millilitre}{\milli\litre}
\newunit{\microliter}{\micro\liter}
\newunit{\milliliter}{\milli\liter}
\newunit{\cubicmetre}{\metre\cubed}
\newunit{\cubiccentimetre}{\centi\metre\cubed}
\newunit{\centimetrecubed}{\centi\metre\cubed}
\newunit{\cubicmicrometre}{\micro\metre\cubed}
\newunit{\cubicmillimetre}{\milli\metre\cubed}
\newunit{\cubicdecimetre}{\cubic\deci\metre}
%    \end{macrocode}
%\end{macro}
%\end{macro}
%\end{macro}
%\end{macro}
%\end{macro}
%\end{macro}
%\end{macro}
%\end{macro}
%\end{macro}
%\end{macro}
%\begin{macro}{\squaremetre}
%\begin{macro}{\squarecentimetre}
%\begin{macro}{\centimetresquared}
%\begin{macro}{\squarekilometre}
% Areas, with metre spelled corrected; \cs{are} and \cs{hectare} are
% in the ``temporarily accepted'' file.
%    \begin{macrocode}
\newunit{\squaremetre}{\Square\metre}
\newunit{\squarecentimetre}{\Square\centi\metre}
\newunit{\centimetresquared}{\centi\metre\squared}
\newunit{\squarekilometre}{\Square\kilo\metre}
%    \end{macrocode}
%\end{macro}
%\end{macro}
%\end{macro}
%\end{macro}
%\begin{macro}{\millijoule}
%\begin{macro}{\kilojoule}
%\begin{macro}{\megajoule}
%\begin{macro}{\millielectronvolt}
%\begin{macro}{\kiloelectronvolt}
%\begin{macro}{\megaelectronvolt}
%\begin{macro}{\gigaelectronvolt}
%\begin{macro}{\teraelectronvolt}
%\begin{macro}{\kilowatthour}
% Some energy is needed by now!  Notice the definition of
% \cs{kilowatthour}
%    \begin{macrocode}
\newunit{\millijoule}{\milli\joule}
\newunit{\kilojoule}{\kilo\joule}
\newunit{\megajoule}{\mega\joule}
\newunit{\millielectronvolt}{\milli\electronvolt}
\newunit{\kiloelectronvolt}{\kilo\electronvolt}
\newunit{\megaelectronvolt}{\mega\electronvolt}
\newunit{\gigaelectronvolt}{\giga\electronvolt}
\newunit{\teraelectronvolt}{\tera\electronvolt}
\newunit[unitsep=none]{\kilowatthour}{\kilo\watt\hour}
%    \end{macrocode}
%\end{macro}
%\end{macro}
%\end{macro}
%\end{macro}
%\end{macro}
%\end{macro}
%\end{macro}
%\end{macro}
%\end{macro}
%\begin{macro}{\millihertz}
%\begin{macro}{\kilohertz}
%\begin{macro}{\megahertz}
%\begin{macro}{\gigahertz}
%\begin{macro}{\terahertz}
% Frequencies.
%    \begin{macrocode}
\newunit{\millihertz}{\milli\hertz}
\newunit{\kilohertz}{\kilo\hertz}
\newunit{\megahertz}{\mega\hertz}
\newunit{\gigahertz}{\giga\hertz}
\newunit{\terahertz}{\tera\hertz}
%    \end{macrocode}
%\end{macro}
%\end{macro}
%\end{macro}
%\end{macro}
%\end{macro}
%\begin{macro}{\millinewton}
%\begin{macro}{\kilonewton}
%\begin{macro}{\hectopascal}
%\begin{macro}{\megabecquerel}
%\begin{macro}{\millisievert}
% A few more from various areas.
%    \begin{macrocode}
\newunit{\millinewton}{\milli\newton}
\newunit{\kilonewton}{\kilo\newton}
\newunit{\hectopascal}{\hecto\pascal}
\newunit{\megabecquerel}{\mega\becquerel}
\newunit{\millisievert}{\milli\sievert}
%    \end{macrocode}
%\end{macro}
%\end{macro}
%\end{macro}
%\end{macro}
%\end{macro}
%\iffalse
%</prefixed>
%<*abbr>
%\fi
%\subsection{Abbreviated units}
%\begin{macro}{\pA}
%\begin{macro}{\nA}
%\begin{macro}{\micA}
%\begin{macro}{\mA}
%\begin{macro}{\kA}
% The abbreviated units are sorted in one file.  To allow
% back-compatibility with \unitsdef, each one is inside an \cs{if}
% block for the appropriate option.  First currents.
%    \begin{macrocode}
\ProvidesFile{si-abbr.cfg}
  [2008/12/22 v1.1j siunitx: Abbreviated units]
\requiresiconfigs{prefix,named,accepted,physical}
\newunit{\pA}{\pico\ampere}
\newunit{\nA}{\nano\ampere}
\newunit{\micA}{\micro\ampere}
\newunit{\mA}{\milli\ampere}
\newunit{\kA}{\kilo\ampere}
%    \end{macrocode}
%\end{macro}
%\end{macro}
%\end{macro}
%\end{macro}
%\end{macro}
%\begin{macro}{\Hz}
%\begin{macro}{\mHz}
%\begin{macro}{\kHz}
%\begin{macro}{\MHz}
%\begin{macro}{\GHz}
%\begin{macro}{\THz}
% Then frequencies.
%    \begin{macrocode}
\newunit{\Hz}{\hertz}
\newunit{\mHz}{\milli\hertz}
\newunit{\kHz}{\kilo\hertz}
\newunit{\MHz}{\mega\hertz}
\newunit{\GHz}{\giga\hertz}
\newunit{\THz}{\tera\hertz}
%    \end{macrocode}
%\end{macro}
%\end{macro}
%\end{macro}
%\end{macro}
%\end{macro}
%\end{macro}
%\begin{macro}{\fmol}
%\begin{macro}{\pmol}
%\begin{macro}{\nmol}
%\begin{macro}{\micmol}
%\begin{macro}{\mmol}
%\begin{macro}{\kmol}
%\changes{v1.1e}{2008/11/18}{New unit}
% Amounts of substance.
%    \begin{macrocode}
\newunit{\fmol}{\femto\mole}
\newunit{\pmol}{\pico\mole}
\newunit{\nmol}{\nano\mole}
\newunit{\micmol}{\micro\mole}
\newunit{\mmol}{\milli\mole}
\newunit{\kmol}{\kilo\mole}
%    \end{macrocode}
%\end{macro}
%\end{macro}
%\end{macro}
%\end{macro}
%\end{macro}
%\end{macro}
%\begin{macro}{\kV}
%\begin{macro}{\mV}
% Potentials.
%    \begin{macrocode}
\newunit{\kV}{\kilo\volt}
\newunit{\mV}{\milli\volt}
%    \end{macrocode}
%\end{macro}
%\end{macro}
%\begin{macro}{\ml}
%\begin{macro}{\micl}
%\begin{macro}{\mL}
%\begin{macro}{\micL}
%\begin{macro}{\cmc}
%\begin{macro}{\dmc}
%\begin{macro}{\cms}
% Volumes and areas
%    \begin{macrocode}
\provideunit{\ml}{\milli\litre}
\provideunit{\micl}{\micro\litre}
\provideunit{\mL}{\milli\liter}
\provideunit{\micL}{\micro\liter}
\newunit{\cmc}{\centi\metre\cubed}
\newunit{\dmc}{\deci\metre\cubed}
\newunit{\cms}{\centi\metre\squared}
%    \end{macrocode}
%\end{macro}
%\end{macro}
%\end{macro}
%\end{macro}
%\end{macro}
%\end{macro}
%\end{macro}
%\begin{macro}{\fg}
%\begin{macro}{\SIfg}
%\begin{macro}{\kg}
%\begin{macro}{\fg}
%\begin{macro}{\pg}
%\begin{macro}{\nanog}
%\begin{macro}{\micg}
%\begin{macro}{\mg}
%\begin{macro}{\amu}
% Masses.
%    \begin{macrocode}
\newunit{\kg}{\kilo\gram}
%    \end{macrocode}
% There is a name clash with \pkg{babel} here in French; hopefully
% there will not be too many complaints.
%    \begin{macrocode}
\AtBeginDocument{\provideunit{\fg}{\femto\gram}}
\newunit{\SIfg}{\femto\gram}
\newunit{\pg}{\pico\gram}
\newunit{\nanog}{\nano\gram}
\newunit{\micg}{\micro\gram}
\newunit{\mg}{\milli\gram}
\newunit{\amu}{\atomicmass}
%    \end{macrocode}
%\end{macro}
%\end{macro}
%\end{macro}
%\end{macro}
%\end{macro}
%\end{macro}
%\end{macro}
%\end{macro}
%\end{macro}
%\begin{macro}{\kJ}
%\begin{macro}{\eV}
%\begin{macro}{\meV}
%\begin{macro}{\keV}
%\begin{macro}{\MeV}
%\begin{macro}{\GeV}
%\begin{macro}{\TeV}
%\begin{macro}{\kWh}
% Energies.
%    \begin{macrocode}
\newunit{\kJ}{\kilo\joule}
\newunit{\eV}{\electronvolt}
\newunit{\meV}{\milli\electronvolt}
\newunit{\keV}{\kilo\electronvolt}
\newunit{\MeV}{\mega\electronvolt}
\newunit{\GeV}{\giga\electronvolt}
\newunit{\TeV}{\tera\electronvolt}
\newunit[unitsep=none]{\kWh}{\kilo\watt\hour}
%    \end{macrocode}
%\end{macro}
%\end{macro}
%\end{macro}
%\end{macro}
%\end{macro}
%\end{macro}
%\end{macro}
%\end{macro}
%\begin{macro}{\picom}
%\begin{macro}{\nm}
%\begin{macro}{\micm}
%\begin{macro}{\mm}
%\begin{macro}{\cm}
%\begin{macro}{\dm}
%\begin{macro}{\km}
% Lengths.
%    \begin{macrocode}
\newunit{\picom}{\pico\metre}
\newunit{\nm}{\nano\metre}
\newunit{\micm}{\micro\metre}
\newunit{\mm}{\milli\metre}
\newunit{\cm}{\centi\metre}
\newunit{\dm}{\deci\metre}
\newunit{\km}{\kilo\metre}
%    \end{macrocode}
%\end{macro}
%\end{macro}
%\end{macro}
%\end{macro}
%\end{macro}
%\end{macro}
%\end{macro}
%\begin{macro}{\Sec}
%\begin{macro}{\as}
%\begin{macro}{\fs}
%\begin{macro}{\ps}
%\begin{macro}{\ns}
%\begin{macro}{\mics}
%\begin{macro}{\ms}
% Finally, times.
%    \begin{macrocode}
\newunit{\Sec}{\second}
\newunit{\as}{\atto\second}
\newunit{\fs}{\femto\second}
%    \end{macrocode}
% The letter class (and others) define \cs{ps} for postscripts, so
% \cs{provideunit} is best here.
%    \begin{macrocode}
\provideunit{\ps}{\pico\second}
\newunit{\ns}{\nano\second}
\newunit{\mics}{\micro\second}
\newunit{\ms}{\milli\second}
%    \end{macrocode}
%\end{macro}
%\end{macro}
%\end{macro}
%\end{macro}
%\end{macro}
%\end{macro}
%\end{macro}
%\iffalse
%</abbr>
%<*addn>
%\fi
%\subsection{Additional (temporary) SI units}
%\begin{macro}{\angstrom}
%\begin{macro}{\are}
%\begin{macro}{\hectare}
%\begin{macro}{\barn}
%\begin{macro}{\BAR}
%\begin{macro}{\bbar}
%\begin{macro}{\millibar}
%\begin{macro}{\gal}
%\begin{macro}{\curie}
%\begin{macro}{\roentgen}
%\begin{macro}{\rad}
%\begin{macro}{\rem}
%\changes{v1.1c}{2008/10/22}{Uses \cs{provideunit}}
% Some units are ``temporarily'' acceptable for use in the SI system.
% These are defined here, although some are in very general use.
%    \begin{macrocode}
\ProvidesFile{si-addn.cfg}
  [2008/12/22 v1.1j siunitx: SI Additional units]
\newunit{\angstrom}{\si@sym@ringA}
\newunit{\are}{a}
\newunit{\hectare}{\hecto\are}
\newunit{\barn}{b}
\newunit{\BAR}{bar}
\newunit{\bbar}{bar}
\newunit{\millibar}{\milli\BAR}
\newunit{\gal}{Gal}
\newunit{\curie}{Ci}
\newunit{\roentgen}{R}
\newunit{\rad}{rad}
\provideunit{\rem}{rem}
%    \end{macrocode}
%\end{macro}
%\end{macro}
%\end{macro}
%\end{macro}
%\end{macro}
%\end{macro}
%\end{macro}
%\end{macro}
%\end{macro}
%\end{macro}
%\end{macro}
%\end{macro}
%\iffalse
%</addn>
%<*accepted>
%\fi
%\subsection{Units accepted for use with SI}
% The units which are accepted but do not fit in the above, plus
% \cs{percent} which seems to fit into this category.
%\begin{macro}{\minute}
%\begin{macro}{\hour}
%\begin{macro}{\Day}
%\begin{macro}{\dday}
%\begin{macro}{\degree}
%\changes{v1.1c}{2008/10/22}{Uses \cs{provideunit}}
%\begin{macro}{\Degree}
%\begin{macro}{\arcmin}
%\begin{macro}{\arcsec}
%\begin{macro}{\litre}
%\begin{macro}{\liter}
%\begin{macro}{\tonne}
%\begin{macro}{\neper}
%\begin{macro}{\bel}
%\begin{macro}{\percent}
%    \begin{macrocode}
\ProvidesFile{si-accepted.cfg}
  [2008/12/22 v1.1j siunitx: SI Accepted units]
\newunit{\minute}{min}
\newunit{\hour}{h}
\newunit{\Day}{d}
\newunit{\dday}{d}
\ifsi@old@OHM
  \AtBeginDocument{
    \provideunit[valuesep=none]{\Degree}{\si@sym@degree}
  }
\else
  \ifsi@gensymb\else
    \AtBeginDocument{
      \provideunit[valuesep=none]{\degree}{\si@sym@degree}
    }
  \fi
\fi
\newunit[valuesep=none]{\arcmin}{\si@sym@minute}
\newunit[valuesep=none]{\arcsec}{\si@sym@second}
\newunit{\litre}{l}
\newunit{\liter}{L}
\newunit{\tonne}{t}
\newunit{\neper}{Np}
\newunit{\bel}{B}
\newunit{\percent}{\%}
%    \end{macrocode}
%\end{macro}
%\end{macro}
%\end{macro}
%\end{macro}
%\end{macro}
%\end{macro}
%\end{macro}
%\end{macro}
%\end{macro}
%\end{macro}
%\end{macro}
%\end{macro}
%\end{macro}
%\end{macro}
%\iffalse
%</accepted>
%<*physical>
%\fi
%\subsection{Units based on physical measurements}
%\begin{macro}{\si@eVspacea}
%\begin{macro}{\si@eVspaceb}
%\begin{macro}{\electronvolt}
%\begin{macro}{\atomicmassunit}
%\begin{macro}{\atomicmass}
% A few units based on physical measurements exist.  For \cs{eV}, the need
% for a negative kern does make things a bit complicated.
%    \begin{macrocode}
\ProvidesFile{si-physical.cfg}
  [2008/12/22 v1.1j siunitx: SI Physically-measured units]
\newcommand*{\si@eVspacea}{\text{\kern-\si@eVcorra}}%
\newcommand*{\si@eVspaceb}{\text{\kern-\si@eVcorrb}}%
\newunit{\electronvolt}{e\protect\si@eVspacea V\protect%
  \si@eVspaceb}
\newunit{\atomicmass}{u}
\newunit{\atomicmassunit}{u}
%    \end{macrocode}
%\end{macro}
%\end{macro}
%\end{macro}
%\end{macro}
%\end{macro}
%\iffalse
%</physical>
%\fi
%
%\section{Additional configurations}
% To provide flexibility for people in specific areas, specialised
% units can be set up.  These are then stored separately to ease use.
%
%\iffalse
%<*synchem>
%\fi
%\subsection{Synthetic chemistry}
%\begin{macro}{\mmHg}
%\begin{macro}{\molar}
%\begin{macro}{\Molar}
%\begin{macro}{\torr}
%\begin{macro}{\dalton}
% Some useful units for synthetic chemists; although \cs{mmHg} and
% \cs{Molar} are outside of the SI system, they are used a lot.
% These are set up using \cs{provideunit} as people may have their
% own definitions.
%    \begin{macrocode}
\ProvidesFile{si-synchem.cfg}
  [2008/12/22 v1.1j siunitx: Units for synthetic chemists]
\requiresiconfigs{prefix}
\newunit{\mmHg}{mmHg}
\newunit{\molar}{\mole\per\cubic\deci\metre}
\newunit{\Molar}{\textsc{m}}
\newunit{\torr}{Torr}
\newunit{\dalton}{Da}
%    \end{macrocode}
%\end{macro}
%\end{macro}
%\end{macro}
%\end{macro}
%\end{macro}
%\iffalse
%</synchem>
%<*hep>
%\fi
%\subsection{High-energy physics}
% Some units inspired by \pkg{hepunits}.
%    \begin{macrocode}
\ProvidesFile{si-hep.cfg}
  [2008/12/22 v1.1j siunitx: Units for high-energy physics]
\requiresiconfigs{prefix,named}
%    \end{macrocode}
%\begin{macro}{\micron}
%\begin{macro}{\mrad}
%\begin{macro}{\gauss}
% These specific to high-energy physics, but are not defined
% elsewhere in \currpkg.\footnote{It is not clear from
% \pkg{hepunits}, but the assumption is that \cs{mrad} represents
% millirad and not milliradian.}
%    \begin{macrocode}
\provideunit{\micron}{\micro\metre}
\newunit{\mrad}{\milli\rad}
\newunit{\gauss}{G}
%    \end{macrocode}
%\end{macro}
%\end{macro}
%\end{macro}
%\begin{macro}{\clight}
%\begin{macro}{\eVperc}
% The speed of light is used in units for the area, although of
% course it is not strictly a unit.  However, this makes it easier to
% use in other units.
%    \begin{macrocode}
\newunit{\clight}{\ensuremath{\mathnormal{c}}}
\newunit{\eVperc}{\eV\per\clight}
%    \end{macrocode}
%\end{macro}
%\end{macro}
%\begin{macro}{\nanobarn}
%\begin{macro}{\picobarn}
%\begin{macro}{\femtobarn}
%\begin{macro}{\attobarn}
%\begin{macro}{\zeptobarn}
%\begin{macro}{\yoctobarn}
% Various prefixed barns.
%    \begin{macrocode}
\newunit{\nanobarn}{\nano\barn}
\newunit{\picobarn}{\pico\barn}
\newunit{\femtobarn}{\femto\barn}
\newunit{\attobarn}{\atto\barn}
\newunit{\zeptobarn}{\zepto\barn}
\newunit{\yoctobarn}{\yocto\barn}
%    \end{macrocode}
%\end{macro}
%\end{macro}
%\end{macro}
%\end{macro}
%\end{macro}
%\end{macro}
%\begin{macro}{\nb}
%\begin{macro}{\pb}
%\begin{macro}{\fb}
%\begin{macro}{\ab}
%\begin{macro}{\zb}
%\begin{macro}{\yb}
% Abbreviations for the same, but using \cs{provideunit} to avoid any
% clashes.
%    \begin{macrocode}
\provideunit{\nb}{\nano\barn}
\provideunit{\pb}{\pico\barn}
\provideunit{\fb}{\femto\barn}
\provideunit{\ab}{\atto\barn}
\provideunit{\zb}{\zepto\barn}
\provideunit{\yb}{\yocto\barn}
%    \end{macrocode}
%\end{macro}
%\end{macro}
%\end{macro}
%\end{macro}
%\end{macro}
%\end{macro}
%\iffalse
%</hep>
%<*astro>
%\fi
%\subsection{Astronomy}
%\begin{macro}{\parsec}
%\begin{macro}{\lightyear}
% For astronomy, the \cs{parsec} unit is needed.
%    \begin{macrocode}
\ProvidesFile{si-astro.cfg}
  [2008/12/22 v1.1j siunitx: Units for astronomy]
\newunit{\parsec}{pc}
\newunit{\lightyear}{ly}
%    \end{macrocode}
%\end{macro}
%\end{macro}
%\iffalse
%</astro>
%<*geophys>
%\fi
%\subsection{Geophysics}
%\begin{macro}{\lightyear}
% The unit \cs{gon} is useful here.
%    \begin{macrocode}
\ProvidesFile{si-geophys.cfg}
  [2008/12/22 v1.1j siunitx: Units for astronomy]
\newunit{\gon}{gon}
%    \end{macrocode}
%\end{macro}
%\iffalse
%</geophys>
%<*binary>
%\fi
%\subsection{Binary units}
%\begin{macro}{\kibi}
%\begin{macro}{\mebi}
%\begin{macro}{\gibi}
%\begin{macro}{\tebi}
%\begin{macro}{\pebi}
%\begin{macro}{\exbi}
%\begin{macro}{\zebi}
%\begin{macro}{\yobi}
% The binary units, as specified by the IEC and made available by
% \SIunits. First, the binary prefixes.
%    \begin{macrocode}
\ProvidesFile{si-binary.cfg}
  [2008/12/22 v1.1j siunitx: Binary units]
\newprefix[binary]{\kibi}{10}{Ki}
\newprefix[binary]{\mebi}{20}{Mi}
\newprefix[binary]{\gibi}{30}{Gi}
\newprefix[binary]{\tebi}{40}{Ti}
\newprefix[binary]{\pebi}{50}{Pi}
\newprefix[binary]{\exbi}{60}{Ei}
\newprefix[binary]{\zebi}{70}{Zi}
\newprefix[binary]{\yobi}{80}{Yi}
%    \end{macrocode}
%\end{macro}
%\end{macro}
%\end{macro}
%\end{macro}
%\end{macro}
%\end{macro}
%\end{macro}
%\end{macro}
%\begin{macro}{\bit}
%\begin{macro}{\byte}
% Now the units.
%    \begin{macrocode}
\newunit{\bit}{bit}
\newunit{\byte}{B}
%    \end{macrocode}
%\end{macro}
%\end{macro}
%
%\iffalse
%</binary>
%\fi
%
%\section{Loadable locales}
% Some short files to provide the correct settings for various places.
%
%\iffalse
%<*UK>
%\fi
%\subsection{United Kingdom}
% This is identical to the USA locale, and contains the package
% default values.\footnote{The package author is in the UK.}
%    \begin{macrocode}
\ProvidesFile{si-UK.cfg}
  [2008/12/22 v1.1j siunitx: UK locale]
\sisetup{
  unitsep=thin,
  expproduct=times,
  valuesep=thin,
  decimalsymbol=fullstop,
  digitsep=thin,
  sepfour=false}
%    \end{macrocode}
%\iffalse
%</UK>
%<*USA>
%\fi
%\subsection{United States}
% The same as for the UK.
%    \begin{macrocode}
\ProvidesFile{si-USA.cfg}
  [2008/12/22 v1.1j siunitx: USA locale]
\sisetup{
  unitsep=thin,
  expproduct=times,
  valuesep=thin,
  decimalsymbol=fullstop,
  digitsep=thin,
  sepfour=false}
%    \end{macrocode}
%\iffalse
%</USA>
%<*DE>
%\fi
%\subsection{Germany}
% Germany, hopefully.
%    \begin{macrocode}
\ProvidesFile{si-DE.cfg}
  [2008/12/22 v1.1j siunitx: Germany locale]
\sisetup{
  unitsep=cdot,
  valuesep=thin,
  decimalsymbol=comma,
  expproduct=cdot,
  digitsep=thin,
  sepfour=false}
%    \end{macrocode}
%\iffalse
%</DE>
%<*ZA>
%\fi
%\subsection{South Africa}
% Design decisions taken from the same section of \SIstyle.
%    \begin{macrocode}
\ProvidesFile{si-ZA.cfg}
  [2008/12/22 v1.1j siunitx: South Africa locale]
\sisetup{
  unitsep=cdot,
  valuesep=thin,
  expproduct=times,
  decimalsymbol=comma,
  digitsep=thin,
  sepfour=false}
%    \end{macrocode}
%\iffalse
%</ZA>
%\fi
%
%\section{Emulation code}
% Each emulation mode loads an appropriate definition file.  This
% then alters the package defaults, and defines new macros provided
% by the emulated package.
%
%\iffalse
%<*units>
%\fi
%\subsection{\units}
% The very first thing to do here is a reload check, as things
% could go wrong with \unitsdef emulation.
%    \begin{macrocode}
\si@ifloaded{units}{\endinput}{}
%    \end{macrocode}
% The \units package is quite easy to emulate, as it only has a
% few options and macros. There is also no error checking in
% \units for conflicting options, so users probably expect none.
%    \begin{macrocode}
\ProvidesFile{si-units.cfg}
  [2008/12/22 v1.1j siunitx: Emulation of units]
\si@emulating{units}{1998/08/04 v0.9b}
\si@ifloaded{SIunits}
  {\si@emclash{units}{SIunits}\endinput}{}
\si@ifloaded{sistyle}
  {\si@emclash{units}{sistyle}\endinput}{}
%    \end{macrocode}
% To emulate \units, \cs{per} must give fractions.
%    \begin{macrocode}
\sisetup{per=fraction,fraction=nice,obeybold,inlinebold=maths,
  ,obeymode}
\ifsi@old@tight
  \sisetup{valuesep=thin}
\fi
\ifsi@old@loose
  \sisetup{valuesep=space}
\fi
\ifsi@old@ugly
  \sisetup{fraction=ugly}
\fi
%    \end{macrocode}
%\begin{macro}{\unit}
%\doarg{number}
%\darg{unit}
% The \units version of \cs{unit} is similar to \cs{SI}.  Here and
% in \cs{unitfrac} the \cs{num} macro is used; thus the number given
% really has to be a number.  However, if users are using \currpkg
% rather than \units they should expect more checking of input.  As
% the \units package uses the current mode, this has to be
% detected.
%    \begin{macrocode}
\si@newrobustcmd*{\unit}[2][]{%
  \ifmmode
    \SI{#1}{#2}%
  \else
    \SI[obeyfamily,obeyitalic]{#1}{#2}%
  \fi}
%    \end{macrocode}
%\end{macro}
%\begin{macro}{\unitfrac}
%\doarg{number}
%\darg{numerator}
%\darg{denominator}
% \cs{unitfrac} is a bit more of a hack.
%    \begin{macrocode}
\si@newrobustcmd*{\unitfrac}[3][]{%
  \begingroup
    \si@fam@mode%
    \ifmmode\else
      \sisetup{obeyfamily,obeyitalic}%
    \fi
    \si@ifnotmtarg{#1}
      {\num{#1}\ensuremath{\si@valuesep}}%
    \si@frac{#2}{#3}
  \endgroup}
%    \end{macrocode}
%\end{macro}
%\iffalse
%</units>
%\fi
%\iffalse
%<*unitsdef>
%\fi
%\subsection{\unitsdef}
% The package begins with the usual identification of what is
% happening.  Although \file{si-units.cfg} makes the same checks,
% the error will make more sense if it comes here, in the event of a
% clash.
%    \begin{macrocode}
\ProvidesFile{si-unitsdef.cfg}
  [2008/12/22 v1.1j siunitx:
   Emulation of unitsdef]
\si@emulating{unitsdef}{2005/01/04 v0.2}
\si@ifloaded{SIunits}
  {\si@emclash{unitsdef}{SIunits}\endinput}{}
\si@ifloaded{sistyle}
  {\si@emclash{unitsdef}{sistyle}\endinput}{}
%    \end{macrocode}
% Emulation of \units is needed for \unitsdef to work.
%    \begin{macrocode}
\requiresiconfigs{units}
%    \end{macrocode}
% The \unitsdef package loads some packages that \currpkg does not.
% In particular, it loads \pkg{textcomp} and \pkg{fontenc}.  This
% could be important for output, and so the same is done here.
%    \begin{macrocode}
\RequirePackage{textcomp}
\RequirePackage[T1]{fontenc}
%    \end{macrocode}
% The multitude of package options for \unitsdef need to be handled.
%    \begin{macrocode}
\sisetup{mode=text,allowoptarg,prespace}
\ifsi@old@noxspace
  \sisetup{xspace=false}
\fi
%    \end{macrocode}
% The various options for loading unit abbreviations have to be
% handled. Here, any request to avoid abbreviations prevents any
% loading.
%    \begin{macrocode}
\ifsi@old@noabbr
  \sisetup{noload=abbr}
\fi
\ifsi@old@nofrequncyabbr
  \sisetup{noload=abbr}
\fi
\ifsi@old@nomolabbr
  \sisetup{noload=abbr}
\fi
\ifsi@old@novoltageabbr
  \sisetup{noload=abbr}
\fi
\ifsi@old@novolumeabbr
  \sisetup{noload=abbr}
\fi
\ifsi@old@noweightabbr
  \sisetup{noload=abbr}
\fi
\ifsi@old@noenergyabbr
  \sisetup{noload=abbr}
\fi
\ifsi@old@nolengthabbr
  \sisetup{noload=abbr}
\fi
\ifsi@old@notimeabbr
  \sisetup{noload=abbr}
\fi
%    \end{macrocode}
%\begin{macro}{\unitvaluesep}
% To emulate the \cs{unitvaluesep} macro, a hack is needed of the original
% \pkg{xkeyval} macro for \opt{valuesep}, as well of course as a definition
% of the macro itself.
%    \begin{macrocode}
\newcommand*{\unitvaluesep}{\,}
\renewcommand*{\si@valuesep}{\text{\unitvaluesep}}
\define@choicekey*+[si]{key}{valuesep}[\si@tempa]
  {space,thin,med,medium,thick,none}
  {\renewcommand*\unitvaluesep\@nameuse{si@fix@##1}%
   \si@log@debug{Option valuesep set to ##1}}
  {\si@log@debug{Option valuesep set to ##1}%
   \renewcommand*\unitvaluesep{##1}}
%    \end{macrocode}
%\end{macro}
%\begin{macro}{\unitsignonly}
%\begin{macro}{\ilu}
%\begin{macro}{\arc}
% Some rather straight-forward definitions, with just a bit of fun to
% get the spacing correct.
%    \begin{macrocode}
\let\unitsignonly\si
\si@newrobustcmd*{\ilu}[2][]{%
  \begingroup
    #1\unitvaluesep%
    \unit{#2}%
  \endgroup}
\let\arc\ang
%    \end{macrocode}
%\end{macro}
%\end{macro}
%\end{macro}
%\begin{macro}{\unitSIdef}
%\begin{macro}{\si@unitSIdef}
% The \unitsdef package uses a different approach to setting the font
% inside its version of \cs{SI}.  The problem is the same as for
% \cs{unitvaluesep}, but with the added problem that \currpkg uses
% \cs{csname} \ldots \cs{endcsname}.
%    \begin{macrocode}
\newcommand*{\unitSIdef}{\upshape}
\newcommand*{\si@unitSIdef}{\unitSIdef\selectfont}
\sisetup{textrm=si@unitSIdef}
%    \end{macrocode}
%\end{macro}
%\end{macro}
%\begin{macro}{\per}
% Rather awkwardly, \unitsdef uses \cs{per} in a different way to
% \currpkg.
%    \begin{macrocode}
\let\per\relax
\si@newrobustcmd*{\per}[2]{%
  \begingroup
    \si@xspacefalse
    \renewcommand*{\unitvaluesep}{}%
    \unitfrac{#1}{#2}%
  \endgroup}
%    \end{macrocode}
%\end{macro}
%\begin{macro}{\unittimes}
%\begin{macro}{\unitsep}
%\begin{macro}{\unitsuperscript}
% Some pretty straight-forward stuff again; notice that the automatic
% analyser for units has to be turned off for this to work.
%    \begin{macrocode}
\newcommand*{\unittimes}{\ensuremath{\cdot}}
\newcommand*{\unitsep}{\,}
\renewcommand*{\si@unt@unithook}{\si@unt@litouttrue}
\sisetup{unitsep=none}
\newcommand*{\unitsuperscript}{\tothe}
%    \end{macrocode}
%\end{macro}
%\end{macro}
%\end{macro}
%\begin{macro}{\newnosepunit}
%\begin{macro}{\renewnosepunit}
% Simple aliases.
%    \begin{macrocode}
\newcommand*{\newnosepunit}{\newunit[valuesep=none]}
\newcommand*{\renewnosepunit}{\renewunit[valuesep=none]}
%    \end{macrocode}
%\end{macro}
%\end{macro}
%\begin{macro}{\setTextOmega}
%\begin{macro}{\setMathOmega}
%\begin{macro}{\setTextmu}
%\begin{macro}{\setMathmu}
%\begin{macro}{\setTextCelsius}
%\begin{macro}{\setMathCelsius}
%\begin{macro}{\setMathDegree}
%\begin{macro}{\setTextDegree}
% Controlling symbols is a simple translation job; as only one
% setting is used by \currpkg in text mode, a bit of extra work is
% needed.
%    \begin{macrocode}
\newcommand*{\setTextOmega}[2]{%
  \renewcommand*{\si@textOmega}{%
      \begingroup
      \edef\si@tempa{\sfdefault}%
      \ifx\f@family\si@tempa
        \expandafter#2%
      \else
        \expandafter#1%
      \fi
    \endgroup}}
\newcommand*{\setMathOmega}[1]{\sisetup{mathsOmega=#1}}
\newcommand*{\setTextmu}[2]{%
  \renewcommand*{\si@textmu}{%
    \begingroup
      \edef\si@tempa{\sfdefault}%
      \ifx\f@family\si@tempa
        \expandafter#2%
      \else
        \expandafter#1%
      \fi
    \endgroup}}
\newcommand*{\setMathmu}[1]{\sisetup{mathsmu=#1}}
\newcommand*{\setTextCelsius}[2]{%
  \renewcommand*{\si@textcelsius}{%
    \begingroup
      \edef\si@tempa{\sfdefault}%
      \ifx\f@family\si@tempa
        \expandafter#2%
      \else
        \expandafter#1%
      \fi
    \endgroup}}
\newcommand*{\setMathCelsius}[1]{\sisetup{mathscelsius=#1}}
\newcommand*{\setMathDegree}[2]{%
  \renewcommand*{\si@textdegree}{%
    \begingroup%
      \edef\si@tempa{\sfdefault}%
      \ifx\f@family\si@tempa
        \expandafter#2%
      \else
        \expandafter#1%
      \fi
    \endgroup}}
\newcommand*{\setTextDegree}[1]{\sisetup{textdegree=#1}}
%    \end{macrocode}
%\end{macro}
%\end{macro}
%\end{macro}
%\end{macro}
%\end{macro}
%\end{macro}
%\end{macro}
%\end{macro}
% The \opt{ohm} and \opt{OHM} options are checked, and some sanity is
% ensured.  This needs to happen before loading the configuration
% files.
%    \begin{macrocode}
\ifsi@old@OHM
  \ifsi@old@ohm
    \si@log@inf{Both `ohm' and `OHM' options given\MessageBreak
      Using default behaviour for unitsdef}
    \expandafter\expandafter\expandafter\si@old@OHMfalse
  \fi
\fi
%    \end{macrocode}
%\begin{macro}{\liter}
%\begin{macro}{\ton}
%\begin{macro}{\days}
% Tonne is spelled as ``ton'' by \unitsdef, which is wrong in the UK
% at least (\SI{1}{ton} $=$ \SI{40}{cwt} $=$ \SI{2240}{lb}!).
%    \begin{macrocode}
\ifsi@old@liter
  \ifsi@old@LITER
    \si@log@inf{Both `liter' and `LITER' options
      given\MessageBreak Using default behaviour for unitsdef}
  \else
    \renewunit{\liter}{l}
  \fi
\fi
\newunit{\ton}{t}
\newunit{\days}{d}
%    \end{macrocode}
%\end{macro}
%\end{macro}
%\end{macro}
%\begin{macro}{\picometer}
%\begin{macro}{\nanometer}
%\begin{macro}{\micrometer}
%\begin{macro}{\millimeter}
%\begin{macro}{\centimeter}
%\begin{macro}{\decimeter}
%\begin{macro}{\kilometer}
% Extra distances.
%    \begin{macrocode}
\newunit{\picometer}{\pico\meter}
\newunit{\nanometer}{\nano\meter}
\newunit{\micrometer}{\micro\meter}
\newunit{\millimeter}{\milli\meter}
\newunit{\centimeter}{\centi\meter}
\newunit{\decimeter}{\deci\meter}
\newunit{\kilometer}{\kilo\meter}
%    \end{macrocode}
%\end{macro}
%\end{macro}
%\end{macro}
%\end{macro}
%\end{macro}
%\end{macro}
%\end{macro}
%\begin{macro}{\femtoliter}
%\begin{macro}{\picoliter}
%\begin{macro}{\nanoliter}
%\begin{macro}{\microliter}
%\begin{macro}{\milliliter}
%\begin{macro}{\centiliter}
%\begin{macro}{\deciliter}
%\begin{macro}{\hectoliter}
%\begin{macro}{\cubicmeter}
%\begin{macro}{\cubicmicrometer}
%\begin{macro}{\cubicmillimeter}
% Volumes with US spellings.
%    \begin{macrocode}
\newunit{\femtoliter}{\femto\liter}
\newunit{\picoliter}{\pico\liter}
\newunit{\nanoliter}{\nano\liter}
\newunit{\microliter}{\micro\liter}
\newunit{\milliliter}{\milli\liter}
\newunit{\centiliter}{\centi\liter}
\newunit{\deciliter}{\deci\liter}
\newunit{\hectoliter}{\hecto\liter}
\newunit{\cubicmeter}{\meter\cubed}
\newunit{\cubicmicrometer}{\micro\meter\cubed}
\newunit{\cubicmillimeter}{\milli\meter\cubed}
%    \end{macrocode}
%\end{macro}
%\end{macro}
%\end{macro}
%\end{macro}
%\end{macro}
%\end{macro}
%\end{macro}
%\end{macro}
%\end{macro}
%\end{macro}
%\end{macro}
%\begin{macro}{\squaremeter}
%\begin{macro}{\squarecentimeter}
%\begin{macro}{\squarekilometer}
%\begin{macro}{\ar}
%\begin{macro}{\hectar}
% Areas, including the mis-spellings for \cs{are} and
% \cs{hectare}.
%    \begin{macrocode}
\newunit{\squaremeter}{\Square\meter}
\newunit{\squarecentimeter}{\Square\centi\meter}
\newunit{\squarekilometer}{\Square\kilo\meter}
\newunit{\ar}{a}
\newunit{\hectar}{\hecto\ar}
%    \end{macrocode}
%\end{macro}
%\end{macro}
%\end{macro}
%\end{macro}
%\end{macro}
%\begin{macro}{\kv}
%\begin{macro}{\mv}
% The code for \unitsdef has the capitalisation wrong for \cs{kV} and
% \cs{mV}.
%    \begin{macrocode}
\ifsi@old@noabbr
\else
  \ifsi@old@novoltageabbr\else
    \newunit{\kv}{\kilo\volt}
    \newunit{\mv}{\milli\volt}
  \fi
\fi
%    \end{macrocode}
%\end{macro}
%\end{macro}
%\begin{macro}{\sek}
%\begin{macro}{\fg}
%\begin{macro}{\fl}
%\begin{macro}{\pl}
%\begin{macro}{\nl}
%\begin{macro}{\micl}
%\begin{macro}{\ml}
%\begin{macro}{\cl}
%\begin{macro}{\dl}
%\begin{macro}{\hl}
% There are some slightly different abbreviations, plus some which
% are not officially allowed.
%    \begin{macrocode}
\ifsi@old@noabbr\else
  \ifsi@old@notimeabbr\else
    \newunit{\sek}{\second}
  \fi
  \ifsi@old@noweightabbr\else
    \newunit{\fg}{\femto\gram}
  \fi
  \ifsi@old@novolumeabbr\else
    \newunit{\fl}{\femto\liter}
    \newunit{\pl}{\pico\liter}
    \newunit{\nl}{\nano\liter}
    \newunit{\micl}{\micro\liter}
    \newunit{\ml}{\milli\liter}
    \newunit{\cl}{\centi\liter}
    \newunit{\dl}{\deci\liter}
    \newunit{\hl}{\hecto\liter}
  \fi
\fi
%    \end{macrocode}
%\end{macro}
%\end{macro}
%\end{macro}
%\end{macro}
%\end{macro}
%\end{macro}
%\end{macro}
%\end{macro}
%\end{macro}
%\end{macro}
%\begin{macro}{\calory}
%\begin{macro}{\kilocalory}
% \unitsdef spells calorie incorrectly, and it is also not an SI unit.
%    \begin{macrocode}
\newunit{\calory}{cal}
\newunit{\kilocalory}{\kilo\calory}
%    \end{macrocode}
%\end{macro}
%\end{macro}
%\begin{macro}{\uBar}
% \unitsdef uses \cs{ubar} for bar.
%    \begin{macrocode}
\newunit{\uBar}{ba}
%    \end{macrocode}
%\end{macro}
%\begin{macro}{\gensymbohm}
%\begin{macro}{\gensymbcelsius}
%\begin{macro}{\gensymbmicro}
%\begin{macro}{\gensymbdegree}
% If the options relating to \pkg{gensymb} are given, then the
% package \emph{has} to be loaded.  The definitions are then renamed;
% a slight awkward feature is that the hyphen character needs to be a
% letter.  To avoid needing to worry about this again, a second
% switch is set up.
%    \begin{macrocode}
\catcode`\-=11\relax
\ifsi@old@redef-gensymb
  \expandafter\si@gensymbtrue
\fi
\catcode`\-=12\relax
\ifsi@gensymb
  \RequirePackage{gensymb}
  \AtBeginDocument{
    \let\gensymbohm\ohm
    \let\gensymbcelsius\celsius
    \let\gensymbmicro\micro
    \let\gensymbdegree\degree
    \let\ohm\@undefined
    \let\celsius\@undefined
    \let\micro\@undefined
    \let\degree\@undefined
    \ifsi@old@OHM\else
      \newunit{\ohm}{\si@sym@Omega}
      \newunit{\celsius}{\si@sym@celsius}
      \newprefix{\micro}{\si@sym@mu}{-6}
      \newunit{\degree}{\si@sym@degree}
    \fi}
\fi
%    \end{macrocode}
%\end{macro}
%\end{macro}
%\end{macro}
%\end{macro}
% The configuration files can now be loaded.
%    \begin{macrocode}
\requiresiconfigs{prefix,named,addn,accepted}
%    \end{macrocode}
% The \opt{noconfig} option could be ignored, but it costs little to
% let is be used.
%    \begin{macrocode}
\ifsi@old@noconfig\else
  \InputIfFileExists{unitsdef.cfg}
    {\si@log@inf{unitsdef config file loaded}}
    {\si@log@inf{unitsdef config file not found}}
\fi
%    \end{macrocode}
%
%\iffalse
%</unitsdef>
%<*sistyle>
%\fi
%\subsection{\SIstyle}
% After setting the necessary defaults, the emulation code defines
% the macros in \SIstyle as given in the manual for that package.
%    \begin{macrocode}
\ProvidesFile{si-sistyle.cfg}
  [2008/12/22 v1.1j siunitx: Emulation of
     SIstyle]
\si@emulating{sistyle}{2006/12/20 v2.3}
\sisetup{%
  sepfour=true,
  obeyfamily,
  obeyitalic=true,
  numsign=+-,
  strictarc=false,
  unitsep=cdot}
%    \end{macrocode}
%\begin{macro}{\SIobeyboldtrue}
%\begin{macro}{\SIobeyboldfalse}
% Some simple switches, but not using \cs{newif}.
%    \begin{macrocode}
\newcommand*{\SIobeyboldtrue}{\sisetup{obeybold=true}}
\newcommand*{\SIobeyboldfalse}{\sisetup{obeybold=false}}
%    \end{macrocode}
%\end{macro}
%\end{macro}
%\begin{macro}{\num}
%\begin{macro}{\si@sis@num}
%\begin{macro}{\si@sis@numstar}
% To get the correct behaviour for \cs{num}, some redefinitions are
% needed to handle to optional |*|.
%    \begin{macrocode}
\let\num\relax
\si@newrobustcmd*{\num}{%
  \@ifstar
    {\si@sis@numstar}
    {\si@sis@num}}
\newcommand*{\si@sis@num}[2][]{%
  \begingroup%
    \sisetup{#1}%
    \expandafter\si@out@num\expandafter{\si@num{#2}}%
  \endgroup}
\newcommand*{\si@sis@numstar}[2][]{%
  \begingroup%
    \sisetup{mode=text,obeybold}%
    \sisetup{#1}%
    \expandafter\si@out@num\expandafter{\si@num{#2}}%
  \endgroup}
%    \end{macrocode}
%\end{macro}
%\end{macro}
%\end{macro}
%\begin{macro}{\pnt}
% The \cs{pnt} macro is needed as |.| is active inside \cs{SI}.  The
% name is exactly the same as in \SIstyle, but the implementation is
% different.  This is not defined by the main package as there are
% better ways of including numbers in the output than this.
%    \begin{macrocode}
\newcommand*{\pnt}{\ensuremath{\si@decimalsymbol}}
%    \end{macrocode}
%\end{macro}
%\begin{macro}{\SIgroupfourtrue}
%\begin{macro}{\SIgroupfourfalse}
% Switches for grouping four characters.
%    \begin{macrocode}
\newcommand*{\SIgroupfourtrue}{\sisetup{sepfour=true}}
\newcommand*{\SIgroupfourfalse}{\sisetup{sepfour=false}}
%    \end{macrocode}
%\end{macro}
%\end{macro}
%\begin{macro}{\SIunitsep}
%\begin{macro}{\SIunitspace}
%\begin{macro}{\SIunitdot}
% Whatever is given here is passed through to \cs{sisetup}.
%    \begin{macrocode}
\newcommand*{\SIunitsep}[1]{\sisetup{valuesep={#1}}}
\newcommand*{\SIunitspace}[1]{\sisetup{unitspace={#1}}}
\newcommand*{\SIunitdot}[1]{\sisetup{unitsep={#1}}}
%    \end{macrocode}
%\end{macro}
%\end{macro}
%\end{macro}
%\begin{macro}{\SIdecimalsymbol}
%\begin{macro}{\SIthousandsep}
%\begin{macro}{\SIproductsign}
%\begin{macro}{\SIdecimalsign}
% The same is true here, with the appropriate translation.
%    \begin{macrocode}
\newcommand*{\SIdecimalsymbol}[1]{\sisetup{decimalsymbol={#1}}}
\newcommand*{\SIthousandsep}[1]{\sisetup{digitsep={#1}}}
\newcommand*{\SIproductsign}[1]{\sisetup{expproduct={#1}}}
\newcommand*{\SIdecimalsign}[1]{\sisetup{decimalsymbol={#1}}}
%    \end{macrocode}
%\end{macro}
%\end{macro}
%\end{macro}
%\end{macro}
%\begin{macro}{\si@sis@savefont}
%\darg{setting}
%\darg{argument}
% The font definitions need a bit of extra work doing.  As both
% settings here have |@| as a letter, all should be fine.
%    \begin{macrocode}
\newcommand*{\si@sis@savefont}[2]{%
  \@namedef{si@sis@#1}{#2}%
  \sisetup{#1=si@sis@#1}}
%    \end{macrocode}
%\end{macro}
%\begin{macro}{\SImathrm}
%\begin{macro}{\SImathsf}
%\begin{macro}{\SImathtt}
% The font control macros have to ensure that a macro name is passed
% to \cs{sisetup}.
%    \begin{macrocode}
\newcommand*{\SImathrm}[1]{\si@sis@savefont{mathrm}{#1}}
\newcommand*{\SImathsf}[1]{\si@sis@savefont{mathsf}{#1}}
\newcommand*{\SImathtt}[1]{\si@sis@savefont{mathtt}{#1}}
%    \end{macrocode}
%\end{macro}
%\end{macro}
%\end{macro}
%\begin{macro}{\SIdefaultMfam}
%\begin{macro}{\SIdefaultNfam}
%\begin{macro}{\SIdefaultTfam}
% The same for the default keys.
%    \begin{macrocode}
\newcommand*{\SIdefaultMfam}[1]{\si@sis@savefont{mathrm}{#1}}
\newcommand*{\SIdefaultNfam}[1]{\si@sis@savefont{mathnumrm}{#1}}
\newcommand*{\SIdefaultTfam}[1]{\si@sis@savefont{textrm}{#1}}
%    \end{macrocode}
%\end{macro}
%\end{macro}
%\end{macro}
%\begin{macro}{\ensureupmath}
% The \cs{ensureupmath} command guarantees processing by the
% font-matching system. The argument cannot be processed here, so
% care is needed.
%    \begin{macrocode}
\si@newrobustcmd*{\ensureupmath}[1]{%
  \begingroup
    \sisetup{mode=maths,obeyitalic=false}%
    \si@out{#1}%
  \endgroup}
%    \end{macrocode}
%\end{macro}
%\begin{macro}{\degC}
%\begin{macro}{\degF}
%\begin{macro}{\arcdeg}
% A few extra symbol names are needed.
%    \begin{macrocode}
\newcommand*{\degC}{\si@sym@celsius}
\newcommand*{\arcdeg}{\si@sym@degree}
\newcommand*{\degF}{\si@sym@degree F}
%    \end{macrocode}
%\end{macro}
%\end{macro}
%\end{macro}
%\begin{macro}{\AddToSIstyle}
%\begin{macro}{\SIstyle}
%\begin{macro}{\SIstyleToLang}
%\begin{macro}{\si@sis@addtolocale}
% Finally, the locale control.
%    \begin{macrocode}
\newcommand*{\SIstyle}[1]{\sisetup{locale=#1}}
\newcommand*{\SIstyleToLang}[2]{\sisetup{loctolang=#1:#2}}
\newcommand*{\AddToSIstyle}{%
  \si@switchfalse
  \@ifstar
    {\si@switchtrue
     \si@sis@addtolocale}
    {\si@sis@addtolocale}}
\newcommand*{\si@sis@addtolocale}[2]{%
  \ifsi@switch
    \expandafter\let\csname si@loc@#1@extra\endcsname\relax
  \fi
  \addtolocale{#1}{#2}}
%    \end{macrocode}
%\end{macro}
%\end{macro}
%\end{macro}
%\end{macro}
%
%\iffalse
%</sistyle>
%<*SIunits>
%\fi
%\subsection{\SIunits}
% \SIunits emulation starts in much the same way.
%    \begin{macrocode}
\ProvidesFile{si-SIunits.cfg}
  [2008/12/22 v1.1j siunitx: Emulation of
     SIunits]
\si@emulating{SIunits}{2007/12/02 v1.36}
\sisetup{
  unitsep=thick,
  valuesep=thick,
  prefixproduct=\si@valuesep,
  trapambigfrac=false,
  stickyper}
\requiresiconfigs{prefix,named,accepted,physical}
%    \end{macrocode}
%\begin{macro}{\reciprocal}
%\begin{macro}{\rp}
%\begin{macro}{\per}
%\begin{macro}{\usk}
%\begin{macro}{\power}
%\begin{macro}{\rpsquare}
%\begin{macro}{\rpcubic}
%\begin{macro}{\fourth}
%\begin{macro}{\rpfourth}
% A few very simple translations, using the internal version of
% \cs{per} to allow changes of output style.
%    \begin{macrocode}
\newcommand*{\reciprocal}{\sisetup{per=reciprocal}\si@per}
\let\rp\reciprocal
\renewcommand*{\per}{\sisetup{per=slash}\si@per}
\newcommand*{\usk}{}
\newcommand*{\power}[1]{#1\tothe}
\newcommand*{\rpsquare}{\sisetup{per=reciprocal}\si@per\Square}
\newcommand*{\rpcubic}{\sisetup{per=reciprocal}\si@per\cubic}
\newpower{\fourth}{4}
\newcommand*{\rpfourth}{\sisetup{per=reciprocal}\si@per\fourth}
%    \end{macrocode}
%\end{macro}
%\end{macro}
%\end{macro}
%\end{macro}
%\end{macro}
%\end{macro}
%\end{macro}
%\end{macro}
%\end{macro}
%\begin{macro}{\rpsquared}
%\begin{macro}{\rpcubed}
% Here, some low-level switch changing is needed.
%    \begin{macrocode}
\newcommand*{\rpsquared}{%
  \sisetup{per=reciprocal}\si@unt@pertrue\si@unt@perseentrue%
    \squared}
\newcommand*{\rpcubed}{%
  \sisetup{per=reciprocal}\si@unt@pertrue\cubed}
%    \end{macrocode}
%\end{macro}
%\end{macro}
%\begin{macro}{\SIsetup}
%\begin{macro}{\si@siu@setup}
% The various package spacing options are processed.  They also have
% to be correctly handled by the \cs{SIsetup} macro.
%    \begin{macrocode}
\newcommand*{\SIsetup}[1]{%
  \@for\si@tempa:=#1\do{%
    \@ifundefined{ifsi@old@#1}
      {\si@log@warn{Unknown SIunits option `#1'}}
      {\csname si@old@#1true\endcsname}}
  \si@siu@setup}
\newcommand*{\si@siu@setup}{%
  \ifsi@old@cdot
    \sisetup{unitsep=cdot}%
  \fi
  \ifsi@old@thickspace
    \sisetup{unitsep=thick}%
  \fi
  \ifsi@old@mediumspace
    \sisetup{unitsep=medium}%
  \fi
  \ifsi@old@thinspace
    \sisetup{unitsep=thin}%
  \fi
  \ifsi@old@thickqspace
    \sisetup{valuesep=thick}%
  \fi
  \ifsi@old@mediumqspace
    \sisetup{valuesep=medium}%
  \fi
  \ifsi@old@thinqspace
    \sisetup{valuesep=thin}%
  \fi}
\si@siu@setup
%    \end{macrocode}
%\end{macro}
%\end{macro}
%\begin{macro}{\square}
%\begin{macro}{\squaren}
% \SIunits does slightly different things about the clash with
% \cs{square}, and either redefines this macro or provides
% \cs{squaren}.
%    \begin{macrocode}
\ifsi@old@squaren
  \newpower{\squaren}{2}
\fi
\AtBeginDocument{%
  \@ifundefined{square}
    {\newpower{\square}{2}}
    {\ifsi@old@amssymb
       \renewpower{\square}{2}
     \else
       \ifsi@old@squaren\else
         \si@log@warn{\string\square\space already
           defined\MessageBreak SIunits mode may cause
           errors}%
       \fi
     \fi}}
%    \end{macrocode}
%\end{macro}
%\end{macro}
%\begin{macro}{\gray}
% The potential clash with \pkg{PStricks} is also handled
% differently; here, \cs{Gray} will already be defined by the
%  \currpkg kernel.
%    \begin{macrocode}
\AtBeginDocument{
  \@ifundefined{gray}
    {\newunit{\gray}{Gy}}
    {\ifsi@old@pstricks
       \renewunit{\gray}{Gy}
     \else
       \ifsi@old@Gray\else
         \si@log@warn{\string\gray\space already
           defined\MessageBreak SIunits mode may cause
           errors}%
       \fi
     \fi}}
%    \end{macrocode}
%\end{macro}
%\begin{macro}{\unit}
%\begin{macro}{\unita}
% The \cs{unit} macro is defined.
%    \begin{macrocode}
\ifsi@old@italian
  \let\unita\SI
\else
  \let\unit\SI
\fi
%    \end{macrocode}
%\end{macro}
%\end{macro}
% The miscellaneous options are moped up.
%    \begin{macrocode}
\ifsi@old@textstyle
  \sisetup{mode=text}
\fi
\ifsi@old@binary
  \sisetup{alsoload=binary}
\fi
\ifsi@old@noams
  \AtBeginDocument{%
    \renewcommand*{\si@textmu}{\ensuremath\si@mathsmu}}
\fi
%    \end{macrocode}
%\begin{macro}{\arcminute}
%\begin{macro}{\arcsecond}
%\begin{macro}{\rperminute}
%\begin{macro}{\ton}
%\begin{macro}{\degreecelsius}
% The unit macros defined by \SIunits that are not defined by
% \currpkg (by default).
%    \begin{macrocode}
\newunit[valuesep=none]{\arcminute}{\si@sym@minute}
\newunit[valuesep=none]{\arcsecond}{\si@sym@second}
\newunit{\rperminute}{r/min}
\newunit{\ton}{t}
\newunit{\degreecelsius}{\celsius}
%    \end{macrocode}
%\end{macro}
%\end{macro}
%\end{macro}
%\end{macro}
%\end{macro}
%\begin{macro}{\addunit}
% This is an alias for \cs{newunit}.
%    \begin{macrocode}
\let\addunit\newunit
%    \end{macrocode}
%\end{macro}
%\begin{macro}{\addprefix}
% A little more work for \cs{addprefix}.
%    \begin{macrocode}
\newcommand*{\addprefix}[2]{\newprefix{#1}{#2}}
%    \end{macrocode}
%\end{macro}
%\begin{macro}{\si@siu@newunit}
%\changes{v1.1a}{2008/09/30}{Corrected very minor bug in definition}
%\begin{macro}{\si@siu@power}
%\begin{macro}{\si@siu@newunithook}
%\doarg{power}
%\darg{numerator}
%\darg{denominator}
% To save some code, making new units which need a \texttt{\ldots np}
% variant is handled by a dedicated macro.
%    \begin{macrocode}
\newcommand*{\si@siu@newunit}[3][]{%
%    \end{macrocode}
% A test is needed to sort out \cs{square}.
%    \begin{macrocode}
  \renewcommand*{\si@tempa}{#1}%
  \renewcommand*{\si@tempb}{square}%
  \def\si@siu@power{}%
  \ifx\@empty\si@tempa\@empty\else
    \ifx\si@tempa\si@tempb
      \renewcommand*{\si@siu@power}{\ssquare}%
    \else
      \edef\si@siu@power{%
        \expandafter\noexpand\csname #1\endcsname}%
    \fi
  \fi
%    \end{macrocode}
% The necessary information is now stored in temporary macros.
%    \begin{macrocode}
  \edef\si@tempa{%
    \expandafter\noexpand\csname #2per#1#3\endcsname}%
  \edef\si@tempb{%
    \expandafter\noexpand\csname #2\endcsname\noexpand\per
    \expandafter\noexpand\si@siu@power
    \expandafter\noexpand\csname #3\endcsname}%
  \expandafter\expandafter\expandafter\newunit\expandafter%
    \expandafter\expandafter{\expandafter\si@tempa\expandafter}%
      \expandafter{\si@tempb}
  \edef\si@tempa{%
    \expandafter\noexpand\csname #2per#1#3np\endcsname}%
  \edef\si@tempb{%
    \expandafter\noexpand\csname #2\endcsname\noexpand
    \reciprocal\expandafter\noexpand\si@siu@power
    \expandafter\noexpand\csname #3\endcsname}%
  \expandafter\expandafter\expandafter\newunit\expandafter
    \expandafter\expandafter{\expandafter\si@tempa\expandafter}%
      \expandafter{\si@tempb}%
  \si@siu@newunithook[#1]{#2}{#3}}
\providecommand*{\si@siu@newunithook}[3][]{}
%    \end{macrocode}
%\end{macro}
%\end{macro}
%\end{macro}
% The basic units are now defined; these only have a single csname in
% each of the numerator and denominator.
%    \begin{macrocode}
\si@siu@newunit{gray}{second}
\si@siu@newunit[square]{metre}{second}
\si@siu@newunit{joule}{mole}
\si@siu@newunit[cubic]{mole}{metre}
\si@siu@newunit[square]{radian}{second}
\si@siu@newunit{radian}{second}
\si@siu@newunit[cubic]{squaremetre}{metre}
\si@siu@newunit[cubic]{katal}{metre}
\si@siu@newunit{coulomb}{mol}
\si@siu@newunit[square]{ampere}{metre}
\si@siu@newunit[cubic]{kilogram}{metre}
\si@siu@newunit[cubic]{coulomb}{metre}
\si@siu@newunit{volt}{metre}
\si@siu@newunit[square]{coulomb}{squaremetre}
\si@siu@newunit{farad}{metre}
\si@siu@newunit[square]{watt}{metre}
\si@siu@newunit[square]{joule}{metre}
\si@siu@newunit[cubic]{newton}{metre}
\si@siu@newunit{newton}{kilogram}
\si@siu@newunit{joule}{kelvin}
\si@siu@newunit{joule}{kilogram}
\si@siu@newunit{coulomb}{kilogram}
\si@siu@newunit{squaremetre}{second}
\si@siu@newunit[square]{squaremetre}{second}
\si@siu@newunit[square]{candela}{metre}
\si@siu@newunit{ampere}{metre}
\si@siu@newunit{joule}{tesla}
\si@siu@newunit{henry}{metre}
\si@siu@newunit{kilogram}{second}
\si@siu@newunit[square]{kilogram}{metre}
\si@siu@newunit{kilogram}{metre}
\si@siu@newunit[square]{newton}{metre}
\si@siu@newunit{watt}{kilogram}
\si@siu@newunit[cubic]{watt}{metre}
\si@siu@newunit{squaremetre}{kilogram}
\si@siu@newunit{cubicmetre}{kilogram}
\si@siu@newunit{newton}{metre}
\si@siu@newunit[cubic]{squaremetre}{second}
\si@siu@newunit{metre}{second}
\si@siu@newunit[cubic]{joule}{metre}
\si@siu@newunit{cubicmetre}{second}
%    \end{macrocode}
%\begin{macro}{\si@siu@newunitx}
%\begin{macro}{\si@siu@newunitxhook}
% For the more complex units, a slightly different approach is used;
% four arguments are required, and have to cover everything.
%    \begin{macrocode}
\newcommand*{\si@siu@newunitx}[4]{%
  \expandafter\newunit\expandafter{\csname #1per#2\endcsname}
    {#3\per#4}%
  \expandafter\newunit\expandafter{\csname #1per#2np\endcsname}
    {#3\reciprocal#4}
  \si@siu@newunitxhook{#1}{#2}{#3}{#4}}
\providecommand*{\si@siu@newunitxhook}[4]{}
%    \end{macrocode}
%\end{macro}
%\end{macro}
% The units are defined.
%    \begin{macrocode}
\si@siu@newunitx{kilogramsquaremetre}{second}
  {\kilogram\squaremetre}{\second}
\si@siu@newunitx{squaremetre}{newtonsecond}{\squaremetre}
  {\newton\second}
\si@siu@newunitx{kilogrammetre}{second}{\kilogram\metre}
  {\second}
\si@siu@newunitx{kilogram}{squaremetresecond}{\kilogram}
  {\squaremetre\second}
\si@siu@newunitx{joule}{molekelvin}{\joule}{\mole\kelvin}
\si@siu@newunitx{kilogram}{kilomole}{\kilogram}{\kilo\mole}
\si@siu@newunitx{kilogrammetre}{squaresecond}{\kilogram\metre}
  {\second\squared}
\si@siu@newunitx{watt}{squaremetresteradian}{\watt}
  {\squaremetre\steradian}
\si@siu@newunitx{joule}{kilogramkelvin}{\joule}
  {\kilogram\kelvin}
\si@siu@newunitx{watt}{metrekelvin}{\watt}{\metre\kelvin}
\si@siu@newunitx{kilogram}{cubicmetrecoulomb}{\kilogram}
  {\cubic\metre\coulomb}
\si@siu@newunitx{kilogram}{secondcubicmetre}{\kilogram}
  {\second\cubicmetre}
%    \end{macrocode}
%\begin{macro}{\si@siu@unity}
% A bit of cleverness to get the ``1'' correct; to avoid any clash, the unit
% is given an internal name.
%    \begin{macrocode}
\newunit{\si@siu@unity}{1}
\si@siu@newunitx{}{squaremetresecond}{\si@siu@unity}
  {\squaremetre\second}
%    \end{macrocode}
%\end{macro}
% A few compound units that are best defined directly.
%    \begin{macrocode}
\newunit{\pascalsecond}{\pascal\second}
\newunit{\amperemetresecond}{\ampere\metre\second}
\newunit{\ohmmetre}{\ohm\metre}
\newunit{\newtonmetre}{\newton\metre}
\let\newtonmetrenp\newtonmetre
\newunit{\kilogramsquaremetre}{\kilogram\squaremetre}
\let\kilogramsquaremetrenp\kilogramsquaremetre
%    \end{macrocode}
%\begin{macro}{\si@siu@newprefix}
%\darg{prefix}
% To generate the prefixes correctly, a small saving in repetition.
%    \begin{macrocode}
\newcommand*{\si@siu@newprefix}[1]{%
  \edef\si@tempa{\expandafter\noexpand\csname #1d\endcsname}%
  \edef\si@tempb{\expandafter\noexpand\csname #1\endcsname}%
  \expandafter\expandafter\expandafter\newcommand\expandafter
    \expandafter\expandafter*\expandafter\expandafter
    \expandafter{\expandafter\si@tempa\expandafter}\expandafter
      {\expandafter\si@prefixsymbolicfalse\si@tempb}}
%    \end{macrocode}
%\end{macro}
% This is now implemented.
%    \begin{macrocode}
\si@siu@newprefix{yocto}
\si@siu@newprefix{zepto}
\si@siu@newprefix{atto}
\si@siu@newprefix{femto}
\si@siu@newprefix{pico}
\si@siu@newprefix{nano}
\si@siu@newprefix{micro}
\si@siu@newprefix{milli}
\si@siu@newprefix{centi}
\si@siu@newprefix{deca}
\si@siu@newprefix{deka}
\si@siu@newprefix{hecto}
\si@siu@newprefix{kilo}
\si@siu@newprefix{mega}
\si@siu@newprefix{giga}
\si@siu@newprefix{tera}
\si@siu@newprefix{peta}
\si@siu@newprefix{exa}
\si@siu@newprefix{zetta}
\si@siu@newprefix{yotta}
\ifsi@old@binary
  \si@siu@newprefix{kibi}
  \si@siu@newprefix{mebi}
  \si@siu@newprefix{gibi}
  \si@siu@newprefix{tebi}
  \si@siu@newprefix{pebi}
  \si@siu@newprefix{exbi}
\fi
%    \end{macrocode}
%\begin{macro}{\derradian}
%\begin{macro}{\dersteradian}
%\begin{macro}{\derhertz}
%\begin{macro}{\dernewton}
%\begin{macro}{\derpascal}
%\begin{macro}{\derjoule}
%\begin{macro}{\derwatt}
%\begin{macro}{\dercoulomb}
%\begin{macro}{\dervolt}
%\begin{macro}{\derfarad}
%\begin{macro}{\derohm}
% The derived units may need to be defined.
%    \begin{macrocode}
\ifsi@old@derived
  \newunit{\derradian}{\metre\reciprocal\metre}
  \newunit{\dersteradian}{\squaremetre\rpsquare\metre}
  \newunit{\derhertz}{\reciprocal\second}
  \newunit{\dernewton}{\metre\kilogram\second\rpsquared}
  \newunit{\derpascal}{\newton\rpsquare\metre}
  \newunit{\derjoule}{\newton\metre}
  \newunit{\derwatt}{\joule\reciprocal\second}
  \newunit{\dercoulomb}{\ampere\second}
  \newunit{\dervolt}{\watt\reciprocal\ampere}
  \newunit{\derfarad}{\coulomb\reciprocal\volt}
  \newunit{\derohm}{\volt\reciprocal\ampere}
%    \end{macrocode}
%\end{macro}
%\end{macro}
%\end{macro}
%\end{macro}
%\end{macro}
%\end{macro}
%\end{macro}
%\end{macro}
%\end{macro}
%\end{macro}
%\end{macro}
%\begin{macro}{\dersiemens}
%\begin{macro}{\derweber}
%\begin{macro}{\dertesla}
%\begin{macro}{\derhenry}
%\begin{macro}{\dercelsius}
%\begin{macro}{\derlumen}
%\begin{macro}{\derlux}
%\begin{macro}{\derbecquerel}
%\begin{macro}{\dergray}
%\begin{macro}{\dersievert}
%\begin{macro}{\derkatal}
% In two blocks!
%    \begin{macrocode}
  \newunit{\dersiemens}{\ampere\reciprocal\volt}
  \newunit{\derweber}
    {\squaremetre\kilogram\second\rpsquared\reciprocal\ampere}
  \newunit{\dertesla}{\weber\rpsquare\metre}
  \newunit{\derhenry}{\weber\reciprocal\ampere}
  \newunit{\dercelsius}{\kelvin}
  \newunit{\derlumen}{\candela\steradian}
  \newunit{\derlux}{\lumen\rpsquare\metre}
  \newunit{\derbecquerel}{\derhertz}
  \newunit{\dergray}{\joule\reciprocal\kilogram}
  \newunit{\dersievert}{\dergray}
  \newunit{\derkatal}{\rp\second\usk\mole}
\fi
%    \end{macrocode}
%\end{macro}
%\end{macro}
%\end{macro}
%\end{macro}
%\end{macro}
%\end{macro}
%\end{macro}
%\end{macro}
%\end{macro}
%\end{macro}
%\end{macro}
%\begin{macro}{\radianbase}
%\begin{macro}{\steradianbase}
%\begin{macro}{\hertzbase}
%\begin{macro}{\newtonbase}
%\begin{macro}{\pascalbase}
%\begin{macro}{\joulebase}
%\begin{macro}{\wattbase}
%\begin{macro}{\coulombbase}
%\begin{macro}{\voltbase}
%\begin{macro}{\faradbase}
%\begin{macro}{\ohmbase}
% Also the ``derived-in-base''.
%    \begin{macrocode}
\ifsi@old@derivedinbase
  \newunit{\radianbase}{\metre\reciprocal\metre}
  \newunit{\steradianbase}{\squaremetre\rpsquare\metre}
  \newunit{\hertzbase}{\reciprocal\second}
  \newunit{\newtonbase}{\metre\kilogram\second\rpsquared}
  \newunit{\pascalbase}{\reciprocal\metre\kilogram\second%
    \rpsquared}
  \newunit{\joulebase}{\squaremetre\kilogram\second\rpsquared}
  \newunit{\wattbase}{\squaremetre\kilogram\rpcubic\second}
  \newunit{\coulombbase}{\ampere\second}
  \newunit{\voltbase}
    {\squaremetre\kilogram\rpcubic\second\reciprocal\ampere}
  \newunit{\faradbase}
    {\rpsquare\metre\reciprocal\kilogram\fourth\second\ampere%
      \squared}
  \newunit{\ohmbase}
    {\squaremetre\kilogram\rpcubic\second\rpsquare\ampere}
%    \end{macrocode}
%\end{macro}
%\end{macro}
%\end{macro}
%\end{macro}
%\end{macro}
%\end{macro}
%\end{macro}
%\end{macro}
%\end{macro}
%\end{macro}
%\end{macro}
%\begin{macro}{\siemensbase}
%\begin{macro}{\weberbase}
%\begin{macro}{\teslabase}
%\begin{macro}{\henrybase}
%\begin{macro}{\celsiusbase}
%\begin{macro}{\lumenbase}
%\begin{macro}{\luxbase}
%\begin{macro}{\becquerelbase}
%\begin{macro}{\graybase}
%\begin{macro}{\sievertbase}
%\begin{macro}{\katalbase}
% Also in two blocks.
%    \begin{macrocode}
  \newunit{\siemensbase}
    {\rpsquare\metre\reciprocal\kilogram\cubic\second\ampere%
       \squared}
  \newunit{\weberbase}
    {\squaremetre\kilogram\second\rpsquared\reciprocal\ampere}
  \newunit{\teslabase}{\kilogram\second\rpsquared\reciprocal%
     \ampere}
  \newunit{\henrybase}
    {\squaremetre\kilogram\second\rpsquared\rpsquare\ampere}
  \newunit{\celsiusbase}{\kelvin}
  \newunit{\lumenbase}{\candela\squaremetre\rpsquare\metre}
  \newunit{\luxbase}{\candela\squaremetre\rpfourth\metre}
  \newunit{\becquerelbase}{\hertzbase}
  \newunit{\graybase}{\squaremetre\second\rpsquared}
  \newunit{\sievertbase}{\graybase}
  \newunit{\katalbase}{\rp\second\mole}
\fi
%    \end{macrocode}
%\end{macro}
%\end{macro}
%\end{macro}
%\end{macro}
%\end{macro}
%\end{macro}
%\end{macro}
%\end{macro}
%\end{macro}
%\end{macro}
%\end{macro}
% Any configuration file is used if found.
%    \begin{macrocode}
\InputIfFileExists{SIunits.cfg}
  {\si@log@inf{SIunits config file loaded}}
  {\si@log@inf{SIunits config file not found}}
%    \end{macrocode}
%\iffalse
%</SIunits>
%<*hepunits>
%\fi
%\subsection{\pkg{hepunits}}
% The \pkg{hepunits} package provides some rather odd unit names,
% which are not really to be encouraged.
%    \begin{macrocode}
\ProvidesFile{si-hepunits.cfg}
  [2008/12/22 v1.1j siunitx: Emulation of
     hepunits]
\si@emulating{hepunits}{2007/09/27}
\requiresiconfigs{SIunits,accepted,prefix,hep}
%    \end{macrocode}
%\begin{macro}{\invbarn}
%\begin{macro}{\invnanobarn}
%\begin{macro}{\invpicobarn}
%\begin{macro}{\invfemtobarn}
%\begin{macro}{\invattobarn}
%\begin{macro}{\invzeptobarn}
%\begin{macro}{\invyoctobarn}
% Inverses barn units.
%    \begin{macrocode}
\ifsi@old@noprefixcmds\else
  \newunit{\invbarn}{\per\barn}
  \newunit{\invnanobarn}{\per\nano\barn}
  \newunit{\invpicobarn}{\per\pico\barn}
  \newunit{\invfemtobarn}{\per\femto\barn}
  \newunit{\invattobarn}{\per\atto\barn}
  \newunit{\invzeptobarn}{\per\zepto\barn}
  \newunit{\invyoctobarn}{\per\yocto\barn}
%    \end{macrocode}
%\end{macro}
%\end{macro}
%\end{macro}
%\end{macro}
%\end{macro}
%\end{macro}
%\begin{macro}{\invnb}
%\begin{macro}{\invpb}
%\begin{macro}{\invfb}
%\begin{macro}{\invab}
%\begin{macro}{\invzb}
%\begin{macro}{\invyb}
% Also available abbreviated.
%    \begin{macrocode}
  \newunit{\invnb}{\per\nano\barn}
  \newunit{\invpb}{\per\pico\barn}
  \newunit{\invfb}{\per\femto\barn}
  \newunit{\invab}{\per\atto\barn}
  \newunit{\invzb}{\per\zepto\barn}
  \newunit{\invyb}{\per\yocto\barn}
\fi
%    \end{macrocode}
%\end{macro}
%\end{macro}
%\end{macro}
%\end{macro}
%\end{macro}
%\end{macro}
%\end{macro}
%\begin{macro}{\invcmsqpersecond}
%\begin{macro}{\invcmsqpersec}
%\begin{macro}{\lumiunits}
% Luminosity.
%    \begin{macrocode}
\newunit{\invcmsqpersecond}{\per\Square\centi\metre\per\second}
\newunit{\invcmsqpersec}{\per\Square\centi\metre\per\second}
\newunit{\lumiunits}{\per\Square\centi\metre\per\second}
%    \end{macrocode}
%\end{macro}
%\end{macro}
%\end{macro}
%\begin{macro}{\inveV}
%\begin{macro}{\minveV}
%\begin{macro}{\minveV}
%\begin{macro}{\kinveV}
%\begin{macro}{\MinveV}
%\begin{macro}{\GinveV}
%\begin{macro}{\TinveV}
% The inverse of an electron-volt, plus prefixes.
%    \begin{macrocode}
\newunit{\inveV}{\per\electronvolt}
\newunit{\minveV}{\per\milli\electronvolt}
\newunit{\kinveV}{\per\kilo\electronvolt}
\newunit{\MinveV}{\per\mega\electronvolt}
\newunit{\GinveV}{\per\giga\electronvolt}
\newunit{\TinveV}{\per\tera\electronvolt}
%    \end{macrocode}
%\end{macro}
%\end{macro}
%\end{macro}
%\end{macro}
%\end{macro}
%\end{macro}
%\end{macro}
%\begin{macro}{\eVoverc}
%\begin{macro}{\eVovercsq}
% Some combinations of electron-volts and the speed of light.  As
% these are called |over|, they are set with a slash.  The
% \opt{eVcorrb} values have been set for Computer Modern.
%    \begin{macrocode}
\newunit[per=slash,eVcorrb=0.6ex]{\eVoverc}
  {\electronvolt\per\clight}
\newunit[per=slash,eVcorrb=0.6ex]{\eVovercsq}
  {\electronvolt\per\Square\clight}
%    \end{macrocode}
%\end{macro}
%\end{macro}
%\begin{macro}{\meVoverc}
%\begin{macro}{\keVoverc}
%\begin{macro}{\MeVoverc}
%\begin{macro}{\GeVoverc}
%\begin{macro}{\TeVoverc}
% Prefixed combinations, first of the speed of light.
%    \begin{macrocode}
\newunit[per=slash,eVcorrb=0.6ex]{\meVoverc}
  {\milli\electronvolt\per\clight}
\newunit[per=slash,eVcorrb=0.6ex]{\keVoverc}
  {\kilo\electronvolt\per\clight}
\newunit[per=slash,eVcorrb=0.6ex]{\MeVoverc}
  {\mega\electronvolt\per\clight}
\newunit[per=slash,eVcorrb=0.6ex]{\GeVoverc}
  {\giga\electronvolt\per\clight}
\newunit[per=slash,eVcorrb=0.6ex]{\TeVoverc}
  {\tera\electronvolt\per\clight}
%    \end{macrocode}
%\end{macro}
%\end{macro}
%\end{macro}
%\end{macro}
%\end{macro}
%\begin{macro}{\meVovercsq}
%\begin{macro}{\keVovercsq}
%\begin{macro}{\MeVovercsq}
%\begin{macro}{\GeVovercsq}
%\begin{macro}{\TeVovercsq}
% Then of the square.
%    \begin{macrocode}
\newunit[per=slash,eVcorrb=0.6ex]{\meVovercsq}
  {\milli\electronvolt\per\Square\clight}
\newunit[per=slash,eVcorrb=0.6ex]{\keVovercsq}
  {\kilo\electronvolt\per\Square\clight}
\newunit[per=slash,eVcorrb=0.6ex]{\MeVovercsq}
  {\mega\electronvolt\per\Square\clight}
\newunit[per=slash,eVcorrb=0.6ex]{\GeVovercsq}
  {\giga\electronvolt\per\Square\clight}
\newunit[per=slash,eVcorrb=0.6ex]{\TeVovercsq}
  {\tera\electronvolt\per\Square\clight}
%    \end{macrocode}
%\end{macro}
%\end{macro}
%\end{macro}
%\end{macro}
%\end{macro}
%\iffalse
%</hepunits>
%<*fancynum>
%\fi
%\subsection{\pkg{fancynum}}
% \pkg{fancynum} only does things with numbers, so there is only
% a little emulation and a few macros needed.
%    \begin{macrocode}
\ProvidesFile{si-fancynum.cfg}
  [2008/12/22 v1.1j siunitx: Emulation of
     fancynum]
\si@emulating{fancynum}{2000/08/08 0.92}
\sisetup{decimalsymbol=cdot,digitsep=comma}
%    \end{macrocode}
%\begin{macro}{\fnum}
% The \cs{fnum} macro is rather restricted, but this is not
% reproduced.  Instead, it is \cs{let} as an alias to the \cs{num}
% macro.
%    \begin{macrocode}
\let\fnum\num
%    \end{macrocode}
%\end{macro}
%\begin{macro}{\setfnumdsym}
%\begin{macro}{\setfnumgsym}
%\begin{macro}{\setfnummsym}
% The control macros are defined.
%    \begin{macrocode}
\newcommand*{\setfnumdsym}[1]{\sisetup{decimalsymbol={#1}}}
\newcommand*{\setfnumgsym}[1]{\sisetup{digitsep={#1}}}
\newcommand*{\setfnummsym}[1]{\sisetup{expproduct={#1}}}
%    \end{macrocode}
%\end{macro}
%\end{macro}
%\end{macro}
% The various package options are now processed if necessary.
%    \begin{macrocode}
\ifsi@old@english
  \sisetup{decimalsymbol=cdot,digitsep=comma}
\fi
\ifsi@old@french
  \sisetup{decimalsymbol=comma,digitsep=fullstop}
\fi
\ifsi@old@tight
  \sisetup{expproduct=tighttimes}
\fi
\ifsi@old@loose
  \sisetup{expproduct=times}
\fi
\ifsi@old@thinspaces
  \sisetup{digitsep=thin}
\fi
\ifsi@old@commas
  \sisetup{digitsep=comma}
\fi
\ifsi@old@plain
  \sisetup{digitsep=none}
\fi
%    \end{macrocode}
%
%\iffalse
%</fancynum>
%<*fancyunits>
%\fi
%\subsection{\pkg{fancyunits}}
% The \pkg{fancyunits} package is not available on \textsc{ctan}, but
% is available from its authors homepage \cite{fancyunits}.  It is
% similar to \SIunits, and so most of the code is shared here.
% However, a few bits of set up occur first, and an emulation-clash
% test is needed.
%    \begin{macrocode}
\ProvidesFile{si-fancyunits.cfg}
  [2008/12/22 v1.1j siunitx: Emulation of
     fancyunits]
\si@emulating{fancyunits}{2007/02/01 v1.0.1}
\si@ifloaded{SIunits}
  {\si@log@err{SIunits emulation loaded\MessageBreak before
    fancyunits emulation}{You need to load the fancyunits
    emualtion\MessageBreak code before that for
    SIunits.\MessageBreak Try emulate=fancyunits as the first
    option when\MessageBreak loading siunitx}}{}
%    \end{macrocode}
%\begin{macro}{\si@siu@newunithook}
%\begin{macro}{\si@siu@newunitxhook}
% To create the extra macros provided by \pkg{fancyunits}, the
% \SIunits emulation code is changed to add the ``\texttt{uf}''
% variants.
%    \begin{macrocode}
\newcommand*{\si@siu@newunithook}[3][]{%
  \edef\si@tempa{%
    \expandafter\noexpand\csname #2per#1#3uf\endcsname}%
  \renewcommand*{\si@tempb}{stickyper,per=fraction,
    fraction=nice}%
  \edef\si@tempc{%
    \noexpand\sisetup{\si@tempb}%
    \expandafter\noexpand\csname #2\endcsname\noexpand\si@per%
    \expandafter\noexpand\si@siu@power%
    \expandafter\noexpand\csname #3\endcsname}%
  \expandafter\expandafter\expandafter\newunit\expandafter
    \expandafter\expandafter{\expandafter\si@tempa\expandafter}%
      \expandafter{\si@tempc}%
  \edef\si@tempa{%
    \expandafter\noexpand\csname #2per#1#3Uf\endcsname}%
  \renewcommand*{\si@tempb}{stickyper,per=fraction,
    fraction=frac}%
  \edef\si@tempc{%
    \noexpand\sisetup{\si@tempb}%
    \noexpand\def\noexpand\si@frc@hook{\noexpand\textstyle}%
    \expandafter\noexpand\csname #2\endcsname\noexpand\si@per%
    \expandafter\noexpand\si@siu@power%
    \expandafter\noexpand\csname #3\endcsname}%
  \expandafter\expandafter\expandafter\newunit\expandafter
    \expandafter\expandafter{\expandafter\si@tempa\expandafter}%
      \expandafter{\si@tempc}%
  \edef\si@tempa{%
    \expandafter\noexpand\csname #2per#1#3UF\endcsname}%
  \edef\si@tempc{%
    \noexpand\sisetup{\si@tempb}%
    \noexpand\def\noexpand\si@frc@hook{\noexpand\displaystyle}%
    \expandafter\noexpand\csname #2\endcsname\noexpand\si@per%
    \expandafter\noexpand\si@siu@power%
    \expandafter\noexpand\csname #3\endcsname}%
  \expandafter\expandafter\expandafter\newunit\expandafter
    \expandafter\expandafter{\expandafter\si@tempa\expandafter}%
      \expandafter{\si@tempc}}
\newcommand*{\si@siu@newunitxhook}[4]{%
  \expandafter\newunit\expandafter{\csname #1per#2uf\endcsname}
    {\sisetup{stickyper,per=fraction,fraction=nice}%
     #3\si@per#4}%
  \expandafter\newunit\expandafter{\csname #1per#2Uf\endcsname}
    {\sisetup{stickyper,per=fraction,fraction=frac}%
     \renewcommand*{\si@frc@hook}{\textstyle}%
     #3\si@per#4}%
  \expandafter\newunit\expandafter{\csname #1per#2UF\endcsname}
    {\sisetup{stickyper,per=fraction,fraction=frac}%
     \renewcommand*{\si@frc@hook}{\displaystyle}%
     #3\si@per#4}}
%    \end{macrocode}
%\end{macro}
%\end{macro}
% With that done, the emulation modules can be loaded.
%    \begin{macrocode}
\requiresiconfigs{SIunits,addn,astro}
\sisetup{obeyall}
%    \end{macrocode}
% There is one \pkg{fancyunits}-specific option to handle.  The other
% options all get sent through to the \SIunits system.\footnote{As
% \SIunits is rather more widely known than \pkg{fancyunits}, any
% other options which could be for either are assumed to be for
% \SIunits.}
%    \begin{macrocode}
\ifsi@old@spaceqspace
  \sisetup{valuesep=space}
\fi
%    \end{macrocode}
%\begin{macro}{\paminute}
%\begin{macro}{\parsecond}
%\begin{macro}{\AstroE}
%\begin{macro}{\oersted}
%\begin{macro}{\ton}
% \pkg{fancyunits} provides some extra units, plus tonne spelled
% incorrectly (again).
%    \begin{macrocode}
\newunit{\paminute}{'}
\newunit{\parsecond}{''}
\newunit{\AstroE}{AE}
\newunit{\oersted}{OE}
\provideunit{\ton}{t}
%    \end{macrocode}
%\end{macro}
%\end{macro}
%\end{macro}
%\end{macro}
%\end{macro}
%\begin{macro}{\decaD}
% An additional prefix.
%    \begin{macrocode}
\let\decaD\decad
%    \end{macrocode}
%\end{macro}
%\begin{macro}{\ufrac}
%\begin{macro}{\Ufrac}
%\begin{macro}{\UFrac}
% The fractional unit macros need to be reproduced.
%    \begin{macrocode}
\newcommand*{\ufrac}[2]{%
  \si[stickyper,per=fraction,fraction=nice]{#1\si@per#2}}
\newcommand*{\Ufrac}[2]{%
  \ensuremath{\textstyle{%
    \si[stickyper,per=fraction,fraction=frac]{#1\si@per#2}}}}
\newcommand*{\UFrac}[2]{%
  \ensuremath{\displaystyle{%
    \si[stickyper,per=fraction,fraction=frac]{#1\si@per#2}}}}
%    \end{macrocode}
%\end{macro}
%\end{macro}
%\end{macro}
%\begin{macro}{\pow}
% A slightly-shorted named \cs{power}.
%    \begin{macrocode}
\let\pow\power
%    \end{macrocode}
%\end{macro}
%\begin{macro}{\Squaremetre}
% An alias for \cs{squaremetre}.
%    \begin{macrocode}
\let\Squaremetre\squaremetre
%    \end{macrocode}
%\end{macro}
% As with \SIunits, there is now a list of compound units to add.
% Only a few are not covered by the \SIunits emulation.
%    \begin{macrocode}
\si@siu@newunit{Gray}{second}
\si@siu@newunit[square]{Squaremetre}{metre}
\si@siu@newunitx{Squaremetre}{newtonsecond}{\Square\metre}
  {\newton\second}
\si@siu@newunit{Squaremetre}{second}
\si@siu@newunit[square]{Squaremetre}{squaresecond}
\si@siu@newunit{Squaremetre}{kilogram}
\si@siu@newunit[cubic]{Squaremetre}{second}
%    \end{macrocode}
%\iffalse
%</fancyunits>
%\fi
%
%\Finale
%\iffalse
%<*refs>
@misc{BIPM:EN,
  note = {\url{http://www.bipm.org/en/si/}},
}

@misc{BIPM:FR,
  note = {\url{http://www.bipm.org/fr/si/}},
}

@misc{BIPM:PDF,
  note = {\url{http://www.bipm.org/en/si/si_brochure/}},
}

@misc{Eijkhout,
  author =       {Victor Eijkhout},
  note =         {\url{http://www.eijkhout.net/tbt/}},
  title =        {{\TeX} by {T}opic},
  year =         {2007},
}

@misc{fancyunits,
  author = {Heiko Bauke},
  note = {\url{http://www.mpi-hd.mpg.de/personalhomes/bauke/LaTeX/Tips_und_Tricks/fancyunits/index.php}},
  title = {\pkg{fancyunits}},
  year = 2007,
}

@misc{IUPAC,
  note = {\url{http://old.iupac.org/publications/books/gbook/green_book_2ed.pdf}},
  title = {The \textsc{iupac} {G}reen {B}ook},
  year = 1993,
}

@misc{NIST,
  note = {\url{http://physics.nist.gov/cuu/Units/index.html}},
}
%</refs>
%<*jawltxdoc>
\NeedsTeXFormat{LaTeX2e}
\ProvidesPackage{jawltxdoc}
\usepackage[T1]{fontenc}
\usepackage{lmodern}
\usepackage[final]{listings,graphicx,microtype}
\usepackage[scaled=0.95]{helvet}
\usepackage[version=3]{mhchem}
\usepackage[osf]{mathpazo}
\usepackage{booktabs,array,url,courier,xspace,varioref}
\usepackage{upgreek,ifpdf,float,caption,longtable,babel}
\begingroup
  \@ifundefined{eTeXversion}
    {\aftergroup\@gobble}
    {\aftergroup\@firstofone}
\endgroup
{\usepackage{etoolbox}}
\floatstyle{plaintop}
\restylefloat{table}
\labelformat{figure}{\figurename~#1}
\labelformat{table}{\tablename~#1}
\IfFileExists{\jobname.sty}
  {\usepackage{\jobname}}{}
\usepackage[numbered]{hypdoc}
\setcounter{IndexColumns}{2}
\newlength\LaTeXwidth
\newlength\LaTeXoutdent
\newlength\LaTeXgap
\setlength\LaTeXgap{1em}
\setlength\LaTeXoutdent{-0.15\textwidth}
\newbox\lst@samplebox
\edef\LaTeXexamplefile{\jobname.tmp}
\lst@RequireAspects{writefile}
\lstnewenvironment{LaTeXexample}[1][example]{%
  \global\let\lst@intname\@empty
  \ifcsname LaTeXcode#1\endcsname
    \expandafter\let\expandafter\LaTeXcode
      \csname LaTeXcode#1\endcsname
    \expandafter\let\expandafter\LaTeXcodeend
      \csname LaTeXcode#1end\endcsname
  \else
    \PackageError{jawltxdoc}
      {Undefined example type `#1'}
      \@ehd
    \let\LaTeXcode\relax
    \let\LaTeXcodeend\relax
  \fi
  \LaTeXcode}
  {\lst@EndWriteFile
   \LaTeXcodeend}
\newcommand*{\LaTeXcodeexample}{%
  \setbox\lst@samplebox=\hbox\bgroup
  \LaTeXcodefloat}
\let\LaTeXcoderesultonly\LaTeXcodeexample
\newcommand*{\LaTeXcodeexampleend}{%
  \egroup
  \setlength\LaTeXwidth{\wd\lst@samplebox}%
  \begin{list}{}{%
    \setlength\itemindent{0pt}
    \setlength\leftmargin\LaTeXoutdent
    \setlength\rightmargin{0pt}}%
    \item
      \setlength\LaTeXoutdent{-0.15\textwidth}
      \begin{minipage}[c]{%
        \textwidth-\LaTeXwidth-\LaTeXoutdent-\LaTeXgap}
        \LaTeXcodefloatend
      \end{minipage}%
      \hfill
      \begin{minipage}[c]{\LaTeXwidth}%
        \hbox to\linewidth{\box\lst@samplebox\hss}%
      \end{minipage}%
  \end{list}}
\newcommand*{\LaTeXcodefloat}{%
  \setkeys{lst}{tabsize=4,gobble=3,breakindent=0pt,
    basicstyle=\small\ttfamily,basewidth=0.51em,
    keywordstyle=\color{blue}}%
  \lst@BeginAlsoWriteFile{\LaTeXexamplefile}}
\let\LaTeXcodenoexample\LaTeXcodefloat
\let\LaTeXcodenoexampleend\@empty
\newcommand*{\LaTeXcodefloatend}{%
  \MakePercentComment\catcode`\^^M=10\relax
  \small
  {\setkeys{lst}{SelectCharTable=\lst@ReplaceInput{\^\^I}%
    {\lst@ProcessTabulator}}%
    \leavevmode \input{\LaTeXexamplefile}}%
  \MakePercentIgnore}
\newcommand*{\LaTeXcoderesultonlyend}{\egroup\LaTeXcodefloatend}
\lstnewenvironment{BibTeXexample}{%
  \global\let\lst@intname\@empty
  \setbox\lst@samplebox=\hbox\bgroup
  \setkeys{lst}{tabsize=4,gobble=3,breakindent=0pt,
    basicstyle=\small\ttfamily,basewidth=0.51em,
    keywordstyle=\color{black}}
  \lst@BeginAlsoWriteFile{\LaTeXexamplefile}}
 {\lst@EndWriteFile
   \LaTeXcodeexampleend}
\newcommand*{\DescribeOption}{%
  \leavevmode\@bsphack\begingroup\MakePrivateLetters
  \Describe@Option}
\newcommand*{\Describe@Option}[1]{\endgroup
              \marginpar{\raggedleft\PrintDescribeEnv{#1}}%
              \SpecialOptionIndex{#1}\@esphack\ignorespaces}
\newcommand*{\SpecialOptionIndex}[1]{\@bsphack
    \index{#1\actualchar{\protect\ttfamily#1}
           (option)\encapchar usage}%
    \index{options:\levelchar#1\actualchar{\protect\ttfamily#1}%
      \encapchar usage}\@esphack}
\newcommand*{\indexopt}[1]{\DescribeOption{#1}\opt{#1}}
\newcommand*{\DescribeOptionInfo}[2]{%
  \DescribeOption{#1}%
  \opt{#1=\meta{#2}}\xspace}
\newcommand*{\ofixarg}[1]{%
  {\ttfamily[}%
  \ifmmode \expandafter \nfss@text \fi
  {%
    \meta@font@select
    \edef\meta@hyphen@restore{%
      \hyphenchar\the\font\the\hyphenchar\font}%
    \hyphenchar\font\m@ne
    \language\l@nohyphenation
    #1\/%
    \meta@hyphen@restore
    }%
    {\ttfamily]}}
\newcommand*{\pkg}[1]{\textsf{#1}}
\newcommand*{\currpkg}{\pkg{\jobname}\xspace}
\newcommand*{\opt}[1]{\texttt{#1}}
\newcommand*{\defaultopt}[1]{\opt{\textbf{#1}}}
\newcommand*{\file}[1]{\texttt{#1}}
\newcommand*{\ext}[1]{\file{.#1}}
\newcommand*{\latin}[1]{\emph{#1}}
\newcommand*{\etc}{%
  \@ifnextchar.
    {\latin{etc}}
    {\latin{etc}.\xspace}}
\newcommand*{\eg}{%
  \@ifnextchar.
    {\latin{e.g}}
    {\latin{e.g}.\xspace}}
\newcommand*{\ie}{%
  \@ifnextchar.
    {\latin{i.e}}
    {\latin{i.e}.\xspace}}
\newcommand*{\etal}{%
  \@ifnextchar.
    {\latin{et~al.}}
    {\latin{et~al}.\xspace}}
\newcommand*{\AMS}{{\protect\usefont{OMS}{cmsy}{m}{n}%
  A\kern-.1667em\lower.5ex\hbox{M}\kern-.125emS}}
\providecommand*{\eTeX}{\ensuremath{\varepsilon}-\TeX}
\DeclareRobustCommand*{\XeTeX}
  {X\kern-.125em\lower.5ex\hbox{\reflectbox{E}}\kern-.1667em\TeX}
\providecommand*{\CTAN}{\textsc{ctan}}
\@ifpackageloaded{etoolbox}
  {\patchcmd{\@addmarginpar}
    {\@latex@warning@no@line {Marginpar on page \thepage\space moved}}
    {\relax}{}{}}
  {}
\newcounter{argument}
\g@addto@macro\endmacro{\setcounter{argument}{0}}
\newcommand*\darg[1]{%
  \stepcounter{argument}%
  {\ttfamily\char`\#\theargument~:~}#1\par\noindent\ignorespaces}
\newcommand*\doarg[1]{%
  \stepcounter{argument}%
  {\ttfamily\makebox[0pt][r]{[}%
   \char`\#\theargument]:~}#1\par\noindent\ignorespaces}
%</jawltxdoc>
%\fi

%
% Documentation:
%    (a) Without write18 enabled:
%          pdflatex siunitx.dtx
%          bibtex8 --wolfgang siunitx
%          makeindex -s gind.ist siunitx.idx
%          makeindex -s gglo.ist -o siunitx.gls siunitx.glo
%          pdflatex siunitx.dtx
%          pdflatex siunitx.dtx
%    (b) With write18 enabled:
%          pdflatex siunitx.dtx
%          pdflatex siunitx.dtx
%          pdflatex siunitx.dtx
%
% Installation:
%     Copy siunitx.sty and the various .cfg files to a location
%     searched by TeX, and if required by your TeX installation,
%     run the appropriate command to build a hash of files
%     (texhash, initexmf --update-fndb, etc.)
%
% Note:
%     The jawltxdoc.sty file is not needed for installation,
%     only for building the documentation; it may be deleted
%     after producing the documentation (if necessary).
%
%<*ignore>
% This is all taken verbatim from Heiko Oberdiek's packages
\begingroup
  \def\x{LaTeX2e}%
\expandafter\endgroup
\ifcase 0\ifx\install y1\fi\expandafter
         \ifx\csname processbatchFile\endcsname\relax\else1\fi
         \ifx\fmtname\x\else 1\fi\relax
\else\csname fi\endcsname
%</ignore>
%<*install>
\input docstrip.tex
\keepsilent
\askforoverwritefalse
\preamble
 ----------------------------------------------------------------
 The siunitx package --- A comprehensive (SI) units package
 Maintained by Joseph Wright
 E-mail: joseph.wright@morningstar2.co.uk
 Released under the LaTeX Project Public License v1.3c or later
 See http://www.latex-project.org/lppl.txt
 ----------------------------------------------------------------

\endpreamble
\Msg{Generating siunitx files:}
\generate{\file{jawltxdoc.sty}{\from{\jobname.dtx}{jawltxdoc}}
}
\usedir{tex/latex/siunitx}
\generate{\file{\jobname.sty}{\from{\jobname.dtx}{package}}
}
\usedir{source/latex/siunitx}
\generate{\file{\jobname.ins}{\from{\jobname.dtx}{install}}
}
\usedir{doc/latex/siunitx}
\generate{\file{si-bug.ltx}{\from{\jobname.dtx}{bug}}
}
\usedir{tex/latex/siunitx/config}
\generate{\file{si-SIunits.cfg}{\from{\jobname.dtx}{SIunits}}
          \file{si-sistyle.cfg}{\from{\jobname.dtx}{sistyle}}
          \file{si-unitsdef.cfg}{\from{\jobname.dtx}{unitsdef}}
          \file{si-units.cfg}{\from{\jobname.dtx}{units}}
          \file{si-hepunits.cfg}{\from{\jobname.dtx}{hepunits}}
          \file{si-fancynum.cfg}{\from{\jobname.dtx}{fancynum}}
          \file{si-fancyunits.cfg}{\from{\jobname.dtx}{fancyunits}}
}
\generate{\file{si-prefix.cfg}{\from{\jobname.dtx}{prefix}}
          \file{si-named.cfg}{\from{\jobname.dtx}{named}}
          \file{si-prefixed.cfg}{\from{\jobname.dtx}{prefixed}}
          \file{si-addn.cfg}{\from{\jobname.dtx}{addn}}
          \file{si-accepted.cfg}{\from{\jobname.dtx}{accepted}}
          \file{si-physical.cfg}{\from{\jobname.dtx}{physical}}
          \file{si-abbr.cfg}{\from{\jobname.dtx}{abbr}}
}
\generate{\file{si-synchem.cfg}{\from{\jobname.dtx}{synchem}}
          \file{si-hep.cfg}{\from{\jobname.dtx}{hep}}
          \file{si-astro.cfg}{\from{\jobname.dtx}{astro}}
          \file{si-geophys.cfg}{\from{\jobname.dtx}{geophys}}
          \file{si-binary.cfg}{\from{\jobname.dtx}{binary}}
}
\generate{\file{si-UK.cfg}{\from{\jobname.dtx}{UK}}
          \file{si-USA.cfg}{\from{\jobname.dtx}{USA}}
          \file{si-DE.cfg}{\from{\jobname.dtx}{DE}}
          \file{si-ZA.cfg}{\from{\jobname.dtx}{ZA}}
}
\nopreamble\nopostamble
\usedir{doc/latex/siunitx}
\generate{\file{README.txt}{\from{\jobname.dtx}{readme}}
          \file{\jobname.bib}{\from{\jobname.dtx}{refs}}
}
\endbatchfile
%</install>
%<*readme>
----------------------------------------------------------------
The siunitx package --- A comprehensive (SI) units package
Maintained by Joseph Wright
E-mail: joseph.wright@morningstar2.co.uk
Released under the LaTeX Project Public License v1.3c or later
See http://www.latex-project.org/lppl.txt
----------------------------------------------------------------

Typesetting values with units requires care to ensure that the
combined mathematical meaning of the value plus unit combination
is clear.  In particular, the SI units system lays down a
consistent set of units with rules on how these are to be used.
However, different countries and publishers have differing
conventions on the exact appearance of numbers (and units).

The siunitx provides a set of tools for authors to typeset
numbers and units in a consistent way.  The package has an
extended set of configuration options which make it possible to
follow varying typographic conventions with the same input
syntax.  The package includes automated processing of numbers
and units, and the ability to control tabular alignment of
numbers.

A number of LaTeX packages have been developed in the past for
formatting units: SIunits, sistyle, unitsdef, unitspkg,
fancyunits and fancynum.  Support for users of all of these
packages is available as emulation modules in siunitx.  In
addition, siunitx can carry out many of the functions of the
dcolumn, rccol and numprint packages.
%</readme>
%<*ignore>
\fi
% Will Robertson's trick
\immediate\write18{bibtex8 --wolfgang \jobname}
\immediate\write18{makeindex -s gind.ist -o \jobname.ind  \jobname.idx}
\immediate\write18{makeindex -s gglo.ist -o \jobname.gls  \jobname.glo}
%</ignore>
%<*driver>
\PassOptionsToClass{a4paper}{article}
\documentclass[german,english,UKenglish]{ltxdoc}
\EnableCrossrefs
\CodelineIndex
\RecordChanges
%\OnlyDescription
% Various adjustments are needed to the packages loaded.  xfrac is
% only loaded if it is available.
\usepackage[final]{graphicx}
% Currently, xfrac is not working with expl3!
%\IfFileExists{xfrac.sty}
%  {\usepackage{xfrac}}
%  {}
\usepackage{jawltxdoc}
\usepackage{textcomp,tikz,pgfplots,tocloft,datatool}
\usepackage[latin1]{inputenc}
\sisetup{
  eVcorra=0.1ex,
  unitsep=cdot,
  alsoload={synchem,hep,binary,astro,geophys},
  loctolang={UK:UKenglish,DE:german}}
\setlength{\cftsubsecnumwidth}{2.5em}
\begin{document}
  \DocInput{\jobname.dtx}
\end{document}
%</driver>
% \fi
%
%\CheckSum{10177}
%
% \CharacterTable
%  {Upper-case    \A\B\C\D\E\F\G\H\I\J\K\L\M\N\O\P\Q\R\S\T\U\V\W\X\Y\Z
%   Lower-case    \a\b\c\d\e\f\g\h\i\j\k\l\m\n\o\p\q\r\s\t\u\v\w\x\y\z
%   Digits        \0\1\2\3\4\5\6\7\8\9
%   Exclamation   \!     Double quote  \"     Hash (number) \#
%   Dollar        \$     Percent       \%     Ampersand     \&
%   Acute accent  \'     Left paren    \(     Right paren   \)
%   Asterisk      \*     Plus          \+     Comma         \,
%   Minus         \-     Point         \.     Solidus       \/
%   Colon         \:     Semicolon     \;     Less than     \<
%   Equals        \=     Greater than  \>     Question mark \?
%   Commercial at \@     Left bracket  \[     Backslash     \\
%   Right bracket \]     Circumflex    \^     Underscore    \_
%   Grave accent  \`     Left brace    \{     Vertical bar  \|
%   Right brace   \}     Tilde         \~}
%
%\GetFileInfo{\jobname.sty}
%
%\DoNotIndex{\@car,\@eha,\@ehc,\@empty,\@firstofone,\@firstoftwo}
%\DoNotIndex{\@for,\@gobble,\@gobbletwo,\@ifl@aded,\@ifnextchar}
%\DoNotIndex{\@ifpackagelater,\@ifpackageloaded,\@ifstar}
%\DoNotIndex{\@ifundefined,\@makeother,\@namedef,\@nameuse,\@ne}
%\DoNotIndex{\@nil,\@secondoftwo,\@temptokena,\@undefined,\AA}
%\DoNotIndex{\active,\addto,\addtolength,\advance,\afterassignment}
%\DoNotIndex{\aftergroup,\AtBeginDocument,\AtEndOfPackage}
%\DoNotIndex{\begingroup,\bfseries,\bgroup,\boldmath,\box}
%\DoNotIndex{\capitalring,\catcode,\cdot,\chardef,\check@mathfonts}
%\DoNotIndex{\circ,\csname,\DeclareFontEncoding,\DeclareFontFamily}
%\DoNotIndex{\DeclareFontShape,\DeclareFontSubstitution}
%\DoNotIndex{\DeclareMathSymbol,\DeclareRobustCommand}
%\DoNotIndex{\DeclareSymbolFont,\DeclareTextSymbol}
%\DoNotIndex{\DeclareTextSymbolDefault,\def,\define@boolkey}
%\DoNotIndex{\define@choicekey,\define@cmdkey,\define@cmdkeys}
%\DoNotIndex{\define@key,\displaystyle,\displaywidth,\divide,\do}
%\DoNotIndex{\edef,\egroup,\else,\encodingdefault,\end,\endcsname}
%\DoNotIndex{\endgroup,\endinput,\endtabular,\ensuremath}
%\DoNotIndex{\expandafter,\f@family,\f@series,\fam,\fi,\font}
%\DoNotIndex{\fontsize,\frac,\gdef,\global,\hbox,\hfil,\hfill,\hskip}
%\DoNotIndex{\hspace,\hss,\ifcsname,\ifdim,\iffalse,\IfFileExists}
%\DoNotIndex{\ifmmode,\ifnum,\ifx,\ignorespaces,\inputencodingname}
%\DoNotIndex{\InputIfFileExists,\key@ifundefine,\let,\long,\m@ne}
%\DoNotIndex{\makeatletter,\math@version,\mathchoice,\mathnormal}
%\DoNotIndex{\mathord,\mathrm,\mathsf,\mathtt,\mdseries}
%\DoNotIndex{\MessageBreak,\mkern,\mp,\NC@find,\NeedsTeXFormat}
%\DoNotIndex{\newbox,\newcolumntype,\newcommand,\newcount,\newdimen}
%\DoNotIndex{\newif,\newlength,\newtoks,\nobreak,\noexpand,\p@}
%\DoNotIndex{\PackageError,\PackageInfo,\PackageWarning,\phantom}
%\DoNotIndex{\ProcessOptionsX,\protect,\protected@edef,\ProvidesFile}
%\DoNotIndex{\ProvidesPackage,\Q,\raisebox,\relax,\renewcommand}
%\DoNotIndex{\RequirePackage,\rlap,\sbox,\scantokens}
%\DoNotIndex{\scriptscriptstyle,\scriptstyle,\selectfont,\setbox}
%\DoNotIndex{\setkeys,\setlength,\settoheight,\sf@size,\sfdefault}
%\DoNotIndex{\sfrac,\skewchar,\sp,\space,\string,\strip@pt}
%\DoNotIndex{\tabularnewline,\text,\textdegree,\textmu,\textohm}
%\DoNotIndex{\textsc,\textstyle,\textsuperscript,\the,\thr@@,\times}
%\DoNotIndex{\ttdefault,\uccode,\unboldmath,\unhbox,\upmu,\Upomega}
%\DoNotIndex{\uppercase,\upshape,\wd,\xdef,\xspace,\z@}
%
%\DoNotIndex{\if,\lowercase,\+,\,,\-,\.,\:,\;,\@argdef,\@cdr}
%\DoNotIndex{\@ifdefinable,\@testopt,\@yargdef,\color,\endlinechar}
%\DoNotIndex{\key@ifundefined,\newrobustcmd,\protected,\scriptspace}
%\DoNotIndex{\usebox,\tw@,\ifinner,\usepackage,\texttt,\textsf}
%\DoNotIndex{\NC@do,\loop,\l@ngrel@x,\g@addto@macro,\documentclass}
%\DoNotIndex{\document\,\begin,\=,\@ehb,\\,\listfiles,\nullfont}
%\DoNotIndex{\normalsize,\futurelet}
%
%\DoNotIndex{\pascalsecond,\amperemetresecond,\ohmmetre}
%\DoNotIndex{\newtonmetre,\newtonmetrenp}
%\DoNotIndex{\kilogramsquaremetre,\kilogramsquaremetrenp}
%
%\changes{v1.0}{2008/06/12}{First official release}
%
%\setkeys{lst}{language=[LaTeX]{TeX},moretexcs={sisetup,num,ang,SI,
%  si,newunit,newpower,newprefix,metre,candela,kelvin,ampere,eV,
%  second,mole,per,kilogram,Square,squared,cubed,cubic,kilo,centi,
%  gram,tothe,toprule,midrule,bottomrule,newton,micro,joule,mega,
%  raiseto,deci,mathnormal,numprint,electronvolt,selectlanguage,
%  kibi,byte,molar,Molar,mmHg,mebi,torr,parsec,lightyear,clight,
%  eVperc,addplot,color,milli,cm,Hz,kg,DeclareRobustCommand,
%  pascal,dalton,addtolocale,DTLnewdb,DTLnewrow,DTLnewdbentry,
%  DTLforeach,DTLiffirstrow,DTLmul,newqualifier,hour,renewunit,
%  degree}}
%
%\title{\currpkg\ --- A comprehensive (SI) units package^^A
%  \thanks{This file describes version \fileversion, last revised
%    \filedate.}}
%\author{Joseph Wright^^A
%  \thanks{E-mail: joseph.wright@morningstar2.co.uk}}
%\date{Released \filedate}
%
%\maketitle
%
%\newcommand*{\SIunits}{\pkg{SIunits}\xspace}
%\newcommand*{\SIstyle}{\pkg{SIstyle}\xspace}
%\newcommand*{\unitsdef}{\pkg{unitsdef}\xspace}
%\newcommand*{\units}{\pkg{units}\xspace}
%\newcommand*{\numprint}{\pkg{numprint}\xspace}
%\begin{abstract}
% Typesetting values with units requires care to ensure that the
% combined mathematical meaning of the value plus unit combination is
% clear.  In particular, the SI units system lays down a consistent
% set of units with rules on how these are to be used.  However,
% different countries and publishers have differing conventions on
% the exact appearance of numbers (and units).
%
% The \currpkg provides a set of tools for authors to typeset numbers
% and units in a consistent way.  The package has an extended set of
% configuration options which make it possible to follow varying
% typographic conventions with the same input syntax.  The package
% includes automated processing of numbers and units, and the ability
% to control tabular alignment of numbers.
%
% A number of \LaTeX\ packages have been developed in the past for
% formatting units: \SIunits, \SIstyle, \unitsdef, \units,
% \pkg{fancyunits} and \pkg{fancynum}. Support for users of all of
% these packages is available as emulation modules in \currpkg.
% In addition, \currpkg can carry out many of the functions of the
% \pkg{dcolumn}, \pkg{rccol} and \numprint packages.
%\end{abstract}
%
%\begin{multicols}{2}
%  \tableofcontents
%\end{multicols}
%
%\clearpage
%\part{Introduction}
% The correct application of units of measurement is very important
% in technical applications.  For this reason, carefully-crafted
% definitions of a coherent units system have been laid down by the
% \latin{Conf\'erence G\'en\'erale des Poids et
% Mesures}\footnote{General Conference on Weights and Measures.}
% (CGPM): this has resulted in the \latin{Syst\`eme International
% d'Unit\'es}\footnote{International System of Units.} (SI).  At the
% same time, typographic conventions for correctly displaying both
% numbers and units exist to ensure that no loss of meaning occurs in
% printed matter.
%
% \currpkg aims to provide a unified method for \LaTeX\ users to
% typeset units and values correctly and easily.  The design
% philosophy of \currpkg is to follow the agreed rules by default,
% but to allow variation through option settings.  In this way,
% users can use \currpkg to follow the requirements of publishers,
% co-authors, universities, \etc without needing to alter the input
% at all.
%
% \currpkg is intended as a complete replacement for \SIunits,
% \SIstyle, \unitsdef, \units, \pkg{fancyunits} and
% \pkg{fancynum}. As such, emulation modes are provided for all of
% these packages.  Where possible, conventions from the existing
% solutions have been used here.  For example, the macros \cs{num},
% \cs{ang} and \cs{SI} act in a very similar fashion to those in
% existing packages.
%
%\clearpage
%\part{Using the \currpkg package}
%
%\section{For the impatient}
% \currpkg provides the user macros:
%\begin{itemize}
%  \item \cs{SI}\oarg{options}\marg{value}\oarg{pre-unit}\marg{unit}
%  \item \cs{si}\oarg{options}\marg{unit}
%  \item \cs{num}\oarg{options}\marg{number}
%  \item \cs{ang}\oarg{options}\marg{angle}
%  \item \cs{sisetup}\marg{options}
%\end{itemize}
% plus the |S| and |s| column types for decimal alignments and units
% in tables.  These macros are designed for typesetting units and
% values with control of appearance and with intelligent processing.
%
% By default, all text is typeset in the current upright, serif maths
% font.  This can be changed by setting the appropriate package
% options: \opt{obeyall} will use the current font for typesetting.
%
% The package includes a ``unit processor'', which allows the use of
% named units or literal values.  Named units are processed to
% correctly include powers.
%\begin{LaTeXexample}
%  \SI{10}{\gram}\\
%  \SI{23.4}{g.cm^3}\\
%  \num{1e34}\\
%  \ang{1;2;3}\\
%  \emph{\SI{16,7}{\metre\per\second}}\\
%  \textbf{\SI{30e3}{\Hz}}\\
%  \SI{1.2 x 3.56 x 9.2}{\milli\metre}\\
%  \sisetup{obeyall}
%  \textbf{\SI{-4.5}{\cm}}\\
%  \si{\joule\per\mole\per\kelvin}\\
%  \si[per=frac]{\joule\per\mole\per\kelvin}\\
%  \num[dp=4]{1.23456}\\
%  \num[dp=4]{9.8}\\
%  \begin{tabular}{S[tabformat=3.2]}
%    \toprule
%    {Heading}\\
%    \midrule
%    1.3 \\
%    134.2 \\
%    3.56 \\
%    74,7 \\
%    \bottomrule
%  \end{tabular}
%\end{LaTeXexample}
%
%\section{Requirements}
% \currpkg requires a reasonably up to date \TeX\ system.  The
% package requires \eTeX-extensions, which should be available on
% most systems.\footnote{If you have an old \LaTeX\, try
% ``\texttt{elatex}'' rather than ``\texttt{latex}''.}  The following
% packages are also needed:
%\begin{itemize}
%  \item \pkg{array} and \pkg{xspace}: from the \pkg{tools} bundle,
%    which should be available to everyone;
%  \item \pkg{xkeyval}: this processes the option handling, and needs
%    to be at least v2.5;
%  \item \pkg{amstext}: from the \AMS\TeX\ support bundle (the
%    \AMS\ fonts are also needed to provide the default upright
%    \si{\micro}).
%\end{itemize}
% Hopefully most people using the package will have access to all of
% those items.
%
% To use the \opt{fraction=sfrac} option, the \pkg{xfrac} package is
% needed.  This needs various experimental \LaTeX3 packages.  As a
% result, \currpkg does not load \pkg{xfrac}.  If you want to use
% \opt{fraction=sfrac}, \emph{you} need to load \pkg{xfrac} in your
% preamble before \currpkg.\footnote{This document has been compiled
% in this way.  You have to load \pkg{xfrac} first as otherwise very
% nasty things happen with \pkg{xkeyval}.  MiK\TeX\ users should note
% that the packaged versions of \pkg{expl3}, \pkg{template} and
% \pkg{xparse} will not work with \pkg{xfrac}: download copies from
% \textsc{ctan}!} If the package is not loaded, \opt{fraction=sfrac}
% falls back on a \pkg{nicefrac}-like method.  The interested user
% should look at the \pkg{xfrac} documentation for reasons this might
% not be ideal.\footnote{On the other hand, some fractional units
% will look really bad with \cs{sfrac}.  Use this option with
% caution.}
%
%\section{Loading the package}
% \currpkg is loaded by the usual \LaTeX\ method.
%\begin{LaTeXexample}[noexample]
%  \usepackage[<options>]{siunitx}
%\end{LaTeXexample}
% As is shown in the example, the package can be loaded with one or
% more options, using the key--value system.  The full range of
% package options are described in Section~\ref{sec:options}; some
% options are described in the along with the appropriate user
% macros.  Most of the user macros accept the same key--value
% settings as an optional argument.
%
%\section{Numbers}
%\DescribeMacro{\num}
% Numbers are automatically formatted by the \cs{num} macro.  This
% takes one optional and one mandatory argument:
% \cs{num}\oarg{options}\marg{number}.  The contents of \meta{number}
% are automatically formatted, in a similar method to that used by
% \numprint.  The formatter removes ``hard'' spaces (\cs{,} and |~|),
% automatically identifies exponents (by default marked using |e| or
% |d|) and adds the appropriate spacing of large numbers.  A leading
% zero is added before a decimal marker, if needed: both ``|.|'' and
% ``|,|'' are recognised as decimal marker.
%\begin{LaTeXexample}
%  \num{1}    \num{123}     \num{1234}  \num{12345}\\
%  \num{0.1}  \num{0.123}   \num{0,1234} \num{.12345}\\
%  \num{1e10} \num{3.45d-4} \num{-e10}
%\end{LaTeXexample}
% Various error-checking systems are built into the package, so that
% if \meta{number} does not contain any numeric characters, a warning
% is issued.  Isolated signs are also detected.  The package
% recognises |(| and |)| as ``extra'' characters, which can be used
% to indicate the error in a number.\footnote{This is common in
% chemical crystallography, for example.}  The \opt{seperr} causes
% this data to be given as a separate error value.  If the number
% also contains an exponent, then brackets are re-added after the
% separation to ensure that meaning is not lost.
%\begin{LaTeXexample}
%  $\num{1.234(5)} = \num[seperr]{1.234(5)}$\\
%  $\num{1.234(5)e6} = \num[seperr]{1.234(5)e6}$
%\end{LaTeXexample}
% The same applies to the unit and value macro \cs{SI}, described
% later, for example the rest mass of an electron \cite{IUPAC}:
%\begin{LaTeXexample}
%   $ m_{\mathrm{e}}
%     = \SI{9.1093897(54)e-31}{\kg} $\\
%   $ m_{\mathrm{e}}
%     = \SI[seperr]{9.1093897(54)e-31}{\kg} $
%\end{LaTeXexample}
%
% A number of effects are available as options.  These are fully
% explained in Section~\ref{sec:options}.  Some of the more useful
% options are illustrated here.  By default, the output of the
% package is typeset in maths mode.  However, the use of the current
% text font can be forced.\footnote{This document is typeset using
% lowercase numbers in text mode, which emphasises the effect here.}
%\begin{LaTeXexample}
%  \num{1234567890} \num[mode=text]{1234567890}
%\end{LaTeXexample}
% \currpkg can automatically add zeros and signs to numbers.  This
% can be altered as desired.
%\begin{LaTeXexample}
%  \num{1.}  \num[padnumber=all]{1.}\\
%  \num{2}   \num[addsign=all]{2}\\
%  \num{3e4} \num[addsign=mant]{3e4} \num[addsign=all]{3e4}\\
%  \num{.5}  \num[padnumber=none]{.5}
%\end{LaTeXexample}
% The separation of digits can be turned on and off, and the output
% changed.
%\begin{LaTeXexample}
%  \num{1234}  \num[sepfour=true]{1234}\\
%  \num{12345} \num[digitsep=comma]{12345}\\
%  \num[digitsep=none]{12345}
%\end{LaTeXexample}
% The formatting of exponents is also customisable.
%\begin{LaTeXexample}
%  \num{1e10} \num[expproduct=cdot]{1e10}\\
%  \num{2e20} \num[expbase=5]{2e20}\\
%  \num{3e30} \num[expproduct=tighttimes]{3e30}
%\end{LaTeXexample}
%
% \currpkg can automatically add colour to negative numbers, which is
% often useful for highlighting purposes.  This is turned on with the
% \opt{colourneg} option; the colour used is set by \opt{negcolour}.
% Both of these are available with the US spellings: \opt{colorneg}
% and \opt{negcolor}.
%\begin{LaTeXexample}
%  \num{-1}\\
%  \sisetup{colourneg} \num{-2}\\
%  \num{3e-3}\\
%  \num[negcolour=blue]{-4}
%\end{LaTeXexample}
%
% \currpkg can automatically zero-fill and round to a fixed number of
% decimal places.  This is controlled by two options \opt{fixdp} and
% \opt{dp}.  The later is an integer which specifies how many places
% to fix to; setting this option automatically sets \opt{fixdp} to
% \opt{true}.  The place-fixing system will only alter pure numbers: for
% example, any error component will result in the input being left
% unchanged.
%\begin{LaTeXexample}
%  \num{1.23456}\\
%  \sisetup{dp=2}
%  \num{1.23456}\\
%  \num{9.8}\\
%  \num{-10.432}\\
%  \num{44.3221(2)}
%\end{LaTeXexample}
%
%\section{Tabular material}
%\subsection{Aligning numbers}
% Centring numbers in tabular content is handled by a new column
% type, the |S| column.  This is based closely on the \pkg{dcolumn}
% method for centring numbers in columns, but adds the functionality
% of the \cs{num} macro.\footnote{The approach used is actually a
% combination of \pkg{dcolumn} for centring the material and
% \numprint for processing it.  It will therefore give rather
% different results than the \texttt{n} and \texttt{N} column types
% in \numprint.}
%
% By default, the decimal marker of the number is placed at the
% centre of the column, which then resizes to accommodate the width
% of the contents (\ref{tab:default}).  This behaviour is set by the
% \opt{tabnumalign=centredecimal} option.  By setting the
% \opt{tabnumalign} option to \opt{centre}, the centre of the space
% reserved for the number is placed at the centre of the column.  The
% space reserved is stored in \opt{tabformat}, which is of the form
% \texttt{\meta{before}\meta{dec}\meta{after}}, where \meta{before}
% is the number of characters before the decimal marker and
% \meta{after} is the number after. Thus in the example,
% \opt{tabformat=2.4} provides space for two digits before the
% decimal marker and four after.  \opt{tabnumalign} can also be set
% to \opt{left} and \opt{right}, with the expected results.
%\begin{LaTeXexample}[float]
%  \begin{table}
%    \caption{Behaviour of \texttt{S} column type}
%    \label{tab:default}
%    \centering
%    \begin{tabular}{%
%      S%
%      S[tabnumalign=centre,tabformat=2.4]%
%      S[tabnumalign=right,tabformat=2.4]%
%      S[tabnumalign=centre,tabformat=2.4,decimalsymbol=comma]}
%    \toprule
%    {Some Values} & {Some Values} & {Some Values} & {Some Values} \\
%    \midrule
%     2.3456  &  2.3456  &  2.3456 &  2.3456 \\
%    34.2345  & 34.2345  & 34.2345 & 34.2345 \\
%    56.7835  & 56.7835  & 56.7835 & 56.7835 \\
%    90.473   & 90.473   & 90.473  & 90.473  \\
%    \bottomrule
%    \end{tabular}
%  \end{table}
%\end{LaTeXexample}
% The \opt{tabformat} setting can also be used to reserve space for
% numbers containing exponents.  This is given in the same format as
% above, but with a mantissa and exponent part (\ref{tab:exptab}).
% Notice that this is designed to expect that numbers will contain a
% mantissa.  Exponents can either be aligned so that the ``$\times$''
% symbols match up vertically, or the exponent part can be allowed to
% move across as needed. Space for signs is added by using any sign
% in the \opt{tabformat}, so for example \opt{tabformat=+2.2} and
% \opt{tabformat=-2.2} have exactly the same effect. Setting
% \opt{tabformat} will automatically switch \opt{tabnumalign} from
% \opt{centredecimal} to \opt{centre}, if the former is
% currently set. In other cases, the current alignment option is
% retained.
%\begin{LaTeXexample}[float]
%  \begin{table}
%    \caption{Exponents in tables}
%    \label{tab:exptab}
%    \centering
%    \begin{tabular}{%
%      S[tabnumalign=right,tabformat=2.2e2]%
%      S[tabnumalign=centre,tabformat=2.2e1.1]%
%      S[tabnumalign=centre,tabformat=2.2e1.1,tabalignexp=false]%
%      S[tabnumalign=centre,tabformat=+2.2]}
%    \toprule
%        {Longer values}
%      & {Longer values}
%      & {Longer values}
%      & {Values} \\
%    \midrule
%    2.3e1    &  2.34e1  &  2.34e1  & +2.31 \\
%    34.23e45 & 34.23e45 & 34.23e45 & 34.23 \\
%    56.78    & 56.78    & 56.78    & -56.78 \\
%    1.0e34   & 1.0e34   & 1.0e34   & +-1.0 \\
%    \bottomrule
%    \end{tabular}
%  \end{table}
%\end{LaTeXexample}
%
% Data not to be processed as a number should be protected by
% wrapping it in braces: this is most likely to be true for column
% headers (again as illustrated).  By default, the contents of
% non-numeric cells are centred.  This can be altered by setting
% \opt{tabtextalign}, which can be set to \opt{left}, \opt{right} or
% \opt{centre}.  The use of digit separators in table columns is
% accounted for: extra space is reserved if digit separators will be
% added.
%
%\subsection{Columns of units}
% As a complement to the |S| column, \currpkg also provides a second
% column type, |s|.  This is intended for producing columns of units.
% The letters chosen are intended to be similar to \cs{SI} and
% \cs{si}, respectively.  The alignment of material in |s| columns is
% governed by the \opt{tabunitalign} option.
%\begin{LaTeXexample}[float]
%  \begin{table}
%  \centering
%  \caption{Number and units in tables}
%  \label{tab:num-unit}
%    \begin{tabular}{%
%      S[tabformat=1.2e-1,tabnumalign=centre]%
%      s[tabunitalign=left]}
%    \toprule
%      {Value} & \multicolumn{1}{c}{Unit} \\
%    \midrule
%      2.16e-5 & \metre\squared\per\second \\
%      2.83e-6 & \metre\squared\per\second \\
%      7.39e3  & \pascal\cubic\metre\per\mole \\
%      1.0e5   & \pascal \\
%    \bottomrule
%    \end{tabular}
%  \end{table}
%\end{LaTeXexample}
%
% As the \cs{si} macro can take literal or macro input, the |s|
% column cannot validate the input.  \emph{Everything} in an |s|
% column is therefore passed to the \cs{si} macro for processing.  To
% prevent this, you have to use \cs{multicolumn}, as is shown in
% \ref{tab:s-limits}.  Notice that the braces do not prevent
% processing and colouring of the cell contents.
%\begin{LaTeXexample}[float]
%  \begin{table}
%  \centering
%  \caption{The \texttt{s} column processes everything}
%  \label{tab:s-limits}
%    \begin{tabular}{%
%      s[colourall,colour=orange]%
%      s[colourall,colour=orange]}
%    \toprule
%      {Unit}    & \multicolumn{1}{c}{Unit}\\
%    \midrule
%      {m^3}     & \multicolumn{1}{c}{\si{m^3}} \\
%      \kilogram & \kilogram \\
%    \bottomrule
%    \end{tabular}
%  \end{table}
%\end{LaTeXexample}
%
%\section{Angles}
%\DescribeMacro{\ang}
% Angles can be typeset using the \cs{ang} command.  This takes two
% arguments, \cs{ang}\oarg{options}\marg{angle}, where \meta{options}
% can be any of the package options to apply only to this value.
% \meta{angle} can be given either as a decimal number or as a
% semi-colon separated list of degrees, minutes and seconds, \ie\
% \cs{ang}\marg{decimal angle} or
% \cs{ang}|{|\meta{degrees}|;|\meta{minutes}|;|\meta{seconds}|}|. By
% default, no space is introduced between angles and the degrees,
% minutes and seconds markers.
%\begin{LaTeXexample}
%  \ang{10} \ang{12.3} \ang{4,5}\\
%  \ang{1;2;3} \ang{;;1}\\
%  \ang{+10;;} \ang{-0;1;}
%\end{LaTeXexample}
%
% By default, angles with no degrees (or minutes) are zero-filled;
% angles with degrees but no minutes or seconds are not filled.  This
% behaviour can be altered using the package options.
%\begin{LaTeXexample}
%  \ang{;;1} \ang[padangle=none]{;;1}\\
%  \ang{2;;} \ang[padangle=all]{2;;}\\
%  \sisetup{padangle=all} \ang{;3;} \ang{4;;} \ang{;;5}
%\end{LaTeXexample}
% The \cs{num} macro is used to typeset each number of the angle, so
% the options for \cs{num} also apply here.  The \opt{anglesep} value
% can be used to separate degrees, minutes and seconds.
%\begin{LaTeXexample}
%  \ang{1.05}    \ang[decimalsymbol=comma]{1.05}\\
%  \ang{3.67890} \ang[digitsep=comma]{3.67890}\\
%  \ang{9;8;7}   \ang[anglesep=thin]{9;8;7}
%\end{LaTeXexample}
% The degrees, minutes and seconds signs can be placed over the
% decimal sign using the \opt{astroang} option.  This is designed on
% the assumption that only the last number given has a decimal
% part.
%\begin{LaTeXexample}
%  \ang{1.2} \ang[astroang]{1.2}\\
%  \ang{1;2.3;} \ang[astroang]{1;2.3;}\\
%  \ang{1;2;3.4} \ang[astroang]{1;2;3.4}
%\end{LaTeXexample}
%
%\section{Units and values}
%\DescribeMacro{\SI}
% The core aim of \currpkg is correctly typesetting values which have
% units. The main output macro here is \cs{SI}, which has the same
% syntax as the macros with the same name in \SIstyle and \unitsdef
% packages. The \cs{SI} macro takes two mandatory arguments, in
% addition to the optional set up argument, and a second optional
% argument:
% \cs{SI}\oarg{options}\marg{number}\oarg{preunit}\marg{unit}.  The
% \meta{number} argument operates in exactly the same manner as the
% equivalent argument of the \cs{num} macro.  \meta{unit} will be
% typeset with a non-breakable space between it and the preceding
% number, with font control as outlined earlier.  Finally,
% \meta{preunit} is a unit to be typeset \emph{before} the numerical
% value (most likely to be a currency).  Some examples illustrate the
% general power of the macro.
%\begin{LaTeXexample}
%  \SI[mode=text]{1.23}{J.mol^{-1}.K^{-1}}\\
%  \SI{.23e7}{\candela}\\
%  \SI[per=slash]{1.99}[\pounds]{\per\kilogram}\\
%  \SI{70}{\metre\per\second}\\
%  \SI[per=frac,fraction=nice]{1,345}{\ampere\per\mole}
%\end{LaTeXexample}
% The use of unit macros outside of the \cs{SI} macro is described
% later.
%
%\subsection{Literal units}
%\changes{v1.1}{2008/07/23}{Unit processor extended to interpret
%  \texttt{\textunderscore} correctly}
% Units can be input in two ways, inspired by \SIstyle and \SIunits.
% The \SIstyle-like method uses literal input.  Five characters have
% a special meaning:
%\begin{itemize}
%  \item ``|^|'' and ``|_|''  The superscript and subscript
%    characters can be used without the usual
%    need for surrounding maths characters (|$|);^^A$
%  \item ``|.|'' and ``|,|'': the full stop (point) symbol and comma
%    are made active, and produce the current contents of the
%    \opt{unitsep} option;
%  \item ``|~|'' The contents of the \opt{unitspace} option are typeset
%    by a tilde.
%\end{itemize}
% This allows ready input of units.
%\begin{LaTeXexample}
%  \SI{10}{kg.m.s^{-2}}\\
%  \SI{1.453}{g/cm^3}\\
%  \SI{6.345}{kg_{pol}.mmol_{cat}^{-1}}\\
%  \SI{33.562}{cd~s}\\
%  \SI[unitsep=medium]{100}{m.s^{-2}}
%\end{LaTeXexample}
% The literal unit system will correctly typeset input containing the
% symbols \si{\micro} (micro), \si{\degree} (degree) and
% \si{\angstrom} (ring-A).\footnote{Currently this works with \XeTeX\
% and \pkg{inputenc} using the \texttt{latin1}, \texttt{latin5} and
% \texttt{latin9} encodings.}
% \begingroup\makeatletter\si@unt@nonlatin
%\begin{LaTeXexample}
%  \SI{10}{�m}\\
%  \SI{20}{�C}\\
%  \SI{30}{�}
%\end{LaTeXexample}
%\endgroup
%
%\subsection{The unit interpreter}
% The second operation mode for the \cs{SI} macro is based on the
% behaviour of \SIunits.  Here, each unit, SI multiple prefix and
% power is given a macro name.  These are entered in a method very
% similar to the reading of the unit name in English.
%\begin{LaTeXexample}
%  \SI{10}{\kilo\gram\metre\per\second\squared}\\
%  \SI{1.453}{\gram\per\cubic\centi\metre}\\
%  \SI{33.562}{\candela\second}\\
%  \SI[unitsep=medium]{100}{\metre\per\Square\second}\\
%  \SI[prefixsymbolic=false]{4.56}{\kilo\metre\per\second}
%\end{LaTeXexample}
% On its own, this is very similar to \SIunits, and is less
% convenient than the direct input method.\footnote{Users of \SIunits
% should note the lack of need for a \cs{usk}-type macro.}  However,
% the package allows you to define new unit macros; a large number of
% pre-defined abbreviations are also supplied.  More importantly, by
% defining macros for units, instead of literal values, new
% functionality is made available.  Units may be re-defined to give
% different output, and handling of reciprocal values can be altered.
%\begin{LaTeXexample}
%  \SI[per=frac,fraction=frac]{10}{\gram\metre\per\second\squared}\\
%  \SI[per=slash]{1.453}{\gram\per\cubic\centi\metre}\\
%  \SI{33.562}{\candela\second}\\
%  \SI[per=frac,fraction=nice]{100}{\metre\per\Square\second}
%\end{LaTeXexample}
% The unit processor will trap \emph{some} errors in the input and
% give the ``best guess'' result.  However, it is down to the user to
% check the output.
%
%\subsection{Powers of units}
%\DescribeMacro{\Square}
%\DescribeMacro{\ssquare}
%\DescribeMacro{\squared}
%\DescribeMacro{\cubic}
%\DescribeMacro{\cubed}
% Including powers in units is handled using a ``natural language''
% method.  Thus preceding a unit by \cs{Square} or \cs{cubic} which
% raise the unit to the appropriate power, while \cs{squared} or
% \cs{cubed} follow the unit they apply to. The \cs{Square} macro is
% capitalised to avoid a name clash with \pkg{pstricks}; the
% alternative \cs{ssquare} is also provided.
%\begin{LaTeXexample}
%  \SI{10}{\metre\squared}\\
%  \SI{20}{\Square\metre}\\
%  \SI{30}{\metre\cubed}\\
%  \SI{40}{\cubic\metre}
%\end{LaTeXexample}
%
%\DescribeMacro{\per}
% The \cs{per} macro intelligently creates reciprocal powers, and
% also adds the power $-1$ when appropriate.
%\begin{LaTeXexample}
%  \SI{10}{\per\second\squared}\\
%  \SI{20}{\per\Square\second}\\
%  \SI[per=frac,fraction=nice]{30}{\per\second\cubed}\\
%  \SI[per=slash]{40}{\per\cubic\second}\\
%  \SI{50}{\per\second}\\
%  \SI{60}{\per\metre\Square\candela}
%\end{LaTeXexample}
%
%\DescribeMacro{\tothe}
%\DescribeMacro{\raiseto}
% For powers not defined above or with \cs{newpower}, the \cs{tothe}
% macro can be used ``in line'' to produce a power.  As follows from
% standard English usage, this comes after the unit. \cs{raiseto}
% achieves the same, but is used \emph{before} a unit to add a power
% \emph{after}.\footnote{\cs{raiseto} acts in the same way as
% \cs{tothe} when used in a literal context: the power will be
% produced where the macro is, rather than moving after the next
% item.}
%\begin{LaTeXexample}
%  \SI{16.86}{\metre\tothe{4}}\\
%  \SI{7.895}{\raiseto{-6}\newton}\\
%  \SI{1.34}{\per\kelvin\tothe{7}}
%\end{LaTeXexample}
%
%\subsection{Units with no values}
%\DescribeMacro{\si}
% For typesetting the symbol for a unit on its own, with the full
% font control and without extra spaces, the \cs{si} macro is
% provided.\footnote{The same effect can be achieved using the
% \cs{SI} macro with an empty numerical argument.}  The macro name
% avoids a clash with the functionality of the earlier packages, but
% is similar to \cs{ilu} from the \unitsdef package.
%\begin{LaTeXexample}
%  \SI{}{kg.m/s^2}\\
%  \si{kg.m/s^2}\\
%  \si[mode=text,unitsep=thin]{\mole\per\cubic\deci\metre}
%\end{LaTeXexample}
%
%\subsection{Free-standing units}
% Users of the \unitsdef package will be accustomed to using unit
% macros on their own (following a value) or with an optional
% argument containing a number.  In both cases, only a single unit
% macro could be used. \currpkg supports both operation modes, with
% the limitation that units trailing values loose font control of the
% value.  When used in this way, the units \emph{do not} take an
% optional \pkg{keyval} argument.
%\begin{LaTeXexample}
%  \sisetup{prespace,allowoptarg}
%  123\metre\\
%  \kelvin[123]\\
%  \sisetup{mode=text} \ampere[234]\\
%  6\second
%\end{LaTeXexample}
%
%\subsection{Pre-defined units, prefixes and powers}
%\newcommand*{\csindex}[1]{^^A
%  \cs{#1}^^A
%  \expandafter\SpecialUsageIndex\csname#1\endcsname}
%\newcommand*{\unitinfo}[1]{^^A
%  #1 & \csindex{#1} &
%  \expandafter\si\expandafter{\csname#1\endcsname}}
%\begin{table}
%  \caption{The seven base SI units}
%  \label{tab:baseunits}
%  \centering
%  \begin{tabular}{lll}
%    \toprule
%      \multicolumn{1}{c}{Unit} &
%      \multicolumn{1}{c}{Macro} &
%      \multicolumn{1}{c}{Symbol} \\
%    \midrule
%    \unitinfo{kilogram}\\
%    \unitinfo{metre}\\
%    \unitinfo{second}\\
%    \unitinfo{mole}\\
%    \unitinfo{kelvin}\\
%    \unitinfo{ampere}\\
%    \unitinfo{candela}\\
%    \bottomrule
%  \end{tabular}
%\end{table}
%\DescribeMacro{\metre}
%\DescribeMacro{\meter}
% The package always defines the seven base SI units, irrespective of
% any package options given (\ref{tab:baseunits}). The kilogram is
% notable as by default it is a \emph{base} unit with a prefix. Thus,
% when the package is loaded with the option \opt{load=\{\}},
% \cs{kilo} and \cs{gram} \emph{are not defined}.  As metre is often
% spelled as ``meter'' in the US, the macro \cs{meter} is provided in
% addition to the \cs{metre} macro.\footnote{The official SI spelling
% for the unit is ``metre''.}
%
% By default, a number of additional definitions are created by the
% package.  These are controlled by the \opt{load} and \opt{noload}
% options. Unless specifically requested with the option
% \opt{noload=prefix}, \currpkg defines the standard prefixes for
% powers of ten (\ref{tab:prefix}).  This leads to the redefinition
% of \cs{kilogram} as \cs{kilo}\cs{gram}.  The macro \cs{deka} is
% provided, as this is used as an alias for \cs{deca} in some places.
% The package also defines a number of derived SI units which have
% assigned names and symbols (\ref{tab:derived}). Note that \cs{Gray}
% is capitalised to avoid a name clash with the \pkg{pstricks}
% package.\footnote{The macros \cs{ohm} and \cs{celsius} are not
% defined by \currpkg if the \pkg{gensymb} package is loaded.}
%\newcommand*{\prefixunitsym}[1]{\si[prefixsymbolic=false]{#1}}
%\newcommand*{\prefixinfo}[1]{^^A
%  #1 & \csindex{#1} &
%    \expandafter\prefixunitsym\expandafter{\csname#1\endcsname} &
%    \expandafter\si\expandafter{\csname#1\endcsname}}
%\begin{table}
%  \caption{The SI prefixes (\opt{load=prefix})}
%  \label{tab:prefix}
%  \centering
%  \begin{tabular}{lllclllc}
%    \toprule
%      \multicolumn{1}{c}{Prefix} &
%      \multicolumn{1}{c}{Macro} &
%      \multicolumn{1}{c}{Power} &
%      Symbol &
%      \multicolumn{1}{c}{Prefix} &
%      \multicolumn{1}{c}{Macro} &
%      \multicolumn{1}{c}{Power} &
%      Symbol \\
%    \midrule
%    \prefixinfo{yocto} &
%    \prefixinfo{deca} \\
%    \prefixinfo{zepto} &
%    \prefixinfo{hecto} \\
%    \prefixinfo{atto} &
%    \prefixinfo{kilo} \\
%    \prefixinfo{femto} &
%    \prefixinfo{mega} \\
%    \prefixinfo{pico} &
%    \prefixinfo{giga} \\
%    \prefixinfo{nano} &
%    \prefixinfo{tera} \\
%    \prefixinfo{micro} &
%    \prefixinfo{peta} \\
%    \prefixinfo{milli} &
%    \prefixinfo{exa} \\
%    \prefixinfo{centi} &
%    \prefixinfo{zetta} \\
%    \prefixinfo{deci} &
%    \prefixinfo{yotta} \\
%    \bottomrule
%  \end{tabular}
%\end{table}
%\begin{table}
%  \caption{The derived SI units with defined names
%    (\opt{load=derived})}
%  \label{tab:derived}
%  \centering
%  \begin{tabular}{llllll}
%    \toprule
%      \multicolumn{1}{c}{Unit} &
%      \multicolumn{1}{c}{Macro} &
%      \multicolumn{1}{c}{Symbol} &
%      \multicolumn{1}{c}{Unit} &
%      \multicolumn{1}{c}{Macro} &
%      \multicolumn{1}{c}{Symbol} \\
%    \midrule
%    \unitinfo{becquerel} &
%    \unitinfo{newton} \\
%    \unitinfo{celsius} &
%    \unitinfo{ohm} \\
%    \unitinfo{coulomb} &
%    \unitinfo{pascal} \\
%    \unitinfo{farad} &
%    \unitinfo{radian} \\
%    \unitinfo{Gray} &
%    \unitinfo{siemens} \\
%      & \csindex{ggray} & \ggray &
%    \unitinfo{sievert} \\
%    \unitinfo{hertz} &
%    \unitinfo{steradian} \\
%    \unitinfo{henry} &
%    \unitinfo{tesla} \\
%    \unitinfo{joule} &
%    \unitinfo{volt} \\
%    \unitinfo{katal} &
%    \unitinfo{watt} \\
%    \unitinfo{lumen} &
%    \unitinfo{weber} \\
%    \unitinfo{lux} \\
%    \bottomrule
%  \end{tabular}
%\end{table}
%
% In addition to these units, there are three other groups of units
% for use with the SI system which do not fit into the above.  These
% are those derived from physical measurements
% (\ref{tab:experimental}), those considered ``accepted''
% (\ref{tab:accepted}), and those accepted temporarily
% (\ref{tab:addn}).\footnote{These are supposed to be replaced over
% time by SI units.}
%\DescribeMacro{\litre}
%\DescribeMacro{\liter}
% The unit ``litre'' is often spelled ``liter'' in the US; both
% spellings are provided by \currpkg, with \cs{liter} giving
% \si{\liter} and \cs{litre} producing \si{\litre}.
%\begin{table}
%  \caption{Units derived from experiments
%    (\opt{load=physical})}
%  \label{tab:experimental}
%  \centering
%  \begin{tabular}{lll}
%    \toprule
%      \multicolumn{1}{c}{Unit} &
%      \multicolumn{1}{c}{Macro} &
%      \multicolumn{1}{c}{Symbol} \\
%    \midrule
%    electron volt & \csindex{electronvolt} & \electronvolt \\
%    unified atomic mass unit
%      & \csindex{atomicmassunit} & \atomicmassunit \\
%      & \csindex{atomicmass} & \atomicmass \\
%    \bottomrule
%  \end{tabular}
%\end{table}
%\begin{table}
%  \caption{Units accepted for use with SI
%    (\opt{load=accepted})}
%  \label{tab:accepted}
%  \centering
%  \begin{tabular}{lll}
%    \toprule
%      \multicolumn{1}{c}{Unit} &
%      \multicolumn{1}{c}{Macro} &
%      \multicolumn{1}{c}{Symbol} \\
%    \midrule
%    \unitinfo{bel} \\
%    day & \csindex{Day} & \Day \\
%     & \csindex{dday} & \dday \\
%    \unitinfo{degree} \\
%    \unitinfo{hour} \\
%    \unitinfo{litre} \\
%     & \csindex{liter} & \liter \\
%    \unitinfo{minute} \\
%    minute (arc) & \csindex{arcmin} & \arcmin \\
%    \unitinfo{neper} \\
%    \unitinfo{percent} \\
%    second (arc) & \csindex{arcsec} & \arcsec \\
%    \unitinfo{tonne} \\
%    \bottomrule
%  \end{tabular}
%\end{table}
%\begin{table}
%  \centering
%  \caption{Additional (temporary) SI units
%    (\opt{load=addn})}
%  \label{tab:addn}
%  \begin{tabular}{lll}
%    \toprule
%      \multicolumn{1}{c}{Unit} &
%      \multicolumn{1}{c}{Macro} &
%      \multicolumn{1}{c}{Symbol} \\
%    \midrule
%    {\aa}ngstr{\"o}m & \csindex{angstrom} & \angstrom \\
%    \unitinfo{are} \\
%    \unitinfo{curie} \\
%    bar & \csindex{BAR} & \BAR \\
%     & \csindex{bbar} & \bbar \\
%    \unitinfo{barn} \\
%    \unitinfo{gal} \\
%    \unitinfo{hectare} \\
%    \unitinfo{millibar} \\
%    \unitinfo{rad} \\
%    \unitinfo{rem} \\
%    \unitinfo{roentgen} \\
%    \bottomrule
%  \end{tabular}
%\end{table}
%
%\subsection{Prefixed and abbreviated units}
% Many basic units have prefixes which are commonly used with the
% unit, such as centimetre or megahertz.  The package therefore
% defines a number of common prefixed units (\opt{load=prefixed}).
% Several of these also have obvious abbreviations (such as \cs{MHz}
% for \cs{megahertz}), which are made available by \opt{load=abbr}.
% In common with the units discussed above, the prefixed and
% abbreviated unit definitions are loaded by default.
%\newcommand*{\tablesubhead}[1]{^^A
%  \addlinespace \emph{#1} \\*}
%\begin{longtable}{llll}
%    \caption{Prefixed (\opt{load=prefixed}) and abbreviated
%      (\opt{load=abbr}) units\label{tab:abbr}}\\
%    \toprule
%    Unit & Macro & Symbol & Abbreviation \\
%    \midrule
%  \endfirsthead
%    \toprule
%    Unit & Macro & Symbol & Abbreviation \\
%    \midrule
%  \endhead
%    \bottomrule
%    \multicolumn{4}{r}{\emph{Continued on next page}}
%  \endfoot
%    \bottomrule
%  \endlastfoot
%
%  \tablesubhead{Masses}
%  \unitinfo{kilogram} & \csindex{kg} \\
%  \unitinfo{femtogram} & \csindex{fg} \\
%  \unitinfo{picogram} & \csindex{pg} \\
%  \unitinfo{nanogram} & \csindex{nanog} \\
%  \unitinfo{microgram} & \csindex{micg} \\
%  \unitinfo{milligram} & \csindex{mg} \\
%  atomic mass & \csindex{atomicmass} & \amu & \csindex{amu} \\
%
%  \tablesubhead{Lengths}
%  \unitinfo{picometre} & \csindex{picom} \\
%  \unitinfo{nanometre} & \csindex{nm} \\
%  \unitinfo{micrometre} & \csindex{micm} \\
%  \unitinfo{millimetre} & \csindex{mm} \\
%  \unitinfo{centimetre} & \csindex{cm} \\
%  \unitinfo{decimetre} & \csindex{dm} \\
%  \unitinfo{kilometre} & \csindex{km} \\
%
%  \tablesubhead{Times}
%  \unitinfo{second} & \csindex{Sec} \\
%  \unitinfo{attosecond} & \csindex{as} \\
%  \unitinfo{femtosecond} & \csindex{fs} \\
%  \unitinfo{picosecond} & \csindex{ps} \\
%  \unitinfo{nanosecond} & \csindex{ns} \\
%  \unitinfo{microsecond} & \csindex{mics} \\
%  \unitinfo{millisecond} & \csindex{ms} \\
%
%  \tablesubhead{Moles}
%  \unitinfo{femtomole} & \csindex{fmol} \\
%  \unitinfo{picomole} & \csindex{pmol} \\
%  \unitinfo{nanomole} & \csindex{nmol} \\
%  \unitinfo{micromole} & \csindex{micmol} \\
%  \unitinfo{millimole} & \csindex{mmol} \\
%  \unitinfo{kilomole} & \csindex{kmol} \\
%
%  \tablesubhead{Currents}
%  \unitinfo{picoampere} & \csindex{pA} \\
%  \unitinfo{nanoampere} & \csindex{nA} \\
%  \unitinfo{microampere} & \csindex{micA} \\
%  \unitinfo{milliampere} & \csindex{mA} \\
%  \unitinfo{kiloampere} & \csindex{kA} \\
%
%  \tablesubhead{Areas}
%  square centimetre & \csindex{squarecentimetre} &
%    \squarecentimetre & \csindex{cms} \\*
%    & \csindex{centimetresquared} & \centimetresquared \\
%  square metre & \csindex{squaremetre} & \squaremetre \\
%  square kilometre & \csindex{squarekilometre} & \squarekilometre \\
%
%  \tablesubhead{Volumes}
%\changes{v1.1d}{2008/10/29}{New units \cs{microliter}, \cs{milliliter},
%  \cs{micL} and \cs{mL}}
%  \unitinfo{microlitre} & \csindex{micl} \\
%  \unitinfo{millilitre} & \csindex{ml} \\
%  \unitinfo{microliter} & \csindex{micL} \\
%  \unitinfo{milliliter} & \csindex{mL} \\
%  cubic centimetre & \csindex{cubiccentimetre} &
%    \cubiccentimetre & \csindex{cmc} \\*
%    & \csindex{centimetrecubed} & \centimetrecubed \\
%  cubic decimetre & \csindex{cubicdecimetre} &
%    \cubicdecimetre & \csindex{dmc} \\
%
%  \tablesubhead{Frequencies}
%  \unitinfo{hertz} & \csindex{Hz} \\
%  \unitinfo{millihertz} & \csindex{mHz} \\
%  \unitinfo{kilohertz} & \csindex{kHz} \\
%  \unitinfo{megahertz} & \csindex{MHz} \\
%  \unitinfo{gigahertz} & \csindex{GHz} \\
%  \unitinfo{terahertz} & \csindex{THz} \\
%
%  \tablesubhead{Potentials}
%  \unitinfo{millivolt} & \csindex{mV} \\
%  \unitinfo{kilovolt} & \csindex{kV} \\
%
%  \tablesubhead{Energies}
%  \unitinfo{kilojoule}  & \csindex{kJ}\\
%  \unitinfo{electronvolt} & \csindex{eV} \\
%  \unitinfo{millielectronvolt} & \csindex{meV} \\
%  \unitinfo{kiloelectronvolt} & \csindex{keV} \\
%  \unitinfo{megaelectronvolt} & \csindex{MeV} \\
%  \unitinfo{gigaelectronvolt} & \csindex{GeV} \\
%  \unitinfo{teraelectronvolt} & \csindex{TeV} \\
%  \unitinfo{kilowatthour} & \cs{kWh} \\
%
%  \tablesubhead{Powers}
%  \unitinfo{milliwatt} \\
%  \unitinfo{kilowatt} \\
%  \unitinfo{megawatt} \\
%
%  \tablesubhead{Capacitances}
%  \unitinfo{femtofarad} \\
%  \unitinfo{picofarad} \\
%  \unitinfo{nanofarad} \\
%  \unitinfo{microfarad} \\
%  \unitinfo{millifarad} \\
%
%  \tablesubhead{Resistances}
%  \unitinfo{kilohm} \\
%  \unitinfo{megohm} \\
%  \unitinfo{gigaohm} \\
%  \unitinfo{millisiemens} \\
%
%  \tablesubhead{Forces}
%  \unitinfo{millinewton} \\
%  \unitinfo{kilonewton} \\
%
%  \tablesubhead{Other units}
%  \unitinfo{hectopascal} \\
%  \unitinfo{megabecquerel} \\
%  \unitinfo{millisievert} \\
%\end{longtable}
%
%\subsection{Defining new units}
%\DescribeMacro{\newunit}
%\DescribeMacro{\renewunit}
%\DescribeMacro{\provideunit}
% New units are produced using the \cs{newunit} macro.  This works as
% might be expected:
% \cs{newunit}\oarg{options}\marg{unit}\marg{symbol}, where
% \meta{symbol} can contain literal values, other units, multiple
% prefixes, powers and \cs{per}.  The \meta{options} argument can be
% any suitable options, and applies the specific unit macro only. The
% most obvious example for using this macro is the \cs{degree}
% unit.\footnote{Although the \cs{ang} macro is preferred for this
% job.}  The (first) optional argument to \cs{SI} and \cs{si} can be
% used to override the settings for the unit.  The \cs{renewunit} and
% \cs{provideunit} macros take the same arguments.
%\begin{LaTeXexample}
%  \SI{3.1415}{\degree}\\
%  \newunit[valuesep=none]{\oddunit}{XXX}
%  \SI{12345}{\oddunit}
%  \SI[valuesep=thick]{67890}{\oddunit}
%\end{LaTeXexample}
%
% As with the \LaTeX\ commands \cs{newcommand}, \etc, the choice of
% \cs{newunit}, \cs{renewunit} or \cs{provideunit} depends on the
% presence of an existing definition.  While \cs{newunit} should be
% used when a unit has not been previously defined, \cs{renewunit}
% will issue a warning if the named unit does not already exist.
% \cs{provideunit} defines the unit if it does not exist, and
% otherwise does nothing at all.  The same behaviour is seen with
% \cs{providepower} and \cs{provideprefix} (\latin{vide infra}).
%
% Output that is only valid in maths mode requires \cs{ensuremath},
% text-only input requires \cs{text}.  In the example below,
% \cs{mathnormal} is used to force the font choice only for the
% single character.\footnote{The \cs{mathrm} font used for this
% document has an ``\ss'' at the $\pi$ position.}
%\begin{LaTeXexample}
%  \newunit{\SIpi}{\ensuremath{\mathnormal{\pi}}}
%  \SI{10}{\metre\per\SIpi\squared}
%\end{LaTeXexample}
%
%\DescribeMacro{\newpower}
%\DescribeMacro{\renewpower}
%\DescribeMacro{\providepower}
% Powers are defined:
% \cs{newpower}\ofixarg{post}\marg{power}\marg{num}. Here,
% \meta{power} is the name of the power macro and \meta{num} is the
% (positive) number it represents.  The later argument is always
% processed internally by \cs{num}, but \emph{must} be a number.
% Giving the optional argument \opt{post} indicates to the package
% that the power will come after the unit it applies to; by default
% it is assumed that it will come before.
%\begin{LaTeXexample}
%  \newpower{\quartic}{4}
%  \newpower[post]{\totheforth}{4}\\
%  \si{\kilogram\totheforth}\\
%  \si{\quartic\metre}
%\end{LaTeXexample}
%
%\DescribeMacro{\newprefix}
%\DescribeMacro{\renewprefix}
%\DescribeMacro{\provideprefix}
% The standard SI powers of ten are defined by the package, and are
% described above.  However, the user can define new prefixes with
% \cs{newprefix}.  This has syntax
% \cs{newunit}\ofixarg{binary}\marg{prefix}\marg{symbol}\marg{powers-ten},
% where \meta{powers-ten} is the number of powers of ten the prefix
% represents.  When the \opt{binary} option is given, the prefix is a
% power of two.  For example, \cs{kilo} and \cs{kibi} are defined:
%\begin{LaTeXexample}[noexample]
%  \newprefix{\kilo}{k}{3}
%  \newprefix[binary]{\kibi}{Ki}{10}
%\end{LaTeXexample}
%
%\changes{v1.1}{2008/09/22}{Qualifiers introduced}
%\DescribeMacro{\newqualifier}
%\DescribeMacro{\renewqualifier}
%\DescribeMacro{\providequalifier}
% It is possible to create unit ``qualifiers'', which add subscript
% descriptions to units.  Although not encouraged, this is sometimes
% necessary to make the meaning clear.  No qualifiers are pre-defined
% by \currpkg, and so the user needs to declare them using
% \cs{newqualifier}. Currently, only subscript qualifiers can be
% created; a more extended set of options is planned for the next
% release of \currpkg.
%\begin{LaTeXexample}
%  \newqualifier{\polymer}{pol}
%  \newqualifier{\catalyst}{cat}
%  \si{\kg\polymer\per\mole\catalyst\per\hour}
%\end{LaTeXexample}
%
%\section{Specialist units}
% In some subject area, there are units which are in common use even
% though they are outside of the SI system.  Unlike the units
% discussed earlier, these specialist units are not loaded by
% default.  In each case, they should be requested with the option
% \opt{alsoload=\meta{name}}.
%
%\subsection{Binary units (\opt{binary})\label{sec:binary}}
%\DescribeMacro{\bit}
%\DescribeMacro{\byte}
% The binary prefixes, \cs{bit}, \cs{byte} (\ref{tab:binary}) are not
% formally part of the SI system. They are available by giving the
% \opt{alsoload=binary} option.
%\begin{LaTeXexample}
%  \SI{100}{\mebi\byte}
%\end{LaTeXexample}
%\begin{table}
%  \caption{Binary prefixes
%    (\opt{alsoload=binary})}
%  \label{tab:binary}
%  \centering
%  \begin{tabular}{lllc}
%    \toprule
%      \multicolumn{1}{c}{Prefix} &
%      \multicolumn{1}{c}{Macro} &
%      \multicolumn{1}{c}{Power} &
%      Symbol \\
%    \midrule
%      \prefixinfo{kibi} \\
%      \prefixinfo{mebi} \\
%      \prefixinfo{gibi} \\
%      \prefixinfo{tebi} \\
%      \prefixinfo{pebi} \\
%      \prefixinfo{exbi} \\
%    \bottomrule
%  \end{tabular}
%\end{table}
%
%\subsection{Synthetic chemistry (\opt{synchem})}
%\DescribeMacro{\mmHg}
%\DescribeMacro{\molar}
%\DescribeMacro{\Molar}
%\DescribeMacro{\torr}
%\DescribeMacro{\dalton}
% The \opt{synchem} file adds the common chemistry units \cs{mmHg},
% \cs{molar}, \cs{Molar}, \cs{torr} and \cs{dalton} to \currpkg.  The
% \cs{Molar} macro is somewhat awkward, as it can be given as either
% ``\textsc{m}'' or ``M''.  The later is obviously easily confused
% with the sign for the prefix mega.  By default, \currpkg uses the
% UK default of a small-caps symbol.  The \cs{dalton} unit is defined
% here as this name is not recognised by the various international
% bodies: the symbol \si{\amu} is preferred.
%\begin{LaTeXexample}
%  \SI{1}{\Molar} HCl\\
%  \SI{760}{\torr}\\
%  \SI{0.01}{\mmHg}\\
%  \SI{3.0}{\molar}\\
%  \SI{106.42}{\dalton}
%\end{LaTeXexample}
%
%\subsection{High-energy physics (\opt{hep})}
% In contrast to \pkg{hepunits}, \currpkg does not define a long list
% of compound units for high-energy physics.\footnote{Using the
% \opt{emulate=hepunits} option will load a file defining those.}
% Instead, a small selection of new units are defined
% (\ref{tab:hep}).  The mechanisms provided by \currpkg should avoid
% the need for large numbers of abbreviations.  For example, the
% \pkg{hepunits} \cs{MinveV} can be given as |\per\MeV| in \currpkg,
% which requires only one more character.
%\begin{table}
%  \caption{High-energy physics units (\opt{alsoload=hep})}
%  \label{tab:hep}
%  \centering
%  \begin{tabular}{lllc}
%    \toprule
%    Unit & Macro & Symbol & Abbreviation \\
%    \midrule
%    \addlinespace \emph{Areas} \\
%    \unitinfo{yoctobarn} & \csindex{yb} \\
%    \unitinfo{zeptobarn} & \csindex{zb} \\
%    \unitinfo{attobarn} & \csindex{ab} \\
%    \unitinfo{femtobarn} & \csindex{fb} \\
%    \unitinfo{picobarn} & \csindex{pb} \\
%    \unitinfo{nanobarn} & \csindex{nb} \\
%    \addlinespace \emph{Other units} \\
%    \unitinfo{micron} \\
%    millirad & \csindex{mrad} & \mrad \\
%    \unitinfo{gauss} \\
%    \bottomrule
%  \end{tabular}
%\end{table}
%
%\DescribeMacro{\clight}
%\DescribeMacro{\eVperc}
% The \opt{hep} option defines two units which are slightly unusual.
% \cs{clight} gives \clight, which is recognised as a unit when used
% in the appropriate circumstances. The second unit provided is
% \cs{eVperc}, which is commonly-used and clear enough for a compound
% definition. Notice that the value of \opt{eVcorrb} will need to be
% adjusted when using this unit.
%\begin{LaTeXexample}
%  \SI[per=slash,eVcorrb=0.4ex]{4.657}{\mega\eVperc\squared}
%\end{LaTeXexample}
%
%\subsection{Astronomy (\opt{astro})}
%\DescribeMacro{\parsec}
%\DescribeMacro{\lightyear}
% For astronomers, the \cs{parsec} and \cs{lightyear} units are
% available, and give the obvious results.
%\begin{LaTeXexample}
%  \SI{12}{\parsec}\\
%  \SI{1}{\lightyear}
%\end{LaTeXexample}
%
%
%\subsection{Geophysics (\opt{geophys})}
%\changes{v1.1j}{2008/12/22}{New module \opt{geophys} containing \cs{gon}
%  unit}
%\DescribeMacro{\gon}
% For geophysics, the unit \cs{gon} is available.
%\begin{LaTeXexample}
%  \SI{12}{\gon}
%\end{LaTeXexample}
%
%\section{Font control}
% Following the lead of \SIstyle, \currpkg provides control over the
% font used to typeset output.  By default, all text is typeset using
% the current upright serif maths font, whether the macros are given
% in text or maths mode.  Some examples will show the effect.
%\begin{LaTeXexample}
%  \num{10} $\num{10}$\\
%  \sffamily \ang{20} $\ang{20}$\\
%  \textbf{\SI{30}{\kilo\gram}}\\
%  \boldmath $\SI{40}{\kilo\gram}$
%  \[ \num{50} \]
%\end{LaTeXexample}
%
% In contrast, by setting \opt{obeyall}, the current font is used:
% this may be maths or text, depending on the context.
%\begin{LaTeXexample}
%  \sisetup{obeyall}\\
%  \ang{1;1;1} $\ang{1;1;1}$\\
%  \sffamily \ang{2;2;2} $\ang{2;2;2}$\\
%  \textbf{\ang{3;3;3}} \boldmath $\ang{3;3;3}$\\
%  \emph{\ang{4;4;4}} \emph{$\ang{4;4;4}$}
%  \[ \ang{5;5;5} \]
%\end{LaTeXexample}
%
% Fine control of which elements of the local font are used is
% available with the \opt{obeyfamily}, \opt{obeybold},
% \opt{obeyitalic} and \opt{obeymode} options.
%\begin{LaTeXexample}
%  \sisetup{obeyfamily}\\
%  \ang{1;1;1} $\ang{1;1;1}$\\
%  \sffamily \ang{2;2;2} $\ang{2;2;2}$\\
%  \sisetup{obeybold}
%  \textbf{\ang{3;3;3}} \boldmath $\ang{3;3;3}$\\
%  \emph{\ang{4;4;4}} \emph{$\ang{4;4;4}$}
%  \[ \ang{5;5;5} \]
%\end{LaTeXexample}
%
%\section{Package options\label{sec:options}}
%\DescribeMacro{\sisetup}
% The ``native'' options for the package are all given using the
% key--value method. Most of the package options can be given both
% when loading the package and at any point in the document.  This is
% achieved using the \cs{sisetup} macro.
%
% The package options take a number of different forms.
%\begin{itemize}
%  \item \opt{option=\meta{bool}} Simple true/false values.  These
%    macros all default to \opt{option=true}, meaning that giving the
%    option name along will set the appropriate flag.
%  \item \opt{option=\meta{choice}} Take a single item from a
%    pre-determined list. Depending on the value, one or more
%    internal states will be altered.  Values not on the list are
%    ignored (with a warning).
%  \item \opt{option=\meta{choice,literal}} If the given value is a
%    \meta{choice}, then the internal settings for that choice are
%    used.  Any other value is used directly.
%  \item \opt{option=\meta{literal}} The given value is used as a
%    literal by the package.
%  \item \opt{option=\meta{csname}} These options expect a command
%    sequence as a value.
%  \item \opt{option=\meta{length}} Requires a \TeX\ length, for
%    example \opt{0.5ex}.
%  \item \opt{option=\meta{list}} Takes a list of one or more items,
%    which are not determined in advance.
%  \item \opt{option=\meta{number}} Takes a number (possibly
%    including an exponent part).
%\end{itemize}
%
% The package has a large range of options, to allow full control of
% the various features of the package.  These control differing
% aspects of the package, and are given below in groups based on
% function.  Where the key has a default value, it is given in bold.
%
%\subsection{Font family and style}
% The font used when typesetting material can be tightly controlled
% using \currpkg.  A number of options affect how the package matches
% the surrounding font, and the font families used to achieve this.
% The default is to use the current upright maths serif font with no
% variation.
%
%\DescribeOption{mode}
%\DescribeOption{textmode}
%\DescribeOption{valuemode}
%\DescribeOption{unitmode}
%\DescribeOption{obeymode}
% The output of \currpkg can occur using either text or maths mode.
% The package option \opt{mode} determines which is used: valid
% options are \defaultopt{maths} and \opt{text}.\footnote{Here and in
% all other cases, either UK or US spelling may be used.  Thus
% \opt{mode=maths} or \opt{mode=math} have exactly the same effect.}
% The shortcut \opt{textmode} is provided for setting \opt{mode=text}
% quickly. Further refinement is possible using the \opt{valuemode}
% and \opt{unitmode} options.  These apply to numbers (the output of
% \cs{num} and the first mandatory argument of \cs{SI}) and units
% (all other output), respectively. By setting the \opt{obeymode}
% flag, the package will use the local typesetting mode (maths or
% text).
%
%\DescribeOption{obeyall}
%\DescribeOption{obeyfamily}
%\DescribeOption{obeybold}
%\DescribeOption{obeyitalic}
% The detection and matching of surrounding text can be controlled
% using a number of Boolean package options. \opt{obeyall} turns on
% all of the detection.  Thus output with \opt{obeyall} in force will
% always match the local text appearance.  \opt{obeyfamily} instructs
% the package on detecting the surrounding font family (Roman, sans
% serif, fixed width), but does not detect bold or italic.
% \opt{obeybold} detects the local bold setting, whilst
% \opt{obeyitalic} picks up italic fonts.
%
% Bold detection is influenced by the value of \opt{inlinebold},
% which takes values \defaultopt{text} and \opt{maths}.  The package
% can detect the local value of bold for either the surrounding text,
% or the surrounding inline (|$|\ldots|$|) maths.  The
% \opt{obeyitalic} option does \emph{not} have the same facility
% (maths is italic anyway).
%
%\DescribeOption{mathsrm}
%\DescribeOption{mathssf}
%\DescribeOption{mathstt}
%\SpecialOptionIndex{mathrm}
%\SpecialOptionIndex{mathsf}
%\SpecialOptionIndex{mathtt}
%\DescribeOption{textrm}
%\DescribeOption{textsf}
%\DescribeOption{texttt}
% The font commands used by the package to achieve the above are all
% available for user modification.  The options \opt{mathsrm},
% \opt{mathssf} and \opt{mathstt} hold the command sequences used in
% maths mode,\footnote{These can also be set using \opt{mathrm},
% \opt{mathsf} and \opt{mathtt}} while \opt{textrm}, \opt{textsf} and
% \opt{texttt} do the same for text mode.  By default, these contain
% the obvious command names, for example \opt{mathsrm=mathrm} and
% \opt{texttt=ttfamily}.  However, they can be set at will: the macro
% names indicate the nature of the surrounding text detected.  For
% example, the value of \opt{mathssf} is used in maths mode when the
% surrounding text is sans serif.
%
%\SpecialOptionIndex{valuemathsrm}
%\SpecialOptionIndex{valuemathssf}
%\SpecialOptionIndex{valuemathstt}
%\SpecialOptionIndex{valuemathrm}
%\SpecialOptionIndex{valuemathsf}
%\SpecialOptionIndex{valuemathtt}
%\SpecialOptionIndex{unitmathsrm}
%\SpecialOptionIndex{unitmathssf}
%\SpecialOptionIndex{unitmathstt}
%\SpecialOptionIndex{valuetextrm}
%\SpecialOptionIndex{valuetextsf}
%\SpecialOptionIndex{valuetexttt}
%\SpecialOptionIndex{valuetextrm}
%\SpecialOptionIndex{valuetextsf}
%\SpecialOptionIndex{valuetexttt}
%\SpecialOptionIndex{unittextrm}
%\SpecialOptionIndex{unittextsf}
%\SpecialOptionIndex{unittexttt}
% Each of the font options can be given separately for the contents
% of numbers and units.  The option names include \opt{value} or
% \opt{unit} before the mode name.  For example, the \opt{mathsrm}
% option may be split into \opt{valuemathsrm} and \opt{unitmathsrm}.
%
%\DescribeOption{detectdisplay}
% The font detection system can treat displayed mathematical content
% in two ways.  This is controlled by the \opt{detectdisplay} option.
% When set to \defaultopt{true}, display mathematics is treated
% independently from the body of the document.  Thus the local
% \emph{maths} font is checked for matching.  In contrast, when set
% to \opt{false}, display material is treated with the current
% running text font.
%\begin{LaTeXexample}
%  \sffamily
%  Some text
%  \sisetup{obeyall}
%  \[ x = \SI{1.2e3}{\kg\kelvin\candela} \]
%  More text
%  \sisetup{detectdisplay=false}
%  \[ y = \SI{3}{\metre\second\mole} \]
%\end{LaTeXexample}
%
%\subsection{Spacing and separators}
%\DescribeOption{unitsep}
%\DescribeOption{valuesep}
%\DescribeOption{digitsep}
%\DescribeOption{anglesep}
% The separators between items can all be set using options taking a
% list of pre-defined items or a literal value.  The ``\opt{sep}''
% options (\opt{unitsep}, \opt{valuesep}, \opt{digitsep} and
% \opt{anglesep}) all recognise \opt{thin}, \opt{medium}, \opt{med},
% \opt{thick}, \opt{space}, \opt{cdot}, \opt{times}, \opt{tightcdot},
% \opt{tighttimes}, \opt{fullstop}, \opt{stop}, \opt{period} and
% \opt{none}.  The named spaces are the normal maths separations,
% with \opt{space} representing a full (non-breakable text) space,
% and with the obvious meanings for \opt{cdot} and \opt{times}.  The
% \opt{tight} variants reduce the spacing available. Three possible
% values are provided for ``|.|'', and \opt{none} yields no space at
% all. In all cases, other values are treated literally and are
% typeset in maths mode.  The default value is \defaultopt{thin} for
% all separations except \opt{anglesep}, which is set to
% \defaultopt{none}.
%
%\DescribeOption{unitspace}
%\DescribeOption{errspace}
% The \opt{unitspace} and \opt{errspace} options again take a list
% or literal value, but only the ``real'' spaces \opt{thin},
% \opt{medium}, \opt{med}, \opt{thick}, \opt{space} and \opt{none}
% are recognised in the list. The \opt{unitspace} option controls the
% output generated by an explicit space (|~|) inside a unit macro,
% while \opt{errspace} is used to separate a bracketed error from the
% main number.
%
%\subsection{Number formatting}
%\DescribeOption{numdigits}
%\DescribeOption{numdecimal}
%\DescribeOption{numdiv}
%\DescribeOption{numexp}
%\DescribeOption{numsign}
%\DescribeOption{numaddn}
%\DescribeOption{numgobble}
% There are two groups of options for formatting numbers.  The first
% group all begin with ``\opt{num}'', and take literal values used by
% the package to parse numbers.  \opt{numdigits} contains the valid
% number symbols (\defaultopt{0123456789}), with \opt{numdecimal}
% containing the decimal markers (\defaultopt{.,}).  As in the
% \numprint package, \opt{numexp} (the list of exponent markers)
% recognises \defaultopt{deDE} as valid by default.  \opt{numsign}
% contains the sign markers for numbers
% (\defaultopt{+-\cs{pm}\cs{mp}}).  \opt{numdiv} holds the division
% marker in numbers (\defaultopt{/}). \opt{numaddn} and \opt{numgobble}
% both control which other characters do not give an error when present
% in a number.  \opt{numaddn} contains valid characters which should be
% included in the final output ``as is'', whereas \opt{numgobble} lists
% the characters that are completely ignored.  In all cases, the
% content of the options is a simple string, for example
% \opt{numdigits=1234567890}.
%
%\DescribeOption{decimalsymbol}
%\DescribeOption{digitsep}
%\DescribeOption{sepfour}
% The second group of number options control the output of numbers
% after parsing. The symbol used by \currpkg as a decimal marker is
% set by the \opt{decimalsymbol} option, which can take a list of
% choices or a literal.  The valid choices here are
% \defaultopt{fullstop}, \opt{comma}, \opt{cdot} and
% \opt{tightcdot}.\footnote{\opt{fullstop} also has aliases
% \opt{stop} and \opt{period}.}  Notice that this does not have to
% agree with the input marker.  The other separator for numerical
% output is the division of digits into groups of three.  The result
% is dependent on two options.  The previously-described
% \opt{digitsep} option controls the spacing added between groups of
% three numbers.  For numbers consisting of exactly four digits, the
% \opt{sepfour} Boolean option controls whether separation occurs in
% these cases.  The default is \defaultopt{false}.
%\begin{LaTeXexample}
%  \num{1234}\\
%  \num[sepfour]{1234}
%\end{LaTeXexample}
%
%\DescribeOption{seperr}
%\DescribeOption{numopenerr}
%\DescribeOption{numcloseerr}
% For numbers given with an error [\eg \num{1.23(4)}], the package
% can separate out the error part, to give for example
% \num[seperr]{1.23(4)}.  This behaviour is activated by the
% \opt{seperr} option, and requires that \opt{numopenerr} and
% \opt{numcloseerr} contain the left- and right-hand delimiters for
% the error [defaults \opt{numopenerr=(} and \opt{numcloseerr=)}].
%\begin{LaTeXexample}
%  \sisetup{seperr}\\
%  \num{1.234(5)}\\
%  \num{1.234(5)e6}
%\end{LaTeXexample}
%\DescribeOption{trapambigerr}
%\DescribeOption{openerr}
%\DescribeOption{closeerr}
%\DescribeOption{tightpm}
% If the number has an exponent, or if units are not repeated, then
% the result can be considered ambiguous.  By default, the package
% adds the markers stored in \opt{openerr} and \opt{closeerr} to
% remove the ambiguity; the options have the same default values as
% the input error markers.  Detection of a potentially-ambiguous
% error is controlled by the \opt{trapambigerr} option, although for
% numbers with units the \opt{repeatunits} option is also important.
% The spacing around the $\pm$ sign is normally set by \TeX. However,
% using the \opt{tightpm} option will cause this to be reduced to a
% minimum.
%\begin{LaTeXexample}
%  \sisetup{seperr}\\
%  \num[trapambigerr=false]{1.234(5)e6}\\
%  \SI{1.234(5)}{\metre}\\
%  \sisetup{repeatunits=false}
%  \SI{1.234(5)}{\metre}\\
%  \SI[trapambigerr=false]{1.234(5)}{\metre}\\
%  \num[tightpm]{1.234(5)}
%\end{LaTeXexample}
%
%\DescribeOption{numprod}
%\DescribeOption{repeatunits}
% The number processor can recognise products in the numerical input;
% the symbols used for products are stored in \opt{numprod}, with the
% default value of ``\opt{x}''.  By default, the \cs{SI} macro will
% repeat the units for a number given in this way.  This behaviour is
% altered  by the \opt{repeatunits} option, which takes the values
% \defaultopt{true}, \opt{false} and \opt{power}.  The later applies
% only when providing multiplied numbers with units, and converts the
% unit to the appropriate power rather than repeating the
% units.\footnote{This is a very simply option: do not expect it to
% work with anything except areas and volumes.} Notice that when
% applied to errors, \opt{repeatunits} takes priority over
% \opt{trapambigerr}.
%\begin{LaTeXexample}
%  \num{1 x 2 x 3}\\
%  \SI{4 x 5 x 6}{\metre}\\
%  \sisetup{repeatunits=false}
%  \SI{1.2 x 3.4 x 5.6}{\milli\metre\cubed}\\
%  \SI[seperr]{1.234(5)}{\metre}\\
%  \SI[seperr,
%    trapambigerr=false]
%    {9.1093897(54)e-31}{\kilo\gram}\\
%  \SI[seperr,repeatunits,
%    trapambigerr=false]
%    {9.1093897(54)e-31}{\kilo\gram}\\
%  \SI[repeatunits=power]{1 x 2 x 3}{\metre}
%\end{LaTeXexample}
%
%\DescribeOption{expproduct}
%\DescribeOption{expbase}
%\DescribeOption{allowzeroexp}
% The formatting of exponents is controlled by \opt{expproduct} and
% \opt{expbase}.  \opt{expproduct} sets the symbol used to indicate a
% product for exponents (\eg the $\times$ in \num{2e2}), while the
% value of \opt{expbase} sets the power used (the \num{10} in the
% example).  Both options accept a very short list of options:
% \defaultopt{times}, \opt{tighttimes}, \opt{cdot} and
% \opt{tightcdot} for the product, and \defaultopt{ten} and \opt{two}
% for the power.\footnote{The \opt{tighttimes} and \opt{tightcdot}
% options give the rather questionable results:
% \num[expproduct=tighttimes]{1e2} and
% \num[expproduct=tightcdot]{1e2}, as opposed to \num{1e2} and
% \num[expproduct=cdot]{1e2}.} Other choices are used literally. Also
% relevant to exponent processing is the \opt{allowzeroexp} option.
% By default, the package will suppress a zero exponent, but setting
% the flag will allow the output of \num[allowzeroexp]{e0}.
%
%\DescribeOption{addsign}
%\DescribeOption{sign}
%\DescribeOption{retainplus}
% Additions to the input can take the form of implicit signs and
% padded zeros.  The \opt{addsign} option takes a list of potential
% sites to add a sign: \defaultopt{none}, \opt{mantissa},
% \opt{exponent} and \opt{both}.\footnote{Aliases are provided:
% \opt{mant} = \opt{mantissa}, \opt{exp} = \opt{exponent}, \opt{all}
% = \opt{true} = \opt{both}, \opt{false} = \opt{none}.}  If no sign
% is given in the input, the setting here determines if one is added.
% The sign to add is stored in \opt{sign}, which takes the list of
% choices \defaultopt{plus}, \opt{minus}, \opt{pm} and \opt{mp}, or
% uses the input literally (in maths mode).  For positive numbers,
% the \opt{retainplus} option causes a $+$ sign explicitly in the
% input to be retained.  By default, the package will remove such
% signs.
%
%\DescribeOption{padnumber}
% The \opt{padnumber} option controls the addition of zeros to the
% input, to ``pad'' the result.  The option takes a list of choices:
% \defaultopt{leading}, \opt{trailing}, \opt{both} and
% \opt{none}.\footnote{Aliases: \opt{all} = \opt{true} = \opt{both},
% \opt{false} = \opt{none}.}  No additional precision is added by
% this option; integer input will not add a decimal point.
%\begin{LaTeXexample}
%  \num[padnumber=leading]{.1}\\
%  \num[padnumber=leading]{2.}\\
%  \num[padnumber=trailing]{3.}\\
%  \num[padnumber=both]{4.}\\
%  \num[padnumber=both]{.5}\\
%  \num[padnumber=both]{6}\\
%  \num[padnumber=none]{7.}\\
%  \num[padnumber=none]{.8}
%\end{LaTeXexample}
%
%\DescribeOption{fixdp}
%\DescribeOption{dp}
% In contrast to the \opt{padnumber} option, the package can alter
% the precision of the input number if the \opt{fixdp} option is set.
% The \opt{fixdp} option will fix the decimal places of the output to
% the number stored in the \opt{dp} option.  The later should be a 
% positive integer or zero.
%\begin{LaTeXexample}
%  \sisetup{fixdp,dp=3}
%  \num{1}\\
%  \num{53.9}\\
%  \num{4.56783}\\
%  \num{-1.2942}\\
%  \num{-1.2959}\\
%  \num{9.9999}\\
%  \num{2.1264e90}\\
%  \num[dp=0]{12.345}
%\end{LaTeXexample}
%
%\DescribeOption{fixsf}
%\changes{v1.1}{2008/08/22}{New \opt{fixsf} option}
%\DescribeOption{sf}
%\changes{v1.1}{2008/08/22}{New \opt{sf} option}
% In contrast to the \opt{dp} option, the \opt{fixsf} and \opt{sf}
% options control the number of significant figures in the output.
% This never adds extra digits to the output, as significant figures
% are concerned with accuracy.
%\begin{LaTeXexample}
%  \sisetup{fixsf,sf=3}
%  \num{1}\\
%  \num{53.9}\\
%  \num{4.56783}\\
%  \num{-1.2942}\\
%  \num{-1.2959}\\
%  \num{9.9999}\\
%  \num{2.1264e90}\\
%  \num{773322}
%\end{LaTeXexample}
%
%\subsection{Angle formatting}
%\DescribeOption{padangle}
%\DescribeOption{strictarc}
% The angle formatter uses \cs{num} to format numbers: any options
% for numbers are therefore applicable here.  The \opt{padangle}
% option takes choices \defaultopt{small}, \opt{large}, \opt{all} and
% \opt{none}, and controls how angles are padded when given in
% degrees, minutes and seconds.\footnote{Aliases: \opt{all} =
% \opt{true} = \opt{both}, \opt{false} = \opt{none}.}  When giving
% angles as arcs (in degrees, minutes and seconds), the package can
% detect if the correct number of semi-colons have been given. This
% is controlled by the \opt{strictarc} option, which is a Boolean
% switch with a \defaultopt{true} default.  With \opt{strictarc} set
% to \opt{false}, an incomplete arc is interpreted as degrees and
% minute, while an over-complete one will drop excess input.
%\begin{LaTeXexample}
%  \ang[padangle=none]{1;;}\\
%  \ang[padangle=large]{2;;}\\
%  \ang[padangle=small]{3;;}\\
%  \ang[padangle=both]{;4;}\\
%  \ang[padangle=none]{;5;5}\\
%  \ang[strictarc=false]{1;2}\\
%  \ang[strictarc=false]{1;2;3;4;}
%\end{LaTeXexample}
%
%\DescribeOption{angformat}
% The angle formatting system can convert between decimal angles and
% those given as degrees, minutes and seconds.  This is controlled by
% the \opt{angformat} option, which takes choices
% \defaultopt{unchanged}, \opt{decimal} and
% \opt{arc}.\footnote{Aliases: \opt{decimal} = \opt{dec}, \opt{arc} =
% \opt{dms}, \opt{unchanged} = \opt{none}.}  When set to
% \opt{unchanged}, nothing is done to the input.  The conversion is
% based on \TeX\ dimensions, and is therefore limited in accuracy.
% For this reason, the output is automatically rounded: output as a
% decimal angle is limited to three places, and that as an arc is
% given to a single decimal place for the seconds component.
%\begin{LaTeXexample}
%  $\ang{1;2;3} = \ang[angformat=dec]{1;2;3}$\\
%  $\ang{4.56} = \ang[angformat=arc]{4.56}$
%\end{LaTeXexample}
%
%\DescribeOption{astroang}
% For astronomers, the \opt{astroang} option is provided.  This
% moves the degrees, minutes or seconds symbol (as appropriate)
% over the decimal marker rather than after the number.
%\begin{LaTeXexample}
%  \ang{1;2;3.4}\\
%  \ang[astroang]{5;6;7.8}
%\end{LaTeXexample}
%
%\subsection{Tabular material}
%\DescribeOption{tabnumalign}
% Material typeset in |S| columns is processed internally by the
% \cs{num} macro.  Thus, as with angles, the number options also
% apply here.  The positioning of tabular material is controlled by
% the two options \opt{tabnumalign} and \opt{tabformat}.
% \opt{tabnumalign} takes values \defaultopt{centredecimal},
% \opt{centre}, \opt{left} and \opt{right}.\footnote{Aliases
% \opt{centerdecimal} = \opt{centredecimal}; \opt{center} =
% \opt{centre}.}  When using \opt{centredecimal}, the package places
% the decimal marker of the mantissa at the centre of the column,
% which then grows to accommodate the widest number given.  For equal
% numbers of digits before and after the decimal sign, this is the
% easiest option.  The other choices use a fixed-width box to store
% the number; the box is then aligned with the edges of the column.
%
%\DescribeOption{tabformat}
% The \opt{tabformat} option sets the amount of space reserved by
% \currpkg for the alignment box when not using the
% \opt{centredecimal} setting of \opt{tabnumalign}.  The numerical
% parts of \opt{tabformat} are interpreted as
% \texttt{\meta{pre}\meta{dec}\meta{post}}; \meta{pre} and
% \meta{post} are the number of digits before and after the decimal
% sign, respectively.  Both signs and exponents can be included in
% \opt{tabformat}, resulting in appropriate space being reserved. The
% entire \opt{tabformat} input is processed using the \cs{num} macro
% internally.  Thus the decimal and exponent signs used in
% \opt{tabformat} are checked against \opt{numdecimal} and
% \opt{numexp}, respectively.
%
%\DescribeOption{tabalignexp}
%\DescribeOption{tabexpalign}
%\changes{v1.1d}{2008/10/29}{Added \opt{tabexpalign} as an alias for
%  \opt{tabalignexp}}
% When \opt{tabformat} contains exponents, two possibilities are
% available for alignment.  The first method is to place the exponent
% parts so that the ``$\times 10$'' parts form a column, with
% whitespace after shorter mantissa components.  In the second
% method, no additional space is added after the mantissa, and the
% exponents do no line up (\ref{tab:alignexp}).  This is controlled
% by the \opt{tabalignexp} option, which can be set to
% \defaultopt{true} or \opt{false}. The alias \opt{tabexpalign} is
% available for this option.
%\begin{LaTeXexample}[float]
%  \begin{table}
%    \centering
%    \caption{The \opt{tabalignexp} option}
%    \label{tab:alignexp}
%    \sisetup{tabformat=1.3e2,tabnumalign=centre}
%    \begin{tabular}{SS[tabalignexp=false]}
%      \toprule
%      {Header} & {Header} \\
%      \midrule
%      1.2e3    & 1.2e3    \\
%      1.234e56 & 1.234e56 \\
%      \bottomrule
%    \end{tabular}
%  \end{table}
%\end{LaTeXexample}
%
%\DescribeOption{tabtextalign}
%\DescribeOption{tabunitalign}
%\DescribeOption{tabalign}
% Cells containing no numbers are handled by \currpkg in a manner
% similar to \cs{multicolumn}.  The setting of \opt{tabtextalign} is
% taken from the list \defaultopt{centre}, \opt{right} and
% \opt{left}.\footnote{Alias \opt{centre} = \opt{center}.}  As would
% be expected, these settings centre, right- or left-align the cell
% contents. In |s| columns, all content is treated as input to the
% \cs{si} macro.  The alignment of the contents relative to the cell
% is controlled by the \opt{tabunitalign} option, which takes options
% \defaultopt{left}, \opt{right} and \opt{centre}.  The settings for
% \opt{tabnumalign}, \opt{tabtextalign} and \opt{tabunitalign} can be
% set to the same value in one go with the \opt{tabalign} option.
%
%\DescribeOption{tabautofit}
% The contents of table cells can automatically be rounded or
% zero-filled to the number of decimal places given in
% \opt{tabformat}.  This is activated by the \opt{tabautofit} Boolean
% option. As \opt{tabformat} does not apply to columns with alignment
% \opt{centredecimal}, \opt{tabautofit} is also inactive for these
% columns (\ref{tab:autofit}).
%\begin{LaTeXexample}[float]
%  \begin{table}
%    \centering
%    \caption{The \opt{tabautofit} option}
%    \label{tab:autofit}
%    \sisetup{tabformat=1.3,tabnumalign=centre}
%    % Notice the overfull hbox which results with
%    % the first column
%    \begin{tabular}{%
%      S%
%      S[tabautofit]%
%      S[tabautofit,tabnumalign=centredecimal]}
%      \toprule
%      {Header} & {Header} & {Header} \\
%      \midrule
%      1.2    & 1.2    & 1.2    \\
%      1.2345 & 1.2345 & 1.2345 \\
%      \bottomrule
%    \end{tabular}
%  \end{table}
%\end{LaTeXexample}
%
%\DescribeOption{tabparseonly}
% It is possible to turn off the automatic alignment of numbers
% while retaining the ability to parse the |S| column input.
% Setting \opt{tabparseonly} to \opt{true} will still parse the
% column contents, but will not align the decimal markers.  The
% alignment of the cell is governed by \opt{tabnumalign}
% (\ref{tab:parse}).
%\begin{LaTeXexample}[float]
%  \begin{table}
%    \centering
%    \caption{The \opt{tabparseonly} option}
%    \label{tab:parse}
%    \begin{tabular}{%
%      S[tabparseonly]%
%      S[tabparseonly,tabnumalign=left]%
%      S[tabparseonly,tabnumalign=right]}
%      \toprule
%      {Header} & {Header} & {Header} \\
%      \midrule
%      14.2    & 14.2    & 14.2    \\
%      1.23456 & 1.23456 & 1.23456 \\
%      1.2e3   & 1.2e3   & 1.2e3   \\
%      \bottomrule
%    \end{tabular}
%  \end{table}
%\end{LaTeXexample}
%
%\subsection{Units}
%\DescribeOption{per}
% Most of the unit options are concerned with the processing of named
% units.  The processor for units given as macro names can be
% influenced to give a variety of output formats.  The \opt{per}
% option defines how the keyword macro \cs{per} is handled.  This
% option takes a choice from the list \defaultopt{reciprocal},
% \opt{slash} and \opt{fraction}.\footnote{Aliases \opt{reciprocal} =
% \opt{rp} = \opt{power}, \opt{fraction} = \opt{frac}.}  The default
% option uses \cs{per} to indicate reciprocal powers, whereas
% \opt{slash} causes the package to use ``$/$'' to show division.
%
%\DescribeOption{fraction}
%\DescribeOption{slash}
% The \opt{fraction} option defines how \opt{per=fraction} is
% interpreted.  The list of applicable values here is
% \defaultopt{frac}, \opt{nice}, \opt{ugly} and \opt{sfrac}.  In each
% case, the unit is typeset as a fraction, but the macro use to
% achieve this varies. \opt{frac} uses the \TeX\ \cs{frac}, macro,
% while \opt{nice} makes use of a \cs{nicefrac}-like method.  The
% \opt{ugly} option uses a slash in text mode and \cs{frac} in maths
% mode.\footnote{Similar to the \pkg{ugly} option of the
% \pkg{nicefrac} package.}  Finally, the setting \opt{fraction=sfrac}
% uses the \cs{sfrac} macro from the \pkg{xfrac} package, when
% available.\footnote{\pkg{xfrac} is part of the experimental system
% for \LaTeX3.  As it requires a number of additional packages to
% work, \currpkg does not load \pkg{xfrac}. If it is unavailable, the
% \opt{sfrac} setting will fall back to using \cs{nicefrac}.  See the
% \pkg{xfrac} documentation for reasons to prefer \cs{sfrac} to
% \cs{nicefrac}.}  The \opt{slash} option sets the symbol used when
% \opt{per=slash} is in force.  This recognises the single keyword
% \opt{slash}; anything else is used literally.
%\begin{LaTeXexample}
%  \sisetup{per=fraction}
%  \si[fraction=frac]{\metre\per\second}\\
%  \si[fraction=nice]{\metre\per\second}\\
%  \si[fraction=sfrac]{\metre\per\second}\\
%  \si[per=slash]{\metre\per\second}
%\end{LaTeXexample}
%
%\DescribeOption{stickyper}
% By default, \cs{per} applies only to the next unit
% given.\footnote{This is the standard method of reading units in
% English: for example, \si{\joule\per\mole\per\kelvin} is pronounced
% ``joules per mole per kelvin''.}  By setting the \opt{stickyper}
% flag, this behaviour is changed so that \cs{per} applies to all
% subsequent units.\footnote{This is the behaviour in \SIunits.}
%\begin{LaTeXexample}
%  \si{\kilogram\per\metre\ampere}\\
%  \si[stickyper]{\kilogram\per\metre\ampere}
%\end{LaTeXexample}
%
%\DescribeOption{trapambigfrac}
%\DescribeOption{openfrac}
%\DescribeOption{closefrac}
% When using \opt{per=slash}, multiple units in the denominator will
% yield a potentially ambiguous result.  The \opt{trapambigfrac}
% determines whether the package checks for this: this takes
% \defaultopt{true} and \opt{false}.  When set, the contents of
% \opt{openfrac} are inserted before the denominator, and
% \opt{closefrac} is inserted after.
%\begin{LaTeXexample}
%  \sisetup{trapambigfrac,openfrac=(,closefrac=),per=slash}\\
%  \si{\metre\per\second\per\kilogram}\\
%  \si[trapambigfrac=false]{\metre\per\second\per\kilogram}
%\end{LaTeXexample}
%
%\DescribeOption{prefixsymbolic}
%\DescribeOption{prefixbase}
%\DescribeOption{prefixproduct}
% The unit prefixes (\cs{kilo}, \etc) are normally given as letters.
% However, the package can convert these into numerical
% powers.\footnote{Provided things are not too complex!}  This is
% controlled by the \opt{prefixsymbolic} Boolean option, which by
% default is \defaultopt{true}.  If \opt{prefixsymbolic} is set to
% \opt{false}, the format of the prefix is controlled by
% \opt{prefixbase} and \opt{prefixproduct}, which work in the same
% way as \opt{expbase} and \opt{expproduct}.
%
%\DescribeOption{prespace}
%\DescribeOption{xspace}
%\DescribeOption{allowoptarg}
% By default, the single unit macros (\eg \cs{metre}) add no space
% either before or after the unit.  Setting the \opt{xspace} flag to
% true means that the single macros are followed by the \cs{xspace}
% command (when used outside of \cs{SI}/\cs{si}).  For users of
% \unitsdef, the \opt{prespace} macro changes the behaviour of the
% unit macros, so that they can immediately follow a number.  As a
% result, the unit macros will \emph{always} be preceded by a fixed
% space when the \opt{prespace} flag is true: this will be in
% addition to any other space. Also relevant to users moving from
% \unitsdef is the \opt{allowoptarg} option.  This allows single unit
% macros to take an optional numerical argument, in the same way that
% occurs in that package.
%\begin{LaTeXexample}
%  \metre is the symbol for metres\\
%  \sisetup{xspace}
%  No, \metre is the correct symbol\\
%  \sisetup{prespace}
%  30\metre\\
%  Do not use \metre in running text\\
%  \sisetup{allowoptarg}
%  \metre[40]
%\end{LaTeXexample}
%
%\subsection{Symbols}
% User access to control the symbols used for \si{\ohm}, \si{\micro},
% \si{\degree}, \si{\arcmin}, \si{\arcsec}, \si{\angstrom} and
% \si{\celsius} is provided here. These are all literal options,
% which are available in text and maths mode variants.  For example,
% \opt{textmicro} is the code used for the \si{\micro} symbol in text
% mode.  The text mode macros should be safe when forced into text,
% and the maths ones when forced into maths.  The symbols defined in
% this way are:
%\newcommand*{\listsymbopt}[1]{\item \opt{#1}\SpecialOptionIndex{#1}}
%\begin{itemize}
%  \listsymbopt{textOmega};
%  \listsymbopt{mathsOmega};
%  \listsymbopt{textmu};
%  \listsymbopt{mathsmu};
%  \listsymbopt{textdegree};
%  \listsymbopt{mathsdegree};
%  \listsymbopt{textminute};
%  \listsymbopt{mathsminute};
%  \listsymbopt{textsecond};
%  \listsymbopt{mathssecond};
%  \listsymbopt{textringA};
%  \listsymbopt{mathsringA};
%  \listsymbopt{textcelsius};
%  \listsymbopt{mathscelsius}.
%\end{itemize}
%
%\DescribeOption{redefsymbols}
% When \currpkg is loaded, it can check for the presence of the
% \pkg{textcomp} and \pkg{upgreek} packages, to provide better
% symbols for certain items. To prevent this, set the
% \opt{redefsymbols} option \opt{false} (the default is
% \defaultopt{true}).
%
%\DescribeOption{eVcorra}
%\DescribeOption{eVcorrb}
% The \si{\electronvolt} symbol requires some fine-tuning, and so has
% two options of its own, both \TeX\ lengths.  \opt{eVcorra} is the
% correction applied to the gap between ``e'' and ``V'' of the unit:
% the default is \defaultopt{0.3ex}. \opt{eVcorrb} is the correction
% applied to the gap between ``V'' of the unit and whatever follows;
% the default is \defaultopt{0ex}.  The optimal value for these
% options will depend on the current font settings.\footnote{This
% document uses \opt{eVcorra=0.1ex}.}
%\begin{LaTeXexample}
%  \si[per=slash]{\electronvolt\per\metre}\\
%  \si[per=slash,eVcorrb=0.7ex]{\electronvolt\per\metre}
%\end{LaTeXexample}
%
%\subsection{Colour}
%\DescribeOption{colourall}
%\DescribeOption{colourunits}
%\DescribeOption{colourvalues}
%\SpecialOptionIndex{colorall}
%\SpecialOptionIndex{colorunit}
%\SpecialOptionIndex{colorvalue}
% The package provides internal hooks for applying colour to part or
% all of the output.  This requires the user to load the \pkg{color}
% or \pkg{xcolor} package to support colour in the output; \currpkg
% will ignore a colour request if support is unavailable. The Boolean
% options \opt{colourall}, \opt{colourunits} and \opt{colourvalues}
% are used to turn application of a given colour on or off for all
% output, only units and only values, respectively.  All three
% switches are available with US spelling, \eg \opt{colorall} and
% \opt{colourall} behave in the same way.  With colour turned off, no
% \cs{color} command is issued internally, and output follows the
% surrounding text.
%
%\DescribeOption{colour}
%\DescribeOption{unitcolour}
%\DescribeOption{valuecolour}
%\SpecialOptionIndex{color}
%\SpecialOptionIndex{unitcolor}
%\SpecialOptionIndex{valuecolor}
% The colour names to use for colouring output are set by the
% \opt{colour}, \opt{unitcolour} and \opt{valuecolour} options (all
% also available with US spelling).  The \opt{colour} option
% internally sets both \opt{unitcolour} and \opt{valuecolour}.
%\begin{LaTeXexample}
%  {\color{brown} Text \num{50}}\\
%  \sisetup{colourall,colour=green}
%  \SI{10}{\metre}\\
%  {\color{purple} Text \num{70}}\\
%  \sisetup{colourunits=true,colourvalues=false}
%  \SI{40}{\second}
%\end{LaTeXexample}
%
%\DescribeOption{colourneg}
%\DescribeOption{negcolour}
%\SpecialOptionIndex{colorneg}
%\SpecialOptionIndex{negcolor}
% \currpkg can automatically add a colour to negative numbers.  This
% is turned on using the \opt{colourneg} switch.  The colour used is
% set by the \opt{negcolour} option; both options are available using
% US spellings.
%
%\subsection{International support}
%\DescribeOption{locale}
% \currpkg allows the user to switch between the typographic
% conventions of different (geographical) areas by using
% \latin{locales}.  Currently, the package is supplied with
% configurations for locales \opt{UK}, \opt{USA}, \opt{DE} (Germany)
% and \opt{ZA} (South Africa). The \opt{locale} option is used to
% switch to a particular locale.
%\begin{LaTeXexample}
%  \SI{1.234}{\metre}\\
%  \SI[locale=DE]{6.789}{\metre}
%\end{LaTeXexample}
%
%\DescribeOption{loctolang}
% Locales are distinct from \pkg{babel} languages, as typographic
% conventions are not tightly integrated with language.  However, it
% is useful to be able to associate a particular locale with a
% \pkg{babel} language.  The option \opt{loctolang} handles this, and
% expects pairs of values:
% \opt{loctolang=\meta{locale}:\meta{language}}.
%\begin{LaTeXexample}
%  \sisetup{loctolang={UK:UKenglish,DE:german}}\\
%  \SI{6.022e23}{\per\mole}\\
%  \selectlanguage{german}\\
%  \SI{6.022e23}{\per\mole}
%\end{LaTeXexample}
%
%\subsection{Package control}
%\DescribeOption{load}
%\DescribeOption{noload}
%\DescribeOption{alsoload}
% The package keeps most of the unit and abbreviations definitions in
% files separate from \file{siunitx.sty}.  To control what is loaded,
% three complementary options are provided, all of which take a list
% of one or more choices.  \opt{load} and \opt{alsoload} define which
% support configuration files are loaded.  The list in \opt{load}
% recognises the value \opt{default}, which is expanded to the normal
% list before loading. The difference between \opt{load} and
% \opt{alsoload} is that \opt{load} specifies the \emph{complete}
% list of files to load, whereas \opt{alsoload} adds to the existing
% list. To use the \opt{load} option successfully requires knowing
% everything that is needed. The \opt{noload} option can be used to
% delete one or more items from the \opt{load} list, without needing
% to know what is on it.\footnote{\opt{noload} does not prevent the
% loading of a file needed by one which is loaded.  Thus the package
% may internally override a \opt{noload} value if needed.}
%
%\DescribeOption{log}
%\DescribeOption{debug}
% To control data written to the \ext{log} file, the \opt{log} option
% is provided.  This takes a value from the list \defaultopt{normal},
% \opt{none}, \opt{minimal}, \opt{errors} and \opt{debug}.  As would
% be expected, these indicate the amount of detail written to the log
% file. As a shortcut to \opt{log=debug}, the package also recognises
% the \opt{debug} option directly.
%
%\DescribeOption{strict}
% Some users will want to stick closely to the official rules for
% typesetting units.  This could be made complicated if the options
% for non-standards behaviour could not be turned off.  The load-time
% option \opt{strict} resets package behaviour to follow the rules
% closely, and disables options which deviate from this.  If the
% package is loaded with the \opt{strict} option, all output is made
% in maths mode using the upright serif font.
%
%\subsection{Back-compatibility options}
%\DescribeOption{emulate}
% The package can emulate \SIunits, \SIstyle, \unitsdef, \units,
% \pkg{hepunits}, \pkg{fancyunits} and \pkg{fancynum}.  Giving the
% \opt{emulate=\meta{package}} option will give the desired
% emulation, and combinations which would be possible with the real
% packages will also work here. The package will recognise the
% options of the emulated packages.  This will automatically cause
% emulation to be switched on.
%
%\subsection{Summary of all options}
% \ref{tab:allopt} lists a summary of the package options (excluding
% those for backward-compatibility).  A reminder or the input format
% is also provided.
%\begin{longtable}{>{\ttfamily}lll}
%    \caption{All package options\label{tab:allopt}}\\
%    \toprule
%    Option & Type & Description \\
%    \midrule
%  \endfirsthead
%    \toprule
%    Option & Type & Description \\
%    \midrule
%  \endhead
%    \bottomrule
%    \multicolumn{3}{r}{\emph{Continued on next page}}
%  \endfoot
%    \bottomrule
%  \endlastfoot
%  addsign & List & Add sign to number \\
%  allowzeroexp & Boolean & Allow \num[allowzeroexp]{e0} \\
%  angformat & List & Conversion of angle format \\
%  anglesep & List or literal & Space between angle components \\
%  astroang & Boolean & Astronomy-style angles \\
%  closeerr & Literal & Closes potential-ambiguous error \\
%  closefrac & Literal & Closes potential-ambiguous fraction \\
%  color & Literal & Colour used for units and values \\
%  colour & Literal & Colour used for units and values \\
%  colorall & Boolean & Switch for colouring all output \\
%  colourall & Boolean & Switch for colouring all output \\
%  colorneg & Boolean & Colour negative numbers \\
%  colourneg & Boolean & Colour negative numbers \\
%  colorunits & Boolean & Switch for colouring units \\
%  colourunits & Boolean & Switch for colouring units \\
%  colorvalues & Boolean & Switch for colouring values \\
%  colourvalues & Boolean & Switch for colouring values \\
%  decimalsymbol & List or literal & Decimal symbol \\
%  debug & Boolean & Write debugging data to log \\
%  detectdisplay & Boolean & Treat display maths separately \\
%  digitsep & List & Separation of digits in large numbers \\
%  dp & Integer & Number of decimal places to \\
%   & & output numbers to \\
%  emulate & Modules & Emulation modules to load \\
%  errspace & List & Spacing of bracketed error \\
%  eVcorra & Length & Spacing correction in \eV \\
%  eVcorrb & Length & Spacing correction after \eV \\
%  expbase & List or literal & Base used for exponents \\
%  expproduct & List or literal & Product sign for exponents \\
%  fixdp & Boolean & Switch for fixing decimal \\
%   & & places of numbers \\
%  fixsf & Boolean & Switch for fixing significant \\
%    & & figures of numbers \\
%  fraction & List & Method used when setting \opt{per=frac} \\
%  inlinebold & List & Select how inline bold is tested \\
%  load & Modules & Modules to load \\
%  locale & Modules & Locale to follow \\
%  loctolang & Special & Associate locale with \pkg{babel}
%    language \\
%  log & List & Amount of data added to log \\
%  mathOmega & Literal & ``$\makeatletter\si@mathsOmega\makeatother$''
%    symbol in maths mode \\
%  mathcelsius & Literal & ``$\makeatletter\si@mathscelsius\makeatother$''
%    symbol in maths mode \\
%  mathdegree & Literal & ``$\makeatletter\si@mathsdegree\makeatother$''
%    symbol in maths mode \\
%  mathminute & Literal & ``$\makeatletter\si@mathsminute\makeatother$''
%    symbol in maths mode \\
%  mathmu & Literal & ``$\makeatletter\si@mathsmu\makeatother$''
%    symbol in maths mode \\
%  mathringA & Literal & ``$\makeatletter\si@mathsringA\makeatother$''
%    symbol in maths mode \\
%  mathrm & Csname & Roman maths font \\
%  mathsOmega & Literal & ``$\makeatletter\si@mathsOmega\makeatother$''
%    symbol in maths mode \\
%  mathscelsius & Literal & ``$\makeatletter\si@mathscelsius\makeatother$''
%    symbol in maths mode \\
%  mathsdegree & Literal & ``$\makeatletter\si@mathsdegree\makeatother$''
%    symbol in maths mode \\
%  mathsecond & Literal & ``$\makeatletter\si@mathssecond\makeatother$''
%    symbol in maths mode \\
%  mathsf & Csname & Sans serif maths font \\
%  mathsminute & Literal & ``$\makeatletter\si@mathsminute\makeatother$''
%    symbol in maths mode \\
%  mathsmu & Literal & ``$\makeatletter\si@mathsmu\makeatother$''
%    symbol in maths mode \\
%  mathsringA & Literal & ``$\makeatletter\si@mathsringA\makeatother$''
%    symbol in maths mode \\
%  mathsrm & Csname & Roman maths font \\
%  mathssecond & Literal & ``$\makeatletter\si@mathssecond\makeatother$''
%    symbol in maths mode \\
%  mathssf & Csname & Sans serif maths font \\
%  mathstt & Csname & Fixed-width maths font \\
%  mathtt & Csname & Fixed-width maths font \\
%  mode & List & Use text or maths mode for \\
%   & & typesetting \\
%  negcolor & Literal & Colour used for negative numbers \\
%  negcolour & Literal & Colour used for negative numbers \\
%  noload & Modules & Modules not to load \\
%  numaddn & Literal & Additional input allowed in numbers \\
%  numcloseerr & Literal & Character indicating end of \\
%   & & numerical error\\
%  numdecimal & Literal & Decimal symbols in numbers \\
%  numdigits & Literal & Digit characters in numbers \\
%  numdiv & Literal & Division characters in numbers \\
%  numexp & Literal & Exponent characters in numbers \\
%  numgobble & Literal & Characters to ignore in numbers \\
%  numopenerr & Literal & Character indicating start of \\
%   & & numerical error\\
%  numprod & Literal & Characters used for a product \\
%  numsign & Literal & Sign characters in numbers \\
%  obeyall & Boolean & Combination of \opt{obeybold}, \\
%   & & \opt{obeyitalic} and \opt{obeymode} \\
%  obeybold & Boolean & Check local bold setting \\
%  obeyitalic & Boolean & Check local italic setting \\
%  obeymode & Boolean & Check local mode (text/maths) \\
%  openerr & Literal & Opens potentially-ambiguous error \\
%  openfrac & Literal & Opens potentially-ambiguous fraction \\
%  padangle & List & Add zeros to blank parts of angles \\
%  padnumber & List & Add zeros to blank parts of numbers \\
%  per & List & Behaviour of \cs{per} \\
%  prefixbase & List or literal & Base used when making prefixes \\
%   & & numerical \\
%  prefixproduct & List or literal & Product sign for prefixes \\
%  prefixsymbolic & Boolean & Behaviour of unit prefixes\\
%  prespace & Boolean & Add space before units \\
%  redefsymbols & Boolean & Use better symbols if available \\
%  repeatunits & List & Repeat units with separated errors \\
%   & &  and products \\
%  retainplus & Boolean & Retain explicit plus sign\\
%  seperr & Boolean & Separate number and error \\
%  sepfour & Boolean & Separate four-digit numbers \\
%  sf & Integer & Number of significant figures \\
%   & & to output numbers to \\
%  sign & List or literal & Sign to add to numbers \\
%  slash & List or literal & Symbol used for ``/'' \\
%  stickyper & Boolean & Require \cs{per} only once \\
%  strict & Boolean & Obey the rules strictly \\
%  strictarc & Boolean & Require exactly zero or \\
%   & & two semi-colons \\
%  tabalign & List & Positioning of all column data \\
%  tabalignexp & Boolean & Alignment of exponents in tables \\
%  tabautofit & Boolean & Switch for rounding numbers to \\
%   & & length given by \opt{tabformat} \\
%  tabformat & Number & Space reserved in table for numbers \\
%  tabnumalign & List & Alignment of |S| column numbers \\
%  tabparseonly & Boolean & Do not align |S| columns \\
%   & & on decimal marker \\
%  tabtextalign & List & Positioning of text in |S| columns \\
%  tabunitalign & List & Positioning of units in |s| columns \\
%  textcelsius & Literal & ``$^{\circ}$C'' symbol in text mode \\
%  textdegree & Literal & ``${\circ}$'' symbol in text mode \\
%  textminute & Literal & ``$'$'' symbol in text mode \\
%  textmu & Literal & ``$\upmu$'' symbol in text mode \\
%  textomega & Literal & ``$\Omega$'' symbol in text mode \\
%  textringa & Literal & ``\AA'' symbol in maths mode \\
%  textrm & Csname & Roman text font \\
%  textsecond & Literal & ``$''$'' symbol in text mode \\
%  textsf & Csname & Sans serif text font \\
%  texttt & Csname & Fixed-width text font \\
%  tightpm & Boolean & Reduce space around $\pm$ \\
%  trapambigerr & Boolean & Check for ambiguous errors \\
%  trapambigfrac & Boolean & Check for ambiguous fractions \\
%  unitcolor & Literal & Switch for colouring units \\
%  unitcolour & Literal & Switch for colouring units \\
%  unitmathrm & Csname & Roman maths font for units \\
%  unitmathsf & Csname & Sans serif maths font for units \\
%  unitmathsrm & Csname & Roman maths font for units \\
%  unitmathssf & Csname & Sans serif maths font for units \\
%  unitmathstt & Csname & Fixed-width maths font for units \\
%  unitmathtt & Csname & Fixed-width maths font for units \\
%  unitmode & List &  As \opt{mode}, for units only \\
%  unitsep & List or literal & Separator for units \\
%  unitspace & List or literal & Space used for ``|~|'' in units \\
%  valuecolor & Literal & Switch for colouring value \\
%  valuecolour & Literal & Switch for colouring value \\
%  valuemathrm & Csname & Roman maths font for values \\
%  valuemathsf & Csname & Sans serif maths font for values \\
%  valuemathsrm & Csname & Roman maths font for values \\
%  valuemathssf & Csname & Sans serif maths font for values \\
%  valuemathstt & Csname & Fixed-width maths font for values \\
%  valuemathtt & Csname & Fixed-width maths font for values \\
%  valuemode & List &  As \opt{mode}, for values only \\
%  valuesep & List or literal & Separator between value and unit \\
%  xspace & Boolean & Use \pkg{xspace} after units \\
%\end{longtable}
%
%\section{Emulation of other packages}
% \currpkg has been designed as a replacement for \SIunits, \SIstyle,
% \unitsdef, \units, \pkg{hepunits}, \pkg{fancyunits} and
% \pkg{fancynum}. It therefore provides options reproduce the
% functions of all of these packages.  In this way, \currpkg should
% be usable as a straight replacement for the older
% packages.\footnote{User macros means that they are described in the
% package documentation; simply not containing an \texttt{@} does not
% mean they will have been emulated.}  This means for example that
% the \cs{num} macro takes an optional star when emulating \SIstyle.
% However, there are some points that should be remembered.  In
% particular, \currpkg validates numerical input, meaning that places
% where a number is expected in the older packages \emph{require} a
% number when emulated by \currpkg.
%
% The \numprint package has provided many useful ideas for the code
% used here for number formatting. The basic use of the \cs{numprint}
% (or \cs{np}) macro can be reproduced using \currpkg.  However,
% \numprint is large and complex, with its own backward-compatibility
% options.  As a result, emulation of \numprint is not provided here.
% To use a \numprint document with \currpkg, the \cs{numprint} macro
% could be provided using the following code.
%^^A Need to deal with \cs{numprint} being defined.
%\let\numprint\relax
%\begin{LaTeXexample}
%  \newcommand*{\numprint}[2][]{\SI[obeymode]{#2}{#1}}\\
%  \numprint{-123456} \\
%  \numprint[N/mm^2]{-123456}
%\end{LaTeXexample}
%^^A Back to the normal - with \cs{newcommand}
%\newcommand*{\numprint}{\pkg{numprint}\xspace}
%
% \currpkg can be used more-or-less directly to replace both
% \pkg{dcolumn} and \pkg{rccol}.  As is explained in the code
% section, much of the column-alignment system here is taken from
% \pkg{dcolumn}, while \pkg{rccol} provided a model for a
% customisable system.  However, neither package has been directly
% emulated here.  The |S| column type can be used to replace both |D|
% and |R| columns by setting the appropriate package options.
%
%\section{Configuration files}
% \currpkg is a modular package.  The unit definitions, abbreviations
% and locales are all stored in configuration files.  These all take
% names of the form \file{si-\meta{name}.cfg}, where \meta{name} is
% the part of the filename used as an option in \cs{sisetup} or when
% loading \currpkg.  Producing new configuration files therefore
% consists of making a suitably-named file and adding it to the path
% searched by \TeX.  The files should normally consist of settings
% (in \cs{sisetup}) and unit definitions, \etc.
%
%\DescribeMacro{\addtolocale}
% To allow arbitrary macros to be stored in locales, the
% \cs{addtolocale} macro is provided.  This ensures that arbitrary
% text is only executed when using a locale, not when loading it. The
% macro takes two arguments, \marg{locale} and \marg{code}.
%\begin{LaTeXexample}
%  \addtolocale{DE}{TEST}
%  Some text as filler \\
%  \sisetup{locale=DE}
%  This example is rather trivial!
%\end{LaTeXexample}
%
%\DescribeMacro{\requiresiconfigs}
% To load one or more configuration files from inside another
% configuration file, the \cs{requiresiconfigs} macro is provided.
% This accepts a comma-separated list of configuration names, in the
% same way as \opt{load} or \opt{noload}.
%
% In addition to the various configuration files provided with the
% package, a local file \file{siunitx.cfg} may be provided.  This is
% read at the end of loading \currpkg, and allows the user to include
% any local definitions or settings easily.
%
%\section{Common questions}
%\subsection{Why do I need \cs{per} more than once?}
% The unit engine of \currpkg is based around the English method for
% reading units out loud. Thus \si{\joule\per\mole\per\kelvin} is
% pronounced ``joules per mole per kelvin''.  Hence by default you
% need to put \cs{per} before each item in the denominator of a unit.
% The behaviour can be altered by setting the \opt{stickyper} option.
%
%\subsection{Why is the order of my units changed?}
% Then using \opt{per=reciprocal}, the units are typeset as given.
% However, when using \opt{per=slash} or \opt{per=fraction}, the
% package needs to find which ones are in the denominator.  It then
% prints the numerator and denominator separately.  So if you give a
% unit in the denominator \emph{before} one in the numerator, they
% have to be re-ordered.\footnote{``Double division''
% (\SI{1}{s/m/kg}) is mathematically incorrect.}
%\begin{LaTeXexample}
%  \SI{88}{\per\kilogram\metre} \\
%  \SI[per=slash]{66}{\per\kilogram\metre}
%\end{LaTeXexample}
%
%\subsection{Why are compound units not recommended outside of
%  \cs{SI}/\cs{si}?}
% To fully process units made up of several parts, the processor has
% to know where the end of the unit is.  When the unit macros are
% used outside of \cs{SI}/\cs{si}, this is not the case.  The package
% therefore does its best, but results may be sub-optimal.  To get
% consistent results, either define a new \emph{single} unit, or keep
% compound units inside \cs{SI} and \cs{si}.
%\begin{LaTeXexample}
%  \metre\per\micro\second \\
%  \si{\metre\per\micro\second} \\
%  \newunit{\myunit}{\metre\per\micro\second}
%  \myunit
%  \newunit{\myunittwo}{\kilogram\myunit} \\
%  \myunittwo \\
%  \kilogram\myunit
%\end{LaTeXexample}
% Notice the difference in behaviour of \cs{per} in the first two
% lines, and the spacing error on the last line in the example.
%
%\subsection{How do I set superscripts to use lining numbers?}
% Lowercase (``old style'') numbers are favoured by many people for
% use in running text.  However, this does not necessarily look good
% in superscripts. The mode used to typeset data can be varied, so
% that maths mode numbers are used for the unit part of the output.
%\begin{LaTeXexample}
%  \SI[mode=text]{1234}{\metre\per\second}\\
%  \SI[valuemode=text,unitmode=maths]{1234}{\metre\per\second}
%\end{LaTeXexample}
%
%\subsection{Why do most of the examples use
%  \texorpdfstring{\si{\joule\per\mole\per\kelvin}}{J mol-1 K-1}?}
% The package author is a chemist, and this is the unit of entropy
% (disorder).  It nicely demonstrates the use of the \cs{per} macro,
% and so it crops up a lot.  It is also in the subsection heading
% here to act as a test with \pkg{hyperref} and moving arguments!
%
%\subsection{What can \numprint do that \currpkg cannot?}
% \currpkg uses a lot of ideas from \numprint: a reasonable amount of
% the number-processing code here is based on that in \numprint.
% However, the two packages have somewhat different aims, and as a
% result there are things that \numprint can do that \currpkg does
% not implement.  The main features of \numprint not available here
% are:
%\begin{itemize}
%  \item General support for numbers with base other than $10$ (see
%    \pkg{nbaseprt});
%  \item Alignment of the decimal marker of powers in tables;
%  \item Alignment of numbers in running text;
%  \item Specific formatting commands for \TeX\ counters and lengths.
%\end{itemize}
%
%\section{Tricks and known issues}
%
%\subsection{Ensuring maths mode}
% Due to the possibility of output in either maths or text mode, any
% input which requires a particular mode needs to be protected.  You
% cannot use |$|\ldots|$|, as this can get ``caught out'', but also
% as it may give hard-to-follow errors.  Always use \cs{ensuremath}
% to force maths processing, and \cs{text} (from the \AMS\TeX\
% bundle) to ensure text mode.
%
%\subsection{Using \texttt{.} and fixed spaces in units}
% To use a literal |.| in a unit, it has to be within an extra set of
% braces.  This does not need any extra protection, unlike the
% situation with \SIstyle (for example, no \cs{text} macro is
% needed).  The fixed space (|~|) is more problematic: set
% \opt{unitspace=space} to get a full space here.
%\begin{LaTeXexample}
%  \newunit[unitspace=space]{\myunit}{V~vs{.}~NHE}%
%  \SI{10}{\myunit}
%\end{LaTeXexample}
%
%\subsection{Passing unprocessed digits through an \texttt{S}
%  column}
% The method used to detect numbers in an |S| column will pick up
% material wrapped inside braces if there is more than a single
% character inside the braces.  If you want to pass a \emph{single}
% numerical character without processing it, you need two sets of
% braces (\ref{tab:S-limits}).
%\begin{LaTeXexample}[float]
%  \begin{table}
%    \centering
%    \caption{Passing single digit characters}
%    \label{tab:S-limits}
%    \begin{tabular}{S[colourall,colour=orange]}
%      \toprule
%        {Heading} \\
%      \midrule
%        {1-2} \\
%        {1}   \\
%        {{1}} \\
%        {{-}} \\ % Using {-} gives an error!
%      \bottomrule
%    \end{tabular}
%  \end{table}
%\end{LaTeXexample}
%
%\subsection{Limitations of \cs{mathrm}}
% The package uses the \cs{mathrm} font family by default to typeset
% output in maths mode.  This however has a few side-effects.  For
% example, the Greek alphabet can give odd results.\footnote{This
% depends on your font setup; this document uses T1 encoding, which
% shows the issue, whereas using OT1 does not.}  The use of the
% \cs{mathnormal} font \emph{may} get around this issue.
%\begin{LaTeXexample}
%  \num[numaddn=\pi]{4\pi e-7}\\
%  \num[numaddn=\pi,mathsrm=mathnormal]{4\pi e-7}
%\end{LaTeXexample}
% On the other hand, you may want to use text mode, in which case
% \cs{ensuremath} is needed.  Depending on the exact circumstances,
% the \LaTeX\ protection mechanism (\cs{DeclareRobustCommand}) may be
% sufficient; in some cases, this will fail and the \eTeX\
% \cs{protected} system may be required. There are several potential
% pitfalls in this area; experimentation may well be needed.
%\begin{LaTeXexample}
%  \DeclareRobustCommand*{\numpi}{\ensuremath{\pi}}
%  \num[numaddn=\numpi,mode=text]{4\numpi e-7}
%\end{LaTeXexample}
%
%\subsection{Entire document in sans serif font}
% If your entire document is not in a Roman font, using
% the font detection system is not the most efficient method
% for setting the \currpkg output.  Instead, the \opt{mathrm}
% and \opt{textrm} package options can be redefined.
%\begin{LaTeXexample}
%  \sffamily
%  Some text \\
%  \sisetup{obeyfamily=false,mathrm=mathsf,textrm=sffamily}
%  \num{1e2} \\
%  \SI{3}{\newton}
%  \[ \SI{4e5}{\pascal} \]
%\end{LaTeXexample}
%
%\subsection{Effects of emulation}
% The package has been designed so that almost everything can be set
% using the options.  In the emulation code, some internal macros are
% redefined.  This is because the legacy packages do odd things,
% which are deliberately not implemented by \currpkg.  Using an
% emulation file will prevent subsequent loading of the real package.
% This is to prevent errors or, worse, difficult to diagnose changes
% to output.
%
%\subsection{Centring columns on non-decimal input}
% The \pkg{dcolumn} manual suggests using that package to align a
% column on a $\pm$ sign. The same type of output is possible using
% \currpkg, but some care is needed (\ref{tab:dcolumn}).  Odd things
% may happen: use with care!
%\begin{LaTeXexample}[float]
%  \begin{table}
%    \centering
%    \caption{Non-standard \texttt{S} column}
%    \label{tab:dcolumn}
%    \begin{tabular}{%
%      S[digitsep=none,decimalsymbol={\,\pm\,},
%        numdigits={0123456789.},numdecimal=+]}
%    \toprule
%    {Some Values} \\
%    \midrule
%    2.3456  + 0.02  \\
%    34.2345 + 0.001 \\
%    56.7835 + 0.067 \\
%    90.473  + 0.021 \\
%    \bottomrule
%    \end{tabular}
%  \end{table}
%\end{LaTeXexample}
%
%\subsection{Expanding content in tables}
% When processing |S| columns, \currpkg works hard to expand any
% items which may give numbers.  So for example you can define values
% as macros (\ref{tab:vmacros}). To avoid the expansion of single
% macros, you must either wrap them in two sets of braces or make
% them robust (using either \cs{DeclareRobustCommand} or
% \cs{protected}).
%\begin{LaTeXexample}[float]
%  \begin{table}
%    \centering
%    \caption{Values as macros}
%    \label{tab:vmacros}
%    \newcommand*{\myvala}{1.234}%
%    \newcommand*{\myvalb}{20.345}%
%    \newcommand*{\myvalc}{0.987654}%
%    \DeclareRobustCommand*{\myvald}{88.88}%
%    \begin{tabular}{S[tabtextalign=left]}
%    \toprule
%    {Some Values} \\
%    \midrule
%    \myvala  \\
%    \myvalb \\
%    \myvalc \\
%    {{\myvala}} \\
%    \myvald \\
%    \bottomrule
%    \end{tabular}
%  \end{table}
%\end{LaTeXexample}
%
% It is possible to use calculated values in tables. For this to
% work, the calculation must take place before attempting to parse
% the number. This means that any calculation code should be wrapped
% in braces and appear before the use of the number. This is
% illustrated by using the \pkg{datatool} package to set up a
% miniature database, and then output the values doubled
% (\ref{tab:calc}). This example also shows that macros such as
% \cs{DLTforeach}, which construct a table while \TeX\ is running,
% cannot be placed inside an |S| column (or an |s| one). Instead, an
% additional dummy column has been added with no inter-column space.
% This is used to contain the tale-building macro.
%\begin{LaTeXexample}[float]
%  \DTLnewdb{data}
%  \DTLnewrow{data}\DTLnewdbentry{data}{value}{66.7012}
%  \DTLnewrow{data}\DTLnewdbentry{data}{value}{66.0212}
%  \DTLnewrow{data}\DTLnewdbentry{data}{value}{64.9026}
%  \begin{table}
%    \caption{Calculated values}
%    \label{tab:calc}
%    \centering
%    \sisetup{tabformat=2.4}
%    \begin{tabular}{SS@{}l}
%      \toprule
%      {Value} & {Doubled}  &
%      \DTLforeach{data}{\myvalue=value}{%
%        \DTLiffirstrow {\\ \midrule}{\\}%
%        \myvalue & % First column
%        {\DTLmul{\myvalue}{\myvalue}{2}} \myvalue % second column
%        & }\\
%      \bottomrule
%    \end{tabular}
%  \end{table}
%\end{LaTeXexample}
%
%\subsection{Adding items after the last column of a tabular}
% If you use an |S| or |s| column as the last one in a tabular, and
% you use the \pkg{array} ``|<|'' construction to add items after it,
% the spacing may be wrong.  This will occur if the column contents
% are of differing widths.  Changing the \LaTeX\ |\\| line ending for
% the plain \TeX\ \cs{cr} will give the correct spacing, but does not
% allow adjustment of inter-row distance (\ref{tab:cr}).\footnote{For
% the \TeX\ experts, the issue here is that the system to gather up
% cell contents is added in using the \texttt{<} construction.
% Normally, this comes after the cell contents and any other
% \texttt{<} arguments, so collects the user additions.  However, in
% the last cell the contents include \texttt{\bslash\bslash}, which
% is converted to \cs{cr} before gathering can occur.  By using
% \cs{cr} directly, the gathering process receives all of the cell
% contents as normal.}  In most cases, this should not be a serious
% issue. Notice that an extra set of braces is needed here, compared
% to usual \LaTeX\ use; this is to prevent any expansion of the
% material by \currpkg.
%\begin{LaTeXexample}[float]
%  \begin{table}
%    \caption{Correcting spacing in last \texttt{S} column}
%    \label{tab:cr}
%    \hfil
%    \begin{tabular}{S<{{\,\si{\kg}}}S<{{\,\si{\kg}}}}
%      \toprule
%      \multicolumn{1}{c}{Long header} &
%        \multicolumn{1}{c}{Long header} \\
%      \midrule
%      1.23 & 1.23 \\
%      4.56 & 4.56 \\
%      7.8  & 7.8  \\
%      \bottomrule
%    \end{tabular}
%    \hfil
%    \begin{tabular}{S<{{\,\si{\kg}}}S<{{\,\si{\kg}}}}
%      \toprule
%      \multicolumn{1}{c}{Long header} &
%        \multicolumn{1}{c}{Long header} \\
%      \midrule
%      1.23 & 1.23 \cr
%      4.56 & 4.56 \cr
%      7.8  & 7.8  \cr
%      \bottomrule
%    \end{tabular}
%    \hfil
%  \end{table}
%\end{LaTeXexample}
%
%\subsection{Using \currpkg\ with the \pkg{cellspace} package}
% Both \currpkg\ and \pkg{cellspace} use the letter \texttt{S} for a 
% new column type. This obviously leads to a problem. If both are 
% loaded, \currpkg\ will retain the \texttt{S} column, and moves the
% functionality of \pkg{cellspace} to the letter \texttt{C}. This 
% allows the normal use of \pkg{cellspace} with standard column types:
% it does \emph{not} work with the \currpkg\ \texttt{S} or \texttt{s}
% columns.
% 
%\subsection{Using \currpkg\ with the \pkg{mathabx} package}
%\changes{v1.1g}{2008/11/26}{Respect loading of \pkg{mathabx}}
% The \pkg{mathabx} package defines its own \cs{second} and 
% \cs{degree} macros.  This is respected by \currpkg: if you want the
% \currpkg\ definitions instead, put the lines
%\begin{LaTeXexample}[noexample]
%  \makeatletter
%  \renewunit{\second}{s}
%  \renewunit[valuesep=none]{\degree}{\si@sym@degree}
%  \makeatother
%\end{LaTeXexample}
% after |\begin{document}|.
%
%\iffalse
%<*bug>
%\fi
%\section{Reporting a problem}
% \currpkg is quite long and complicated, and works hard to cover all
% possible eventualities.  However, there will be bugs in the code
% and unexpected interactions with other packages.  If you think you
% have found a bug, please report it.  A short test-case
% demonstrating the problem would be very welcome.  The following is
% a suitable template, and is available as \file{si-bug.ltx} in the
% \texttt{doc/latex/siunitx} directory or by running the \ext{dtx} or
% \ext{ins} file through \TeX.
%    \begin{macrocode}
\listfiles
\documentclass{article}
%    \end{macrocode}
% Load other packages needed here.
%    \begin{macrocode}
\usepackage{siunitx}
%    \end{macrocode}
% Normally, debugging the load procedure will not be needed; the
% debug option here means that all run-time information is logged.
%    \begin{macrocode}
\sisetup{debug}
\begin{document}
This is the bug test-case document for the \textsf{siunitx}
package.\\
Please put your demonstration here, and e-mail to the package
author.
\begin{center}
  \texttt{joseph.wright@morningtar2.co.uk}
\end{center}
\end{document}
%    \end{macrocode}
%\iffalse
%</bug>
%\fi
%
%\section{Feature requests}
% Feature requests for \currpkg are welcome.  The package maintainer
% will consider any ideas within the remit of the package (units and
% values).  If suggesting a new feature, an example of how it should
% work would be appreciated.  If new controls are needed, some
% suggestions for option names would be welcome.
%
%\section{Acknowledgements}
% Many thanks indeed to Stefan Pinnow, who has made a very large
% number of suggestions and found numerous bugs in the package.  His
% contribution to the package has been vital. The package author has
% learned \LaTeX\ tricks from far too many people to thank all of
% them.  However, for this package specific thanks must go to the
% authors of the existing ``unit'' packages: Danie Els (\SIstyle),
% Marcel Heldoorn (\SIunits), Patrick Happel (\unitsdef), Axel
% Reichert (\units) and Harald Harders (\numprint).  Will Robertson
% and Heiko Oberdiek deserve much credit for demonstrating \LaTeX\
% coding best practice. Victor Eijkhout's excellent (and free)
% \emph{\TeX\ by Topic} has provided some useful coding hints
% \cite{Eijkhout}.  The idea for combining and extending unit
% provision in \LaTeX\ was heavily inspired by Philip Lehmann's
% \pkg{biblatex}. Thanks to the various contributors of ideas for the
% package: Donald Arseneau, Michele Dondi, Paul Gans, Ben Morrow, Lan
% Thuy Pham, Alan Ristow, Patrick Heinze, Andrea Blomenhofer, Morten
% H{\o}gholm, Burkhard Moddemann  and Patrick Steegstra.
%
%\clearpage
%\part{Correct application of (SI) units}
%\section{Background}
% Consistent and logical units are a necessity for scientific work,
% and have applicability everywhere.  Historically, a number of
% systems have been used for physical units.  SI units were
% introduced by the \latin{Conf\'erence G\'en\'erale des Poids et
% Mesures} (CGPM) in 1960.  SI units are a coherent system based on
% seven base units, from which all other units may be derived.
%
% At the same time, physical quantities with units are mathematical
% entities, and as such way that they are typeset is important.  In
% mathematics, changes of type (such as using bold, italic, sans
% serif typeface and so on) convey information.  This means that
% rules exist not only for the type of units to be used under the SI
% system, but also the way they should appear in print.  Advice on
% best practice has been made available by the \emph{National
% Institute of Standards and Technology} (NIST) \cite{NIST}.
%
% As befits an agreed international standard, the full rules are
% detailed. It is not appropriate to reproduce these in totality
% here; instead, a useful summary of the key points is provided. The
% full details are available from the \latin{Bureau International des
% Poids et Mesures} (BIPM) in French \cite{BIPM:FR} and English
% \cite{BIPM:EN}.  They also publish a very useful and detailed guide
% to using units, values and so on, available online in a number of
% different formats \cite{BIPM:PDF}.
%
% \currpkg takes account of the information given here, so far as is
% possible.  Thus the package defaults follow the recommendations
% made for typesetting units and values.  Spacing and so forth is
% handled in such a way as to make implementing the rules
% (relatively) easy.
%
%\section{Units}
%\subsection{SI base units}
% There are seven base SI units, listed in \ref{tab:baseunits}. Apart
% from the kilogram, these are defined in terms of a measurable
% physical quantity needing the definition alone.\footnote{Some base
% units need others defined first; there is therefore a required
% order of definition.}  The base units have been chosen such that
% all physical quantities can be expressed using an appropriative
% combination of these units, needing no others and with no
% redundancy.  The kilogram is slightly different from the other base
% units as it is still defined in terms of a ``prototype'' held in
% Paris.\footnote{At the time of writing, this is under review and
% will be altered.}
%
%\subsection{SI derived units}
% All other units within the SI system are regarded as ``derived''
% from the seven base units.  At the most basic, all other SI units
% can be expressed as combinations of the base units.  However, many
% units (listed in \ref{tab:derived}, \ref{tab:experimental} and
% \ref{tab:accepted}) have a special name and symbol.\footnote{The
% nautical mile has a given name but no agreed symbol, and although
% accepted by the SI is not provided by \currpkg as a unit macro.}
% Most of these units are simple combinations of one or more base
% units (raised to powers as appropriate).  A small number of units
% derived from experimental data are allowed as SI units
% (\ref{tab:experimental}).
%
% Some of these units (in \ref{tab:accepted}) are regarded as only
% ``temporarily'' accepted, as the use of only the base and
% fully-consistent derived units in \ref{tab:derived} is encouraged.
% They are accepted as they are in common use in one or more
% disciplines; some are still very widespread in the appropriate
% areas.  These units are mainly multiples of base units (for
% example, a tonne is \SI{1000}{\kilo\gram}).
%
% One point to note is that ``unitless ratios'' are regarded as
% having base units which cancel out.  For example, the radian is
% regarded as having base unit \si{\metre\per\metre}.  The result of
% this division (``$1$'') is therefore regarded as a derived SI unit
% in this context.
%
%\subsection{SI prefixes}
% A series of SI prefixes for decimal multiples and submultiples are
% provided, and can be used as modifiers for any SI unit (either base
% or derived units) with the exception of the kilogram.  The prefixes
% are listed in \ref{tab:prefix}.  No space should be used between a
% prefix and the unit, and only a single prefix should be used.  Even
% the degree Celsius can be given a prefix, for example
% \SI{1}{\milli\celsius}.  The only exception to this rule is for
% degrees, minutes and seconds of an arc: \ang{1;2;3}.
%
% It is important to note that the kilogram is the only SI unit with
% a prefix as part of its name and symbol.  Only  single prefix may
% be used, and so in the case of the kilogram prefix names are used
% with the unit name ``gram'' and the prefix symbols are used with
% the unit symbol \si{\gram}.  For example $\SI{1e-6}{\kilogram} =
% \SI{1}{\milli\gram}$.
% 
% Three unit-prefix combinations deserve special note. The units
% hectare, kilohm and megohm elide the additional vowel (these
% would otherwise be hectoare, kiloohm and megaohm).  All other
% vowel combinations are retained.\footnote{Thanks to Charles
% Cameron for raising this issue with the package author.}
%
%\subsection{Other units}
% The application of SI units is meant to provide a single set of
% units which ensure consistency and clarity across all areas.
% However, other units are common is many areas, and are not without
% merit. The units provided by \currpkg by default do not included
% any of these; only units which are part of the SI set or are
% accepted for use with SI units are defined.  However, several other
% sets of units can be loaded as optional modules. The binary
% prefixes and units (Section~\ref{sec:binary} and \ref{tab:binary})
% are the most obvious example.  These are \emph{not} part of the SI
% specifications, but the prefix names are derived from those in
% \ref{tab:prefix}.
%
% Other units (such as those provided by the modules \opt{synchem},
% \opt{hep} and \opt{astro}) are normally to be avoided where
% possible.  SI units should, in the main, be preferred due to the
% advantages of clear definition and self-consistency this brings.
% However, there will probably always be a place for specialist or
% non-standard units.  This is particulary true of units derived from
% basic physical constants; for example reason, the \opt{hep} module
% defines the speed of light, \si{\clight}, as a unit.  For work in
% basic science, a small number of physical constants are recognised
% as units provided the results for comparison with experiment are
% given in SI units.
%
% There are also many areas where non-standard units are used so
% commonly that to do otherwise is difficult or impossible.  For
% example, most synthetic chemists measure the pressure inside vacuum
% apparatus in \si{\mmHg}, partly because the most common gauge for
% the task still uses a column of mercury metal.  For these reasons,
% \currpkg does define non-SI units.
%
%\section{Units and values in print}
%\subsection{Mathematical meaning}
% As explained earlier, a unit--value combination is a single
% mathematical entity.  This has implications for how both the number
% and the unit should be printed.  Firstly, the two parts should not
% be separated.  With the exception of the symbols for plane angles
% (\si{\degree}, \si{\arcmin} and \si{\arcsec}), it is usual to have
% a space between the unit and the value.  This should therefore be a
% non-breaking space between the two.  Different geographical areas
% have different conventions on the size of this space; a ``small''
% space (|\,|) is the \currpkg default.
%\begin{LaTeXexample}
%  A space for \SI{10}{\percent}\\
%  and also for \SI{100}{\celsius}\\
%  but not for \ang{1.23}.
%\end{LaTeXexample}
%
% The mathematical meaning of units also means that the shape, weight
% and family are important.  Units are supposed to be typeset in an
% upright, medium weight serif font.  Italic, bold and sans serif are
% all used mathematically to convey other meanings.  \currpkg package
% defaults again follow this convention: any local settings are
% ignored, and uses the current upright serif maths font.  However,
% there are occasions where this may not be the most desirable
% behaviour.  A classic example would be in an all-bold section
% heading.  As the surrounding text is bold, some people feel that
% any units should follow this.
%\begin{LaTeXexample}
%  Units should \textbf{not be bold: \SI{54}{\farad}}\\
%  \textbf{But perhaps in a running block,\\
%  it might look better: \SI[obeybold]{54}{\farad}}
%\end{LaTeXexample}
%
%\subsection{Unit multiplication and division}
% Symbols for units formed from other units by multiplication are
% indicated by means of either a half-height (that is, centred) dot
% or a (thin) space.  This document uses a half-height dot as
% (\emph{i}) this is the recommendation of \textsc{nist}, amongst
% others and (\emph{ii}) it avoids potential confusion between unit
% prefixes and multiplied units.
%\begin{LaTeXexample}
%  $\si{\metre\second} = \mbox{metre second}$ \\
%  $\si{\milli\second} = \mbox{millisecond}$ \\
%  \sisetup{unitsep=thin}
%  $\si{\metre\second} = \mbox{metre second}$ \\
%  $\si{\milli\second} = \mbox{millisecond}$
%\end{LaTeXexample}
% There are some circumstances under which it is permissable to omit
% any spaces.  The classic example is \si{\kWh}, where
% ``\si[unitsep=thin]{\kWh}'' does not add any useful information.
% If using such a unit repeatedly, users of \currpkg are advised to
% create a custom unit to ensure consistency.\footnote{\cs{kWh} and
% \cs{kilowatthour} are defined by \currpkg in this way.}
%
% Symbols for units formed from other units by division are indicated
% by means of a virgule (oblique stroke, slash, |/|), a horizontal
% line, or negative exponents.\footnote{Notice that a virgule and a
% solidus are not the same symbol.}  However, to avoid ambiguity, the
% virgule must not be repeated on the same line unless parentheses
% are used. This is ensured when using named unit macros in \currpkg,
% which will ``trap'' repeated division and format it correctly.  In
% complicated cases, negative exponents are to be preferred over
% other formats.
%\begin{LaTeXexample}
%  \si{\joule\per\mole\per\kelvin}\\
%  \si[per=fraction]{\joule\per\mole\per\kelvin}\\
%  \si[per=slash]{\joule\per\mole\per\kelvin}
%\end{LaTeXexample}
%
%\subsection{Repeating units}
% Products and errors should show what unit applies to each value
% given.  Thus \SI{2x3}{\metre} is an ordered set of lengths of a
% geometric area, whereas \SI[repeatunits=false]{2x3}{\metre} is a
% length (and equal to \SI{6}{\metre}). Thus, $\times$ is not a
% product but is a mathematical operator; in the same way, a $2
% \times 3$ matrix is not a $6$ matrix!  In some areas, areas and
% volumes are given with separated units but a unit raised to the
% appropriate power: \SI[repeatunits=power]{2 x 3}{\metre}. Although
% this does display the correct overall units, it is
% potentially-confusing and is not encouraged.
%
%\subsection{Clarity in writing values of quantities}
% Care must be taken when writing ranges of numbers.  For purely
% numerical values, it is common to use a en-dash between values, for
% example ``see pages 1--5''.  On the other hand, values with units
% could be misinterpret as negative values if written in this way. As
% the unit--value combination is a single mathematic entity, writing
% the values with an en-dash followed by a single unit is also
% incorrect.  As a result, using the word ``to'' is strongly
% recommended.
%\begin{LaTeXexample}
%  \SI{1}{\metre} to \SI{5}{\metre} long.
%\end{LaTeXexample}
%
%\subsection{Graphs and tables}
% In graphs and tables, repetition of the units following each entry
% or axis mark is confusing and repetitive.  It is therefore best to
% place the unit in the label part of the information.  Placing the
% unit in square brackets is common but mathematically
% poor.\footnote{For example, for an acceleration $a$, the expression
% $[a]$ is the dimensions of $a$, \ie length per time squared in this
% case.}  Much better is to show division of all values by the unit,
% which leaves the entries as unitless ratios.  This is illustrated
% in \ref{tab:label} and \ref{fig:label}.
%\begin{LaTeXexample}[float]
%  \begin{table}
%    \centering
%    \caption{An example of table labelling}
%    \label{tab:label}
%    \begin{tabular}{cS[tabformat=1.4,tabnumalign=centre]}
%      \toprule
%      {Entry} & {Length/\si{\metre}} \\
%      \midrule
%      1 & 1.1234 \\
%      2 & 1.1425 \\
%      3 & 1.7578 \\
%      4 & 1.9560 \\
%      \bottomrule
%    \end{tabular}
%  \end{table}
%\end{LaTeXexample}
%\begin{LaTeXexample}[float]
%  \begin{figure}
%    \centering
%    \begin{tikzpicture}
%      \begin{axis}[xlabel=$t/\si{\second}$,ylabel=$d/\si{\metre}$]
%        \addplot[smooth,mark=x]
%          plot coordinates {
%            (0,0)
%            (1,5)
%            (2,8)
%            (3,9)
%            (4,8)
%            (5,5)
%            (6,0)
%           };
%      \end{axis}
%    \end{tikzpicture}
%    \caption{An example of graph labelling}
%    \label{fig:label}
%  \end{figure}
%\end{LaTeXexample}
%
% In most cases, adding exponent values in the body of a table is
% less desirable than adding a fixed exponent to column headers.  An
% example is shown in \ref{tab:exp}.  The use of \cs{multicolumn} is
% needed here due to the ``|<|''; without \cs{multicolumn}, the
% titles are followed by ``\si{\kilo\gram}''!
%\begin{LaTeXexample}[float]
%  \begin{table}
%    \centering
%    \caption{Good and bad columns}
%    \label{tab:exp}
%    \sisetup{tabnumalign=centre}
%    \begin{tabular}{%
%      c%
%      S[tabformat=1.3e1]<{\,\si{\kilogram}}%
%      S[tabformat=2.2]}
%      \toprule
%      Entry & \multicolumn{1}{c}{Mass} &
%        {Mass/\SI{e3}{\kilogram}} \\
%      \midrule
%      1 & 4.56e3  & 4.56 \\
%      2 & 2.40e3  & 2.40 \\
%      3 & 1.345e4 & 13.45 \\
%      4 & 4.5e2   & 0.45 \\
%      \bottomrule
%    \end{tabular}
%  \end{table}
%\end{LaTeXexample}
%
%\bibliographystyle{unsrt}
%\StopEventually{%
%  \clearpage
%  \part{Notes}
%  \PrintChanges
%  \PrintIndex
%  \bibliography{siunitx}}
%
% \iffalse
%<*package>
% \fi
%
%\clearpage
%\part{Implementation}
%\section{Main package}
% Much of the code here is taken, with little or no modification,
% from the existing packages.  These are all released under the LPPL,
% and so this use is entirely allowed.  Rather than confuse the
% source here with repeated references, note that code here could be
% copied from \SIstyle, \SIunits, \numprint, \unitsdef or \units.
% Some ideas have also been borrowed from \pkg{biblatex}; again these
% will not be specifically noted. Code from other packages will be
% marked when used.
%
% User-space commands (those not containing |@|) defined here should
% give the same result as macros with the same name in the older
% packages.\footnote{Although extra optional arguments may be added.}
% However, internal package macros will behave differently; if the
% user has redefined internal macros, then compatibility will be
% impaired.
%
% The code used here uses \LaTeX\ rather than \TeX\ commands where
% possible.\footnote{This applies to \LaTeX\ kernel commands only;
% for example, \pkg{ifthenelse} is not used.}  For example,
% \cs{newcommand*} is used in place of \cs{def}, unless custom
% parameters are needed.  Hopefully, this will aid future
% maintenance.  Grouping is used where possible to limit the scope of
% temporary assignments.
%
%\subsection{Setup code}
% As always, the package starts with identification.
%    \begin{macrocode}
\NeedsTeXFormat{LaTeX2e}
\ProvidesPackage{siunitx}
  [2008/12/22 v1.1j A comprehensive (SI) units package]
%    \end{macrocode}
% The package requires \eTeX, so the usual test is made.
%    \begin{macrocode}
\begingroup
  \@ifundefined{eTeXversion}
    {\PackageError{siunitx}
       {Not running under e-TeX}
       {This package requires e-TeX. Try compiling the document
        with\MessageBreak `elatex' instead of `latex'. When
        using pdfTeX, try `pdfelatex'\MessageBreak instead of
        `pdflatex'}%
     \endgroup\endinput}
    {\endgroup}
%    \end{macrocode}
%\begin{macro}{\si@catcodes}
% There are circumstances under which some odd category codes might
% be in place.  The category codes for |{|, |}|, |[|, |]| and |#| are
% assumed to be okay; if issues arise, this can be altered.  \LaTeX\
% will have set |@| as a letter on loading \currpkg.
%    \begin{macrocode}
\edef\si@catcodes{%
  \catcode\string`\string ` \the\catcode\string`\`\relax
  \catcode\string`\string = \the\catcode\string`\=\relax
  \catcode\string`\string ^ \the\catcode\string`\^\relax
  \catcode\string`\string _ \the\catcode\string`\_\relax
  \catcode\string`\string ~ \the\catcode\string`\~\relax
  \catcode\string`\string : \the\catcode\string`\:\relax
  \catcode\string`\string - \the\catcode\string`\-\relax
  \catcode\string`\string + \the\catcode\string`\+\relax
  \catcode\string`\string ; \the\catcode\string`\;\relax
  \catcode\string`\string , \the\catcode\string`\,\relax
  \catcode\string`\string . \the\catcode\string`\.\relax}
\catcode\string`\` 12\relax
\catcode`\= 12\relax
\catcode`\^ 7\relax
\catcode`\_ 8\relax
\catcode`\~ \active\relax
\@makeother{\:}
\@makeother{\-}
\@makeother{\+}
\@makeother{\;}
\@makeother{\,}
\@makeother{\.}
%    \end{macrocode}
%\end{macro}
% Packages needed for functionality are loaded.  \pkg{xkeyval}
% handles the package options, while \pkg{amstext} from the \AMS\
% bundle is needed for \cs{text}.  \pkg{array} is needed for the new
% column type for tabular material.  \pkg{xspace} provides ``magic''
% spacing after macros, if requested.
%    \begin{macrocode}
\RequirePackage{xkeyval}[2005/05/07]
\RequirePackage{amstext,array,xspace}
%    \end{macrocode}
%\begin{macro}{\si@tempa}
%\begin{macro}{\si@tempb}
%\begin{macro}{\si@tempc}
% Some scratch commands are defined; apart from where a known value
% is carried through, these could contain anything.
%    \begin{macrocode}
\newcommand*{\si@tempa}{}
\newcommand*{\si@tempb}{}
\newcommand*{\si@tempc}{}
%    \end{macrocode}
%\end{macro}
%\end{macro}
%\end{macro}
%\begin{macro}{\ifsi@switch}
% Various items will need a switch.  To avoid name pollution, a
% single switch is defined here; grouping will keep the definition
% local.
%    \begin{macrocode}
\newif\ifsi@switch
%    \end{macrocode}
%\end{macro}
%\begin{macro}{\si@tempboxa}
%\begin{macro}{\si@tempboxb}
%\begin{macro}{\si@tempboxc}
%\begin{macro}{\si@tempboxd}
% Some boxes are also needed.
%    \begin{macrocode}
\newbox\si@tempboxa
\newbox\si@tempboxb
\newbox\si@tempboxc
\newbox\si@tempboxd
%    \end{macrocode}
%\end{macro}
%\end{macro}
%\end{macro}
%\end{macro}
%\begin{macro}{\si@temptoks}
% A token register is also handy.
%    \begin{macrocode}
\newtoks{\si@temptoks}
%    \end{macrocode}
%\end{macro}
%\begin{macro}{\si@packagecheck}
%\begin{macro}{\si@blockpkgs}
%\begin{macro}{\si@checkpkgs}
% As \currpkg is intended to replace the other unit-management
% packages, these are tested for before any further processing.  If
% any are loaded, the package halts compilation; name clashes or
% unexpected results could occur if this is not tested.  Notice that
% \SIunits and \SIstyle could be loaded with variable capitalisation
% (at least on Windows); both possibilities are tested.
%    \begin{macrocode}
\newcommand*{\si@blockpkgs}{SIunits,sistyle,siunits,SIstyle,%
  unitsdef,fancyunits}
\newcommand*{\si@checkpkgs}{units,hepunits,fancynum}
\newcommand*{\si@packagecheck}{%
  \begingroup
  \@for\si@tempa:=\si@blockpkgs\do{
    \@ifpackageloaded{\si@tempa}
      {\PackageError{siunitx}
         {Package `\si@tempa' incompatible}
         {The \si@tempa\space package and siunitx are
          incompatible.\MessageBreak Use the
          `emulate=\si@tempa' package option when loading
          siunitx}}
      {}}
%    \end{macrocode}
% Some packages should not cause a clash, but are emulated and would
% be better handled that way.
%    \begin{macrocode}
  \@for\si@tempa:=\si@checkpkgs\do{%
    \@ifpackageloaded{\si@tempa}
      {\PackageWarning{siunitx}
         {Consider loading the siunitx package
          with\MessageBreak option `emulate=\si@tempa', rather
          than\MessageBreak loading both \si@tempa\space and
          siunitx}}
      {}}
  \endgroup}
%    \end{macrocode}
% The check is carried out on loading and at the beginning of the
% document, so that packages loaded both before and after \currpkg
% are caught.
%    \begin{macrocode}
\si@packagecheck
\AtBeginDocument{\si@packagecheck}
%    \end{macrocode}
%\end{macro}
%\end{macro}
%\end{macro}
%\begin{macro}{\si@ifdefinable}
%\darg{macro}
% Using \cs{@ifdefinable} to check macro definitions gives a generic
% error.  To give something more helpful, \cs{@ifundefined} is used,
% but this needs some \cs{expandafter} work.  This way it can also be
% used as a form of \cs{@ifundefined} for macro names.
%    \begin{macrocode}
\newcommand*{\si@ifdefinable}[1]{%
  \expandafter\expandafter\expandafter\@ifundefined%
    \expandafter\expandafter\expandafter%
      {\expandafter\@gobble\string#1}}
%    \end{macrocode}
%\end{macro}
%\begin{macro}{\si@addtolist}
%\darg{macro}
%\darg{items}
% It is quite useful to be able to add to a comma-separated list of
% expandable items.
%    \begin{macrocode}
\newcommand*{\si@addtolist}[2]{%
  \ifx\@empty#1\@empty
    \edef#1{#2}%
  \else
    \edef#1{#1,#2}%
  \fi}
%    \end{macrocode}
%\end{macro}
%\begin{macro}{\si@addtocsname}
%\darg{csname}
%\darg{tokens}
% A second item to add to a command sequence.
%    \begin{macrocode}
\newcommand*{\si@addtocsname}[2]{%
  \@ifundefined{#1}
    {\expandafter\gdef\csname #1\endcsname{#2}}
    {\si@temptoks\expandafter\expandafter\expandafter{%
       \csname #1\endcsname#2}%
     \expandafter\xdef\csname #1\endcsname{\the\si@temptoks}}}
%    \end{macrocode}
%\end{macro}
%\begin{macro}{\si@ifmtarg}
%\begin{macro}{\si@xifmtarg}
%\begin{macro}{\si@ifnotmtarg}
% To keep down dependance on other packages, the very short code
% block from \pkg{ifmtarg} is copied here with an internal name.
%    \begin{macrocode}
\begingroup
  \catcode`\Q=3
  \long\gdef\si@ifmtarg#1{%
    \si@xifmtarg#1QQ\@secondoftwo\@firstoftwo\@nil}
  \long\gdef\si@xifmtarg#1#2Q#3#4#5\@nil{#4}
  \long\gdef\si@ifnotmtarg#1{%
    \si@xifmtarg#1QQ\@firstofone\@gobble\@nil}
\endgroup
%    \end{macrocode}
%\end{macro}
%\end{macro}
%\end{macro}
%\begin{macro}{\si@newrobustcmd}
%\begin{macro}{\si@newcommand}
%\begin{macro}{\si@newcmd}
%\begin{macro}{\si@xargdef}
%\darg{macro}
%\doarg{num-args}
%\doarg{default}
%\darg{def}
% Some more copying, this time from \pkg{etoolbox}.  Various macros
% need to be \emph{really} robust.  This is achieved using the \eTeX\
% \cs{protected} primitive in various places.  However, it would be
% nice to have a \cs{protected} version of \cs{newcommand}.
% \pkg{etoolbox} has code for that, but to avoid needing to load it,
% the necessary stuff is copied here.  The only changes form the
% original are names, and the use of \cs{newcommand*} for the
% \cs{si@newcommand} macro.
%    \begin{macrocode}
\@ifpackageloaded{etoolbox}
  {\let\si@newrobustcmd\newrobustcmd}
  {\protected\def\si@newrobustcmd{%
     \@ifstar
       {\let\l@ngrel@x\protected\si@newcommand}
       {\def\l@ngrel@x{\protected\long}\si@newcommand}}
   \newcommand*{\si@newcommand}[1]{\@testopt{\si@newcmd#1}0}
   \def\si@newcmd#1[#2]{%
     \@ifnextchar[%]
       {\si@xargdef#1[#2]}
       {\@argdef#1[#2]}}
   \long\def\si@xargdef#1[#2][#3]#4{%
     \@ifdefinable#1{%
       \expandafter\protected
       \expandafter\def
       \expandafter#1%
       \expandafter{%
         \expandafter\@testopt
         \csname\string#1\endcsname{#3}}%
       \expandafter\@yargdef
       \csname\string#1\endcsname\tw@{#2}{#4}}}}
%    \end{macrocode}
%\end{macro}
%\end{macro}
%\end{macro}
%\end{macro}
%
%\subsection{Logging}
%\begin{macro}{\ifsi@debug}
%\begin{macro}{\ifsi@logmin}
%\begin{macro}{\ifsi@lognone}
% To control logging, some new switches are declared.
%    \begin{macrocode}
\newif\ifsi@debug
\newif\ifsi@logmin
\newif\ifsi@lognone
%    \end{macrocode}
%\end{macro}
%\end{macro}
%\end{macro}
%\begin{macro}{\si@log@err}
%\darg{error}
%\darg{explanation}
% Some handy re-usable macros are defined here.  These all take names
% beginning  These pop up in various places.  First errors are
% handled.
%    \begin{macrocode}
\newcommand*{\si@log@err}[2]{%
  \ifsi@lognone\else
    \ifsi@logmin
      \PackageWarning{siunitx}{#1}%
    \else
      \PackageError{siunitx}{#1}{#2}%
    \fi
  \fi}
%    \end{macrocode}
%\end{macro}
%\begin{macro}{\si@log@warn}
%\begin{macro}{\si@log@inf}
%\darg{text}
% Then warnings and information.
%    \begin{macrocode}
\newcommand*{\si@log@warn}[1]{%
  \ifsi@lognone\else
    \ifsi@logmin\else
      \PackageWarning{siunitx}{#1}%
    \fi
  \fi}
\newcommand*{\si@log@inf}[1]{%
  \ifsi@lognone\else
    \ifsi@logmin\else
      \PackageInfo{siunitx}{#1}%
    \fi
  \fi}
%    \end{macrocode}
%\end{macro}
%\end{macro}
%\begin{macro}{\si@log@debug}
%\darg{debug-info}
% The debug macro only gives output if the appropriate package option
% is set.
%    \begin{macrocode}
\newcommand*{\si@log@debug}[1]{%
  \ifsi@lognone\else
    \ifsi@debug
      \PackageInfo{siunitx}{#1}%
    \fi
  \fi}
%    \end{macrocode}
%\end{macro}
%
%\subsection{String comparison}
%\begin{macro}{\si@str@ifchrstr}
%\begin{macro}{\si@str@chrstr}
%\darg{char}
%\darg{string}
% At various points, the package needs to compare two strings, to
% find if one occurs in the other.  The first test is if a single
% character is part of a second string; this is used, for example, to
% check that a character is valid as input.  The first argument is
% not expanded further, but the second is two allow division into
% individual units.
%    \begin{macrocode}
\newcommand*{\si@str@ifchrstr}[2]{%
  \begingroup
    \si@switchfalse
    \renewcommand*{\si@tempa}{#1}%
    \protected@edef\si@tempb{#2}%
    \expandafter\si@str@chrstr\si@tempb\@empty\@empty\@empty
    \ifsi@switch
      \aftergroup\@firstoftwo
    \else
      \aftergroup\@secondoftwo
    \fi
  \endgroup}
\def\si@str@chrstr#1#2\@empty{%
  \renewcommand*{\si@tempc}{#1}%
  \ifx\si@tempa\si@tempc
    \expandafter\si@switchtrue
  \else
    \ifx\@empty#2\@empty\else
      \si@str@chrstr#2\@empty\@empty
    \fi
  \fi}
%    \end{macrocode}
%\end{macro}
%\end{macro}
%\begin{macro}{\si@str@ifonlychrs}
%\begin{macro}{\si@str@onlychrs}
%\darg{string}
%\darg{chars}
% The second test builds on the first.  Here, a check is made to see
% if the first string is made up only of characters from the second
% string.  In this case, the first string is expanded before
% testing.  The second string will be expanded by the internal
% character by character test.
%    \begin{macrocode}
\newcommand*{\si@str@ifonlychrs}[2]{%
  \begingroup
    \si@switchtrue
    \protected@edef\si@tempa{#1}%
    \renewcommand*{\si@tempb}{#2}%
    \expandafter\si@str@onlychrs\si@tempa\@empty\@empty\@empty
    \ifsi@switch
      \aftergroup\@firstoftwo
    \else
      \aftergroup\@secondoftwo
    \fi
  \endgroup}
\def\si@str@onlychrs#1#2\@empty{%
  \si@str@ifchrstr{#1}{\si@tempb}
    {}{\si@switchfalse}%
  \ifx\@empty#2\@empty\else
    \si@str@onlychrs#2\@empty\@empty
  \fi}
%    \end{macrocode}
%\end{macro}
%\end{macro}
%
%\subsection{Option handling}
%\begin{macro}{\sisetup}
%\darg{options}
% To allow modification of options at run time, a setup macro is
% provided.
%    \begin{macrocode}
\newcommand*{\sisetup}{\setkeys[si]{key}}
%    \end{macrocode}
%\end{macro}
%\begin{macro}{\si@opt@key}
%\darg{keyname}
%\darg{code}
% To aid maintenance, some shortcuts are defined for generating keys.
% These also allow the debugging messages to be added automatically
% to every key.  First of all the basic key definition.
%    \begin{macrocode}
\newcommand*{\si@opt@key}[2]{%
  \define@key[si]{key}{#1}
  {#2\si@log@debug{Option #1 set to ##1}}}
%    \end{macrocode}
%\end{macro}
%\begin{macro}{\si@opt@cmdkey}
%\doarg{default}
%\darg{keyname}
%\darg{function}
% For a single command key, a function must be specified.
%    \begin{macrocode}
\newcommand*{\si@opt@cmdkey}[3][]{%
  \define@cmdkey[si]{key}[si@]{#2}[#1]{#3}}
%    \end{macrocode}
%\end{macro}
%\begin{macro}{\si@opt@cmdkeys}
%\doarg{default}
%\darg{keyname}
% Whereas multiple definitions do not take a function.
%    \begin{macrocode}
\newcommand*{\si@opt@cmdkeys}[2][]{%
  \define@cmdkeys[si]{key}[si@]{#2}[#1]}
%    \end{macrocode}
%\end{macro}
%\begin{macro}{\si@opt@boolkey}
%\doarg{function}
%\darg{keyname}
% Keys which only take switch values; anything other than \opt{true}
% or \opt{false} will generate a warning from \pkg{xkeyval}.
%    \begin{macrocode}
\newcommand*{\si@opt@boolkey}[2][]{%
  \define@boolkey[si]{key}[si@]{#2}[true]
  {#1\si@log@debug{Option #2 set to ##1}}}
%    \end{macrocode}
%\end{macro}
%\begin{macro}{\si@opt@choicekey}
%\doarg{default}
%\darg{keyname}
%\darg{choices}
%\darg{function}
% A ``fill in the blanks'' choice key.  In all cases, \cs{si@tempa}
% is used to hold the value given to the key, so that \cs{ifx}
% testing can occur.
%    \begin{macrocode}
\newcommand*{\si@opt@choicekey}[4][]{%
  \define@choicekey*+[si]{key}{#2}[\si@tempa]{#3}[#1]
    {#4\si@log@debug{Option #2 set to ##1}}
    {\si@log@warn{Unknown value `##1' for option #2}}}
%    \end{macrocode}
%\end{macro}
%\begin{macro}{\si@opt@xchoicekey}
%\darg{keyname}
%\darg{choices}
%\darg{initial-value}
% Several of the package options can take either a choice from a list
% of known options, or a value to be interpreted literally.  To aid
% maintenance, the necessary code can be set up here.  These keys all
% define a new macro, which must exist.  The \cs{si@opt@xchoicekey}
% macro therefore ensures that this is defined, as well as setting up
% the \pkg{xkeyval} key.
%    \begin{macrocode}
\newcommand*{\si@opt@xchoicekey}[3]{%
  \define@choicekey*+[si]{key}{#1}[\si@tempa]{#2}[#1]
%    \end{macrocode}
% This code will execute if the option is on the list.  There will be
% a ``fixed'' macro with a matching name, which is used for this.
%    \begin{macrocode}
    {\si@log@debug{Option #1 set to ##1}%
     \expandafter\renewcommand\expandafter*\expandafter{%
      \csname si@#1\endcsname}{\@nameuse{si@fix@##1}}}
%    \end{macrocode}
% The user has given something that is not on the list as an
% argument.  It is used literally.
%    \begin{macrocode}
    {\si@log@debug{Option #1 set to ##1}%
     \expandafter\renewcommand\expandafter*\expandafter{%
      \csname si@#1\endcsname}{##1}}
%    \end{macrocode}
% Finally, the initial value of the macro is set up.
%    \begin{macrocode}
  \expandafter\newcommand\expandafter*\expandafter{%
    \csname si@#1\endcsname}%
    {\@nameuse{si@fix@#3}}}
%    \end{macrocode}
%\end{macro}
%\begin{macro}{\si@opt@compatkey}
%\darg{package}
%\darg{keyname}
% An all-in-one definition for a back-compatibility key.  These
% should only be used at load time, so are automatically disabled
% once the package is loaded.  Emulation is also automatically turned
% on.
%    \begin{macrocode}
\newcommand*{\si@opt@compatkey}[2]{%
  \define@boolkey[si]{key}[si@old@]{#2}[true]
  {\si@log@debug{Emulating #1 package option\MessageBreak #2}%
   \sisetup{emulate=#1}%
   \si@log@debug{Option #2 set to ##1}}
   \AtEndOfPackage{\si@opt@disablekey{#2}
     {Compatibility option #2 only\MessageBreak
        available when loading siunitx package}}}
%    \end{macrocode}
%\end{macro}
%\begin{macro}{\si@opt@disablekey}
%\darg{keyname}
%\darg{text}
% The ability to disable a key with a meaningful message is a must;
% the warning will come from \currpkg, and not from \pkg{xkeyval}.
%    \begin{macrocode}
\newcommand*{\si@opt@disablekey}[2]{%
  \key@ifundefined[si]{key}{#1}
    {}
    {\si@log@debug{Disabling key #1}%
     \si@opt@key{#1}{\si@log@warn{#2}}}}
%    \end{macrocode}
%\end{macro}
%
% The \pkg{xkeyval} package option for logging is declared. This is
% then processed to set the switches correctly.
%    \begin{macrocode}
\si@opt@choicekey[normal]{log}{debug,verbose,normal,errors,none}
%    \end{macrocode}
% A series of comparisons are made to assign the logging mode.  The
% \opt{normal} option is not tested, as executing the option sets the
% switches appropriately.
%    \begin{macrocode}
  {\si@debugfalse
   \si@logminfalse
   \si@lognonefalse
   \renewcommand*{\si@tempb}{none}%
   \ifx\si@tempa\si@tempb
     \si@lognonetrue
   \fi
   \renewcommand*{\si@tempb}{minimal}%
   \ifx\si@tempa\si@tempb
     \si@logmintrue
   \fi
   \renewcommand*{\si@tempb}{debug}%
   \ifx\si@tempa\si@tempb
     \si@debugtrue
   \fi
   \renewcommand*{\si@tempb}{verbose}%
   \ifx\si@tempa\si@tempb
     \si@debugtrue
   \fi}
%    \end{macrocode}
% A quick method to set \opt{log=debug}.
%    \begin{macrocode}
\si@opt@boolkey{debug}
%    \end{macrocode}
%\begin{macro}{\ifsi@strict}
% It would be useful to be able to disable some keys, when strict
% interpretation of the rules is desired.  This is a load-time
% option, and has to disable various options.
%    \begin{macrocode}
\si@opt@boolkey[%
  \ifsi@strict
    \sisetup{
      obeymode=false,
      obeybold=false,
      obeyitalic=false,
      mode=maths,
      repeatunits=true,
      trapambigerr=true,
      trapambigfrac=true}
    \@for\si@tempa:=obeyall,obeymode,obeyitalic,mode,unitmode,%
      valuemode,textmode,obeybold,repeatunits,trapambigerr,%
      trapambigfrac\do{%
      \begingroup
        \edef\si@tempb{\endgroup
          \noexpand\si@opt@disablekey{\si@tempa}
          {Option `\si@tempa' forbidden in strict mode}}%
      \si@tempb}
  \fi]{strict}
\AtEndOfPackage{
  \si@opt@disablekey{strict}
    {Option `strict' only available when\MessageBreak
     loading package}}
%    \end{macrocode}
%\end{macro}
%\begin{macro}{\si@emulate}
% The \opt{emulate} option is used for back-compatibility mode; the
% option is only valid when loading \currpkg.
%    \begin{macrocode}
\newcommand*{\si@emulate}{}
\si@opt@key{emulate}{\si@addtolist{\si@emulate}{#1}}
\AtEndOfPackage{
  \si@opt@disablekey{emulate}
    {Option `emulate' only available when\MessageBreak
     loading package}}
%    \end{macrocode}
%\end{macro}
%\begin{macro}{\si@unitsep}
%\begin{macro}{\si@unitspace}
%\begin{macro}{\si@errspace}
%\begin{macro}{\si@valuesep}
% The two \opt{\dots sep} options control the size of spaces between
% the number and the unit (\cs{si@valuesep}), and that used to
% represent a product (\cs{si@unitsep}).  Known values here are
% \opt{thin}, \opt{med}, \opt{medium}, \opt{thick}, \opt{cdot},
% \opt{tightcdot}\footnote{Both \cs{cdot}-based options only valid
% for \opt{unitsep}.} and \opt{none};\footnote{Only valid for
% \opt{valuesep}.} other entries will be treated as custom spaces.
%    \begin{macrocode}
\si@opt@xchoicekey{unitsep}
  {thin,med,medium,thick,none,comma,stop,fullstop,period,
    times,tighttimes,cdot,tightcdot}{thin}
\si@opt@xchoicekey{unitspace}{space,thin,med,medium,thick,
  none}{thin}
\si@opt@xchoicekey{errspace}{space,thin,med,medium,thick,
  none}{none}
\si@opt@xchoicekey{valuesep}
  {thin,med,medium,thick,none,comma,stop,fullstop,period,
    times,tighttimes,cdot,tightcdot}{thin}
%    \end{macrocode}
%\end{macro}
%\end{macro}
%\end{macro}
%\end{macro}
%\begin{macro}{\si@digitsep}
% Separation of digits in large numbers is controlled by the
% \opt{digitsep} option.  As with the other \opt{sep} values, this
% one has a choice of possible values.  The list is quite long, so
% that a range of options are handled automatically.  Notice that
% \opt{digitsep=none} will be used for no separation at all.
%    \begin{macrocode}
\si@opt@xchoicekey{digitsep}
  {thin,med,medium,thick,none,comma,stop,fullstop,period,
    times,tighttimes,cdot,tightcdot}{thin}
%    \end{macrocode}
%\end{macro}
%\begin{macro}{\si@decimalsymbol}
% The symbol used for the decimal position is varied here.  There are
% only two real options, but options are given for the name of a full
% stop.
%    \begin{macrocode}
\si@opt@xchoicekey{decimalsymbol}{comma,stop,fullstop,period,
  cdot,tightcdot}{fullstop}
%    \end{macrocode}
%\end{macro}
%\begin{macro}{\si@anglesep}
% The separator between degrees and minutes, and between minutes and
% seconds, when using \cs{ang}.
%    \begin{macrocode}
\si@opt@xchoicekey{anglesep}
    {thin,med,medium,thick,none,comma,stop,fullstop,period,
      times,tighttimes,cdot,tightcdot}{none}
%    \end{macrocode}
%\end{macro}
%\begin{macro}{\ifsi@obeymode}
% The first test for the font control is whether to respect the
% surrounding maths or text mode.
%    \begin{macrocode}
\si@opt@boolkey{obeymode}
%    \end{macrocode}
%\end{macro}
%\begin{macro}{\ifsi@numtextmode}
%\begin{macro}{\ifsi@unittextmode}
% The output of the package can be typeset using either text or maths
% mode fonts. Two switches are needed, for numbers and units.
%    \begin{macrocode}
\newif\ifsi@numtextmode
\newif\ifsi@unittextmode
%    \end{macrocode}
%\end{macro}
%\end{macro}
% The \opt{textmode} option has to set both flags.
%    \begin{macrocode}
\si@opt@choicekey[true]{textmode}{true,false}
  {\si@numtextmodefalse
   \si@unittextmodefalse
   \renewcommand*{\si@tempb}{true}%
   \ifx\si@tempa\si@tempb
     \si@numtextmodetrue
     \si@unittextmodetrue
   \fi}
%    \end{macrocode}
% The \opt{mode} option applies to numbers and units.
%    \begin{macrocode}
\si@opt@choicekey{mode}{math,maths,text}
  {\si@numtextmodefalse
   \si@unittextmodefalse
   \renewcommand*{\si@tempb}{text}%
   \ifx\si@tempa\si@tempb
     \si@numtextmodetrue
     \si@unittextmodetrue
   \fi}
%    \end{macrocode}
% Now the two options for numbers or units alone.
%    \begin{macrocode}
\si@opt@choicekey{valuemode}{math,maths,text}
  {\si@numtextmodefalse
   \renewcommand*{\si@tempb}{text}%
   \ifx\si@tempa\si@tempb
     \si@numtextmodetrue
   \fi}
\si@opt@choicekey{unitmode}{math,maths,text}
  {\si@unittextmodefalse
   \renewcommand*{\si@tempb}{text}%
   \ifx\si@tempa\si@tempb
     \si@unittextmodetrue
   \fi}
%    \end{macrocode}
%\begin{macro}{\ifsi@obeyfamily}
% The package can work to match the font family (serif, sans serif,
% typewriter) of the surrounding text.  This is controlled by a
% Boolean option.
%    \begin{macrocode}
\si@opt@boolkey{obeyfamily}
%    \end{macrocode}
%\end{macro}
%\begin{macro}{\ifsi@obeybold}
% The package can attempt to respect bold, or may ignore it.
%    \begin{macrocode}
\si@opt@boolkey{obeybold}
%    \end{macrocode}
%\end{macro}
%\begin{macro}{\ifsi@inlinebtext}
% For inline maths, two options for checking what is bold are
% available, the maths environment (\ie \cs{boldmath}) and the
% surrounding text (\cs{textbf} or \cs{bffamily}).
%    \begin{macrocode}
\newif\ifsi@inlinebtext
\si@opt@choicekey{inlinebold}{text,maths,math}
  {\si@inlinebtextfalse
   \renewcommand*{\si@tempb}{text}%
   \ifx\si@tempa\si@tempb
     \si@inlinebtexttrue
   \fi}
%    \end{macrocode}
%\end{macro}
%\begin{macro}{\ifsi@obeyitalic}
% Italic is slightly different to bold, as there is no convenient
% switch for maths.  Thus a choice key is used, with the appropriate
% check.
%    \begin{macrocode}
\si@opt@boolkey{obeyitalic}
%    \end{macrocode}
%\end{macro}
%\begin{macro}{\ifsi@detectdisplay}
% For handling display mathematics, a setting is needed for whether to
% treat it differently from other maths.
%    \begin{macrocode}
\si@opt@boolkey{detectdisplay}
%    \end{macrocode}
%\end{macro}
% The option to obey all font switching commands is Boolean-like but
% needs alternative setup.
%    \begin{macrocode}
\si@opt@choicekey[true]{obeyall}{true,false}
  {\si@obeyboldfalse
   \si@obeyitalicfalse
   \si@obeymodefalse
   \si@obeyfamilyfalse
   \renewcommand*{\si@tempb}{true}%
   \ifx\si@tempa\si@tempb
     \si@obeyboldtrue
     \si@obeyitalictrue
     \si@obeymodetrue
     \si@obeyfamilytrue
   \fi}
%    \end{macrocode}
%\begin{macro}{\si@valuemathsrm}
%\begin{macro}{\si@valuemathssf}
%\begin{macro}{\si@valuemathstt}
%\begin{macro}{\si@unitmathsrm}
%\begin{macro}{\si@unitmathssf}
%\begin{macro}{\si@unitmathstt}
% The fonts used by the package default to the obvious \LaTeX\ ones;
% however, this needs to be exposed to user modification.  First the
% maths mode fonts are sorted out.
%    \begin{macrocode}
\si@opt@cmdkeys{valuemathsrm,valuemathssf,valuemathstt}
\si@opt@cmdkeys{unitmathsrm,unitmathssf,unitmathstt}
%    \end{macrocode}
%\end{macro}
%\end{macro}
%\end{macro}
%\end{macro}
%\end{macro}
%\end{macro}
% To make life easier for the user, US spellings are provided for the
% maths keys.
%    \begin{macrocode}
\si@opt@key{valuemathrm}{\sisetup{valuemathsrm=#1}}
\si@opt@key{valuemathsf}{\sisetup{valuemathssf=#1}}
\si@opt@key{valuemathtt}{\sisetup{valuemathstt=#1}}
\si@opt@key{unitmathrm}{\sisetup{unitmathsrm=#1}}
\si@opt@key{unitmathsf}{\sisetup{unitmathssf=#1}}
\si@opt@key{unitmathtt}{\sisetup{unitmathstt=#1}}
%    \end{macrocode}
% The combined options are given, for setting numbers and units
% at the same time.
%    \begin{macrocode}
\si@opt@key{mathsrm}{\sisetup{valuemathsrm=#1,unitmathsrm=#1}}
\si@opt@key{mathssf}{\sisetup{valuemathssf=#1,unitmathssf=#1}}
\si@opt@key{mathstt}{\sisetup{valuemathstt=#1,unitmathstt=#1}}
\si@opt@key{mathrm}{\sisetup{valuemathsrm=#1,unitmathsrm=#1}}
\si@opt@key{mathsf}{\sisetup{valuemathssf=#1,unitmathssf=#1}}
\si@opt@key{mathtt}{\sisetup{valuemathstt=#1,unitmathstt=#1}}
%    \end{macrocode}
%\begin{macro}{\si@valuetextrm}
%\begin{macro}{\si@valuetextsf}
%\begin{macro}{\si@valuetexttt}
%\begin{macro}{\si@unittextrm}
%\begin{macro}{\si@unittextsf}
%\begin{macro}{\si@unittexttt}
% The same thing for text mode fonts.  Once again the default values
% are pretty obvious.
%    \begin{macrocode}
\si@opt@cmdkeys{valuetextrm,valuetextsf,valuetexttt}
\si@opt@cmdkeys{unittextrm,unittextsf,unittexttt}
\si@opt@key{textrm}{\sisetup{unittextrm=#1,valuetextrm=#1}}
\si@opt@key{textsf}{\sisetup{unittextsf=#1,valuetextsf=#1}}
\si@opt@key{texttt}{\sisetup{unittexttt=#1,valuetexttt=#1}}
%    \end{macrocode}
%\end{macro}
%\end{macro}
%\end{macro}
%\end{macro}
%\end{macro}
%\end{macro}
%\changes{v1.1}{2008/09/22}{Added \opt{numdiv} option}
%\begin{macro}{\si@numdigits}
%\begin{macro}{\si@numdiv}
%\begin{macro}{\si@numdecimal}
%\begin{macro}{\si@numexp}
%\begin{macro}{\si@numprod}
%\begin{macro}{\si@numgobble}
%\begin{macro}{\si@numsign}
%\begin{macro}{\si@numcloseerr}
%\begin{macro}{\si@numopenerr}
%\begin{macro}{\si@numaddn}
% The list of possible valid characters for parsing numbers is set
% up.  This is similar to \numprint, but with the \opt{extra} class,
% and with characters ignored with no output renamed as \opt{gobble}.
%    \begin{macrocode}
\si@opt@cmdkeys{numdigits,numdecimal,numexp,numgobble,numsign,%
  numcloseerr,numopenerr,numaddn,numprod,numdiv}
%    \end{macrocode}
%\end{macro}
%\end{macro}
%\end{macro}
%\end{macro}
%\end{macro}
%\end{macro}
%\end{macro}
%\end{macro}
%\end{macro}
%\end{macro}
%\begin{macro}{\si@numvalid}
%\begin{macro}{\si@numextra}
% The various valid characters are collected together in a single
% macro for later.  In common with the above macros, this one starts
% \cs{si@num\ldots}.  The order here is the order the values are
% tested later on.
%    \begin{macrocode}
\newcommand*{\si@numextra}{%
  \si@numopenerr\si@numcloseerr\si@numaddn}
\newcommand*{\si@numvalid}{\si@numgobble\si@numexp\si@numsign
  \si@numdecimal\si@numdigits\si@numextra\si@numprod\si@numdiv}
%    \end{macrocode}
%\end{macro}
%\end{macro}
%\begin{macro}{\ifsi@seperr}
%\begin{macro}{\ifsi@trapambigerr}
%\begin{macro}{\si@openerr}
%\begin{macro}{\si@closeerr}
% An option to control whether numerical error values are printed with
% or separate from the number.
%    \begin{macrocode}
\si@opt@boolkey{seperr}
\si@opt@boolkey{trapambigerr}
\si@opt@cmdkeys{openerr,closeerr}
%    \end{macrocode}
%\end{macro}
%\end{macro}
%\end{macro}
%\end{macro}
%\begin{macro}{\ifsi@sepfour}
% With four digits in a number, separating may or may not be desired.
% Note that this option is the same as one for \numprint.
%    \begin{macrocode}
\si@opt@boolkey{sepfour}
%    \end{macrocode}
%\end{macro}
%\begin{macro}{\ifsi@retainplus}
% An option to keep an explicit positive sign.
%    \begin{macrocode}
\si@opt@boolkey{retainplus}
%    \end{macrocode}
%\end{macro}
%\begin{macro}{\si@expbase}
%\begin{macro}{\si@expproduct}
% The options for exponents are set up.
%    \begin{macrocode}
\si@opt@xchoicekey{expproduct}{times,tighttimes,
  cdot,tightcdot}{times}
\si@opt@xchoicekey{expbase}{ten}{ten}
%    \end{macrocode}
%\end{macro}
%\end{macro}
%\begin{macro}{\ifsi@allowzeroexp}
% The package normally prevents \num[allowzeroexp]{e0}.
%    \begin{macrocode}
\si@opt@boolkey{allowzeroexp}
%    \end{macrocode}
%\end{macro}
%\begin{macro}{\si@prefixproduct}
% The marker for multiplication in prefixes.
%    \begin{macrocode}
\si@opt@xchoicekey{prefixproduct}{times,tighttimes,cdot,
  tightcdot,none}{times}
%    \end{macrocode}
%\end{macro}
%\begin{macro}{\si@prefixbase}
% In the same area, the power for prefixes is variable.  Here, two
% choices ar needed.
%    \begin{macrocode}
\si@opt@xchoicekey{prefixbase}{ten,two}{ten}
%    \end{macrocode}
%\end{macro}
%\begin{macro}{\ifsi@prefixsymbolic}
% Unit prefixes can be given as either symbols or numerically.
%    \begin{macrocode}
\si@opt@boolkey{prefixsymbolic}
%    \end{macrocode}
%\end{macro}
%\begin{macro}{\ifsi@num@padlead}
%\begin{macro}{\ifsi@num@padtrail}
% A setting is needed to indicate when to add zeros to decimal
% numbers, either before the decimal marker (|.1| giving
% ``\num[padnumber=all]{.1}'') or after (|1.| giving
% ``\num[padnumber=all]{1.}'').
%    \begin{macrocode}
\newif\ifsi@num@padlead
\newif\ifsi@num@padtrail
\si@opt@choicekey[all]{padnumber}
  {leading,lead,trailing,trail,all,both,true,none,false}
  {\si@num@padleadfalse
   \si@num@padtrailfalse
   \renewcommand*{\si@tempb}{leading}%
   \ifx\si@tempa\si@tempb
     \si@num@padleadtrue
   \fi
   \renewcommand*{\si@tempb}{lead}%
   \ifx\si@tempa\si@tempb
     \si@num@padleadtrue
   \fi
   \renewcommand*{\si@tempb}{trailing}%
   \ifx\si@tempa\si@tempb
     \si@num@padtrailtrue
   \fi
   \renewcommand*{\si@tempb}{trail}%
   \ifx\si@tempa\si@tempb
     \si@num@padtrailtrue
   \fi
   \renewcommand*{\si@tempb}{all}%
   \ifx\si@tempa\si@tempb
     \si@num@padleadtrue
     \si@num@padtrailtrue
   \fi
   \renewcommand*{\si@tempb}{true}%
   \ifx\si@tempa\si@tempb
     \si@num@padleadtrue
     \si@num@padtrailtrue
   \fi
   \renewcommand*{\si@tempb}{both}%
   \ifx\si@tempa\si@tempb
     \si@num@padleadtrue
     \si@num@padtrailtrue
   \fi}
%    \end{macrocode}
%\end{macro}
%\end{macro}
%\begin{macro}{\si@sign}
%\begin{macro}{\ifsi@num@signmant}
%\begin{macro}{\ifsi@num@signexp}
% Some new switches for adding signs to numbers
%    \begin{macrocode}
\newif\ifsi@num@signmant
\newif\ifsi@num@signexp
%    \end{macrocode}
%\end{macro}
%\end{macro}
% Signs can be added to numbers by default. Two options are needed
% here; whether to add a sign by default, and what the sign is.
%    \begin{macrocode}
\si@opt@xchoicekey{sign}{plus,minus,pm,mp}{plus}
\si@opt@choicekey[all]{addsign}
  {mantissa,exponent,mant,exp,all,both,true,none,false}
%    \end{macrocode}
%\end{macro}
% The option is now processed.
%    \begin{macrocode}
  {\si@num@signmantfalse
   \si@num@signexpfalse
   \renewcommand*{\si@tempb}{mantissa}%
   \ifx\si@tempa\si@tempb
     \si@num@signmanttrue
   \fi
   \renewcommand*{\si@tempb}{mant}%
   \ifx\si@tempa\si@tempb
     \si@num@signmanttrue
   \fi
   \renewcommand*{\si@tempb}{exponent}%
   \ifx\si@tempa\si@tempb
     \si@num@signexptrue
   \fi
   \renewcommand*{\si@tempb}{exp}%
   \ifx\si@tempa\si@tempb
     \si@num@signexptrue
   \fi
   \renewcommand*{\si@tempb}{all}%
   \ifx\si@tempa\si@tempb
     \si@num@signmanttrue
     \si@num@signexptrue
   \fi
   \renewcommand*{\si@tempb}{true}%
   \ifx\si@tempa\si@tempb
     \si@num@signmanttrue
     \si@num@signexptrue
   \fi
   \renewcommand*{\si@tempb}{both}%
   \ifx\si@tempa\si@tempb
     \si@num@signmanttrue
     \si@num@signexptrue
   \fi}
%    \end{macrocode}
%\begin{macro}{\ifsi@tightpm}
%\begin{macro}{\si@pm}
% To reduce spacing, it might be necessary to use a ``tight'' $\pm$
% sign.
%    \begin{macrocode}
\si@opt@boolkey{tightpm}
\newcommand*{\si@pm}{%
  \ifsi@tightpm
    \si@fix@tightpm
  \else
    \si@fix@pm
  \fi}
%    \end{macrocode}
%\end{macro}
%\end{macro}
%\begin{macro}{\ifsi@ang@padsmall}
%\begin{macro}{\ifsi@ang@padlarge}
% A switch for determining whether to typeset |\ang{;;1}| as
% \ang[padangle=both]{;;1} or \ang[padangle=none]{;;1}.  First, two
% new Boolean switches are needed to indicate padding.
%    \begin{macrocode}
\newif\ifsi@ang@padsmall
\newif\ifsi@ang@padlarge
\si@opt@choicekey[all]{padangle}
  {small,large,all,both,true,none,false}
  {\si@ang@padsmallfalse
   \si@ang@padlargefalse
   \renewcommand*{\si@tempb}{small}%
   \ifx\si@tempa\si@tempb
     \si@ang@padsmalltrue
   \fi
   \renewcommand*{\si@tempb}{large}%
   \ifx\si@tempa\si@tempb
     \si@ang@padlargetrue
   \fi
   \renewcommand*{\si@tempb}{all}%
   \ifx\si@tempa\si@tempb
     \si@ang@padsmalltrue
     \si@ang@padlargetrue
   \fi
   \renewcommand*{\si@tempb}{true}%
   \ifx\si@tempa\si@tempb
     \si@ang@padsmalltrue
     \si@ang@padlargetrue
   \fi
   \renewcommand*{\si@tempb}{both}%
   \ifx\si@tempa\si@tempb
     \si@ang@padsmalltrue
     \si@ang@padlargetrue
   \fi}
%    \end{macrocode}
%\end{macro}
%\end{macro}
%\begin{macro}{\ifsi@ang@toarc}
%\begin{macro}{\ifsi@ang@todec}
% To control whether angles are formatted as decimals or
% degrees--minutes--seconds, a package option plus two switches are
% needed.  The later format is referred to as ``\opt{arc}'' most
% readily.  An option to leave the input unchanged is also provided.
%    \begin{macrocode}
\newif\ifsi@ang@toarc
\newif\ifsi@ang@todec
\si@opt@choicekey[all]{angformat}
  {dec,decimal,arc,dms,unchanged,none}
  {\si@ang@toarcfalse
   \si@ang@todecfalse
   \renewcommand*{\si@tempb}{dec}%
   \ifx\si@tempa\si@tempb
     \si@ang@todectrue
   \fi
   \renewcommand*{\si@tempb}{decimal}%
   \ifx\si@tempa\si@tempb
     \si@ang@todectrue
   \fi
   \renewcommand*{\si@tempb}{arc}%
   \ifx\si@tempa\si@tempb
     \si@ang@toarctrue
   \fi
   \renewcommand*{\si@tempb}{dms}%
   \ifx\si@tempa\si@tempb
     \si@ang@toarctrue
   \fi}
%    \end{macrocode}
%\end{macro}
%\end{macro}
%\begin{macro}{\ifsi@astroang}
% A slightly odd option to allow the method used by astronomers for
% angles.
%    \begin{macrocode}
\si@opt@boolkey{astroang}
%    \end{macrocode}
%\end{macro}
%\begin{macro}{\ifsi@strictarc}
% For the \cs{ang} macro, the default is to require two semi-colons
% in the input for arc angles.  This is controlled here.
%    \begin{macrocode}
\si@opt@boolkey{strictarc}
%    \end{macrocode}
%\end{macro}
%\changes{v1.1}{2008/09/03}{Added \opt{tabparseonly} option}
%\begin{macro}{\ifsi@tab@fixed}
%\begin{macro}{\ifsi@tabparseonly}
%\begin{macro}{\si@tabnumalign}
%\begin{macro}{\si@tab@rfill@S}
%\begin{macro}{\si@tab@lfill@S}
% To determine the control of table alignment, two options are
% provided.  The \opt{tabnumalign} option controls which centring method
% is used, and the fills used for achieving this.
%    \begin{macrocode}
\newif\ifsi@tab@fixed
\si@opt@boolkey{tabparseonly}
\si@opt@choicekey{tabnumalign}
  {centredecimal,centerdecimal,right,left,centre,center}
  {\si@tab@fixedtrue
   \def\si@tab@rfill@S{0.5}%
   \def\si@tab@lfill@S{0.5}%
   \renewcommand*{\si@tempb}{right}%
   \ifx\si@tempa\si@tempb
     \def\si@tab@rfill@S{0}%
     \def\si@tab@lfill@S{1}%
   \fi
   \renewcommand*{\si@tempb}{left}%
   \ifx\si@tempa\si@tempb
     \def\si@tab@rfill@S{1}%
     \def\si@tab@lfill@S{0}%
   \fi
   \renewcommand*{\si@tempb}{centredecimal}%
   \ifx\si@tempa\si@tempb
     \expandafter\si@tab@fixedfalse
   \fi
   \renewcommand*{\si@tempb}{centerdecimal}%
   \ifx\si@tempa\si@tempb
     \expandafter\si@tab@fixedfalse
   \fi}
\si@opt@key{tabalign}{\sisetup{tabnumalign=#1,tabtextalign=#1,
  tabunitalign=#1}}
%    \end{macrocode}
%\end{macro}
%\end{macro}
%\end{macro}
%\end{macro}
%\end{macro}
%\begin{macro}{\ifsi@tabalignexp}
% A switch for aligning exponents.
%    \begin{macrocode}
\si@opt@boolkey{tabalignexp}
\si@opt@boolkey[%
  \si@tabalignexpfalse
  \ifsi@tabexpalign
    \expandafter\si@tabalignexptrue
  \fi
  ]{tabexpalign}
%    \end{macrocode}
%\end{macro}
%\begin{macro}{\si@tab@mantprecnt}
%\begin{macro}{\si@tab@mantpostcnt}
%\begin{macro}{\si@tab@expprecnt}
%\begin{macro}{\si@tab@exppostcnt}
%\begin{macro}{\ifsi@tab@mantsign}
%\begin{macro}{\ifsi@tab@expsign}
% To process the format information, various internal
% number-processing macros are used.  First, some
% storage areas are created.
%    \begin{macrocode}
\newcount\si@tab@mantprecnt
\newcount\si@tab@mantpostcnt
\newcount\si@tab@expprecnt
\newcount\si@tab@exppostcnt
\newif\ifsi@tab@mantsign
\newif\ifsi@tab@expsign
%    \end{macrocode}
%\end{macro}
%\end{macro}
%\end{macro}
%\end{macro}
%\end{macro}
%\end{macro}
% The input is split into a mantissa and exponent, then passed to a
% re-useable macro for further processing.
%    \begin{macrocode}
\si@opt@cmdkey{tabformat}
  {\si@num@fixpm
   \renewcommand*{\si@num@arg}{tabformat data}%
   \renewcommand*{\si@num@exp}{}%
   \renewcommand*{\si@num@mant}{}%
   \si@tab@mantsignfalse
   \si@tab@expsignfalse
   \si@switchfalse
   \si@num@sepmantexp{#1}%
%    \end{macrocode}
% When checking for a sign, the internal flag for finding but
% deleting a plus sign is used.
%    \begin{macrocode}
   \si@num@sepsign{mant}%
   \ifx\@empty\si@num@mantsign\@empty
     \ifsi@num@delplus
       \expandafter\expandafter\expandafter\si@tab@mantsigntrue
     \fi
   \else
     \expandafter\si@tab@mantsigntrue
   \fi
   \si@num@sepsign{exp}%
   \ifx\@empty\si@num@expsign\@empty
     \ifsi@num@delplus
       \expandafter\expandafter\expandafter\si@tab@expsigntrue
     \fi
   \else
     \expandafter\si@tab@expsigntrue
   \fi
   \si@opt@proctform{mant}%
   \si@opt@proctform{exp}%
%    \end{macrocode}
%\changes{v1.0h}{2008/08/02}{Simplified warnings for \opt{tabformat}}
% If both the integer and decimal parts are empty, then something has
% probably gone wrong.
%    \begin{macrocode}
   \ifnum\si@tab@mantpostcnt=\z@\relax
     \ifnum\si@tab@mantprecnt=\z@\relax
       \si@log@err{Empty mantissa argument for tabformat}
         {The argument `#1' contains no valid entry for
          a mantissa\MessageBreak It should be of the
          form `m.n', where  m and n are integers}%
     \fi
   \fi
%    \end{macrocode}
% If \opt{tabformat} has been given with
% \opt{tabnumalign=centredecimal} active, then the alignment is
% changed to centred.
%    \begin{macrocode}
   \ifsi@tab@fixed\else
     \sisetup{tabnumalign=centre}%
   \fi
   \let\pm\si@num@pm
   \let\mp\si@num@mp}
%    \end{macrocode}
%\begin{macro}{\si@opt@proctform}
%\darg{either \texttt{mant} or \texttt{exp}}
% Processing the number further uses the \cs{si@num@digits} macro.
% The results are stored in the appropriate counter.
%    \begin{macrocode}
\newcommand*{\si@opt@proctform}[1]{%
  \renewcommand*{\si@num@predec}{}%
  \renewcommand*{\si@num@postdec}{}%
  \si@switchfalse
  \expandafter\si@ifnotmtarg\expandafter{%
    \csname si@num@#1\endcsname}
     {\expandafter\expandafter\expandafter\si@num@digits
        \csname si@num@#1\endcsname\@empty\@empty}%
   \csname si@tab@#1precnt\endcsname\z@\relax
   \csname si@tab@#1postcnt\endcsname\z@\relax
   \ifx\@empty\si@num@predec\@empty\else
     \csname si@tab@#1precnt\endcsname\si@num@predec\relax
   \fi
   \ifx\@empty\si@num@postdec\@empty\else
     \csname si@tab@#1postcnt\endcsname\si@num@postdec\relax
   \fi}
%    \end{macrocode}
%\end{macro}
% The alignment of tabular material when not processed by \cs{num}
% needs to be available.
%    \begin{macrocode}
\si@opt@choicekey{tabtextalign}{left,right,centre,center}
%    \end{macrocode}
%\begin{macro}{\si@tab@rfill@t}
%\begin{macro}{\si@tab@lfill@t}
% By default, centring happens on both sides of the content of
% tabular material.
%    \begin{macrocode}
  {\def\si@tab@rfill@t{0.5}%
   \def\si@tab@lfill@t{0.5}%
   \renewcommand*{\si@tempb}{right}%
   \ifx\si@tempa\si@tempb
     \def\si@tab@rfill@t{0}%
     \def\si@tab@lfill@t{1}%
   \fi
   \renewcommand*{\si@tempb}{left}%
   \ifx\si@tempa\si@tempb
     \def\si@tab@rfill@t{1}%
     \def\si@tab@lfill@t{0}%
   \fi}
%    \end{macrocode}
%\end{macro}
%\end{macro}
% The alignment of unit columns for tabular material has a similar
% control option.
%    \begin{macrocode}
\si@opt@choicekey{tabunitalign}{left,right,centre,center}
%    \end{macrocode}
%\begin{macro}{\si@tab@rfill@s}
%\begin{macro}{\si@tab@lfill@s}
% By default, centring happens on both sides of the content of
% tabular material.
%    \begin{macrocode}
  {\def\si@tab@rfill@s{0.5}%
   \def\si@tab@lfill@s{0.5}%
   \renewcommand*{\si@tempb}{right}%
   \ifx\si@tempa\si@tempb
     \def\si@tab@rfill@s{0}%
     \def\si@tab@lfill@s{1}%
   \fi
   \renewcommand*{\si@tempb}{left}%
   \ifx\si@tempa\si@tempb
     \def\si@tab@rfill@s{1}%
     \def\si@tab@lfill@s{0}%
   \fi}
%    \end{macrocode}
%\end{macro}
%\end{macro}
%\begin{macro}{\ifsi@fixdp}
%\changes{v1.1}{2008/08/22}{Modified to work with \opt{fixsf} option}
%\begin{macro}{\si@num@dp}
%\changes{v1.1}{2008/08/22}{Modified to work with \opt{fixsf} option}
% To allow control of rounding, two options are needed.  One sets how
% many fixed digits to use, the second turns this function on and
% off.
%    \begin{macrocode}
\si@opt@boolkey[%
  \ifsi@fixdp
    \expandafter\si@fixsffalse
  \fi]{fixdp}
\newcount\si@num@dp
\si@opt@key{dp}{%
  \si@str@ifonlychrs{#1}{0123456789}
    {\si@num@dp#1\relax
     \si@fixdptrue
     \si@fixsffalse}
    {\si@log@err{Invalid input for dp option}
      {The dp option must be given a positive integer}}}
%    \end{macrocode}
%\end{macro}
%\end{macro}
%\begin{macro}{\ifsi@fixsf}
%\begin{macro}{\si@num@sf}
% A similar approach for fixing significant figures.
%    \begin{macrocode}
\si@opt@boolkey[%
  \ifsi@fixsf
    \expandafter\si@fixdpfalse
  \fi]{fixsf}
\newcount\si@num@sf
\si@opt@key{sf}{%
  \si@str@ifonlychrs{#1}{0123456789}
    {\si@num@sf#1\relax
     \ifnum\si@num@sf=\z@\relax
       \si@log@warn{Setting sf=0 disables function}%
       \si@fixsffalse
     \else
       \si@fixsftrue
       \si@fixdpfalse
     \fi}
    {\si@log@err{Invalid input for sf option}
      {The sf option must be given a positive integer}}}
%    \end{macrocode}
%\end{macro}
%\end{macro}
%\begin{macro}{\ifsi@tabautofit}
% To apply rounding automatically in a table, a separate option is
% used.
%    \begin{macrocode}
\si@opt@boolkey{tabautofit}
%    \end{macrocode}
%\end{macro}
%\begin{macro}{\ifsi@xspace}
% Unit macros on their own may need \pkg{xpsace}.
%    \begin{macrocode}
\si@opt@boolkey{xspace}
%    \end{macrocode}
%\end{macro}
%\begin{macro}{\ifsi@prespace}
%    \begin{macrocode}
\si@opt@boolkey
  [\si@unt@numfalse
   \ifsi@prespace
     \si@unt@numtrue
  \fi]
  {prespace}
%    \end{macrocode}
%\end{macro}
%\begin{macro}{\ifsi@allowoptarg}
% For \unitsdef users, a method to absorb optional arguments is
% needed.
%    \begin{macrocode}
\si@opt@boolkey{allowoptarg}
%    \end{macrocode}
%\end{macro}
%\changes{v1.1d}{2008/10/29}{Minor bug with \opt{per} option fixed}
%\begin{macro}{\ifsi@frac}
%\begin{macro}{\ifsi@slash}
%\begin{macro}{\ifsi@stickyper}
% The option processing for formatting units with \cs{per} in them
% needs two switches.
%    \begin{macrocode}
\newif\ifsi@slash
\newif\ifsi@frac
\si@opt@boolkey{stickyper}
\si@opt@choicekey[reciprocal]{per}
  {reciprocal,rp,power,slash,frac,fraction}
  {\si@slashfalse
   \si@fracfalse
   \let\si@frac\si@frc@frac
   \renewcommand*{\si@tempb}{slash}%
   \ifx\si@tempa\si@tempb
     \si@fractrue
     \si@slashtrue
     \let\si@frac\si@frc@slash
   \fi
   \renewcommand*{\si@tempb}{frac}%
   \ifx\si@tempa\si@tempb
     \si@fractrue
   \fi
   \renewcommand*{\si@tempb}{fraction}%
   \ifx\si@tempa\si@tempb
     \si@fractrue
   \fi}
%    \end{macrocode}
%\end{macro}
%\end{macro}
%\end{macro}
%\begin{macro}{\si@slash}
% For the \opt{slash} option, the separator can be customised.
%    \begin{macrocode}
\si@opt@xchoicekey{slash}{slash}{slash}
%    \end{macrocode}
%\end{macro}
%\begin{macro}{\ifsi@repeatunits}
%\begin{macro}{\ifsi@addunitpower}
% An option is needed for cases where units should be repeated.
%    \begin{macrocode}
\newif\ifsi@repeatunits
\newif\ifsi@addunitpower
\si@opt@choicekey[true]{repeatunits}{true,false,power}
  {\si@repeatunitsfalse
   \si@addunitpowerfalse
   \renewcommand*{\si@tempb}{true}%
   \ifx\si@tempa\si@tempb
     \si@repeatunitstrue
   \fi
   \renewcommand*{\si@tempb}{power}%
   \ifx\si@tempa\si@tempb
     \si@addunitpowertrue
   \fi}
%    \end{macrocode}
%\end{macro}
%\end{macro}
%\begin{macro}{\ifsi@trapambigfrac}
%\begin{macro}{\si@closefrac}
%\begin{macro}{\si@openfrac}
% Macros for the right and left brackets added to potentially
% ambiguous denominators.
%    \begin{macrocode}
\si@opt@boolkey{trapambigfrac}
\si@opt@cmdkeys{closefrac,openfrac}
%    \end{macrocode}
%\end{macro}
%\end{macro}
%\end{macro}
% In the case of fractional handling of the \cs{per} operator,
% further refinement is available.
%    \begin{macrocode}
\si@opt@choicekey[frac]{fraction}
  {frac,nicefrac,nice,sfrac,xfrac,uglyfrac,ugly}
  {\let\si@frac\si@frc@frac
   \renewcommand*{\si@tempb}{nicefrac}%
   \ifx\si@tempa\si@tempb
     \let\si@frac\si@frc@nice
   \fi
   \renewcommand*{\si@tempb}{uglyfrac}%
   \ifx\si@tempa\si@tempb
     \let\si@frac\si@frc@ugly
   \fi
   \renewcommand*{\si@tempb}{nice}%
   \ifx\si@tempa\si@tempb
     \let\si@frac\si@frc@nice
   \fi
   \renewcommand*{\si@tempb}{sfrac}%
   \ifx\si@tempa\si@tempb
     \let\si@frac\si@frc@sfrac
   \fi
   \renewcommand*{\si@tempb}{xfrac}%
   \ifx\si@tempa\si@tempb
     \let\si@frac\si@frc@sfrac
   \fi
   \renewcommand*{\si@tempb}{ugly}%
   \ifx\si@tempa\si@tempb
     \let\si@frac\si@frc@ugly
   \fi}
%    \end{macrocode}
%\begin{macro}{\si@load}
%\begin{macro}{\si@noload}
% Loading of support files is controlled by two keys.  The first
% defines a list of files that may be loaded, the second a list that
% will not.  This makes it easy to exclude a single file from a long
% list.
%    \begin{macrocode}
\si@opt@cmdkeys{load,noload}
\si@opt@key{alsoload}{\si@addtolist{\si@load}{#1}}
\AtEndOfPackage{
  \si@opt@disablekey{load}
    {Configuration files can only be used\MessageBreak
     when loading package}
  \si@opt@disablekey{noload}
    {Configuration files can only be used\MessageBreak
     when loading package}}
\AtEndOfPackage{
  \si@opt@key{alsoload}{%
    \@for\si@tempa:=#1\do{\si@loadfile{\si@tempa}}}}
%    \end{macrocode}
%\end{macro}
%\end{macro}
%\begin{macro}{\ifsi@colourunits}
%\begin{macro}{\ifsi@colourvalues}
%\begin{macro}{\si@unitcolour}
%\begin{macro}{\si@valuecolour}
% Colour is turned on and off using two switches and the appropriate
% options.  US spellings are also provided.
%    \begin{macrocode}
\si@opt@boolkey{colourunits}
\si@opt@boolkey{colourvalues}
\si@opt@choicekey[true]{colorunits}
  {true,false}
  {\si@colourunitsfalse
   \renewcommand*{\si@tempb}{true}%
   \ifx\si@tempa\si@tempb
     \si@colourunitstrue
   \fi}
\si@opt@choicekey[true]{colorvalues}
  {true,false}
  {\si@colourvaluesfalse
   \renewcommand*{\si@tempb}{true}%
   \ifx\si@tempa\si@tempb
     \si@colourvaluestrue
   \fi}
\si@opt@choicekey[true]{colorall}
  {true,false}
  {\si@colourvaluesfalse
   \si@colourunitsfalse
   \renewcommand*{\si@tempb}{true}%
   \ifx\si@tempa\si@tempb
     \si@colourunitstrue
     \si@colourvaluestrue
   \fi}
\si@opt@choicekey[true]{colourall}
  {true,false}
  {\si@colourvaluesfalse
   \si@colourunitsfalse
   \renewcommand*{\si@tempb}{true}%
   \ifx\si@tempa\si@tempb
     \si@colourunitstrue
     \si@colourvaluestrue
   \fi}
\si@opt@cmdkeys{unitcolour,valuecolour}
\si@opt@key{unitcolor}{\sisetup{unitcolour=#1}}
\si@opt@key{valuecolor}{\sisetup{valuecolour=#1}}
\si@opt@key{colour}{\sisetup{unitcolour=#1,valuecolour=#1}}
\si@opt@key{color}{\sisetup{unitcolour=#1,valuecolour=#1}}
%    \end{macrocode}
%\end{macro}
%\end{macro}
%\end{macro}
%\end{macro}
%\begin{macro}{\ifsi@colourneg}
%\begin{macro}{\si@negcolour}
% The set up for colouring negative numbers is similar.
%    \begin{macrocode}
\si@opt@boolkey{colourneg}
\si@opt@choicekey[true]{colorneg}
  {true,false}
  {\si@colournegfalse
   \renewcommand*{\si@tempb}{true}%
   \ifx\si@tempa\si@tempb
     \si@colournegtrue
   \fi}
\si@opt@cmdkeys{negcolour}
\si@opt@key{negcolor}{\sisetup{negcolour=#1}}
%    \end{macrocode}
%\end{macro}
%\end{macro}
%\begin{macro}{\si@textOmega}
%\begin{macro}{\si@mathsOmega}
% The various non-Latin symbols need to be handled, and given user interfaces.
% Some definitions are more complex than others; for \si{\ohm}
% things are easy.
%    \begin{macrocode}
\si@opt@cmdkeys{textOmega,mathsOmega}
\si@opt@key{mathOmega}{\sisetup{mathsOmega=#1}}
\newcommand*{\si@mathsOmega}{\text{\ensuremath{\Omega}}}
\newcommand*{\si@textOmega}{\ensuremath{\Omega}}
%    \end{macrocode}
%\end{macro}
%\end{macro}
%\changes{v1.0c}{2008/06/25}{Fixed excess loading of maths fonts}
%\begin{macro}{\si@textmu}
%\begin{macro}{\si@mathsmu}
% For the \si{\micro} symbol, some direct loading of symbols is
% needed as the maths mu sign ($\mu$) is wrong.
%    \begin{macrocode}
\si@opt@cmdkeys{textmu,mathsmu}
\si@opt@key{mathmu}{\sisetup{mathsmu=#1}}
\DeclareFontEncoding{TS1}{}{}
\DeclareFontSubstitution{TS1}{cmr}{m}{n}
\DeclareTextSymbol{\si@textmu}{TS1}{181}
\DeclareTextSymbolDefault{\si@textmu}{TS1}
\@ifpackageloaded{upgreek}
  {}
  {\DeclareFontFamily{OML}{eur}{\skewchar\font'177}
   \DeclareFontShape{OML}{eur}{m}{n}{%
    <-6> eurm5 <6-8> eurm7 <8-> eurm10}{}}
\AtBeginDocument{
  \@ifpackageloaded{upgreek}
    {\let\si@mathsmu\upmu}
    {\DeclareSymbolFont{si@greek}{OML}{eur}{m}{n}
     \DeclareMathSymbol{\si@mathsmu}{\mathord}{si@greek}{"16}}}
%    \end{macrocode}
%\end{macro}
%\end{macro}
%\begin{macro}{\si@textdegree}
%\begin{macro}{\si@mathsdegree}
%\begin{macro}{\si@textminute}
%\begin{macro}{\si@mathsminute}
%\begin{macro}{\si@textsecond}
%\begin{macro}{\si@mathssecond}
% The angle signs.
%    \begin{macrocode}
\si@opt@cmdkeys{textdegree,mathsdegree,textminute,mathsminute,
  textsecond,mathssecond}
\si@opt@key{mathdegree}{\sisetup{mathsdegree=#1}}
\si@opt@key{mathminute}{\sisetup{mathsminute=#1}}
\si@opt@key{mathsecond}{\sisetup{mathssecond=#1}}
\newcommand*{\si@textdegree}{\ensuremath{{}^{\circ}}}
\newcommand*{\si@mathsdegree}{{}^{\circ}}
\newcommand*{\si@textminute}{\ensuremath{{}^{\prime}}}
\newcommand*{\si@mathsminute}{{}^{\prime}}
\newcommand*{\si@textsecond}{\ensuremath{{}^{\prime\prime}}}
\newcommand*{\si@mathssecond}{{}^{\prime\prime}}
%    \end{macrocode}
%\end{macro}
%\end{macro}
%\end{macro}
%\end{macro}
%\end{macro}
%\end{macro}
%\begin{macro}{\si@textcelsius}
%\changes{v1.1e}{2008/11/18}{Bug fix in appearance}
%\begin{macro}{\si@mathscelsius}
% Finally, degrees Celsius, which may need the degree symbol.
%    \begin{macrocode}
\si@opt@cmdkeys{textcelsius,mathscelsius}
\si@opt@key{mathcelsius}{\sisetup{mathscelsius=#1}}
\newcommand*{\si@textcelsius}{%
  \si@textdegree\kern-\scriptspace C}
\newcommand*{\si@mathscelsius}{%
  \si@mathsdegree\kern-\scriptspace C}
%    \end{macrocode}
%\end{macro}
%\end{macro}
%\begin{macro}{\si@textringA}
%\begin{macro}{\si@mathsringA}
% The \si{\angstrom} sign.
%    \begin{macrocode}
\si@opt@cmdkeys{textringA,mathsringA}
\si@opt@key{mathringA}{\sisetup{mathsringA=#1}}
\newcommand*{\si@textringA}{\AA}
\newcommand*{\si@mathsringA}{\text{\AA}}
%    \end{macrocode}
%\end{macro}
%\end{macro}
%\begin{macro}{\ifsi@redefsymbols}
% A flag for using \pkg{textcomp} and \pkg{upgreek} to provide better
% symbols.
%    \begin{macrocode}
\si@opt@boolkey{redefsymbols}
\AtBeginDocument{
  \si@opt@disablekey{redefsymbols}
    {Symbols can only be redefined\MessageBreak
     when loading siunitx}}
%    \end{macrocode}
%\end{macro}
%\begin{macro}{\si@eVcorra}
%\begin{macro}{\si@eVcorrb}
%    \begin{macrocode}
\newlength\si@eVcorra
\newlength\si@eVcorrb
\si@opt@key{eVcorra}{\setlength\si@eVcorra{#1}}
\si@opt@key{eVcorrb}{\setlength\si@eVcorrb{#1}}
%    \end{macrocode}
%\end{macro}
%\end{macro}
%\begin{macro}{\si@locale}
%\begin{macro}{\si@loctolang}
% Handling typographic conventions needs three keys.  \opt{locale} is
% used to set the locale, \opt{loctolang} to bind to \pkg{babel}.
%    \begin{macrocode}
\si@opt@key{locale}{%
  \si@loc@load{#1}%
  \si@loc@set{#1}}%
\si@opt@key{loctolang}{\si@loc@ltol{#1}}
%    \end{macrocode}
%\end{macro}
%\end{macro}
%
%\subsection{Compatibility options}
%\begin{macro}{\ifsi@old@ugly}
%\begin{macro}{\ifsi@old@nice}
%\begin{macro}{\ifsi@old@loose}
%\begin{macro}{\ifsi@old@tight}
% With the options for the package set up, the next stage is to
% provide support for users of the older packages.  These all set up
% switches, but do not do anything.  That is left to the emulation
% files, loaded at the end of the package. First of all, the
% \units options are dealt with; there are not many.
%    \begin{macrocode}
\si@opt@compatkey{units}{ugly}
\si@opt@compatkey{units}{nice}
\si@opt@compatkey{units}{loose}
\si@opt@compatkey{units}{tight}
%    \end{macrocode}
%\end{macro}
%\end{macro}
%\end{macro}
%\end{macro}
%\begin{macro}{\ifsi@old@OHM}
%\begin{macro}{\ifsi@old@ohm}
%\begin{macro}{\ifsi@old@redef-gensymb}
%\begin{macro}{\ifsi@gensymb}
% The \unitsdef package is unfortunately much more profligate with
% options. The first set are to do with support for \pkg{gensymb}.
%    \begin{macrocode}
\si@opt@compatkey{unitsdef}{OHM}
\si@opt@compatkey{unitsdef}{ohm}
\si@opt@compatkey{unitsdef}{redef-gensymb}
\newif\ifsi@gensymb
%    \end{macrocode}
%\end{macro}
%\end{macro}
%\end{macro}
%\end{macro}
%\begin{macro}{\ifsi@old@LITER}
%\begin{macro}{\ifsi@old@liter}
%\begin{macro}{\ifsi@old@noxspace}
%\begin{macro}{\ifsi@old@noconfig}
% The second set are more general functionality.
%    \begin{macrocode}
\si@opt@compatkey{unitsdef}{LITER}
\si@opt@compatkey{unitsdef}{liter}
\si@opt@compatkey{unitsdef}{noxspace}
\si@opt@compatkey{unitsdef}{noconfig}
%    \end{macrocode}
%\end{macro}
%\end{macro}
%\end{macro}
%\end{macro}
%\begin{macro}{\ifsi@old@noabbr}
%\begin{macro}{\ifsi@old@noamperageabbr}
%\begin{macro}{\ifsi@old@nofrequncyabbr}
%\begin{macro}{\ifsi@old@nomolabbr}
%\begin{macro}{\ifsi@old@novoltageabbr}
%\begin{macro}{\ifsi@old@novolumeabbr}
%\begin{macro}{\ifsi@old@noweightabbr}
%\begin{macro}{\ifsi@old@noenergyabbr}
%\begin{macro}{\ifsi@old@nolengthabbr}
%\begin{macro}{\ifsi@old@notimeabbr}
% The final set are for control of abbreviations, and are a good
% demonstration of why to use \pkg{xkeyval}!
%    \begin{macrocode}
\si@opt@compatkey{unitsdef}{noabbr}
\si@opt@compatkey{unitsdef}{noampereageabbr}
\si@opt@compatkey{unitsdef}{nofrequncyabbr}
\si@opt@compatkey{unitsdef}{nomolabbr}
\si@opt@compatkey{unitsdef}{novoltageabbr}
\si@opt@compatkey{unitsdef}{novolumeabbr}
\si@opt@compatkey{unitsdef}{noweightabbr}
\si@opt@compatkey{unitsdef}{noenergyabbr}
\si@opt@compatkey{unitsdef}{nolengthabbr}
\si@opt@compatkey{unitsdef}{notimeabbr}
%    \end{macrocode}
%\end{macro}
%\end{macro}
%\end{macro}
%\end{macro}
%\end{macro}
%\end{macro}
%\end{macro}
%\end{macro}
%\end{macro}
%\end{macro}
%\begin{macro}{\ifsi@old@cdot}
%\begin{macro}{\ifsi@old@thickspace}
%\begin{macro}{\ifsi@old@mediumspace}
%\begin{macro}{\ifsi@old@thinspace}
%\begin{macro}{\ifsi@old@thickqspace}
%\begin{macro}{\ifsi@old@mediumqspace}
%\begin{macro}{\ifsi@old@thinqspace}
% The \SIunits package has lots of options.  These ones are all
% related to spacing.
%    \begin{macrocode}
\si@opt@compatkey{SIunits}{cdot}
\si@opt@compatkey{SIunits}{thickspace}
\si@opt@compatkey{SIunits}{mediumspace}
\si@opt@compatkey{SIunits}{thinspace}
\si@opt@compatkey{SIunits}{thickqspace}
\si@opt@compatkey{SIunits}{mediumqspace}
\si@opt@compatkey{SIunits}{thinqspace}
%    \end{macrocode}
%\end{macro}
%\end{macro}
%\end{macro}
%\end{macro}
%\end{macro}
%\end{macro}
%\end{macro}
%\begin{macro}{\ifsi@old@amssymb}
%\begin{macro}{\ifsi@old@squaren}
%\begin{macro}{\ifsi@old@pstricks}
%\begin{macro}{\ifsi@old@Gray}
%\begin{macro}{\ifsi@old@italian}
% These options are used by \SIunits to control clashes with other
% packages.
%    \begin{macrocode}
\si@opt@compatkey{SIunits}{amssymb}
\si@opt@compatkey{SIunits}{squaren}
\si@opt@compatkey{SIunits}{pstricks}
\si@opt@compatkey{SIunits}{Gray}
\si@opt@compatkey{SIunits}{italian}
%    \end{macrocode}
%\end{macro}
%\end{macro}
%\end{macro}
%\end{macro}
%\end{macro}
%\begin{macro}{\ifsi@old@textstyle}
%\begin{macro}{\ifsi@old@binary}
%\begin{macro}{\ifsi@old@noams}
%\begin{macro}{\ifsi@old@derivedinbase}
%\begin{macro}{\ifsi@old@derived}
% The miscellaneous options.
%    \begin{macrocode}
\si@opt@compatkey{SIunits}{textstyle}
\si@opt@compatkey{SIunits}{binary}
\si@opt@compatkey{SIunits}{noams}
\si@opt@compatkey{SIunits}{derivedinbase}
\si@opt@compatkey{SIunits}{derived}
%    \end{macrocode}
%\end{macro}
%\end{macro}
%\end{macro}
%\end{macro}
%\end{macro}
%\begin{macro}{\ifsi@old@noprefixcmds}
% The \pkg{hepunits} package only has one option.
%    \begin{macrocode}
\si@opt@compatkey{hepunits}{noprefixcmds}
%    \end{macrocode}
%\end{macro}
%\begin{macro}{\ifsi@old@english}
%\begin{macro}{\ifsi@old@french}
% \pkg{fancynum} provides a few options.  First the rather oddly
% named \opt{english} and \opt{french} ones.
%    \begin{macrocode}
\si@opt@compatkey{fancynum}{english}
\si@opt@compatkey{fancynum}{french}
%    \end{macrocode}
%\end{macro}
%\end{macro}
%\begin{macro}{\ifsi@old@tight}
%\begin{macro}{\ifsi@old@loose}
% A couple for spacing multiplication.
%    \begin{macrocode}
\si@opt@compatkey{fancynum}{tight}
\si@opt@compatkey{fancynum}{loose}
%    \end{macrocode}
%\end{macro}
%\end{macro}
%\begin{macro}{\ifsi@old@thinspaces}
%\begin{macro}{\ifsi@old@commas}
%\begin{macro}{\ifsi@old@plain}
% Three for digit separation.
%    \begin{macrocode}
\si@opt@compatkey{fancynum}{thinspaces}
\si@opt@compatkey{fancynum}{commas}
\si@opt@compatkey{fancynum}{plain}
%    \end{macrocode}
%\end{macro}
%\end{macro}
%\end{macro}
%\begin{macro}{\ifsi@old@spaceqspace}
% The \pkg{fancyunits} package provides one option not available with
% \SIunits.
%    \begin{macrocode}
\si@opt@compatkey{fancyunits}{spaceqspace}
%    \end{macrocode}
%\end{macro}
%
%\subsection{Constants}
% A number of macros are needed by the package that provide a
% non-changing output.  These are defined here; the intention is that
% these should not be macros that the user is likely to need to
% alter.  All of these macros have preface \cs{si@fix@}, to flag that
% that are intended as constants.  The package may rely on the
% contents of these macros for functionality.
%
%\begin{macro}{\si@fix@thin}
%\begin{macro}{\si@fix@med}
%\begin{macro}{\si@fix@medium}
%\begin{macro}{\si@fix@thick}
%\begin{macro}{\si@fix@space}
%\changes{v1.0a}{2008/06/20}{Fixed problem with \opt{space} option}
% First, there are the various space macros.  To allow both \opt{med}
% and \opt{medium} to be used as a space description, two macros are
% needed for the same output.
%    \begin{macrocode}
\newcommand*{\si@fix@thin}{\,}
\newcommand*{\si@fix@med}{\:}
\newcommand*{\si@fix@medium}{\:}
\newcommand*{\si@fix@thick}{\;}
\newcommand*{\si@fix@space}{\text{ }}
%    \end{macrocode}
%\end{macro}
%\end{macro}
%\end{macro}
%\end{macro}
%\end{macro}
%\begin{macro}{\si@fix@cdot}
%\begin{macro}{\si@fix@comma}
%\begin{macro}{\si@fix@stop}
%\begin{macro}{\si@fix@fullstop}
%\begin{macro}{\si@fix@period}
%\begin{macro}{\si@fix@times}
%\begin{macro}{\si@fix@tighttimes}
%\begin{macro}{\si@fix@tightcdot}
% Next there are macros for material that is not simply whitespace.
% To allow several options, the full-stop gets lots of names.
%    \begin{macrocode}
\newcommand*{\si@fix@cdot}{{}\cdot{}}
\newcommand*{\si@fix@comma}{{,}}
\newcommand*{\si@fix@stop}{{.}}
\newcommand*{\si@fix@fullstop}{{.}}
\newcommand*{\si@fix@period}{{.}}
\newcommand*{\si@fix@times}{\times}
\newcommand*{\si@fix@tighttimes}{\bgroup\times\egroup}
\newcommand*{\si@fix@tightcdot}{\bgroup\cdot\egroup}
%    \end{macrocode}
%\end{macro}
%\end{macro}
%\end{macro}
%\end{macro}
%\end{macro}
%\end{macro}
%\end{macro}
%\end{macro}
%\begin{macro}{\si@fix@plus}
%\begin{macro}{\si@fix@minus}
%\begin{macro}{\si@fix@pm}
%\begin{macro}{\si@fix@tightpm}
%\begin{macro}{\si@fix@mp}
% Signs for numbers are needed.
%    \begin{macrocode}
\newcommand*{\si@fix@plus}{+}
\newcommand*{\si@fix@minus}{-}
\newcommand*{\si@fix@pm}{\pm}
\newcommand*{\si@fix@tightpm}{\bgroup\pm\egroup}
\newcommand*{\si@fix@mp}{\mp}
%    \end{macrocode}
%\end{macro}
%\end{macro}
%\end{macro}
%\end{macro}
%\end{macro}
%\begin{macro}{\si@fix@two}
%\begin{macro}{\si@fix@ten}
% The literals ``2'' and ``10'' are needed for exponents.
%    \begin{macrocode}
\newcommand*{\si@fix@two}{2}
\newcommand*{\si@fix@ten}{10}
%    \end{macrocode}
%\end{macro}
%\end{macro}
%\begin{macro}{\si@fix@slash}
% Another optional component that will probably not be used by many
% people.
%    \begin{macrocode}
\newcommand*{\si@fix@slash}{/}
%    \end{macrocode}
%\end{macro}
%\begin{macro}{\si@fix@none}
% Finally for spacing, there is the possibility of nothing at all
%    \begin{macrocode}
\newcommand*{\si@fix@none}{}
%    \end{macrocode}
%\end{macro}
%
%\subsection{Symbols}
%\begin{macro}{\si@symbol}
% Each of the symbol macros needs to be set up; the options give a
% maths and text mode sign, but internally a single macro is needed
% for each.
%    \begin{macrocode}
\newcommand*{\si@symbol}[1]{%
  \expandafter\protected\expandafter\def
    \csname si@sym@#1\endcsname{%
    \ifmmode
      \expandafter\csname si@maths#1\expandafter\endcsname
    \else
      \expandafter\csname si@text#1\expandafter\endcsname
    \fi}}
%    \end{macrocode}
%\end{macro}
%\begin{macro}{\si@sym@Omega}
%\begin{macro}{\si@sym@ringA}
%\begin{macro}{\si@sym@mu}
%\begin{macro}{\si@sym@degree}
%\begin{macro}{\si@sym@minute}
%\begin{macro}{\si@sym@second}
%\begin{macro}{\si@sym@celsius}
% The various symbols are now declared.
%    \begin{macrocode}
\si@symbol{Omega}
\si@symbol{ringA}
\si@symbol{mu}
\si@symbol{degree}
\si@symbol{minute}
\si@symbol{second}
\si@symbol{celsius}
%    \end{macrocode}
%\end{macro}
%\end{macro}
%\end{macro}
%\end{macro}
%\end{macro}
%\end{macro}
%\end{macro}
% The issue of redefinition of symbols now arises.  \currpkg can
% check for the loading of a number of support package, and can then
% redefine the appropriate symbols.
%    \begin{macrocode}
\AtBeginDocument{%
  \ifsi@redefsymbols
    \@ifpackageloaded{textcomp}
      {\si@log@debug{Redefining symbols using textcomp}%
       \renewcommand*{\si@textdegree}{\textdegree}%
       \renewcommand*{\si@mathsdegree}{\text{\textdegree}}%
%    \end{macrocode}
% \pkg{mathptmx} will give issues with \pkg{textcomp} and the
% \si{\ohm} sign.
%    \begin{macrocode}
       \@ifpackageloaded{mathptmx}{}
         {\renewcommand*{\si@textmu}{\textmu}%
          \renewcommand*{\si@textOmega}{\textohm}}%
%    \end{macrocode}
% The \si{\angstrom} symbol is only redefined if the encoding is
% OT1; other encodings should have a proper glyph used for \cs{AA}.
% The \cs{encodingdefault} macro is \cs{long} for some reason.
%    \begin{macrocode}
       \renewcommand*{\si@tempa}{OT1}%
       \ifx\si@tempa\encodingdefault
         \renewcommand*{\si@mathsringA}{%
           \text{\capitalring{A}}}%
         \renewcommand*{\si@textringA}{\capitalring{A}}
       \fi}{}
    \@ifpackageloaded{upgreek}
      {\si@log@debug{Redefining symbols using upgreek}%
       \renewcommand*{\si@mathsOmega}{\Upomega}}{}
  \fi}
%    \end{macrocode}
%
%\subsection{Handling fractions}
%\begin{macro}{\si@frac}
%\begin{macro}{\si@frc@hook}
%\begin{macro}{\si@frc@frac}
%\begin{macro}{\si@frc@slash}
%\begin{macro}{\si@frc@nice}
%\begin{macro}{\si@frc@sfrac}
% Various methods of handling fractions are provided.
%    \begin{macrocode}
\newcommand*{\si@frc@frac}[2]{%
  \ensuremath{\si@frc@hook\frac{%
    \expandafter\si@unt@out\expandafter{#1}}%
    {\expandafter\si@unt@out\expandafter{#2}}}}
\let\si@frac\si@frc@frac
\newcommand*{\si@frc@hook}{}
\newcommand*{\si@frc@slash}[2]{%
    \expandafter\si@unt@out\expandafter{#1}%
    \si@out{\ensuremath{\si@slash}}%
    \expandafter\si@unt@out\expandafter{#2}}
\newcommand*{\si@frc@nice}[2]{%
  \ensuremath{\si@frc@nicefrac{\expandafter\si@unt@out%
    \expandafter{#1}}{\expandafter\si@unt@out\expandafter
      {#2}}}}
\newcommand*{\si@frc@sfrac}[2]{%
    \sfrac{\expandafter\si@unt@out\expandafter{#1}}%
      {\expandafter\si@unt@out\expandafter{#2}}}
\AtBeginDocument{
  \@ifpackageloaded{xfrac}
    {}
    {\si@log@inf{xfrac package unavailable\MessageBreak
       using `fraction=sfrac' will fall back on\MessageBreak
       nicefrac-like method}%
     \renewcommand*{\si@frc@sfrac}[2]{%
      \si@log@warn{xfrac package unavailable}%
      \si@frc@nice{#1}{#2}}}}
%    \end{macrocode}
%\end{macro}
%\end{macro}
%\end{macro}
%\end{macro}
%\end{macro}
%\end{macro}
%\begin{macro}{\si@frc@nicefrac}
%\begin{macro}{\si@frc@displen}
%\begin{macro}{\si@frc@textlen}
%\begin{macro}{\si@frc@suplen}
%\begin{macro}{\si@frc@ssuplen}
% To avoid needing \units installed, the \cs{nicefrac} macro needs
% to be emulated here. The code is taken (with permission) from
% \kg{nicefrac}.\footnote{The original is licensed under the GPL;
% thanks to the author Axel Reichert for permission to copy the code
% here.}
%    \begin{macrocode}
\newlength\si@frc@displen
\newlength\si@frc@textlen
\newlength\si@frc@suplen
\newlength\si@frc@ssuplen
\newcommand*{\si@frc@nicefrac}{%
  \ifmmode
    \expandafter\si@frc@mathsnf
  \else
    \expandafter\si@frc@textnf
  \fi}
%    \end{macrocode}
%\end{macro}
%\end{macro}
%\end{macro}
%\end{macro}
%\end{macro}
%\begin{macro}{\si@frc@mathsnf}
%\darg{numerator}
%\darg{denominator}
% The maths mode system.
%    \begin{macrocode}
\newcommand*{\si@frc@mathsnf}[2]{%
  \begingroup
    \settoheight{\si@frc@displen}{\ensuremath{%
      \displaystyle{M}}}%
    \settoheight{\si@frc@textlen}{\ensuremath{%
      \textstyle{M}}}%
    \settoheight{\si@frc@suplen}{\ensuremath{%
      \scriptstyle{M}}}%
    \settoheight{\si@frc@ssuplen}{%
      \ensuremath{\scriptscriptstyle{M}}}%
    \addtolength{\si@frc@displen}{-\si@frc@ssuplen}%
    \addtolength{\si@frc@textlen}{-\si@frc@ssuplen}%
    \addtolength{\si@frc@suplen}{-\si@frc@ssuplen}%
    \mathchoice
      {\raisebox{\si@frc@displen}{\ensuremath{%
        \scriptstyle{#1}}}}%
      {\raisebox{\si@frc@textlen}{\ensuremath{%
         \scriptstyle{#1}}}}%
      {\raisebox{\si@frc@suplen}%
        {\ensuremath{\scriptscriptstyle{#1}}}}%
      {\raisebox{\si@frc@ssuplen}%
        {\ensuremath{\scriptscriptstyle{#1}}}}%
    \mkern-2mu\relax/\mkern-1mu\relax
    \bgroup
      \mathchoice
        {\scriptstyle}%
        {\scriptstyle}%
        {\scriptscriptstyle}%
        {\scriptscriptstyle}%
      {#2}%
    \egroup
  \endgroup}
%    \end{macrocode}
%\end{macro}
%\begin{macro}{\si@frc@textnf}
%\darg{numerator}
%\darg{denominator}
% A stripped down version of the \pkg{nicefrac} system for text mode.
%    \begin{macrocode}
\newcommand*{\si@frc@textnf}[2]{%
  \begingroup
    \settoheight{\si@frc@textlen}{M}%
    \settoheight{\si@frc@ssuplen}{\fontsize\sf@size\z@\relax
      \selectfont{M}}%
    \addtolength{\si@frc@textlen}{-\si@frc@ssuplen}%
    \raisebox{\si@frc@textlen}{\fontsize\sf@size\z@\relax
      \selectfont{#1}}%
    \hspace{-0.25ex}/\hspace{-0.25ex}%
    \hbox{\fontsize\sf@size\z@\selectfont{#2}}%
  \endgroup}
%    \end{macrocode}
%\end{macro}
%\begin{macro}{\si@frc@ugly}
%\darg{numerator}
% The \cs{si@frc@ugly} macro is needed to emulate the \opt{ugly}
% option in \units, where output depends on the current mode.
%    \begin{macrocode}
\newcommand*{\si@frc@ugly}[1]{%
  \renewcommand*{\si@tempa}{#1}%
  \ifmmode
    \expandafter\si@frc@frac
  \else
    \renewcommand*{\si@tempb}{1}%
    \ifx\si@tempa\si@tempb
%    \end{macrocode}
% The slash switch cannot be used, so the possibility of the
% numerator being one is handled here.
%    \begin{macrocode}
      \setbox\si@tempboxa=\hbox{\ensuremath{\si@valuesep}}%
      \hskip-\wd\si@tempboxa\relax
      \renewcommand*{\si@tempa}{}%
    \fi
    \expandafter\si@frc@slash
  \fi
  {\si@tempa}}
%    \end{macrocode}
%\end{macro}
%
%\subsection{Font control}
% A number of controls and tests are needed to control the font used
% for output.  Underlying all of this is the \AMS\ package
% \pkg{amstext} package, providing the \cs{text} command.  Much of
% the font control system here is taken more or less verbatim from
% \SIstyle; modifications have been made to fit the \currpkg
% interface.
%
%\begin{macro}{\si@fam@sf}
%\begin{macro}{\si@fam@tt}
% The package needs to know the maths font families in use.  This is
% set right at the start of the document, after any changes can have
% been made by, for example, \pkg{fontspec}.
%    \begin{macrocode}
\g@addto@macro{\document}{%
  \ifdefined\mathsf
    \setbox\si@tempboxa=\hbox{%
      $\mathsf{\global\chardef\si@fam@sf=\fam}$}%
  \else
    \si@log@inf{\string\mathsf not found}%
    \global\chardef\si@fam@sf=99\relax
  \fi
  \ifdefined\mathtt
    \setbox\si@tempboxa=\hbox{%
      $\mathtt{\global\chardef\si@fam@tt=\fam}$}%
  \else
    \si@log@inf{\string\mathtt not found}%
    \global\chardef\si@fam@tt=99\relax
  \fi}
%    \end{macrocode}
%\end{macro}
%\end{macro}
%\begin{macro}{\si@fam@ifbtext}
%\begin{macro}{\si@fam@ifbmaths}
%\darg{code}
% These tests check for bold in text and maths mode, respectively.
%    \begin{macrocode}
\newcommand*{\si@fam@ifbtext}[1]{%
  \if b\expandafter\@car\f@series\@nil
    #1\fi}
\newcommand*{\si@fam@ifbmaths}[1]{%
  \renewcommand*{\si@tempa}{bold}%
  \ifx\math@version\si@tempa
    #1\fi}
%    \end{macrocode}
%\end{macro}
%\end{macro}
%\begin{macro}{\si@fam@ifbinline}
% For compatibility with \units, a method to change the behaviour
% when in inline maths is needed for the bold detector.
%    \begin{macrocode}
\newcommand*{\si@fam@ifbinline}{%
  \ifsi@inlinebtext
    \expandafter\si@fam@ifbtext
  \else
    \expandafter\si@fam@ifbmaths
  \fi}
%    \end{macrocode}
%\end{macro}
%\begin{macro}{\si@fam@ifitext}
%\darg{code}
% This test check for italic or slanted text in text mode, by
% negation (upright text is |n|).
%    \begin{macrocode}
\newcommand*{\si@fam@ifitext}[1]{%
  \if n\expandafter\@car\f@series\@nil\else
    #1\fi}
%    \end{macrocode}
%\end{macro}
%\begin{macro}{\si@fam@mode}
% Detection of the current mode needs to happen``early'' (before any
% change of \cs{ensuremath}).  So a short macro is provided to do the
% job.
%    \begin{macrocode}
\newcommand*{\si@fam@mode}{%
  \ifsi@obeymode
    \ifmmode
      \sisetup{mode=maths}%
    \else
      \sisetup{mode=text}%
    \fi
  \fi}
%    \end{macrocode}
%\end{macro}
%\begin{macro}{\si@fam@colourcmd}
% The colour command is set up.
%    \begin{macrocode}
\AtBeginDocument{
  \@ifpackageloaded{color}
    {\let\si@fam@colourcmd\color}
    {\let\si@fam@colourcmd\@gobble}}
%    \end{macrocode}
%\end{macro}
%
%\begin{macro}{\ifsi@fam@set}
%\begin{macro}{\ifsi@textmode}
% A marker is set up to check if font-matching has been taken place.
% A second flag is used to track the use of text mode.
%    \begin{macrocode}
\newif\ifsi@fam@set
\newif\ifsi@textmode
%    \end{macrocode}
%\end{macro}
%\end{macro}
%\begin{macro}{\si@fam@set}
% Using the code from \SIstyle as a base, a set of tests are used to
% set the current font families and weights.  To begin with, the
% mode to use is set up.
%    \begin{macrocode}
\newcommand*{\si@fam@set}{%
  \ifsi@out@num
    \ifsi@numtextmode
      \expandafter\expandafter\expandafter\si@textmodetrue
    \else
      \expandafter\expandafter\expandafter\si@textmodefalse
    \fi
  \else
    \ifsi@unittextmode
      \expandafter\expandafter\expandafter\si@textmodetrue
    \else
      \expandafter\expandafter\expandafter\si@textmodefalse
    \fi
  \fi
%    \end{macrocode}
%\begin{macro}{\si@mathsrm}
%\begin{macro}{\si@mathssf}
%\begin{macro}{\si@mathstt}
%\begin{macro}{\si@textrm}
%\begin{macro}{\si@textsf}
%\begin{macro}{\si@texttt}
%\begin{macro}{\si@colourcmd}
%\begin{macro}{\si@colour}
% The appropriate font macros are now established, if necessary.
%    \begin{macrocode}
  \ifsi@fam@set\else
    \let\si@colourcmd\@gobble
    \ifsi@out@num
      \let\si@mathsrm\si@valuemathsrm
      \let\si@mathssf\si@valuemathssf
      \let\si@mathstt\si@valuemathstt
      \let\si@textrm\si@valuetextrm
      \let\si@textsf\si@valuetextsf
      \let\si@texttt\si@valuetexttt
      \ifsi@colourvalues
        \let\si@colourcmd\si@fam@colourcmd
      \fi
      \let\si@colour\si@valuecolour
    \else
      \let\si@mathsrm\si@unitmathsrm
      \let\si@mathssf\si@unitmathssf
      \let\si@mathstt\si@unitmathstt
      \let\si@textrm\si@unittextrm
      \let\si@textsf\si@unittextsf
      \let\si@texttt\si@unittexttt
      \ifsi@colourunits
        \let\si@colourcmd\si@fam@colourcmd
      \fi
      \let\si@colour\si@unitcolour
    \fi
  \fi
  \si@fam@settrue
%    \end{macrocode}
%\end{macro}
%\end{macro}
%\end{macro}
%\end{macro}
%\end{macro}
%\end{macro}
%\end{macro}
%\end{macro}
% The temporary macros are needed for the \cs{ifx} tests, which need to
% be expanded once.
%    \begin{macrocode}
  \edef\si@tempa{\sfdefault}%
  \edef\si@tempb{\ttdefault}%
%    \end{macrocode}
%\begin{macro}{\si@fam@maths}
%\begin{macro}{\si@fam@text}
% The surrounding font family is only tested if matching is
% requested.  First, the defaults are set up assuming no detection
% takes place.
%    \begin{macrocode}
  \expandafter\let\expandafter\si@fam@maths
    \csname\si@mathsrm\endcsname
  \expandafter\let\expandafter\si@fam@text
    \csname\si@textrm\endcsname
  \ifsi@obeyfamily
    \si@log@debug{Font detection: checking font}%
%    \end{macrocode}
%\end{macro}
%\end{macro}
% The detection code has to check the mode currently in operation.
% Display mathematics can be handled in two ways, so this means some
% code is repeated: it is spun out to separate routines.
%    \begin{macrocode}
    \ifmmode
      \ifinner
        \si@log@debug{Font detection: inline maths}%
        \si@fam@dettext
      \else
        \si@log@debug{Font detection: display maths}%
        \ifsi@detectdisplay
          \si@fam@detmaths
        \else
          \si@fam@dettext
        \fi
      \fi
    \else
      \si@log@debug{Font detection: text}%
      \si@fam@dettext
    \fi
  \else
    \si@log@debug{Font detection: inactive}%
  \fi
%    \end{macrocode}
%\begin{macro}{\si@fam@bold}
% With the font family set, the next check is for bold text.  This
% again needs to examine the current mode. Things are a bit more
% complex than in \SIstyle as it is possible to be typesetting in
% either text or maths mode.  The bold command is set up with
% \cs{def}, as nested calls can occur.
%    \begin{macrocode}
  \def\si@fam@bold{\unboldmath\mdseries}%
  \ifsi@obeybold
    \si@log@debug{Weight detection: checking weight}%
    \ifmmode
      \ifdim\displaywidth>0pt\relax
        \ifsi@detectdisplay
          \expandafter\si@fam@ifbmaths
        \else
          \expandafter\si@fam@ifbtext
        \fi
        \si@fam@setbold
      \else
        \si@fam@ifbinline\si@fam@setbold
      \fi
    \else
      \si@fam@ifbtext\si@fam@setbold
    \fi
  \fi
%    \end{macrocode}
%\end{macro}
%\begin{macro}{\si@fam@italic}
% The value of \opt{obeyitalic} is now tested; as this does nothing
% in maths mode, a reminder is added to the log.
%    \begin{macrocode}
  \let\si@fam@italic\upshape
  \ifsi@obeyitalic
    \si@log@debug{Italic detection: checking italic}%
    \si@fam@ifitext
      {\let\si@fam@italic\relax
       \si@log@debug{Italic detection: italic}}%
  \fi}
%    \end{macrocode}
%\end{macro}
%\end{macro}
%\begin{macro}{\si@fam@detmaths}
%\begin{macro}{\si@fam@dettext}
% Two detection macros are needed for maths and text mode. This
% allows handling of the various combinations without needing too
% many code lines.
%    \begin{macrocode}
\newcommand*{\si@fam@detmaths}{%
  \ifnum\the\fam=\si@fam@sf
    \si@log@debug{Font detection: sf}%
    \expandafter\let\expandafter\si@fam@maths
      \csname\si@mathssf\endcsname
    \expandafter\let\expandafter\si@fam@text
      \csname\si@textsf\endcsname
  \else
    \ifnum\the\fam=\si@fam@tt
      \si@log@debug{Font detection: tt}%
      \expandafter\let\expandafter\si@fam@maths
        \csname\si@mathstt\endcsname
      \expandafter\let\expandafter\si@fam@text
        \csname\si@texttt\endcsname
    \else
      \si@log@debug{Font detection: rm}%
      \expandafter\let\expandafter\si@fam@maths
        \csname\si@mathsrm\endcsname
      \expandafter\let\expandafter\si@fam@text
        \csname\si@textrm\endcsname
    \fi
  \fi}
\newcommand*{\si@fam@dettext}{%
  \ifx\f@family\si@tempa
    \si@log@debug{Font detection: sf}%
    \expandafter\let\expandafter\si@fam@maths
      \csname\si@mathssf\endcsname
    \expandafter\let\expandafter\si@fam@text
      \csname\si@textsf\endcsname
  \else
    \ifx\f@family\si@tempb
      \si@log@debug{Font detection: tt}%
      \expandafter\let\expandafter\si@fam@maths
        \csname\si@mathstt\endcsname
      \expandafter\let\expandafter\si@fam@text
        \csname\si@texttt\endcsname
    \else
      \si@log@debug{Font detection: rm}%
      \expandafter\let\expandafter\si@fam@maths
        \csname\si@mathsrm\endcsname
      \expandafter\let\expandafter\si@fam@text
        \csname\si@textrm\endcsname
    \fi
  \fi}
%    \end{macrocode}
%\end{macro}
%\end{macro}
%\begin{macro}{\si@fam@setbold}
%\begin{macro}{\si@fam@boldify}
% For setting bold, a couple of control macros are needed.
%    \begin{macrocode}
\newcommand*{\si@fam@setbold}{%
  \si@log@debug{Weight detection: bold weight}%
  \let\si@fam@bold\si@fam@boldify}
\newcommand*{\si@fam@boldify}{\boldmath\bfseries}
%    \end{macrocode}
%\end{macro}
%\end{macro}
%
%\subsection{Formatting numbers}
%\begin{macro}{\num}
%\doarg{keyval options}
%\darg{number}
% The system used here is modelled on that in \numprint; the
% input is broken down into single tokens, each one is examined and
% the result is re-assembled into an output number.  However, various
% changes have been made to the system used, and so the macros here
% are not simply renamed copies of those in \numprint.  The user
% macro \cs{num} sets any local keys, then calls the number
% formatting macro on the processed number.
%    \begin{macrocode}
\si@newrobustcmd*{\num}[2][]{%
  \begingroup
    \sisetup{#1}%
    \si@fam@mode
    \si@num@intabfalse
    \si@log@debug{Processing \string\num\space input `#2'}%
    \expandafter\si@out@num\expandafter{\si@num{#2}}%
  \endgroup}
%    \end{macrocode}
%\end{macro}
%\begin{macro}{\ifsi@num@intab}
% A flag for processing inside a table is needed.
%    \begin{macrocode}
\newif\ifsi@num@intab
%    \end{macrocode}
%\end{macro}
%\begin{macro}{\si@num}
%\darg{number}
% This is the main processing macro.  Unlike the related macro in
% \numprint, the output of this macro is not subjected to any font
% changes.  That is left to one of the \cs{si@out@\ldots} macros.  No
% grouping is applied here; any call to \cs{si@num} (or any of the
% sub-macros) must be within a group as the definitions used rely on
% this.  Grouping is not applied here so that other macros can get
% the various separated parts of the input.
%    \begin{macrocode}
\newcommand*{\si@num}[1]{%
%    \end{macrocode}
% The argument of the macro is fully expanded before any processing.
% By using \cs{scantokens}, any odd problems from packages with
% active characters can be avoided.
%    \begin{macrocode}
  \si@num@fixpm
  \begingroup
    \makeatletter
    \@makeother{\,}%
    \@makeother{\.}%
    \@makeother{\+}%
    \@makeother{\-}%
    \def~{}%
    \def\,{}%
    \catcode`\~=\active\relax
    \catcode`\^=\active\relax
    \catcode`\_=\active\relax
    \everyeof{\noexpand}%
    \endlinechar\m@ne
    \protected@xdef\si@tempa{\scantokens{#1}}%
  \endgroup
%    \end{macrocode}
% Processing only takes place if there is actually something in the
% argument.  This is tested once ``hard'' spaces have been stripped
% out; if there is input other than spaces, the processor first
% checks the validity of the input, then moves on to format it.
%    \begin{macrocode}
  \si@ifnotmtarg{\si@tempa}
    {\si@num@ifvalid{\si@tempa}
      {\si@num@format{\si@tempa}}
%    \end{macrocode}
% The parser has to bailed-out, and so no further processing of
% the input is done.  Instead, whatever was passed to the macro is
% returned as supplied.
%    \begin{macrocode}
      {\si@log@err{Invalid character `#1' in numerical input}%
         {Only characters from the list
          `\si@numvalid'\MessageBreak should be present in the
          argument of the \string\num\space  macro\MessageBreak
         (or derivative such as an `s' column)}%
       {#1}}}}
%    \end{macrocode}
%\end{macro}
%\begin{macro}{\si@num@fixpm}
%\begin{macro}{\si@num@pm}
%\begin{macro}{\si@num@mp}
%\begin{macro}{\pm}
%\begin{macro}{\mp}
% With certain packages loaded, there can be issues with \cs{pm} and
% \cs{mp}.  To avoid this, the \eTeX\ \cs{protected} system is
% employed; this is only used within local groups.
%    \begin{macrocode}
\newcommand*{\si@num@fixpm}{%
  \let\si@num@pm\pm
  \let\si@num@mp\mp
  \protected\def\pm{\si@num@pm}%
  \protected\def\mp{\si@num@mp}}
%    \end{macrocode}
%\end{macro}
%\end{macro}
%\end{macro}
%\end{macro}
%\end{macro}
%\begin{macro}{\si@num@ifvalid}
%\begin{macro}{\si@num@valid}
%\darg{chars}
% Assuming that there is a non-space argument to \cs{si@num}, every
% character is checked to ensure it is valid in the context, so that
% further processing can occur without sanity checks.  If the
% character is valid, recursion occurs.
%    \begin{macrocode}
\newcommand*{\si@num@ifvalid}[1]{%
  \begingroup
    \si@switchtrue
    \expandafter\si@num@valid#1\@empty\@empty
    \ifsi@switch
      \aftergroup\@firstoftwo
    \else
      \aftergroup\@secondoftwo
    \fi
  \endgroup}
\def\si@num@valid#1#2\@empty{%
  \si@str@ifchrstr{#1}{\si@numvalid}
    {\ifx\@empty#2\@empty\else
       \si@num@valid#2\@empty\@empty\@empty
     \fi}
    {\si@switchfalse}}
%    \end{macrocode}
%\end{macro}
%\end{macro}
%\begin{macro}{\si@num@in}
%\begin{macro}{\si@num@out}
%\begin{macro}{\si@num@exp}
%\begin{macro}{\si@num@expsign}
%\begin{macro}{\si@num@mant}
%\begin{macro}{\si@num@mantsign}
%\begin{macro}{\si@num@err}
%\begin{macro}{\si@num@xpart}
%\begin{macro}{\si@num@ambig}
%\begin{macro}{\si@tab@out}
%\begin{macro}{\si@tab@expout}
% Various storage macros are needed.
%    \begin{macrocode}
\newcommand*{\si@num@in}{}
\newcommand*{\si@num@out}{}
\newcommand*{\si@num@exp}{}
\newcommand*{\si@num@expsign}{}
\newcommand*{\si@num@mant}{}
\newcommand*{\si@num@mantsign}{}
\newcommand*{\si@num@err}{}
\newcommand*{\si@num@xpart}{}
\newcommand*{\si@num@ambig}{}
\newcommand*{\si@tab@out}{}
\newcommand*{\si@tab@expout}{}
%    \end{macrocode}
%\end{macro}
%\end{macro}
%\end{macro}
%\end{macro}
%\end{macro}
%\end{macro}
%\end{macro}
%\end{macro}
%\end{macro}
%\end{macro}
%\end{macro}
%\begin{macro}{\ifsi@num@erropen}
% A flag is set up for tracking unclosed errors.
%    \begin{macrocode}
\newif\ifsi@num@erropen
%    \end{macrocode}
%\end{macro}
%\begin{macro}{\si@num@format}
%\begin{macro}{\si@num@arg}
%\darg{number}
% The number processor starts by saving |#1| (odd things happen
% otherwise).  A hook is also provided to allow modifications by
% other macros.
%    \begin{macrocode}
\newcommand*{\si@num@arg}{}
\newcommand*{\si@num@format}[1]{%
  \protected@edef\si@num@arg{#1}%
  \si@log@debug{Formatting number `\si@num@arg'}%
%    \end{macrocode}
%\end{macro}
% The storage areas are emptied.
%    \begin{macrocode}
  \renewcommand*{\si@num@in}{}%
  \renewcommand*{\si@num@exp}{}%
  \renewcommand*{\si@num@expsign}{}%
  \renewcommand*{\si@num@mant}{}%
  \renewcommand*{\si@num@mantsign}{}%
  \renewcommand*{\si@num@err}{}%
  \renewcommand*{\si@num@xpart}{}%
%    \end{macrocode}
% Any ``x-part'' is now found, leaving the first number in
% \cs{si@num@in} and anything else in \cs{si@num@xpart}.
%    \begin{macrocode}
  \si@switchfalse
  \expandafter\si@num@findxpart\si@num@arg\@empty\@empty
%    \end{macrocode}
% The input is split into an mantissa and an exponent; the flag is
% used here to indicate if an exponent is found.  The mantissa will
% end up in \cs{si@num@mant}, and the exponent in \cs{si@num@exp}.
%    \begin{macrocode}
  \si@switchfalse
  \si@num@sepmantexp{\si@num@in}%
%    \end{macrocode}
% The sign and value of the mantissa and exponent are separated; the
% mantissa is done after the exponent as this makes life easier when
% using the table-formatting fork. Checks are then needed, as a sign
% with no value is potentially-valid for the mantissa (for example
% \num{-e10}).
%    \begin{macrocode}
  \si@num@sepsign{exp}%
  \si@num@sepsign{mant}%
  \ifx\@empty\si@num@exp\@empty
    \ifx\@empty\si@num@expsign\@empty\else
      \si@log@warn{Sign but no number for `\si@num@arg'}%
    \fi
    \let\si@num@expsign\@empty
  \fi
  \ifx\@empty\si@num@mant\@empty
    \ifx\@empty\si@num@mantsign\@empty\else
      \ifx\@empty\si@num@exp\@empty
        \si@log@warn{Sign but no number for `\si@num@arg'}%
        \let\si@num@mantsign\@empty
      \fi
    \fi
  \fi
%    \end{macrocode}
% A check for negative mantissa values is made, to allow some
% colour-based trickery.
%    \begin{macrocode}
  \renewcommand*{\si@tempa}{{-}}%
  \ifx\si@num@mantsign\si@tempa
    \ifsi@colourneg
      \expandafter\expandafter\expandafter\si@fam@colourcmd
    \else
      \expandafter\expandafter\expandafter\@gobble
    \fi
  \else
    \expandafter\@gobble
  \fi
  {\si@negcolour}%
%    \end{macrocode}
% The next stage is to process the remaining data, to find the
% decimal marker and reformat correctly.  These two macros control
% this entire process.  The exponent is processed first as this makes
% life easier when the system is used to typeset tabular
% material.\footnote{The contents of \cs{si@num@predec} and
% \cs{si@num@postdec} are needed for the mantissa.}
%    \begin{macrocode}
  \si@num@procnum{exp}%
  \si@num@procnum{mant}%
%    \end{macrocode}
% If the exponent is zero, then if might need to be deleted.
%    \begin{macrocode}
  \si@str@ifonlychrs{\si@num@exp}{0\si@numdecimal}
    {\ifsi@allowzeroexp\else
       \renewcommand*{\si@num@exp}{}%
       \ifx\@empty\si@num@mant\@empty
         \renewcommand*{\si@num@mant}{1}%
       \fi
     \fi}{}%
%    \end{macrocode}
% To build up the number for typesetting, a rather complex series
% of tests is needed.  First, the ``ambiguous error'' flag is
% set if there is an exponent and the package has been asked to check
% this.  The same flag will already be true if a unit is present and
% checking is active.
%    \begin{macrocode}
  \ifx\@empty\si@num@exp\@empty\else
    \ifsi@trapambigerr
      \expandafter\expandafter\expandafter\si@num@ambigerrtrue
    \fi
  \fi
%    \end{macrocode}
%\begin{macro}{\si@num@out}
%\begin{macro}{\si@tab@out}
%\begin{macro}{\si@tab@expout}
% Processing now divides, as when used with the |S| column some extra
% steps are needed.  Inside a table, the macro \cs{si@tab@out} holds
% the part of the mantissa before the decimal sign.  The post-decimal
% part then ends up in \cs{i@num@out}.  On the other hand, under
% normal circumstances the entire mantissa should be copied to
% \cs{si@num@out}.
%    \begin{macrocode}
  \protected@edef\si@num@out{%
    \ensuremath{{\si@num@mantsign}}\si@num@mant}%
  \renewcommand*{\si@tempa}{num}%
  \ifsi@num@intab
    \protected@edef\si@tab@out{%
      \ensuremath{{\si@num@mantsign}}\si@num@predec}%
    \protected@edef\si@num@out{\si@num@postdec}%
    \renewcommand*{\si@tempa}{tab}%
  \fi
%    \end{macrocode}
% Now there is the error part to handle.  For tables, a check has to
% be made so the error ends up in the correct part of the number. The
% value of \cs{si@tempa} is used to track this., and so is set up
% here.
%    \begin{macrocode}
  \ifx\@empty\si@num@postdec\@empty\else
    \renewcommand*{\si@tempa}{num}%
  \fi
  \ifx\@empty\si@num@err\@empty\else
%    \end{macrocode}
% If there is an error, and it is begin separated, the problem arises
% of the potential for ambiguous values.  This can only apply outside
% of a table, as \opt{seperr} is disabled in tabular material.
%    \begin{macrocode}
    \ifsi@seperr
      \ifsi@num@ambigerr
        \protected@edef\si@num@out{%
          \ensuremath{\si@openerr}\si@num@out}%
        \si@repeatunitsfalse
        \expandafter\si@num@erropentrue
      \else
%    \end{macrocode}
% If there is an exponential part and an error, errors are being
% separated and ambiguous errors are not trapped, then there is work
% to do.  The exponent part is added to the \emph{error}, and deleted
% from the mantissa if necessary.
%    \begin{macrocode}
        \ifsi@trapambigerr\else
          \ifx\@empty\si@num@exp\@empty\else
            \protected@edef\si@num@err{%
              \si@num@err\expandafter\@car\si@numexp\@nil
              \si@num@expsign\si@num@exp}%
            \ifsi@repeatunits\else
              \renewcommand*{\si@num@exp}{}%
              \renewcommand*{\si@num@expsign}{}%
            \fi
          \fi
        \fi
      \fi
%    \end{macrocode}
% If \opt{seperr} is not in force, the error and mantissa have to be
% recombined.  This is handled so that the same macro deals with
% tables and normal processing.
%    \begin{macrocode}
    \else
      \expandafter\protected@edef\csname
        si@\si@tempa @out\endcsname{%
          \si@num@out\ensuremath{\si@errspace}\ensuremath
          {\si@openerr}\si@num@err\ensuremath{\si@closeerr}}%
      \renewcommand*{\si@num@err}{}%
    \fi
  \fi
%    \end{macrocode}
% The main reconstruction can now occur.  This performs various
% checks on the validity of the input, and adds the necessary
% ``filler'' between the supplied data.
%    \begin{macrocode}
  \renewcommand*{\si@tempa}{num@out}%
  \ifsi@num@erropen
    \renewcommand*{\si@tempa}{num@ambig}%
  \fi
  \ifsi@num@intab
    \renewcommand*{\si@tempa}{tab@expout}%
  \fi
  \ifx\@empty\si@num@exp\@empty
    \ifx\@empty\si@num@mant\@empty
      \si@log@err{Invalid number format `\si@num@arg'}
        {Something is wrong with the number format; does it
         contain \MessageBreak any numbers (from the list
         `\si@numdigits')?}%
      \renewcommand*\si@num@out{}%
    \fi
  \else
    \ifx\@empty\si@num@mant\@empty\else
      \expandafter\protected@edef\csname
        si@\si@tempa\endcsname{%
          \csname si@\si@tempa\endcsname\ensuremath{{}%
          \si@expproduct{}}}%
    \fi
    \expandafter\protected@edef\csname
      si@\si@tempa\endcsname{%
        \csname si@\si@tempa\endcsname\si@expbase
        \textsuperscript{\ensuremath{\si@num@expsign}%
        \si@num@exp}}%
  \fi
%    \end{macrocode}
%\end{macro}
%\end{macro}
%\end{macro}
% With everything done, the result is output.
%    \begin{macrocode}
  \ifsi@num@intab\else
    \expandafter\si@num@out
  \fi
%    \end{macrocode}
% If there is anything inside the ``x-part'', then there is now
% more work to do.
%    \begin{macrocode}
  \ifx\@empty\si@num@err\@empty\else
    \expandafter\si@num@procerr
  \fi
  \ifsi@num@erropen
    \expandafter\si@out@num\expandafter{%
      \ensuremath{\si@closeerr}}%
      \ifx\@empty\si@num@ambig\@empty\else
        \expandafter\si@out@num\expandafter{\si@num@ambig}%
        \renewcommand*{\si@num@ambig}{}%
    \fi
  \fi
  \si@num@erropenfalse
  \ifx\@empty\si@num@xpart\@empty\else
    \expandafter\si@num@sepxpart
  \fi}
%    \end{macrocode}
%\end{macro}
%\begin{macro}{\si@num@findxpart}
% Before processing the number, any multiplied parts have to be found
% and removed.  As there can be more than one product sign, the
% building process here has no error checking.
%    \begin{macrocode}
\def\si@num@findxpart#1#2\@empty{%
  \si@str@ifchrstr{#1}{\si@numprod}
    {\si@switchtrue\si@seperrfalse}{}%
  \ifsi@switch
    \protected@edef\si@num@xpart{\si@num@xpart#1}%
  \else
    \protected@edef\si@num@in{\si@num@in#1}%
  \fi
  \ifx\@empty#2\@empty\else
    \si@num@findxpart#2\@empty
  \fi}
%    \end{macrocode}
%\end{macro}
%\begin{macro}{\si@num@sepmantexp}
%\begin{macro}{\si@num@mantexp}
%\darg{number}
% Splitting the mantissa and exponent first checks for characters to
% gobble, which are simply thrown away.  For any other input, there
% are two possibilities.  If the character is an exponent marker,
% then the package switches from collecting the mantissa to
% collecting the exponent (after a sanity check).  All other
% characters are added to either the mantissa or the exponent, as
% appropriate.
%    \begin{macrocode}
\newcommand*{\si@num@sepmantexp}[1]{%
  \expandafter\si@num@mantexp#1\@empty\@empty}
\def\si@num@mantexp#1#2\@empty{%
  \si@str@ifchrstr{#1}{\si@numgobble}
    {\si@log@debug{Gobbling `#1' in \si@num@arg}}
    {\si@str@ifchrstr{#1}{\si@numexp}
       {\ifsi@switch
          \si@log@err{Duplicate exponent marker found}
            {Only a single exponent character \MessageBreak
            (from the list `\si@numexp')\MessageBreak may
            occur in a numerical argument}%
        \else
          \si@log@debug{Exponent marker `#1' found in
            `\si@num@arg'}%
        \fi
        \si@switchtrue}%
       {\ifsi@switch
          \expandafter\si@num@addexp
        \else
          \expandafter\si@num@addmnt
        \fi
        {#1}}}%
%    \end{macrocode}
% If the recursion has not bottomed out, another loop occurs.
%    \begin{macrocode}
  \ifx\@empty#2\@empty
    \expandafter\@gobble
  \else
    \expandafter\si@num@sepmantexp
  \fi
  {#2}}
%    \end{macrocode}
%\end{macro}
%\end{macro}
%\begin{macro}{\si@num@addmnt}
%\begin{macro}{\si@num@addexp}
%\darg{char}
% To allow \cs{expandafter} use in the above, the actual addition to
% the appropriate macro is handled here.  Two helper macros are
% needed.
%    \begin{macrocode}
\newcommand*{\si@num@addmnt}[1]{%
  \si@num@addmntexp{#1}{mant}{mantissa}}
\newcommand*{\si@num@addexp}[1]{%
  \si@num@addmntexp{#1}{exp}{exponent}}
%    \end{macrocode}
%\end{macro}
%\end{macro}
%\begin{macro}{\si@num@addmntexp}
%\darg{char}
%\darg{either \texttt{mant} or \texttt{exp}}
%\darg{info-text}
% Then the business end.
%    \begin{macrocode}
\newcommand*{\si@num@addmntexp}[3]{%
  \si@log@debug{Adding `#1' to #3 for `\si@num@arg'}%
  \expandafter\protected@edef\csname si@num@#2\endcsname{%
    \csname si@num@#2\endcsname#1}}
%    \end{macrocode}
%\end{macro}
%\begin{macro}{\si@num@sepsign}
%\darg{either \texttt{mant} or \texttt{exp}}
% The input is now tested for a sign.  If one exists, it is
% transferred into the \cs{si@num@}|#1sign| storage macro, with the
% remained of the number in \cs{si@num@}|#1|.  The only check made
% directly here is that there is something to process.
%    \begin{macrocode}
\newcommand*{\si@num@sepsign}[1]{%
  \expandafter\ifx\expandafter\@empty
    \csname si@num@#1\endcsname\@empty
    \expandafter\@gobble
  \else
    \expandafter\si@num@gensign
  \fi
  {#1}}
%    \end{macrocode}
%\end{macro}
%\begin{macro}{\si@num@gensign}
%\darg{either \texttt{mant} or \texttt{exp}}
% The sign generator starts by calling the procedure to check if the
% input contains a valid one- or two-digit sign.  The results are
% returned as \cs{si@num@sign} and \cs{si@num@value}.
%    \begin{macrocode}
\newcommand*{\si@num@gensign}[1]{%
  \expandafter\expandafter\expandafter\si@num@findsign
    \csname si@num@#1\endcsname\@empty\@empty
%    \end{macrocode}
% If no sign has been found, then there may be a need to add one
% anyway.
%    \begin{macrocode}
  \ifx\@empty\si@num@sign\@empty
    \ifx\@empty\si@num@value\@empty
      \expandafter\expandafter\expandafter\@gobble
    \else
      \expandafter\expandafter\expandafter\si@num@addsign
    \fi
  \else
    \expandafter\@gobble
  \fi
  {#1}%
%    \end{macrocode}
% The appropriate storage areas are now assigned.
%    \begin{macrocode}
  \expandafter\let\csname si@num@#1sign\endcsname\si@num@sign
  \expandafter\let\csname si@num@#1\endcsname\si@num@value}
%    \end{macrocode}
%\end{macro}
%\begin{macro}{\si@num@findsign}
%\begin{macro}{\si@num@sign}
%\begin{macro}{\si@num@value}
% The first one or two characters of the mantissa or exponent may
% contain a sign.  To test for this, the first two characters of the
% number are split off, and examined.  Two characters are used so
% that \cs{pm} and \cs{mp} can be represented by |+-| and |-+|,
% respectively. To allow the user to alter the valid signs, but
% retain this conversion, the generic character test is used before
% checking specific matches.
%    \begin{macrocode}
\newcommand*{\si@num@sign}{}
\def\si@num@findsign#1#2#3\@empty{%
  \si@num@delplusfalse
  \si@str@ifchrstr{#1}{\si@numsign}{%
    \si@str@ifchrstr{#2}{\si@numsign}{%
      \if +#1%
        \if -#2%
          \si@log@debug{Found sign combination +- for
            `\si@num@arg'}%
          \renewcommand*{\si@num@sign}{{\si@pm}}%
        \else
          \si@log@inf{Unknown sign combination `#1#2'}%
          \renewcommand*{\si@num@sign}{{#1#2}}%
        \fi
      \else
        \if -#1%
          \if +#2%
            \si@log@debug{Found sign combination -+ for
              `\si@num@arg'}%
            \renewcommand*{\si@num@sign}{{\mp}}%
          \else
            \si@log@inf{Unknown sign combination `#1#2'}%
            \renewcommand*{\si@num@sign}{{#1#2}}%
          \fi
        \else
          \si@log@inf{Unknown sign combination `#1#2'}%
          \renewcommand*{\si@num@sign}{{#1#2}}%
        \fi
      \fi
      \protected@edef\si@num@value{#3}}%
%    \end{macrocode}
% Only one valid sign character.
%    \begin{macrocode}
      {\si@log@debug{Found single sign character `#1' for
         `\si@num@arg'}%
       \renewcommand*{\si@num@sign}{{#1}}%
       \if +#1%
         \ifsi@retainplus\else
           \expandafter\expandafter\expandafter\si@num@killsign
         \fi
       \fi
       \protected@edef\si@num@value{#2#3}}}%
%    \end{macrocode}
% No valid sign, so \cs{@empty} is returned for the sign .
%    \begin{macrocode}
    {\si@log@debug{No sign found for `\si@num@arg'}%
     \renewcommand*{\si@num@sign}{}%
     \protected@edef\si@num@value{#1#2#3}}}
%    \end{macrocode}
%\end{macro}
%\end{macro}
%\end{macro}
%\begin{macro}{\ifsi@num@delplus}
%\begin{macro}{\si@num@killsign}
% A simple spin-out to remove a plus sign.  A flag is set as it might
% be useful to know this.
%    \begin{macrocode}
\newif\ifsi@num@delplus
\newcommand*{\si@num@killsign}{%
  \si@num@delplustrue
  \renewcommand*{\si@num@sign}{}}
%    \end{macrocode}
%\end{macro}
%\end{macro}
%\begin{macro}{\si@num@addsign}
%\begin{macro}{\si@num@asign}
%\darg{either \texttt{mant} or \texttt{exp}}
% The macro to add a sign to an unsigned number has to check whether
% this is a mantissa or an exponent.  The result is still placed in
% \cs{si@num@sign} for ease of processing later.
%    \begin{macrocode}
\newcommand*{\si@num@addsign}[1]{%
  \begingroup
    \renewcommand*{\si@tempa}{#1}%
    \renewcommand*{\si@tempb}{mant}%
    \ifx\si@tempa\si@tempb
      \aftergroup\@firstoftwo
    \else
      \aftergroup\@secondoftwo
    \fi
  \endgroup
  {\ifsi@num@signmant
     \expandafter\si@num@asign
   \else
     \expandafter\@gobble
   \fi
   {mantissa}}
  {\ifsi@num@signexp
     \expandafter\si@num@asign
   \else
     \expandafter\@gobble
   \fi
   {exponent}}}
\newcommand*{\si@num@asign}[1]{%
  \let\si@num@sign\si@sign
  \si@log@debug{Adding sign \si@sign\space to #1 for
    `\si@num@arg'}}
%    \end{macrocode}
%\end{macro}
%\end{macro}
%\begin{macro}{\si@num@procnum}
%\darg{either \texttt{mant} or \texttt{exp}}
% The control macro for processing the number (plus any extra
% characters).
%    \begin{macrocode}
\newcommand*{\si@num@procnum}[1]{%
  \expandafter\ifx\expandafter\@empty
    \csname si@num@#1\endcsname\@empty
    \expandafter\@gobble
  \else
    \expandafter\si@num@finddigits
  \fi
  {#1}}
%    \end{macrocode}
%\end{macro}
%\begin{macro}{\si@num@predec}
%\begin{macro}{\si@num@postdec}
% Two new storage areas are defined.
%    \begin{macrocode}
\newcommand*{\si@num@predec}{}
\newcommand*{\si@num@postdec}{}
%    \end{macrocode}
%\end{macro}
%\end{macro}
%\begin{macro}{\si@num@finddigits}
%\darg{either \texttt{mant} or \texttt{exp}}
% The core digit processor divides the number into the parts before
% and after the decimal point marker.  The temporary switch is used
% to indicate finding a decimal marker.
%    \begin{macrocode}
\newcommand*{\si@num@finddigits}[1]{%
  \renewcommand*{\si@num@predec}{}%
  \renewcommand*{\si@num@postdec}{}%
  \si@switchfalse
  \expandafter\expandafter\expandafter\si@num@digits
    \csname si@num@#1\endcsname\@empty\@empty
%    \end{macrocode}
% Tests are now made to see if padding zeros are needed.  The
% trailing test needs to verify if a decimal marker was found, as
% well as if a zero is needed.
%    \begin{macrocode}
  \ifx\@empty\si@num@predec\@empty
    \ifsi@num@padlead
      \expandafter\expandafter\expandafter\si@num@addprezero
    \fi
  \fi
  \ifx\@empty\si@num@postdec\@empty
    \ifsi@num@padtrail
      \ifsi@switch
        \expandafter\expandafter\expandafter\expandafter
          \expandafter\expandafter\expandafter
          \si@num@addpostzero
      \fi
    \fi
  \fi
%    \end{macrocode}
% The next checks to make concern input validity in a more
% mathematical sense.  First, if the number is zero, then no sign
% should be given under any circumstances.  Then leading zeros need
% to be removed from the input.  This is slightly complicated by the
% potential presence of ``extra'' characters.
%    \begin{macrocode}
  \si@num@unsign{#1}%
  \ifx\@empty\si@num@predec\@empty
  \else
    \expandafter\si@num@nozero
  \fi
%    \end{macrocode}
% A sanity check is made to ensure that the supplied number consisted
% of more than a decimal marker.
%    \begin{macrocode}
  \ifx\@empty\si@num@predec\@empty
    \ifx\@empty\si@num@postdec\@empty
      \expandafter\expandafter\expandafter\@gobble
    \else
      \expandafter\expandafter\expandafter\si@num@sepdigits
    \fi
  \else
    \expandafter\si@num@sepdigits
  \fi
  {#1}}
%    \end{macrocode}
%\end{macro}
%\begin{macro}{\si@num@digits}
% The \cs{si@num@digits} macro compares each character in the input
% against the list of characters valid at this stage: numbers,
% decimal markers and ``extra'' characters.
%    \begin{macrocode}
\def\si@num@digits#1#2\@empty{%
  \si@str@ifchrstr{#1}{\si@numdecimal}
    {\ifsi@switch
       \si@log@err{Duplicate decimal marker in `\si@num@arg'}
         {Only a single decimal marker (from the list
            `\si@numdecimal')\MessageBreak may occur in a
            numerical argument}%
     \else
       \si@log@debug{Found decimal marker `#1' in
         `\si@num@arg'}%
       \expandafter\si@switchtrue
     \fi}
%    \end{macrocode}
% The earlier code only checks for a sign at the start of the text. A
% check is therefore needed for a sign after the first two
% characters; if one is found, it is ignored.
%    \begin{macrocode}
    {\si@str@ifchrstr{#1}{\si@numsign}
       {\si@log@err{Misplaced sign character
          `#1' in `\si@num@arg'}
          {Sign characters `\si@numsign' can only
            occur\MessageBreak at the start of a number}}
%    \end{macrocode}
% The current character is added to the appropriate stack; this is
% ``spun out'' to avoid problems with expansion of the switch code.
%    \begin{macrocode}
       {\ifsi@switch
          \expandafter\si@num@post
        \else
          \expandafter\si@num@pre
        \fi
        {#1}}}%
  \ifx\@empty#2\@empty\else
    \si@num@digits#2\@empty\@empty
  \fi}
%    \end{macrocode}
%\end{macro}
%\begin{macro}{\si@num@pre}
%\begin{macro}{\si@num@post}
%\darg{char}
% Storage of the result is spun out.
%    \begin{macrocode}
\newcommand*{\si@num@pre}[1]{%
  \si@num@prepost{#1}{pre}{integer}}
\newcommand*{\si@num@post}[1]{%
  \si@num@prepost{#1}{post}{decimal}}
%    \end{macrocode}
%\end{macro}
%\end{macro}
%\begin{macro}{\si@num@prepost}
%\darg{char}
%\darg{either \texttt{pre} or \texttt{post}}
%\darg{info-text}
% Then actually stored here.
%    \begin{macrocode}
\newcommand*{\si@num@prepost}[3]{%
  \expandafter\protected@edef\csname si@num@#2dec\endcsname{%
    \csname si@num@#2dec\endcsname#1}%
  \si@log@debug{Adding `#1' to #3 part for `\si@num@arg'}}
%    \end{macrocode}
%\end{macro}
%\begin{macro}{\si@num@addprezero}
%\begin{macro}{\si@num@addpostzero}
% A similar set of macros are used for the padding zeros.
%    \begin{macrocode}
\newcommand*{\si@num@addprezero}{%
  \si@num@addpzero{pre}{leading}}
\newcommand*{\si@num@addpostzero}{%
  \si@num@addpzero{post}{trailing}}
%    \end{macrocode}
%\end{macro}
%\end{macro}
%\begin{macro}{\si@num@addpzero}
%\darg{either \texttt{pre} or \texttt{post}}
% In this case, there is no argument to be passed along.
%    \begin{macrocode}
\newcommand*{\si@num@addpzero}[2]{%
  \si@log@debug{Adding #2 zero for `\si@num@arg'}%
  \@namedef{si@num@#1dec}{0}}
%    \end{macrocode}
%\end{macro}
%\begin{macro}{\si@num@unsign}
%\begin{macro}{\si@num@nosign}
%\darg{either \texttt{mant} or \texttt{exp}}
% The trap for a sign with zero numerical input is made.  First, a
% check is made to see if there is a sign to worry about.  The pre-
% and post-decimal parts are then examined, to see if they contain
% something other than ``0'' or an extra character.
%    \begin{macrocode}
\newcommand*{\si@num@unsign}[1]{%
  \expandafter\ifx\expandafter\@empty
    \csname si@num@#1sign\endcsname\@empty
    \expandafter\@gobble
  \else
    \expandafter\si@num@nosign
  \fi
  {#1}}
\newcommand*{\si@num@nosign}[1]{%
  \begingroup
    \si@switchtrue
    \si@str@ifonlychrs{\si@num@predec\si@num@postdec}{0}
      {\si@switchfalse}{}%
    \ifsi@switch
      \aftergroup\@gobble
    \else
      \aftergroup\@firstofone
    \fi
  \endgroup
  {\si@log@debug{Zero value: removing any sign}%
   \ifsi@ang@sign\else
     \@namedef{si@num@#1sign}{}%
   \fi}}
%    \end{macrocode}
%\end{macro}
%\end{macro}
%\begin{macro}{\si@num@nozero}
%  A very short test for a totally zero pre-decimal component.
%    \begin{macrocode}
\newcommand*{\si@num@nozero}{%
  \si@str@ifonlychrs{\si@num@predec}{0}
    {\renewcommand*{\si@num@predec}{0}}{}}
%    \end{macrocode}
%\end{macro}
%\begin{macro}{\si@num@decimalhook}
% A hook is needed to attach things inside the group to happen
% afterwards, if the number is a decimal.
%    \begin{macrocode}
\newcommand*{\si@num@decimalhook}{}
%    \end{macrocode}
%\end{macro}
%\begin{macro}{\si@num@sepdigits}
%\darg{either \texttt{mant} or \texttt{exp}}
% The \cs{si@num@sepdigits} macro is only called if at least one of
% the mantissa and exponent contain something to output.
%    \begin{macrocode}
\newcommand*{\si@num@sepdigits}[1]{%
%    \end{macrocode}
% First an overall check is needed for additional characters.  By
% altering the contents of \cs{si@numextra}, the same code can be
% shared by two different checks.
%    \begin{macrocode}
  \begingroup
    \let\si@numextra\si@numaddn
    \protected@edef\si@tempa{\si@num@predec\si@num@postdec}%
    \si@num@ifextra{\si@tempa}
      {\aftergroup\@gobble}
      {\aftergroup\@firstofone}%
  \endgroup
%    \end{macrocode}
% Separation of the error in a number from the number itself is only
% attempted for the mantissa.
%    \begin{macrocode}
  {\renewcommand*{\si@tempb}{mant}%
   \renewcommand*{\si@tempc}{#1}%
   \ifx\si@tempb\si@tempc
     \expandafter\si@num@checkerr
   \fi}%
%    \end{macrocode}
%\changes{v1.1}{2008/08/22}{Fixed-length code generalised}
%\changes{v1.1}{2008/08/26}{Length-fixing only applies to
%  manitssa}
% If both parts of the number contain only digits, then any rounding
% can be attempted if there is no error part.  Otherwise, the input
% must be left alone.
%    \begin{macrocode}
  \protected@edef\si@tempa{\si@num@predec\si@num@postdec}%
  \expandafter\si@str@ifonlychrs\expandafter{\si@tempa}
    {0123456789}
    {\ifx\@empty\si@num@err\@empty
       \renewcommand*{\si@tempb}{mant}%
       \renewcommand*{\si@tempc}{#1}%
       \ifx\si@tempb\si@tempc
         \expandafter\expandafter\expandafter\si@num@fixlength
       \fi
     \fi}{}%
%    \end{macrocode}
% If the pre-decimal part contains nothing except numbers, then
% digit separation is carried out.
%    \begin{macrocode}
  \si@num@ifextra{\si@num@predec}{}
    {\expandafter\si@num@int\expandafter{\si@num@predec}}%
%    \end{macrocode}
% A decision is made about the decimal sign, then digit separation
% occurs on the post-decimal part of the number.
%    \begin{macrocode}
  \renewcommand*{\si@tempc}{}%
  \ifx\@empty\si@num@postdec\@empty\else
    \si@num@decimalhook
    \renewcommand*{\si@tempc}{%
      \ensuremath{{\si@decimalsymbol}}}%
    \si@num@ifextra{\si@num@postdec}{}
      {\expandafter\si@num@dec\expandafter{\si@num@postdec}}%
  \fi
%    \end{macrocode}
% The construction is finalised by re-combining the number.
%    \begin{macrocode}
  \expandafter\protected@edef\csname si@num@#1\endcsname
    {\si@num@predec\si@tempc\si@num@postdec}}
%    \end{macrocode}
%\end{macro}
%\begin{macro}{\si@num@ifextra}
%\begin{macro}{\si@num@extra}
%\darg{number}
% A relatively simple test for ``extra'' characters.  Once again, a
% bit of group trickery is used.
%    \begin{macrocode}
\newcommand*{\si@num@ifextra}[1]{%
  \begingroup
    \si@switchfalse
    \expandafter\si@num@extra#1\@empty\@empty
    \ifsi@switch
      \si@log@debug{Found `extra' characters in `#1'}%
      \aftergroup\@firstoftwo
    \else
      \aftergroup\@secondoftwo
    \fi
  \endgroup}
\def\si@num@extra#1#2\@empty{%
  \ifx\@empty#1\@empty\else
    \si@str@ifchrstr{#1}{\si@numextra}{\si@switchtrue}{}%
    \ifx\@empty#2\@empty\else
      \si@num@extra#2\@empty\@empty
    \fi
  \fi}
%    \end{macrocode}
%\end{macro}
%\end{macro}
%\begin{macro}{\ifsi@num@ambigerr}
%\begin{macro}{\si@num@checkerr}
% When separating out an error from a number, the first step is to
% see if there is a decimal part to the number.  If so, any error
% must be in that part of the decimal part; it is not possible to
% have an error starting in the integer part and continuing into the
% decimal part.
%    \begin{macrocode}
\newif\ifsi@num@ambigerr
\newcommand*{\si@num@checkerr}{%
  \ifx\@empty\si@num@postdec\@empty
    \expandafter\si@num@preerr
  \else
    \expandafter\si@num@posterr
  \fi}
%    \end{macrocode}
%\end{macro}
%\end{macro}
%\begin{macro}{\si@num@preerr}
% Errors in integers are easy to handle.  After finding the error,
% the result is simply stored in the error macro \cs{si@num@err}.
%    \begin{macrocode}
\newcommand*{\si@num@preerr}{%
  \si@num@seperr{pre}%
  \ifx\@empty\si@tempb\@empty\else
    \expandafter\renewcommand\expandafter*\expandafter
      \si@num@err\expandafter{\si@tempb}%
  \fi}
%    \end{macrocode}
%\end{macro}
%\begin{macro}{\si@num@posterr}
%\begin{macro}{\si@num@psterr}
% Life is more complex for an error in the decimal part.  This is
% found, then it may need zero-filling or the insertion of a decimal
% point.  This depends on whether the number of error digits is
% larger than the number of post-decimal digits.
%    \begin{macrocode}
\newcommand*{\si@num@posterr}{%
  \si@num@seperr{post}%
  \ifx\@empty\si@tempb\@empty\else
    \ifsi@seperr
      \expandafter\expandafter\expandafter\si@num@psterr
    \else
      \let\si@num@err\si@tempb
    \fi
  \fi}
\newcommand*{\si@num@psterr}{%
    \si@num@cntdigits{\si@tempb}%
    \si@tempcntb\si@tempcnta\relax
    \si@num@cntdigits{\si@num@postdec}%
    \ifnum\si@tempcnta<\si@tempcntb\relax
      \expandafter\si@num@largeerr
    \else
      \expandafter\si@num@smallerr
    \fi}
%    \end{macrocode}
%\end{macro}
%\end{macro}
%\begin{macro}{\si@num@seperr}
%\begin{macro}{\si@num@finderr}
%\darg{either \texttt{pre} or \texttt{post}}
% The usual hand-off is made for searching for an error.
%    \begin{macrocode}
\newcommand*{\si@num@seperr}[1]{%
  \si@switchfalse
  \renewcommand*{\si@tempa}{}%
  \renewcommand*{\si@tempb}{}%
  \expandafter\expandafter\expandafter\si@num@finderr
    \csname si@num@#1dec\endcsname\@empty\@empty
  \ifx\@empty\si@tempb\@empty\else
    \expandafter\let\csname si@num@#1dec\endcsname\si@tempa
  \fi}
\def\si@num@finderr#1#2\@empty{%
%    \end{macrocode}
% First a check for the opening character of an error.
%    \begin{macrocode}
  \si@str@ifchrstr{#1}{\si@numopenerr}
%    \end{macrocode}
% If the switch is set when an error is found, then something is
% wrong.
%    \begin{macrocode}
    {\ifsi@switch
       \si@log@err{Invalid error in number}
         {The numerical argument \si@num@arg\space has two (or
           more)\MessageBreak error-opening characters}%
     \else
       \expandafter\si@switchtrue
     \fi}
%    \end{macrocode}
% The end-of-error character has to be the last item of the input,
% and an error needs to start before it ends.
%    \begin{macrocode}
    {\si@str@ifchrstr{#1}{\si@numcloseerr}
       {\ifsi@switch
          \ifx\@empty#2\@empty\else
            \si@log@err{Invalid error in number}
              {The numerical argument \si@num@arg\space has an
              error-closing before the last character}%
          \fi
        \else
          \si@log@err{Invalid error in number}
           {The numerical argument \si@num@arg\space has an
              error-closing character\MessageBreak but no
              error-opening one}%
        \fi}
%    \end{macrocode}
% The input must be a digit.  It is stored in the appropriate area.
%    \begin{macrocode}
       {\ifsi@switch
          \expandafter\si@num@addtmpb
        \else
          \expandafter\si@num@addtmpa
        \fi
        {#1}}}%
  \ifx\@empty#2\@empty\else
    \si@num@finderr#2\@empty
  \fi}
%    \end{macrocode}
%\end{macro}
%\end{macro}
%\begin{macro}{\si@num@addtmpa}
%\begin{macro}{\si@num@addtmpb}
%\darg{number}
% Some quick methods for adding to the temporary macros with \cs{if}
% expansion.
%    \begin{macrocode}
\newcommand*{\si@num@addtmpa}[1]{\si@num@addtmp{a}{#1}}
\newcommand*{\si@num@addtmpb}[1]{\si@num@addtmp{b}{#1}}
%    \end{macrocode}
%\end{macro}
%\end{macro}
%\begin{macro}{\si@num@addtmp}
%\darg{either \texttt{a} or \texttt{b}}
%\darg{number}
% The addition is committed.
%    \begin{macrocode}
\newcommand*{\si@num@addtmp}[2]{%
  \expandafter\protected@edef\csname si@temp#1\endcsname{%
    \csname si@temp#1\endcsname#2}}
%    \end{macrocode}
%\end{macro}
%\begin{macro}{\si@num@cntdigits}
%\begin{macro}{\si@num@cntdgt}
%\darg{number}
% A recursive system for counting the number of characters in a given
% input; only used here to count digits in a number.
%    \begin{macrocode}
\newcommand*{\si@num@cntdigits}[1]{%
  \si@tempcnta\z@\relax
  \expandafter\si@num@cntdgt#1\@empty\@empty}
\def\si@num@cntdgt#1#2\@empty{%
  \ifx\@empty#1\@empty\else
    \advance\si@tempcnta\@ne\relax
  \fi
  \ifx\@empty#2\@empty\else
    \expandafter\si@num@cntdgt#2\@empty
  \fi}
%    \end{macrocode}
%\end{macro}
%\end{macro}
%\begin{macro}{\si@num@smallerr}
%\begin{macro}{\si@num@serr}
% For handling an error with no more digits than the post-decimal
% part of a number, zero-padding is undertaken before adding a
% decimal sign and (possibly) leading zero.
%    \begin{macrocode}
\newcommand*{\si@num@smallerr}{%
  \si@tempcntb\si@tempcnta\relax
  \si@num@serr
  \protected@edef\si@num@err{%
    \ifsi@num@padlead0\fi\expandafter\@car\si@numdecimal\@nil
    \si@tempb}}
\newcommand*{\si@num@serr}{%
  \si@num@cntdigits{\si@tempb}%
  \ifnum\si@tempcnta=\si@tempcntb\relax\else
    \protected@edef\si@tempb{0\si@tempb}%
    \expandafter\si@num@serr
  \fi}
%    \end{macrocode}
%\end{macro}
%\end{macro}
%\begin{macro}{\si@num@largeerr}
%\changes{v1.0m}{2008/09/16}{Fixed problem with error separation
%  crossing decimal boundary}
%\begin{macro}{\si@num@lerr}
%\begin{macro}{\si@num@movedigit}
% When the error has more digits than the decimal part, some
% shuffling is needed.
%    \begin{macrocode}
\newcommand*{\si@num@largeerr}{%
  \renewcommand*{\si@tempa}{}%
  \si@tempcntb\si@tempcnta\relax
  \si@num@lerr
  \protected@edef\si@num@err{%
    \si@tempa\expandafter\@car\si@numdecimal\@nil\si@tempb}}
\newcommand*{\si@num@lerr}{%
  \si@num@cntdigits{\si@tempb}%
  \ifnum\si@tempcnta=\si@tempcntb\relax\else
    \expandafter\si@num@movedigit\si@tempb\@empty\@empty
    \si@num@lerr
  \fi}
\def\si@num@movedigit#1#2\@empty{%
  \protected@edef\si@tempa{\si@tempa#1}%
  \protected@edef\si@tempb{#2}}
%    \end{macrocode}
%\end{macro}
%\end{macro}
%\end{macro}
%\begin{macro}{\si@num@fixlength}
% The checks for fixing number length are spun out here.
%    \begin{macrocode}
\newcommand*{\si@num@fixlength}{%
  \ifsi@fixdp
    \expandafter\si@num@fixdp
  \else
    \ifsi@fixsf
      \expandafter\expandafter\expandafter\si@num@fixsf
    \fi
  \fi}
%    \end{macrocode}
%\end{macro}
%\begin{macro}{\si@num@fixdp}
% The test for fixing decimal places starts by counting up the number
% of decimal digits.  The number is then padded, rounded or left
% alone.
%    \begin{macrocode}
\newcommand*{\si@num@fixdp}{%
  \si@num@cntdigits{\si@num@postdec}%
  \ifx\@empty\si@num@postdec\@empty
    \si@tempcnta\z@\relax
  \fi
  \ifnum\si@tempcnta>\si@num@dp\relax
    \expandafter\si@num@round
  \else
    \ifnum\si@tempcnta<\si@num@dp\relax
      \expandafter\expandafter\expandafter\si@num@pad
    \fi
  \fi}
%    \end{macrocode}
%\end{macro}
%\begin{macro}{\si@num@pad}
%\begin{macro}{\si@num@pd}
% Padding a number is a relatively easy matter.  The number of digits
% is increased by adding ``$0$'' recursively.
%    \begin{macrocode}
\newcommand*{\si@num@pad}{%
  \si@log@debug{Padding to \the\si@num@dp\space digits}%
  \loop\ifnum\si@tempcnta<\si@num@dp\si@num@pd\repeat}
\newcommand*{\si@num@pd}{%
  \advance\si@tempcnta\@ne\relax
  \protected@edef\si@num@postdec{\si@num@postdec0}}
%    \end{macrocode}
%\end{macro}
%\end{macro}
%\begin{macro}{\si@num@rndtarget}
%\changes{v1.1}{2008/08/25}{New counter}
% A scratch counter is created to allow some code sharing between
% fixing decimal places and significant figures.
%    \begin{macrocode}
\newcount\si@num@rndtarget
%    \end{macrocode}
%\end{macro}
%\begin{macro}{\si@num@round}
%\begin{macro}{\si@num@prernd}
%\begin{macro}{\si@num@postrnd}
%\begin{macro}{\si@num@rndstart}
%\changes{v1.1}{2008/08/25}{Separated out from \cs{si@num@round}}
% Rounding a number is more complicated.  The main macro here relies
% on several others, and is simply a set-up and hand-off system.
%    \begin{macrocode}
\newcommand*{\si@num@prernd}{}
\newcommand*{\si@num@postrnd}{}
\newcommand*{\si@num@round}{%
  \si@log@debug{Rounding to \the\si@num@dp\space digits}%
  \si@num@rndtarget\si@num@dp\relax
  \si@num@rndstart}
\newcommand*{\si@num@rndstart}{%
  \si@num@reverse{\si@num@postdec}%
  \si@num@reverse{\si@num@predec}%
  \let\si@num@prernd\si@num@predec
  \let\si@num@postrnd\si@num@postdec
  \renewcommand*{\si@num@predec}{}%
  \renewcommand*{\si@num@postdec}{}%
  \si@switchfalse
  \si@num@rnd}
%    \end{macrocode}
%\end{macro}
%\end{macro}
%\end{macro}
%\end{macro}
%\begin{macro}{\si@num@reverse}
%\begin{macro}{\si@num@rev}
%\darg{storage-macro}
% The initial stage is to reverse the entire number, to make life
% easier.  This is then undone by the other macros as the output is
% constructed.
%    \begin{macrocode}
\newcommand*{\si@num@reverse}[1]{%
  \renewcommand*{\si@tempa}{}%
  \expandafter\si@num@rev#1\@empty\@empty
  \let#1\si@tempa}
\def\si@num@rev#1#2\@empty{%
  \edef\si@tempa{#1\si@tempa}%
  \ifx\@empty#2\@empty\else
    \si@num@rev#2\@empty\@empty
  \fi}
%    \end{macrocode}
%\end{macro}
%\end{macro}
%\begin{macro}{\si@num@rnd}
%\changes{v1.1}{2008/08/25}{Altered check for ending to generalise
%  code}
%\begin{macro}{\si@num@rndpre}
%\begin{macro}{\si@num@rndpost}
%\changes{v1.0b}{2008/06/23}{Corrected bug in rounding code}
% The core looping macro of the system simply divides the flow
% between rounding before and after the decimal.  The two routines
% are quite similar, but have subtly different requirements.  Both
% take one character at a time from the input, increment if there is
% a carry digit, check its value, and add to the output string. This
% is very similar to the same routines used in other macro packages
% to achieve the same thing.
%    \begin{macrocode}
\newcommand*{\si@num@rnd}{%
  \ifx\@empty\si@num@postrnd\@empty
    \expandafter\si@num@rndpre
  \else
    \expandafter\si@num@rndpost
  \fi}
\newcommand*{\si@num@rndpre}{%
  \expandafter\edef\expandafter\si@tempa\expandafter{%
    \expandafter\@car\si@num@prernd\@nil}%
  \expandafter\edef\expandafter\si@num@prernd\expandafter{%
    \expandafter\@cdr\si@num@prernd\@nil}%
  \si@tempcntb\si@tempa\relax
  \ifsi@switch
    \advance\si@tempcntb\@ne\relax
  \fi
  \si@switchfalse
  \ifnum\si@tempcntb=10\relax
    \si@tempcntb\z@\relax
    \expandafter\expandafter\expandafter\si@switchtrue
  \fi
  \edef\si@num@predec{\the\si@tempcntb\si@num@predec}%
  \ifx\@empty\si@num@prernd\@empty
    \ifsi@switch
      \edef\si@num@predec{1\si@num@predec}%
    \fi
  \else
    \expandafter\si@num@rnd
  \fi}
\newcommand*{\si@num@rndpost}{%
  \expandafter\edef\expandafter\si@tempa\expandafter{%
    \expandafter\@car\si@num@postrnd\@nil}%
  \expandafter\edef\expandafter\si@num@postrnd\expandafter{%
    \expandafter\@cdr\si@num@postrnd\@nil}%
  \si@tempcntb\si@tempa\relax
  \ifsi@switch
    \advance\si@tempcntb\@ne\relax
  \fi
  \si@switchfalse
  \advance\si@num@rndtarget\@ne\relax
  \ifnum\si@tempcnta>\si@num@rndtarget\relax
    \advance\si@num@rndtarget\m@ne\relax
  \else
    \advance\si@num@rndtarget\m@ne\relax
    \ifnum\si@tempcnta>\si@num@rndtarget\relax
      \ifnum\si@tempcntb>4\relax
        \expandafter\expandafter\expandafter\si@switchtrue
      \fi
    \else
      \ifnum\si@tempcntb=10\relax
        \si@tempcntb\z@\relax
        \expandafter\expandafter\expandafter\si@switchtrue
      \fi
      \edef\si@num@postdec{\the\si@tempcntb\si@num@postdec}%
    \fi
  \fi
  \advance\si@tempcnta\m@ne\relax
%    \end{macrocode}
% If \cs{si@tempcnta} is negative, this must be the significant
% figures routine and one digit needs to be thrown away.
%    \begin{macrocode}
  \ifnum\si@tempcnta=\m@ne\relax
    \si@num@throw
  \fi
  \si@num@rnd}
%    \end{macrocode}
%\end{macro}
%\end{macro}
%\end{macro}
%\begin{macro}{\si@num@throw}
% Throw one digit away from the end of the post-decimal part.
%    \begin{macrocode}
\newcommand*{\si@num@throw}{%
  \si@num@reverse{\si@num@postdec}%
  \edef\si@num@postdec{\expandafter\@cdr\si@num@postdec\@nil}%
  \si@num@reverse{\si@num@postdec}}
%    \end{macrocode}
%\end{macro}
%\begin{macro}{\si@num@fixsf}
% Setting up a fixed number of significant figures is rather more
% complicated than fixing the decimal places.  There are a number of
% tests needed to see what needs to be done.
%    \begin{macrocode}
\newcommand*{\si@num@fixsf}{%
  \si@num@cntdigits{\si@num@predec}%
  \ifx\@empty\si@num@predec\@empty
    \si@tempcnta\z@\relax
  \fi
%    \end{macrocode}
% If the pre-decimal component has only a single digit, it could be
% ``0'': if so, only the significant figures in the decimal part are
% counted.
%    \begin{macrocode}
  \ifnum\si@tempcnta=\@ne\relax
    \renewcommand*{\si@tempa}{0}%
    \ifx\si@num@predec\si@tempa
%    \end{macrocode}
% The number of significant digits after the decimal point have to be
% counted up. If there are enough digits there, then some rounding
% will again be needed.
%    \begin{macrocode}
      \ifx\@empty\si@num@postdec\@empty\else
        \si@tempcnta\z@\relax
        \si@switchfalse
        \expandafter\si@num@cntpostsf\si@num@postdec
          \@empty\@empty
        \ifnum\si@tempcnta>\si@num@sf\relax
          \si@log@debug{Rounding to \the\si@num@sf\space
            figures}%
          \si@num@rndtarget\si@num@sf\relax
          \si@num@rndstart
        \fi
      \fi
    \else
      \si@num@cntprepost
    \fi
  \else
%    \end{macrocode}
% If there are too many digits in the pre-decimal part, a check is
% first made to see if there may be a need to round from the first
% post-decimal digit. If not, the post decimal part is thrown away.
%    \begin{macrocode}
    \ifnum\si@tempcnta<\si@num@sf\relax
      \si@num@cntprepost
    \else
      \advance\si@tempcnta\m@ne\relax
      \si@switchfalse
      \ifnum\si@tempcnta>\si@num@sf\relax\else
        \ifx\@empty\si@num@postdec\@empty\else
          \edef\si@num@postdec{\expandafter\@car
            \si@num@postdec\@nil}%
          \ifnum\si@num@postdec>4\relax
            \expandafter\si@switchtrue
          \fi
        \fi
      \fi
      \advance\si@tempcnta\@ne\relax
      \renewcommand*{\si@num@postdec}{}%
      \si@num@reverse{\si@num@predec}%
      \let\si@num@prernd\si@num@predec
      \renewcommand*{\si@num@predec}{}%
      \si@num@sfpre
      \ifsi@switch
        \edef\si@num@predec{1\si@num@predec}%
      \fi
    \fi
  \fi}
%    \end{macrocode}
%\end{macro}
%\begin{macro}{\si@num@cntpostsf}
% Counting up significant figures in the post-decimal part needs an
% auxiliary macro.
%    \begin{macrocode}
\def\si@num@cntpostsf#1#2\@empty{%
  \ifsi@switch
    \advance\si@tempcnta\@ne\relax
  \else
    \if 0#1\relax\else
      \si@tempcnta\@ne\relax
      \expandafter\expandafter\expandafter\si@switchtrue
    \fi
  \fi
  \ifx\@empty#2\@empty\else
    \si@num@cntpostsf#2\@empty\@empty
  \fi}
%    \end{macrocode}
%\end{macro}
%\begin{macro}{\si@num@cntprepost}
% There are two routes to find that the number of digits in both
% parts of the number are important.  In either case, rounding is
% only needed if the two parts have too many digits.
%    \begin{macrocode}
\newcommand*{\si@num@cntprepost}{%
  \si@tempcntb\si@tempcnta\relax
  \si@num@cntdigits{\si@num@postdec}%
  \advance\si@tempcnta\si@tempcntb\relax
  \ifnum\si@tempcnta>\si@num@sf
    \si@log@debug{Rounding to \the\si@num@sf\space figures}%
    \si@num@rndtarget\si@num@sf\relax
    \si@num@rndstart
    \ifsi@switch
      \si@num@throw
    \fi
  \fi}
%    \end{macrocode}
%\end{macro}
%\begin{macro}{\si@num@sfpre}
% The code here is very similar to that used to round to a fixed
% number of decimal places.
%    \begin{macrocode}
\newcommand*{\si@num@sfpre}{%
  \expandafter\edef\expandafter\si@tempa\expandafter{%
    \expandafter\@car\si@num@prernd\@nil}%
  \expandafter\edef\expandafter\si@num@prernd\expandafter{%
    \expandafter\@cdr\si@num@prernd\@nil}%
  \si@tempcntb\si@tempa\relax
  \ifsi@switch
    \advance\si@tempcntb\@ne\relax
  \fi
  \si@switchfalse
  \advance\si@num@sf\@ne\relax
  \ifnum\si@tempcnta>\si@num@sf\relax
    \advance\si@num@sf\m@ne\relax
    \edef\si@num@predec{0\si@num@predec}%
  \else
    \advance\si@num@sf\m@ne\relax
    \ifnum\si@tempcnta>\si@num@sf\relax
      \ifnum\si@tempcntb>4\relax
        \expandafter\expandafter\expandafter\si@switchtrue
      \fi
      \edef\si@num@predec{0\si@num@predec}%
    \else
      \ifnum\si@tempcntb=10\relax
        \si@tempcntb\z@\relax
        \expandafter\expandafter\expandafter\si@switchtrue
      \fi
      \edef\si@num@predec{\the\si@tempcntb\si@num@predec}%
    \fi
  \fi
  \advance\si@tempcnta\m@ne\relax
  \ifx\@empty\si@num@prernd\@empty\else
    \expandafter\si@num@sfpre
  \fi}
%    \end{macrocode}
%\end{macro}
%
%\begin{macro}{\si@num@int}
%\darg{integer-part}
% The formatting code for separating thousands is taken more-or-less
% directly from \SIstyle.  A few changes are made to fit the various
% conventions here.  Following on from the code above, \cs{si@tempa}
% is used to store the integer part of the number, and \cs{si@tempb}
% is used for the decimal part.
%    \begin{macrocode}
\newcommand*{\si@num@int}[1]{%
  \renewcommand*{\si@num@predec}{}%
  \ifsi@sepfour
    \si@num@intfmt{}#1\@empty\@empty\@empty
  \else
    \si@num@iffive{#1}
      {\si@num@intfmt{}#1\@empty\@empty\@empty}
      {\renewcommand*{\si@num@predec}{#1}}%
  \fi}
%    \end{macrocode}
%\end{macro}
%\begin{macro}{\si@num@iffive}
%\begin{macro}{\si@num@five}
%\darg{number}
% A test is needed for the presence of more than four characters.
%    \begin{macrocode}
\newcommand*{\si@num@iffive}[1]{%
  \si@num@five#1\@empty\@empty\@empty\@empty\@empty\end}
\def\si@num@five#1#2#3#4#5\end{%
   \ifx\@empty#5\@empty
      \expandafter\@secondoftwo
   \else
      \expandafter\@firstoftwo
   \fi}
%    \end{macrocode}
%\end{macro}
%\end{macro}
%\begin{macro}{\si@num@intfmt}
%\begin{macro}{\si@num@fiint}
% The business end of the integer formatter.
%    \begin{macrocode}
\newcommand*{\si@num@intfmt}[4]{%
  \ifx\@empty#2\@empty
    \si@num@intsep#1\relax
  \else
    \ifx\@empty#3\@empty
      \si@num@intsep\@empty\@empty#1#2\relax
    \else
      \ifx\@empty#4\@empty
        \si@num@intsep\@empty#1#2#3\relax
      \else
        \si@num@fiint{#1#2#3#4}%
      \fi
    \fi
  \fi}
\def\si@num@fiint#1\fi\fi\fi{\fi\fi\fi\si@num@intfmt{#1}}
%    \end{macrocode}
%\end{macro}
%\end{macro}
%\begin{macro}{\si@num@intsep}
% For adding separation to integers, an extra function is needed.
%    \begin{macrocode}
\newcommand*{\si@num@intsep}[4]{%
  \protected@edef\si@num@predec{\si@num@predec#1#2#3}%
  \if\relax#4\relax\else
    \protected@edef\si@num@predec{%
      \si@num@predec\ensuremath{\noexpand\si@digitsep}}%
    \expandafter\si@num@intsep\expandafter#4%
  \fi}
%    \end{macrocode}
%\end{macro}
%\begin{macro}{\si@num@dec}
%\begin{macro}{\si@num@decfmt}
%\darg{decimal-part}
% Formatting a decimal uses a similar mechanism, but with a few
% alterations needed.
%    \begin{macrocode}
\newcommand*{\si@num@dec}[1]{%
  \renewcommand*{\si@num@postdec}{}%
  \ifsi@sepfour
    \si@num@decfmt#1\@empty\@empty\@empty\@empty
  \else
    \si@num@iffive{#1}
      {\si@num@decfmt#1\@empty\@empty\@empty\@empty}
      {\protected@edef\si@num@postdec{\si@num@postdec#1}}%
  \fi}
\newcommand*{\si@num@decfmt}[4]{%
  \protected@edef\si@num@postdec{\si@num@postdec#1#2#3}%
  \ifx\@empty#4\@empty%
  \else
    \protected@edef\si@num@postdec{%
      \si@num@postdec\ensuremath{\noexpand\si@digitsep}}%
    \expandafter\si@num@decfmt\expandafter#4%
  \fi}
%    \end{macrocode}
%\end{macro}
%\end{macro}
%\begin{macro}{\si@num@procerr}
% Any error is recycled to to formatted correctly.  The $\pm$ sign,
% and any unit, are also added.
%    \begin{macrocode}
\newcommand*{\si@num@procerr}{%
  \si@num@addunit
  \ensuremath{\si@pm}%
  \expandafter\si@num\expandafter{\si@num@err}}
%    \end{macrocode}
%\end{macro}
%\begin{macro}{\si@num@sepxpart}
% A similar approach for numbers containing products, except the
% first token of the input has to be deleted.
%    \begin{macrocode}
\newcommand*{\si@num@sepxpart}{%
  \si@num@addunit
  \ensuremath{\times}%
  \expandafter\expandafter\expandafter\si@num\expandafter
    \expandafter\expandafter{%
      \expandafter\@cdr\si@num@xpart\@nil}}
%    \end{macrocode}
%\end{macro}
%\begin{macro}{\si@num@addunit}
% A short macro to add units if needed.
%    \begin{macrocode}
\newcommand*{\si@num@addunit}{%
  \si@unt@numtrue
  \ifx\@empty\si@unt@unitarg\@empty\else
    \ifsi@repeatunits
      \si@unt@printunit{\si@unt@unitarg}%
    \fi
  \fi}
%    \end{macrocode}
%\end{macro}
%
%\subsection{Formatting angles}
%\begin{macro}{\ang}
%\doarg{keyval options}
%\darg{angle, either in decimal or deg;min;sec format}
% The approach used here is similar to that in \SIstyle, but has been
% modified in a few ways.
%    \begin{macrocode}
\si@newrobustcmd*{\ang}[2][]{%
  \begingroup
    \sisetup{#1}%
    \si@fam@mode
    \si@log@debug{Processing \string\ang\space input `#2'}%
    \@makeother{\;}%
    \makeatletter
    \scantokens{\si@ang@parse#2;;;\@nil}}
%    \end{macrocode}
%\end{macro}
%\begin{macro}{\si@ang@parse}
% With the correct catcodes in place, processing can take place strip
% out the semi-colons.  Here, the input must either contain no
% semi-colons or two semi-colons.
%    \begin{macrocode}
\def\si@ang@parse#1;#2;#3;#4\@nil{%
  \let\ifsi@ang@fixdp\ifsi@fixdp
  \si@fixdpfalse
  \si@ifmtarg{#4}
    {\si@log@debug{Angle argument contains no
       semi-colons:\MessageBreak decimal angle}%
     \si@ang@dec{#1}{}{}}
    {\si@log@debug{Angle argument contains
       semi-colons:\MessageBreak degree-minute-second angle}%
     \renewcommand*{\si@tempa}{#4}%
     \renewcommand*{\si@tempb}{;;}%
     \ifx\si@tempa\si@tempb\else
       \ifsi@strictarc
         \renewcommand*{\si@tempb}{;}%
         \ifx\si@tempa\si@tempb
           \si@log@err{Insufficient semi-colons in argument
             of \string\ang}{The argument of
             \string\ang\space  must contain either no
             semi-colons or exactly two}%
         \else
           \si@log@err{Excess semi-colons in argument of
             \string\ang}{The argument of \string\ang\space
             must contain either no semi-colons or exactly
             two}%
         \fi
       \fi
     \fi
     \si@ang@arc{#1}{#2}{#3}}}
%    \end{macrocode}
%\end{macro}
%\begin{macro}{\si@ang@dec}
%\begin{macro}{\si@ang@arc}
% Two tests are needed, in case the input format requires conversion.
%    \begin{macrocode}
\newcommand*{\si@ang@dec}{%
  \let\si@ang@fix\@gobble
  \ifsi@ang@toarc
    \expandafter\si@ang@dectoarc
  \else
    \sisetup{padangle=none}\expandafter\si@ang@typeset
  \fi}
\newcommand*{\si@ang@arc}{%
  \let\si@ang@fix\si@ang@arcfix
  \ifsi@ang@todec
    \expandafter\si@ang@arctodec
  \else
    \expandafter\si@ang@typeset
  \fi}
%    \end{macrocode}
%\end{macro}
%\end{macro}
%\begin{macro}{\ifsi@ang@fixdp}
%\begin{macro}{\si@ang@fix}
%\begin{macro}{\si@ang@arcfix}
%\darg{number}
% A check is needed so that rounding and zero filling are only
% applied to the seconds of an arc angle.
%    \begin{macrocode}
\newif\ifsi@ang@fixdp
\newcommand*{\si@ang@fix}[1]{}
\newcommand*{\si@ang@arcfix}[1]{%
  \renewcommand*{\si@tempa}{second}%
  \renewcommand*{\si@tempb}{#1}%
  \ifx\si@tempa\si@tempb
    \ifsi@ang@fixdp
      \expandafter\expandafter\expandafter\si@fixdptrue
    \else
      \expandafter\expandafter\expandafter\si@fixdpfalse
    \fi
  \else
    \expandafter\si@fixdpfalse
  \fi}
%    \end{macrocode}
%\end{macro}
%\end{macro}
%\end{macro}
%\begin{macro}{\si@ang@ifnum}
%\darg{number}
% A test is required to check that the provided data consists of
% numbers which can actually be processed by \TeX.  This is achieved
% by using \cs{si@num@ifvalid} with a fixed list of valid
% characters.
%    \begin{macrocode}
\newcommand*{\si@ang@ifnum}[1]{%
  \begingroup
    \renewcommand*{\si@numvalid}{0123456789,.+-}%
    \ifx\@empty#1\@empty
      \aftergroup\@firstoftwo
    \else
      \si@num@ifvalid{#1}
        {\aftergroup\@firstoftwo}
        {\aftergroup\@secondoftwo}%
    \fi
  \endgroup}
%    \end{macrocode}
%\end{macro}
%\begin{macro}{\si@ang@arctodec}
%\darg{degrees}
%\darg{minutes}
%\darg{seconds}
% The business end of converting arcs to decimal angles is relatively
% straight-forward, as it only needs one calculation.  This has to be
% divided up, so that disaster does not strike if there are empty
% arguments; a check is also needed for the sign of the angle,
% so that the maths makes sense.
%    \begin{macrocode}
\newcommand*{\si@ang@arctodec}[3]{%
  \let\si@ang@fix\@gobble
  \ifnum\si@num@dp>\thr@@\relax
    \si@num@dp\thr@@\relax
  \fi
  \si@fixdptrue
  \si@ang@ifnum{#1}
    {\si@ang@ifnum{#2}
      {\si@ang@ifnum{#3}
        {\si@tempdima\z@\relax
         \renewcommand*{\si@tempa}{+}%
         \ifx\@empty#1\@empty\else
           \si@tempdima #1pt\relax
         \fi
         \ifdim\si@tempdima<\z@\relax
           \renewcommand*{\si@tempa}{-}%
         \fi
         \ifx\@empty#2\@empty\else
           \si@tempdima\dimexpr\si@tempdima\si@tempa
             #2pt/60\relax
         \fi
         \ifdim\si@tempdima<\z@\relax
           \renewcommand*{\si@tempa}{-}%
         \else
         \fi
         \ifx\@empty#3\@empty\else
           \si@tempdima\dimexpr\si@tempdima\si@tempa
             #3pt/3600\relax
         \fi
         \sisetup{numdecimal=.}%
         \expandafter\si@ang@typeset\expandafter{%
           \strip@pt\si@tempdima}{}{}}
        {\si@ang@notnum{#1}{#2}{#3}}}
      {\si@ang@notnum{#1}{#2}{#3}}}
    {\si@ang@notnum{#1}{#2}{#3}}}
%    \end{macrocode}
%\end{macro}
%\begin{macro}{\si@ang@dectoarc}
%\begin{macro}{\si@ang@arcdeg}
%\begin{macro}{\si@ang@arcmin}
%\begin{macro}{\si@ang@arcsec}
%\darg{number}
% The conversion macros for decimal to arc angles.  Life is more
% complex here than above, even without the need for checks on the
% input.  A number of separation steps are needed, each of which
% needs a separate macro.
%    \begin{macrocode}
\newcommand*{\si@ang@dectoarc}[1]{%
  \let\si@ang@fix\si@ang@arcfix
  \si@ang@fixdptrue
  \ifnum\si@num@dp>\@ne\relax
    \si@num@dp\@ne\relax
  \fi
  \si@ang@ifnum{#1}
    {\si@tempdima\z@\relax
     \ifx\@empty#1\@empty\else
       \si@tempdima #1pt\relax
     \fi
     \si@ang@sepint{deg}%
     \si@tempdima\dimexpr\si@tempdima *60\relax
     \si@ang@sepint{min}%
     \edef\si@tempa{\the\dimexpr\si@tempdima *60\relax}%
     \expandafter\newcommand\expandafter*\expandafter{%
       \expandafter\si@ang@arcsec\expandafter}\expandafter{%
       \expandafter\si@ang@strippt\si@tempa}%
%    \end{macrocode}
% A check is made for ``0.0'' seconds, which should be converted to
% simply ``0''.
%    \begin{macrocode}
     \si@tempdima\z@\relax
     \edef\si@tempa{\the\si@tempdima}%
     \expandafter\renewcommand\expandafter*\expandafter{%
       \expandafter\si@tempa\expandafter}\expandafter{%
       \expandafter\si@ang@strippt\si@tempa}%
     \ifx\si@tempa\si@ang@arcsec
       \renewcommand*{\si@ang@arcsec}{0}%
     \fi
%    \end{macrocode}
% To avoid adding zeros where they are not required, each part of the
% angle is now checked.
%    \begin{macrocode}
     \renewcommand*{\si@tempa}{0}%
     \ifx\si@ang@arcdeg\si@tempa
       \si@temptoks{{}}%
     \else
       \si@temptoks{{\si@ang@arcdeg}}%
     \fi
     \ifx\si@ang@arcmin\si@tempa
       \si@temptoks\expandafter{\the\si@temptoks{}}%
     \else
       \si@temptoks\expandafter{\the\si@temptoks{%
         \si@ang@arcmin}}%
     \fi
     \ifx\si@ang@arcsec\si@tempa
       \si@temptoks\expandafter{\the\si@temptoks{}}%
     \else
       \si@temptoks\expandafter{\the\si@temptoks{%
         \si@ang@arcsec}}%
     \fi
     \expandafter\si@ang@typeset\the\si@temptoks}
    {\si@ang@notnum{#1}{}{}}}
%    \end{macrocode}
%\end{macro}
%\end{macro}
%\end{macro}
%\end{macro}
%\begin{macro}{\si@ang@sepint}
%\begin{macro}{\si@ang@sint}
%\begin{macro}{\si@ang@strippt}
%\darg{either \texttt{deg} or \texttt{min}}
% Support macros for the conversion from decimal to arc angles.
%    \begin{macrocode}
\newcommand*{\si@ang@sepint}[1]{%
  \expandafter\si@ang@sint\the\si@tempdima\@empty
  \expandafter\let\csname si@ang@arc#1\endcsname\si@tempa}
\def\si@ang@sint#1.#2\@empty{%
  \renewcommand*{\si@tempa}{#1}%
  \si@tempdima 0.#2\relax}
\begingroup
  \catcode`P=12
  \catcode`T=12
  \lowercase{
    \renewcommand*{\si@tempa}{%
      \def\si@ang@strippt##1PT{##1}}}
  \expandafter\endgroup
\si@tempa
%    \end{macrocode}
%\end{macro}
%\end{macro}
%\end{macro}
%\begin{macro}{\si@ang@notnum}
%\darg{degrees}
%\darg{minutes}
%\darg{seconds}
% Not a \TeX\ number: complain.
%    \begin{macrocode}
\newcommand*{\si@ang@notnum}[3]{%
  \si@log@warn{Angle `#1;#2;#3' is not a pure
    number:\MessageBreak output will be as given}%
  \si@ang@typeset{#1}{#2}{#3}}
%    \end{macrocode}
%\end{macro}
%\begin{macro}{\ifsi@ang@sign}
% A flag is needed to leave signs along for angles, where
% a zero value may still need a sign.
%    \begin{macrocode}
\newif\ifsi@ang@sign
%    \end{macrocode}
%\end{macro}
%\begin{macro}{\si@ang@typeset}
%\darg{degrees}
%\darg{minutes}
%\darg{seconds}
% The \cs{si@ang@set} macro does the work of assigning the degrees,
% minutes and seconds, and actually typesetting the result.
%    \begin{macrocode}
\newcommand*{\si@ang@typeset}[3]{%
%    \end{macrocode}
%\begin{macro}{\si@ang@degs}
%\begin{macro}{\si@ang@mins}
%\begin{macro}{\si@ang@secs}
% First, the three macros that will contain the measures must exist.
%    \begin{macrocode}
  \ifsi@ang@padlarge
    \newcommand*{\si@ang@degs}{0\si@sym@degree}%
    \newcommand*{\si@ang@mins}{0\si@sym@minute}%
    \newcommand*{\si@ang@secs}{0\si@sym@second}%
  \else
    \newcommand*{\si@ang@degs}{}%
    \newcommand*{\si@ang@mins}{}%
    \newcommand*{\si@ang@secs}{}%
  \fi
%    \end{macrocode}
%\begin{macro}{\si@ang@decimalsymbol}
% The current definition of \cs{si@decimalsymbol} needs to be saved.
%    \begin{macrocode}
  \protected@edef\si@ang@decimalsymbol{\si@decimalsymbol}%
%    \end{macrocode}
%\end{macro}
%\begin{macro}{\si@ang@movesign}
% Either the signs need to be moved, or this needs to be killed off.
%    \begin{macrocode}
  \ifsi@astroang
    \let\si@ang@movesign\si@ang@astrosign
  \else
    \let\si@ang@movesign\@gobble
  \fi
%    \end{macrocode}
%\end{macro}
%\begin{macro}{\si@ang@secnum}
%\begin{macro}{\si@ang@minnum}
% The arguments are now examined in reverse order.  If they are
% empty, then nothing is done.  Otherwise, the larger measures are
% zero-filled, if this has been requested. Some steps are needed
% to allow for addition of signs to numbers.
%    \begin{macrocode}
  \newcommand*{\si@ang@secnum}{\si@ang@num{second}}%
  \newcommand*{\si@ang@minnum}{\si@ang@num{minute}}%
   \si@ifnotmtarg{#3}
     {\si@log@debug{Found seconds `#3'}%
      \si@ang@ifnum{#3}
        {\ifdim #3 pt=\z@\relax\else
           \si@ang@signtrue
         \fi}{}%
      \renewcommand*{\si@ang@secs}
       {\si@ang@secnum{#3}\si@sym@second}%
      \renewcommand*{\si@ang@mins}
        {\si@ang@pad{0\si@sym@minute}}%
      \renewcommand*{\si@ang@degs}
        {\si@ang@pad{0\si@sym@degree}}}%
   \si@ifnotmtarg{#2}
     {\si@log@debug{Found minutes `#2'}%
      \si@ang@ifnum{#2}
        {\ifdim #2 pt=\z@\relax\else
           \si@ang@signtrue
         \fi}{}%
      \renewcommand*{\si@ang@secnum}{%
        \si@ang@signlessnum{second}}%
      \renewcommand*{\si@ang@mins}
       {\si@ang@minnum{#2}\si@sym@minute}%
      \renewcommand*{\si@ang@degs}
        {\si@ang@pad{0\si@sym@degree}}}%
   \si@ifnotmtarg{#1}
     {\si@log@debug{Found degrees `#1'}%
      \renewcommand*{\si@ang@secnum}{%
        \si@ang@signlessnum{second}}%
      \renewcommand*{\si@ang@minnum}{%
        \si@ang@signlessnum{minute}}%
      \renewcommand*{\si@ang@degs}
%    \end{macrocode}
% The group here is needed to get the mechanism to move the symbol to
% work properly.
%    \begin{macrocode}
       {\si@ang@num{degree}{#1}%
        \si@sym@degree}}%
   \si@out@num
     {\si@ang@degs\si@anglesep\si@ang@mins\si@anglesep
        \si@ang@secs}%
%    \end{macrocode}
%\end{macro}
%\end{macro}
% The group opened by \cs{ang} is closed.
%    \begin{macrocode}
   \endgroup}
%    \end{macrocode}
%\end{macro}
%\end{macro}
%\end{macro}
%\end{macro}
%\begin{macro}{\si@ang@pad}
%\darg{number}
% Padding is only added if requested; the zero is a literal.
%    \begin{macrocode}
\newcommand*{\si@ang@pad}[1]{\ifsi@ang@padsmall #1\fi}
%    \end{macrocode}
%\end{macro}
%\begin{macro}{\si@ang@num}
%\begin{macro}{\si@ang@signlessnum}
%\darg{one of \texttt{degree}, \texttt{minute} or \texttt{second}}
%\darg{number}
% Modified versions of \cs{num}, one to typeset angles without a
% leading sign and the other with.
%    \begin{macrocode}
\newcommand*{\si@ang@num}[2]{%
  \begingroup
    \si@ang@fix{#1}%
    \si@ang@movesign{#1}%
    \si@num{#2}%
  \endgroup}
\newcommand*{\si@ang@signlessnum}[2]{%
  \begingroup
    \si@ang@fix{#1}%
    \si@ang@movesign{#1}%
    \sisetup{addsign=none}%
    \si@num{#2}%
  \endgroup}
%    \end{macrocode}
%\end{macro}
%\end{macro}
%\begin{macro}{\si@ang@killdegree}
%\begin{macro}{\si@ang@killminute}
%\begin{macro}{\si@ang@killsecond}
% A mechanism is needed to handle moving the angle unit signs for the
% \opt{astroang} option.  First, some support macros are needed.
%    \begin{macrocode}
\newcommand*{\si@ang@killdegree}{\let\si@sym@degree\relax}
\newcommand*{\si@ang@killminute}{\let\si@sym@minute\relax}
\newcommand*{\si@ang@killsecond}{\let\si@sym@second\relax}
%    \end{macrocode}
%\end{macro}
%\end{macro}
%\end{macro}
%\begin{macro}{\si@ang@astrosign}
%\darg{one of \texttt{degree}, \texttt{minute} or \texttt{second}}
% The method needs two steps, producing the sign over the decimal
% sign and preventing duplicate symbols appearing. This is based on a
% suggestion from Morten H{\o}gholm, but using \TeX\ internals as
% \cs{makebox} does not work here.  Note the need to correct for
% \cs{scriptspace} (thanks to Donald Arseneau for that).
%    \begin{macrocode}
\newcommand*{\si@ang@astrosign}[1]{%
  \renewcommand*{\si@decimalsymbol}{%
    \setbox\si@tempboxa=\hbox{%
      \ensuremath{{\si@ang@decimalsymbol}}}%
    \si@tempdima\wd\si@tempboxa\relax
    \setbox\si@tempboxb=\hbox to\z@{%
      \hss\unhbox\si@tempboxa\hss}%
    \setbox\si@tempboxa=\hbox{%
      \csname si@sym@#1\endcsname\hskip-\scriptspace}%
    \si@tempdimb\wd\si@tempboxa\relax
    \setbox\si@tempboxc=\hbox to\z@{%
      \hss\unhbox\si@tempboxa\hss}%
    \setbox\si@tempboxd=\hbox{%
      \usebox\si@tempboxb\usebox\si@tempboxc}%
    \ifdim\si@tempdima>\si@tempdimb\relax
      \setbox\si@tempboxa=\hbox to\si@tempdima{%
        \hss\unhbox\si@tempboxd\hss}%
    \else
      \setbox\si@tempboxa=\hbox to\si@tempdimb{%
        \hss\unhbox\si@tempboxd\hss}%
    \fi
    \usebox\si@tempboxa%
    \ifdim\si@tempdima>\si@tempdimb\relax\else
      \hskip\scriptspace
    \fi}%
  \renewcommand*{\si@num@decimalhook}{\expandafter\aftergroup
    \csname si@ang@kill#1\endcsname}}%
%    \end{macrocode}
%\end{macro}
%
%\subsection{Tabular material}
% The automatic formatting and alignment of numerical data in columns
% is handled here.  The various other packages that work in this area
% are basically ripped-off here.  The letters |D|, |N| and |R| are
% already taken by the other packages for numerical alignment, and so
% |S| ($=$ \currpkg) is chosen for the alignment of numerical
% material.  The package also provides a second column type, |s|, for
% units (the letter is taken from \cs{si}).
%
%\begin{macro}{\NC@list}
%\changes{v1.1e}{2008/11/18}{Fixed issue with \pkg{cellspace} package}
% The first part of the job is to create the basic apparatus for the
% columns using the \pkg{array} package.  To prevent any issues with
% the content of optional arguments to the new columns, so
% rearrangement is needed.  The \currpkg columns have to come
% \emph{before} any other column definitions.  This is achieved by
% saving \cs{NC@list}, creating the columns then restoring the list
% with the appropriate extra parts.
%    \begin{macrocode}
\@ifpackageloaded{cellspace}
  {\newcolumntype{C}[1]{>{\bcolumn #1\@nil}#1<{\ecolumn}}}
  {\AtBeginDocument{
    \@ifpackageloaded{cellspace}
     {\newcolumntype{C}[1]{>{\bcolumn #1\@nil}#1<{\ecolumn}}%
      \renewcommand*{\NC@rewrite@S}[1][]{%
        \edef\si@tempa{\the\@temptokena
        >{\noexpand\si@tab@begin@S[#1]}c%
        <{\noexpand\si@tab@end@S}}%
        \@temptokena\expandafter{\si@tempa}%
        \NC@find}}
     {}}}
\edef\si@tempa{%
  \noexpand\NC@do S\noexpand\NC@do s\the\NC@list}
\newcolumntype{S}{}
\newcolumntype{s}{}
\NC@list\expandafter{\si@tempa}
%    \end{macrocode}
%\end{macro}
%\changes{v1.1h}{2008/11/30}{Fixed problem with \pkg{colortbl} package}
%\changes{v1.1i}{2008/11/30}{Better solution to \pkg{colortbl} issue}
%\begin{macro}{\si@tab@corr@colortbl}
% A correction if \pkg{colortbl} is loaded.
%    \begin{macrocode}
\newcommand*{\si@tab@corr@colortbl}{\relax}
\AtBeginDocument{
  \@ifpackageloaded{colortbl}{%
    \renewcommand*{\si@tab@corr@colortbl}{%
      \hskip\stretch{-0.5}\kern\z@
    }
  }{}
}
%    \end{macrocode}
%\end{macro}
%\begin{macro}{\NC@rewrite@S}
%\begin{macro}{\NC@rewrite@s}
%\doarg{keyval options}
% Following the \numprint approach, the \cs{NC@rewrite@\ldots} macros
% are rewritten to collect the cell contents.  This means messing
% with the internal macros of another package, but there is no other
% way to do this.  As \pkg{array} is a standard package from the
% \pkg{tools} bundle, this should be reasonably safe. Here the begin
% and end code needed is added to the existing list if
% \cs{@temptokena}, with the start and end macros unexpanded.
% Argument |#1| contains any user setup options for this column.
% Passing an argument at this stage will cause issues, so each column
% type needs its own \texttt{begin} and \texttt{end} macros.
%    \begin{macrocode}
\renewcommand*{\NC@rewrite@S}[1][]{%
  \edef\si@tempa{\the\@temptokena
    >{\noexpand\si@tab@begin@S[#1]}c%
    <{\noexpand\si@tab@end@S}}%
  \@temptokena\expandafter{\si@tempa}%
  \NC@find}
\renewcommand*{\NC@rewrite@s}[1][]{%
  \edef\si@tempa{\the\@temptokena
    >{\noexpand\si@tab@begin@s[#1]}c%
    <{\noexpand\si@tab@end@s}}%
  \@temptokena\expandafter{\si@tempa}%
  \NC@find}
%    \end{macrocode}
%\end{macro}
%\end{macro}
%\begin{macro}{\si@tab@begin@S}
%\begin{macro}{\si@tab@begin@s}
%\begin{macro}{\si@tab@gettok}
%\doarg{keyval options}
% At this stage, the appropriate token gathering macro is activated,
% and the common starting macro is called.  For the |S| column, the
% \opt{seperr} is turned off, and an error is set to be raised by any
% ``x-part'' input.
%    \begin{macrocode}
\newcommand*{\si@tab@begin@S}[1][]{%
  \si@log@debug{Processing S column cell contents}%
  \let\si@tab@gettok\si@tab@gettok@S
  \si@seperrfalse
  \renewcommand*{\si@num@sepxpart}{%
    \si@log@err{Multiple numbers not allowed in
       tables\MessageBreak Only the first number used}
       \@ehb}%
  \si@tab@begin[#1]}
\newcommand*{\si@tab@begin@s}[1][]{%
  \si@log@debug{Processing s column cell contents}%
  \let\si@tab@gettok\si@tab@gettok@s
  \si@tab@begin[#1]}
%    \end{macrocode}
%\end{macro}
%\end{macro}
%\end{macro}
%\begin{macro}{\si@tab@toks}
%\begin{macro}{\si@tab@pretoks}
%\begin{macro}{\si@tab@posttoks}
% Some storage is needed for the data to build up.  In common with
% \pkg{rccol} and \numprint, token registers are used for this (thus
% leaving problematic input to be handled later).
%    \begin{macrocode}
\newtoks\si@tab@toks
\newtoks\si@tab@pretoks
\newtoks\si@tab@posttoks
%    \end{macrocode}
%\end{macro}
%\end{macro}
%\end{macro}
%\begin{macro}{\si@tab@begin}
%\doarg{keyval options}
% The macro for gathering up input is common to both column types. It
% uses the method for \pkg{rccol}; the cell contents are collected by
% a second macro, which then stores all of the data in an appropriate
% token store.
%    \begin{macrocode}
\newcommand*{\si@tab@begin}[1][]{%
  \begingroup
    \sisetup{#1}%
    \si@tab@toks{}%
    \si@tab@pretoks{}%
    \si@tab@posttoks{}%
    \si@switchfalse
    \si@tab@gettok}
%    \end{macrocode}
%\end{macro}
%\begin{macro}{\si@tab@next}
% A macro is needed for recursion when collecting cell contents.
%    \begin{macrocode}
\newcommand*{\si@tab@next}{}
%    \end{macrocode}
%\end{macro}
%\begin{macro}{\si@tab@gettok@S}
%\changes{v1.0j}{2008/08/21}{Token gathering code now checks for
%  expandable content}
%\darg{token}
% The cell-content collection system is inherited from \pkg{rccol}.
% First, the current input is compared with a fixed list of macros
% which need custom handling.
%    \begin{macrocode}
\newcommand*{\si@tab@gettok@S}[1]{%
  \ifx\tabularnewline#1\relax
    \let\si@tab@next\si@tab@newline@S
  \else
    \ifx\end#1\relax
      \let\si@tab@next\end
    \else
      \ifx\si@tab@end@S#1\relax
        \let\si@tab@next\si@tab@end@S
      \else
        \ifx\endtabular#1\relax
          \let\si@tab@next\endtabular
        \else
          \ifx\csname#1\relax
            \let\si@tab@next\csname
          \else
            \ifx\relax#1\relax
              \let\si@tab@next\relax
            \else
%    \end{macrocode}
% If the code gets to here, then recursion should occur.
%    \begin{macrocode}
              \let\si@tab@next\si@tab@gettok@S
%    \end{macrocode}
% Cells always contain \cs{ignorespaces} and \cs{unskip}. These can
% both be filtered out here and are saved to either the pre-number or
% post-number token store, as appropriate. This is done without
% fixing the destination in case the user adds addition copies to the
% input.
%    \begin{macrocode}
              \ifx\ignorespaces#1\relax
                \si@tab@othertok{#1}%
              \else
                \ifx\unskip#1\relax
                  \si@tab@othertok{#1}%
                \else
%    \end{macrocode}
% If the current token is a valid numerical character, then the flag
% is set and the input is saved into the ``processing'' token
% register.
%    \begin{macrocode}
                  \si@str@ifchrstr{#1}{\si@numvalid}
                     {\si@switchtrue
                      \si@log@debug{Found numerical cell
                        contents `#1'}%
                      \si@tab@toks=\expandafter{%
                        \the\si@tab@toks#1}}
%    \end{macrocode}
% For non-numerical content, some testing is needed. A check is made
% to see if the current ``token'' is a single item. If it was wrapped
% in one set of braces before being picked up here, they will have
% been removed. So the input may now have more than one character to
% it. If it is a single token, testing for expendability first
% requires a check for another set of braces.  This is handed off
% using \cs{futurelet}.
%    \begin{macrocode}
                     {\si@tab@ifonechar{#1}
                       {\futurelet\si@tempa\si@tab@bracetest#1}%
                       {\si@tab@othertok{#1}}}%
                \fi
              \fi
            \fi
          \fi
        \fi
      \fi
    \fi
  \fi
  \si@tab@next}
%    \end{macrocode}
%\end{macro}
%\begin{macro}{\si@tab@othertok}
%\darg{token}
% Tokens which are not part of a number can occur before or after a
% number. This is all sorted out here, with some debugging
% information for good measure.
%    \begin{macrocode}
\newcommand*{\si@tab@othertok}[1]{%
  \si@log@debug{Found other cell contents `\unexpanded{#1}'}%
  \ifsi@switch
    \si@log@debug{Adding to post-numeral part}%
    \si@tab@posttoks=\expandafter{\the\si@tab@posttoks#1}%
  \else
    \si@log@debug{Adding to pre-numeral part}%
    \si@tab@pretoks=\expandafter{\the\si@tab@pretoks#1}%
  \fi}
%    \end{macrocode}
%\end{macro}
%\begin{macro}{\si@tab@ifonechar}
%\darg{token}
%\begin{macro}{\si@tab@onechar}
% A quick check for a single character.
%    \begin{macrocode}
\newcommand*{\si@tab@ifonechar}[1]{\si@tab@onechar#1\@empty}
\def\si@tab@onechar#1#2\@empty{%
  \ifx\@empty#2\@empty
    \expandafter\@firstoftwo
  \else
    \expandafter\@secondoftwo
  \fi}
%    \end{macrocode}
%\end{macro}
%\end{macro}
%\begin{macro}{\si@tab@bracetest}
% For the \cs{futurelet} test, a separate macro is needed. If the
% current token starts with a brace, then it is added to the
% ``other'' stack.  If not, a test can be made on its definition.
%    \begin{macrocode}
\newcommand*{\si@tab@bracetest}{%
  \ifx\si@tempa\bgroup
    \expandafter\si@tab@othertok
  \else
    \expandafter\si@tab@exptest
  \fi}
%    \end{macrocode}
%\end{macro}
%\begin{macro}{\si@tab@exptest}
%\begin{macro}{\si@tab@meaning}
%\begin{macro}{\si@tab@strip}
%\darg{token}
% The meaning of the token is compared to that for a short macro. If
% the two are of the same type, expansion is attempted. In this case,
% the current pre-number tokens are also typeset.
%    \begin{macrocode}
\newcommand*{\si@tab@exptest}[1]{%
  \edef\si@tempa{\si@tab@meaning{#1}}%
  \ifx\si@tempa\si@tab@macro
    \si@log@debug{Expanding cell contents `\unexpanded{#1}'}%
    \the\si@tab@pretoks
    \si@tab@pretoks{}%
    \protected@edef\si@tab@next{\noexpand\si@tab@gettok@S#1}%
  \else
    \si@tab@othertok{#1}%
  \fi}
\newcommand*{\si@tab@meaning}[1]{%
  \expandafter\si@tab@strip\meaning#1:->:->\@empty}
\def\si@tab@strip#1:->#2:->#3\@empty{#1}
%    \end{macrocode}
%\end{macro}
%\end{macro}
%\end{macro}
%\begin{macro}{\si@tab@macro}
% A supporter macro for the above which always expands to ``macro''
% with everything with category code 12.
%    \begin{macrocode}
\newcommand*{\si@tab@macro}{}
\edef\si@tab@macro{\si@tab@meaning{\si@tab@macro}}
%    \end{macrocode}
%\end{macro}
%\begin{macro}{\si@tab@gettok@s}
%\changes{v1.0a}{2008/06/20}{Fixed issues with alignment of
%  \texttt{s} column contents}
%\darg{token}
% The situation for the |s| column is less complex, as there is only
% one store and expansion is not an issue.
%    \begin{macrocode}
\newcommand*{\si@tab@gettok@s}[1]{%
  \ifx\tabularnewline#1\relax
    \let\si@tab@next\si@tab@newline@s
  \else
    \ifx\end#1\relax
      \let\si@tab@next\end
    \else
      \ifx\si@tab@end@s#1\relax
        \let\si@tab@next\si@tab@end@s
      \else
        \ifx\endtabular#1\relax
          \let\si@tab@next\endtabular
        \else
          \ifx\csname#1\relax
            \let\si@tab@next\csname
          \else
            \ifx\relax#1\relax
              \let\si@tab@next\relax
            \else
              \let\si@tab@next\si@tab@gettok@s
              \ifx\ignorespaces#1\relax\else
                \ifx\unskip#1\relax\else
                  \si@tab@toks=\expandafter{%
                    \the\si@tab@toks#1}%
                  \si@log@debug{Found cell contents `#1'}%
                \fi
              \fi
            \fi
          \fi
        \fi
      \fi
    \fi
  \fi
  \si@tab@next}
%    \end{macrocode}
%\end{macro}
%\begin{macro}{\si@tab@end@S}
%\changes{v1.0a}{2008/06/20}{Fixed mixed number/text column
%  alignment}
%\changes{v1.0k}{2008/09/02}{Fixed problem with coloured cells}
%\begin{macro}{\si@tab@rfill}
%\begin{macro}{\si@tab@lfill}
%\begin{macro}{\si@tab@end@s}
% The end macros are similar, but with some minor differences. In
% both cases, the appropriate filling is carried out.  For the |S|
% column, this depends on the cell contents, whereas for the |s|
% column the fill is always the same.  Output of a number in an |S|
% column is only attempted if one was found, otherwise the cell
% contents will all be in \cs{si@tab@pretoks}.
%    \begin{macrocode}
\newcommand*{\si@tab@end@S}{%
    \ifsi@switch
      \let\si@tab@lfill\si@tab@lfill@S
      \let\si@tab@rfill\si@tab@rfill@S
    \else
      \let\si@tab@rfill\si@tab@rfill@t
      \let\si@tab@lfill\si@tab@lfill@t
    \fi
    \si@tab@corr@colortbl
    \hskip\stretch{\si@tab@lfill}\kern\z@
    \ifsi@switch
      \the\si@tab@pretoks
      \si@tab@numout
      \the\si@tab@posttoks
    \else
      \the\si@tab@pretoks
    \fi
    \hskip\stretch{\si@tab@rfill}\kern\z@
    \si@tab@corr@colortbl
  \endgroup}
\newcommand*{\si@tab@end@s}{%
    \si@tab@corr@colortbl
    \hskip\stretch{\si@tab@lfill@s}\kern\z@ 
    \ignorespaces
    \expandafter\si\expandafter{\the\si@tab@toks}%
    \unskip
    \hskip\stretch{\si@tab@rfill@s}\kern\z@ 
    \si@tab@corr@colortbl
  \endgroup}
%    \end{macrocode}
%\end{macro}
%\end{macro}
%\end{macro}
%\end{macro}
%\begin{macro}{\si@tab@newline@S}
%\begin{macro}{\si@tab@newline@s}
%\begin{macro}{\si@tab@end}
% If the column is the final one read, then some work is needed with
% the \cs{tabularnewline} macro.  Output has to happen \emph{before}
% the new line, then the ending macro is made safe before calling the
% \LaTeX\ line end.  If the user makes use of the primitive \cs{cr}
% then this problem does not arise as \cs{si@tab@end@\ldots} is
% called correctly.
%    \begin{macrocode}
\newcommand*{\si@tab@newline@S}{%
  \si@tab@end@S
  \hfil\relax
  \let\si@tab@end\si@tab@end@S
  \renewcommand*{\si@tab@end@S}{\let\si@tab@end@S\si@tab@end}%
  \tabularnewline}
\newcommand*{\si@tab@newline@s}{%
  \si@tab@end@s
  \hfil\relax
  \let\si@tab@end\si@tab@end@s
  \renewcommand*{\si@tab@end@s}{\let\si@tab@end@s\si@tab@end}%
  \tabularnewline}
%    \end{macrocode}
%\end{macro}
%\end{macro}
%\end{macro}
%
%\begin{macro}{\si@tempcnta}
%\begin{macro}{\si@tempcntb}
% The second part of the tabular code is concerned with typesetting
% numbers in |S| columns with the appropriate alignment. Counters are
% needed for the digit-counting system.
%    \begin{macrocode}
\newcount\si@tempcnta
\newcount\si@tempcntb
%    \end{macrocode}
%\end{macro}
%\end{macro}
%\begin{macro}{\si@tab@numout}
%\changes{v1.1}{2008/09/03}{Extended for non-aligned output}
% If a number is found, then some secondary processing is needed to
% format it correctly.
%    \begin{macrocode}
\newcommand*{\si@tab@numout}{%
  \si@num@intabtrue
  \ifsi@tabparseonly\else
    \ifsi@tab@fixed
      \ifsi@tabautofit
        \si@num@dp\si@tab@mantpostcnt\relax
        \expandafter\expandafter\expandafter\expandafter
          \expandafter\expandafter\expandafter\si@fixdptrue
      \fi
    \fi
  \fi
  \expandafter\si@num\expandafter{\the\si@tab@toks}%
  \si@tab@format}
%    \end{macrocode}
%\end{macro}
%\begin{macro}{\si@tab@prebox}
%\begin{macro}{\si@tab@postbox}
%\begin{macro}{\si@tab@midbox}
%\begin{macro}{\si@tab@expbox}
% The various boxes needed for the column centring are declared
% Unlike the \pkg{dcolumn} original, private boxes are used here.
% \cs{si@tempboxa} is used when a space to measure one of the
% constituents is needed; it is never used for output.
%    \begin{macrocode}
\newbox\si@tab@prebox
\newbox\si@tab@midbox
\newbox\si@tab@postbox
\newbox\si@tab@expbox
%    \end{macrocode}
%\end{macro}
%\end{macro}
%\end{macro}
%\end{macro}
%\begin{macro}{\si@tab@format}
%\changes{v1.1}{2008/09/03}{Extended for non-aligned output}
% The formatting set up is taken from \pkg{dcolumn}, but with
% control of the output form the stored information.  The choice
% of a variable (decimal-centred) column or fixed width boxes is
% made.
%    \begin{macrocode}
\newcommand*{\si@tab@format}{%
  \ifsi@tabparseonly
    \expandafter\si@tab@outonly
  \else
    \ifsi@tab@fixed
      \expandafter\expandafter\expandafter\si@tab@fixed
    \else
      \expandafter\expandafter\expandafter\si@tab@unfixed
    \fi
  \fi
%    \end{macrocode}
% A hack to get the correct colour everywhere without too much work.
%    \begin{macrocode}
  \ifsi@colourvalues
    \si@fam@colourcmd{\si@valuecolour}%
  \fi
  \box\si@tab@prebox\box\si@tab@midbox\box\si@tab@postbox
    \box\si@tab@expbox}
%    \end{macrocode}
%\end{macro}
%\begin{macro}{\si@tab@outonly}
% The non-alignment option means that no real processing is done: only font
% control is required.
%    \begin{macrocode}
\newcommand*{\si@tab@outonly}{%
  \si@log@debug{Using non-aligned S column}%
  \setbox\si@tab@prebox=\hbox{}%
  \setbox\si@tab@midbox=\hbox{}%
  \setbox\si@tab@postbox=\hbox{}%
  \setbox\si@tab@expbox=\hbox{}%
    \protected@edef\si@tempa{%
      \si@tab@out
    \ifx\@empty\si@num@out\@empty\else
      \ensuremath{{\si@decimalsymbol}}%
    \fi
    \si@num@out
    \si@tab@expout}%
  \expandafter\si@out@num\expandafter{\si@tempa}}
%    \end{macrocode}
%\end{macro}
%\begin{macro}{\si@tab@unfixed}
%\changes{v1.0l}{2008/09/03}{Fixed issue with decimal sign and
%  integers}
% When the width of the contents is not fixed, the system creates a
% block in which the decimal marker is always at the centre. This is
% achieved by placing the pre- and post-decimal parts of the number
% in boxes.  The wider one is then used to set up the column width,
% by resizing the other one.
%    \begin{macrocode}
\newcommand*{\si@tab@unfixed}{%
  \si@log@debug{Using variable width S column}%
  \ifx\@empty\si@num@out\@empty
    \setbox\si@tab@midbox=\hbox
      {\phantom{\ensuremath{{\si@decimalsymbol}}}}%
  \else
    \setbox\si@tab@midbox=\hbox
      {\ensuremath{{\si@decimalsymbol}}}%
  \fi
  \protected@edef\si@num@out{\si@num@out\si@tab@expout}%
  \setbox\si@tab@prebox=\hbox
    {\expandafter\si@out@num\expandafter{\si@tab@out}}%
  \setbox\si@tab@postbox=\hbox
    {\expandafter\si@out@num\expandafter{\si@num@out}}%
  \ifdim\wd\si@tab@prebox>\wd\si@tab@postbox\relax
    \setbox\si@tab@postbox=\hbox to\wd\si@tab@prebox%
      {\unhbox\si@tab@postbox\hfill}%
  \else
    \setbox\si@tab@prebox=\hbox to\wd\si@tab@postbox%
      {\hfill\unhbox\si@tab@prebox}%
  \fi
  \setbox\si@tab@expbox=\hbox{}}
%    \end{macrocode}
%\end{macro}
%\begin{macro}{\si@tab@predim}
%\begin{macro}{\si@tab@postdim}
%\begin{macro}{\si@tab@expdim}
%\begin{macro}{\si@tempdima}
%\begin{macro}{\si@tempdimb}
% Some storage dimensions are declared.
%    \begin{macrocode}
\newdimen\si@tab@predim
\newdimen\si@tab@postdim
\newdimen\si@tab@expdim
\newdimen\si@tempdima
\newdimen\si@tempdimb
%    \end{macrocode}
%\end{macro}
%\end{macro}
%\end{macro}
%\end{macro}
%\end{macro}
%\begin{macro}{\si@tab@sp}
% A short macro to control superscript.
%    \begin{macrocode}
\newcommand*{\si@tab@sp}{}
%    \end{macrocode}
%\end{macro}
%\begin{macro}{\si@tab@fixed}
%\changes{v1.0l}{2008/09/03}{Fixed issue with spacing when no digits
%  present after decimal marker}
% The column is not centred on the decimal marker; the user specifies
% how many characters on each side are allowed for.  First, the width
% of a character is measured, and stored.
%    \begin{macrocode}
\newcommand*{\si@tab@fixed}{%
  \si@log@debug{Using fixed-width S column}%
  \let\si@tab@sp\relax
  \setbox\si@tab@midbox=\hbox{}%
  \setbox\si@tab@expbox=\hbox{}%
  \setbox\si@tempboxa=\hbox{\si@out@num{1}}%
  \si@tempdima\wd\si@tempboxa\relax
%    \end{macrocode}
% The width for the two output boxes is set up.
%    \begin{macrocode}
  \si@tab@predim\the\si@tab@mantprecnt\si@tempdima\relax
  \si@tab@sepcorr{mantpre}{pre}%
  \si@tab@postdim\si@tab@mantpostcnt\si@tempdima\relax
  \setbox\si@tempboxa=\hbox{\ensuremath{{\si@decimalsymbol}}}%
  \ifnum\si@tab@mantpostcnt=\z@\relax\else
    \advance\si@tab@postdim\wd\si@tempboxa\relax
  \fi
  \si@tab@sepcorr{mantpost}{post}%
%    \end{macrocode}
% If space is needed for an exponent, it needs to be allowed for in
% the exponent box dimension.  First, the digits of the two parts are
% checked for; the width of a character is altered to be superscript.
%    \begin{macrocode}
  \setbox\si@tempboxa=\hbox{\si@out@num{^{1}}}%
  \si@tempdima\wd\si@tempboxa\relax
  \ifnum\si@tab@expprecnt>\z@\relax
    \si@tab@expdim\si@tab@expprecnt\si@tempdima\relax
    \si@tab@sepcorr{exppre}{exp}%
  \fi
  \let\si@tab@sp\sp
  \ifnum\si@tab@exppostcnt>\z@\relax
    \advance\si@tab@expdim\si@tab@exppostcnt\si@tempdima\relax
    \setbox\si@tempboxa=\hbox{%
      \ensuremath{^{\si@decimalsymbol}}}%
    \advance\si@tab@expdim\wd\si@tempboxa\relax
    \si@tab@sepcorr{exppost}{exp}%
  \fi
%    \end{macrocode}
% Space is reserved for signs.
%    \begin{macrocode}
  \setbox\si@tempboxa=\hbox{\ensuremath{-}}%
  \ifsi@tab@mantsign
    \advance\si@tab@predim\wd\si@tempboxa\relax
  \fi
  \setbox\si@tempboxa=\hbox{\ensuremath{^{-}}}%
  \ifsi@tab@expsign
    \advance\si@tab@expdim\wd\si@tempboxa\relax
  \fi
%    \end{macrocode}
% Now, if there is space to be saved for an exponent under any
% circumstances, the space for the ``$\times 10$'' part is needed.
% This is done by adding both counters together, then using this
% result for the logic.
%    \begin{macrocode}
  \si@tempcnta\si@tab@expprecnt\relax
  \advance\si@tempcnta\si@tab@exppostcnt\relax
  \ifnum\si@tempcnta>\z@\relax
    \setbox\si@tempboxa=\hbox{\ensuremath{%
      {}\si@expproduct{}\si@expbase}}%
    \advance\si@tab@expdim\wd\si@tempboxa\relax
  \fi
%    \end{macrocode}
% Finally for the box dimensions, if the exponent is not aligned,
% the space reserved for it is added to the post box.
%    \begin{macrocode}
  \ifsi@tabalignexp\else
    \advance\si@tab@postdim\si@tab@expdim\relax
  \fi
%    \end{macrocode}
% With the boxes set up, the contents can be sorted out.  A bit of
% shuffling may be needed, depending on the treatment of exponents.
%    \begin{macrocode}
  \setbox\si@tab@prebox=\hbox to\si@tab@predim{\hss\hfill
    \expandafter\si@out@num\expandafter{\si@tab@out}}%
  \ifx\@empty\si@num@out\@empty
    \setbox\si@tab@postbox=\hbox to\si@tab@postdim{\hfil}%
  \else
    \ifsi@tabalignexp\else
      \protected@edef\si@num@out{\si@num@out\si@tab@expout}%
    \fi
    \setbox\si@tab@postbox=\hbox to\si@tab@postdim
      {\ensuremath{{\si@decimalsymbol}}\expandafter\si@out@num
        \expandafter{\si@num@out}\hfil}%
  \fi
  \ifx\@empty\si@tab@expout\@empty
    \ifsi@tabalignexp
      \setbox\si@tab@expbox=\hbox to\si@tab@expdim{\hfil}%
    \fi
  \else
    \ifsi@tabalignexp
      \setbox\si@tab@expbox=\hbox to\si@tab@expdim
        {\expandafter\si@out@num\expandafter{\si@tab@expout}%
           \hfil}%
    \fi
  \fi}
%    \end{macrocode}
%\end{macro}
%\begin{macro}{\si@tab@sepcorr}
%\darg{one of \texttt{mantpre}, \texttt{mantpost}, \texttt{exppre} or
%  \texttt{exppost}}
%\darg{one of \texttt{pre}, \texttt{post} or \texttt{exp}}
% A spacing correction is needed \emph{if} the number of digits to be
% allowed for will lead to the introduction of a separator.  A counter
% and dimension are needed for the testing.
%    \begin{macrocode}
\newcommand*{\si@tab@sepcorr}[2]{%
  \expandafter\si@tempcnta\expandafter\the
    \csname si@tab@#1cnt\endcsname\relax
%    \end{macrocode}
% Calculate how many groups of three there are, then allow for not
% separating four characters if \cs{ifsi@sepfour} is false.
%    \begin{macrocode}
  \divide\si@tempcnta\thr@@\relax
  \ifsi@sepfour\else
    \expandafter\ifnum\expandafter\the
      \csname si@tab@#1cnt\endcsname=4\relax
      \si@tempcnta\z@\relax
    \fi
  \fi
%    \end{macrocode}
% The width of the separators is measured, and the correct number of
% separator widths are added to the box dimension.
%    \begin{macrocode}
  \setbox\si@tempboxa=\hbox{%
    \ensuremath{\si@tab@sp{\si@digitsep}}}%
  \expandafter\advance\csname si@tab@#2dim\endcsname
    \si@tempcnta\wd\si@tempboxa}
%    \end{macrocode}
%\end{macro}
%
%\subsection{Units}
%\begin{macro}{\SI}
%\doarg{keyval options}
%\darg{number}
% There are two types of user macros for the units system; those for
% defining new units, prefixes and powers, and those for using them.
% There are two macros for using units, \cs{SI} and \cs{si}, which
% work in very similar ways. argument to \cs{SI}
%    \begin{macrocode}
\si@newrobustcmd*{\SI}[2][]{%
  \@ifnextchar[%]
    {\si@SI[#1]{#2}}
    {\si@SI[#1]{#2}[]}}
%    \end{macrocode}
%\end{macro}
%\begin{macro}{\si}
%\doarg{keyval options}
%\darg{unit}
% \cs{si} is just an alias for \cs{SI} with no number; everything is
% handed off into an internal macro.  The internal macro also handles
% the optional prefix
%    \begin{macrocode}
\si@newrobustcmd*{\si}[2][]{\si@SI[#1]{}[]{#2}}
%    \end{macrocode}
%\end{macro}

%\begin{macro}{\newunit}
%\begin{macro}{\renewunit}
%\begin{macro}{\provideunit}
%\doarg{keyval options}
%\darg{unit}
%\darg{symbol}
% The \cs{newunit} and \cs{renewunit} macros create the new unit
% macros. To allow a mechanism for checking an existing definition,
% these macros simply carry out the appropriate tests, before handing
% off to the internal macro.  \cs{@ifdefinable} is not used here as a
% customised error is desirable. Other than that, the code here gives
% very similar results to \cs{newcommand} and \cs{renewcommand}.
% Finally, \cs{provideunit} adds the unit definition only if it does
% not already exist.
%    \begin{macrocode}
\newcommand*{\newunit}[3][]{%
  \si@ifdefinable{#2}
    {\si@unt@defunit[#1]{#2}{#3}}
    {\si@log@err{Unit \string#2 already defined!}\@eha}}
\newcommand*{\renewunit}[3][]{%
  \si@ifdefinable{#2}
    {\si@log@err{Unit \string#2 undefined}\@ehc
     \si@unt@defunit[#1]{#2}{#3}}
    {\si@log@inf{Redefining unit \string#2}%
     \si@unt@defunit[#1]{#2}{#3}}}
\newcommand*{\provideunit}[3][]{%
  \si@ifdefinable{#2}
    {\si@unt@defunit[#1]{#2}{#3}}
    {}}
%    \end{macrocode}
%\end{macro}
%\end{macro}
%\end{macro}
%\begin{macro}{\newprefix}
%\begin{macro}{\renewprefix}
%\begin{macro}{\provideprefix}
%\doarg{\texttt{binary}}
%\darg{multiple}
%\darg{powers-ten}
%\darg{symbol}
% The multiples of units are defined here; very similar code is used
% to the \cs{newunit}, \etc, macros.  The multiple prefixes cannot
% take an optional argument, and must represent some power.  Hence
% the arguments required are different.
%    \begin{macrocode}
\newcommand*{\newprefix}[4][]{%
  \si@ifdefinable{#2}
    {\si@unt@defprefix[#1]{#2}{#3}{#4}}
    {\si@log@err{Prefix \string#2 already defined!}\@eha}}
\newcommand*{\renewprefix}[4][]{%
  \si@ifdefinable{#2}
    {\si@log@err{Prefix \string#2 undefined}\@ehc
     \si@unt@defprefix[#1]{#2}{#3}{#4}}
    {\si@log@inf{Redefining prefix \string#2}%
     \si@unt@defprefix[#1]{#2}{#3}{#4}}}
\newcommand*{\provideprefix}[4][]{%
  \si@ifdefinable{#2}
    {\si@unt@defprefix[#1]{#2}{#3}{#4}}
    {}}
%    \end{macrocode}
%\end{macro}
%\end{macro}
%\end{macro}
%\begin{macro}{\newpower}
%\begin{macro}{\renewpower}
%\begin{macro}{\providepower}
%\doarg{\texttt{post}}
%\darg{number}
%\darg{power}
% Here power multiples for units are set up.  As with units and
% multiples, a layered approach is used to keep things easy to
% maintain.
%    \begin{macrocode}
\newcommand*{\newpower}[3][]{%
  \si@ifdefinable{#2}
    {\si@unt@defpower[#1]{#2}{#3}}
    {\si@log@err{Power \string#2 already defined!}\@eha}}
\newcommand*{\renewpower}[3][]{%
  \si@ifdefinable{#2}
    {\si@log@err{Power \string#2 undefined}\@ehc
     \si@unt@defpower[#1]{#2}{#3}}
    {\si@log@inf{Redefining power \string#2}%
     \si@unt@defpower[#1]{#2}{#3}}}
\newcommand*{\providepower}[3][]{%
  \si@ifdefinable{#2}
    {\si@unt@defpower[#1]{#2}{#3}}
    {}}
%    \end{macrocode}
%\end{macro}
%\end{macro}
%\end{macro}
%\begin{macro}{\newqualifier}
%\changes{v1.1}{2008/09/04}{New user macro}
%\begin{macro}{\renewqualifier}
%\changes{v1.1}{2008/09/04}{New user macro}
%\begin{macro}{\providequalifier}
%\changes{v1.1}{2008/09/04}{New user macro}
%\darg{qualifier}
%\darg{symbol}
% Creating unit qualifiers is much the same method.
%    \begin{macrocode}
\newcommand*{\newqualifier}[2]{%
  \si@ifdefinable{#1}
    {\si@unt@defqual{#1}{#2}}
    {\si@log@err{Qualifier \string#1 already defined!}\@eha}}
\newcommand*{\renewqualifier}[2]{%
  \si@ifdefinable{#1}
    {\si@log@err{Qualifier \string#1 undefined}\@ehc
     \si@unt@defqual{#1}{#2}}
    {\si@unt@defqual{#1}{#2}}}
\newcommand*{\providequalifier}[2]{%
  \si@ifdefinable{#1}
    {\si@unt@defqual{#1}{#2}}
    {}}
%    \end{macrocode}
%\end{macro}
%\end{macro}
%\end{macro}
%\begin{macro}{\ifsi@unt@num}
% A flag is needed to tell the processor whether there is a number,
% to get the correct spacing.
%    \begin{macrocode}
\newif\ifsi@unt@num
%    \end{macrocode}
%\end{macro}
%\begin{macro}{\si@unt@unitarg}
% A storage macro for the argument of the unit macro.
%    \begin{macrocode}
\newcommand*{\si@unt@unitarg}{}
%    \end{macrocode}
%\end{macro}
%\begin{macro}{\si@SI}
%\begin{macro}{\si@unt@SIopts}
%\doarg{keyval options}
%\darg{unit}
%\doarg{preunit}
%\darg{unit}
% The internal processing starts with \cs{si@SI}, which processes the
% second optional argument to \cs{SI} (which is empty for \cs{si}).
% Everything is set up in a group, and processing begins by handling
% the options.
%    \begin{macrocode}
\newcommand*{\si@unt@SIopts}{}
\def\si@SI[#1]#2[#3]#4{%
  \begingroup
    \let\fg\SIfg
    \si@ifnotmtarg{#1}
      {\sisetup{#1}%
       \renewcommand*{\si@unt@SIopts}{#1}}%
%    \end{macrocode}
%\end{macro}
% The prefix unit is handled before any processing of the number; the
% flags are set to get spacing correct.
%    \begin{macrocode}
    \si@unt@numfalse
    \si@xspacefalse
    \si@ifnotmtarg{#3}
      {\si@log@debug{Prefix unit found}%
       \si@unt@printunit{#3}}%
%    \end{macrocode}
% The numerical argument may be empty, in which case no extra space
% should be produced.
%    \begin{macrocode}
    \si@ifnotmtarg{#4}
      {\renewcommand*{\si@unt@unitarg}{#4}}%
    \si@ifnotmtarg{#2}
      {\si@log@debug{Number found in \string\SI\space
         argument}%
       \ifsi@repeatunits\else
         \ifsi@trapambigerr
           \expandafter\expandafter\expandafter
             \si@num@ambigerrtrue
         \fi
       \fi
       \num{#2}%
       \si@unt@numtrue}%
%    \end{macrocode}
% If there is a unit, a check is needed in case the units need to
% have a power added.
%    \begin{macrocode}
    \si@ifnotmtarg{#4}
      {\si@ifmtarg{#2}
         {\si@unt@printunit{#4}}
         {\si@tempcnta\z@\relax
         \ifsi@addunitpower
           \si@unt@countx{#2}%
         \fi
          \ifnum\si@tempcnta>\z@\relax
            \advance\si@tempcnta\@ne\relax
            \edef\si@tempa{\noexpand\tothe{\si@tempcnta}}%
            \renewcommand*{\si@tempb}{#4}%
            \expandafter\expandafter\expandafter
            \si@unt@printunit\expandafter\expandafter
            \expandafter{%
              \expandafter\si@tempb\si@tempa}%
          \else
            \si@unt@printunit{#4}%
          \fi}}%
  \endgroup}
%    \end{macrocode}
%\end{macro}
%\begin{macro}{\si@unt@countx}
% A short macro to count up any multiplication in numerical input.
%    \begin{macrocode}
\newcommand*{\si@unt@countx}[1]{%
  \si@tempcnta\z@\relax
  \expandafter\si@unt@cntx#1\@empty\@empty}
\def\si@unt@cntx#1#2\@empty{%
  \si@str@ifchrstr{#1}{\si@numprod}
    {\advance\si@tempcnta\@ne\relax}
    {}%
  \ifx\@empty#2\@empty\else
    \si@unt@cntx#2\@empty\@empty
  \fi}
%    \end{macrocode}
%\end{macro}
%\begin{macro}{\si@unt@ifliteral}
%\begin{macro}{\ifsi@unt@littest}
%\darg{unit}
% The next stage of the processor is to determine whether or not the
% argument of the unit macro is processable.  For literal arguments,
% this is not the case, and the argument is typeset ``as is''.  On
% the other hand, any units, \etc., declared by the package will work
% with the processor, and so need to be executed before typesetting
% the result.
%    \begin{macrocode}
\newif\ifsi@unt@littest
\newcommand*{\si@unt@ifliteral}[1]{%
  \begingroup
    \si@unt@littesttrue
%    \end{macrocode}
% The test relies on any non-processable test having some width;
% hopefully, this should be the case.
%    \begin{macrocode}
    \setbox\si@tempboxa=\hbox{\si@unt@out{#1}}%
    \ifdim\wd\si@tempboxa>\z@\relax
      \aftergroup\@firstoftwo
    \else
      \aftergroup\@secondoftwo
    \fi
  \endgroup}
%    \end{macrocode}
%\end{macro}
%\end{macro}
%\begin{macro}{\ifsi@unt@litout}
%\begin{macro}{\si@unt@printunit}
%\darg{unit}
% The printing macro uses the above test to determine how to act.  It
% then carries out the appropriate action: either typesetting or
% executing.  A flag is also provided so that any macro units inside
% a partially-literal argument will work (this is also needed to
% emulate \unitsdef).
%    \begin{macrocode}
\newif\ifsi@unt@litout
\newcommand*{\si@unt@printunit}[1]{%
  \si@unt@ifliteral{#1}
%    \end{macrocode}
% The unit includes one or more literal items; typeset using the
% unit typesetting macro.
%    \begin{macrocode}
    {\si@log@debug{Literal items found in unit
      argument:\MessageBreak outputting without further
      processing}%
     \si@unt@litouttrue
     \si@unt@addvaluesep
     \si@unt@out{#1}}
%    \end{macrocode}
% For processable output, the argument is executed; the macros are
% all designed for this.
%    \begin{macrocode}
    {\si@log@debug{Macro unit found:\MessageBreak
       processing to format output}%
     \si@unt@init
     \advance\si@unt@depthcnt\@ne\relax
     #1%
     \si@unt@final}}
%    \end{macrocode}
%\end{macro}
%\end{macro}
%\begin{macro}{\si@unt@addvaluesep}
%\begin{macro}{\si@unt@addvalsep}
%\begin{macro}{\si@unt@litvalsep}
%\begin{macro}{\si@unt@stackvalsep}
% To ensure no problems pop up with expansion, adding the value--unit
% space is handled by a macro.
%    \begin{macrocode}
\newcommand*{\si@unt@addvaluesep}{%
  \ifsi@unt@num
    \expandafter\si@unt@addvalsep
  \fi}
\newcommand*{\si@unt@addvalsep}{%
  \ifsi@unt@litout
    \expandafter\si@unt@litvalsep
  \else
    \expandafter\si@unt@stackvalsep
  \fi}
\newcommand*{\si@unt@stackvalsep}{%
  \protected@edef\si@unt@spstack{\si@valuesep}}
\newcommand*{\si@unt@litvalsep}{%
  \nobreak\ensuremath{\si@valuesep}\nobreak}
%    \end{macrocode}
%\end{macro}
%\end{macro}
%\end{macro}
%\end{macro}
%\begin{macro}{\si@unt@spstack}
%\begin{macro}{\si@unt@stacka}
%\begin{macro}{\si@unt@stackb}
%\begin{macro}{\si@unt@unitcnta}
%\begin{macro}{\si@unt@unitcntb}
%\begin{macro}{\si@unt@depthcnt}
%\begin{macro}{\ifsi@unt@first}
%\begin{macro}{\ifsi@unt@first}
%\begin{macro}{\si@unt@init}
% The initialisation macro sets up the various switches, and clears
% the storage areas for the formatted output.  There are two stacks,
% as when typesetting as fractions, the two parts of the number have
% to be stored separately.  The depth counter is used to tell when
% recursion ends in the processor.  The ``first'' switch is needed as
% the depth counter will not be at one for items processed by
% \cs{SI}.
%    \begin{macrocode}
\newcommand*{\si@unt@spstack}{}
\newcommand*{\si@unt@stacka}{}
\newcommand*{\si@unt@stackb}{}
\newcount\si@unt@unitcnta
\newcount\si@unt@unitcntb
\newcount\si@unt@depthcnt
\newif\ifsi@unt@first
\si@unt@depthcnt\m@ne\relax
\newcommand*{\si@unt@init}{%
  \begingroup
    \si@unt@litoutfalse
    \si@unt@litprefixfalse
    \si@unt@firsttrue
    \si@unt@perfalse
    \si@unt@perseenfalse
    \si@unt@prepowerfalse
    \si@unt@depthcnt\z@\relax
    \si@unt@powerdim\z@\relax
    \si@unt@unitcnta\z@\relax
    \si@unt@unitcntb\z@\relax
    \si@unt@prefixcnt\z@\relax
    \renewcommand*{\si@unt@spstack}{}%
    \renewcommand*{\si@unt@stacka}{}%
    \renewcommand*{\si@unt@stackb}{}%
    \renewcommand*{\si@unt@spacestacka}{}%
    \renewcommand*{\si@unt@spacestackb}{}%
    \renewcommand*{\si@unt@lastadda}{space}%
    \renewcommand*{\si@unt@lastaddb}{space}}
%    \end{macrocode}
%\end{macro}
%\end{macro}
%\end{macro}
%\end{macro}
%\end{macro}
%\end{macro}
%\end{macro}
%\end{macro}
%\end{macro}
%\begin{macro}{\si@unt@final}
% The finalisation macro finishes off the output and resets the flags.
%    \begin{macrocode}
\newcommand*{\si@unt@final}{%
    \si@unt@third
    \si@unt@stackout
  \endgroup
  \ifsi@xspace
    \expandafter\expandafter\expandafter\xspace
  \fi}
%    \end{macrocode}
%\end{macro}
%\begin{macro}{\si@unt@defunit}
%\doarg{keyval options}
%\darg{unit}
%\darg{symbol}
% The internal macro for defining a unit does not check for
% redefinition; that is done by the user macros.
%    \begin{macrocode}
\newcommand*{\si@unt@defunit}[3][]{%
  \si@log@debug{Declaring unit \string#2 with \MessageBreak
    meaning \string#3}%
%    \end{macrocode}
% The optional argument needs to be saved. The macro name is reversed
% so that life is easier with the expansions here.
%    \begin{macrocode}
  \si@ifnotmtarg{#1}
    {\expandafter\@namedef\expandafter{%
      \expandafter\@gobble\string#2@opt@unt@si}{#1}}%
%    \end{macrocode}
% The unit macro itself is now defined.  The definition simply
% selects the correct path for the rest of the processing to go down.
%    \begin{macrocode}
  \protected\def#2{%
    \ifsi@allowoptarg
      \expandafter\si@unt@withopt
    \else
      \expandafter\si@unt@noopt
    \fi
    {#2}{#3}}}
%    \end{macrocode}
%\end{macro}
%\begin{macro}{\si@unt@withopt}
%\begin{macro}{\si@unt@noopt}
%\darg{unit}
%\darg{symbol}
% To allow the correct expansion, the potential optional argument to
% a unit macro has to come last.  So \cs{@ifnextchar} is needed to
% detect it and pass data through.  To keep variation down, when the
% argument is not allowed, the empty |[]| is supplied.
%    \begin{macrocode}
\newcommand*{\si@unt@withopt}[2]{%
  \@ifnextchar[%]
    {\si@unt@opt{#1}{#2}}
    {\si@unt@opt{#1}{#2}[]}}
\newcommand*{\si@unt@noopt}[2]{\si@unt@opt{#1}{#2}[]}
%    \end{macrocode}
%\end{macro}
%\end{macro}
%\begin{macro}{\si@unt@opt}
%\darg{unit}
%\darg{symbol}
%\doarg{number}
% The optional argument to the unit macro (if present) is converted
% to a normal one for ease.  The correct route for processing is then
% picked.
%    \begin{macrocode}
\def\si@unt@opt#1#2[#3]{%
  \ifsi@unt@littest
    \expandafter\si@gobblethree
  \else
%    \end{macrocode}
% For literal output, the second argument is all that is needed.
%    \begin{macrocode}
    \ifsi@unt@litout
      \expandafter\expandafter\expandafter\@gobbletwo
    \else
      \expandafter\expandafter\expandafter\si@unt@unit
    \fi
  \fi
  {#3}{#1}{#2}}
%    \end{macrocode}
%\end{macro}
%\begin{macro}{\si@gobblethree}
% \LaTeX\ does not have a \cs{@gobblethree} macro, but one is needed.
%    \begin{macrocode}
\long\def\si@gobblethree #1#2#3{}
%    \end{macrocode}
%\end{macro}
%\begin{macro}{\si@unt@defprefix}
%\begin{macro}{\ifsi@unt@litprefix}
%\doarg{\texttt{binary}}
%\darg{multiple}
%\darg{powers-ten}
%\darg{symbol}
% As with units, multiples are defined by an internal macro.
%    \begin{macrocode}
\newif\ifsi@unt@litprefix
\si@unt@litprefixtrue
\newcommand*{\si@unt@defprefix}[4][]{%
  \si@log@debug{Declaring multiple \string#1 with\MessageBreak
    meaning \string#4}%
%    \end{macrocode}
% The optional argument is saved, using \cs{def} as no check is made
% on an existing definition of the storage macro.
%    \begin{macrocode}
  \expandafter\expandafter\expandafter\def\expandafter
    \csname\expandafter\@gobble\string#2@opt@si\endcsname{#1}%
  \protected\def#2{%
    \ifsi@unt@littest
      \expandafter\si@gobblethree
    \else
      \ifsi@unt@litout
        \expandafter\expandafter\expandafter\@gobbletwo
      \else
        \ifsi@unt@litprefix
          \expandafter\expandafter\expandafter\expandafter
            \expandafter\expandafter\expandafter\@gobbletwo
        \else
          \expandafter\expandafter\expandafter\expandafter
            \expandafter\expandafter\expandafter\si@unt@prefix
        \fi
      \fi
    \fi
    {#2}{#3}{#4}}}
%    \end{macrocode}
%\end{macro}
%\end{macro}
%\begin{macro}{\si@unt@defpower}
%\doarg{\texttt{post}}
%\darg{number}
%\darg{power}
% The definition of powers is complicated by the need to handle both
% those given before units (such as \cs{cubic}) and those given after
% (\eg \cs{cubed}).  This means that an optional argument is needed.
%    \begin{macrocode}
\newcommand*{\si@unt@defpower}[3][]{%
  \si@log@debug{Declaring power \string#2 with\MessageBreak
    meaning \string#3}%
%    \end{macrocode}
% Once again the optional argument is saved.
%    \begin{macrocode}
  \expandafter\expandafter\expandafter\def\expandafter
    \csname\expandafter\@gobble\string#2@opt@si\endcsname{#1}%
  \protected\def#2{%
    \ifsi@unt@littest
      \expandafter\@gobbletwo
    \else
%    \end{macrocode}
% The literal output here does not need to gobble anything.
%    \begin{macrocode}
      \ifsi@unt@litout
        \expandafter\expandafter\expandafter\si@unt@litpower
      \else
        \expandafter\expandafter\expandafter\si@unt@power
      \fi
    \fi
    {#2}{#3}}}
%    \end{macrocode}
%\end{macro}
%\begin{macro}{\si@unt@defqual}
%\darg{qualifier}
%\darg{symbol}
% Qualifiers are defined in a similar manner.
%    \begin{macrocode}
\newcommand*{\si@unt@defqual}[2]{%
  \si@log@debug{Declaring qualifier \string#1 with\MessageBreak
    meaning \string#2}%
  \protected\def#1{%
    \ifsi@unt@littest
      \expandafter\@gobble
    \else
      \ifsi@unt@litout
        \expandafter\expandafter\expandafter\si@unt@litqual
      \else
        \expandafter\expandafter\expandafter\si@unt@qual
      \fi
    \fi
    {#2}}}
%    \end{macrocode}
%\end{macro}
%
%\begin{macro}{\si@unt@unithook}
%\begin{macro}{\si@unt@unit}
%\darg{number}
%\darg{unit}
%\darg{symbol}
% The macro for units is actually a processor, rather than
% typesetting anything, which is handled elsewhere.  The first
% argument to the macro is optional, but does not have square
% brackets to keep things simple with gobbling.
%    \begin{macrocode}
\newcommand*{\si@unt@unithook}{}
\newcommand*{\si@unt@unit}[3]{%
%    \end{macrocode}
% When the count is minus one at the start of the processor, then the
% unit is begin used on its own: initialisation occurs.
%    \begin{macrocode}
  \ifnum\si@unt@depthcnt=\m@ne\relax
    \expandafter\si@unt@init
  \fi
  \advance\si@unt@depthcnt\@ne\relax
  \si@log@debug{Unit processing: level \the\si@unt@depthcnt,
    \MessageBreak unit \string#2}%
  \si@unt@firstorsecond{#1}{#2}%
%    \end{macrocode}
% The core of the \cs{si@unt@unit} macro is testing if the symbol for
% the unit is a literal value or another macro.  Depending on the
% result, the symbol is either used as a literal or executed.
%    \begin{macrocode}
  \si@unt@ifliteral{#3}
    {\si@unt@addtostack{unit}{#3}%
     \ifsi@unt@prepower
       \expandafter\si@unt@stkpower
     \fi}
    {#3}%
%    \end{macrocode}
% The counter is now stepped down, before checking if this is
% the end of a compound unit.
%    \begin{macrocode}
  \advance\si@unt@depthcnt\m@ne\relax
  \ifnum\si@unt@depthcnt=\z@\relax
    \expandafter\si@unt@final
  \fi}
%    \end{macrocode}
%\end{macro}
%\end{macro}
%\begin{macro}{\si@unt@firstorsecond}
%\darg{number}
%\darg{unit-macro}
% At this stage, the flag will be set for the first item to be
% processed whichever route the unit has been called by.
%    \begin{macrocode}
\newcommand*{\si@unt@firstorsecond}[2]{%
  \ifsi@unt@first
    \expandafter\si@unt@first
  \else
    \expandafter\si@unt@second
  \fi
  {#1}{#2}}%
%    \end{macrocode}
%\end{macro}
%\begin{macro}{\si@unt@first}
%\darg{number}
%\darg{unit-macro}
% For the first unit in the input, some extra tasks are needed.
% First, the optional argument for the unit macro needs to be
% tested.
%    \begin{macrocode}
\newcommand*{\si@unt@first}[2]{%
  \si@ifnotmtarg{#1}
    {\num{#1}%
     \si@unt@numtrue}%
  \si@unt@unithook
%    \end{macrocode}
% To avoid filling up the macro list with useless values, the \eTeX\
% primitive \cs{ifcsname} is employed here (it also avoids complex
% expansion issues).  If some options exist, they are set.
%    \begin{macrocode}
  \ifcsname\expandafter\@gobble\string#2@opt@unt@si\endcsname
    \expandafter\si@unt@setopts
  \else
    \expandafter\@gobble
  \fi
  {#2}%
  \si@unt@addvaluesep
  \si@unt@firstfalse}
%    \end{macrocode}
%\end{macro}
%\begin{macro}{\si@unt@setopts}
%\begin{macro}{\si@unt@setSIopts}
%\darg{unit}
% A rather long set of \cs{expandafter} commands to get the options
% to set safely.
%    \begin{macrocode}
\newcommand*{\si@unt@setopts}[1]{%
  \expandafter\expandafter\expandafter\expandafter\expandafter
    \expandafter\expandafter\si@temptoks\expandafter
    \expandafter\expandafter\expandafter\expandafter
    \expandafter\expandafter{\expandafter%
      \csname\expandafter\@gobble\string#1@opt@unt@si%
      \endcsname}%
  \expandafter\sisetup\expandafter{\the\si@temptoks}%
  \si@log@debug{Applying options `\the\si@temptoks'
    for\MessageBreak unit \string#1}%
%    \end{macrocode}
% The user options are reloaded, if defined, to ensure that they
% still work as expected.
%    \begin{macrocode}
  \@ifundefined{si@unt@SIopts}{}
    {\ifx\@empty\si@unt@SIopts\@empty\else
       \expandafter\expandafter\si@unt@setSIopts
     \fi}}
\newcommand*{\si@unt@setSIopts}{%
  \expandafter\si@temptoks\expandafter{\si@unt@SIopts}%
  \expandafter\sisetup\expandafter{\the\si@temptoks}}
%    \end{macrocode}
%\end{macro}
%\end{macro}
%\begin{macro}{\si@unt@second}
%\begin{macro}{\si@unt@third}
%\darg{number}
%\darg{unit}
% For everything apart from the first item to be processed, spacing
% may need to be added to separated different units.  The macro is
% divided into two, so that everything except the space can be added
% in finalisation.
%    \begin{macrocode}
\newcommand*{\si@unt@second}[2]{%
  \si@ifnotmtarg{#1}
    {\si@log@warn{Optional argument to unit macro\MessageBreak
       allowed only for outer unit}}%
  \si@unt@third
  \si@unt@addtostack{space}{\ensuremath{\si@unitsep}}}
\newcommand*{\si@unt@third}{%
  \ifsi@unt@prepower\else
    \expandafter\si@unt@stkpower
  \fi
%    \end{macrocode}
% A check is made to avoid adding $-1$ to prefixes.  If \opt{frac} is
% active, then the |b| stack will be in use, otherwise it will be
% |a|.
%    \begin{macrocode}
  \renewcommand*{\si@tempa}{prefix}%
  \expandafter\ifx
    \csname si@unt@lastadd\si@unt@checkstack\endcsname\si@tempa
  \else
    \expandafter\si@unt@spacecheck
  \fi
  \ifsi@unt@per
    \expandafter\si@unt@perseentrue
  \fi}
%    \end{macrocode}
%\end{macro}
%\end{macro}
%\begin{macro}{\si@unt@spacecheck}
% A check to prevent adding $-1$ at the very beginning of the unit,
% where there is a space on the stack.
%    \begin{macrocode}
\newcommand*{\si@unt@spacecheck}{%
  \renewcommand*{\si@tempa}{space}%
  \expandafter\ifx
    \csname si@unt@lastadd\si@unt@checkstack\endcsname\si@tempa
  \else
    \expandafter\si@unt@reciptest
  \fi}
%    \end{macrocode}
%\end{macro}
%\begin{macro}{\si@unt@prefix}
%\darg{multiple}
%\darg{power-ten}
%\darg{symbol}
% Actual output of the prefixes.
%    \begin{macrocode}
\newcommand*{\si@unt@prefix}[3]{%
  \si@unt@firstorsecond{}{#1}%
  \ifsi@prefixsymbolic
    \expandafter\si@unt@addprefix
  \else
    \expandafter\si@unt@countprefix
  \fi
  {#1}{#2}{#3}}
%    \end{macrocode}
%\end{macro}
%\begin{macro}{\si@unt@addprefix}
%\darg{gobbled}
%\darg{gobbled}
%\darg{symbol}
% To add the prefix, a little translation is needed.
%    \begin{macrocode}
\newcommand*{\si@unt@addprefix}[3]{%
  \si@unt@addtostack{prefix}{#3}}
%    \end{macrocode}
%\end{macro}
%\begin{macro}{\si@unt@prefixcnt}
% A storage area is created.
%    \begin{macrocode}
\newcount\si@unt@prefixcnt
%    \end{macrocode}
%\end{macro}
%\begin{macro}{\si@unt@countprefix}
%\begin{macro}{\si@unt@invprefix}
%\darg{multiple}
%\darg{powers-ten}
%\darg{gobbled}
% To count the prefix numeral, the symbol is thrown away.  First, a
% check is made for binary units.
%    \begin{macrocode}
\newcommand*{\si@unt@countprefix}[3]{%
  \renewcommand*{\si@tempa}{binary}%
  \expandafter\expandafter\expandafter\ifx\expandafter
    \csname\expandafter\@gobble\string#1@opt@si\endcsname
    \si@tempa
    \expandafter\sisetup
  \else
    \expandafter\@gobble
  \fi
  {prefixbase=two}%
  \si@tempcnta#2\relax
  \ifsi@unt@per
    \expandafter\si@unt@invprefix
  \fi
  \advance\si@unt@prefixcnt\si@tempcnta\relax}
\newcommand*{\si@unt@invprefix}{%
  \si@tempcntb\si@tempcnta\relax
  \si@tempcnta -\si@tempcntb\relax}
%    \end{macrocode}
%\end{macro}
%\end{macro}
%\begin{macro}{\si@unt@litpower}
%\darg{gobbled}
%\darg{number}
% For literal power output, the number can't simply be dumped, so
% a macro is needed.
%    \begin{macrocode}
\newcommand*{\si@unt@litpower}[2]{\textsuperscript{#2}}
%    \end{macrocode}
%\end{macro}
%\begin{macro}{\ifsi@unt@prepower}
% If a power is seen before a unit, tracking is needed.
%    \begin{macrocode}
\newif\ifsi@unt@prepower
%    \end{macrocode}
%\end{macro}
%\begin{macro}{\si@unt@power}
%\darg{power}
%\darg{number}
% The handling of powers starts by ensuring that ``pre'' powers
% follow \cs{per} cleanly.  Then a check is needed for inversion.
%    \begin{macrocode}
\newcommand*{\si@unt@power}[2]{%
  \renewcommand*{\si@tempa}{post}%
  \expandafter\expandafter\expandafter\ifx\expandafter
    \csname\expandafter\@gobble\string#1@opt@si\endcsname
    \si@tempa
    \expandafter\@gobbletwo
  \else
    \expandafter\si@unt@firstorsecond
  \fi
  {}{\power}%
  \si@unt@powerdim #2 pt\relax
  \ifsi@frac\else
    \ifsi@unt@per
      \expandafter\expandafter\expandafter\si@unt@invpower
    \fi
  \fi
  \renewcommand*{\si@tempa}{post}%
  \si@unt@prepowertrue
  \expandafter\expandafter\expandafter\ifx\expandafter
    \csname\expandafter\@gobble\string#1@opt@si\endcsname
    \si@tempa
    \expandafter\si@unt@stackpower
  \else
    \si@log@debug{Power \strip@pt\si@unt@powerdim\space saved
      to be added after\MessageBreak next unit}%
  \fi}
%    \end{macrocode}
%\end{macro}
%\begin{macro}{\si@unt@powerdim}
% To do sign-inversion on the power, a dimension is used (this allows
% non-integer values to be handled).
%    \begin{macrocode}
\newdimen\si@unt@powerdim
%    \end{macrocode}
%\end{macro}
%\begin{macro}{\si@unt@stackpower}
%\begin{macro}{\si@unt@stkpower}
%\begin{macro}{\si@unt@stkpwr}
%\begin{macro}{\si@unt@skpwr}
% Adding powers to the stack should also clear the power list.  If
% the number is already zero, then  of course nothing happens.
%    \begin{macrocode}
\newcommand*{\si@unt@stackpower}{%
  \si@unt@prepowerfalse
%    \end{macrocode}
% A trap is used for $-1$ added to the denominator of a fraction.
%    \begin{macrocode}
  \si@unt@stkpower
  \ifsi@stickyper\else
    \si@unt@perfalse
    \si@unt@perseenfalse
  \fi}
%    \end{macrocode}
% The \cs{si@unt@stkpower} macro needs to be called from a few
% places, so is spun out from the above.
%    \begin{macrocode}
\newcommand*{\si@unt@stkpower}{%
  \ifdim\si@unt@powerdim=\m@ne pt\relax
    \ifsi@frac\else
      \expandafter\expandafter\expandafter\si@unt@stkpwr
    \fi
  \else
    \expandafter\si@unt@stkpwr
  \fi}
%    \end{macrocode}
%\changes{v1.1}{2008/09/04}{Altered to account for qualifiers}
% Finally, the actual adding (set up to avoid problems with the
% \cs{if} above).
%    \begin{macrocode}
\newcommand*{\si@unt@stkpwr}{%
  \ifdim\si@unt@powerdim=\z@\relax\else
    \renewcommand*{\si@tempa}{unit}%
    \expandafter\ifx
      \csname si@unt@lastadd\si@unt@checkstack\endcsname
      \si@tempa
      \si@unt@skpwr
    \else
      \renewcommand*{\si@tempa}{qual}%
      \expandafter\ifx
        \csname si@unt@lastadd\si@unt@checkstack\endcsname
        \si@tempa
        \si@unt@skpwrqual
      \fi
    \fi
  \fi
  \si@unt@powerdim\z@\relax}
\newcommand*{\si@unt@skpwr}{%
  \si@log@debug{Adding power
    \strip@pt\si@unt@powerdim\space to output stack}%
  \si@unt@addtostack{power}{^{\num[fixdp=false]{%
    \strip@pt\si@unt@powerdim}}}}
%    \end{macrocode}
%\end{macro}
%\end{macro}
%\end{macro}
%\end{macro}
%\begin{macro}{\si@unt@invpower}
% A macro to change the sign of the current power.
%    \begin{macrocode}
\newcommand*{\si@unt@invpower}{%
  \si@tempdima\si@unt@powerdim\relax
  \si@unt@powerdim -\si@tempdima\relax
%    \end{macrocode}
% The power might end up as ``1'', which is not wanted.  So it is
% chucked away.
%    \begin{macrocode}
  \ifdim\si@unt@powerdim=\p@\relax
    \si@unt@powerdim\z@\relax
  \fi}
%    \end{macrocode}
%\end{macro}
%\begin{macro}{\si@unt@litqual}
%    \begin{macrocode}
\newcommand*{\si@unt@litqual}[1]{(#1)}
%    \end{macrocode}
%\end{macro}
%\begin{macro}{\si@unt@qual}
%    \begin{macrocode}
\newcommand*{\si@unt@qual}[1]{%
  \ifsi@unt@first
    \si@log@err{Qualifier before units}
      {A unit qualifier can only be given after\MessageBreak
       one or more units.  You tried to open a unit
       with\MessageBreak \string#1}%
  \fi
%    \end{macrocode}
% The space after the current unit is shuffled so it will come
% after the qualifier.
%    \begin{macrocode}
  \expandafter\let\expandafter\si@unt@spacestack
    \csname si@unt@spacestack\si@unt@checkstack\endcsname
  \expandafter\let\csname si@unt@spacestack\si@unt@checkstack
    \endcsname\@empty
%    \end{macrocode}
%\begin{macro}{\si@unt@skpwrqual}
% Some protection against trying to add two powers.
%    \begin{macrocode}
  \renewcommand*{\si@tempa}{power}%
    \expandafter\ifx
    \csname si@unt@lastadd\si@unt@checkstack\endcsname\si@tempa
    \let\si@unt@skpwrqual\relax
  \else
    \let\si@unt@skpwrqual\si@unt@skpwr
  \fi
  \si@unt@addtostack{qual}{_{#1}}%
  \expandafter\let\csname si@unt@spacestack\si@unt@checkstack
    \endcsname\si@unt@spacestack}
%    \end{macrocode}
%\end{macro}
%\end{macro}
%\begin{macro}{\ifsi@unt@per}
%\begin{macro}{\ifsi@unt@perseen}
%\begin{macro}{\per}
%\begin{macro}{\si@per}
%\begin{macro}{\si@unt@per}
% The \cs{per} macro sets the correct flags; almost everything else
% is done elsewhere.  There is always the case of two \cs{per}
% instructions; so the flag is flipped rather than set arbitrarily.
% The second flag is needed so that \cs{per} can give powers of $-1$
% properly.
%    \begin{macrocode}
\newif\ifsi@unt@per
\newif\ifsi@unt@perseen
\si@newrobustcmd*{\si@per}{%
  \ifsi@unt@littest\else
    \ifsi@unt@litout
      \expandafter\expandafter\expandafter /%
    \else
      \ifsi@unt@litprefix
        \expandafter\expandafter\expandafter\expandafter
          \expandafter\expandafter\expandafter /%
      \else
        \expandafter\expandafter\expandafter\expandafter
          \expandafter\expandafter\expandafter\si@unt@per
      \fi
    \fi
  \fi}
\newcommand*{\si@unt@per}{%
  \si@unt@firstorsecond{}{\per}%
  \ifsi@unt@per
    \ifsi@stickyper\else
      \expandafter\expandafter\expandafter\si@unt@perfalse
    \fi
  \else
    \expandafter\si@unt@pertrue
  \fi}
\let\per\si@per
%    \end{macrocode}
%\end{macro}
%\end{macro}
%\end{macro}
%\end{macro}
%\end{macro}
%\begin{macro}{\si@unt@reciptest}
%\begin{macro}{\si@unt@recip}
% A test is needed for adding $-1$ when needed. The second macro is
% fired only if the power should be reciprocal.
%    \begin{macrocode}
\newcommand*{\si@unt@reciptest}{%
  \ifsi@unt@per
    \ifsi@unt@perseen
      \expandafter\expandafter\expandafter\si@unt@recip
    \fi
  \fi}
\newcommand*{\si@unt@recip}{%
  \si@unt@powerdim\m@ne pt\relax
  \si@unt@stackpower}
%    \end{macrocode}
%\end{macro}
%\end{macro}
%\begin{macro}{\si@unt@lastadda}
%\begin{macro}{\si@unt@lastaddb}
% Items cannot be added directly to the stacks (except the spacing
% stack, a) as the fractional handling may need to add the item to
% either storage area. First, a track is kept of what has
% been added at each stage.
%    \begin{macrocode}
\newcommand*{\si@unt@lastadda}{}
\newcommand*{\si@unt@lastaddb}{}
%    \end{macrocode}
%\end{macro}
%\end{macro}
%\begin{macro}{\si@unt@addtostack}
%\changes{v1.1b}{2008/10/12}{Added warning for multiple prefixes}
%\darg{type}
%\darg{token}
% By indicating the type of data to be added to the stack, problems
% can be avoided.
%    \begin{macrocode}
\newcommand*{\si@unt@addtostack}[2]{%
  \renewcommand*{\si@tempa}{#1}%
%    \end{macrocode}
% Two consecutive items cannot be of the same type; there must be spaces
% between units, units between prefixes, \etc.
%    \begin{macrocode}
  \expandafter\ifx
    \csname si@unt@lastadd\si@unt@checkstack\endcsname\si@tempa
      \renewcommand*{\si@tempa}{prefix}%
      \expandafter\ifx\csname si@unt@lastadd\si@unt@checkstack
        \endcsname\si@tempa
        \si@log@warn{Two prefixes found: only first printed}%
      \fi
    \expandafter\@gobbletwo
  \else
    \expandafter\si@unt@preplussp
  \fi
  {#1}{#2}}
%    \end{macrocode}
%\end{macro}
%\begin{macro}{\si@unt@preplussp}
%\darg{type}
%\darg{stack}
%\darg{token}
%\darg{gobbled}
% The space added after a prefix needs to be ignored.
%    \begin{macrocode}
\newcommand*{\si@unt@preplussp}[2]{%
  \renewcommand*{\si@tempa}{prefix+space}%
  \edef\si@tempb{%
    \csname si@unt@lastadd\si@unt@checkstack\endcsname+#1}%
  \ifx\si@tempa\si@tempb
    \expandafter\@gobbletwo
  \else
    \expandafter\si@unt@stack
  \fi
  {#1}{#2}}
%    \end{macrocode}
%\end{macro}
%\begin{macro}{\si@unt@stack}
%\darg{type}
%\darg{token}
% The macro for actually doing the stacking up.
%    \begin{macrocode}
\newcommand*{\si@unt@stack}[2]{%
  \expandafter\renewcommand\expandafter*\expandafter{%
    \csname si@unt@lastadd\si@unt@checkstack\endcsname}{#1}%
%    \end{macrocode}
% A count is kept of the number of \emph{units} added to each stack.
%    \begin{macrocode}
  \renewcommand*{\si@tempa}{#1}%
  \renewcommand*{\si@tempb}{unit}%
  \ifx\si@tempa\si@tempb
    \expandafter\si@unt@inccnt
  \fi
%    \end{macrocode}
% If a space is added, it is actually held until the next add.
%    \begin{macrocode}
  \renewcommand*{\si@tempb}{space}%
  \ifx\si@tempa\si@tempb
    \expandafter\si@unt@holdspace
  \else
    \expandafter\si@unt@addstack
  \fi
  {#2}}
%    \end{macrocode}
%\end{macro}
%\begin{macro}{\si@unt@inccnt}
% The appropriate counter is added to.
%    \begin{macrocode}
\newcommand*{\si@unt@inccnt}{%
  \expandafter\advance
    \csname si@unt@unitcnt\si@unt@checkstack\endcsname
    \@ne\relax}
%    \end{macrocode}
%\end{macro}
%\begin{macro}{\si@unt@spacestack}
%\begin{macro}{\si@unt@spacestacka}
%\begin{macro}{\si@unt@spacestackb}
% The stacked material needs somewhere to live.
%    \begin{macrocode}
\newcommand*{\si@unt@spacestack}{}
\newcommand*{\si@unt@spacestacka}{}
\newcommand*{\si@unt@spacestackb}{}
%    \end{macrocode}
%\end{macro}
%\end{macro}
%\end{macro}
%\begin{macro}{\si@unt@holdspace}
%\begin{macro}{\si@unt@addstack}
%\darg{token}
% Depending on the nature of the addition, it is either held or added
% to the stack.  For the ``b'' space stack, a check is made to ensure
% that a space cannot be added before the first item.
%    \begin{macrocode}
\newcommand*{\si@unt@holdspace}[1]{%
  \renewcommand*{\si@tempa}{b}%
  \edef\si@tempb{\si@unt@checkstack}%
  \ifx\si@tempa\si@tempb
    \ifx\@empty\si@unt@stackb\@empty
    \else
      \protected@edef\si@unt@spacestackb{#1}%
    \fi
  \else
    \protected@edef\si@unt@spacestacka{#1}%
  \fi}
\newcommand*{\si@unt@addstack}[1]{%
  \expandafter\protected@edef
    \csname si@unt@stack\si@unt@checkstack\endcsname
    {\csname si@unt@stack\si@unt@checkstack\endcsname
     \csname si@unt@spacestack\si@unt@checkstack\endcsname#1}%
  \expandafter\renewcommand\expandafter*\expandafter{%
    \csname si@unt@spacestack\si@unt@checkstack\endcsname}{}}
%    \end{macrocode}
%\end{macro}
%\end{macro}
%\begin{macro}{\si@unt@stackout}
% The stack contents are actually typeset here.  First the spacing
% between units and values is added.
%    \begin{macrocode}
\newcommand*{\si@unt@stackout}{%
  \si@unt@litouttrue
  \ifsi@frac
    \expandafter\si@unt@fracout
  \else
    \expandafter\si@unt@normout
  \fi}
%    \end{macrocode}
%\end{macro}
%\begin{macro}{\si@unt@checkstack}
% Which stack is in use needs to be tested.
%    \begin{macrocode}
\newcommand*{\si@unt@checkstack}{%
  \ifsi@frac
    \ifsi@unt@per
      \expandafter\expandafter\expandafter b%
    \else
      \expandafter\expandafter\expandafter a%
    \fi
  \else
    \expandafter a%
  \fi}
%    \end{macrocode}
%\end{macro}
%\begin{macro}{\si@unt@spaceout}
% The space before a unit might not be needed, so it crops up a few
% times in the output routine.
%    \begin{macrocode}
\newcommand*{\si@unt@spaceout}{%
   \ensuremath{\si@unt@spstack}}
%    \end{macrocode}
%\end{macro}
%\begin{macro}{\si@unt@prefixout}
% If the prefix counter is not zero, then there is something to
% typeset.
%    \begin{macrocode}
\newcommand*{\si@unt@prefixout}{%
  \ifnum\si@unt@prefixcnt=\z@\relax\else
    \ifsi@unt@num
      \si@out{\ensuremath{{}\si@prefixproduct{}}}%
    \fi
    \si@unt@stackvalsep
    \let\si@expbase\si@prefixbase
    \num[fixdp=false]{e\the\si@unt@prefixcnt}%
  \fi}
%    \end{macrocode}
%\end{macro}
%\begin{macro}{\si@unt@normout}
% The normal output mode is set up here; nothing much needs to be
% done as there is no need for complex checks.
%    \begin{macrocode}
\newcommand*{\si@unt@normout}{%
  \si@unt@prefixout
  \si@unt@spaceout
  \expandafter\si@unt@out\expandafter{\si@unt@stacka}}
%    \end{macrocode}
%\end{macro}
%\begin{macro}{\si@unt@fracout}
% For fractions, some checks are needed.
%    \begin{macrocode}
\newcommand*{\si@unt@fracout}{%
  \si@unt@notambig
  \ifx\@empty\si@unt@stacka\@empty
    \ifx\@empty\si@unt@stackb\@empty
      \ifsi@unt@litout\else
        \si@log@err{Empty fractional unit}{The unit
          argument\MessageBreak given does not contain any
          symbols}%
      \fi
    \else
%    \end{macrocode}
% With an empty numerator, no space is added
%    \begin{macrocode}
      \ifsi@slash
        \si@unt@prefixout
        \si@frac{}{\si@unt@stackb}%
      \else
        \si@unt@prefixout
        \si@unt@spaceout
        \si@frac{1}{\si@unt@stackb}%
      \fi
    \fi
  \else
%    \end{macrocode}
% If the denominator is empty, then the usual output system can be
% used.
%    \begin{macrocode}
    \ifx\@empty\si@unt@stackb\@empty
      \si@unt@normout
    \else
      \si@unt@prefixout
      \si@unt@spaceout
      \si@frac{\si@unt@stacka}{\si@unt@stackb}%
    \fi
  \fi}
%    \end{macrocode}
%\end{macro}
%\begin{macro}{\si@unt@notambig}
%\begin{macro}{\si@unt@notabg}
% A trap is set for adding brackets to units using a slash, when more
% than one unit is in the denominator.
%    \begin{macrocode}
\newcommand*{\si@unt@notambig}{%
  \ifnum\si@unt@unitcntb>\@ne\relax
    \ifsi@slash
        \ifsi@trapambigfrac
          \expandafter\expandafter\expandafter\expandafter
            \expandafter\expandafter\expandafter\si@unt@notabg
        \fi
    \fi
  \fi}
\newcommand*{\si@unt@notabg}{%
  \protected@edef\si@unt@stackb{\si@openfrac\si@unt@stackb
    \si@closefrac}}
%    \end{macrocode}
%\end{macro}
%\end{macro}
%
%\begin{macro}{\si@unt@out}
%\changes{v1.0d}{2008/06/25}{Sanitise one level of unit input
%  only}
%\darg{unit}
% The final part of the units system is the output routine.  This has
% to cope with units not only as macros but also as direct input
% (\SIstyle-type input).  Non-Latin characters are also handled
% cleanly.  The \cs{scantokens} system deals with everything except
% full stops; these are left out so that a single level system can be
% used \latin{via} a token register.
%    \begin{macrocode}
\begingroup
  \catcode`\~=\active
  \catcode`\.=\active
  \gdef\si@unt@out#1{%
    \si@temptoks{#1}%
    \si@unt@fullstop
    \def.{\ensuremath{\si@unitsep}}%
    \def~{\ensuremath{\si@unitspace}}%
    \expandafter\protected@edef\expandafter\si@tempa
      \expandafter{\the\si@temptoks}%
    \begingroup
      \si@unt@nonlatin
      \makeatletter
      \endlinechar\m@ne
      \expandafter\si@out\expandafter{%
        \expandafter\scantokens\expandafter{\si@tempa}}%
    \endgroup}
\endgroup
%    \end{macrocode}
%\end{macro}
%\begin{macro}{\si@unt@fullstop}
%\begin{macro}{\si@unt@stp}
% Two macros modified from \pkg{kvsetkeys} to deal with a single
% level of active full stops only.
%    \begin{macrocode}
\begingroup
  \catcode`\.=\active
  \catcode`\&=12\relax
  \begingroup
    \lccode`\.=`\.\relax
    \lccode`\&=`\.\relax
  \lowercase{\endgroup
    \gdef\si@unt@fullstop{%
      \si@temptoks\expandafter{\expandafter}\expandafter
      \si@unt@stp\the\si@temptoks&\@nil}
  \gdef\si@unt@stp#1&#2\@nil{%
    \edef\si@tempa{\the\si@temptoks}%
    \ifx\si@tempa\@empty
      \expandafter\@firstoftwo
    \else
      \expandafter\@secondoftwo
    \fi
      {\si@temptoks{#1}}
      {\si@temptoks\expandafter{\the\si@temptoks.#1}}%
    \si@ifmtarg{#2}
      {}
      {\si@unt@stp#2\@nil}}}
\endgroup
%    \end{macrocode}
%\end{macro}
%\end{macro}
%\begin{macro}{\si@unt@nonlatin}
% To handle non-Latin symbols in the input, a single macro is
% provided.  If \XeTeX\ is in use, this can be detected immediately.
%    \begin{macrocode}
\newcommand*{\si@unt@nonlatin}{}
\ifdefined\XeTeXrevision
  \renewcommand*{\si@unt@nonlatin}{%
    \catcode176=\active
    \catcode181=\active
    \catcode197=\active
    \si@unt@sym{176}{\si@sym@degree}%
    \si@unt@sym{181}{\si@sym@mu}%
    \si@unt@sym{197}{\si@sym@ringA}}%
\fi
%    \end{macrocode}
% If \pkg{inputenc} has been loaded, then a check is made that the
% encoding is correct.  If all is well, the non-Latin symbols are
% handled.  The degree symbol is character 176, the micro
% symbol is character 181 and ring capital A is character 197 in
% latin1.
%    \begin{macrocode}
\AtBeginDocument{
  \@ifpackageloaded{inputenc}
    {\@for\si@tempa:=latin1,latin5,latin9\do{
       \ifx\inputencodingname\si@tempa
         \renewcommand*{\si@unt@nonlatin}{%
           \catcode176=\active
           \catcode181=\active
           \catcode197=\active
           \si@unt@sym{176}{\si@sym@degree}%
           \si@unt@sym{181}{\si@sym@mu}%
           \si@unt@sym{197}{\si@sym@ringA}}%
       \fi}}
    {}}
%    \end{macrocode}
%\end{macro}
%\begin{macro}{\si@unt@sym}
%\darg{charcode}
% A macro for declaring symbols: a copy of \cs{DeclareInputMath} from
% \pkg{inputenc}.
%    \begin{macrocode}
\newcommand*{\si@unt@sym}[1]{%
  \bgroup
    \uccode`\~#1%
    \uppercase{%
      \egroup
      \def~}}
%    \end{macrocode}
%\end{macro}
%
%\begin{macro}{\kilogram}
%\begin{macro}{\metre}
%\begin{macro}{\meter}
%\begin{macro}{\mole}
%\begin{macro}{\kelvin}
%\begin{macro}{\candela}
%\begin{macro}{\second}
%\begin{macro}{\ampere}
% With the system set up, the basic unit macros are implemented.  The
% only units defined whatever options are given are the base SI
% units.
%    \begin{macrocode}
\newunit{\kilogram}{kg}
\newunit{\metre}{m}
\newunit{\meter}{\metre}
\newunit{\mole}{mol}
\AtBeginDocument{\provideunit{\second}{s}}
\newunit{\ampere}{A}
\newunit{\kelvin}{K}
\newunit{\candela}{cd}
%    \end{macrocode}
%\end{macro}
%\end{macro}
%\end{macro}
%\end{macro}
%\end{macro}
%\end{macro}
%\end{macro}
%\end{macro}
%\begin{macro}{\Square}
%\changes{v1.1b}{2008/10/12}{Only defined if not otherwise available}
%\begin{macro}{\ssquare}
%\begin{macro}{\squared}
%\begin{macro}{\cubic}
%\begin{macro}{\cubed}
% Unlike multiples (which can be skipped if needed), the basic powers
% are also always defined.
%    \begin{macrocode}
\AtBeginDocument{
  \@ifundefined{Square}{%
    \newpower{\Square}{2}}{}}
\newpower{\ssquare}{2}
\newpower[post]{\squared}{2}
\newpower{\cubic}{3}
\newpower[post]{\cubed}{3}
%    \end{macrocode}
%\end{macro}
%\end{macro}
%\end{macro}
%\end{macro}
%\end{macro}
%\begin{macro}{\tothe}
%\begin{macro}{\raiseto}
% The user macros for arbitraty powers are simple
% calling macros to a common internal macro.
%    \begin{macrocode}
\newcommand*{\tothe}{\si@tothe{\tothe}}
\newcommand*{\raiseto}{\si@tothe{\raiseto}}
%    \end{macrocode}
%\end{macro}
%\end{macro}
%\begin{macro}{\si@tothe}
%\begin{macro}{\tothe@opt@si}
%\begin{macro}{\raiseto@opt@si}
%\darg{either \texttt{\cs{tothe}} or \texttt{\cs{raiseto}}}
%\darg{number}
% A macro for arbitrary powers, which comes after the unit and so
% needs to be marked as such.  Two fake option-storage macros are
% created so that everything works correctly.
%    \begin{macrocode}
\newcommand*{\si@tothe}[2]{%
  \ifsi@unt@littest
    \expandafter\@gobbletwo
  \else
    \ifsi@unt@litout
      \expandafter\expandafter\expandafter\si@unt@litpower
    \else
      \expandafter\expandafter\expandafter\si@unt@power
    \fi
  \fi
  {#1}{#2}}
\newcommand*{\tothe@opt@si}{post}
\newcommand*{\raiseto@opt@si}{}
%    \end{macrocode}
%\end{macro}
%\end{macro}
%\end{macro}
%
%\subsection{Locales}
%\begin{macro}{\si@loc@load}
%\begin{macro}{\si@loc@sisetup}
%\darg{locale}
% When loading a locale, the setup is saved rather than applied.
% Anything other than simple settings should be inside
% \cs{addtolocale}, which is already defined.
%    \begin{macrocode}
\newcommand*{\si@loc@load}[1]{%
  \let\si@loc@sisetup\sisetup
  \renewcommand*{\sisetup}[1]{%
    \expandafter\gdef\csname si@loc@#1\endcsname{##1}}%
  \si@loadfile{#1}%
  \let\sisetup\si@loc@sisetup}
%    \end{macrocode}
%\end{macro}
%\end{macro}
%\begin{macro}{\si@loc@set}
%\darg{locale}
% Setting the locale transfers the settings to \cs{sisetup}, and runs
% any extra macros.
%    \begin{macrocode}
\newcommand*{\si@loc@set}[1]{%
  \ifcsname si@loc@#1\endcsname
    \si@log@inf{Setting locale to `#1'}%
    \expandafter\expandafter\expandafter\expandafter
      \expandafter\expandafter\expandafter\si@temptoks
      \expandafter\expandafter\expandafter\expandafter
      \expandafter\expandafter\expandafter{%
        \expandafter\csname si@loc@#1\endcsname}%
    \expandafter\sisetup\expandafter{\the\si@temptoks}%
    \ifcsname si@loc@#1@extra\endcsname
      \csname si@loc@#1@extra\endcsname
    \fi
  \else
    \ifcsname si@loc@#1@extra\endcsname
      \si@log@inf{Setting locale to `#1'}%
      \csname si@loc@#1@extra\endcsname
    \else
      \si@log@warn{Unknown locale `#1'}%
    \fi
  \fi}
%    \end{macrocode}
%\end{macro}
%\begin{macro}{\si@loc@ltol}
%\changes{v1.0a}{2008/06/20}{\pkg{babel} support fixed for default
%  document language}
%\darg{list of locales and languages}
% The necessary loading for language modules occurs.
%    \begin{macrocode}
\newcommand*{\si@loc@ltol}[1]{%
  \def\si@tempa##1:##2\@nil{\si@loc@load{##1}}
  \@for\si@tempb:=#1\do{%
    \expandafter\si@tempa\si@tempb:\@nil}
  \AtBeginDocument{
    \@ifpackageloaded{babel}
      {\def\si@tempa##1:##2:##3\@nil{%
        \expandafter\addto\expandafter{%
          \csname extras##2\endcsname}%
          {\si@loc@set{##1}}}%
       \@for\si@tempb:=#1\do{%
         \expandafter\si@tempa\si@tempb::\@nil}%
       \expandafter\selectlanguage\expandafter{\languagename}}
      {\si@log@warn{babel not loaded \MessageBreak
        loctolang option ignored}}}}
\AtBeginDocument{
  \@ifpackageloaded{babel}
    {\renewcommand*{\si@loc@ltol}[1]{%
       \def\si@tempa##1:##2\@nil{\si@loc@load{##1}}%
       \@for\si@tempb:=#1\do{%
         \expandafter\si@tempa\si@tempb:\@nil}%
       \def\si@tempa##1:##2:##3\@nil{%
         \expandafter\addto\expandafter{%
           \csname extras##2\endcsname}%
           {\si@loc@set{##1}}}%
       \@for\si@tempb:=#1\do{%
         \expandafter\si@tempa\si@tempb::\@nil}}}
    {\renewcommand*{\si@loc@ltol}[1]{%
       \si@log@warn{babel not loaded \MessageBreak
         loctolang option ignored}}}}
%    \end{macrocode}
%\end{macro}
%\begin{macro}{\addtolocale}
%\darg{locale}
%\darg{commands}
% Arbitrary macros may need to be added to the locale.
%    \begin{macrocode}
\newcommand*{\addtolocale}[2]{%
  \si@addtocsname{si@loc@#1@extra}{#2}}
%    \end{macrocode}
%\end{macro}
%
%\subsection{Output routine}
%\begin{macro}{\si@out}
%\changes{v1.1}{2008/07/22}{Extended handling to subscripts}
%\darg{text}
% With all of the setup done, the text can finally be typeset.  This
% is done inside a \cs{text} block, which takes care of
% \cs{ensuremath}, \etc.  First of all, the various catcode settings
% needed to get maths-in-text mode are made.  \cs{makeatletter} is
% needed so that \cs{scantokens} still allows internal macros to
% work.
%    \begin{macrocode}
\begingroup
  \catcode`\^=\active
  \catcode`\-=\active
  \catcode`\_=\active
  \gdef\si@out#1{%
    \begingroup
      \catcode`\^=\active
      \catcode`\_=\active
      \makeatletter
      \endlinechar\m@ne
%    \end{macrocode}
%\changes{v1.0f}{2008/07/17}{Fixed bug with colour and spacing}
% The various font families can now be set up.  First a check is made
% in case there are nested calls to \cs{si@out@text}.
%    \begin{macrocode}
      \ifsi@fam@set\else
        \expandafter\si@fam@set
      \fi
      \text{%
        \si@colourcmd{\si@colour}%
        \si@fam@italic\si@fam@bold\si@fam@text
%    \end{macrocode}
%  The correct mode is selected, and the input is handed off for
%  typesetting.
%    \begin{macrocode}
        \ifsi@textmode
          \expandafter\si@out@text
        \else
          \expandafter\si@out@maths
        \fi
        {\scantokens{#1}}}%
    \endgroup
    \check@mathfonts}
%    \end{macrocode}
%\end{macro}
%\begin{macro}{\si@out@text}
%\begin{macro}{\si@out@maths}
%\darg{text}
% Output occurs with the correct changes to superscript behaviour.
%    \begin{macrocode}
  \gdef\si@out@text#1{%
    \let^\si@out@sp
    \let_\si@out@sb
    \let\textsuperscript\si@out@sp
    \let\textsubscript\si@out@sb
    \catcode`\-=\active\relax
    \let-\si@out@minus
    #1}
  \gdef\si@out@maths#1{%
    \let^\sp
    \let_\sb
    \let\textsuperscript\sp
    \let\textsubscript\sb
    $\si@fam@maths{#1}$}
\endgroup
%    \end{macrocode}
%\end{macro}
%\end{macro}
%\begin{macro}{\si@out@sp}
%\begin{macro}{\si@out@sb}
%\darg{text}
% \cs{textsuperscript} gives slightly different alignment of numbers
% to using |^| in text mode.  To avoid this, a slightly different
% definition is used.  Elsewhere \cs{textsuperscript} is used, as the
% code above sorts out the text/maths mode issues. The same is done
% for subscripts.
%    \begin{macrocode}
\newcommand*{\si@out@sp}[1]{\ensuremath{^{\text{#1}}}}
\newcommand*{\si@out@sb}[1]{\ensuremath{_{\text{#1}}}}
%    \end{macrocode}
%\end{macro}
%\end{macro}
%\begin{macro}{\si@out@minus}
%\changes{v1.1j}{2008/12/22}{Fixed infinite loop with \pkg{breqn}}
% An active minus sign is needed.
%    \begin{macrocode}
\newcommand*{\si@out@minus}{}
\DeclareTextSymbolDefault{\si@out@minus}{TS1}
\DeclareTextSymbol{\si@out@minus}{TS1}{61}
%    \end{macrocode}
%\end{macro}
%\begin{macro}{\ifsi@out@num}
% A flag is needed to control output settings.  This will be
% false unless inside \cs{si@out@num}.
%    \begin{macrocode}
\newif\ifsi@out@num
%    \end{macrocode}
%\end{macro}
%\begin{macro}{\si@out@num}
%\darg{number}
% For numerical output, the default fonts are controlled slightly
% differently to text output.
%    \begin{macrocode}
\newcommand*{\si@out@num}[1]{%
  \begingroup
    \si@out@numtrue
    \si@out{#1}%
  \endgroup}
%    \end{macrocode}
%\end{macro}
%
%\subsection{Finalisation}
%\begin{macro}{\si@extension}
%\begin{macro}{\si@fileprefix}
% To keep the code easy to maintain, the reusable filename components
% are macros rather than literals.
%    \begin{macrocode}
\newcommand*{\si@extension}{cfg}
\newcommand*{\si@fileprefix}{si-}
%    \end{macrocode}
%\end{macro}
%\end{macro}
%\begin{macro}{\si@ifl@aded}
%\begin{macro}{\si@ifloaded}
%\darg{package}
% A bit of borrowing from the \LaTeX\ kernel.  A copy of
% \cs{@ifl@aded} is needed as things aren't always done in the
% preamble by \currpkg.
%    \begin{macrocode}
\newcommand*{\si@ifl@aded}{}
\let\si@ifl@aded\@ifl@aded
\newcommand*{\si@ifloaded}[1]{%
  \si@ifl@aded\si@extension{\si@fileprefix#1}}
%    \end{macrocode}
%\end{macro}
%\end{macro}
%\begin{macro}{\si@loadfile}
%\darg{file}
% Loading configuration files is handled here.
%    \begin{macrocode}
\newcommand*{\si@loadfile}[1]{%
  \si@ifloaded{#1}{}
    {\si@InputIfFileExists{\si@fileprefix#1.\si@extension}
      {}
      {\si@log@err{Failed to load file
         \si@fileprefix#1.\si@extension}
         {The configuration file requested could not be
          found}}}}
%    \end{macrocode}
%\end{macro}
%\begin{macro}{\ifsi@outerinput}
% At various points, the package needs to know if the
% currently-read configuration file is the
% outer one.
%    \begin{macrocode}
\newif\ifsi@outerinput\si@outerinputtrue
%    \end{macrocode}
%\end{macro}
%\begin{macro}{\si@InputIfFileExists}
% To allow reading configuration files after the start of the
% document, a private copy of \cs{InputIfFileExists} is needed.  The
% main macro here simply checks if the modified version is needed.
%    \begin{macrocode}
\let\si@InputIfFileExists\InputIfFileExists
\AtBeginDocument{
  \renewcommand*{\si@InputIfFileExists}{%
    \ifsi@outerinput
      \expandafter\si@outerinput
    \else
      \expandafter\InputIfFileExists
    \fi}}
%    \end{macrocode}
%\end{macro}
%\begin{macro}{\si@outerinput}
%\darg{file}
%\darg{true-code}
%\darg{false-code}
% The outer call to load a file needs to set |@| to a letter and turn
% of printing using \cs{nullfont}.
%    \begin{macrocode}
\newcommand*{\si@outerinput}[3]{%
  \makeatletter
  \nullfont
  \si@outerinputfalse
  \IfFileExists{#1}%
    {#2\@addtofilelist{#1}\@@input \@filef@und\normalsize}
    {#3}%
  \normalsize
  \makeatother}
%    \end{macrocode}
%\end{macro}
%\begin{macro}{\requiresiconfigs}
%\darg{config-file}
% The configuration files depend on each other.
%    \begin{macrocode}
\newcommand*{\requiresiconfigs}[1]{%
  \@for\si@tempb:=#1\do{\si@loadfile{\si@tempb}}}
%    \end{macrocode}
%\end{macro}
%\begin{macro}{\si@loademfile}
%\darg{file}
% For emulation files, an additional check is made.
%    \begin{macrocode}
\newcommand*{\si@loademfile}[1]{%
  \@ifpackageloaded{#1}
    {\si@log@err{Emulation clash for package `#1'}
      {You have asked for emulation of package
       `#1'\MessageBreak
       (perhaps by giving siunitx a back-compatibility
       option)\MessageBreak but the package is already
       loaded!}}
    {\si@loadfile{#1}}}
%    \end{macrocode}
%\end{macro}
%\begin{macro}{\si@emclash}
%\darg{package}
%\darg{package}
% A macro for emulation clashes.
%    \begin{macrocode}
\newcommand*{\si@emclash}[2]{%
  \si@log@err{Emulation clash: `#1' and `#2'}
    {You have asked for emulation of package `#1'\MessageBreak
     but have already loaded emulation of `#2'}}
%    \end{macrocode}
%\end{macro}
%\begin{macro}{\si@emulating}
%\darg{package}
%\darg{version}
% For packages that are emulated, the \LaTeX\ mechanism to prevent
% re-loading is used.  The list of packages to check at the start
% of the document also has to be altered.
%    \begin{macrocode}
\newcommand*{\si@emulating}[2]{%
  \@namedef{ver@#1.sty}{#2 siunitx emulation of #1}%
  \let\si@tempa\si@blockpkgs
  \renewcommand*{\si@blockpkgs}{}%
  \@for\si@tempb:=\si@tempa\do{%
    \renewcommand*{\si@tempa}{#1}%
    \ifx\si@tempa\si@tempb\else
      \lowercase{\edef\si@tempa{#1}}%
      \lowercase{\edef\si@tempc{\si@tempb}}%
      \ifx\si@tempa\si@tempc
        \@namedef{ver@\si@tempc.sty}{#2 siunitx emulation of
          #1}%
      \else
        \si@addtolist{\si@blockpkgs}{\si@tempb}%
      \fi
    \fi}%
  \let\si@tempa\si@checkpkgs
  \renewcommand*{\si@checkpkgs}{}%
  \renewcommand*{\si@tempb}{#1}%
  \@for\si@tempc:=\si@tempa\do{%
    \ifx\si@tempb\si@tempc\else
      \si@addtolist{\si@checkpkgs}{\si@tempc}%
    \fi}}
%    \end{macrocode}
%\end{macro}
% With the \currpkg kernel macros defined, the package can now run
% through finalisation.  First, the default key values are set. The
% user options are then processed.
%    \begin{macrocode}
\sisetup{
  addsign=none,
  allowzeroexp=false,
  angformat=unchanged,
  astroang=false,%(
  closeerr=),%(
  closefrac=),
  colour=black,
  colourall=false,
  colourneg=false,
  decimalsymbol=fullstop,
  detectdisplay=true,
  digitsep=thin,
  dp=3,
  eVcorra=0.3ex,
  eVcorrb=0ex,
  errspace=none,
  fixdp=false,
  inlinebold=text,
  load=default,
  mathsrm=mathrm,
  mathssf=mathsf,
  mathstt=mathtt,
  mode=maths,
  negcolour=red,
  noload={},
  numaddn={},%(
  numcloseerr=),
  numdecimal={.,},
  numdigits=0123456789,
  numdiv=/,
  numexp=eEdD,
  numgobble={},
  numopenerr=(,%)
  numprod=x,
  numsign=+-\pm\mp,
  obeybold=false,
  obeyitalic=false,
  obeymode=false,
  openerr=(,%)
  openfrac=(,%)
  padangle=small,
  padnumber=lead,
  per=reciprocal,
  prefixbase=ten,
  prefixproduct=times,
  prefixsymbolic=true,
  prespace=false,
  redefsymbols=true,
  repeatunits=true,
  retainplus=false,
  seperr=false,
  sepfour=false,
  sign=plus,
  slash=slash,
  stickyper=false,
  strictarc=true,
  tabalignexp=true,
  tabautofit=false,
  tabformat=3.2,
  tabnumalign=centredecimal,
  tabparseonly=false,
  tabtextalign=centre,
  tabunitalign=left,
  textrm=rmfamily,
  textsf=sffamily,
  texttt=ttfamily,
  tightpm=false,
  trapambigerr=true,
  trapambigfrac=true,
  unitsep=thin,
  valuesep=thin,
  xspace=false}
\ProcessOptionsX[si]<key>
%    \end{macrocode}
% A check is now made so that emulation takes place one file at a
% time, and that each file is loaded only once.
%    \begin{macrocode}
\ifx\@empty\si@emulate\@empty\else
  \@for\si@tempa:=\si@emulate\do{%
    \si@loademfile{\si@tempa}}
\fi
%    \end{macrocode}
%\begin{macro}{\si@expanddefault}
% For turning the list of default choices into a loadable list.
%    \begin{macrocode}
\newcommand*{\si@expanddefault}[2]{%
  \expandafter\ifx\expandafter\@empty\csname si@#1\endcsname
    \@empty
  \else
    \renewcommand*{\si@tempb}{default}%
    \renewcommand*{\si@tempc}{}%
    \expandafter\@for\expandafter\si@tempa\expandafter
      :\expandafter=\csname si@#1\endcsname\do{%
      \ifx\si@tempa\si@tempb
        \si@addtolist{\si@tempc}{#2}%
      \else
        \si@addtolist{\si@tempc}{\si@tempa}%
      \fi}
    \expandafter\edef\csname si@#1\endcsname{\si@tempc}%
    \expandafter\si@addtolist\expandafter{%
      \csname si@no#1\endcsname}%
      {default}%
    \renewcommand*{\si@tempc}{}%
    \expandafter\@for\expandafter\si@tempa\expandafter
      :\expandafter=\csname si@#1\endcsname\do{%
      \si@switchfalse
      \expandafter\@for\expandafter\si@tempb\expandafter
        :\expandafter=\csname si@no#1\endcsname\do{%
        \ifx\si@tempa\si@tempb
          \si@switchtrue
        \fi
        \ifsi@switch\else
          \si@addtolist{\si@tempc}{\si@tempa}%
        \fi}}
    \@for\si@tempa:=\si@tempc\do{%
      \si@loadfile{\si@tempa}}%
  \fi}
%    \end{macrocode}
%\end{macro}
% The configuration and abbreviation files are loaded.
%    \begin{macrocode}
\si@expanddefault{load}{prefix,named,addn,prefixed,accepted,%
  physical,abbr}
%    \end{macrocode}
% The very last job is to load a local configuration file, if one
% exists, and restore catcodes.
%    \begin{macrocode}
\IfFileExists{siunitx.cfg}
  {\si@log@inf{Local configuration file found}%
   \InputIfFileExists{siunitx.cfg}{}{}}
  {}
\si@catcodes
%    \end{macrocode}
%
%\iffalse
%</package>
%\fi
%
%\section{Loadable modules}
% To keep the package relatively clear, and to make maintenance
% easier, the only units defined in the package itself are the base
% units. Everything else is a loadable module (similar to the
% approach in \unitsdef).
%
%\iffalse
%<*prefix>
%\fi
%\subsection{Multiple prefixes}
%\begin{macro}{\yocto}
%\begin{macro}{\zepto}
%\begin{macro}{\atto}
%\begin{macro}{\femto}
%\begin{macro}{\pico}
%\begin{macro}{\nano}
%\begin{macro}{\micro}
%\begin{macro}{\Micro}
%\begin{macro}{\milli}
%\begin{macro}{\centi}
%\begin{macro}{\deci}
% The various SI multiple prefixes are defined here.  First the small
% powers.
%    \begin{macrocode}
\ProvidesFile{si-prefix.cfg}
  [2008/12/22 v1.1j siunitx: SI Multiple prefixes]
\newprefix{\yocto}{-24}{y}
\newprefix{\zepto}{-21}{z}
\newprefix{\atto}{-18}{a}
\newprefix{\femto}{-15}{f}
\newprefix{\pico}{-12}{p}
\newprefix{\nano}{-9}{n}
%    \end{macrocode}
% Some testing is needed for \unitsdef compatibility.
%    \begin{macrocode}
\ifsi@old@OHM
  \newprefix{\Micro}{-6}{\si@sym@mu}
\else
  \ifsi@gensymb\else
    \newprefix{\micro}{-6}{\si@sym@mu}
  \fi
\fi
\newprefix{\milli}{-3}{m}
\newprefix{\centi}{-2}{c}
\newprefix{\deci}{-1}{d}
%    \end{macrocode}
%\end{macro}
%\end{macro}
%\end{macro}
%\end{macro}
%\end{macro}
%\end{macro}
%\end{macro}
%\end{macro}
%\end{macro}
%\end{macro}
%\end{macro}
%\begin{macro}{\deca}
%\begin{macro}{\hecto}
%\begin{macro}{\kilo}
%\begin{macro}{\mega}
%\begin{macro}{\giga}
%\begin{macro}{\tera}
%\begin{macro}{\peta}
%\begin{macro}{\exa}
%\begin{macro}{\zetta}
%\begin{macro}{\yotta}
% Now the large ones.
%    \begin{macrocode}
\newprefix{\deca}{1}{da}
\newprefix{\hecto}{2}{h}
\newprefix{\kilo}{3}{k}
\newprefix{\mega}{6}{M}
\newprefix{\giga}{9}{G}
\newprefix{\tera}{12}{T}
\newprefix{\peta}{15}{P}
\newprefix{\exa}{18}{E}
\newprefix{\zetta}{21}{Z}
\newprefix{\yotta}{24}{Y}
%    \end{macrocode}
%\end{macro}
%\end{macro}
%\end{macro}
%\end{macro}
%\end{macro}
%\end{macro}
%\end{macro}
%\end{macro}
%\end{macro}
%\end{macro}
%\begin{macro}{\deka}
% Apparently, ``deka'' is common in the US for deca.
%    \begin{macrocode}
\newprefix{\deka}{1}{da}
%    \end{macrocode}
%\end{macro}
%\begin{macro}{\gram}
%\begin{macro}{\kilogram}
% As the base unit of mass is the kilogram, rather than the gram, a
% bit of extra work is needed; by default the package only defines
% \cs{kilogram}, but with the prefixes available, this is altered to
% be \cs{kilo}\cs{gram}.  For that, the \cs{gram} must be defined first.
%    \begin{macrocode}
\newunit{\gram}{g}
\renewunit{\kilogram}{\kilo\gram}
%    \end{macrocode}
%\end{macro}
%\end{macro}
%\iffalse
%</prefix>
%<*named>
%\fi
%\subsection{Derived units with specific names}
%\begin{macro}{\becquerel}
%\begin{macro}{\coulomb}
%\begin{macro}{\farad}
%\begin{macro}{\Gray}
%\begin{macro}{\ggray}
%\begin{macro}{\hertz}
%\begin{macro}{\henry}
%\begin{macro}{\joule}
%\begin{macro}{\katal}
%\begin{macro}{\lumen}
%\begin{macro}{\lux}
%\begin{macro}{\newton}
% Derived units with specific names and symbols are defined.  Litre
% is an awkward one, but here the UK standard is used.
%    \begin{macrocode}
\ProvidesFile{si-named.cfg}
  [2008/12/22 v1.1j siunitx: SI Named units]
\newunit{\becquerel}{Bq}
\newunit{\coulomb}{C}
\newunit{\farad}{F}
\newunit{\Gray}{Gy}
\newunit{\ggray}{Gy}
\newunit{\hertz}{Hz}
\newunit{\henry}{H}
\newunit{\joule}{J}
\newunit{\katal}{kat}
\newunit{\lumen}{lm}
\newunit{\lux}{lx}
\newunit{\newton}{N}
%    \end{macrocode}
%\end{macro}
%\end{macro}
%\end{macro}
%\end{macro}
%\end{macro}
%\end{macro}
%\end{macro}
%\end{macro}
%\end{macro}
%\end{macro}
%\end{macro}
%\end{macro}
%\begin{macro}{\ohm}
%\begin{macro}{\Ohm}
%\begin{macro}{\pascal}
%\begin{macro}{\siemens}
%\begin{macro}{\sievert}
%\begin{macro}{\tesla}
%\begin{macro}{\volt}
%\begin{macro}{\watt}
%\begin{macro}{\weber}
% Some testing is needed for \unitsdef compatibility.
%    \begin{macrocode}
\ifsi@old@OHM
  \newunit{\Ohm}{\si@sym@Omega}
\else
  \ifsi@gensymb\else
%    \end{macrocode}
% To be on the safe side, use \cs{provideunit}.
%    \begin{macrocode}
    \provideunit{\ohm}{\si@sym@Omega}
  \fi
\fi
\newunit{\pascal}{Pa}
\newunit{\siemens}{S}
\newunit{\sievert}{Sv}
\newunit{\tesla}{T}
\newunit{\volt}{V}
\newunit{\watt}{W}
\newunit{\weber}{Wb}
%    \end{macrocode}
%\end{macro}
%\end{macro}
%\end{macro}
%\end{macro}
%\end{macro}
%\end{macro}
%\end{macro}
%\end{macro}
%\end{macro}
%\begin{macro}{\celsius}
%\begin{macro}{\Celsius}
% The degree celsius is a named unit.
%    \begin{macrocode}
\ifsi@old@OHM
  \newunit{\Celsius}{\si@sym@celsius}
\else
  \ifsi@gensymb\else
    \newunit{\celsius}{\si@sym@celsius}
  \fi
\fi
%    \end{macrocode}
%\end{macro}
%\end{macro}
%\begin{macro}{\radian}
%\begin{macro}{\steradian}
% The radian and steradian are officially derived units.
%    \begin{macrocode}
\newunit{\radian}{rad}
\newunit{\steradian}{sr}
%    \end{macrocode}
%\end{macro}
%\end{macro}
%\iffalse
%</named>
%<*prefixed>
%\fi
%\subsection{Units with prefixes}
% As in \unitsdef, some commonly used prefixed units are set up.
% This requires \file{si-prefix.cfg} and \file{si-named.cfg}.
%    \begin{macrocode}
\ProvidesFile{si-prefixed.cfg}
  [2008/12/22 v1.1j siunitx: SI Prefixed units]
\requiresiconfigs{prefix,named,accepted,physical}
%    \end{macrocode}
%\begin{macro}{\picometre}
%\begin{macro}{\nanometre}
%\begin{macro}{\micrometre}
%\begin{macro}{\millimetre}
%\begin{macro}{\centimetre}
%\begin{macro}{\decimetre}
%\begin{macro}{\kilometre}
% Extra distances.
%    \begin{macrocode}
\newunit{\picometre}{\pico\metre}
\newunit{\nanometre}{\nano\metre}
\newunit{\micrometre}{\micro\metre}
\newunit{\millimetre}{\milli\metre}
\newunit{\centimetre}{\centi\metre}
\newunit{\decimetre}{\deci\metre}
\newunit{\kilometre}{\kilo\metre}
%    \end{macrocode}
%\end{macro}
%\end{macro}
%\end{macro}
%\end{macro}
%\end{macro}
%\end{macro}
%\end{macro}
%\begin{macro}{\femtogram}
%\begin{macro}{\picogram}
%\begin{macro}{\nanogram}
%\begin{macro}{\microgram}
%\begin{macro}{\milligram}
% Extra masses.
%    \begin{macrocode}
\newunit{\femtogram}{\femto\gram}
\newunit{\picogram}{\pico\gram}
\newunit{\nanogram}{\nano\gram}
\newunit{\microgram}{\micro\gram}
\newunit{\milligram}{\milli\gram}
%    \end{macrocode}
%\end{macro}
%\end{macro}
%\end{macro}
%\end{macro}
%\end{macro}
%\begin{macro}{\femtomole}
%\begin{macro}{\picomole}
%\begin{macro}{\nanomole}
%\begin{macro}{\micromole}
%\begin{macro}{\millimole}
%\begin{macro}{\kilomole}
%\changes{v1.1e}{2008/11/18}{New unit}
% Now some moles.
%    \begin{macrocode}
\newunit{\femtomole}{\femto\mole}
\newunit{\picomole}{\pico\mole}
\newunit{\nanomole}{\nano\mole}
\newunit{\micromole}{\micro\mole}
\newunit{\millimole}{\milli\mole}
\newunit{\kilomole}{\kilo\mole}
%    \end{macrocode}
%\end{macro}
%\end{macro}
%\end{macro}
%\end{macro}
%\end{macro}
%\end{macro}
%\begin{macro}{\attosecond}
%\begin{macro}{\femtosecond}
%\begin{macro}{\picosecond}
%\begin{macro}{\nanosecond}
%\begin{macro}{\microsecond}
%\begin{macro}{\millisecond}
% Prefixed seconds.
%    \begin{macrocode}
\newunit{\attosecond}{\atto\second}
\newunit{\femtosecond}{\femto\second}
\newunit{\picosecond}{\pico\second}
\newunit{\nanosecond}{\nano\second}
\newunit{\microsecond}{\micro\second}
\newunit{\millisecond}{\milli\second}
%    \end{macrocode}
%\end{macro}
%\end{macro}
%\end{macro}
%\end{macro}
%\end{macro}
%\end{macro}
%\begin{macro}{\picoampere}
%\begin{macro}{\nanoampere}
%\begin{macro}{\microampere}
%\begin{macro}{\milliampere}
%\begin{macro}{\kiloampere}
% The last prefixed base units are amperes.
%    \begin{macrocode}
\newunit{\picoampere}{\pico\ampere}
\newunit{\nanoampere}{\nano\ampere}
\newunit{\microampere}{\micro\ampere}
\newunit{\milliampere}{\milli\ampere}
\newunit{\kiloampere}{\kilo\ampere}
%    \end{macrocode}
%\end{macro}
%\end{macro}
%\end{macro}
%\end{macro}
%\end{macro}
%\begin{macro}{\millivolt}
%\begin{macro}{\kilovolt}
%\begin{macro}{\milliwatt}
%\begin{macro}{\kilowatt}
%\begin{macro}{\megawatt}
%\begin{macro}{\femtofarad}
%\begin{macro}{\picofarad}
%\begin{macro}{\nanofarad}
%\begin{macro}{\microfarad}
%\begin{macro}{\millifarad}
%\begin{macro}{\millisiemens}
% More electricity-related units.
%    \begin{macrocode}
\newunit{\millivolt}{\milli\volt}
\newunit{\kilovolt}{\kilo\volt}
\newunit{\milliwatt}{\milli\watt}
\newunit{\kilowatt}{\kilo\watt}
\newunit{\megawatt}{\mega\watt}
\newunit{\femtofarad}{\femto\farad}
\newunit{\picofarad}{\pico\farad}
\newunit{\nanofarad}{\nano\farad}
\newunit{\microfarad}{\micro\farad}
\newunit{\millifarad}{\milli\farad}
\newunit{\millisiemens}{\milli\siemens}
%    \end{macrocode}
%\end{macro}
%\end{macro}
%\end{macro}
%\end{macro}
%\end{macro}
%\end{macro}
%\end{macro}
%\end{macro}
%\end{macro}
%\end{macro}
%\end{macro}
%\begin{macro}{\kilohm}
%\changes{v1.1f}{2008/11/25}{Renamed from \cs{kiloohm}}
%\begin{macro}{\megohm}
%\changes{v1.1f}{2008/11/25}{Renamed from \cs{megaohm}}
%\begin{macro}{\gigaohm}
% For resistance, checks are needed again for the definition of
% \cs{ohm}.
%    \begin{macrocode}
\ifsi@old@OHM
  \newunit{\kilohm}{\kilo\Ohm}
  \newunit{\megohm}{\mega\Ohm}
  \newunit{\gigaohm}{\giga\Ohm}
\else
  \ifsi@gensymb\else
  \newunit{\kilohm}{\kilo\ohm}
  \newunit{\megohm}{\mega\ohm}
  \newunit{\gigaohm}{\giga\ohm}
  \fi
\fi
%    \end{macrocode}
%\end{macro}
%\end{macro}
%\end{macro}
%\begin{macro}{\microlitre}
%\begin{macro}{\millilitre}
%\begin{macro}{\microliter}
%\begin{macro}{\milliliter}
%\begin{macro}{\cubicmetre}
%\begin{macro}{\cubiccentimetre}
%\begin{macro}{\centimetrecubed}
%\begin{macro}{\cubicmicrometre}
%\begin{macro}{\cubicmillimetre}
%\begin{macro}{\cubicdecimetre}
% Volumes (unlike \unitsdef, with litre and metre spelled
% correctly).  \cs{millilitre} and \cs{microlitre} are defined
% as they are the two officially-sanctioned prefixes for the litre.
%    \begin{macrocode}
\newunit{\microlitre}{\micro\litre}
\newunit{\millilitre}{\milli\litre}
\newunit{\microliter}{\micro\liter}
\newunit{\milliliter}{\milli\liter}
\newunit{\cubicmetre}{\metre\cubed}
\newunit{\cubiccentimetre}{\centi\metre\cubed}
\newunit{\centimetrecubed}{\centi\metre\cubed}
\newunit{\cubicmicrometre}{\micro\metre\cubed}
\newunit{\cubicmillimetre}{\milli\metre\cubed}
\newunit{\cubicdecimetre}{\cubic\deci\metre}
%    \end{macrocode}
%\end{macro}
%\end{macro}
%\end{macro}
%\end{macro}
%\end{macro}
%\end{macro}
%\end{macro}
%\end{macro}
%\end{macro}
%\end{macro}
%\begin{macro}{\squaremetre}
%\begin{macro}{\squarecentimetre}
%\begin{macro}{\centimetresquared}
%\begin{macro}{\squarekilometre}
% Areas, with metre spelled corrected; \cs{are} and \cs{hectare} are
% in the ``temporarily accepted'' file.
%    \begin{macrocode}
\newunit{\squaremetre}{\Square\metre}
\newunit{\squarecentimetre}{\Square\centi\metre}
\newunit{\centimetresquared}{\centi\metre\squared}
\newunit{\squarekilometre}{\Square\kilo\metre}
%    \end{macrocode}
%\end{macro}
%\end{macro}
%\end{macro}
%\end{macro}
%\begin{macro}{\millijoule}
%\begin{macro}{\kilojoule}
%\begin{macro}{\megajoule}
%\begin{macro}{\millielectronvolt}
%\begin{macro}{\kiloelectronvolt}
%\begin{macro}{\megaelectronvolt}
%\begin{macro}{\gigaelectronvolt}
%\begin{macro}{\teraelectronvolt}
%\begin{macro}{\kilowatthour}
% Some energy is needed by now!  Notice the definition of
% \cs{kilowatthour}
%    \begin{macrocode}
\newunit{\millijoule}{\milli\joule}
\newunit{\kilojoule}{\kilo\joule}
\newunit{\megajoule}{\mega\joule}
\newunit{\millielectronvolt}{\milli\electronvolt}
\newunit{\kiloelectronvolt}{\kilo\electronvolt}
\newunit{\megaelectronvolt}{\mega\electronvolt}
\newunit{\gigaelectronvolt}{\giga\electronvolt}
\newunit{\teraelectronvolt}{\tera\electronvolt}
\newunit[unitsep=none]{\kilowatthour}{\kilo\watt\hour}
%    \end{macrocode}
%\end{macro}
%\end{macro}
%\end{macro}
%\end{macro}
%\end{macro}
%\end{macro}
%\end{macro}
%\end{macro}
%\end{macro}
%\begin{macro}{\millihertz}
%\begin{macro}{\kilohertz}
%\begin{macro}{\megahertz}
%\begin{macro}{\gigahertz}
%\begin{macro}{\terahertz}
% Frequencies.
%    \begin{macrocode}
\newunit{\millihertz}{\milli\hertz}
\newunit{\kilohertz}{\kilo\hertz}
\newunit{\megahertz}{\mega\hertz}
\newunit{\gigahertz}{\giga\hertz}
\newunit{\terahertz}{\tera\hertz}
%    \end{macrocode}
%\end{macro}
%\end{macro}
%\end{macro}
%\end{macro}
%\end{macro}
%\begin{macro}{\millinewton}
%\begin{macro}{\kilonewton}
%\begin{macro}{\hectopascal}
%\begin{macro}{\megabecquerel}
%\begin{macro}{\millisievert}
% A few more from various areas.
%    \begin{macrocode}
\newunit{\millinewton}{\milli\newton}
\newunit{\kilonewton}{\kilo\newton}
\newunit{\hectopascal}{\hecto\pascal}
\newunit{\megabecquerel}{\mega\becquerel}
\newunit{\millisievert}{\milli\sievert}
%    \end{macrocode}
%\end{macro}
%\end{macro}
%\end{macro}
%\end{macro}
%\end{macro}
%\iffalse
%</prefixed>
%<*abbr>
%\fi
%\subsection{Abbreviated units}
%\begin{macro}{\pA}
%\begin{macro}{\nA}
%\begin{macro}{\micA}
%\begin{macro}{\mA}
%\begin{macro}{\kA}
% The abbreviated units are sorted in one file.  To allow
% back-compatibility with \unitsdef, each one is inside an \cs{if}
% block for the appropriate option.  First currents.
%    \begin{macrocode}
\ProvidesFile{si-abbr.cfg}
  [2008/12/22 v1.1j siunitx: Abbreviated units]
\requiresiconfigs{prefix,named,accepted,physical}
\newunit{\pA}{\pico\ampere}
\newunit{\nA}{\nano\ampere}
\newunit{\micA}{\micro\ampere}
\newunit{\mA}{\milli\ampere}
\newunit{\kA}{\kilo\ampere}
%    \end{macrocode}
%\end{macro}
%\end{macro}
%\end{macro}
%\end{macro}
%\end{macro}
%\begin{macro}{\Hz}
%\begin{macro}{\mHz}
%\begin{macro}{\kHz}
%\begin{macro}{\MHz}
%\begin{macro}{\GHz}
%\begin{macro}{\THz}
% Then frequencies.
%    \begin{macrocode}
\newunit{\Hz}{\hertz}
\newunit{\mHz}{\milli\hertz}
\newunit{\kHz}{\kilo\hertz}
\newunit{\MHz}{\mega\hertz}
\newunit{\GHz}{\giga\hertz}
\newunit{\THz}{\tera\hertz}
%    \end{macrocode}
%\end{macro}
%\end{macro}
%\end{macro}
%\end{macro}
%\end{macro}
%\end{macro}
%\begin{macro}{\fmol}
%\begin{macro}{\pmol}
%\begin{macro}{\nmol}
%\begin{macro}{\micmol}
%\begin{macro}{\mmol}
%\begin{macro}{\kmol}
%\changes{v1.1e}{2008/11/18}{New unit}
% Amounts of substance.
%    \begin{macrocode}
\newunit{\fmol}{\femto\mole}
\newunit{\pmol}{\pico\mole}
\newunit{\nmol}{\nano\mole}
\newunit{\micmol}{\micro\mole}
\newunit{\mmol}{\milli\mole}
\newunit{\kmol}{\kilo\mole}
%    \end{macrocode}
%\end{macro}
%\end{macro}
%\end{macro}
%\end{macro}
%\end{macro}
%\end{macro}
%\begin{macro}{\kV}
%\begin{macro}{\mV}
% Potentials.
%    \begin{macrocode}
\newunit{\kV}{\kilo\volt}
\newunit{\mV}{\milli\volt}
%    \end{macrocode}
%\end{macro}
%\end{macro}
%\begin{macro}{\ml}
%\begin{macro}{\micl}
%\begin{macro}{\mL}
%\begin{macro}{\micL}
%\begin{macro}{\cmc}
%\begin{macro}{\dmc}
%\begin{macro}{\cms}
% Volumes and areas
%    \begin{macrocode}
\provideunit{\ml}{\milli\litre}
\provideunit{\micl}{\micro\litre}
\provideunit{\mL}{\milli\liter}
\provideunit{\micL}{\micro\liter}
\newunit{\cmc}{\centi\metre\cubed}
\newunit{\dmc}{\deci\metre\cubed}
\newunit{\cms}{\centi\metre\squared}
%    \end{macrocode}
%\end{macro}
%\end{macro}
%\end{macro}
%\end{macro}
%\end{macro}
%\end{macro}
%\end{macro}
%\begin{macro}{\fg}
%\begin{macro}{\SIfg}
%\begin{macro}{\kg}
%\begin{macro}{\fg}
%\begin{macro}{\pg}
%\begin{macro}{\nanog}
%\begin{macro}{\micg}
%\begin{macro}{\mg}
%\begin{macro}{\amu}
% Masses.
%    \begin{macrocode}
\newunit{\kg}{\kilo\gram}
%    \end{macrocode}
% There is a name clash with \pkg{babel} here in French; hopefully
% there will not be too many complaints.
%    \begin{macrocode}
\AtBeginDocument{\provideunit{\fg}{\femto\gram}}
\newunit{\SIfg}{\femto\gram}
\newunit{\pg}{\pico\gram}
\newunit{\nanog}{\nano\gram}
\newunit{\micg}{\micro\gram}
\newunit{\mg}{\milli\gram}
\newunit{\amu}{\atomicmass}
%    \end{macrocode}
%\end{macro}
%\end{macro}
%\end{macro}
%\end{macro}
%\end{macro}
%\end{macro}
%\end{macro}
%\end{macro}
%\end{macro}
%\begin{macro}{\kJ}
%\begin{macro}{\eV}
%\begin{macro}{\meV}
%\begin{macro}{\keV}
%\begin{macro}{\MeV}
%\begin{macro}{\GeV}
%\begin{macro}{\TeV}
%\begin{macro}{\kWh}
% Energies.
%    \begin{macrocode}
\newunit{\kJ}{\kilo\joule}
\newunit{\eV}{\electronvolt}
\newunit{\meV}{\milli\electronvolt}
\newunit{\keV}{\kilo\electronvolt}
\newunit{\MeV}{\mega\electronvolt}
\newunit{\GeV}{\giga\electronvolt}
\newunit{\TeV}{\tera\electronvolt}
\newunit[unitsep=none]{\kWh}{\kilo\watt\hour}
%    \end{macrocode}
%\end{macro}
%\end{macro}
%\end{macro}
%\end{macro}
%\end{macro}
%\end{macro}
%\end{macro}
%\end{macro}
%\begin{macro}{\picom}
%\begin{macro}{\nm}
%\begin{macro}{\micm}
%\begin{macro}{\mm}
%\begin{macro}{\cm}
%\begin{macro}{\dm}
%\begin{macro}{\km}
% Lengths.
%    \begin{macrocode}
\newunit{\picom}{\pico\metre}
\newunit{\nm}{\nano\metre}
\newunit{\micm}{\micro\metre}
\newunit{\mm}{\milli\metre}
\newunit{\cm}{\centi\metre}
\newunit{\dm}{\deci\metre}
\newunit{\km}{\kilo\metre}
%    \end{macrocode}
%\end{macro}
%\end{macro}
%\end{macro}
%\end{macro}
%\end{macro}
%\end{macro}
%\end{macro}
%\begin{macro}{\Sec}
%\begin{macro}{\as}
%\begin{macro}{\fs}
%\begin{macro}{\ps}
%\begin{macro}{\ns}
%\begin{macro}{\mics}
%\begin{macro}{\ms}
% Finally, times.
%    \begin{macrocode}
\newunit{\Sec}{\second}
\newunit{\as}{\atto\second}
\newunit{\fs}{\femto\second}
%    \end{macrocode}
% The letter class (and others) define \cs{ps} for postscripts, so
% \cs{provideunit} is best here.
%    \begin{macrocode}
\provideunit{\ps}{\pico\second}
\newunit{\ns}{\nano\second}
\newunit{\mics}{\micro\second}
\newunit{\ms}{\milli\second}
%    \end{macrocode}
%\end{macro}
%\end{macro}
%\end{macro}
%\end{macro}
%\end{macro}
%\end{macro}
%\end{macro}
%\iffalse
%</abbr>
%<*addn>
%\fi
%\subsection{Additional (temporary) SI units}
%\begin{macro}{\angstrom}
%\begin{macro}{\are}
%\begin{macro}{\hectare}
%\begin{macro}{\barn}
%\begin{macro}{\BAR}
%\begin{macro}{\bbar}
%\begin{macro}{\millibar}
%\begin{macro}{\gal}
%\begin{macro}{\curie}
%\begin{macro}{\roentgen}
%\begin{macro}{\rad}
%\begin{macro}{\rem}
%\changes{v1.1c}{2008/10/22}{Uses \cs{provideunit}}
% Some units are ``temporarily'' acceptable for use in the SI system.
% These are defined here, although some are in very general use.
%    \begin{macrocode}
\ProvidesFile{si-addn.cfg}
  [2008/12/22 v1.1j siunitx: SI Additional units]
\newunit{\angstrom}{\si@sym@ringA}
\newunit{\are}{a}
\newunit{\hectare}{\hecto\are}
\newunit{\barn}{b}
\newunit{\BAR}{bar}
\newunit{\bbar}{bar}
\newunit{\millibar}{\milli\BAR}
\newunit{\gal}{Gal}
\newunit{\curie}{Ci}
\newunit{\roentgen}{R}
\newunit{\rad}{rad}
\provideunit{\rem}{rem}
%    \end{macrocode}
%\end{macro}
%\end{macro}
%\end{macro}
%\end{macro}
%\end{macro}
%\end{macro}
%\end{macro}
%\end{macro}
%\end{macro}
%\end{macro}
%\end{macro}
%\end{macro}
%\iffalse
%</addn>
%<*accepted>
%\fi
%\subsection{Units accepted for use with SI}
% The units which are accepted but do not fit in the above, plus
% \cs{percent} which seems to fit into this category.
%\begin{macro}{\minute}
%\begin{macro}{\hour}
%\begin{macro}{\Day}
%\begin{macro}{\dday}
%\begin{macro}{\degree}
%\changes{v1.1c}{2008/10/22}{Uses \cs{provideunit}}
%\begin{macro}{\Degree}
%\begin{macro}{\arcmin}
%\begin{macro}{\arcsec}
%\begin{macro}{\litre}
%\begin{macro}{\liter}
%\begin{macro}{\tonne}
%\begin{macro}{\neper}
%\begin{macro}{\bel}
%\begin{macro}{\percent}
%    \begin{macrocode}
\ProvidesFile{si-accepted.cfg}
  [2008/12/22 v1.1j siunitx: SI Accepted units]
\newunit{\minute}{min}
\newunit{\hour}{h}
\newunit{\Day}{d}
\newunit{\dday}{d}
\ifsi@old@OHM
  \AtBeginDocument{
    \provideunit[valuesep=none]{\Degree}{\si@sym@degree}
  }
\else
  \ifsi@gensymb\else
    \AtBeginDocument{
      \provideunit[valuesep=none]{\degree}{\si@sym@degree}
    }
  \fi
\fi
\newunit[valuesep=none]{\arcmin}{\si@sym@minute}
\newunit[valuesep=none]{\arcsec}{\si@sym@second}
\newunit{\litre}{l}
\newunit{\liter}{L}
\newunit{\tonne}{t}
\newunit{\neper}{Np}
\newunit{\bel}{B}
\newunit{\percent}{\%}
%    \end{macrocode}
%\end{macro}
%\end{macro}
%\end{macro}
%\end{macro}
%\end{macro}
%\end{macro}
%\end{macro}
%\end{macro}
%\end{macro}
%\end{macro}
%\end{macro}
%\end{macro}
%\end{macro}
%\end{macro}
%\iffalse
%</accepted>
%<*physical>
%\fi
%\subsection{Units based on physical measurements}
%\begin{macro}{\si@eVspacea}
%\begin{macro}{\si@eVspaceb}
%\begin{macro}{\electronvolt}
%\begin{macro}{\atomicmassunit}
%\begin{macro}{\atomicmass}
% A few units based on physical measurements exist.  For \cs{eV}, the need
% for a negative kern does make things a bit complicated.
%    \begin{macrocode}
\ProvidesFile{si-physical.cfg}
  [2008/12/22 v1.1j siunitx: SI Physically-measured units]
\newcommand*{\si@eVspacea}{\text{\kern-\si@eVcorra}}%
\newcommand*{\si@eVspaceb}{\text{\kern-\si@eVcorrb}}%
\newunit{\electronvolt}{e\protect\si@eVspacea V\protect%
  \si@eVspaceb}
\newunit{\atomicmass}{u}
\newunit{\atomicmassunit}{u}
%    \end{macrocode}
%\end{macro}
%\end{macro}
%\end{macro}
%\end{macro}
%\end{macro}
%\iffalse
%</physical>
%\fi
%
%\section{Additional configurations}
% To provide flexibility for people in specific areas, specialised
% units can be set up.  These are then stored separately to ease use.
%
%\iffalse
%<*synchem>
%\fi
%\subsection{Synthetic chemistry}
%\begin{macro}{\mmHg}
%\begin{macro}{\molar}
%\begin{macro}{\Molar}
%\begin{macro}{\torr}
%\begin{macro}{\dalton}
% Some useful units for synthetic chemists; although \cs{mmHg} and
% \cs{Molar} are outside of the SI system, they are used a lot.
% These are set up using \cs{provideunit} as people may have their
% own definitions.
%    \begin{macrocode}
\ProvidesFile{si-synchem.cfg}
  [2008/12/22 v1.1j siunitx: Units for synthetic chemists]
\requiresiconfigs{prefix}
\newunit{\mmHg}{mmHg}
\newunit{\molar}{\mole\per\cubic\deci\metre}
\newunit{\Molar}{\textsc{m}}
\newunit{\torr}{Torr}
\newunit{\dalton}{Da}
%    \end{macrocode}
%\end{macro}
%\end{macro}
%\end{macro}
%\end{macro}
%\end{macro}
%\iffalse
%</synchem>
%<*hep>
%\fi
%\subsection{High-energy physics}
% Some units inspired by \pkg{hepunits}.
%    \begin{macrocode}
\ProvidesFile{si-hep.cfg}
  [2008/12/22 v1.1j siunitx: Units for high-energy physics]
\requiresiconfigs{prefix,named}
%    \end{macrocode}
%\begin{macro}{\micron}
%\begin{macro}{\mrad}
%\begin{macro}{\gauss}
% These specific to high-energy physics, but are not defined
% elsewhere in \currpkg.\footnote{It is not clear from
% \pkg{hepunits}, but the assumption is that \cs{mrad} represents
% millirad and not milliradian.}
%    \begin{macrocode}
\provideunit{\micron}{\micro\metre}
\newunit{\mrad}{\milli\rad}
\newunit{\gauss}{G}
%    \end{macrocode}
%\end{macro}
%\end{macro}
%\end{macro}
%\begin{macro}{\clight}
%\begin{macro}{\eVperc}
% The speed of light is used in units for the area, although of
% course it is not strictly a unit.  However, this makes it easier to
% use in other units.
%    \begin{macrocode}
\newunit{\clight}{\ensuremath{\mathnormal{c}}}
\newunit{\eVperc}{\eV\per\clight}
%    \end{macrocode}
%\end{macro}
%\end{macro}
%\begin{macro}{\nanobarn}
%\begin{macro}{\picobarn}
%\begin{macro}{\femtobarn}
%\begin{macro}{\attobarn}
%\begin{macro}{\zeptobarn}
%\begin{macro}{\yoctobarn}
% Various prefixed barns.
%    \begin{macrocode}
\newunit{\nanobarn}{\nano\barn}
\newunit{\picobarn}{\pico\barn}
\newunit{\femtobarn}{\femto\barn}
\newunit{\attobarn}{\atto\barn}
\newunit{\zeptobarn}{\zepto\barn}
\newunit{\yoctobarn}{\yocto\barn}
%    \end{macrocode}
%\end{macro}
%\end{macro}
%\end{macro}
%\end{macro}
%\end{macro}
%\end{macro}
%\begin{macro}{\nb}
%\begin{macro}{\pb}
%\begin{macro}{\fb}
%\begin{macro}{\ab}
%\begin{macro}{\zb}
%\begin{macro}{\yb}
% Abbreviations for the same, but using \cs{provideunit} to avoid any
% clashes.
%    \begin{macrocode}
\provideunit{\nb}{\nano\barn}
\provideunit{\pb}{\pico\barn}
\provideunit{\fb}{\femto\barn}
\provideunit{\ab}{\atto\barn}
\provideunit{\zb}{\zepto\barn}
\provideunit{\yb}{\yocto\barn}
%    \end{macrocode}
%\end{macro}
%\end{macro}
%\end{macro}
%\end{macro}
%\end{macro}
%\end{macro}
%\iffalse
%</hep>
%<*astro>
%\fi
%\subsection{Astronomy}
%\begin{macro}{\parsec}
%\begin{macro}{\lightyear}
% For astronomy, the \cs{parsec} unit is needed.
%    \begin{macrocode}
\ProvidesFile{si-astro.cfg}
  [2008/12/22 v1.1j siunitx: Units for astronomy]
\newunit{\parsec}{pc}
\newunit{\lightyear}{ly}
%    \end{macrocode}
%\end{macro}
%\end{macro}
%\iffalse
%</astro>
%<*geophys>
%\fi
%\subsection{Geophysics}
%\begin{macro}{\lightyear}
% The unit \cs{gon} is useful here.
%    \begin{macrocode}
\ProvidesFile{si-geophys.cfg}
  [2008/12/22 v1.1j siunitx: Units for astronomy]
\newunit{\gon}{gon}
%    \end{macrocode}
%\end{macro}
%\iffalse
%</geophys>
%<*binary>
%\fi
%\subsection{Binary units}
%\begin{macro}{\kibi}
%\begin{macro}{\mebi}
%\begin{macro}{\gibi}
%\begin{macro}{\tebi}
%\begin{macro}{\pebi}
%\begin{macro}{\exbi}
%\begin{macro}{\zebi}
%\begin{macro}{\yobi}
% The binary units, as specified by the IEC and made available by
% \SIunits. First, the binary prefixes.
%    \begin{macrocode}
\ProvidesFile{si-binary.cfg}
  [2008/12/22 v1.1j siunitx: Binary units]
\newprefix[binary]{\kibi}{10}{Ki}
\newprefix[binary]{\mebi}{20}{Mi}
\newprefix[binary]{\gibi}{30}{Gi}
\newprefix[binary]{\tebi}{40}{Ti}
\newprefix[binary]{\pebi}{50}{Pi}
\newprefix[binary]{\exbi}{60}{Ei}
\newprefix[binary]{\zebi}{70}{Zi}
\newprefix[binary]{\yobi}{80}{Yi}
%    \end{macrocode}
%\end{macro}
%\end{macro}
%\end{macro}
%\end{macro}
%\end{macro}
%\end{macro}
%\end{macro}
%\end{macro}
%\begin{macro}{\bit}
%\begin{macro}{\byte}
% Now the units.
%    \begin{macrocode}
\newunit{\bit}{bit}
\newunit{\byte}{B}
%    \end{macrocode}
%\end{macro}
%\end{macro}
%
%\iffalse
%</binary>
%\fi
%
%\section{Loadable locales}
% Some short files to provide the correct settings for various places.
%
%\iffalse
%<*UK>
%\fi
%\subsection{United Kingdom}
% This is identical to the USA locale, and contains the package
% default values.\footnote{The package author is in the UK.}
%    \begin{macrocode}
\ProvidesFile{si-UK.cfg}
  [2008/12/22 v1.1j siunitx: UK locale]
\sisetup{
  unitsep=thin,
  expproduct=times,
  valuesep=thin,
  decimalsymbol=fullstop,
  digitsep=thin,
  sepfour=false}
%    \end{macrocode}
%\iffalse
%</UK>
%<*USA>
%\fi
%\subsection{United States}
% The same as for the UK.
%    \begin{macrocode}
\ProvidesFile{si-USA.cfg}
  [2008/12/22 v1.1j siunitx: USA locale]
\sisetup{
  unitsep=thin,
  expproduct=times,
  valuesep=thin,
  decimalsymbol=fullstop,
  digitsep=thin,
  sepfour=false}
%    \end{macrocode}
%\iffalse
%</USA>
%<*DE>
%\fi
%\subsection{Germany}
% Germany, hopefully.
%    \begin{macrocode}
\ProvidesFile{si-DE.cfg}
  [2008/12/22 v1.1j siunitx: Germany locale]
\sisetup{
  unitsep=cdot,
  valuesep=thin,
  decimalsymbol=comma,
  expproduct=cdot,
  digitsep=thin,
  sepfour=false}
%    \end{macrocode}
%\iffalse
%</DE>
%<*ZA>
%\fi
%\subsection{South Africa}
% Design decisions taken from the same section of \SIstyle.
%    \begin{macrocode}
\ProvidesFile{si-ZA.cfg}
  [2008/12/22 v1.1j siunitx: South Africa locale]
\sisetup{
  unitsep=cdot,
  valuesep=thin,
  expproduct=times,
  decimalsymbol=comma,
  digitsep=thin,
  sepfour=false}
%    \end{macrocode}
%\iffalse
%</ZA>
%\fi
%
%\section{Emulation code}
% Each emulation mode loads an appropriate definition file.  This
% then alters the package defaults, and defines new macros provided
% by the emulated package.
%
%\iffalse
%<*units>
%\fi
%\subsection{\units}
% The very first thing to do here is a reload check, as things
% could go wrong with \unitsdef emulation.
%    \begin{macrocode}
\si@ifloaded{units}{\endinput}{}
%    \end{macrocode}
% The \units package is quite easy to emulate, as it only has a
% few options and macros. There is also no error checking in
% \units for conflicting options, so users probably expect none.
%    \begin{macrocode}
\ProvidesFile{si-units.cfg}
  [2008/12/22 v1.1j siunitx: Emulation of units]
\si@emulating{units}{1998/08/04 v0.9b}
\si@ifloaded{SIunits}
  {\si@emclash{units}{SIunits}\endinput}{}
\si@ifloaded{sistyle}
  {\si@emclash{units}{sistyle}\endinput}{}
%    \end{macrocode}
% To emulate \units, \cs{per} must give fractions.
%    \begin{macrocode}
\sisetup{per=fraction,fraction=nice,obeybold,inlinebold=maths,
  ,obeymode}
\ifsi@old@tight
  \sisetup{valuesep=thin}
\fi
\ifsi@old@loose
  \sisetup{valuesep=space}
\fi
\ifsi@old@ugly
  \sisetup{fraction=ugly}
\fi
%    \end{macrocode}
%\begin{macro}{\unit}
%\doarg{number}
%\darg{unit}
% The \units version of \cs{unit} is similar to \cs{SI}.  Here and
% in \cs{unitfrac} the \cs{num} macro is used; thus the number given
% really has to be a number.  However, if users are using \currpkg
% rather than \units they should expect more checking of input.  As
% the \units package uses the current mode, this has to be
% detected.
%    \begin{macrocode}
\si@newrobustcmd*{\unit}[2][]{%
  \ifmmode
    \SI{#1}{#2}%
  \else
    \SI[obeyfamily,obeyitalic]{#1}{#2}%
  \fi}
%    \end{macrocode}
%\end{macro}
%\begin{macro}{\unitfrac}
%\doarg{number}
%\darg{numerator}
%\darg{denominator}
% \cs{unitfrac} is a bit more of a hack.
%    \begin{macrocode}
\si@newrobustcmd*{\unitfrac}[3][]{%
  \begingroup
    \si@fam@mode%
    \ifmmode\else
      \sisetup{obeyfamily,obeyitalic}%
    \fi
    \si@ifnotmtarg{#1}
      {\num{#1}\ensuremath{\si@valuesep}}%
    \si@frac{#2}{#3}
  \endgroup}
%    \end{macrocode}
%\end{macro}
%\iffalse
%</units>
%\fi
%\iffalse
%<*unitsdef>
%\fi
%\subsection{\unitsdef}
% The package begins with the usual identification of what is
% happening.  Although \file{si-units.cfg} makes the same checks,
% the error will make more sense if it comes here, in the event of a
% clash.
%    \begin{macrocode}
\ProvidesFile{si-unitsdef.cfg}
  [2008/12/22 v1.1j siunitx:
   Emulation of unitsdef]
\si@emulating{unitsdef}{2005/01/04 v0.2}
\si@ifloaded{SIunits}
  {\si@emclash{unitsdef}{SIunits}\endinput}{}
\si@ifloaded{sistyle}
  {\si@emclash{unitsdef}{sistyle}\endinput}{}
%    \end{macrocode}
% Emulation of \units is needed for \unitsdef to work.
%    \begin{macrocode}
\requiresiconfigs{units}
%    \end{macrocode}
% The \unitsdef package loads some packages that \currpkg does not.
% In particular, it loads \pkg{textcomp} and \pkg{fontenc}.  This
% could be important for output, and so the same is done here.
%    \begin{macrocode}
\RequirePackage{textcomp}
\RequirePackage[T1]{fontenc}
%    \end{macrocode}
% The multitude of package options for \unitsdef need to be handled.
%    \begin{macrocode}
\sisetup{mode=text,allowoptarg,prespace}
\ifsi@old@noxspace
  \sisetup{xspace=false}
\fi
%    \end{macrocode}
% The various options for loading unit abbreviations have to be
% handled. Here, any request to avoid abbreviations prevents any
% loading.
%    \begin{macrocode}
\ifsi@old@noabbr
  \sisetup{noload=abbr}
\fi
\ifsi@old@nofrequncyabbr
  \sisetup{noload=abbr}
\fi
\ifsi@old@nomolabbr
  \sisetup{noload=abbr}
\fi
\ifsi@old@novoltageabbr
  \sisetup{noload=abbr}
\fi
\ifsi@old@novolumeabbr
  \sisetup{noload=abbr}
\fi
\ifsi@old@noweightabbr
  \sisetup{noload=abbr}
\fi
\ifsi@old@noenergyabbr
  \sisetup{noload=abbr}
\fi
\ifsi@old@nolengthabbr
  \sisetup{noload=abbr}
\fi
\ifsi@old@notimeabbr
  \sisetup{noload=abbr}
\fi
%    \end{macrocode}
%\begin{macro}{\unitvaluesep}
% To emulate the \cs{unitvaluesep} macro, a hack is needed of the original
% \pkg{xkeyval} macro for \opt{valuesep}, as well of course as a definition
% of the macro itself.
%    \begin{macrocode}
\newcommand*{\unitvaluesep}{\,}
\renewcommand*{\si@valuesep}{\text{\unitvaluesep}}
\define@choicekey*+[si]{key}{valuesep}[\si@tempa]
  {space,thin,med,medium,thick,none}
  {\renewcommand*\unitvaluesep\@nameuse{si@fix@##1}%
   \si@log@debug{Option valuesep set to ##1}}
  {\si@log@debug{Option valuesep set to ##1}%
   \renewcommand*\unitvaluesep{##1}}
%    \end{macrocode}
%\end{macro}
%\begin{macro}{\unitsignonly}
%\begin{macro}{\ilu}
%\begin{macro}{\arc}
% Some rather straight-forward definitions, with just a bit of fun to
% get the spacing correct.
%    \begin{macrocode}
\let\unitsignonly\si
\si@newrobustcmd*{\ilu}[2][]{%
  \begingroup
    #1\unitvaluesep%
    \unit{#2}%
  \endgroup}
\let\arc\ang
%    \end{macrocode}
%\end{macro}
%\end{macro}
%\end{macro}
%\begin{macro}{\unitSIdef}
%\begin{macro}{\si@unitSIdef}
% The \unitsdef package uses a different approach to setting the font
% inside its version of \cs{SI}.  The problem is the same as for
% \cs{unitvaluesep}, but with the added problem that \currpkg uses
% \cs{csname} \ldots \cs{endcsname}.
%    \begin{macrocode}
\newcommand*{\unitSIdef}{\upshape}
\newcommand*{\si@unitSIdef}{\unitSIdef\selectfont}
\sisetup{textrm=si@unitSIdef}
%    \end{macrocode}
%\end{macro}
%\end{macro}
%\begin{macro}{\per}
% Rather awkwardly, \unitsdef uses \cs{per} in a different way to
% \currpkg.
%    \begin{macrocode}
\let\per\relax
\si@newrobustcmd*{\per}[2]{%
  \begingroup
    \si@xspacefalse
    \renewcommand*{\unitvaluesep}{}%
    \unitfrac{#1}{#2}%
  \endgroup}
%    \end{macrocode}
%\end{macro}
%\begin{macro}{\unittimes}
%\begin{macro}{\unitsep}
%\begin{macro}{\unitsuperscript}
% Some pretty straight-forward stuff again; notice that the automatic
% analyser for units has to be turned off for this to work.
%    \begin{macrocode}
\newcommand*{\unittimes}{\ensuremath{\cdot}}
\newcommand*{\unitsep}{\,}
\renewcommand*{\si@unt@unithook}{\si@unt@litouttrue}
\sisetup{unitsep=none}
\newcommand*{\unitsuperscript}{\tothe}
%    \end{macrocode}
%\end{macro}
%\end{macro}
%\end{macro}
%\begin{macro}{\newnosepunit}
%\begin{macro}{\renewnosepunit}
% Simple aliases.
%    \begin{macrocode}
\newcommand*{\newnosepunit}{\newunit[valuesep=none]}
\newcommand*{\renewnosepunit}{\renewunit[valuesep=none]}
%    \end{macrocode}
%\end{macro}
%\end{macro}
%\begin{macro}{\setTextOmega}
%\begin{macro}{\setMathOmega}
%\begin{macro}{\setTextmu}
%\begin{macro}{\setMathmu}
%\begin{macro}{\setTextCelsius}
%\begin{macro}{\setMathCelsius}
%\begin{macro}{\setMathDegree}
%\begin{macro}{\setTextDegree}
% Controlling symbols is a simple translation job; as only one
% setting is used by \currpkg in text mode, a bit of extra work is
% needed.
%    \begin{macrocode}
\newcommand*{\setTextOmega}[2]{%
  \renewcommand*{\si@textOmega}{%
      \begingroup
      \edef\si@tempa{\sfdefault}%
      \ifx\f@family\si@tempa
        \expandafter#2%
      \else
        \expandafter#1%
      \fi
    \endgroup}}
\newcommand*{\setMathOmega}[1]{\sisetup{mathsOmega=#1}}
\newcommand*{\setTextmu}[2]{%
  \renewcommand*{\si@textmu}{%
    \begingroup
      \edef\si@tempa{\sfdefault}%
      \ifx\f@family\si@tempa
        \expandafter#2%
      \else
        \expandafter#1%
      \fi
    \endgroup}}
\newcommand*{\setMathmu}[1]{\sisetup{mathsmu=#1}}
\newcommand*{\setTextCelsius}[2]{%
  \renewcommand*{\si@textcelsius}{%
    \begingroup
      \edef\si@tempa{\sfdefault}%
      \ifx\f@family\si@tempa
        \expandafter#2%
      \else
        \expandafter#1%
      \fi
    \endgroup}}
\newcommand*{\setMathCelsius}[1]{\sisetup{mathscelsius=#1}}
\newcommand*{\setMathDegree}[2]{%
  \renewcommand*{\si@textdegree}{%
    \begingroup%
      \edef\si@tempa{\sfdefault}%
      \ifx\f@family\si@tempa
        \expandafter#2%
      \else
        \expandafter#1%
      \fi
    \endgroup}}
\newcommand*{\setTextDegree}[1]{\sisetup{textdegree=#1}}
%    \end{macrocode}
%\end{macro}
%\end{macro}
%\end{macro}
%\end{macro}
%\end{macro}
%\end{macro}
%\end{macro}
%\end{macro}
% The \opt{ohm} and \opt{OHM} options are checked, and some sanity is
% ensured.  This needs to happen before loading the configuration
% files.
%    \begin{macrocode}
\ifsi@old@OHM
  \ifsi@old@ohm
    \si@log@inf{Both `ohm' and `OHM' options given\MessageBreak
      Using default behaviour for unitsdef}
    \expandafter\expandafter\expandafter\si@old@OHMfalse
  \fi
\fi
%    \end{macrocode}
%\begin{macro}{\liter}
%\begin{macro}{\ton}
%\begin{macro}{\days}
% Tonne is spelled as ``ton'' by \unitsdef, which is wrong in the UK
% at least (\SI{1}{ton} $=$ \SI{40}{cwt} $=$ \SI{2240}{lb}!).
%    \begin{macrocode}
\ifsi@old@liter
  \ifsi@old@LITER
    \si@log@inf{Both `liter' and `LITER' options
      given\MessageBreak Using default behaviour for unitsdef}
  \else
    \renewunit{\liter}{l}
  \fi
\fi
\newunit{\ton}{t}
\newunit{\days}{d}
%    \end{macrocode}
%\end{macro}
%\end{macro}
%\end{macro}
%\begin{macro}{\picometer}
%\begin{macro}{\nanometer}
%\begin{macro}{\micrometer}
%\begin{macro}{\millimeter}
%\begin{macro}{\centimeter}
%\begin{macro}{\decimeter}
%\begin{macro}{\kilometer}
% Extra distances.
%    \begin{macrocode}
\newunit{\picometer}{\pico\meter}
\newunit{\nanometer}{\nano\meter}
\newunit{\micrometer}{\micro\meter}
\newunit{\millimeter}{\milli\meter}
\newunit{\centimeter}{\centi\meter}
\newunit{\decimeter}{\deci\meter}
\newunit{\kilometer}{\kilo\meter}
%    \end{macrocode}
%\end{macro}
%\end{macro}
%\end{macro}
%\end{macro}
%\end{macro}
%\end{macro}
%\end{macro}
%\begin{macro}{\femtoliter}
%\begin{macro}{\picoliter}
%\begin{macro}{\nanoliter}
%\begin{macro}{\microliter}
%\begin{macro}{\milliliter}
%\begin{macro}{\centiliter}
%\begin{macro}{\deciliter}
%\begin{macro}{\hectoliter}
%\begin{macro}{\cubicmeter}
%\begin{macro}{\cubicmicrometer}
%\begin{macro}{\cubicmillimeter}
% Volumes with US spellings.
%    \begin{macrocode}
\newunit{\femtoliter}{\femto\liter}
\newunit{\picoliter}{\pico\liter}
\newunit{\nanoliter}{\nano\liter}
\newunit{\microliter}{\micro\liter}
\newunit{\milliliter}{\milli\liter}
\newunit{\centiliter}{\centi\liter}
\newunit{\deciliter}{\deci\liter}
\newunit{\hectoliter}{\hecto\liter}
\newunit{\cubicmeter}{\meter\cubed}
\newunit{\cubicmicrometer}{\micro\meter\cubed}
\newunit{\cubicmillimeter}{\milli\meter\cubed}
%    \end{macrocode}
%\end{macro}
%\end{macro}
%\end{macro}
%\end{macro}
%\end{macro}
%\end{macro}
%\end{macro}
%\end{macro}
%\end{macro}
%\end{macro}
%\end{macro}
%\begin{macro}{\squaremeter}
%\begin{macro}{\squarecentimeter}
%\begin{macro}{\squarekilometer}
%\begin{macro}{\ar}
%\begin{macro}{\hectar}
% Areas, including the mis-spellings for \cs{are} and
% \cs{hectare}.
%    \begin{macrocode}
\newunit{\squaremeter}{\Square\meter}
\newunit{\squarecentimeter}{\Square\centi\meter}
\newunit{\squarekilometer}{\Square\kilo\meter}
\newunit{\ar}{a}
\newunit{\hectar}{\hecto\ar}
%    \end{macrocode}
%\end{macro}
%\end{macro}
%\end{macro}
%\end{macro}
%\end{macro}
%\begin{macro}{\kv}
%\begin{macro}{\mv}
% The code for \unitsdef has the capitalisation wrong for \cs{kV} and
% \cs{mV}.
%    \begin{macrocode}
\ifsi@old@noabbr
\else
  \ifsi@old@novoltageabbr\else
    \newunit{\kv}{\kilo\volt}
    \newunit{\mv}{\milli\volt}
  \fi
\fi
%    \end{macrocode}
%\end{macro}
%\end{macro}
%\begin{macro}{\sek}
%\begin{macro}{\fg}
%\begin{macro}{\fl}
%\begin{macro}{\pl}
%\begin{macro}{\nl}
%\begin{macro}{\micl}
%\begin{macro}{\ml}
%\begin{macro}{\cl}
%\begin{macro}{\dl}
%\begin{macro}{\hl}
% There are some slightly different abbreviations, plus some which
% are not officially allowed.
%    \begin{macrocode}
\ifsi@old@noabbr\else
  \ifsi@old@notimeabbr\else
    \newunit{\sek}{\second}
  \fi
  \ifsi@old@noweightabbr\else
    \newunit{\fg}{\femto\gram}
  \fi
  \ifsi@old@novolumeabbr\else
    \newunit{\fl}{\femto\liter}
    \newunit{\pl}{\pico\liter}
    \newunit{\nl}{\nano\liter}
    \newunit{\micl}{\micro\liter}
    \newunit{\ml}{\milli\liter}
    \newunit{\cl}{\centi\liter}
    \newunit{\dl}{\deci\liter}
    \newunit{\hl}{\hecto\liter}
  \fi
\fi
%    \end{macrocode}
%\end{macro}
%\end{macro}
%\end{macro}
%\end{macro}
%\end{macro}
%\end{macro}
%\end{macro}
%\end{macro}
%\end{macro}
%\end{macro}
%\begin{macro}{\calory}
%\begin{macro}{\kilocalory}
% \unitsdef spells calorie incorrectly, and it is also not an SI unit.
%    \begin{macrocode}
\newunit{\calory}{cal}
\newunit{\kilocalory}{\kilo\calory}
%    \end{macrocode}
%\end{macro}
%\end{macro}
%\begin{macro}{\uBar}
% \unitsdef uses \cs{ubar} for bar.
%    \begin{macrocode}
\newunit{\uBar}{ba}
%    \end{macrocode}
%\end{macro}
%\begin{macro}{\gensymbohm}
%\begin{macro}{\gensymbcelsius}
%\begin{macro}{\gensymbmicro}
%\begin{macro}{\gensymbdegree}
% If the options relating to \pkg{gensymb} are given, then the
% package \emph{has} to be loaded.  The definitions are then renamed;
% a slight awkward feature is that the hyphen character needs to be a
% letter.  To avoid needing to worry about this again, a second
% switch is set up.
%    \begin{macrocode}
\catcode`\-=11\relax
\ifsi@old@redef-gensymb
  \expandafter\si@gensymbtrue
\fi
\catcode`\-=12\relax
\ifsi@gensymb
  \RequirePackage{gensymb}
  \AtBeginDocument{
    \let\gensymbohm\ohm
    \let\gensymbcelsius\celsius
    \let\gensymbmicro\micro
    \let\gensymbdegree\degree
    \let\ohm\@undefined
    \let\celsius\@undefined
    \let\micro\@undefined
    \let\degree\@undefined
    \ifsi@old@OHM\else
      \newunit{\ohm}{\si@sym@Omega}
      \newunit{\celsius}{\si@sym@celsius}
      \newprefix{\micro}{\si@sym@mu}{-6}
      \newunit{\degree}{\si@sym@degree}
    \fi}
\fi
%    \end{macrocode}
%\end{macro}
%\end{macro}
%\end{macro}
%\end{macro}
% The configuration files can now be loaded.
%    \begin{macrocode}
\requiresiconfigs{prefix,named,addn,accepted}
%    \end{macrocode}
% The \opt{noconfig} option could be ignored, but it costs little to
% let is be used.
%    \begin{macrocode}
\ifsi@old@noconfig\else
  \InputIfFileExists{unitsdef.cfg}
    {\si@log@inf{unitsdef config file loaded}}
    {\si@log@inf{unitsdef config file not found}}
\fi
%    \end{macrocode}
%
%\iffalse
%</unitsdef>
%<*sistyle>
%\fi
%\subsection{\SIstyle}
% After setting the necessary defaults, the emulation code defines
% the macros in \SIstyle as given in the manual for that package.
%    \begin{macrocode}
\ProvidesFile{si-sistyle.cfg}
  [2008/12/22 v1.1j siunitx: Emulation of
     SIstyle]
\si@emulating{sistyle}{2006/12/20 v2.3}
\sisetup{%
  sepfour=true,
  obeyfamily,
  obeyitalic=true,
  numsign=+-,
  strictarc=false,
  unitsep=cdot}
%    \end{macrocode}
%\begin{macro}{\SIobeyboldtrue}
%\begin{macro}{\SIobeyboldfalse}
% Some simple switches, but not using \cs{newif}.
%    \begin{macrocode}
\newcommand*{\SIobeyboldtrue}{\sisetup{obeybold=true}}
\newcommand*{\SIobeyboldfalse}{\sisetup{obeybold=false}}
%    \end{macrocode}
%\end{macro}
%\end{macro}
%\begin{macro}{\num}
%\begin{macro}{\si@sis@num}
%\begin{macro}{\si@sis@numstar}
% To get the correct behaviour for \cs{num}, some redefinitions are
% needed to handle to optional |*|.
%    \begin{macrocode}
\let\num\relax
\si@newrobustcmd*{\num}{%
  \@ifstar
    {\si@sis@numstar}
    {\si@sis@num}}
\newcommand*{\si@sis@num}[2][]{%
  \begingroup%
    \sisetup{#1}%
    \expandafter\si@out@num\expandafter{\si@num{#2}}%
  \endgroup}
\newcommand*{\si@sis@numstar}[2][]{%
  \begingroup%
    \sisetup{mode=text,obeybold}%
    \sisetup{#1}%
    \expandafter\si@out@num\expandafter{\si@num{#2}}%
  \endgroup}
%    \end{macrocode}
%\end{macro}
%\end{macro}
%\end{macro}
%\begin{macro}{\pnt}
% The \cs{pnt} macro is needed as |.| is active inside \cs{SI}.  The
% name is exactly the same as in \SIstyle, but the implementation is
% different.  This is not defined by the main package as there are
% better ways of including numbers in the output than this.
%    \begin{macrocode}
\newcommand*{\pnt}{\ensuremath{\si@decimalsymbol}}
%    \end{macrocode}
%\end{macro}
%\begin{macro}{\SIgroupfourtrue}
%\begin{macro}{\SIgroupfourfalse}
% Switches for grouping four characters.
%    \begin{macrocode}
\newcommand*{\SIgroupfourtrue}{\sisetup{sepfour=true}}
\newcommand*{\SIgroupfourfalse}{\sisetup{sepfour=false}}
%    \end{macrocode}
%\end{macro}
%\end{macro}
%\begin{macro}{\SIunitsep}
%\begin{macro}{\SIunitspace}
%\begin{macro}{\SIunitdot}
% Whatever is given here is passed through to \cs{sisetup}.
%    \begin{macrocode}
\newcommand*{\SIunitsep}[1]{\sisetup{valuesep={#1}}}
\newcommand*{\SIunitspace}[1]{\sisetup{unitspace={#1}}}
\newcommand*{\SIunitdot}[1]{\sisetup{unitsep={#1}}}
%    \end{macrocode}
%\end{macro}
%\end{macro}
%\end{macro}
%\begin{macro}{\SIdecimalsymbol}
%\begin{macro}{\SIthousandsep}
%\begin{macro}{\SIproductsign}
%\begin{macro}{\SIdecimalsign}
% The same is true here, with the appropriate translation.
%    \begin{macrocode}
\newcommand*{\SIdecimalsymbol}[1]{\sisetup{decimalsymbol={#1}}}
\newcommand*{\SIthousandsep}[1]{\sisetup{digitsep={#1}}}
\newcommand*{\SIproductsign}[1]{\sisetup{expproduct={#1}}}
\newcommand*{\SIdecimalsign}[1]{\sisetup{decimalsymbol={#1}}}
%    \end{macrocode}
%\end{macro}
%\end{macro}
%\end{macro}
%\end{macro}
%\begin{macro}{\si@sis@savefont}
%\darg{setting}
%\darg{argument}
% The font definitions need a bit of extra work doing.  As both
% settings here have |@| as a letter, all should be fine.
%    \begin{macrocode}
\newcommand*{\si@sis@savefont}[2]{%
  \@namedef{si@sis@#1}{#2}%
  \sisetup{#1=si@sis@#1}}
%    \end{macrocode}
%\end{macro}
%\begin{macro}{\SImathrm}
%\begin{macro}{\SImathsf}
%\begin{macro}{\SImathtt}
% The font control macros have to ensure that a macro name is passed
% to \cs{sisetup}.
%    \begin{macrocode}
\newcommand*{\SImathrm}[1]{\si@sis@savefont{mathrm}{#1}}
\newcommand*{\SImathsf}[1]{\si@sis@savefont{mathsf}{#1}}
\newcommand*{\SImathtt}[1]{\si@sis@savefont{mathtt}{#1}}
%    \end{macrocode}
%\end{macro}
%\end{macro}
%\end{macro}
%\begin{macro}{\SIdefaultMfam}
%\begin{macro}{\SIdefaultNfam}
%\begin{macro}{\SIdefaultTfam}
% The same for the default keys.
%    \begin{macrocode}
\newcommand*{\SIdefaultMfam}[1]{\si@sis@savefont{mathrm}{#1}}
\newcommand*{\SIdefaultNfam}[1]{\si@sis@savefont{mathnumrm}{#1}}
\newcommand*{\SIdefaultTfam}[1]{\si@sis@savefont{textrm}{#1}}
%    \end{macrocode}
%\end{macro}
%\end{macro}
%\end{macro}
%\begin{macro}{\ensureupmath}
% The \cs{ensureupmath} command guarantees processing by the
% font-matching system. The argument cannot be processed here, so
% care is needed.
%    \begin{macrocode}
\si@newrobustcmd*{\ensureupmath}[1]{%
  \begingroup
    \sisetup{mode=maths,obeyitalic=false}%
    \si@out{#1}%
  \endgroup}
%    \end{macrocode}
%\end{macro}
%\begin{macro}{\degC}
%\begin{macro}{\degF}
%\begin{macro}{\arcdeg}
% A few extra symbol names are needed.
%    \begin{macrocode}
\newcommand*{\degC}{\si@sym@celsius}
\newcommand*{\arcdeg}{\si@sym@degree}
\newcommand*{\degF}{\si@sym@degree F}
%    \end{macrocode}
%\end{macro}
%\end{macro}
%\end{macro}
%\begin{macro}{\AddToSIstyle}
%\begin{macro}{\SIstyle}
%\begin{macro}{\SIstyleToLang}
%\begin{macro}{\si@sis@addtolocale}
% Finally, the locale control.
%    \begin{macrocode}
\newcommand*{\SIstyle}[1]{\sisetup{locale=#1}}
\newcommand*{\SIstyleToLang}[2]{\sisetup{loctolang=#1:#2}}
\newcommand*{\AddToSIstyle}{%
  \si@switchfalse
  \@ifstar
    {\si@switchtrue
     \si@sis@addtolocale}
    {\si@sis@addtolocale}}
\newcommand*{\si@sis@addtolocale}[2]{%
  \ifsi@switch
    \expandafter\let\csname si@loc@#1@extra\endcsname\relax
  \fi
  \addtolocale{#1}{#2}}
%    \end{macrocode}
%\end{macro}
%\end{macro}
%\end{macro}
%\end{macro}
%
%\iffalse
%</sistyle>
%<*SIunits>
%\fi
%\subsection{\SIunits}
% \SIunits emulation starts in much the same way.
%    \begin{macrocode}
\ProvidesFile{si-SIunits.cfg}
  [2008/12/22 v1.1j siunitx: Emulation of
     SIunits]
\si@emulating{SIunits}{2007/12/02 v1.36}
\sisetup{
  unitsep=thick,
  valuesep=thick,
  prefixproduct=\si@valuesep,
  trapambigfrac=false,
  stickyper}
\requiresiconfigs{prefix,named,accepted,physical}
%    \end{macrocode}
%\begin{macro}{\reciprocal}
%\begin{macro}{\rp}
%\begin{macro}{\per}
%\begin{macro}{\usk}
%\begin{macro}{\power}
%\begin{macro}{\rpsquare}
%\begin{macro}{\rpcubic}
%\begin{macro}{\fourth}
%\begin{macro}{\rpfourth}
% A few very simple translations, using the internal version of
% \cs{per} to allow changes of output style.
%    \begin{macrocode}
\newcommand*{\reciprocal}{\sisetup{per=reciprocal}\si@per}
\let\rp\reciprocal
\renewcommand*{\per}{\sisetup{per=slash}\si@per}
\newcommand*{\usk}{}
\newcommand*{\power}[1]{#1\tothe}
\newcommand*{\rpsquare}{\sisetup{per=reciprocal}\si@per\Square}
\newcommand*{\rpcubic}{\sisetup{per=reciprocal}\si@per\cubic}
\newpower{\fourth}{4}
\newcommand*{\rpfourth}{\sisetup{per=reciprocal}\si@per\fourth}
%    \end{macrocode}
%\end{macro}
%\end{macro}
%\end{macro}
%\end{macro}
%\end{macro}
%\end{macro}
%\end{macro}
%\end{macro}
%\end{macro}
%\begin{macro}{\rpsquared}
%\begin{macro}{\rpcubed}
% Here, some low-level switch changing is needed.
%    \begin{macrocode}
\newcommand*{\rpsquared}{%
  \sisetup{per=reciprocal}\si@unt@pertrue\si@unt@perseentrue%
    \squared}
\newcommand*{\rpcubed}{%
  \sisetup{per=reciprocal}\si@unt@pertrue\cubed}
%    \end{macrocode}
%\end{macro}
%\end{macro}
%\begin{macro}{\SIsetup}
%\begin{macro}{\si@siu@setup}
% The various package spacing options are processed.  They also have
% to be correctly handled by the \cs{SIsetup} macro.
%    \begin{macrocode}
\newcommand*{\SIsetup}[1]{%
  \@for\si@tempa:=#1\do{%
    \@ifundefined{ifsi@old@#1}
      {\si@log@warn{Unknown SIunits option `#1'}}
      {\csname si@old@#1true\endcsname}}
  \si@siu@setup}
\newcommand*{\si@siu@setup}{%
  \ifsi@old@cdot
    \sisetup{unitsep=cdot}%
  \fi
  \ifsi@old@thickspace
    \sisetup{unitsep=thick}%
  \fi
  \ifsi@old@mediumspace
    \sisetup{unitsep=medium}%
  \fi
  \ifsi@old@thinspace
    \sisetup{unitsep=thin}%
  \fi
  \ifsi@old@thickqspace
    \sisetup{valuesep=thick}%
  \fi
  \ifsi@old@mediumqspace
    \sisetup{valuesep=medium}%
  \fi
  \ifsi@old@thinqspace
    \sisetup{valuesep=thin}%
  \fi}
\si@siu@setup
%    \end{macrocode}
%\end{macro}
%\end{macro}
%\begin{macro}{\square}
%\begin{macro}{\squaren}
% \SIunits does slightly different things about the clash with
% \cs{square}, and either redefines this macro or provides
% \cs{squaren}.
%    \begin{macrocode}
\ifsi@old@squaren
  \newpower{\squaren}{2}
\fi
\AtBeginDocument{%
  \@ifundefined{square}
    {\newpower{\square}{2}}
    {\ifsi@old@amssymb
       \renewpower{\square}{2}
     \else
       \ifsi@old@squaren\else
         \si@log@warn{\string\square\space already
           defined\MessageBreak SIunits mode may cause
           errors}%
       \fi
     \fi}}
%    \end{macrocode}
%\end{macro}
%\end{macro}
%\begin{macro}{\gray}
% The potential clash with \pkg{PStricks} is also handled
% differently; here, \cs{Gray} will already be defined by the
%  \currpkg kernel.
%    \begin{macrocode}
\AtBeginDocument{
  \@ifundefined{gray}
    {\newunit{\gray}{Gy}}
    {\ifsi@old@pstricks
       \renewunit{\gray}{Gy}
     \else
       \ifsi@old@Gray\else
         \si@log@warn{\string\gray\space already
           defined\MessageBreak SIunits mode may cause
           errors}%
       \fi
     \fi}}
%    \end{macrocode}
%\end{macro}
%\begin{macro}{\unit}
%\begin{macro}{\unita}
% The \cs{unit} macro is defined.
%    \begin{macrocode}
\ifsi@old@italian
  \let\unita\SI
\else
  \let\unit\SI
\fi
%    \end{macrocode}
%\end{macro}
%\end{macro}
% The miscellaneous options are moped up.
%    \begin{macrocode}
\ifsi@old@textstyle
  \sisetup{mode=text}
\fi
\ifsi@old@binary
  \sisetup{alsoload=binary}
\fi
\ifsi@old@noams
  \AtBeginDocument{%
    \renewcommand*{\si@textmu}{\ensuremath\si@mathsmu}}
\fi
%    \end{macrocode}
%\begin{macro}{\arcminute}
%\begin{macro}{\arcsecond}
%\begin{macro}{\rperminute}
%\begin{macro}{\ton}
%\begin{macro}{\degreecelsius}
% The unit macros defined by \SIunits that are not defined by
% \currpkg (by default).
%    \begin{macrocode}
\newunit[valuesep=none]{\arcminute}{\si@sym@minute}
\newunit[valuesep=none]{\arcsecond}{\si@sym@second}
\newunit{\rperminute}{r/min}
\newunit{\ton}{t}
\newunit{\degreecelsius}{\celsius}
%    \end{macrocode}
%\end{macro}
%\end{macro}
%\end{macro}
%\end{macro}
%\end{macro}
%\begin{macro}{\addunit}
% This is an alias for \cs{newunit}.
%    \begin{macrocode}
\let\addunit\newunit
%    \end{macrocode}
%\end{macro}
%\begin{macro}{\addprefix}
% A little more work for \cs{addprefix}.
%    \begin{macrocode}
\newcommand*{\addprefix}[2]{\newprefix{#1}{#2}}
%    \end{macrocode}
%\end{macro}
%\begin{macro}{\si@siu@newunit}
%\changes{v1.1a}{2008/09/30}{Corrected very minor bug in definition}
%\begin{macro}{\si@siu@power}
%\begin{macro}{\si@siu@newunithook}
%\doarg{power}
%\darg{numerator}
%\darg{denominator}
% To save some code, making new units which need a \texttt{\ldots np}
% variant is handled by a dedicated macro.
%    \begin{macrocode}
\newcommand*{\si@siu@newunit}[3][]{%
%    \end{macrocode}
% A test is needed to sort out \cs{square}.
%    \begin{macrocode}
  \renewcommand*{\si@tempa}{#1}%
  \renewcommand*{\si@tempb}{square}%
  \def\si@siu@power{}%
  \ifx\@empty\si@tempa\@empty\else
    \ifx\si@tempa\si@tempb
      \renewcommand*{\si@siu@power}{\ssquare}%
    \else
      \edef\si@siu@power{%
        \expandafter\noexpand\csname #1\endcsname}%
    \fi
  \fi
%    \end{macrocode}
% The necessary information is now stored in temporary macros.
%    \begin{macrocode}
  \edef\si@tempa{%
    \expandafter\noexpand\csname #2per#1#3\endcsname}%
  \edef\si@tempb{%
    \expandafter\noexpand\csname #2\endcsname\noexpand\per
    \expandafter\noexpand\si@siu@power
    \expandafter\noexpand\csname #3\endcsname}%
  \expandafter\expandafter\expandafter\newunit\expandafter%
    \expandafter\expandafter{\expandafter\si@tempa\expandafter}%
      \expandafter{\si@tempb}
  \edef\si@tempa{%
    \expandafter\noexpand\csname #2per#1#3np\endcsname}%
  \edef\si@tempb{%
    \expandafter\noexpand\csname #2\endcsname\noexpand
    \reciprocal\expandafter\noexpand\si@siu@power
    \expandafter\noexpand\csname #3\endcsname}%
  \expandafter\expandafter\expandafter\newunit\expandafter
    \expandafter\expandafter{\expandafter\si@tempa\expandafter}%
      \expandafter{\si@tempb}%
  \si@siu@newunithook[#1]{#2}{#3}}
\providecommand*{\si@siu@newunithook}[3][]{}
%    \end{macrocode}
%\end{macro}
%\end{macro}
%\end{macro}
% The basic units are now defined; these only have a single csname in
% each of the numerator and denominator.
%    \begin{macrocode}
\si@siu@newunit{gray}{second}
\si@siu@newunit[square]{metre}{second}
\si@siu@newunit{joule}{mole}
\si@siu@newunit[cubic]{mole}{metre}
\si@siu@newunit[square]{radian}{second}
\si@siu@newunit{radian}{second}
\si@siu@newunit[cubic]{squaremetre}{metre}
\si@siu@newunit[cubic]{katal}{metre}
\si@siu@newunit{coulomb}{mol}
\si@siu@newunit[square]{ampere}{metre}
\si@siu@newunit[cubic]{kilogram}{metre}
\si@siu@newunit[cubic]{coulomb}{metre}
\si@siu@newunit{volt}{metre}
\si@siu@newunit[square]{coulomb}{squaremetre}
\si@siu@newunit{farad}{metre}
\si@siu@newunit[square]{watt}{metre}
\si@siu@newunit[square]{joule}{metre}
\si@siu@newunit[cubic]{newton}{metre}
\si@siu@newunit{newton}{kilogram}
\si@siu@newunit{joule}{kelvin}
\si@siu@newunit{joule}{kilogram}
\si@siu@newunit{coulomb}{kilogram}
\si@siu@newunit{squaremetre}{second}
\si@siu@newunit[square]{squaremetre}{second}
\si@siu@newunit[square]{candela}{metre}
\si@siu@newunit{ampere}{metre}
\si@siu@newunit{joule}{tesla}
\si@siu@newunit{henry}{metre}
\si@siu@newunit{kilogram}{second}
\si@siu@newunit[square]{kilogram}{metre}
\si@siu@newunit{kilogram}{metre}
\si@siu@newunit[square]{newton}{metre}
\si@siu@newunit{watt}{kilogram}
\si@siu@newunit[cubic]{watt}{metre}
\si@siu@newunit{squaremetre}{kilogram}
\si@siu@newunit{cubicmetre}{kilogram}
\si@siu@newunit{newton}{metre}
\si@siu@newunit[cubic]{squaremetre}{second}
\si@siu@newunit{metre}{second}
\si@siu@newunit[cubic]{joule}{metre}
\si@siu@newunit{cubicmetre}{second}
%    \end{macrocode}
%\begin{macro}{\si@siu@newunitx}
%\begin{macro}{\si@siu@newunitxhook}
% For the more complex units, a slightly different approach is used;
% four arguments are required, and have to cover everything.
%    \begin{macrocode}
\newcommand*{\si@siu@newunitx}[4]{%
  \expandafter\newunit\expandafter{\csname #1per#2\endcsname}
    {#3\per#4}%
  \expandafter\newunit\expandafter{\csname #1per#2np\endcsname}
    {#3\reciprocal#4}
  \si@siu@newunitxhook{#1}{#2}{#3}{#4}}
\providecommand*{\si@siu@newunitxhook}[4]{}
%    \end{macrocode}
%\end{macro}
%\end{macro}
% The units are defined.
%    \begin{macrocode}
\si@siu@newunitx{kilogramsquaremetre}{second}
  {\kilogram\squaremetre}{\second}
\si@siu@newunitx{squaremetre}{newtonsecond}{\squaremetre}
  {\newton\second}
\si@siu@newunitx{kilogrammetre}{second}{\kilogram\metre}
  {\second}
\si@siu@newunitx{kilogram}{squaremetresecond}{\kilogram}
  {\squaremetre\second}
\si@siu@newunitx{joule}{molekelvin}{\joule}{\mole\kelvin}
\si@siu@newunitx{kilogram}{kilomole}{\kilogram}{\kilo\mole}
\si@siu@newunitx{kilogrammetre}{squaresecond}{\kilogram\metre}
  {\second\squared}
\si@siu@newunitx{watt}{squaremetresteradian}{\watt}
  {\squaremetre\steradian}
\si@siu@newunitx{joule}{kilogramkelvin}{\joule}
  {\kilogram\kelvin}
\si@siu@newunitx{watt}{metrekelvin}{\watt}{\metre\kelvin}
\si@siu@newunitx{kilogram}{cubicmetrecoulomb}{\kilogram}
  {\cubic\metre\coulomb}
\si@siu@newunitx{kilogram}{secondcubicmetre}{\kilogram}
  {\second\cubicmetre}
%    \end{macrocode}
%\begin{macro}{\si@siu@unity}
% A bit of cleverness to get the ``1'' correct; to avoid any clash, the unit
% is given an internal name.
%    \begin{macrocode}
\newunit{\si@siu@unity}{1}
\si@siu@newunitx{}{squaremetresecond}{\si@siu@unity}
  {\squaremetre\second}
%    \end{macrocode}
%\end{macro}
% A few compound units that are best defined directly.
%    \begin{macrocode}
\newunit{\pascalsecond}{\pascal\second}
\newunit{\amperemetresecond}{\ampere\metre\second}
\newunit{\ohmmetre}{\ohm\metre}
\newunit{\newtonmetre}{\newton\metre}
\let\newtonmetrenp\newtonmetre
\newunit{\kilogramsquaremetre}{\kilogram\squaremetre}
\let\kilogramsquaremetrenp\kilogramsquaremetre
%    \end{macrocode}
%\begin{macro}{\si@siu@newprefix}
%\darg{prefix}
% To generate the prefixes correctly, a small saving in repetition.
%    \begin{macrocode}
\newcommand*{\si@siu@newprefix}[1]{%
  \edef\si@tempa{\expandafter\noexpand\csname #1d\endcsname}%
  \edef\si@tempb{\expandafter\noexpand\csname #1\endcsname}%
  \expandafter\expandafter\expandafter\newcommand\expandafter
    \expandafter\expandafter*\expandafter\expandafter
    \expandafter{\expandafter\si@tempa\expandafter}\expandafter
      {\expandafter\si@prefixsymbolicfalse\si@tempb}}
%    \end{macrocode}
%\end{macro}
% This is now implemented.
%    \begin{macrocode}
\si@siu@newprefix{yocto}
\si@siu@newprefix{zepto}
\si@siu@newprefix{atto}
\si@siu@newprefix{femto}
\si@siu@newprefix{pico}
\si@siu@newprefix{nano}
\si@siu@newprefix{micro}
\si@siu@newprefix{milli}
\si@siu@newprefix{centi}
\si@siu@newprefix{deca}
\si@siu@newprefix{deka}
\si@siu@newprefix{hecto}
\si@siu@newprefix{kilo}
\si@siu@newprefix{mega}
\si@siu@newprefix{giga}
\si@siu@newprefix{tera}
\si@siu@newprefix{peta}
\si@siu@newprefix{exa}
\si@siu@newprefix{zetta}
\si@siu@newprefix{yotta}
\ifsi@old@binary
  \si@siu@newprefix{kibi}
  \si@siu@newprefix{mebi}
  \si@siu@newprefix{gibi}
  \si@siu@newprefix{tebi}
  \si@siu@newprefix{pebi}
  \si@siu@newprefix{exbi}
\fi
%    \end{macrocode}
%\begin{macro}{\derradian}
%\begin{macro}{\dersteradian}
%\begin{macro}{\derhertz}
%\begin{macro}{\dernewton}
%\begin{macro}{\derpascal}
%\begin{macro}{\derjoule}
%\begin{macro}{\derwatt}
%\begin{macro}{\dercoulomb}
%\begin{macro}{\dervolt}
%\begin{macro}{\derfarad}
%\begin{macro}{\derohm}
% The derived units may need to be defined.
%    \begin{macrocode}
\ifsi@old@derived
  \newunit{\derradian}{\metre\reciprocal\metre}
  \newunit{\dersteradian}{\squaremetre\rpsquare\metre}
  \newunit{\derhertz}{\reciprocal\second}
  \newunit{\dernewton}{\metre\kilogram\second\rpsquared}
  \newunit{\derpascal}{\newton\rpsquare\metre}
  \newunit{\derjoule}{\newton\metre}
  \newunit{\derwatt}{\joule\reciprocal\second}
  \newunit{\dercoulomb}{\ampere\second}
  \newunit{\dervolt}{\watt\reciprocal\ampere}
  \newunit{\derfarad}{\coulomb\reciprocal\volt}
  \newunit{\derohm}{\volt\reciprocal\ampere}
%    \end{macrocode}
%\end{macro}
%\end{macro}
%\end{macro}
%\end{macro}
%\end{macro}
%\end{macro}
%\end{macro}
%\end{macro}
%\end{macro}
%\end{macro}
%\end{macro}
%\begin{macro}{\dersiemens}
%\begin{macro}{\derweber}
%\begin{macro}{\dertesla}
%\begin{macro}{\derhenry}
%\begin{macro}{\dercelsius}
%\begin{macro}{\derlumen}
%\begin{macro}{\derlux}
%\begin{macro}{\derbecquerel}
%\begin{macro}{\dergray}
%\begin{macro}{\dersievert}
%\begin{macro}{\derkatal}
% In two blocks!
%    \begin{macrocode}
  \newunit{\dersiemens}{\ampere\reciprocal\volt}
  \newunit{\derweber}
    {\squaremetre\kilogram\second\rpsquared\reciprocal\ampere}
  \newunit{\dertesla}{\weber\rpsquare\metre}
  \newunit{\derhenry}{\weber\reciprocal\ampere}
  \newunit{\dercelsius}{\kelvin}
  \newunit{\derlumen}{\candela\steradian}
  \newunit{\derlux}{\lumen\rpsquare\metre}
  \newunit{\derbecquerel}{\derhertz}
  \newunit{\dergray}{\joule\reciprocal\kilogram}
  \newunit{\dersievert}{\dergray}
  \newunit{\derkatal}{\rp\second\usk\mole}
\fi
%    \end{macrocode}
%\end{macro}
%\end{macro}
%\end{macro}
%\end{macro}
%\end{macro}
%\end{macro}
%\end{macro}
%\end{macro}
%\end{macro}
%\end{macro}
%\end{macro}
%\begin{macro}{\radianbase}
%\begin{macro}{\steradianbase}
%\begin{macro}{\hertzbase}
%\begin{macro}{\newtonbase}
%\begin{macro}{\pascalbase}
%\begin{macro}{\joulebase}
%\begin{macro}{\wattbase}
%\begin{macro}{\coulombbase}
%\begin{macro}{\voltbase}
%\begin{macro}{\faradbase}
%\begin{macro}{\ohmbase}
% Also the ``derived-in-base''.
%    \begin{macrocode}
\ifsi@old@derivedinbase
  \newunit{\radianbase}{\metre\reciprocal\metre}
  \newunit{\steradianbase}{\squaremetre\rpsquare\metre}
  \newunit{\hertzbase}{\reciprocal\second}
  \newunit{\newtonbase}{\metre\kilogram\second\rpsquared}
  \newunit{\pascalbase}{\reciprocal\metre\kilogram\second%
    \rpsquared}
  \newunit{\joulebase}{\squaremetre\kilogram\second\rpsquared}
  \newunit{\wattbase}{\squaremetre\kilogram\rpcubic\second}
  \newunit{\coulombbase}{\ampere\second}
  \newunit{\voltbase}
    {\squaremetre\kilogram\rpcubic\second\reciprocal\ampere}
  \newunit{\faradbase}
    {\rpsquare\metre\reciprocal\kilogram\fourth\second\ampere%
      \squared}
  \newunit{\ohmbase}
    {\squaremetre\kilogram\rpcubic\second\rpsquare\ampere}
%    \end{macrocode}
%\end{macro}
%\end{macro}
%\end{macro}
%\end{macro}
%\end{macro}
%\end{macro}
%\end{macro}
%\end{macro}
%\end{macro}
%\end{macro}
%\end{macro}
%\begin{macro}{\siemensbase}
%\begin{macro}{\weberbase}
%\begin{macro}{\teslabase}
%\begin{macro}{\henrybase}
%\begin{macro}{\celsiusbase}
%\begin{macro}{\lumenbase}
%\begin{macro}{\luxbase}
%\begin{macro}{\becquerelbase}
%\begin{macro}{\graybase}
%\begin{macro}{\sievertbase}
%\begin{macro}{\katalbase}
% Also in two blocks.
%    \begin{macrocode}
  \newunit{\siemensbase}
    {\rpsquare\metre\reciprocal\kilogram\cubic\second\ampere%
       \squared}
  \newunit{\weberbase}
    {\squaremetre\kilogram\second\rpsquared\reciprocal\ampere}
  \newunit{\teslabase}{\kilogram\second\rpsquared\reciprocal%
     \ampere}
  \newunit{\henrybase}
    {\squaremetre\kilogram\second\rpsquared\rpsquare\ampere}
  \newunit{\celsiusbase}{\kelvin}
  \newunit{\lumenbase}{\candela\squaremetre\rpsquare\metre}
  \newunit{\luxbase}{\candela\squaremetre\rpfourth\metre}
  \newunit{\becquerelbase}{\hertzbase}
  \newunit{\graybase}{\squaremetre\second\rpsquared}
  \newunit{\sievertbase}{\graybase}
  \newunit{\katalbase}{\rp\second\mole}
\fi
%    \end{macrocode}
%\end{macro}
%\end{macro}
%\end{macro}
%\end{macro}
%\end{macro}
%\end{macro}
%\end{macro}
%\end{macro}
%\end{macro}
%\end{macro}
%\end{macro}
% Any configuration file is used if found.
%    \begin{macrocode}
\InputIfFileExists{SIunits.cfg}
  {\si@log@inf{SIunits config file loaded}}
  {\si@log@inf{SIunits config file not found}}
%    \end{macrocode}
%\iffalse
%</SIunits>
%<*hepunits>
%\fi
%\subsection{\pkg{hepunits}}
% The \pkg{hepunits} package provides some rather odd unit names,
% which are not really to be encouraged.
%    \begin{macrocode}
\ProvidesFile{si-hepunits.cfg}
  [2008/12/22 v1.1j siunitx: Emulation of
     hepunits]
\si@emulating{hepunits}{2007/09/27}
\requiresiconfigs{SIunits,accepted,prefix,hep}
%    \end{macrocode}
%\begin{macro}{\invbarn}
%\begin{macro}{\invnanobarn}
%\begin{macro}{\invpicobarn}
%\begin{macro}{\invfemtobarn}
%\begin{macro}{\invattobarn}
%\begin{macro}{\invzeptobarn}
%\begin{macro}{\invyoctobarn}
% Inverses barn units.
%    \begin{macrocode}
\ifsi@old@noprefixcmds\else
  \newunit{\invbarn}{\per\barn}
  \newunit{\invnanobarn}{\per\nano\barn}
  \newunit{\invpicobarn}{\per\pico\barn}
  \newunit{\invfemtobarn}{\per\femto\barn}
  \newunit{\invattobarn}{\per\atto\barn}
  \newunit{\invzeptobarn}{\per\zepto\barn}
  \newunit{\invyoctobarn}{\per\yocto\barn}
%    \end{macrocode}
%\end{macro}
%\end{macro}
%\end{macro}
%\end{macro}
%\end{macro}
%\end{macro}
%\begin{macro}{\invnb}
%\begin{macro}{\invpb}
%\begin{macro}{\invfb}
%\begin{macro}{\invab}
%\begin{macro}{\invzb}
%\begin{macro}{\invyb}
% Also available abbreviated.
%    \begin{macrocode}
  \newunit{\invnb}{\per\nano\barn}
  \newunit{\invpb}{\per\pico\barn}
  \newunit{\invfb}{\per\femto\barn}
  \newunit{\invab}{\per\atto\barn}
  \newunit{\invzb}{\per\zepto\barn}
  \newunit{\invyb}{\per\yocto\barn}
\fi
%    \end{macrocode}
%\end{macro}
%\end{macro}
%\end{macro}
%\end{macro}
%\end{macro}
%\end{macro}
%\end{macro}
%\begin{macro}{\invcmsqpersecond}
%\begin{macro}{\invcmsqpersec}
%\begin{macro}{\lumiunits}
% Luminosity.
%    \begin{macrocode}
\newunit{\invcmsqpersecond}{\per\Square\centi\metre\per\second}
\newunit{\invcmsqpersec}{\per\Square\centi\metre\per\second}
\newunit{\lumiunits}{\per\Square\centi\metre\per\second}
%    \end{macrocode}
%\end{macro}
%\end{macro}
%\end{macro}
%\begin{macro}{\inveV}
%\begin{macro}{\minveV}
%\begin{macro}{\minveV}
%\begin{macro}{\kinveV}
%\begin{macro}{\MinveV}
%\begin{macro}{\GinveV}
%\begin{macro}{\TinveV}
% The inverse of an electron-volt, plus prefixes.
%    \begin{macrocode}
\newunit{\inveV}{\per\electronvolt}
\newunit{\minveV}{\per\milli\electronvolt}
\newunit{\kinveV}{\per\kilo\electronvolt}
\newunit{\MinveV}{\per\mega\electronvolt}
\newunit{\GinveV}{\per\giga\electronvolt}
\newunit{\TinveV}{\per\tera\electronvolt}
%    \end{macrocode}
%\end{macro}
%\end{macro}
%\end{macro}
%\end{macro}
%\end{macro}
%\end{macro}
%\end{macro}
%\begin{macro}{\eVoverc}
%\begin{macro}{\eVovercsq}
% Some combinations of electron-volts and the speed of light.  As
% these are called |over|, they are set with a slash.  The
% \opt{eVcorrb} values have been set for Computer Modern.
%    \begin{macrocode}
\newunit[per=slash,eVcorrb=0.6ex]{\eVoverc}
  {\electronvolt\per\clight}
\newunit[per=slash,eVcorrb=0.6ex]{\eVovercsq}
  {\electronvolt\per\Square\clight}
%    \end{macrocode}
%\end{macro}
%\end{macro}
%\begin{macro}{\meVoverc}
%\begin{macro}{\keVoverc}
%\begin{macro}{\MeVoverc}
%\begin{macro}{\GeVoverc}
%\begin{macro}{\TeVoverc}
% Prefixed combinations, first of the speed of light.
%    \begin{macrocode}
\newunit[per=slash,eVcorrb=0.6ex]{\meVoverc}
  {\milli\electronvolt\per\clight}
\newunit[per=slash,eVcorrb=0.6ex]{\keVoverc}
  {\kilo\electronvolt\per\clight}
\newunit[per=slash,eVcorrb=0.6ex]{\MeVoverc}
  {\mega\electronvolt\per\clight}
\newunit[per=slash,eVcorrb=0.6ex]{\GeVoverc}
  {\giga\electronvolt\per\clight}
\newunit[per=slash,eVcorrb=0.6ex]{\TeVoverc}
  {\tera\electronvolt\per\clight}
%    \end{macrocode}
%\end{macro}
%\end{macro}
%\end{macro}
%\end{macro}
%\end{macro}
%\begin{macro}{\meVovercsq}
%\begin{macro}{\keVovercsq}
%\begin{macro}{\MeVovercsq}
%\begin{macro}{\GeVovercsq}
%\begin{macro}{\TeVovercsq}
% Then of the square.
%    \begin{macrocode}
\newunit[per=slash,eVcorrb=0.6ex]{\meVovercsq}
  {\milli\electronvolt\per\Square\clight}
\newunit[per=slash,eVcorrb=0.6ex]{\keVovercsq}
  {\kilo\electronvolt\per\Square\clight}
\newunit[per=slash,eVcorrb=0.6ex]{\MeVovercsq}
  {\mega\electronvolt\per\Square\clight}
\newunit[per=slash,eVcorrb=0.6ex]{\GeVovercsq}
  {\giga\electronvolt\per\Square\clight}
\newunit[per=slash,eVcorrb=0.6ex]{\TeVovercsq}
  {\tera\electronvolt\per\Square\clight}
%    \end{macrocode}
%\end{macro}
%\end{macro}
%\end{macro}
%\end{macro}
%\end{macro}
%\iffalse
%</hepunits>
%<*fancynum>
%\fi
%\subsection{\pkg{fancynum}}
% \pkg{fancynum} only does things with numbers, so there is only
% a little emulation and a few macros needed.
%    \begin{macrocode}
\ProvidesFile{si-fancynum.cfg}
  [2008/12/22 v1.1j siunitx: Emulation of
     fancynum]
\si@emulating{fancynum}{2000/08/08 0.92}
\sisetup{decimalsymbol=cdot,digitsep=comma}
%    \end{macrocode}
%\begin{macro}{\fnum}
% The \cs{fnum} macro is rather restricted, but this is not
% reproduced.  Instead, it is \cs{let} as an alias to the \cs{num}
% macro.
%    \begin{macrocode}
\let\fnum\num
%    \end{macrocode}
%\end{macro}
%\begin{macro}{\setfnumdsym}
%\begin{macro}{\setfnumgsym}
%\begin{macro}{\setfnummsym}
% The control macros are defined.
%    \begin{macrocode}
\newcommand*{\setfnumdsym}[1]{\sisetup{decimalsymbol={#1}}}
\newcommand*{\setfnumgsym}[1]{\sisetup{digitsep={#1}}}
\newcommand*{\setfnummsym}[1]{\sisetup{expproduct={#1}}}
%    \end{macrocode}
%\end{macro}
%\end{macro}
%\end{macro}
% The various package options are now processed if necessary.
%    \begin{macrocode}
\ifsi@old@english
  \sisetup{decimalsymbol=cdot,digitsep=comma}
\fi
\ifsi@old@french
  \sisetup{decimalsymbol=comma,digitsep=fullstop}
\fi
\ifsi@old@tight
  \sisetup{expproduct=tighttimes}
\fi
\ifsi@old@loose
  \sisetup{expproduct=times}
\fi
\ifsi@old@thinspaces
  \sisetup{digitsep=thin}
\fi
\ifsi@old@commas
  \sisetup{digitsep=comma}
\fi
\ifsi@old@plain
  \sisetup{digitsep=none}
\fi
%    \end{macrocode}
%
%\iffalse
%</fancynum>
%<*fancyunits>
%\fi
%\subsection{\pkg{fancyunits}}
% The \pkg{fancyunits} package is not available on \textsc{ctan}, but
% is available from its authors homepage \cite{fancyunits}.  It is
% similar to \SIunits, and so most of the code is shared here.
% However, a few bits of set up occur first, and an emulation-clash
% test is needed.
%    \begin{macrocode}
\ProvidesFile{si-fancyunits.cfg}
  [2008/12/22 v1.1j siunitx: Emulation of
     fancyunits]
\si@emulating{fancyunits}{2007/02/01 v1.0.1}
\si@ifloaded{SIunits}
  {\si@log@err{SIunits emulation loaded\MessageBreak before
    fancyunits emulation}{You need to load the fancyunits
    emualtion\MessageBreak code before that for
    SIunits.\MessageBreak Try emulate=fancyunits as the first
    option when\MessageBreak loading siunitx}}{}
%    \end{macrocode}
%\begin{macro}{\si@siu@newunithook}
%\begin{macro}{\si@siu@newunitxhook}
% To create the extra macros provided by \pkg{fancyunits}, the
% \SIunits emulation code is changed to add the ``\texttt{uf}''
% variants.
%    \begin{macrocode}
\newcommand*{\si@siu@newunithook}[3][]{%
  \edef\si@tempa{%
    \expandafter\noexpand\csname #2per#1#3uf\endcsname}%
  \renewcommand*{\si@tempb}{stickyper,per=fraction,
    fraction=nice}%
  \edef\si@tempc{%
    \noexpand\sisetup{\si@tempb}%
    \expandafter\noexpand\csname #2\endcsname\noexpand\si@per%
    \expandafter\noexpand\si@siu@power%
    \expandafter\noexpand\csname #3\endcsname}%
  \expandafter\expandafter\expandafter\newunit\expandafter
    \expandafter\expandafter{\expandafter\si@tempa\expandafter}%
      \expandafter{\si@tempc}%
  \edef\si@tempa{%
    \expandafter\noexpand\csname #2per#1#3Uf\endcsname}%
  \renewcommand*{\si@tempb}{stickyper,per=fraction,
    fraction=frac}%
  \edef\si@tempc{%
    \noexpand\sisetup{\si@tempb}%
    \noexpand\def\noexpand\si@frc@hook{\noexpand\textstyle}%
    \expandafter\noexpand\csname #2\endcsname\noexpand\si@per%
    \expandafter\noexpand\si@siu@power%
    \expandafter\noexpand\csname #3\endcsname}%
  \expandafter\expandafter\expandafter\newunit\expandafter
    \expandafter\expandafter{\expandafter\si@tempa\expandafter}%
      \expandafter{\si@tempc}%
  \edef\si@tempa{%
    \expandafter\noexpand\csname #2per#1#3UF\endcsname}%
  \edef\si@tempc{%
    \noexpand\sisetup{\si@tempb}%
    \noexpand\def\noexpand\si@frc@hook{\noexpand\displaystyle}%
    \expandafter\noexpand\csname #2\endcsname\noexpand\si@per%
    \expandafter\noexpand\si@siu@power%
    \expandafter\noexpand\csname #3\endcsname}%
  \expandafter\expandafter\expandafter\newunit\expandafter
    \expandafter\expandafter{\expandafter\si@tempa\expandafter}%
      \expandafter{\si@tempc}}
\newcommand*{\si@siu@newunitxhook}[4]{%
  \expandafter\newunit\expandafter{\csname #1per#2uf\endcsname}
    {\sisetup{stickyper,per=fraction,fraction=nice}%
     #3\si@per#4}%
  \expandafter\newunit\expandafter{\csname #1per#2Uf\endcsname}
    {\sisetup{stickyper,per=fraction,fraction=frac}%
     \renewcommand*{\si@frc@hook}{\textstyle}%
     #3\si@per#4}%
  \expandafter\newunit\expandafter{\csname #1per#2UF\endcsname}
    {\sisetup{stickyper,per=fraction,fraction=frac}%
     \renewcommand*{\si@frc@hook}{\displaystyle}%
     #3\si@per#4}}
%    \end{macrocode}
%\end{macro}
%\end{macro}
% With that done, the emulation modules can be loaded.
%    \begin{macrocode}
\requiresiconfigs{SIunits,addn,astro}
\sisetup{obeyall}
%    \end{macrocode}
% There is one \pkg{fancyunits}-specific option to handle.  The other
% options all get sent through to the \SIunits system.\footnote{As
% \SIunits is rather more widely known than \pkg{fancyunits}, any
% other options which could be for either are assumed to be for
% \SIunits.}
%    \begin{macrocode}
\ifsi@old@spaceqspace
  \sisetup{valuesep=space}
\fi
%    \end{macrocode}
%\begin{macro}{\paminute}
%\begin{macro}{\parsecond}
%\begin{macro}{\AstroE}
%\begin{macro}{\oersted}
%\begin{macro}{\ton}
% \pkg{fancyunits} provides some extra units, plus tonne spelled
% incorrectly (again).
%    \begin{macrocode}
\newunit{\paminute}{'}
\newunit{\parsecond}{''}
\newunit{\AstroE}{AE}
\newunit{\oersted}{OE}
\provideunit{\ton}{t}
%    \end{macrocode}
%\end{macro}
%\end{macro}
%\end{macro}
%\end{macro}
%\end{macro}
%\begin{macro}{\decaD}
% An additional prefix.
%    \begin{macrocode}
\let\decaD\decad
%    \end{macrocode}
%\end{macro}
%\begin{macro}{\ufrac}
%\begin{macro}{\Ufrac}
%\begin{macro}{\UFrac}
% The fractional unit macros need to be reproduced.
%    \begin{macrocode}
\newcommand*{\ufrac}[2]{%
  \si[stickyper,per=fraction,fraction=nice]{#1\si@per#2}}
\newcommand*{\Ufrac}[2]{%
  \ensuremath{\textstyle{%
    \si[stickyper,per=fraction,fraction=frac]{#1\si@per#2}}}}
\newcommand*{\UFrac}[2]{%
  \ensuremath{\displaystyle{%
    \si[stickyper,per=fraction,fraction=frac]{#1\si@per#2}}}}
%    \end{macrocode}
%\end{macro}
%\end{macro}
%\end{macro}
%\begin{macro}{\pow}
% A slightly-shorted named \cs{power}.
%    \begin{macrocode}
\let\pow\power
%    \end{macrocode}
%\end{macro}
%\begin{macro}{\Squaremetre}
% An alias for \cs{squaremetre}.
%    \begin{macrocode}
\let\Squaremetre\squaremetre
%    \end{macrocode}
%\end{macro}
% As with \SIunits, there is now a list of compound units to add.
% Only a few are not covered by the \SIunits emulation.
%    \begin{macrocode}
\si@siu@newunit{Gray}{second}
\si@siu@newunit[square]{Squaremetre}{metre}
\si@siu@newunitx{Squaremetre}{newtonsecond}{\Square\metre}
  {\newton\second}
\si@siu@newunit{Squaremetre}{second}
\si@siu@newunit[square]{Squaremetre}{squaresecond}
\si@siu@newunit{Squaremetre}{kilogram}
\si@siu@newunit[cubic]{Squaremetre}{second}
%    \end{macrocode}
%\iffalse
%</fancyunits>
%\fi
%
%\Finale
%\iffalse
%<*refs>
@misc{BIPM:EN,
  note = {\url{http://www.bipm.org/en/si/}},
}

@misc{BIPM:FR,
  note = {\url{http://www.bipm.org/fr/si/}},
}

@misc{BIPM:PDF,
  note = {\url{http://www.bipm.org/en/si/si_brochure/}},
}

@misc{Eijkhout,
  author =       {Victor Eijkhout},
  note =         {\url{http://www.eijkhout.net/tbt/}},
  title =        {{\TeX} by {T}opic},
  year =         {2007},
}

@misc{fancyunits,
  author = {Heiko Bauke},
  note = {\url{http://www.mpi-hd.mpg.de/personalhomes/bauke/LaTeX/Tips_und_Tricks/fancyunits/index.php}},
  title = {\pkg{fancyunits}},
  year = 2007,
}

@misc{IUPAC,
  note = {\url{http://old.iupac.org/publications/books/gbook/green_book_2ed.pdf}},
  title = {The \textsc{iupac} {G}reen {B}ook},
  year = 1993,
}

@misc{NIST,
  note = {\url{http://physics.nist.gov/cuu/Units/index.html}},
}
%</refs>
%<*jawltxdoc>
\NeedsTeXFormat{LaTeX2e}
\ProvidesPackage{jawltxdoc}
\usepackage[T1]{fontenc}
\usepackage{lmodern}
\usepackage[final]{listings,graphicx,microtype}
\usepackage[scaled=0.95]{helvet}
\usepackage[version=3]{mhchem}
\usepackage[osf]{mathpazo}
\usepackage{booktabs,array,url,courier,xspace,varioref}
\usepackage{upgreek,ifpdf,float,caption,longtable,babel}
\begingroup
  \@ifundefined{eTeXversion}
    {\aftergroup\@gobble}
    {\aftergroup\@firstofone}
\endgroup
{\usepackage{etoolbox}}
\floatstyle{plaintop}
\restylefloat{table}
\labelformat{figure}{\figurename~#1}
\labelformat{table}{\tablename~#1}
\IfFileExists{\jobname.sty}
  {\usepackage{\jobname}}{}
\usepackage[numbered]{hypdoc}
\setcounter{IndexColumns}{2}
\newlength\LaTeXwidth
\newlength\LaTeXoutdent
\newlength\LaTeXgap
\setlength\LaTeXgap{1em}
\setlength\LaTeXoutdent{-0.15\textwidth}
\newbox\lst@samplebox
\edef\LaTeXexamplefile{\jobname.tmp}
\lst@RequireAspects{writefile}
\lstnewenvironment{LaTeXexample}[1][example]{%
  \global\let\lst@intname\@empty
  \ifcsname LaTeXcode#1\endcsname
    \expandafter\let\expandafter\LaTeXcode
      \csname LaTeXcode#1\endcsname
    \expandafter\let\expandafter\LaTeXcodeend
      \csname LaTeXcode#1end\endcsname
  \else
    \PackageError{jawltxdoc}
      {Undefined example type `#1'}
      \@ehd
    \let\LaTeXcode\relax
    \let\LaTeXcodeend\relax
  \fi
  \LaTeXcode}
  {\lst@EndWriteFile
   \LaTeXcodeend}
\newcommand*{\LaTeXcodeexample}{%
  \setbox\lst@samplebox=\hbox\bgroup
  \LaTeXcodefloat}
\let\LaTeXcoderesultonly\LaTeXcodeexample
\newcommand*{\LaTeXcodeexampleend}{%
  \egroup
  \setlength\LaTeXwidth{\wd\lst@samplebox}%
  \begin{list}{}{%
    \setlength\itemindent{0pt}
    \setlength\leftmargin\LaTeXoutdent
    \setlength\rightmargin{0pt}}%
    \item
      \setlength\LaTeXoutdent{-0.15\textwidth}
      \begin{minipage}[c]{%
        \textwidth-\LaTeXwidth-\LaTeXoutdent-\LaTeXgap}
        \LaTeXcodefloatend
      \end{minipage}%
      \hfill
      \begin{minipage}[c]{\LaTeXwidth}%
        \hbox to\linewidth{\box\lst@samplebox\hss}%
      \end{minipage}%
  \end{list}}
\newcommand*{\LaTeXcodefloat}{%
  \setkeys{lst}{tabsize=4,gobble=3,breakindent=0pt,
    basicstyle=\small\ttfamily,basewidth=0.51em,
    keywordstyle=\color{blue}}%
  \lst@BeginAlsoWriteFile{\LaTeXexamplefile}}
\let\LaTeXcodenoexample\LaTeXcodefloat
\let\LaTeXcodenoexampleend\@empty
\newcommand*{\LaTeXcodefloatend}{%
  \MakePercentComment\catcode`\^^M=10\relax
  \small
  {\setkeys{lst}{SelectCharTable=\lst@ReplaceInput{\^\^I}%
    {\lst@ProcessTabulator}}%
    \leavevmode \input{\LaTeXexamplefile}}%
  \MakePercentIgnore}
\newcommand*{\LaTeXcoderesultonlyend}{\egroup\LaTeXcodefloatend}
\lstnewenvironment{BibTeXexample}{%
  \global\let\lst@intname\@empty
  \setbox\lst@samplebox=\hbox\bgroup
  \setkeys{lst}{tabsize=4,gobble=3,breakindent=0pt,
    basicstyle=\small\ttfamily,basewidth=0.51em,
    keywordstyle=\color{black}}
  \lst@BeginAlsoWriteFile{\LaTeXexamplefile}}
 {\lst@EndWriteFile
   \LaTeXcodeexampleend}
\newcommand*{\DescribeOption}{%
  \leavevmode\@bsphack\begingroup\MakePrivateLetters
  \Describe@Option}
\newcommand*{\Describe@Option}[1]{\endgroup
              \marginpar{\raggedleft\PrintDescribeEnv{#1}}%
              \SpecialOptionIndex{#1}\@esphack\ignorespaces}
\newcommand*{\SpecialOptionIndex}[1]{\@bsphack
    \index{#1\actualchar{\protect\ttfamily#1}
           (option)\encapchar usage}%
    \index{options:\levelchar#1\actualchar{\protect\ttfamily#1}%
      \encapchar usage}\@esphack}
\newcommand*{\indexopt}[1]{\DescribeOption{#1}\opt{#1}}
\newcommand*{\DescribeOptionInfo}[2]{%
  \DescribeOption{#1}%
  \opt{#1=\meta{#2}}\xspace}
\newcommand*{\ofixarg}[1]{%
  {\ttfamily[}%
  \ifmmode \expandafter \nfss@text \fi
  {%
    \meta@font@select
    \edef\meta@hyphen@restore{%
      \hyphenchar\the\font\the\hyphenchar\font}%
    \hyphenchar\font\m@ne
    \language\l@nohyphenation
    #1\/%
    \meta@hyphen@restore
    }%
    {\ttfamily]}}
\newcommand*{\pkg}[1]{\textsf{#1}}
\newcommand*{\currpkg}{\pkg{\jobname}\xspace}
\newcommand*{\opt}[1]{\texttt{#1}}
\newcommand*{\defaultopt}[1]{\opt{\textbf{#1}}}
\newcommand*{\file}[1]{\texttt{#1}}
\newcommand*{\ext}[1]{\file{.#1}}
\newcommand*{\latin}[1]{\emph{#1}}
\newcommand*{\etc}{%
  \@ifnextchar.
    {\latin{etc}}
    {\latin{etc}.\xspace}}
\newcommand*{\eg}{%
  \@ifnextchar.
    {\latin{e.g}}
    {\latin{e.g}.\xspace}}
\newcommand*{\ie}{%
  \@ifnextchar.
    {\latin{i.e}}
    {\latin{i.e}.\xspace}}
\newcommand*{\etal}{%
  \@ifnextchar.
    {\latin{et~al.}}
    {\latin{et~al}.\xspace}}
\newcommand*{\AMS}{{\protect\usefont{OMS}{cmsy}{m}{n}%
  A\kern-.1667em\lower.5ex\hbox{M}\kern-.125emS}}
\providecommand*{\eTeX}{\ensuremath{\varepsilon}-\TeX}
\DeclareRobustCommand*{\XeTeX}
  {X\kern-.125em\lower.5ex\hbox{\reflectbox{E}}\kern-.1667em\TeX}
\providecommand*{\CTAN}{\textsc{ctan}}
\@ifpackageloaded{etoolbox}
  {\patchcmd{\@addmarginpar}
    {\@latex@warning@no@line {Marginpar on page \thepage\space moved}}
    {\relax}{}{}}
  {}
\newcounter{argument}
\g@addto@macro\endmacro{\setcounter{argument}{0}}
\newcommand*\darg[1]{%
  \stepcounter{argument}%
  {\ttfamily\char`\#\theargument~:~}#1\par\noindent\ignorespaces}
\newcommand*\doarg[1]{%
  \stepcounter{argument}%
  {\ttfamily\makebox[0pt][r]{[}%
   \char`\#\theargument]:~}#1\par\noindent\ignorespaces}
%</jawltxdoc>
%\fi
